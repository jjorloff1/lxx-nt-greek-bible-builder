% !TeX program = lualatex
% Layout
\documentclass[9pt,openany]{book}
\usepackage[paperwidth=6in,paperheight=9in,top=0.3in,bottom=0.7in,left=0.4in,right=0.4in,bindingoffset=0.2in,includehead,headsep=7pt]{geometry}
\usepackage{setspace} % for line spacing
\usepackage{microtype} % helps with formatting
\usepackage{emptypage}
\usepackage[greek]{babel}
\usepackage[all]{nowidow}
\usepackage{ragged2e} % Nicer ragged edges. Use \Center to get nice balanced lines
\usepackage{bookmark} %add PDF bookmarks for each \chapter
\usepackage{hyperref} % Make the TOC clickable
\usepackage{fontspec}
\setmainfont{Gentium}

% Define colors
\usepackage{xcolor}
\definecolor{bookheadingcolor}{HTML}{9B3A3F}
\definecolor{versenumbercolor}{HTML}{909090}

% % Use a different font for Greek (with Latin fallback)
% % https://tex.stackexchange.com/a/619560/319804
% \directlua{luaotfload.add_fallback("myfallback",{"Gentium:mode=harf;",})}
% \setmainfont{GFS Porson}[RawFeature={fallback=myfallback}]

\flushbottom
\frenchspacing % stops extra space after a period or colon
\hyphenpenalty=2000
\emergencystretch=1pt
\setlength{\parindent}{.2in} % Set paragraph indentation
\setlength{\parskip}{0\baselineskip}

\usepackage{titlesec}
\renewcommand{\thechapter}{} % remove numbers from chapter
% Show chapter titles with a line after them
\titleformat{\chapter}[display]
{\filcenter\normalfont\Huge}{}{0pt}{}
[\vspace{.5ex}\rule{1.5in}{0.4pt}]

% Add the multicol package for two-column layout
\usepackage{multicol}
\setlength{\columnsep}{.4cm} % Set the columnn width

% Adjust spacing and format of TOC 
\usepackage{tocloft}
\renewcommand{\cftchapfont}{\raggedright}    % Book names left-justified
\renewcommand{\cftchappagefont}{\raggedleft} % Page numbers right-justified
\renewcommand{\cftdotsep}{2}                 % change the default spacing of dots
\renewcommand{\cftchapleader}{\cftdotfill{\cftdotsep}} % Add leader dots for chapters
\setlength{\cftbeforechapskip}{1pt}          % Adjust the space before each TOC chapter entry

% Adjust the width of the TOC
\setlength{\cftchapindent}{0em}              % No indent
\setlength{\cftchapnumwidth}{0em}            % No space reserved for chapter numbers

\newcommand{\psalmheading}[1]{%
    \begin{Center}%
        {#1}%
    \end{Center}%
}

% Set up chapter/verse references in headers
% https://tex.stackexchange.com/a/657575/319804
\usepackage{fancyhdr}
\pagestyle{fancy}
\newcounter{mychapter}
\newcounter{verse}
\def\book{}
\fancyhead{}
\fancyhead[LE]{\rightmark}
\fancyhead[OR]{\leftmark}
\fancyfoot{}
\fancyfoot[C]{\thepage}
\renewcommand{\headrulewidth}{0pt}

\newcommand{\ch}[1]{%
  \setcounter{mychapter}{#1}%
  \markboth{\book\ \themychapter:1}{\book\ \themychapter:1}%
  \textbf{#1}%
}
\newcommand{\vs}[1]{%
  \textsuperscript{\textcolor{versenumbercolor}{#1}}%
  \markboth{\book\ \themychapter:#1}{\book\ \themychapter:#1}%
}

\let\biblebook\chapter % rename \chapter so it's not confusing

\newenvironment{psalmhead}[1]{%
    \par\addvspace{\baselineskip}%
    \noindent%
    \begin{minipage}[b]{\columnwidth}%
        \begin{Center}%
            {#1}%
        \end{Center}%
}{%
    \end{minipage}%
}

% Chapter and Heading Styles

\let\oldbiblebook\biblebook
\renewcommand{\biblebook}[1]{\textcolor{bookheadingcolor}{\oldbiblebook{#1}}}

\usepackage{marginnote}
\let\oldch\ch
\renewcommand{\ch}[1]{%
  \setcounter{mychapter}{#1}%
  \markboth{\book\ \themychapter:1}{\book\ \themychapter:1}%
  \leavevmode\llap{\textcolor{bookheadingcolor}{\textbf{\oldch{#1}}\hspace*{2em}}}%
}

\usepackage{lettrine}
\usepackage{calc}

\newcommand{\postdropcapindent}{\hspace*{.2in + .8em}}

\usepackage{etoolbox}
% \AtBeginEnvironment{quote}{\vspace{-0.5\baselineskip}}
\AtEndEnvironment{quote}{\vspace{-.5\baselineskip}}
% \renewcommand{\ch}[1]{%
%   \setcounter{mychapter}{#1}%
%   \markboth{\book\ \themychapter:1}{\book\ \themychapter:1}%
%   \marginnote[\textcolor{bookheadingcolor}{\textbf{\oldch{#1}}}]{}%
% }
% \renewcommand{\ch}[1]{\textcolor{bookheadingcolor}{\oldch{#1}}}

\title{Η ΑΓΙΑ ΓΡΑΦΗ}
\author{}
\date{}

\begin{document}
\begin{spacing}{1.1}
\maketitle

\cleardoublepage
\begin{titlepage}
  \begin{center}
    \textcolor{bookheadingcolor}{\Huge Preface}\par
  \end{center}
  \vspace{2em}
  
  This project was undertaken in love and respect for the Bible, with a desire to have an accessible and
  beautiful Greek Bible available in print to anyone who would like one. While there are many great Greek
  New Testaments available in print, the Septuagint has been less accessible, particularly in a format that
  is both compact and minimalist. Most of the Septuagints available in print are quite large. Additionally,
  there are almost no complete Greek Bibles available to purchase for a reasonable price. I have undertaken
  this project for those out there who, like me, want to take a physical Greek Bible along 
  with them where ever they want to go.

  When I started this project, I went hunting for open and public domain editions of the Septuagint and the NT
  that I could use as the texts for this Bible. While there are several great options out there, I settled on
  the Brenton Septuagint and the OpenGNT new testament. The reason for choosing Brenton's Septuagint was simple:
  I found a great open source project that had already digitized the text and prepared it for print: 
  https://github.com/mrgreekgeek/Brenton-LXX-Latex-print-project/. Starting with this baseline, I was able to
  style the text in a way that I liked. Then I had to find and prepare a NT Text.

  For the NT I chose the Open GNT (https://opengnt.com/) which was prepared by Eliran Wong and released under
  the Creative Commons Attribution 4.0 International License (CC BY 4.0). This project was created "to offer a 
  FREE NA-equivalent text of Greek New Testament, compiled from open-resources" and provided access to the text
  in a format that I could adapt to my needs.

  As for formatting, I was inspired by some of the beautiful minimalist reader Bibles available in English. As
  much as possible, I wanted to keep the text front and center, eliminating distractions and unnecessary elements.
  I have tried to mitigate the distraction from things like section headings, spacing between chapters, and even
  chapter numbers to some degree. I ultimately decided to leave verse numbers in place, because I think navigating
  the Old Testament may have been more difficult without them; however, I tried to minimize their visual impact.
  My goal is to facilitate a novel-like reading experience, free of distractions.

  The source code that I have used to extract, process, and format the texts used for this Bible is available 
  free of charge at https://github.com/jjorloff1/lxx-nt-greek-bible-builder.

  I hope that this Greek Bible will serve you well as you study and meditate on the Scriptures.
  Glory to God!

  \vfill
  \begin{flushright}
    {\large\textit{Jesse Orloff}\par}
    {\large www.jesseorloff.com\par}
    {\large August 2025\par}
  \end{flushright}  
\end{titlepage}

\cleardoublepage
\pagestyle{empty}
\begingroup
\centering
{\huge \textcolor{bookheadingcolor}{Table of Contents} \par}
\endgroup

\begin{multicols}{2}
\makeatletter
\renewcommand{\tableofcontents}{\@starttoc{toc}}
\makeatother
\tableofcontents
\end{multicols}
\pagestyle{fancy}

\cleardoublepage
\thispagestyle{empty}
\vspace*{3cm}
\phantomsection
\addcontentsline{toc}{part}{Η ΠΑΛΑΙΑ ΔΙΑΘΗΚΗ}
\begin{center}
  {\Huge Η ΠΑΛΑΙΑ ΔΙΑΘΗΚΗ}\\[2em]
  {\large Ἡ μετάφρασις τῶν Ἑβδομήκοντα}
\end{center}
\newpage
\thispagestyle{empty}
\null

\def\book{ΓΕΝΕΣΙΣ}
\biblebook{ΓΕΝΕΣΙΣ}


\lettrine[lines=2, loversize=0.2, nindent=0em, findent=.25em]{\textcolor{bookheadingcolor}{Ἐ}}{Ν} ἀρχῇ ἐποίησεν ὁ Θεὸς τὸν οὐρανὸν καὶ τὴν γῆν.
\vs{2}Ἡ δὲ γῆ ἦν ἀόρατος καὶ ἀκατασκεύαστος, καὶ σκότος ἐπάνω τῆς ἀβύσσου· καὶ πνεῦμα Θεοῦ ἐπεφέρετο ἐπάνω τοῦ ὕδατος.
\vs{3}Καὶ εἶπεν ὁ Θεὸς, γενηθήτω φῶς· καὶ ἐγένετο φῶς.
\vs{4}Καὶ εἶδεν ὁ Θεὸς τὸ φῶς, ὅτι καλόν· καὶ διεχώρισεν ὁ Θεὸς ἀνὰ μέσον τοῦ φωτὸς, καὶ ἀνὰ μέσον τοῦ σκότους.
\vs{5}Καὶ ἐκάλεσεν ὁ Θεὸς τὸ φῶς ἡμέραν, καὶ τὸ σκότος ἐκάλεσε νύκτα. Καὶ ἐγένετο ἑσπέρα, καὶ ἐγένετο πρωῒ, ἡμέρα μία.

\vs{6}Καὶ εἶπεν ὁ Θεὸς, γενηθήτω στερέωμα ἐν μέσῳ τοῦ ὕδατος· καὶ ἔστω διαχωρίζον ἀνὰ μέσον ὕδατος καὶ ὕδατος· καὶ ἐγένετο οὕτως.
\vs{7}Καὶ ἐποίησεν ὁ Θεὸς τὸ στερέωμα· καὶ διεχώρισεν ὁ Θεὸς ἀνὰ μέσον τοῦ ὕδατος, ὃ ἦν ὑποκάτω τοῦ στερεώματος, καὶ ἀνὰ μέσον τοῦ ὕδατος, τοῦ ἐπάνω τοῦ στερεώματος.
\vs{8}Καὶ ἐκάλεσεν ὁ Θεὸς τὸ στερέωμα οὐρανόν· καὶ εἶδεν ὁ Θεὸς ὅτι καλόν· καὶ ἐγένετο ἑσπέρα, καὶ ἐγένετο πρωῒ, ἡμέρα δευτέρα.

\vs{9}Καὶ εἶπεν ὁ Θεὸς, συναχθήτω τὸ ὕδωρ τὸ ὑποκάτω τοῦ οὐρανοῦ εἰς συναγωγὴν μίαν, καὶ ὀφθήτω ἡ ξηρά· καὶ ἐγένετο οὕτως· καὶ συνήχθη τὸ ὕδωρ τὸ ὑποκάτω τοῦ οὐρανοῦ εἰς τὰς συναγωγὰς αὐτῶν, καὶ ὤφθη ἡ ξηρά.
\vs{10}Καὶ ἐκάλεσεν ὁ Θεὸς τὴν ξηρὰν, γῆν· καὶ τὰ συστήματα τῶν ὑδάτων ἐκάλεσε θαλάσσας· καὶ εἶδεν ὁ Θεὸς ὅτι καλόν.
\vs{11}Καὶ εἶπεν ὁ Θεὸς, βλαστησάτω ἡ γῆ βοτάνην χόρτου, σπεῖρον σπέρμα κατὰ γένος καὶ καθʼ ὁμοιότητα, καὶ ξύλον κάρπιμον ποιοῦν καρπὸν, οὗ τὸ σπέρμα αὐτοῦ ἐν αὐτῷ κατὰ γένος ἐπὶ τῆς γῆς· καὶ ἐγένετο οὕτως.
\vs{12}Καὶ ἐξήνεγκεν ἡ γῆ βοτάνην χόρτου, σπεῖρον σπέρμα κατὰ γένος καὶ καθʼ ὁμοιότητα, καὶ ξύλον κάρπιμον ποιοῦν καρπὸν, οὗ τὸ σπέρμα αὐτοῦ ἐν αὐτῷ κατὰ γένος ἐπὶ τῆς γῆς· καὶ εἶδεν ὁ Θεὸς ὅτι καλόν.
\vs{13}Καὶ ἐγένετο ἑσπέρα, καὶ ἐγένετο πρωῒ, ἡμέρα τρίτη.

\vs{14}Καὶ εἶπεν ὁ Θεὸς, γενηθήτωσαν φωστῆρες ἐν τῷ στερεώματι τοῦ οὐρανοῦ εἰς φαῦσιν ἐπὶ τῆς γῆς, τοῦ διαχωρίζειν ἀνὰ μέσον τῆς ἡμέρας καὶ ἀνὰ μέσον τῆς νυκτός· καὶ ἔστωσαν εἰς σημεῖα, καὶ εἰς καιροὺς, καὶ εἰς ἡμέρας, καὶ εἰς ἐνιαυτούς.
\vs{15}Καὶ ἔστωσαν εἰς φαῦσιν ἐν τῷ στερεώματι τοῦ οὐρανοῦ, ὥστε φαίνειν ἐπὶ τῆς γῆς· καὶ ἐγένετο οὕτως.
\vs{16}Καὶ ἐποίησεν ὁ Θεὸς τοὺς δύο φωστῆρας τοὺς μεγάλους· τὸν φωστῆρα τὸν μέγαν εἰς ἀρχὰς τῆς ἡμέρας, καὶ τὸν φωστῆρα τὸν ἐλάσσω εἰς ἀρχὰς τῆς νυκτὸς, καὶ τοὺς ἀστέρας.
\vs{17}Καὶ ἔθετο αὐτοὺς ὁ Θεὸς ἐν τῷ στερεώματι τοῦ οὐρανοῦ, ὥστε φαίνειν ἐπὶ τῆς γῆς,
\vs{18}καὶ ἄρχειν τῆς ἡμέρας καὶ τῆς νυκτὸς, καὶ διαχωρίζειν ἀνὰ μέσον τοῦ φωτὸς, καὶ ἀνὰ μέσον τοῦ σκότους· καὶ εἶδεν ὁ Θεὸς ὅτι καλόν.
\vs{19}Καὶ ἐγένετο ἑσπέρα καὶ ἐγένετο πρωῒ, ἡμέρα τετάρτη.

\vs{20}Καὶ εἶπεν ὁ Θεὸς, ἐξαγαγέτω τὰ ὕδατα ἑρπετὰ ψυχῶν ζωσῶν, καὶ πετεινὰ πετόμενα ἐπὶ τῆς γῆς κατὰ τὸ στερέωμα τοῦ οὐρανοῦ· καὶ ἐγένετο οὕτως.
\vs{21}Καὶ ἐποίησεν ὁ Θεὸς τὰ κήτη τὰ μεγάλα, καὶ πᾶσαν ψυχὴν ζώων ἑρπετῶν, ἃ ἐξήγαγε τὰ ὕδατα κατὰ γένη αὐτῶν, καὶ πᾶν πετεινὸν πτερωτὸν κατὰ γένος· καὶ εἶδεν ὁ Θεὸς ὅτι καλά.
\vs{22}Καὶ εὐλόγησεν αὐτὰ ὁ Θεὸς, λέγων, αὐξάνεσθε καὶ πληθύνεσθε, καὶ πληρώσατε τὰ ὕδατα ἐν ταῖς θαλάσσαις, καὶ τὰ πετεινὰ πληθυνέσθωσαν ἐπὶ τῆς γῆς.
\vs{23}Καὶ ἐγένετο ἑσπέρα, καὶ ἐγένετο πρωῒ, ἡμέρα πέμπτη.

\vs{24}Καὶ εἶπεν ὁ Θεὸς, ἐξαγαγέτω ἡ γῆ ψυχὴν ζῶσαν κατὰ γένος, τετράποδα, καὶ ἑρπετὰ, καὶ θηρία τῆς γῆς κατὰ γένος· καὶ ἐγένετο οὕτως.
\vs{25}Καὶ ἐποίησεν ὁ Θεὸς τὰ θηρία τῆς γῆς κατὰ γένος, καὶ τὰ κτήνη κατὰ γένος αὐτῶν, καὶ πάντα τὰ ἑρπετὰ τῆς γῆς κατὰ γένος· καὶ εἶδεν ὁ Θεὸς ὅτι καλά.

\vs{26}Καὶ εἶπεν ὁ Θεός, Ποιήσωμεν ἄνθρωπον κατʼ εἰκόνα ἡμετέραν καὶ καθʼ ὁμοίωσιν· καὶ ἀρχέτωσαν τῶν ἰχθύων τῆς θαλάσσης, καὶ τῶν πετεινῶν τοῦ οὐρανοῦ, καὶ τῶν κτηνῶν, καὶ πάσης τῆς γῆς, καὶ πάντων τῶν ἑρπετῶν τῶν ἑρπόντων ἐπὶ τῆς γῆς.
\vs{27}Καὶ ἐποιήσεν ὁ Θεὸς τὸν ἄνθρωπον· κατʼ εἰκόνα Θεοῦ ἐποίησεν αὐτόν· ἄρσεν καὶ θῆλυ ἐποίησεν αὐτούς.
\vs{28}Καὶ εὐλόγησεν αὐτοὺς ὁ Θεὸς, λέγων, αὐξάνεσθε καὶ πληθύνεσθε, καὶ πληρώσατε τὴν γῆν, καὶ κατακυριεύσατε αὐτῆς· καὶ ἄρχετε τῶν ἰχθύων τῆς θαλάσσης, καὶ τῶν πετεινῶν τοῦ οὐρανοῦ, καὶ πάντων τῶν κτηνῶν, καὶ πάσης τῆς γῆς, καὶ πάντων τῶν ἑρπετῶν τῶν ἑρπόντων ἐπὶ τῆς γῆς.
\vs{29}Καὶ εἶπεν ὁ Θεός, Ἰδοὺ δέδωκα ὑμῖν πάντα χόρτον σπόριμον σπεῖρον σπέρμα, ὅ ἐστιν ἐπάνω πάσης τῆς γῆς· καὶ πᾶν ξύλον, ὃ ἔχει ἐν ἑαυτῷ καρπὸν σπέρματος σπορίμου, ὑμῖν ἔσται εἰς βρῶσιν,
\vs{30}καὶ πᾶσι τοῖς θηρίοις τῆς γῆς, καὶ πᾶσι τοῖς πετεινοῖς τοῦ οὐρανοῦ, καὶ παντὶ ἑρπετῷ ἕρποντι ἐπὶ τῆς γῆς, ὃ ἔχει ἐν ἑαυτῷ ψυχὴν ζωῆς, καὶ πάντα χόρτον χλωρὸν εἰς βρῶσιν· καὶ ἐγένετο οὕτως.
\vs{31}Καὶ εἶδεν ὁ Θεὸς τὰ πάντα, ὅσα ἐποίησε, καὶ ἰδοὺ καλὰ λίαν· καὶ ἐγένετο ἑσπέρα, καὶ ἐγένετο πρωῒ, ἡμέρα ἕκτη.

\ch{2}Καὶ συνετελέσθησαν ὁ οὐρανὸς καὶ ἡ γῆ, καὶ πᾶς ὁ κόσμος αὐτῶν.

\vs{2}Καὶ συνετέλεσεν ὁ Θεὸς ἐν τῇ ἡμέρᾳ τῇ ἕκτῃ τὰ ἔργα αὐτοῦ, ἃ ἐποίησε· καὶ κατέπαυσε τῇ ἡμέρᾳ τῇ ἑβδόμῃ ἀπὸ πάντων τῶν ἔργων αὐτοῦ, ὧν ἐποίησε.
\vs{3}Καὶ εὐλόγησεν ὁ Θεὸς τὴν ἡμέραν τὴν ἑβδόμην, καὶ ἡγίασεν αὐτήν, ὅτι ἐν αὐτῇ κατέπαυσεν ἀπὸ πάντων τῶν ἔργων αὐτοῦ, ὧν ἤρξατο ὁ Θεὸς ποιῆσαι.

\vs{4}Αὕτη ἡ βίβλος γενέσεως οὐρανοῦ καὶ γῆς, ὅτε ἐγένετο, ᾗ ἡμέρᾳ ἐποίησε Κύριος ὁ Θεὸς τὸν οὐρανὸν καὶ τὴν γῆν,
\vs{5}καὶ πᾶν χλωρὸν ἀγροῦ πρὸ τοῦ γενέσθαι ἐπὶ τῆς γῆς, καὶ πάντα χόρτον ἀγροῦ πρὸ τοῦ ἀνατεῖλαι· οὐ γὰρ ἔβρεξεν ὁ Θεὸς ἐπὶ τὴν γῆν, καὶ ἄνθρωπος οὐκ ἦν ἐργάζεσθαι αὐτήν.
\vs{6}Πηγὴ δὲ ἀνέβαινεν ἐκ τῆς γῆς, καὶ ἐπότιζε πᾶν τὸ πρόσωπον τῆς γῆς.
\vs{7}Καὶ ἔπλασεν ὁ Θεὸς τὸν ἄνθρωπον, χοῦν ἀπὸ τῆς γῆς· καὶ ἐνεφύσησεν εἰς τὸ πρόσωπον αὐτοῦ πνοὴν ζωῆς, καὶ ἐγένετο ὁ ἄνθρωπος εἰς ψυχὴν ζῶσαν.

\vs{8}Καὶ ἐφύτευσεν ὁ Θεὸς παράδεισον ἐν Ἐδὲμ κατὰ ἀνατολάς· καὶ ἔθετο ἐκεῖ τὸν ἄνθρωπον, ὃν ἔπλασε.
\vs{9}Καὶ ἐξανέτειλεν ὁ Θεὸς ἔτι ἐκ τῆς γῆς πᾶν ξύλον ὡραῖον εἰς ὅρασιν, καὶ καλὸν εἰς βρῶσιν, καὶ τὸ ξύλον τῆς ζωῆς ἐν μέσῳ τοῦ παραδείσου, καὶ τὸ ξύλον τοῦ εἰδέναι γνωστὸν καλοῦ καὶ πονηροῦ.
\vs{10}Ποταμὸς δὲ ἐκπορεύεται ἐξ Ἐδὲμ ποτίζειν τὸν παράδεισον· ἐκεῖθεν ἀφορίζεται εἰς τέσσαρας ἀρχάς.
\vs{11}Ὄνομα τῷ ἑνὶ, Φισῶν· οὗτος ὁ κυκλῶν πᾶσαν τὴν γῆν Εὐιλάτ· ἐκεῖ οὗ ἐστι τὸ χρυσίον.
\vs{12}Τὸ δὲ χρυσίον τῆς γῆς ἐκείνης καλόν· καὶ ἐκεῖ ἐστιν ὁ ἄνθραξ, καὶ ὁ λίθος ὁ πράσινος.
\vs{13}Καὶ ὄνομα τῷ ποταμῷ τῷ δευτέρῳ, Γεῶν· οὗτος ὁ κυκλῶν πᾶσαν τὴν γὴν Αἰθιοπίας.
\vs{14}Καὶ ὁ ποταμὸς ὁ τρίτος, Τίγρις· οὗτος ὁ προπορευόμενος κατέναντι Ἀσσυρίων· ὁ δὲ ποταμὸς ὁ τέταρτος, Εὐφράτης.
\vs{15}Καὶ ἔλαβε Κύριος ὁ Θεὸς τὸν ἄνθρωπον ὃν ἔπλασε, καὶ ἔθετο αὐτὸν ἐν τῷ παραδείσῳ τῆς τρυφῆς, ἐργάζεσθαι αὐτὸν καὶ φυλάσσειν.
\vs{16}Καὶ ἐνετείλατο Κύριος ὁ Θεὸς τῷ Ἀδὰμ, λέγων, ἀπὸ παντὸς ξύλου τοῦ ἐν τῷ παραδείσῳ βρώσει φαγῇ.
\vs{17}Ἀπὸ δὲ τοῦ ξύλου τοῦ γινώσκειν καλὸν καὶ πονηρὸν, οὐ φάγεσθε ἀπʼ αὐτοῦ· ᾗ δʼ ἂν ἡμέρᾳ φάγητε ἀπʼ αὐτοῦ, θανάτῳ ἀποθανεῖσθε.

\vs{18}Καὶ εἶπε Κύριος ὁ Θεὸς, οὐ καλὸν εἶναι τὸν ἄνθρωπον μόνον· ποιήσωμεν αὐτῷ βοηθὸν κατʼ αὐτόν.
\vs{19}Καὶ ἔπλασεν ὁ Θεὸς ἔτι ἐκ τῆς γῆς πάντα τὰ θηρία τοῦ ἀγροῦ, καὶ πάντα τὰ πετεινὰ τοῦ οὐρανοῦ· καὶ ἤγαγεν αὐτὰ πρὸς τὸν Ἀδὰμ, ἰδεῖν τί καλέσει αὐτά· καὶ πᾶν ὃ ἐὰν ἐκάλεσεν αὐτὸ Ἀδὰμ ψυχὴν ζῶσαν, τοῦτο ὄνομα αὐτῷ.
\vs{20}Καὶ ἐκάλεσεν Ἀδὰμ ὀνόματα πᾶσι τοῖς κτήνεσι, καὶ πᾶσι τοῖς πετεινοῖς τοῦ οὐρανοῦ, καὶ πᾶσι τοῖς θηρίοις τοῦ ἀγροῦ· τῷ δὲ Ἀδὰμ οὐχ εὑρέθη βοηθὸς ὅμοιος αὐτῷ.
\vs{21}Καὶ ἐπέβαλεν ὁ Θεὸς ἔκστασιν ἐπὶ τὸν Ἀδὰμ, καὶ ὕπνωσε· καὶ ἔλαβε μίαν τῶν πλευρῶν αὐτοῦ, καὶ ἀνεπλήρωσε σάρκα ἀντʼ αὐτῆς.
\vs{22}Καὶ ᾠκοδόμησεν ὁ Θεὸς τὴν πλευρὰν, ἣν ἔλαβεν ἀπὸ τοῦ Ἀδὰμ εἰς γυναῖκα· καὶ ἤγαγεν αὐτὴν πρὸς τὸν Ἀδάμ.
\vs{23}Καὶ εἶπεν Ἀδάμ· τοῦτο νῦν ὀστοῦν ἐκ τῶν ὀστέων μου, καὶ σὰρξ ἐκ τῆς σαρκός μου· αὕτη κληθήσεται γυνὴ, ὅτι ἐκ τοῦ ἀνδρὸς αὐτῆς ἐλήφθη.
\vs{24}Ἕνεκεν τούτου καταλείψει ἄνθρωπος τὸν πατέρα αὐτοῦ καὶ τὴν μητέρα, καὶ προσκολληθήσεται πρὸς τὴν γυναῖκα αὐτοῦ· καὶ ἔσονται οἱ δύο εἰς σάρκα μίαν.
\vs{25}Καὶ ἦσαν οἱ δύο γυμνοὶ, ὅ, τε Ἀδὰμ καὶ ἡ γυνὴ αὐτοῦ, καὶ οὐκ ᾐσχύνοντο.

\ch{3}
Ὁ δὲ ὄφις ἦν φρονιμώτατος πάντων τῶν θηρίων τῶν ἐπὶ τῆς γῆς, ὧν ἐποίησε Κύριος ὁ Θεός· καὶ εἶπεν ὁ ὄφις τῇ γυναικὶ, τί ὅτι εἶπεν ὁ Θεός, οὐ μὴ φάγητε ἀπὸ παντὸς ξύλου τοῦ παραδείσου;
\vs{2}Καὶ εἶπεν ἡ γυνὴ τῷ ὄφει, ἀπὸ καρποῦ τοῦ ξύλου τοῦ παραδείσου φαγούμεθα·
\vs{3}Ἀπὸ δὲ τοῦ καρποῦ τοῦ ξύλου, ὅ ἐστιν ἐν μέσῳ τοῦ παραδείσου, εἶπεν ὁ Θεός, οὐ φάγεσθε ἀπʼ αὐτοῦ, οὐδὲ μὴ ἅψησθε αὐτοῦ, ἵνα μὴ ἀποθάνητε.
\vs{4}Καὶ εἶπεν ὁ ὄφις τῇ γυναικί· οὐ θανάτῳ ἀποθανεῖσθε·
\vs{5}Ἤδει γὰρ ὁ Θεὸς, ὅτι ᾗ ἂν ἡμέρᾳ φάγητε ἀπʼ αὐτοῦ, διανοιχθήσονται ὑμῶν οἱ ὀφθαλμοί, καὶ ἔσεσθε ὡς θεοί, γινώσκοντες καλὸν καὶ πονηρόν.
\vs{6}Καὶ εἶδεν ἡ γυνὴ, ὅτι καλὸν τὸ ξύλον εἰς βρῶσιν, καὶ ὅτι ἀρεστὸν τοῖς ὀφθαλμοῖς ἰδεῖν, καὶ ὡραῖόν ἐστι τοῦ κατανοῆσαι· καὶ λαβοῦσα ἀπὸ τοῦ καρποῦ αὐτοῦ, ἔφαγε· καὶ ἔδωκε καὶ τῷ ἀνδρὶ αὐτῆς μετʼ αὐτῆς, καὶ ἔφαγον.
\vs{7}Καὶ διηνοίχθησαν οἱ ὀφθαλμοὶ τῶν δύο, καὶ ἔγνωσαν ὅτι γυμνοὶ ἦσαν· καὶ ἔῤῥαψαν φύλλα συκῆς, καὶ ἐποίησαν ἑαυτοῖς περιζώματα.
\vs{8}Καὶ ἤκουσαν τὴς φωνὴς Κυρίου τοῦ Θεοῦ περιπατοῦντος ἐν τῷ παραδείσῳ τὸ δειλινόν· καὶ ἐκρύβησαν ὅ, τε Ἀδὰμ καὶ ἡ γυνὴ αὐτοῦ ἀπὸ προσώπου Κυρίου τοῦ Θεοῦ ἐν μέσῳ τοῦ ξύλου τοῦ παραδείσου.
\vs{9}Καὶ ἐκάλεσεν Κύριος ὁ Θεὸς τὸν Ἀδὰμ, καὶ εἶπεν αὐτῷ· Ἀδὰμ ποῦ εἶ;
\vs{10}Καὶ εἶπεν αὐτῷ· τὴς φωνῆς σου ἤκουσα περιπατοῦντος ἐν τῷ παραδείσῳ, καὶ ἐφοβήθην ὅτι γυμνός εἰμι, καὶ ἐκρύβην.
\vs{11}Καὶ εἶπεν αὐτῷ ὁ Θεὸς, τὶς ἀνήγγειλέ σοι ὅτι γυμνὸς εἶ, εἰ μὴ ἀπὸ τοῦ ξύλου, οὗ ἐνετειλάμην σοι τούτου μόνου μὴ φαγεῖν, ἀπʼ αὐτοῦ ἔφαγες;
\vs{12}Καὶ εἶπεν ὁ Ἀδάμ· ἡ γυνή, ἣν ἔδωκας μετʼ ἐμοῦ, αὕτη μοι ἔδωκεν ἀπὸ τοῦ ξύλου, καὶ ἔφαγον.
\vs{13}Καὶ εἶπε Κύριος ὁ Θεὸς τῇ γυναικί· τί τοῦτο ἐποιήσας; καὶ εἶπεν ἡ γυνὴ, ὁ ὄφις ἠπάτησέ με, καὶ ἔφαγον.

\vs{14}Καὶ εἶπε Κύριος ὁ Θεὸς τῷ ὄφει· ὅτι ἐποίησας τοῦτο, ἐπικατάρατος σὺ ἀπὸ πάντων τῶν κτηνῶν, καὶ ἀπὸ πάντων τῶν θηρίων τῶν ἐπὶ τῆς γῆς· ἐπὶ τῷ στήθει σου καὶ τῇ κοιλίᾳ πορεύσῃ, καὶ γῆν φαγῃ πάσας τὰς ἡμέρας τῆς ζωῆς σου.
\vs{15}Καὶ ἔχθραν θήσω ἀνὰ μέσον σοῦ καὶ ἀνὰ μέσον τῆς γυναικὸς, καὶ ἀνὰ μέσον τοῦ σπέρματός σου, καὶ ἀνὰ μέσον τοῦ σπέρματος αὐτῆς· αὐτός σοῦ τηρήσει κεφαλὴν, καὶ σὺ τηρήσεις αὐτοῦ πτέρναν.
\vs{16}Καὶ τῇ γυναικὶ εἶπε· πληθύνων πληθυνῶ τὰς λύπας σου, καὶ τὸν στεναγμόν σου· ἐν λύπαις τέξῃ τέκνα, καὶ πρὸς τὸν ἄνδρα σου ἡ ἀποστροφή σου· καὶ αὐτός σου κυριεύσει.
\vs{17}Τῷ δὲ Ἀδὰμ εἶπεν· ὅτι ἤκουσας τῆς φωνῆς τῆς γυναικός σου, καὶ ἔφαγες ἀπὸ τοῦ ξύλου, οὗ ἐνετειλάμην σοι τούτου μόνου μὴ φαγεῖν, ἀπʼ αὐτοῦ ἔφαγες, ἐπικατάρατος ἡ γῆ ἐν τοῖς ἔργοις σου· ἐν λύπαις φάγῃ αὐτὴν πάσας τὰς ἡμέρας τῆς ζωῆς σου.
\vs{18}Ἀκάνθας καὶ τριβόλους ἀνατελεῖ σοι, καὶ φαγῇ τὸν χόρτον τοῦ ἀγροῦ.
\vs{19}Ἐν ἱδρῶτι τοῦ προσώπου σου φαγῃ τὸν ἄρτον σου, ἕως τοῦ ἀποστρέψαι σε εἰς τὴν γῆν ἐξ ἧς ἐλήμφθης· ὅτι γῆ εἶ, καὶ εἰς γῆν ἀπελεύσῃ.
\vs{20}Καὶ ἐκάλεσεν Ἀδὰμ τὸ ὄνομα τῆς γυναικὸς αὐτοῦ Ζωή, ὅτι μήτηρ πάντων τῶν ζώντων.
\vs{21}Καὶ ἐποίησε Κύριος ὁ Θεὸς τῷ Ἀδὰμ, καὶ τῇ γυναικὶ αὐτοῦ χιτῶνας δερματίνους, καὶ ἐνέδυσεν αὐτούς.

\vs{22}Καὶ εἶπεν ὁ Θεός, ἰδοὺ Ἀδὰμ γέγονεν ὡς εἷς ἐξ ἡμῶν, τοῦ γινώσκειν καλὸν καὶ πονηρόν· καὶ νῦν μή ποτε ἐκτείνῃ τὴν χεῖρα αὐτοῦ, καὶ λάβῃ τοῦ ξύλου τῆς ζωῆς καὶ φάγῃ, καὶ ζήσεται εἰς τὸν αἰῶνα.
\vs{23}Καὶ ἐξαπέστειλεν αὐτὸν Κύριος ὁ Θεὸς ἐκ τοῦ παραδείσου τῆς τρυφῆς, ἐργάζεσθαι τὴν γῆν ἐξ ἧς ἐλήμφθη.
\vs{24}Καὶ ἐξέβαλεν τὸν Ἀδὰμ, καὶ κατῴκισεν αὐτὸν ἀπέναντι τοῦ παραδείσου τῆς τρυφῆς· καὶ ἔταξε τὰ χερουβὶμ· καὶ τὴν φλογίνην ῥομφαίαν τὴν στρεφομένην, φυλάσσειν τὴν ὁδὸν τοῦ ξύλου τῆς ζωῆς.

\ch{4}
Ἀδὰμ δὲ ἔγνω Εὔαν τὴν γυναῖκα αὐτοῦ, καὶ συλλαβοῦσα ἔτεκε τὸν Κάϊν· καὶ εἶπεν, ἐκτησάμην ἄνθρωπον διὰ τοῦ Θεοῦ.
\vs{2}Καὶ προσέθηκε τεκεῖν τὸν ἀδελφὸν αὐτοῦ τὸν Ἄβελ· καὶ ἐγένετο Ἄβελ ποιμὴν προβάτων, Κάϊν δὲ ἦν ἐργαζόμενος τὴν γῆν.
\vs{3}Καὶ ἐγένετο μεθʼ ἡμέρας ἤνεγκε Κάϊν ἀπὸ τῶν καρπῶν τῆς γῆς θυσίαν τῷ Κυρίῳ·
\vs{4}Καὶ Ἄβελ ἤνεγκε καὶ αὐτὸς ἀπὸ τῶν πρωτοτόκων τῶν προβάτων αὐτοῦ, καὶ ἀπὸ τῶν στεάτων αὐτῶν· καὶ ἐπεῖδεν ὁ Θεὸς ἐπὶ Ἄβελ, καὶ ἐπὶ τοῖς δώροις αὐτοῦ.
\vs{5}Ἐπὶ δὲ Κάϊν, καὶ ἐπὶ ταῖς θυσίαις αὐτοῦ, οὐ προσέσχε· καὶ ἐλυπήθη Κάϊν λίαν, καὶ συνέπεσε τῷ προσώπῳ αὐτοῦ.
\vs{6}Καὶ εἶπε Κύριος ὁ Θεὸς τῷ Κάϊν, ἵνα τί περίλυπος ἐγένου, καὶ ἵνα τί συνέπεσε τὸ πρόσωπόν σου;
\vs{7}Οὐκ ἐὰν ὀρθῶς προσενέγκῃς, ὀρθῶς δὲ μὴ διέλῃς, ἥμαρτες; ἡσυχασον· πρός σὲ ἡ ἀποστροφὴ αὐτοῦ, καὶ σὺ ἄρξεις αὐτοῦ.

\vs{8}Καὶ εἶπεν Κάϊν πρὸς Ἄβελ τὸν ἀδελφὸν αὐτοῦ, διέλθωμεν εἰς τὸ πεδίον· καὶ ἐγένετο ἐν τῷ εἶναι αὐτοὺς ἐν τῷ πεδίῳ, ἀνέστη Κάϊν ἐπὶ Ἄβελ τὸν ἀδελφὸν αὐτοῦ, καὶ ἀπέκτεινεν αὐτόν.
\vs{9}Καὶ εἶπε Κύπιος ὁ Θεὸς πρὸς Κάϊν· ποῦ ἔστιν Ἄβελ ὁ ἀδελφός σου; καὶ εἶπεν, οὐ γινώσκω· μὴ φύλαξ τοῦ ἀδελφοῦ μου εἰμὶ ἐγώ;
\vs{10}Καὶ εἶπε Κύριος, τί πεποίηκας; φωνὴ αἵματος τοῦ ἀδελφοῦ σου βοᾷ πρός με ἐκ τῆς γῆς.
\vs{11}Καὶ νῦν ἐπικατάρατος σὺ ἀπὸ τῆς γῆς, ἣ ἔχανε τὸ στόμα αὐτῆς δέξασθαι τὸ αἷμα τοῦ ἀδελφοῦ σου ἐκ τῆς χειρός σου.
\vs{12}Ὅτε ἐργᾷ τὴν γῆν, καὶ οὐ προσθήσει τὴν ἰσχὺν αὐτῆς δοῦναί σοι· στένων καὶ τρέμων ἐσῃ ἐπὶ τῆς γῆς.
\vs{13}Καὶ εἶπε Κάϊν πρὸς Κύριον τὸν Θεὸν, μείζων ἡ αἰτία μου τοῦ ἀφεθῆναί με.
\vs{14}Εἰ ἐκβάλλεις με σήμερον ἀπὸ προσώπου τῆς γῆς, καὶ ἀπὸ τοῦ προσώπου σου κρυβήσομαι, καὶ ἔσομαι στένων καὶ τρέμων ἐπὶ τῆς γῆς, καὶ ἔσται πᾶς ὁ εὑρίσκων με, ἀποκτενεῖ με.
\vs{15}Καὶ εἴπεν αὐτῷ Κύριος ὁ Θεὸς, οὐχ οὕτω· πᾶς ὁ ἀποκτείνας Κάϊν, ἑπτὰ ἐκδικούμενα παραλύσει. Καὶ ἔθετο Κύριος ὁ Θεὸς σημεῖον τῷ Κάϊν, τοῦ μὴ ἀνελεῖν αὐτὸν πάντα τὸν εὑρίσκοντα αὐτόν.
\vs{16}Ἐξῆλθεν δὲ Κάϊν ἀπὸ προσώπου τοῦ Θεοῦ, καὶ ᾤκησεν ἐν γῇ Ναὶδ κατέναντι Ἐδέμ.

\vs{17}Καὶ ἔγνω Κάϊν τὴν γυναῖκα αὐτοῦ· καὶ συλλαβοῦσα ἔτεκε τὸν Ἐνώχ. Καὶ ἦν οἰκοδομῶν πόλιν· καὶ ἐπῳνόμασε τὴν πόλιν ἐπὶ τῷ ὀνόματι τοῦ υἱοῦ αὐτοῦ, Ἐνώχ.
\vs{18}Ἐγενήθη δὲ τῷ Ἐνὼχ Γαϊδάδ· καὶ Γαϊδὰδ ἐγέννησε τὸν Μαλελεὴλ· καὶ Μαλελεὴλ ἐγέννησε τὸν Μαθουσάλα· καὶ Μαθουσάλα ἐγέννησε τὸν Λάμεχ.

\vs{19}Καὶ ἔλαβεν ἑαυτῷ Λάμεχ δύο γυναῖκας· ὄνομα τῇ μιᾷ, Ἀδά· καὶ ὄνομα τῇ δευτέρᾳ, Σελλά.
\vs{20}Καὶ ἔτεκεν Ἀδὰ τὸν Ἰωβήλ· οὗτος ἦν πατὴρ οἰκούντων ἐν σκηναῖς κτηνοτρόφων.
\vs{21}Καὶ ὄνομα τῷ ἀδελφῷ αὐτοῦ, Ἰουβάλ· οὗτος ἦν ὁ καταδείξας ψαλτήριον καὶ κιθάραν.
\vs{22}Σελλὰ δὲ καὶ αὐτὴ ἔτεκε τὸν Θόβελ· καὶ ἦν σφυροκόπος χαλκεὺς χαλκοῦ καὶ σιδήρου. ἀδελφὴ δὲ Θόβελ, Νοεμά.
\vs{23}Εἶπε δὲ Λάμεχ ταῖς ἑαυτοῦ γυναιξίν, Ἀδὰ καὶ Σελλὰ, ἀκούσατέ μου τῆς φωνῆς, γυναῖκες Λάμεχ, ἐνωτίσασθέ μου τοὺς λόγους· ὅτι ἄνδρα ἀπέκτεινα εἰς τραῦμα ἐμοὶ, καὶ νεανίσκον εἰς μώλωπα ἐμοί.
\vs{24}Ὅτι ἑπτάκις ἐκδεδίκηται ἐκ Κάϊν· ἐκ δὲ Λάμεχ, ἑβδομηκοντάκις ἑπτά.

\vs{25}Ἔγνω δὲ Ἀδὰμ Εὔαν τὴν γυναῖκα αὐτοῦ· καὶ συλλαβοῦσα ἔτεκεν υἱόν· καὶ ἐπωνόμασε τὸ ὄνομα αὐτοῦ Σὴθ, λέγουσα, ἐξανέστησε γάρ μοι ὁ Θεὸς σπέρμα ἕτερον ἀντὶ Ἄβελ, ὃν ἀπέκτεινε Κάϊν.
\vs{26}Καὶ τῷ Σὴθ ἐγένετο υἱός· ἐπωνόμασε δὲ τὸ ὄνομα αὐτοῦ, Ἑνώς· οὗτος ἤλπισεν ἐπικαλεῖσθαι τὸ ὄνομα Κυρίου τοῦ Θεοῦ.

\ch{5}
Αὕτη ἡ βίβλος γενέσεως ἀνθρώπων· ᾗ ἡμέρᾳ ἐποίησεν ὁ Θεὸς τὸν Ἀδὰμ, κατʼ εἰκόνα Θεοῦ ἐποίησεν αὐτόν·
\vs{2}Ἄρσεν καὶ θῆλυ ἐποίησεν αὐτούς· καὶ εὐλόγησεν αὐτούς· καὶ ἐπωνόμασε τὸ ὄνομα αὐτοῦ Ἀδὰμ, ᾗ ἡμέρᾳ ἐποίησεν αὐτούς.
\vs{3}Ἔζησεν δὲ Ἀδὰμ τριάκοντα καὶ διακόσια ἔτη, καὶ ἐγέννησε κατὰ τὴν ἰδέαν αὐτοῦ, καὶ κατὰ τὴν εἰκόνα αὐτοῦ, καὶ ἐπωνόμασε τὸ ὄνομα αὐτοῦ, Σήθ.
\vs{4}Ἐγένοντο δὲ αἱ ἡμέραι Ἀδὰμ, ἃς ἔζησε μετὰ τὸ γεννῆσαι αὐτὸν τὸν Σὴθ, ἔτη ἑπτακόσια· καὶ ἐγέννησεν υἱοὺς καὶ θυγατέρας.
\vs{5}Καὶ ἐγένοντο πᾶσαι αἱ ἡμέραι Ἀδὰμ, ἃς ἔζησε, τριάκοντα καὶ ἐννακόσια ἔτη· καὶ ἀπέθανεν.
\vs{6}Ἔζησε δὲ Σὴθ πέντε καὶ διακόσια ἔτη· καὶ ἐγέννησε τὸν Ἐνώς.
\vs{7}Καὶ ἔζησε Σὴθ μετὰ τὸ γεννῆσαι αὐτὸν τὸν Ἐνὼς, ἑπτὰ ἔτη καὶ ἑπτακόσια· καὶ ἐγέννησεν υἱοὺς καὶ θυγατέρας.
\vs{8}Καὶ ἐγένοντο πᾶσαι αἱ ἡμέραι Σὴθ δώδεκα καὶ ἐννακόσια ἔτη· καὶ ἀπέθανε.
\vs{9}Καὶ ἔζησεν Ἐνὼς ἔτη ἑκατὸν ἐννεήκοντα· καὶ ἐγέννησε τὸν Καϊνᾶν.
\vs{10}Καὶ ἔζησεν Ἐνὼς μετὰ τὸ γεννῆσαι αὐτὸν τὸν Καϊνᾶν, πεντεκαίδεκα ἔτη καὶ ἑπτκόσια· καὶ ἐγέννησεν υἱοὺς καὶ θυγατέρας.
\vs{11}Καὶ ἐγένοντο πᾶσαι αἱ ἡμέραι Ἐνὼς πέντε ἔτη καὶ ἐννακόσια· καὶ ἀπέθανε.
\vs{12}Καὶ ἔζησεν Καϊνᾶν ἑβδομήκοντα καὶ ἑκατὸν ἔτη· καὶ ἐγέννησε τὸν Μαλελεήλ.
\vs{13}Καὶ ἔζησε Καϊνᾶν μετὰ τὸ γεννῆσαι αὐτὸν τὸν Μαλελεὴλ, τεσσεράκοντα καὶ ἑπτακόσια ἔτη· καὶ ἐγέννησεν υἱοὺς καὶ θυγατέρας.
\vs{14}Καὶ ἐγένοντο πᾶσαι αἱ ἡμέραι Καϊνᾶν δέκα ἔτη καὶ ἐννακόσια· καὶ ἀπέθανε.

\vs{15}Καὶ ἔζησε Μαλελεὴλ πέντε καὶ ἑξήκοντα καὶ ἑκατὸν ἔτη· καὶ ἐγέννησε τὸν Ἰάρεδ.
\vs{16}Καὶ ἔζησε Μαλελεὴλ μετὰ τὸ γεννῆσαι αὐτὸν τὸν Ἰάρεδ, ἔτη τριάκοντα καὶ ἑπτακόσια· καὶ ἐγέννησεν υἱοὺς καὶ θυγατέρας.
\vs{17}Καὶ ἐγένοντο πᾶσαι αἱ ἡμέραι Μαλελεὴλ, ἔτη πέντε καὶ ἐννενήκοντα καὶ ὀκτακόσια· καὶ ἀπέθανε.
\vs{18}Καὶ ἔζησεν Ἰάρεδ δύο καὶ ἑξήκοντα ἔτη καὶ ἑκατὸν· καὶ ἐγέννησε τὸν Ἐνώχ.
\vs{19}Καὶ ἔζησεν Ἰάρεδ μετὰ τὸ γεννῆσαι αὐτὸν τὸν Ἐνὼχ, ὀκτακόσια ἔτη· καὶ ἐγέννησεν υἱοὺς καὶ θυγατέρας.
\vs{20}Καὶ ἐγένοντο πᾶσαι αἱ ἡμέραι Ἰάρεδ, δύο καὶ ἑξήκοντα καὶ ἐννακόσια ἔτη· καὶ ἀπέθανε.
\vs{21}Καὶ ἔζησεν Ἐνὼχ πέντε καὶ ἑξήκοντα καὶ ἑκατὸν ἔτη· καὶ ἐγέννησε τὸν Μαθουσάλα.
\vs{22}Εὐηρέστησε δὲ Ἐνὼχ τῷ Θεῷ μετὰ τὸ γεννῆσαι αὐτὸν τὸν Μαθουσάλα, διακόσια ἔτη· καὶ ἐγέννησεν υἱοὺς καὶ θυγατέρας.
\vs{23}Καὶ ἐγένοντο πᾶσαι αἱ ἡμέραι Ἐνὼχ, πέντε καὶ ἑξήκοντα καὶ τριακόσια ἔτη.
\vs{24}Καὶ εὐηρέστησεν Ἐνὼχ τῷ Θεῷ· καὶ οὐχ εὑρίσκετο, ὅτι μετέθηκεν αὐτὸν ὁ Θεός.
\vs{25}Καὶ ἔζησε Μαθουσάλα ἑπτὰ ἔτη καὶ ἑξήκοντα καὶ ἑκατόν· καὶ ἐγέννησε τὸν Λάμεχ.
\vs{26}Καὶ ἔζησε Μαθουσάλα μετὰ τὸ γεννῆσαι αὐτὸν τὸν Λάμεχ, δύο καὶ ὀκτακόσια ἔτη· καὶ ἐγέννησεν υἱοὺς καὶ θυγατέρας.
\vs{27}Καὶ ἐγένοντο πᾶσαι αἱ ἡμέραι Μαθουσάλα ἃς ἔζησεν, ἐννέα καὶ ἑξήκοντα καὶ ἐννακόσια ἔτη· καὶ ἀπέθανε.
\vs{28}Καὶ ἔζησε Λάμεχ ὀκτὼ καὶ ὀγδοήκοντα καὶ ἑκατὸν ἔτη· καὶ ἐγέννησεν υἱόν.
\vs{29}Καὶ ἐπωνόμασε τὸ ὄνομα αὐτοῦ Νῶε, λέγων, οὗτος διαναπαύσει ἡμᾶς ἀπὸ τῶν ἔργων ἡμῶν, καὶ ἀπὸ τῶν λυπῶν τῶν χειρῶν ἡμῶν, καὶ ἀπὸ τῆς γῆς, ἧς κατηράσατο Κύριος ὁ Θεός.
\vs{30}Καὶ ἔζησε Λάμεχ μετὰ τὸ γεννῆσαι αὐτὸν τὸν Νῶε, πεντακόσια καὶ ἑξήκοντα καὶ πέντε ἔτη· καὶ ἐγέννησεν υἱοὺς καὶ θυγατέρας.
\vs{31}Καὶ ἐγένοντο πᾶσαι αἱ ἡμέραι Λάμεχ, ἑπτακόσια καὶ πεντήκοντα τρία ἔτη· καὶ ἀπέθανε.
\vs{32}Καὶ ἦν Νῶε ἐτῶν πεντακοσίων· καὶ ἐγέννησε τρεῖς υἱοὺς, τὸν Σὴμ, τὸν Χὰμ, τὸν Ἰάφεθ.

\ch{6}
Καὶ ἐγένετο ἡνίκα ἤρξαντο οἱ ἄνθρωποι πολλοὶ γίνεσθαι ἐπὶ τῆς γῆς, καὶ θυγατέρες ἐγεννήθησαν αὐτοῖς.
\vs{2}Ἰδόντες δὲ υἱοὶ τοῦ Θεοῦ τὰς θυγατέρας τῶν ἀνθρώπων, ὅτι καλαί εἰσιν, ἔλαβον ἑαυτοῖς γυναῖκας ἀπὸ πασῶν, ὧν ἐξελέξαντο.
\vs{3}Καὶ εἶπε Κύριος ὁ Θεὸς, οὐ μὴ καταμείνῃ τὸ πνεῦμά μου ἐν τοῖς ἀνθρώποις τούτοις εἰς τὸν αἰῶνα, διὰ τὸ εἶναι αὐτοὺς σάρκας· ἔσονται δὲ αἱ ἡμέραι αὐτῶν, ἑκατὸν εἴκοσιν ἔτη.
\vs{4}Οἱ δὲ γίγαντες ἦσαν ἐπὶ τῆς γῆς ἐν ταῖς ἡμέραις ἐκείναις, καὶ μετʼ ἐκεῖνο, ὡς ἂν εἰσεπορεύοντο οἱ υἱοὶ τοῦ Θεοῦ πρὸς τὰς θυγατέρας τῶν ἀνθρώπων, καὶ ἐγεννῶσαν αὐτοῖς· ἐκεῖνοι ἦσαν οἱ γίγαντες οἱ ἀπʼ αἰῶνος, οἱ ἄνθρωποι οἱ ὀνομαστοί.

\vs{5}Ἰδὼν δὲ Κύριος ὁ Θεὸς, ὅτι ἐπληθύνθησαν αἱ κακίαι τῶν ἀνθρώπων ἐπὶ τῆς γῆς, καὶ πᾶς τις διανοεῖται ἐν τῇ καρδίᾳ αὐτοῦ ἐπιμελῶς ἐπὶ τὰ πονηρὰ πάσας τὰς ἡμέρας·
\vs{6}Καὶ ἐνεθυμήθη ὁ Θεὸς, ὅτι ἐποίησε τὸν ἄνθρωπον ἐπὶ τῆς γῆς, καὶ διενοήθη.
\vs{7}Καὶ εἶπεν ὁ Θεὸς, ἀπαλείψω τὸν ἄνθρωπον, ὃν ἐποίησα, ἀπὸ προσώπου τῆς γῆς, ἀπὸ ἀνθρώπου ἕως κτήνους, καὶ ἀπὸ ἑρπετῶν ἕως πετεινῶν τοῦ οὐρανοῦ· ὅτι ἐνεθυμήθην, ὅτι ἐποίησα αὐτούς.

\vs{8}Νῶε δὲ εὗρε χάριν ἐναντίον Κυρίου τοῦ Θεοῦ.
\vs{9}Αὗται δὲ αἱ γενέσεις Νῶε. Νῶε ἄνθρωπος δίκαιος, τέλειος ὢν ἐν τῇ γενεᾷ αὐτοῦ, τῷ Θεῷ εὐηρέστησε Νῶε.
\vs{10}Ἐγέννησε δὲ Νῶε τρεῖς υἱοὺς, τὸν Σὴμ, τὸν Χὰμ, τὸν Ἰάφεθ.
\vs{11}Ἐφθάρη δὲ ἡ γῆ ἐναντίον τοῦ Θεοῦ, καὶ ἐπλήσθη ἡ γῆ ἀδικίας.
\vs{12}Καὶ εἶδε Κύριος ὁ Θεὸς τὴν γῆν, καὶ ἦν κατεφθαρμένη· ὅτι κατέφθειρε πᾶσα σὰρξ τὴν ὁδὸν αὐτοῦ ἐπὶ τῆς γῆς.
\vs{13}Καὶ εἶπε Κύριος ὁ Θεὸς τῷ Νῶε, καιρὸς παντὸς ἀνθρώπου ἥκει ἐναντίον μου, ὅτι ἐπλήσθη ἡ γῆ ἀδικίας ἀπʼ αὐτῶν· καὶ ἰδοὺ ἐγὼ καταφθείρω αὐτοὺς καὶ τὴν γῆν.

\vs{14}Ποίησον οὖν σεαυτῷ κιβωτὸν ἐκ ξύλων τετραγώνων· νοσσιὰς ποιήσεις τὴν κιβωτόν· καὶ ἀσφαλτώσεις αὐτὴν ἔσωθεν καὶ ἔξωθεν τῇ ἀσφάλτῳ.
\vs{15}Καὶ οὕτω ποιήσεις τὴν κιβωτόν· τριακοσίων πήχεων τὸ μῆκος τῆς κιβωτοῦ, καὶ πεντήκοντα πήχεων τὸ πλάτος, καὶ τριάκοντα πήχεων τὸ ὕψος αὐτῆς.
\vs{16}Ἐπισυνάγων ποιήσεις τὴν κιβωτὸν, καὶ εἰς πῆχυν συντελέσεις αὐτὴν ἄνωθεν· τὴν δὲ θύραν τῆς κιβωτοῦ ποιήσεις ἐκ πλαγίων· κατάγαια διώροφα καὶ τριώροφα ποιήσεις αὐτήν.
\vs{17}Ἐγὼ δὲ ἰδοὺ ἐπάγω τὸν κατακλυσμὸν, ὕδωρ ἐπὶ τὴν γῆν, καταφθεῖραι πᾶσαν σάρκα, ἐν ᾗ ἐστι πνεῦμα ζωῆς ὑποκάτω τοῦ οὐρανοῦ· καὶ ὅσα ἂν ᾖ ἐπὶ τῆς γῆς, τελευτήσει.

\vs{18}Καὶ στήσω τὴν διαθήκην μου μετά σου· εἰσελεύσῃ δὲ εἰς τὴν κιβωτὸν σὺ, καὶ οἱ υἱοί σου, καὶ ἡ γυνή σου, καὶ αἱ γυναῖκες τῶν υἱῶν σου μετά σου.
\vs{19}Καὶ ἀπὸ πάντων τῶν κτηνῶν, καὶ ἀπὸ πάντων τῶν ἑρπετῶν, καὶ ἀπὸ πάντων τῶν θηρίων, καὶ ἀπὸ πάσης σαρκὸς δύο δύο ἀπὸ πάντων εἰσάξεις εἰς τὴν κιβωτὸν, ἵνα τρέφῃς μετὰ σεαυτοῦ· ἄρσεν καὶ θῆλυ ἔσονται.
\vs{20}Ἀπὸ πάντων τῶν ὀρνέων τῶν πετεινῶν κατὰ γένος, καὶ ἀπὸ πάντων τῶν κτηνῶν κατὰ γένος, καὶ ἀπὸ πάντων τῶν ἑρπετῶν τῶν ἑρπόντων ἐπὶ τῆς γῆς κατὰ γένος αὐτῶν, δύο δύο ἀπὸ πάντων εἰσελεύσονται πρὸς σὲ τρέφεσθαι μετά σου, ἄρσεν καὶ θῆλυ.
\vs{21}Σὺ δὲ λήψῃ σεαυτῷ ἀπὸ πάντων τῶν βρωμάτων ἃ ἔδεσθε, καὶ συνάξεις πρὸς σεαυτὸν, καὶ ἔσται σοι καὶ ἐκείνοις φαγεῖν.
\vs{22}Καὶ ἐποίησε Νῶε πάντα ὅσα ἐνετείλατο αὐτῷ Κύριος ὁ Θεὸς, οὕτως ἐποίησε.

\ch{7}
Καὶ εἶπε Κύριος ὁ Θεὸς πρὸς Νῶε, εἴσελθε σὺ καὶ πᾶς ὁ οἶκός σου εἰς τὴν κιβωτὸν, ὅτι σὲ εἶδον δίκαιον ἐναντίον μου ἐν τῇ γενεᾷ ταύτῃ.
\vs{2}Ἀπὸ δὲ τῶν κτηνῶν τῶν καθαρῶν εἰσάγαγε πρὸς σὲ ἑπτὰ ἑπτὰ ἄρσεν καὶ θῆλυ, ἀπὸ δὲ τῶν κτηνῶν τῶν μὴ καθαρῶν δύο δύο ἄρσεν καὶ θῆλυ.
\vs{3}Καὶ ἀπὸ τῶν πετεινῶν τοῦ οὐρανοῦ τῶν καθαρῶν ἑπτὰ ἑπτὰ ἄρσεν καὶ θῆλυ, καὶ ἀπὸ πάντων τῶν πετεινῶν τῶν μὴ καθαρῶν δύο δύο ἄρσεν καὶ θῆλυ, διαθρέψαι σπέρμα ἐπὶ πᾶσαν τὴν γῆν.
\vs{4}Ἔτι γὰρ ἡμερῶν ἑπτὰ ἐγὼ ἐπάγω ὑετὸν ἐπὶ τὴν γῆν, τεσσαράκοντα ἡμέρας καὶ τεσσαράκοντα νύκτας· καὶ ἐξαλείψω πᾶν τὸ ἀνάστημα, ὃ ἐποίησα ἀπὸ προσώπου πάσης τῆς γῆς.
\vs{5}Καὶ ἐποίησε Νῶε πάντα, ὅσα ἐνετείλατο αὐτῷ Κύριος ὁ Θεός.
\vs{6}Νῶε δὲ ἦν ἐτῶν ἑξακοσίων, καὶ ὁ κατακλυσμὸς τοῦ ὕδατος ἐγένετο ἐπὶ τῆς γῆς.
\vs{7}Εἰσῆλθε δὲ Νῶε καὶ οἱ υἱοὶ αὐτοῦ, καὶ ἡ γυνὴ αὐτοῦ, καὶ αἱ γυναῖκες τῶν υἱῶν αὐτοῦ μετʼ αὐτοῦ εἰς τὴν κιβωτὸν, διὰ τὸ ὕδωρ τοῦ κατακλυσμοῦ.
\vs{8}Καὶ ἀπὸ τῶν πετεινῶν τῶν καθαρῶν, καὶ ἀπὸ τῶν πετεινῶν τῶν μὴ καθαρῶν, καὶ ἀπὸ τῶν κτηνῶν τῶν καθαρῶν, καὶ ἀπὸ τῶν κτηνῶν τῶν μὴ καθαρῶν, καὶ ἀπὸ πάντων τῶν ἑρπόντων ἐπὶ τῆς γῆς,
\vs{9}δύο δύο εἰσῆλθον πρὸς Νῶε εἰς τὴν κιβωτὸν ἄρσεν καὶ θῆλυ, καθὰ ἐνετείλατο ὁ Θεὸς τῷ Νῶε.
\vs{10}Καὶ ἐγένετο μετὰ τὰς ἑπτὰ ἡμέρας, καὶ τὸ ὕδωρ τοῦ κατακλυσμοῦ ἐγένετο ἐπὶ τῆς γῆς.
\vs{11}Ἐν τῷ ἑξακοσιοστῷ ἔτει ἐν τῇ ζωῇ τοῦ Νῶε, τοῦ δευτέρου μηνὸς, ἑβδόμῃ καὶ εἰκάδι τοῦ μηνὸς, τῇ ἡμέρᾳ ταύτῃ ἐῤῥάγησαν πᾶσαι αἱ πηγαὶ τῆς ἀβύσσου, καὶ οἱ καταῤῥάκται τοῦ οὐρανοῦ ἠνεῴχθησαν.
\vs{12}Καὶ ἐγένετο ὁ ὑετὸς ἐπὶ τῆς γῆς τεσσαράκοντα ἡμέρας καὶ τεσσαράκοντα νύκτας.
\vs{13}Ἐν τῇ ἡμέρᾳ ταύτῃ εἰσῆλθε Νῶε, Σὴμ, Χὰμ, Ἰάφεθ, οἱ υἱοὶ Νῶε, καὶ ἡ γυνὴ Νῶε, καὶ αἱ τρεῖς γυναῖκες τῶν υἱῶν αὐτοῦ μετʼ αὐτοῦ, εἰς τὴν κιβωτόν.
\vs{14}Καὶ πάντα τὰ θηρία κατὰ γένος, καὶ πάντα τὰ κτήνη κατὰ γένος, καὶ πᾶν ἑρπετὸν κινούμενον ἐπὶ τῆς γῆς κατὰ γένος, καὶ πᾶν ὄρνεον πετεινὸν κατὰ γένος αὐτοῦ,
\vs{15}εἰσῆλθον πρὸς Νῶε εἰς τὴν κιβωτὸν, δύο δύο ἄρσεν καὶ θῆλυ ἀπὸ πάσης σαρκὸς, ἐν ᾧ ἐστι πνεῦμα ζωῆς.
\vs{16}Καὶ τὰ εἰσπορευόμενα ἄρσεν καὶ θῆλυ ἀπὸ πάσης σαρκὸς εἰσῆλθε, καθὰ ἐνετείλατο ὁ Θεὸς τῷ Νῶε· καὶ ἔκλεισε Κύριος ὁ Θεὸς τὴν κιβωτὸν ἔξωθεν αὐτοῦ.

\vs{17}Καὶ ἐγένετο ὁ κατακλυσμὸς τεσσαράκοντα ἡμέρας καὶ τεσσαράκοντα νύκτας ἐπὶ τῆς γῆς· καὶ ἐπεπληθύνθη τὸ ὕδωρ· καὶ ἐπῇρε τὴν κιβωτὸν, καὶ ὑψώθη ἀπὸ τῆς γῆς.
\vs{18}Καὶ ἐπεκράτει τὸ ὕδωρ, καὶ ἐπληθύνετο σφόδρα ἐπὶ τῆς γῆς· καὶ ἐπεφέρετο ἡ κιβωτὸς ἐπάνω τοῦ ὕδατος.
\vs{19}Τὸ δὲ ὕδωρ ἐπεκράτει σφόδρα σφόδρα ἐπὶ τῆς γῆς· καὶ ἐκάλυψε πάντα τὰ ὄρη τὰ ὑψηλὰ, ἃ ἦν ὑποκάτω τοῦ οὐρανοῦ.
\vs{20}Πεντεκαίδεκα πήχεις ὑπεράνω ὑψώθη τὸ ὕδωρ· καὶ ἐπεκάλυψε πάντα τὰ ὄρη τὰ ὑψηλά.
\vs{21}Καὶ ἀπέθανε πᾶσα σὰρξ κινουμένη ἐπὶ τῆς γῆς τῶν πετεινῶν, καὶ τῶν κτηνῶν, καὶ τῶν θηρίων· καὶ πᾶν ἑρπετὸν κινούμενον ἐπὶ τῆς γῆς, καὶ πᾶς ἄνθρωπος.
\vs{22}Καὶ πάντα ὅσα ἔχει πνοὴν ζωῆς, καὶ πᾶν ὃ ἦν ἐπὶ τῆς ξηρᾶς, ἀπέθανε.
\vs{23}Καὶ ἐξήλειψε πᾶν τὸ ἀνάστημα, ὃ ἦν ἐπὶ προσώπου τῆς γῆς, ἀπὸ ἀνθρώπου ἕως κτήνους, καὶ ἑρπετῶν, καὶ τῶν πετεινῶν τοῦ οὐρανοῦ· καὶ ἐξηλείφησαν ἀπὸ τῆς γῆς· καὶ κατελείφθη μόνος Νῶε, καὶ οἱ μετʼ αὐτοῦ ἐν τῇ κιβωτῷ.
\vs{24}Καὶ ὑψώθη τὸ ὕδωρ ἐπὶ τῆς γῆς ἡμέρας ἑκατὸν πεντήκοντα.

\ch{8}
Καὶ ἀνεμνήσθη ὁ Θεὸς τοῦ Νῶε, καὶ πάντων τῶν θηρίων, καὶ πάντων τῶν κτηνῶν, καὶ πάντων τῶν πετεινῶν, καὶ πάντων τῶν ἑρπετῶν τῶν ἑρπόντων, ὅσα ἦν μετʼ αὐτοῦ ἐν τῇ κιβωτῷ· καὶ ἐπήγαγεν ὁ Θεὸς πνεῦμα ἐπὶ τὴν γῆν, καὶ ἐκόπασε τὸ ὕδωρ.
\vs{2}Καὶ ἐπεκαλύφθησαν αἱ πηγαὶ τῆς ἀβύσσου, καὶ οἱ καταῤῥάκται τοῦ οὐρανοῦ, καὶ συνεσχέθη ὁ ὑετὸς ἀπὸ τοῦ οὐρανοῦ.
\vs{3}Καὶ ἐνεδίδου τὸ ὕδωρ πορευόμενον ἀπὸ τῆς γῆς· καὶ ἠλαττονοῦτο τὸ ὕδωρ μετὰ πεντήκοντα καὶ ἑκατὸν ἡμέρας.
\vs{4}Καὶ ἐκάθισεν ἡ κιβωτὸς ἐν μηνὶ τῷ ἑβδόμῳ, ἑβδόμῃ καὶ εἰκάδι τοῦ μηνὸς, ἐπὶ τὰ ὄρη τὰ Ἀραράτ.
\vs{5}Τὸ δὲ ὕδωρ ἠλαττονοῦτο ἕως τοῦ δεκάτου μηνός. Καὶ ἐν τῷ δεκάτῳ μηνὶ, τῇ πρώτῃ τοῦ μηνὸς, ὤφθησαν αἱ κεφαλαὶ τῶν ὀρέων.
\vs{6}Καὶ ἐγένετο μετὰ τεσσαράκοντα ἡμέρας ἠνέῳξε Νῶε τὴν θυρίδα τῆς κιβωτοῦ, ἣν ἐποίησε.
\vs{7}Καὶ ἀπέστειλε τὸν κόρακα· καὶ ἐξελθὼν, οὐκ ἀνέστρεψεν ἕως τοῦ ξηρανθῆναι τὸ ὕδωρ ἀπὸ τῆς γῆς.
\vs{8}Καὶ ἀπέστειλε τὴν περιστερὰν ὀπίσω αὐτοῦ, ἰδεῖν εἰ κεκόπακε τὸ ὕδωρ ἀπὸ τῆς γῆς.
\vs{9}Καὶ οὐχ εὑροῦσα ἡ περιστερὰ ἀνάπαυσιν τοῖς ποσὶν αὐτῆς, ἀνέστρεψε πρὸς αὐτὸν εἰς τὴν κιβωτὸν, ὅτι ὕδωρ ἦν ἐπὶ πᾶν τὸ πρόσωπον τῆς γῆς· καὶ ἐκτείνας τὴν χεῖρα ἔλαβεν αὐτὴν, καὶ εἰσήγαγεν αὐτὴν πρὸς ἑαυτὸν εἰς τὴν κιβωτόν.
\vs{10}Καὶ ἐπισχὼν ἔτι ἡμέρας ἑπτὰ ἑτέρας, πάλιν ἐξαπέστειλε τὴν περιστερὰν ἐκ τῆς κιβωτοῦ.
\vs{11}Καὶ ἀνέστρεψε πρὸς αὐτὸν ἡ περιστερὰ τὸ πρὸς ἑσπέραν· καὶ εἶχε φύλλον ἐλαίας κάρφος ἐν τῷ στόματι αὐτῆς· καὶ ἔγνω Νῶε, ὅτι κεκόπακε τὸ ὕδωρ ἀπὸ τῆς γῆς.
\vs{12}Καὶ ἐπισχὼν ἔτι ἡμέρας ἑπτὰ ἑτέρας, πάλιν ἐξαπέστειλε τὴν περιστερὰν, καὶ οὐ προσέθετο τοῦ ἐπιστρέψαι πρὸς αὐτὸν ἔτι.
\vs{13}Καὶ ἐγένετο ἐν τῷ ἑνὶ καὶ ἑξακοσιοστῷ ἔτει ἐν τῇ ζωῇ τοῦ Νῶε, τοῦ πρώτου μηνὸς, μιᾷ τοῦ μηνὸς, ἐξέλιπε τὸ ὕδωρ ἀπὸ τῆς γῆς. Καὶ ἀπεκάλυψε Νῶε τὴν στέγην τῆς κιβωτοῦ, ἣν ἐποίησε· καὶ εἶδεν ὅτι ἐξέλιπε τὸ ὕδωρ ἀπὸ προσώπου τῆς γῆς.
\vs{14}Ἐν δὲ τῷ δευτέρῳ μηνὶ ἐξηράνθη ἡ γῆ, ἑβδόμῃ καὶ εἰκάδι τοῦ μηνός.

\vs{15}Καὶ εἶπε Κύριος ὁ Θεὸς πρὸς Νῶε, λέγων,
\vs{16}Ἔξελθε ἐκ τῆς κιβωτοῦ σὺ, καὶ ἡ γυνή σου, καὶ οἱ υἱοί σου, καὶ αἱ γυναῖκες τῶν υἱῶν σου μετὰ σοῦ,
\vs{17}Καὶ πάντα τὰ θηρία ὅσα ἐστὶ μετὰ σοῦ, καὶ πᾶσα σὰρξ ἀπὸ πετεινῶν ἕως κτηνῶν, καὶ πᾶν ἑρπετὸν κινούμενον ἐπὶ τῆς γῆς, ἐξάγαγε μετὰ σεαυτοῦ. καὶ αὐξάνεσθε καὶ πληθύνεσθε ἐπὶ τῆς γῆς.
\vs{18}Καὶ ἐξῆλθε Νῶε, καὶ ἡ γυνὴ αὐτοῦ, καὶ οἱ υἱοὶ αὐτοῦ, καὶ αἱ γυναῖκες τῶν υἱῶν αὐτοῦ μετʼ αὐτοῦ·
\vs{19}Καὶ πάντα τὰ θηρία, καὶ πάντα τὰ κτήνη, καὶ πᾶν πετεινὸν, καὶ πᾶν ἑρπετὸν κινούμενον ἐπὶ τῆς γῆς κατὰ γένος αὐτῶν, ἐξήλθοσαν ἐκ τῆς κιβωτοῦ.

\vs{20}Καὶ ᾠκοδόμησε Νῶε θυσιαστήριον τῷ Κυρίῳ· καὶ ἔλαβεν ἀπὸ πάντων τῶν κτηνῶν τῶν καθαρῶν, καὶ ἀπὸ πάντων τῶν πετεινῶν τῶν καθαρῶν, καὶ ἀνήνεγκεν εἰς ὁλοκάρπωσιν ἐπὶ τὸ θυσιαστήριον.
\vs{21}Καὶ ὠσφράνθη Κύριος ὁ Θεὸς ὀσμὴν εὐωδίας. Καὶ εἶπε Κύριος ὁ Θεὸς διανοηθείς, οὐ προσθήσω ἔτι καταράσασθαι τὴν γῆν διὰ τὰ ἔργα τῶν ἀνθρώπων· ὅτι ἔγκειται ἡ διάνοια τοῦ ἀνθρώπου ἐπιμελῶς ἐπὶ τὰ πονηρὰ ἐκ νεότητος αὐτοῦ· οὐ προσθήσω οὖν ἔτι πατάξαι πᾶσαν σάρκα ζῶσαν, καθὼς ἐποίησα.
\vs{22}Πάσας τὰς ἡμέρας τῆς γῆς, σπέρμα καὶ θερισμὸς, ψύχος καὶ καῦμα, θέρος καὶ ἔαρ, ἡμέραν καὶ νύκτα, οὐ καταπαύσουσι.

\ch{9}
Καὶ εὐλόγησεν ὁ Θεὸς τὸν Νῶε, καὶ τοὺς υἱοὺς αὐτοῦ· καὶ εἶπεν αὐτοῖς· αὐξάνεσθε καὶ πληθύνεσθε, καὶ πληρώσατε τὴν γῆν, καὶ κατακυριεύσατε αὐτῆς.
\vs{2}Καὶ ὁ τρόμος, καὶ ὁ φόβος ὑμῶν, ἔσται ἐπὶ πᾶσι τοῖς θηρίοις τῆς γῆς, ἐπὶ πάντα τὰ πετεινὰ τοῦ οὐρανοῦ, καὶ ἐπὶ πάντα τὰ κινούμενα ἐπὶ τῆς γῆς, καὶ ἐπὶ πάντας τοὺς ἰχθύας τῆς θαλάσσης· ὑπὸ χεῖρας ὑμῖν δέδωκα.
\vs{3}Καὶ πᾶν ἑρπετὸν, ὅ ἐστι ζῶν, ὑμῖν ἔσται εἰς βρῶσιν· ὡς λάχανα χόρτου δέδωκα ὑμῖν τὰ πάντα.
\vs{4}Πλὴν κρέας ἐν αἵματι ψυχῆς οὐ φάγεσθε.
\vs{5}Καὶ γὰρ τὸ ὑμέτερον αἷμα τῶν ψυχῶν ὑμῶν ἐκ χειρὸς πάντων τῶν θηρίων ἐκζητήσω αὐτό· καὶ ἐκ χειρὸς ἀνθρώπου ἀδελφοῦ ἐκζητήσω τὴν ψυχὴν τοῦ ἀνθρώπου.
\vs{6}Ὁ ἐκχέων αἷμα ἀνθρώπου, ἀντὶ τοῦ αἵματος αὐτοῦ ἐκχυθήσεται, ὅτι ἐν εἰκόνι Θεοῦ ἐποίησα τὸν ἄνθρωπον.
\vs{7}Ὑμεῖς δὲ αὐξάνεσθε, καὶ πληθύνεσθε, καὶ πληρώσατε τὴν γῆν, καὶ κατακυριεύσατε αὐτῆς.

\vs{8}Καὶ εἶπεν ὁ Θεὸς τῷ Νῶε καὶ τοῖς υἱοῖς αὐτοῦ, μετʼ αὐτοῦ λέγων,
\vs{9}καὶ ἰδοὺ ἐγὼ ἀνίστημι τὴν διαθήκην μου ὑμῖν, καὶ τῷ σπέρματι ὑμῶν μεθʼ ὑμᾶς,
\vs{10}καὶ πάσῃ ψυχῇ ζώσῃ μεθʼ ὑμῶν, ἀπὸ ὀρνέων, καὶ ἀπὸ κτηνῶν· καὶ πᾶσι τοῖς θηρίοις τῆς γῆς, ὅσα ἐστὶ μεθʼ ὑμῶν ἀπὸ πάντων τῶν ἐξελθόντων ἐκ τῆς κιβωτοῦ.
\vs{11}Καὶ στήσω τὴν διαθήκην μου πρὸς ὑμᾶς· καὶ οὐκ ἀποθανεῖται πᾶσα σὰρξ ἔτι ἀπὸ τοῦ ὕδατος τοῦ κατακλυσμοῦ· καὶ οὐκ ἔτι ἔσται κατακλυσμὸς ὕδατος, καταφθεῖραι πᾶσαν τὴν γῆν.
\vs{12}Καὶ εἶπε Κύριος ὁ Θεὸς πρὸς Νῶε· τοῦτο τὸ σημεῖον τῆς διαθήκης, ὃ ἐγὼ δίδωμι ἀνὰ μέσον ἐμοῦ καὶ ὑμῶν, καὶ ἀνὰ μέσον πάσης ψυχῆς ζώσης, ἥ ἐστι μεθʼ ὑμῶν εἰς γενεὰς αἰωνίους.
\vs{13}Τὸ τόξον μου τίθημι ἐν τῇ νεφέλῃ, καὶ ἔσται εἰς σημεῖον διαθήκης ἀνὰ μέσον ἐμοῦ καὶ τῆς γῆς.
\vs{14}Καὶ ἔσται ἐν τῷ συννεφεῖν με νεφέλας ἐπὶ τὴν γῆν, ὀφθήσεται τὸ τόξον ἐν τῇ νεφέλῃ.
\vs{15}Καὶ μνησθήσομαι τῆς διαθήκης μου, ἥ ἐστιν ἀνὰ μέσον ἐμοῦ καὶ ὑμῶν, καὶ ἀνὰ μέσον πάσης ψυχῆς ζώσης ἐν πάσῃ σαρκί· καὶ οὐκ ἔσται ἔτι τὸ ὕδωρ εἰς κατακλυσμὸν, ὥστε ἐξαλεῖψαι πᾶσαν σάρκα.
\vs{16}Καὶ ἔσται τὸ τόξον μου ἐν τῇ νεφέλῃ· καὶ ὄψομαι τοῦ μνησθῆναι διαθήκην αἰώνιον ἀνὰ μέσον ἐμοῦ καὶ τῆς γῆς, καὶ ἀνὰ μέσον ψυχῆς ζώσης ἐν πάσῃ σαρκὶ, ἥ ἐστιν ἐπὶ τῆς γῆς.
\vs{17}Καὶ εἶπεν ὁ Θεὸς τῷ Νῶε, τοῦτο τὸ σημεῖον τῆς διαθήκης, ἧς διεθέμην ἀνὰ μέσον ἐμοῦ, καὶ ἀνὰ μέσον πάσης σαρκὸς, ἥ ἐστιν ἐπὶ τῆς γῆς.

\vs{18}Ἦσαν δὲ οἱ υἱοὶ Νῶε, οἱ ἐξελθόντες ἐκ τῆς κιβωτοῦ, Σὴμ, Χὰμ, Ἰάφεθ. Χὰμ δὲ ἦν πατὴρ Χαναάν.
\vs{19}Τρεῖς οὗτοί εἰσιν υἱοὶ Νῶε· ἀπὸ τούτων διεσπάρησαν ἐπὶ πᾶσαν τὴν γῆν.
\vs{20}Καὶ ἤρξατο Νῶε ἄνθρωπος γεωργὸς γῆς, καὶ ἐφύτευσεν ἀμπελῶνα.
\vs{21}Καὶ ἔπιεν ἐκ τοῦ οἴνου, καὶ ἐμεθύσθη, καὶ ἐγυμνώθη ἐν τῷ οἴκῳ αὐτοῦ.
\vs{22}Καὶ εἶδε Χὰμ ὁ πατὴρ Χαναὰν τὴν γύμνωσιν τοῦ πατρὸς αὐτοῦ, καὶ ἐξελθὼν ἀνήγγειλε τοῖς δυσὶν ἀδελφοῖς αὐτοῦ ἔξω.
\vs{23}Καὶ λαβόντες Σὴμ καὶ Ἰάφεθ τὸ ἱμάτιον, ἐπέθεντο ἐπὶ τὰ δύο νῶτα αὐτῶν, καὶ ἐπορεύθησαν ὀπισθοφανῶς, καὶ συνεκάλυψαν τὴν γύμνωσιν τοῦ πατρὸς αὐτῶν· καὶ τὸ πρόσωπον αὐτῶν ὀπισθοφανῶς, καὶ τὴν γύμνωσιν τοῦ πατρὸς αὐτῶν οὐκ εἶδον.
\vs{24}Ἐξένηψε δὲ Νῶε ἀπὸ τοῦ οἴνου, καὶ ἔγνω ὅσα ἐποίησεν αὐτῷ ὁ υἱὸς αὐτοῦ ὁ νεώτερος.
\vs{25}Καὶ εἶπεν, ἐπικατάρατος Χαναὰν παῖς· οἰκέτης ἔσται τοῖς ἀδελφοῖς αὐτοῦ.
\vs{26}Καὶ εἶπεν, εὐλογητὸς Κύριος ὁ Θεὸς τοῦ Σήμ· καὶ ἔσται Χαναὰν παῖς οἰκέτης αὐτοῦ.
\vs{27}Πλατύναι ὁ Θεὸς τῷ Ἰάφεθ, καὶ κατοικησάτω ἐν τοῖς οἴκοις τοῦ Σήμ· καὶ γενηθήτω Χαναὰν παῖς αὐτοῦ.

\vs{28}Ἔζησε δὲ Νῶε μετὰ τὸν κατακλυσμὸν ἔτη τριακόσια πεντήκοντα.
\vs{29}Καὶ ἐγένοντο πᾶσαι αἱ ἡμέραι Νῶε ἐννακόσια πεντήκοντα ἔτη· καὶ ἀπέθανεν.

\ch{10}
Αὗται δὲ αἱ γενέσεις τῶν υἱῶν Νῶε, Σὴμ, Χὰμ, Ἰάφεθ· καὶ ἐγεννήθησαν αὐτοῖς υἱοὶ μετὰ τὸν κατακλυσμόν.

\vs{2}Υἱοὶ Ἰάφεθ, Γαμὲρ, καὶ Μαγὼγ, καὶ Μαδοὶ, καὶ Ἰωύαν, καὶ Ἐλισὰ, καὶ Θοβὲλ, καὶ Μοσὸχ, καὶ Θείρας.
\vs{3}Καὶ υἱοὶ Γαμὲρ, Ἀσχανὰζ, καὶ Ῥιφὰθ, καὶ Θοργαμά.
\vs{4}Καὶ υἱοὶ Ἰωύαν, Ἐλισὰ, καὶ Θάρσεις, Κήτιοι, Ῥόδὶοι.
\vs{5}Ἐκ τούτων ἀφωρίσθησαν νῆσοι τῶν ἐθνῶν ἐν τῇ γῇ αὐτῶν· ἕκαστος κατὰ γλῶσσαν ἐν ταῖς φυλαῖς αὐτῶν, καὶ ἐν τοῖς ἔθνεσιν αὐτῶν.

\vs{6}Υἱοὶ δὲ Χὰμ, Χοὺς, καὶ Μεσραῒν, Φοὺδ, καὶ Χαναάν.
\vs{7}Υἱοὶ δὲ Χοὺς, Σαβὰ, καὶ Εὐϊλὰ, καὶ Σαβαθὰ, καὶ Ῥεγμὰ, καὶ Σαβαθακά· υἱοὶ δὲ Ῥεγμὰ, Σαβὰ, καὶ Δαδάν.
\vs{8}Χοὺς δὲ ἐγέννησε τὸν Νεβρώδ· οὗτος ἤρξατο εἶναι γίγας ἐπὶ τῆς γῆς.
\vs{9}Οὗτος ἦν γίγας κυνηγὸς ἐναντίον Κυρίου τοῦ Θεοῦ· διὰ τοῦτο ἐροῦσιν, ὡς Νεβρὼδ γίγας κυνηγὸς ἐναντίον Κυρίου.
\vs{10}Καὶ ἐγένετο ἀρχὴ τῆς βασιλείας αὐτοῦ Βαβυλὼν, καὶ Ὀρὲχ, καὶ Ἀρχὰδ, καὶ Χαλάννη, ἐν τῇ γῇ Σεναάρ.
\vs{11}Ἐκ τῆς γῆς ἐκείνης ἐξῆλθεν Ἀσσούρ· καὶ ᾠκοδόμησε τὴν Νινευῒ, καὶ τὴν Ῥοωβὼθ πόλιν, καὶ τὴν Χαλὰχ,
\vs{12}καὶ τὴν Δασὴ ἀνὰ μέσον Νινευῒ, καὶ ἀνὰ μέσον Χαλάχ· αὕτη ἡ πόλις μεγάλη.
\vs{13}Καὶ Μεσραῒν ἐγέννησε τοὺς Λουδιεὶμ, καὶ τοὺς Νεφθαλεὶμ, καὶ τοὺς Ἐνεμετιεὶμ, καὶ τοὺς Λαβιεὶμ,
\vs{14}καὶ τοὺς Πατροσωνιεὶμ, καὶ τοὺς Χασμωνιεὶμ, ὅθεν ἐξῆλθε Φυλιστιεὶμ, καὶ τοὺς Γαφθοριείμ.
\vs{15}Χαναὰν δὲ ἐγέννησε τὸν Σιδῶνα πρωτότοκον αὐτοῦ, καὶ τὸν Χετταῖον,
\vs{16}καὶ τὸν Ἰεβουσαῖον, καὶ τὸν Ἀμοῤῥαῖον, καὶ τὸν Γεργεσαῖον,
\vs{17}καὶ τὸν Εὐαῖον, καὶ τὸν Ἀρουκαῖον, καὶ τὸν Ἀσενναῖον,
\vs{18}καὶ τον Ἀράδιον, καὶ τὸν Σαμαραῖον, καὶ τὸν Ἀμαθί. Καὶ μετὰ τοῦτο διεσπάρησαν αἱ φυλαὶ τῶν Χαναναίων.
\vs{19}Καὶ ἐγένετο τὰ ὅρια τῶν Χαναναίων ἀπὸ Σιδῶνος ἕως ἐλθεῖν εἰς Γεραρὰ καὶ Γαζὰν, ἕως ἐλθεῖν ἕως Σοδόμων καὶ Γομόῤῥας, Ἀδαμὰ καὶ Σεβωῒμ ἕως Δασά.
\vs{20}Οὗτοι υἱοὶ Χὰμ, ἐν ταῖς φυλαῖς αὐτῶν, κατὰ γλώσσας αὐτῶν, ἐν ταῖς χώραις αὐτῶν, καὶ ἐν τοῖς ἔθνεσιν αὐτῶν.

\vs{21}Καὶ τῷ Σὴμ ἐγεννήθη καὶ αὐτῷ πατρὶ πάντων τῶν υἱῶν Ἕβερ, ἀδελφῷ Ἰάφεθ τοῦ μείζονος.
\vs{22}Υἱοὶ Σὴμ, Ἐλὰμ, καὶ Ἀσσοὺρ, καὶ Ἀρφαξὰδ, καὶ Λοὺδ, καὶ Ἀρὰμ, καὶ Καϊνᾶν.
\vs{23}Καὶ υἱοὶ Ἀρὰμ, Οὒζ, καὶ Οὒλ, καὶ Γατὲρ, καὶ Μοσόχ.
\vs{24}Καὶ Ἀρφαξὰδ ἐγέννησε τὸν Καϊνᾶν, καὶ Καϊνᾶν ἐγέννησε τὸν Σαλά· Σαλὰ δὲ ἐγέννησε τὸν Ἕβερ.
\vs{25}Καὶ τῷ Ἕβερ ἐγεννήθησαν δύο υἱοί· ὄνομα τῷ ἑνὶ, Φαλὲγ, ὅτι ἐν ταῖς ἡμέραις αὐτοῦ διεμερίσθη ἡ γῆ· καὶ ὄνομα τῷ ἀδελφῷ αὐτοῦ Ἰεκτάν.
\vs{26}Ἰεκτὰν δὲ ἐγέννησε τὸν Ἐλμωδὰδ, καὶ Σαλὲθ, καὶ τὸν Σαρμὼθ, καὶ Ἰαρὰχ,
\vs{27}καὶ Ὁδοῤῥὰ, καὶ Αἰβὴλ, καὶ Δεκλὰ, καὶ Εὐὰλ,
\vs{28}καὶ Ἀβιμαὲλ, καὶ Σαβὰ,
\vs{29}καὶ Οὐφεὶρ, καὶ Εὑεϊλὰ, καὶ Ἰωβάβ· πάντες οὗτοι υἱοὶ Ἰεκτάν.
\vs{30}Καὶ ἐγένετο ἡ κατοίκησις αὐτῶν, ἀπὸ Μασσῆ ἕως ἐλθεῖν εἰς Σαφηρὰ ὄρος ἀνατολῶν.
\vs{31}Οὗτοι υἱοὶ Σὴμ, ἐν ταῖς φυλαῖς αὐτῶν, κατὰ γλώσσας αὐτῶν, ἐν ταῖς χώραις αὐτῶν, καὶ ἐν τοῖς ἔθνεσιν αὐτῶν.
\vs{32}Αὗται αἱ φυλαὶ υἱῶν Νῶε κατὰ γενέσεις αὐτῶν, κατὰ ἔθνη αὐτῶν· ἀπὸ τούτων διεσπάρησαν νῆσοι τῶν ἐθνῶν ἐπὶ τῆς γῆς μετὰ τὸν κατακλυσμόν.

\ch{11}
Καὶ ἦν πᾶσα ἡ γῆ χεῖλος ἓν, καὶ φωνὴ μία πᾶσι.
\vs{2}Καὶ ἐγένετο ἐν τῷ κινῆσαι αὐτοὺς ἀπὸ ἀνατολῶν, εὗρον πεδίον ἐν γῇ Σεναὰρ, καὶ κατῴκησαν ἐκεῖ.
\vs{3}Καὶ εἶπεν ἄνθρωπος τῷ πλησίον αὐτοῦ, δεῦτε πλινθεύσωμεν πλίνθους, καὶ ὀπτήσωμεν αὐτὰς πυρί· καὶ ἐγένετο αὐτοῖς ἡ πλίνθος εἰς λίθον, καὶ ἄσφαλτος ἦν αὐτοῖς ὁ πηλός.
\vs{4}Καὶ εἶπαν, δεῦτε οἰκοδομήσωμεν ἑαυτοῖς πόλιν καὶ πύργον, οὗ ἔσται ἡ κεφαλὴ ἕως τοῦ οὐρανοῦ, καὶ ποιήσωμεν ἑαυτοῖς ὄνομα, πρὸ τοῦ διασπαρῆναι ἡμᾶς ἐπὶ προσώπου πάσης τῆς γῆς.
\vs{5}Καὶ κατέβη Κύριος ἰδεῖν τὴν πόλιν καὶ τὸν πύργον, ὃν ᾠκοδόμησαν οἱ υἱοὶ τῶν ἀνθρώπων.
\vs{6}Καὶ εἶπε Κύριος, ἰδοὺ γένος ἓν, καὶ χεῖλος ἓν πάντων, καὶ τοῦτο ἤρξαντο ποιῆσαι, καὶ νῦν οὐκ ἐκλείψει ἀπʼ αὐτῶν πάντα ὅσα ἂν ἐπιθῶνται ποιεῖν.
\vs{7}Δεῦτε, καὶ καταβάντες συγχέωμεν αὐτῶν ἐκεῖ τὴν γλῶσσαν, ἵνα μὴ ἀκούσωσιν ἕκαστος τὴν φωνὴν τοῦ πλησίον.
\vs{8}Καὶ διέσπειρεν αὐτοὺς Κύριος ἐκεῖθεν ἐπὶ πρόσωπον πάσης τῆς γῆς· καὶ ἐπαύσαντο οἰκοδομοῦντες τῆν πόλιν καὶ τὸν πύργον.
\vs{9}Διὰ τοῦτο ἐκλήθη τὸ ὄνομα αὐτῆς, Σύγχυσις, ὅτι ἐκεῖ συνέχεε Κύριος τὰ χείλη πάσης τῆς γῆς, καὶ ἐκεῖθεν διέσπειρεν αὐτοὺς Κύριος ἐπὶ πρόσωπον πάσης τῆς γῆς.

\vs{10}Καὶ αὗται αἱ γενέσεις Σήμ· καὶ ἦν Σὴμ υἱὸς ἑκατὸν ἐτῶν, ὅτε ἐγέννησε τὸν Ἀρφαξὰδ, δευτέρου ἔτους μετὰ τὸν κατακλυσμόν.
\vs{11}Καὶ ἔζησε Σὴμ, μετὰ τὸ γεννῆσαι αὐτὸν τὸν Ἀρφαξὰδ, ἔτη πεντακόσια, καὶ ἐγέννησεν υἱοὺς καὶ θυγατέρας, καὶ ἀπέθανε.
\vs{12}Καὶ ἔζησεν Ἀρφαξὰδ ἑκατὸν τριακονταπέντε ἔτη, καὶ ἐγέννησε τὸν Καϊνᾶν.
\vs{13}Καὶ ἔζησεν Ἀρφαξὰδ, μετὰ τὸ γεννῆσαι αὐτὸν τὸν Καϊνᾶν, ἔτη τετρακόσια, καὶ ἐγέννησεν υἱοὺς καὶ θυγατέρας, καὶ ἀπέθανε. Καὶ ἔζησε Καϊνᾶν ἑκατὸν καὶ τριάκοντα ἔτη, καὶ ἐγέννησε τὸν Σαλά· καὶ ἔξησε Καϊνᾶν, μετὰ τὸ γεννῆσαι αὐτὸν τὸν Σαλὰ, ἔτη τριακόσια τριάκοντα, καὶ ἐγέννησεν υἱοὺς καὶ θυγατέρας, καὶ ἀπέθανε.
\vs{14}Καὶ ἔζησε Σαλὰ ἑκατὸν τριάκοντα ἔτη, καὶ ἐγέννησε τὸν Ἕβερ.
\vs{15}Καὶ ἔζησε Σαλὰ μετὰ τὸ γεννῆσαι αὐτὸν τὸν Ἕβερ, τριακόσια τριάκοντα ἔτη, καὶ ἐγέννησεν υἱοὺς καὶ θυγατέρας· καὶ ἀπέθανε.
\vs{16}Καὶ ἔζησεν Ἕβερ ἑκατὸν τριάκοντα τέσσαρα ἔτη, καὶ ἐγέννησε τὸν Φαλέγ.
\vs{17}Καὶ ἔξησεν Ἕβερ, μετὰ τὸ γεννῆσαι αὐτὸν τὸν Φαλὲγ, ἔτη διακόσια ἑβδομήκοντα, καὶ ἐγέννησεν υἱοὺς καὶ θυγατέρας, καὶ ἀπέθανε.
\vs{18}Καὶ ἔζησε Φαλὲγ τριάκοντα καὶ ἑκατὸν ἔτη, καὶ ἐγέννησε τὸν Ῥαγαῦ.
\vs{19}Καὶ ἔζησε Φαλὲγ, μετὰ τὸ γεννῆσαι αὐτὸν τὸν Ῥαγαῦ, ἐννέα καὶ διακόσια ἔτη, καὶ ἐγέννησεν υἱοὺς και θυγατέρας, καὶ ἀπέθανε.
\vs{20}Καὶ ἔζησε Ῥαγαὺ ἑκατὸν τριάκοντα καὶ δύο ἔτη, καὶ ἐγέννησε τὸν Σερούχ.
\vs{21}Καὶ ἔζησε Ῥαγαῦ, μετὰ τὸ γεννῆσαι αὐτὸν τὸν Σεροὺχ, διακόσια ἑπτὰ ἔτη, καὶ ἐγέννησεν υἱοὺς καὶ θυγατέρας, καὶ ἀπέθανε.
\vs{22}Καὶ ἔζησε Σεροὺχ ἑκατὸν τριάκοντα ἔτη, καὶ ἐγέννησε τὸν Ναχώρ.
\vs{23}Καὶ ἔζησε Σεροὺχ, μετὰ τὸ γεννῆσαι αὐτὸν τὸν Ναχὼρ, ἔτη διακόσια, καὶ ἐγέννησεν υἱοὺς καὶ θυγατέρας, καὶ ἀπέθανε.
\vs{24}Καὶ ἔζησε Ναχὼρ ἔτη ἑκατὸν ἑβδομηκονταεννέα, καὶ ἐγέννησε τὸν Θάῤῥα.
\vs{25}Καὶ ἔζησε Ναχὼρ, μετὰ τὸ γεννῆσαι αὐτὸν τὸν Θάῤῥα, ἔτη ἑκατὸν εἰκοσιπὲντε, καὶ ἐγεννησεν υἱοὺς καὶ θυγατέρας, καὶ ἀπέθανε.
\vs{26}Καὶ ἔζησε Θάῤῥα ἑβδομήκοντα ἔτη, καὶ ἐγέννησε τὸν Ἄβραμ, καὶ τὸν Ναχὼρ, καὶ τὸν Ἀῤῥάν.

\vs{27}Αὗται δὲ αἱ γενέσεις Θάῤῥα· Θάῤῥα ἐγέννησε τὸν Ἅβραμ, καὶ τὸν Ναχὼρ, καὶ τὸν Ἀῤῥάν· καὶ Ἀῤῥὰν ἐγέννησε τὸν Λώτ.
\vs{28}Καὶ ἀπέθανεν Ἀῤῥὰν ἐνώπιον Θάῤῥα τοῦ πατρὸς αὐτοῦ ἐν τῇ γῇ ᾗ ἐγενήθη, ἐν τῇ χώρᾳ τῶν Χαλδαίων.
\vs{29}Καὶ ἔλαβον Ἅβραμ καὶ Ναχὼρ ἑαυτοῖς γυναῖκας· ὄνομα τῇ γυναικὶ Ἅβραμ, Σάρα, καὶ ὄνομα τῇ γυναικὶ Ναχὼρ, Μελχά, θυγάτηρ Ἀῤῥάν· καὶ πατὴρ Μελχὰ, καὶ πατὴρ Ἰεσχά.
\vs{30}Καὶ ἦν Σάρα στεῖρα, καὶ οὐκ ἐτεκνοποίει.
\vs{31}Καὶ ἔλαβε Θάῤῥα τὸν Ἅβραμ υἱὸν αὐτοῦ, καὶ τὸν Λὼτ υἱὸν Ἀῤῥάν, υἱὸν τοῦ υἱοῦ αὐτοῦ, καὶ τὴν Σάραν τὴν νύμφην αὐτοῦ, γυναῖκα Ἅβραμ τοῦ υἱοῦ αὐτοῦ, καὶ ἐξήγαγεν αὐτοὺς ἐκ τῆς χώρας τῶν Χαλδαίων, πορευθῆναι εἰς γῆν Χαναάν· καὶ ἦλθον ἕως Χαῤῥὰν, καὶ κατῴκησεν ἐκεῖ.
\vs{32}Καὶ ἐγένοντο πᾶσαι αἱ ἡμέραι Θάῤῥα ἐν γῇ Χαῤῥὰν, διακόσια πέντε ἔτη· καὶ ἀπέθανε Θάῤῥα ἐν Χαῤῥάν.

\ch{12}
Καὶ εἶπε Κύριος τῷ Ἅβραμ, ἔξελθε ἐκ τῆς γῆς σου, καὶ ἐκ τῆς συγγενείας σου, καὶ ἐκ τοῦ οἴκου τοῦ πατρός σου, καὶ δεῦρο εἰς τὴν γῆν, ἣν ἄν σοι δείξω.
\vs{2}Καὶ ποιήσω σε εἰς ἔθνος μέγα, καὶ εὐλογήσω σε, καὶ μεγαλυνῶ τὸ ὄνομά σου, καὶ ἔσῃ εὐλογημένος.
\vs{3}Καὶ εὐλογήσω τοὺς εὐλογοῦντάς σε, καὶ τοὺς καταρωμένους σε καταράσομαι, καὶ ἐνευλογηθήσονται ἐν σοὶ πᾶσαι αἱ φυλαὶ τῆς γῆς.
\vs{4}Καὶ ἐπορεύθη Ἅβραμ, καθάπερ ἐλάλησεν αὐτῷ Κύριος, καὶ ᾤχετο μετʼ αὐτοῦ Λώτ· Ἅβραμ δὲ ἦν ἐτῶν ἑβδομηκονταπέντε, ὅτε ἐξῆλθεν ἐκ Χαῤῥάν.
\vs{5}Καὶ ἔλαβεν Ἅβραμ Σάραν τὴν γυναῖκα αὐτοῦ, καὶ τὸν Λὼτ υἱὸν τοῦ ἀδελφοῦ αὐτοῦ, καὶ πάντα τὰ ὑπάρχοντα αὐτῶν ὅσα ἐκτήσαντο, καὶ πᾶσαν ψυχὴν ἣν ἐκτήσαντο, ἐκ Χαῤῥάν, καὶ ἐξήλθοσαν πορευθῆναι εἰς γῆν Χανάαν.
\vs{6}Καὶ διώδευσεν Ἅβραμ τὴν γῆν εἰς τὸ μῆκος αὐτῆς ἕως τοῦ τόπου Συχέμ, ἐπὶ τὴν δρῦν τὴν ὑψηλήν· οἱ δὲ Χαναναῖοι τότε κατῴκουν τὴν γῆν.
\vs{7}Καὶ ὤφθη Κύριος τῷ Ἅβραμ, καὶ εἶπεν αὐτῷ, τῷ σπέρματί σου δώσω τὴν γῆν ταύτην· καὶ ᾠκοδόμησεν ἐκεῖ Ἅβραμ θυσιαστήριον Κυρίῳ τῷ ὀφθέντι αὐτῷ.
\vs{8}Καὶ ἀπέστη ἐκεῖθεν εἰς τὸ ὄρος κατὰ ἀνατολὰς Βαιθήλ· καὶ ἔστησεν ἐκεῖ τὴν σκηνὴν αὐτοῦ ἐν Βαιθὴλ κατὰ θάλασσαν, καὶ Ἀγγαὶ κατὰ ἀνατολάς· καὶ ᾠκοδόμησεν ἐκεῖ θυσιαστήριον τῷ Κυρίῳ, καὶ ἐπεκαλέσατο ἐπὶ τῷ ὀνόματι Κυρίου.
\vs{9}Καὶ ἀπῇρεν Ἅβραμ, καὶ πορευθεὶς ἐστρατοπέδευσεν ἐν τῇ ἐρήμῳ.

\vs{10}Καὶ ἐγένετο λιμὸς ἐπὶ τῆς γῆς· καὶ κατέβη Ἅβραμ εἰς Αἴγυπτον παροικῆσαι ἐκεῖ, ὅτι ἐνίσχυσεν ὁ λιμὸς ἐπὶ τῆς γῆς.
\vs{11}Ἐγένετο δὲ ἡνίκα ἤγγισεν Ἅβραμ εἰσελθεῖν εἰς Αἴγυπτον, εἶπεν Ἅβραμ Σάρα τῇ γυναικὶ, γινώσκω ἐγὼ, ὅτι γυνὴ εὐπρόσωπος εἶ.
\vs{12}Ἔσται οὖν ὡς ἂν ἴδωσί σε οἱ Αἰγύπτιοι, ἐροῦσιν ὅτι γυνὴ αὐτοῦ ἐστιν αὐτὴ, καὶ ἀποκτενοῦσί με, σὲ δὲ περιποιήσονται.
\vs{13}Εἶπον οὖν, ὅτι ἀδελφὴ αὐτοῦ εἰμι, ὅπως ἄν εὖ μοι γένηται διὰ σὲ, καὶ ζήσεται ἡ ψυχή μου ἕνεκέν σου.
\vs{14}Ἐγένετο δὲ, ἡνίκα εἰσῆλθεν Ἅβραμ εἰς Αἴγυπτον, ἰδόντες οἱ Αἰγύπτιοι τὴν γυναῖκα αὐτοῦ, ὅτι καλὴ ἦν σφόδρα.
\vs{15}Καὶ ἴδον αὐτὴν οἱ ἄρχοντες Φαραὼ, καὶ ἐπῄνεσαν αὐτὴν πρὸς Φαραὼ, καὶ εἰσήγαγον αὐτὴν εἰς τὸν οἶκον Φαραώ.
\vs{16}Καὶ τῷ Ἅβραμ εὖ ἐχρήσαντο διʼ αὐτήν· καὶ ἐγένοντο αὐτῷ πρόβατα, καὶ μόσχοι, καὶ ὄνοι, καὶ παῖδες, καὶ παιδίσκαι, καὶ ἡμίονοι, καὶ κάμηλοι.
\vs{17}Καὶ ἤτασεν ὁ Θεὸς τὸν Φαραὼ ἐτασμοῖς μεγάλοις καὶ πονηροῖς, καὶ τὸν οἶκον αὐτοῦ, περὶ Σάρας τῆς γυναικὸς Ἅβραμ.
\vs{18}Καλέσας δὲ Φαραὼ τὸν Ἅβραμ, εἶπεν, τί τοῦτο ἐποίησάς μοι, ὅτι οὐκ ἀπήγγειλάς μοι, ὅτι γυνή σου ἐστίν;
\vs{19}Ἱνατί εἶπας ὅτι ἀδελφή μου ἐστίν; καὶ ἔλαβον αὐτὴν ἐμαυτῷ γυναῖκα· καὶ νῦν ἰδοὺ ἡ γυνή σου ἔναντί σου, λαβὼν ἀπότρεχε.
\vs{20}Καὶ ἐνετείλατο Φαραὼ ἀνδράσι περὶ Ἅβραμ συμπροπέμψαι αὐτὸν, καὶ τὴν γυναῖκα αὐτοῦ, καὶ πάντα ὅσα ἦν αὐτῷ.

\ch{13}
Ἀνέβη δὲ Ἅβραμ ἐξ Αἰγύπτου αὐτὸς, καὶ ἡ γυνὴ αὐτοῦ, καὶ πάντα τὰ αὐτοῦ, καὶ Λὼτ μετʼ αὐτοῦ, εἰς τὴν ἔρημον.
\vs{2}Ἅβραμ δὲ ἦν πλούσιος σφόδρα κτήνεσι, καὶ ἀργυρίῳ, καὶ χρυσίῳ.
\vs{3}Καὶ ἐπορεύθη ὅθεν ἦλθεν εἰς τὴν ἔρημον ἕως Βαιθὴλ, ἕως τοῦ τόπου οὗ ἦν ἡ σκηνὴ αὐτοῦ τὸ πρότερον, ἀνὰ μέσον Βαιθὴλ καὶ ἀνὰ μέσον Ἀγγαί,
\vs{4}εἰς τὸν τόπον τοῦ θυσιαστηρίου, οὗ ἐποίησεν ἐκεῖ τὴν ἀρχὴν, καὶ ἐπεκαλέσατο ἐκεῖ Ἅβραμ τὸ ὄνομα τοῦ Κυρίου.
\vs{5}Καὶ Λὼτ τῷ συμπορευομένῳ μετὰ Ἅβραμ ἦν πρόβατα, καὶ βόες, καὶ σκηναί.
\vs{6}Καὶ οὐκ ἐχώρει αὐτοὺς ἡ γῆ κατοικεῖν ἅμα, ὅτι ἦν τὰ ὑπάρχοντα αὐτῶν πολλά· καὶ οὐκ ἐχώρει αὐτοὺδ ἡ γῆ κατοικεῖν ἅμα.
\vs{7}Καὶ ἐγενετο μάχη ἀνὰ μέσον τῶν ποιμένων τῶν κτηνῶν τοῦ Ἅβραμ, καὶ ἀνὰ μέσον τῶν ποιμένων τῶν κτηνῶν τοῦ Λώτ· οἱ δὲ Χαναναῖοι καὶ οἱ Φερεζαῖοι τότε κατῴκουν τὴν γῆν.
\vs{8}Εἶπε δὲ Ἅβραμ τῷ Λὼτ, μὴ ἔστω μάχη ἀνὰ μέσον ἐμοῦ καὶ σοῦ, καὶ ἀνὰ μέσον τῶν ποιμένων μου καὶ ἀνὰ μέσον τῶν ποιμένων σοῦ, ὅτι ἄνθρωποι ἀδελφοὶ ἐσμὲν ἡμεῖς.
\vs{9}Οὐκ ἰδοὺ πᾶσα ἡ γῆ ἐναντίον σου ἐστί; διαχωρίσθητι ἀπʼ ἐμοῦ· εἰ σὺ εἰς ἀριστερὰ, ἐγὼ εἰς δεξιά· εἰ δὲ σὺ εἰς δεξιὰ, ἐγὼ εἰς ἀριστερά.
\vs{10}Καὶ ἐπάρας Λὼτ τοὺς ὀφθαλμοὺς αὐτοῦ, ἐπεῖδε πᾶσαν τὴν περίχωρον τοῦ Ἰορδάνου, ὅτι πᾶσα ἦν ποτιζομένη, πρὸ τοῦ καταστρέψαι τὸν Θεὸν Σόδομα καὶ Γόμοῤῥα, ὡς ὁ παράδεισος τοῦ Θεοῦ, καὶ ὡς ἡ γῆ Αἰγύπτου, ἕως ἐλθεῖν εἰς Ζόγορα.
\vs{11}Καὶ ἐξελέξατο ἑαυτῷ Λὼτ πᾶσαν τὴν περίχωρον τοῦ Ἰορδάνου· καὶ ἀπῇρε Λὼτ ἀπὸ ἀνατολῶν· καὶ διεχωρίσθησαν ἕκαστος ἀπὸ τοῦ ἀδελφοῦ αὐτοῦ.
\vs{12}Ἅβραμ δὲ κατῴκησεν ἐν γῇ Χαναάν· Λὼτ δὲ κατῴκησεν ἐν πόλει τῶν περιχώρων, καὶ ἐσκήνωσεν ἐν Σοδόμοις.
\vs{13}Οἱ δὲ ἄνθρωποι οἱ ἐν Σοδόμοις πονηροὶ καὶ ἁμαρτωλοὶ ἐναντίον τοῦ Θεοῦ σφόδρα.
\vs{14}Ὁ δὲ Θεὸς εἶπε τῷ Ἅβραμ μετὰ τὸ διαχωρισθῆναι τὸν Λὼτ ἀπʼ αὐτοῦ, ἀνάβλεψον τοῖς ὀφθαλμοῖς σου, καὶ ἴδε ἀπὸ τοῦ τόπου οὗ νῦν σὺ εἶ πρὸς βοῤῥὰν καὶ λίβα καὶ ἀνατολὰς καὶ θάλασσαν·
\vs{15}ὅτι πᾶσαν τὴν γῆν, ἣν σὺ ὁρᾷς, σοὶ δώσω αὐτὴν καὶ τῷ σπέρματί σου ἕως αἰῶνος.
\vs{16}Καὶ ποιήσω τὸ σπέρμα σου, ὡς τὴν ἄμμον τῆς γῆς· εἰ δύναταί τις ἐξαριθμῆσαι τὴν ἄμμον τῆς γῆς, καὶ τὸ σπέρμα σου ἐξαριθμηθήσεται.
\vs{17}Ἀναστὰς διόδευσον τὴν γῆν εἴς τε τὸ μῆκος αὐτῆς καὶ εἰς τὸ πλάτος· ὅτι σοι δώσω αὐτὴν καὶ τῷ σπέρματί σου εἰς τὸν αἰῶνα.
\vs{18}Καὶ ἀποσκηνώσας Ἅβραμ, ἐλθὼν κατῴκησε παρὰ τὴν δρῦν τὴν Μαμβρῆ, ἣ ἦν ἐν Χεβρὼμ, καὶ ᾠκοδόμησεν ἐκεῖ θυσιαστήριον τῷ Κυρίῳ.

\ch{14}
Ἐγένετο δὲ ἐν τῇ βασιλείᾳ τῇ Ἀμαρφὰλ βασιλέως Σενναὰρ, καὶ Ἀριὼχ βασιλέως Ἑλλασὰρ, Χοδολλογομὸρ βασιλεὺς Ἐλὰμ, καὶ Θαργὰλ βασιλεὺς ἐθνῶν,
\vs{2}ἐποίησαν πόλεμον μετὰ Βαλλὰ βασιλέως Σοδόμων, καὶ μετὰ Βαρσὰ βασιλέως Γομόῤῥας, καὶ μετὰ Σενναὰρ βασιλέως Ἀδαμὰ, καὶ μετὰ Συμοβὸρ βασιλέως Σεβωεὶμ, καὶ βασιλέως Βαλάκ· αὕτη ἐστὶ Σηγώρ.
\vs{3}Πάντες οὗτοι συνεφώνησαν ἐπὶ τὴν φάραγγα τὴν ἁλυκήν· αὕτη ἡ θάλασσα τῶν ἁλῶν.
\vs{4}Δώδεκα ἔτη αὐτοὶ ἐδούλευσαν τῷ Χοδολλογομόρ· τῷ δὲ τρισκαιδεκάτῳ ἔτει ἀπέστησαν.
\vs{5}Ἐν δὲ τῷ τεσσαρεσκαιδεκάτῳ ἔτει ἦλθε Χοδολλογομὸρ καὶ οἱ βασιλεῖς μετʼ αὐτοῦ, καὶ κατέκοψαν τοὺς γίγαντας τοὺς ἐν Ἀσταρὼθ, καὶ Καρναῒν, καὶ ἔθνη ἰσχυρὰ ἅμα αὐτοῖς, καὶ τοὺς Ὀμμαίους τοὺς ἐν Σαυῇ τῇ πόλει.
\vs{6}Καὶ τοὺς Χοῤῥαίους τοὺς ἐν τοῖς ὄρεσι Σηεὶρ, ἕως τῆς τερεβίνθου τῆς Φαρὰν, ἥ ἐστιν ἐν τῇ ἐρήμῳ.
\vs{7}Καὶ ἀναστρέψαντες ἦλθον ἐπὶ τὴν πηγὴν τῆς κρίσεως· αὕτη ἐστὶ Κάδης· καὶ κατέκοψαν πάντας τοὺς ἄρχοντας Ἀμαλὴκ, καὶ τοὺς Ἀμοῤῥαίους τοὺς κατοικοῦντας ἐν ʼΑσασονθαμὰρ
\vs{8}Ἐξῆλθε δὲ βασιλεὺς Σοδόμων, καὶ βασιλεὺς Γομόῤῥας, καὶ βασιλεὺς Ἀδαμὰ, καὶ βασιλεὺς Σεβωεὶμ, καὶ βασιλεὺς Βαλάκ· αὕτη ἐστὶ Σηγώρ· καὶ παρετάξαντο αὐτοῖς εἰς πόλεμον ἐν τῇ κοιλάδι, τῇ ἁλυκῇ,
\vs{9}πρὸς Χοδολλογομὸρ βασιλέα Ἐλὰμ, καὶ Θαπγὰλ βασιλέα ἐθνῶν, καὶ Ἀμαρφὰλ βασιλέα Σενναὰρ, καὶ Ἀριὼχ βασιλέα Ἑλλασὰρ, οἱ τέσσαρες βασιλεῖς πρὸς τοὺς πέντε.
\vs{10}Ἡ δὲ κοιλὰς ἡ ἁλυκὴ, φρέατα ἀσφάλτου· ἔφυγε δὲ βασιλεὺς Σοδόμων καὶ βασιλεὺς Γομόῤῥας, καὶ ἐνέπεσαν ἐκεῖ· οἱ δὲ καταλειφθέντες εἰς τὴν ὀρεινὴν ἔφυγον.
\vs{11}Ἔλαβον δὲ τὴν ἵππον πᾶσαν τὴν Σοδόμων καὶ Γομόῤῥας, καὶ πάντα τὰ βρώματα αὐτῶν, καὶ ἀπῆλθον.
\vs{12}Ἔλαβον δὲ καὶ τὸν Λὼτ τὸν υἱὸν τοῦ ἀδελφοῦ Ἅβραμ, καὶ τὴν ἀποσκευὴν αὐτοῦ, καὶ ἀπῴχοντο· ἦν γὰρ κατοικῶν ἐν Σοδόμοις.

\vs{13}Παραγενόμενος δὲ τῶν ἀνασωθέντων τις ἀπήγγειλεν Ἅβραμ τῷ περάτῃ· αὐτὸς δὲ κατῴκει παρὰ τῇ δρυῒ τῇ Μαμβρῇ Ἀμοῤῥαίου τοῦ ἀδελφοῦ Ἐσχὼλ, καὶ τοῦ ἀδελφοῦ Αὐνὰν, οἳ ἦσαν συνωμόται τοῦ Ἅβραμ.
\vs{14}Ἀκούσας δὲ Ἅβραμ ὅτι ᾐχμαλώτευται Λὼτ ὁ ἀδελφοῦς αὐτοῦ, ἠρίθμησε τοὺς ἰδίους οἰκογενεῖς αὐτοῦ τριακοσίους δέκα καὶ ὀκτώ· καὶ κατεδίωξεν ὀπίσω αὐτῶν ἕως Δάν.
\vs{15}Καὶ ἐπέπεσεν ἐπʼ αὐτοὺς τὴν νύκτα αὐτὸς, καὶ οἱ παῖδες αὐτοῦ, καὶ ἐπάταξεν αὐτοὺς, καὶ κατεδίωξεν αὐτοὺς ἕως Χοβὰ, ἥ ἐστιν ἐν ἀριστερᾷ Δαμασκοῦ.
\vs{16}Καὶ ἀπέστρεψε πᾶσαν τὴν ἵππον Σοδόμων· καὶ Λὼτ τὸν ἀδελφιδοῦν αὐτοῦ ἀπέστρεψε, καὶ πάντα τὰ ὑπάρχοντα αὐτοῦ, καὶ τὰς γυναῖκας, καὶ τὸν λαόν.
\vs{17}Ἐξῆλθε δὲ βασιλεὺς Σοδόμων εἰς συνάντησιν αὐτῷ, μετὰ τὸ ὑποστρέψαι αὐτὸν ἀπὸ τῆς κοπῆς τοῦ Χοδολλογομὸρ, καὶ τῶν βασιλέων τῶν μετʼ αὐτοῦ εἰς τὴν κοιλάδα τοῦ Σαβύ· τοῦτο ἦν τὸ πεδίον τῶν βασιλέων.

\vs{18}Καὶ Μελχισεδὲκ βασιλεὺς Σαλὴμ ἐξήνεγκεν ἄρτους καὶ οἶνον· ἦν δὲ ἱερεὺς τοῦ Θεοῦ τοῦ ὑψίστου.
\vs{19}Καὶ εὐλόγησε τὸν Ἅβραμ, καὶ εἶπεν, εὐλογημένος Ἅβραμ τῷ Θεῷ τῷ ὑψίστῳ, ὃς ἔκτισε τὸν οὐρανὸν καὶ τὴν γῆν.
\vs{20}Καὶ εὐλογητὸς ὁ Θεὸς ὁ ὕψιστος, ὃς παρέδωκε τοὺς ἐχθρούς σου ὑποχειρίους σοι· καὶ ἔδωκεν αὐτῷ Ἅβραμ δεκάτην ἀπὸ πάντων.
\vs{21}Εἶπε δὲ βασιλεὺς Σοδόμων πρὸς Ἅβραμ, δός μοι τοὺς ἄνδρας, τὴν δὲ ἵππον λάβε σεαυτῷ.
\vs{22}Εἶπε δὲ Ἅβραμ πρὸς τὸν βασιλέα Σοδόμων, ἐκτενῶ τὴν χεῖρά μου πρὸς Κύπιον τὸν Θεὸν τὸν ὕψιστον, ὃς ἔκτισε τὸν οὐρανὸν καὶ τὴν γῆν,
\vs{23}εἰ ἀπὸ σπαρτίου ἕως σφυρωτῆρος ὑποδήματος λήψομαι ἀπὸ πάντων τῶν σῶν, ἵνα μὴ εἴπῃς, ὅτι ἐγὼ ἐπλούτισα τὸν Ἅβραμ.
\vs{24}Πλὴν ὧν ἔφαγον οἱ νεανίσκοι, καὶ τῆς μερίδος τῶν ἀνδρῶν τῶν συμπορευθέντων μετʼ ἐμοῦ Ἐσχὼλ, Αὐνᾶν, Μαμβρῆ· οὗτοι λήψονται μερίδα.

\ch{15}
Μετὰ δὲ τὰ ῥήματα ταῦτα ἐγενήθη ῥῆμα Κυρίου πρὸς Ἅβραμ ἐν ὁράματι, λέγων, μὴ φοβοῦ Ἅβραμ· ἐγὼ ὑπερασπίζω σου· ὁ μισθός σου πολὺς ἔσται σφόδρα.
\vs{2}Δέγει δὲ Ἅβραμ, Δέσποτα Κύριε, τί μοι δώσεις; ἐγὼ δὲ ἀπολύομαι ἄτεκνος· ὁ δὲ υἱὸς Μασὲκ τῆς οἰκογενοῦς μου, οὗτος Δαμασκὸς Ἐλιέζερ.
\vs{3}Καὶ εἶπεν Ἅβραμ, ἐπειδὴ ἐμοὶ οὐκ ἔδωκας σπέρμα, ὁ δὲ οἰκογενής μου κληρονομήσει με.
\vs{4}Καὶ εὐθὺς φωνὴ Κυρίου ἐγένετο πρὸς αὐτὸν, λέγουσα, οὐ κληρονομήσει σε οὗτος· ἀλλʼ ὃς ἐξελεύσεται ἐκ σοῦ, οὗτος κληρονομήσει σε.
\vs{5}Ἐξήγαγε δὲ αὐτὸν ἔξω, καὶ εἶπεν αὐτῷ, ἀνάβλεψον δὴ εἰς τὸν οὐρανὸν, καὶ ἀρίθμησον τοὺς ἀστέρας, εἰ δυνήσῃ ἐξαριθμῆσαι αὐτούς· καὶ εἶπεν, οὕτως ἔσται τὸ σπέρμα σου.
\vs{6}Καὶ ἐπίστευσεν Ἅβραμ τῷ Θεῷ, καὶ ἐλογίσθη αὐτῷ εἰς δικαιοσύνην.
\vs{7}Εἶπε δὲ πρὸς αὐτὸν, ἐγὼ ὁ Θεὸς ὁ ἐξαγαγών σε ἐκ χώρας Χαλδαίων, ὥστε δοῦναί σοι τὴν γῆν ταύτην κληρονομῆσαι.
\vs{8}Εἶπε δέ, Δέσποτα Κύριε, κατὰ τί γνώσομαι, ὅτι κληρονομήσω αὐτήν;
\vs{9}Εἶπε δὲ αὐτῷ, λάβε μοι δάμαλιν τριετίζουσαν, καὶ αἶγα τριετίζουσαν, καὶ κριὸν τριετίζοντα, καὶ τρυγόνα, καὶ περιστεράν.
\vs{10}Ἔλαβε δὲ αὐτῷ πάντα ταῦτα, καὶ διεῖλεν αὐτὰ μέσα, καὶ ἔθηκεν αὐτὰ ἀντιπρόσωπα ἀλλήλοις· τὰ δὲ ὄρνεα οὐ διεῖλε.
\vs{11}Κατέβη δὲ ὄρνεα ἐπὶ τὰ σώματα, ἐπὶ τὰ διχοτομήματα αὐτῶν· καὶ συνεκάθισεν αὐτοῖς Ἅβραμ.
\vs{12}Περὶ δὲ ἡλίου δυσμὰς ἔκστασις ἐπέπεσε τῷ Ἅβραμ, καὶ ἰδοὺ φόβος σκοτεινὸς μέγας ἐπιπίπτει αὐτῷ.
\vs{13}Καὶ ἐῤῥέθη πρὸς Ἅβραμ· γινώσκων γνώσῃ ὅτι πάροικον ἔσται τὸ σπέρμα σου ἐν γῇ οὐκ ἰδίᾳ, καὶ δουλώσουσιν αὐτοὺς, καὶ κακώσουσιν αὐτοὺς, καὶ ταπεινώσουσιν αὐτοὺς, τετρακόσια ἔτη.
\vs{14}Τὸ δὲ ἔθνος, ᾧ ἐὰν δουλεύσωσι, κρινῶ ἐγώ· μετὰ δὲ ταῦτα, ἐξελεύσονται ὧδε μετὰ ἀποσκευῆς πολλῆς.
\vs{15}Σὺ δὲ ἀπελεύσῃ πρὸς τοὺς πατέρας σου ἐν εἰρήνῃ, τραφεὶς ἐν γήρᾳ καλῷ.
\vs{16}Τετάρτῃ δὲ γενεᾷ ἀποστραφήσονται ὧδε· οὔπω γὰρ ἀναπεπλήρωνται αἱ ἁμαρτίαι τῶν Ἀμοῤῥαίων ἕως τοῦ νῦν.
\vs{17}Ἐπεὶ δὲ ὁ ἥλιος ἐγένετο πρὸς δυσμὰς, φλὸξ ἐγένετο· καὶ ἰδοὺ κλίβανος καπνιζόμενος καὶ λαμπάδες πυρός, αἳ διῆλθον ἀνὰ μέσον τῶν διχοτομημάτων τούτων.
\vs{18}Ἐν τῇ ἡμέρᾳ ἐκείνῃ διέθετο Κύριος τῷ Ἅβραμ διαθήκην, λέγων, τῷ σπέρματί σου δώσω τὴν γῆν ταύτην, ἀπὸ τοῦ ποταμοῦ Αἰγύπτου ἕως τοῦ ποταμοῦ τοῦ μεγάλου Εὐφράτου·
\vs{19}Τοὺς Κεναίους, καὶ τοὺς Κενεζαίους, καὶ τοὺς Κεδμωναίους,
\vs{20}καὶ τοὺς Χετταίους, καὶ τοὺς Φερεζαίους, καὶ τοὺς ʼΡαφαεὶν,
\vs{21}καὶ τοὺς Ἀμοῤῥαίους, καὶ τοὺς Χαναναίους, καὶ τοὺς Εὐαίους, καὶ τοὺς Γεργεσαίους, καὶ τοὺς Ἰεβουσαίους.

\ch{16}
Σάρα δὲ ἡ γυνὴ Ἅβραμ οὐκ ἔτικτεν αὐτῷ· ἦν δὲ αὐτῇ παιδίσκη Αἰγυπτία, ᾗ ὄνομα Ἄγαρ.
\vs{2}Εἶπε δὲ Σάρα πρὸς Ἅβραμ, ἰδοὺ συνέκλεισέ με Κύριος τοῦ μὴ τίκτειν· εἴσελθε οὖν πρὸς τὴν παιδίσκην μου, ἵνα τεκνοποιήσωμαι ἐξ αὐτῆς· ὑπήκουσελ δὲ Ἅβραμ τῆς φωνῆς Σάρας.
\vs{3}Καὶ λαβοῦσα Σάρα ἡ γυνὴ Ἅβραμ Ἄγαρ τὴν Αἰγυπτίαν τὴν ἑαυτῆς παιδίσκην, μετὰ δέκα ἔτη τοῦ οἰκῆσαι Ἅβραμ ἐν γῇ Χαναὰν, ἔδωκεν αὐτὴν τῷ Ἅβραμ ἀνδρὶ αὐτῆς αὐτῷ γυναῖκα.
\vs{4}Καὶ εἰσῆλθε πρὸς Ἄγαρ, καὶ συνέλαβε· καὶ εἶδεν ὅτι ἐν γαστρὶ ἔχει, καὶ ἠτιμάσθη ἡ κυρία ἐναντίον αὐτῆς.
\vs{5}Εἶπε δὲ Σάρα πρὸς Ἅβραμ, ἀδικοῦμαι ἐκ σοῦ· ἐγὼ δέδωκα τὴν παιδίσκην μου εἰς τὸν κόλπον σου, ἰδοῦσα δὲ ὅτι ἐν γαστρὶ ἔχει, ἠτιμάσθην ἐναντίον αὐτῆς. κρίναι ὁ Θεὸς ἀνὰ μέσον ἐμοῦ καὶ σου.
\vs{6}Εἶπε δὲ Ἅβραμ πρὸς Σάραν, ἰδοὺ ἡ παιδίσκη σου ἐν ταῖς χερσί σου, χρῶ αὐτῇ ὡς ἄν σοι ἀρεστὸν ᾖ. καὶ ἐκάκωσεν αὐτὴν Σάρα, καὶ ἀπέδρα ἀπὸ προσώπου αὐτῆς.

\vs{7}Εὗρε δὲ αὐτὴν ἄγγελος Κυρίου ἐπὶ τῆς πηγῆς τοῦ ὕδατος ἐν τῇ ἐρήμῳ, ἐπὶ τῆς πηγῆς ἐν τῇ ὁδῷ Σούρ.
\vs{8}Καὶ εἶπεν αὐτῇ ὁ ἄγγελος Κυρίου, Ἄγαρ παιδίσκη Σάρας, πόθεν ἔρχῃ; καὶ ποῦ πορεύῃ; καὶ εἶπεν· ἀπὸ προσώπου Σάρας τῆς κυρίας μου ἐγὼ ἀποδιδράσκω.
\vs{9}Εἶπε δὲ αὐτῇ ὁ ἄγγελος Κυρίου, ἀποστράφηθι πρὸς τὴν κυρίαν σου, καὶ ταπεινώθητι ὑπὸ τὰς χεῖρας αὐτῆς.
\vs{10}Καὶ εἶπεν αὐτῇ ὁ ἄγγελος Κυρίου, πληθύνων πληθυνῶ τὸ σπέρμα σου, καὶ οὐκ ἀριθμηθήσεται ὑπὸ τοῦ πλήθους.
\vs{11}Καὶ εἶπεν αὐτῇ ὁ ἄγγελος Κυρίου, ἰδοὺ σὺ ἐν γαστρὶ ἔχεις, καὶ τέξῃ υἱὸν, καὶ καλέσεις τὸ ὄνομα αὐτοῦ Ἰσμαὴλ, ὅτι ἐπήκουσε Κύριος τῇ ταπεινώσει σου.
\vs{12}Οὗτος ἔσται ἄγροικος ἄνθρωπος· αἱ χεῖρες αὐτοῦ ἐπὶ πάντας, καὶ αἱ χεῖρες πάντων ἐπʼ αὐτόν· καὶ κατὰ πρόσωπον πάντων τῶν ἀδελφῶν αὐτοῦ κατοικήσει.
\vs{13}Καὶ ἐκάλεσε τὸ ὄνομα Κυρίου τοῦ λαλοῦντος πρὸς αὐτὴν, σὺ ὁ Θεὸς ὁ ἐτιδών με· ὅτι εἶπε, καὶ γὰρ ἐνώπιον εἶδον ὀφθέντα μοι.
\vs{14}Ἕνεκεν τούτου ἐκάλεσε τὸ φρέαρ, φρέαρ οὗ ἐνώπιον εἶδον· ἰδοὺ ἀνὰ μέσον Κάδης καὶ ἀνὰ μέσον Βαράδ.
\vs{15}Καὶ ἔτεκεν Ἄγαρ τῷ Ἅβραμ υἱὸν, καὶ ἐκάλεσεν Ἅβραμ τὸ ὄνομα τοῦ υἱοῦ αὐτοῦ, ὃν ἔτεκεν αὐτῷ Ἄγαρ, Ἰσμαήλ.
\vs{16}Ἅβραμ δὲ ἦν ἐτῶν ὀγδοηκονταὲξ, ἡνίκα ἔτεκεν Ἄγαρ τῷ Ἅβραμ τὸν Ἰσμαήλ.

\ch{17}
Ἐγένετο δὲ Ἅβραμ ἐτῶν ἐννενηκονταεννέα. Καὶ ὤφθη Κύριος τῷ Ἅβραμ, καὶ εἶπεν αὐτῷ, ἐγώ εἰμι ὁ Θεός σου· εὐαρέστει ἐνώπιον ἐμοῦ, καὶ γίνου ἄμεμπτος.
\vs{2}Καὶ θήσομαι τὴν διαθήκην μου ἀνὰ μέσον ἐμοῦ, καὶ ἀνὰ μέσον σου, καὶ πληθυνῶ σε σφόδρα.
\vs{3}Καὶ ἔπεσεν Ἅβραμ ἐπὶ πρόσωπον αὐτοῦ.
\vs{4}Καὶ ἐλάλησεν αὐτῷ ὁ Θεὸς, λέγων, Καὶ ἐγὼ ἰδοὺ ἡ διαθήκη μου μετὰ σοῦ· καὶ ἔσῃ πατὴρ πλήθους ἐθνῶν.
\vs{5}Καὶ οὐ κληθήσεται ἔτι τὸ ὄνομά σου Ἅβραμ, ἀλλʼ ἔσται τὸ ὄνομά σου Ἁβραὰμ, ὅτι πατέρα πολλῶν ἐθνῶν τέθεικά σε.
\vs{6}Καὶ αὐξανῶ σε σφόδρα σφόδρα, καὶ θήσω σε εἰς ἔθνη· καὶ βασιλεῖς ἐκ σοῦ ἐξελεύσονται.
\vs{7}Καὶ στήσω τὴν διαθήκην μου ἀνὰ μέσον σου, καὶ ἀνὰ μέσον τοῦ σπέρματός σου μετὰ σὲ εἰς τὰς γενεὰς αὐτῶν, εἰς διαθήκην αἰώνιον εἶναί σου Θεὸς, καὶ τοῦ σπέρματός σου μετὰ σέ.
\vs{8}Καὶ δώσω σοι καὶ τῷ σπέρματί σου μετὰ σὲ τὴν γῆν, ἣν παροικεῖς, πᾶσαν τὴν γῆν Χαναὰν, εἰς κατάσχεσιν αἰώνιον· καὶ ἔσομαι αὐτοῖς εἰς Θεόν.
\vs{9}Καὶ εἶπεν ὁ Θεὸς πρὸς Ἁβραὰμ, σὺ δὲ τὴν. διαθήκην μου διατηρήσεις, σὺ καὶ τὸ σπέρμα σου μετὰ σὲ εἰς τὰς γενεὰς αὐτῶν.
\vs{10}Καὶ αὕτη ἡ διαθήκη, ἣν διατηρήσεις, ἀνὰ μέσον ἐμοῦ καὶ ὑμῶν, καὶ ἀνὰ μέσον τοῦ σπέρματός σου μετὰ σὲ εἰς τὰς γενεὰς αὐτῶν· περιτμηθήσεται ὑμῶν πᾶν ἀρσενικόν.
\vs{11}Καὶ περιτμηθήσεσθε τὴν σάρκα τῆς ἀκροβυστίας ὑμῶν, καὶ ἔσται εἰς σημεῖον διαθήκης ἀνὰ μέσον ἐμοῦ καὶ ὑμῶν.
\vs{12}Καὶ παιδίον ὀκτὼ ἡμερῶν περιτμηθήσεται ὑμῖν, πᾶν ἀρσενικὸν εἰς τὰς γενεὰς ὑμῶν· καὶ οἰκογενὴς καὶ ὁ ἀργυρώνητος ἀπὸ παντὸς υἱοῦ ἀλλοτρίου, ὃς οὐκ ἔστιν ἐκ τοῦ σπέρματός σου,
\vs{13}Περιτομῇ περιτμηθήσεται ὁ οἰκογενὴς τῆς οἰκίας σου, καὶ ὁ ἀργυρώνητος· καὶ ἔσται ἡ διαθήκη μου ἐπὶ τῆς σαρκὸς ὑμῶν εἰς διαθήκην αἰώνιον.
\vs{14}Καὶ ἀπερίτμητος ἄρσην, ὃς οὐ περιτμηθήσεται τὴν σάρκα τῆς ἀκροβυστίας αὐτοῦ τῇ ἡμέρᾳ τῇ ὀγδόῃ, ἐξολοθρευθήσεται ἡ ψυχὴ ἐκείνη ἐκ τοῦ γένους αὐτῆς, ὅτι τὴν διαθήκην μου διεσκέδασε.
\vs{15}Καὶ εἶπεν ὁ Θεὸς τῷ Ἁβραὰμ, Σάρα ἡ γυνή σου, οὐ κληθήσεται τὸ ὄνομα αὐτῆς Σάρα, Σάῤῥα ἔσται τὸ ὄνομα αὐτῆς.
\vs{16}Εὐλογήσω δὲ αὐτὴν, καὶ δώσω σοι ἐξ αὐτῆς τέκνον, καὶ εὐλογήσω αὐτὸ, καὶ ἔσται εἰς ἔθνη, καὶ βασιλεῖς ἐθνῶν ἐξ αὐτοῦ ἔσονται.
\vs{17}Καὶ ἔπεσεν Ἁβραὰμ ἐπὶ πρόσωπον αὐτοῦ, καὶ ἐγέλασε· καὶ εἶπεν ἐν τῇ διανοίᾳ αὐτοῦ, λέγων, εἰ τῷ ἑκατονταετεῖ γενήσεται υἱός; καὶ εἰ ἡ Σάῤῥα ἐννενήκοντα ἐτῶν τέξεται;
\vs{18}Εἶπε δὲ Ἁβραὰμ πρὸς τὸν Θεόν· Ἰσμαὴλ οὗτος ζήτω ἐναντίον σου.
\vs{19}Εἶπε δὲ ὁ Θεὸς πρὸς Ἁβραὰμ, ναί· ἰδοὺ Σάῤῥα ἡ γυνή σου τέξεταί σοι υἱὸν, καὶ καλέσεις τὸ ὄνομα αὐτοῦ Ἰσαάκ· καὶ στήσω τὴν διαθήκην μου πρὸς αὐτὸν, εἰς διαθήκην αἰώνιον, εἶναι αὐτῷ Θεὸς καὶ τῷ σπέρματι αὐτοῦ μετʼ αὐτόν.
\vs{20}Περὶ δὲ Ἰσμαὴλ ἰδοὺ ἐπήκουσά σου· καὶ ἰδοὺ εὐλόγηκα αὐτὸν, καὶ αὐξανῶ αὐτὸν, καὶ πληθυνῶ αὐτὸν σφόδρα δώδεκα ἔθνη γεννήσει, καὶ δώσω αὐτὸν εἰς ἔθνος μέγα.
\vs{21}Τὴν δὲ διαθήκην μου στήσω πρὸς Ἰσαὰκ, ὃν τέξεταί σοι Σάῤῥα εἰς τὸν καιρὸν τοῦτον, ἐν τῷ ἐνιαυτῷ τῷ ἑτέρῳ.
\vs{22}Συνετέλεσε δὲ λαλῶν πρὸς αὐτὸν, καὶ ἀνέβη ὁ Θεὸς ἀπὸ Ἁβραάμ.

\vs{23}Καὶ ἔλαβεν Ἁβραὰμ Ἰσμαὴλ τὸν υἱὸν ἑαυτοῦ, καὶ πάντας τοὺς οἰκογενεῖς αὐτοῦ, καὶ πάντας τοὺς ἀργυρωνήτους, καὶ πᾶν ἄρσεν τῶν ἀνδρῶν τῶν ἐν τῷ οἴκῳ Ἁβραὰμ, καὶ περιέτεμε τὰς ἀκροβυστίας αὐτῶν, ἐν τῷ καιρῷ τῆς ἡμέρας ἐκείνης, καθὰ ἐλάλησεν αὐτῷ ὁ Θεός.
\vs{24}Ἁβραὰμ δὲ ἐννενηκονταεννέα ἦν ἐτῶν, ἡνίκα περιετέμετο τὴν σάρκα τῆς ἀκροβυστίας αὐτοῦ.
\vs{25}Ἰσμαὴλ δὲ ὁ υἱὸς αὐτοῦ ἦν ἐτῶν δεκατριῶν, ἡνίκα περιετέμετο τὴν σάρκα τῆς ἀκροβυστίας αὐτοῦ.
\vs{26}Ἐν δὲ τῷ καιρῷ τῆς ἡμέρας ἐκείνης, περιετμήθη Ἁβραὰμ, καὶ Ἰσμαὴλ ὁ υἱὸς αὐτοῦ,
\vs{27}καὶ πάντες οἱ ἄνδρες τοῦ οἴκου αὐτοῦ, καὶ οἱ οἰκογενεῖς αὐτοῦ, καὶ οἱ ἀργυρώνητοι ἐξ ἀλλογενῶν ἐθνῶν.

\ch{18}
Ὤφθη δὲ αὐτῷ ὁ Θεὸς πρὸς τῇ δρυῒ τῇ Μαμβρῇ, καθημένου αὐτοῦ ἐπὶ τῆς θύρας τῆς σκηνῆς αὐτοῦ μεσημβρίας.
\vs{2}Ἀναβλέψας δὲ τοῖς ὀφθαλμοῖς αὐτοῦ εἶδε, καὶ ἰδοὺ τρεῖς ἄνδρες εἱστήκεισαν ἐπάνω αὐτοῦ· καὶ ἰδὼν, προσέδραμεν εἰς συνάντησιν αὐτοῖς ἀπὸ τῆς θύρας τῆς σκηνῆς αὐτοῦ, καὶ προσεκύνησεν ἐπὶ τὴν γῆν.
\vs{3}Καὶ εἶπε, Κύριε, εἰ ἄρα εὗρον χάριν ἐναντίον σου, μὴ παρέλθῃς τὸν παῖδά σου.
\vs{4}Ληφθήτω δὴ ὕδωρ, καὶ νιψάτωσαν τοὺς πόδας ὑμῶν, καὶ καταψύξατε ὑπὸ τὸ δένδρον.
\vs{5}Καὶ λήψομαι ἄρτον, καὶ φάγεσθε. Καὶ μετὰ τοῦτο παρελεύσεσθε εἰς τὴν ὁδὸν ὑμῶν, οὗ ἕνεκεν ἐξεκλίνατε πρὸς τὸν παῖδα ὑμῶν. Καὶ εἶπεν, οὕτω ποίησον, καθὼς εἴρηκας.
\vs{6}Καὶ ἔσπευσεν Ἁβραὰμ ἐπὶ τὴν σκηνὴν πρὸς Σάῤῥαν, καὶ εἶπεν αὐτῇ, σπεῦσον, καὶ φύρασον τρία μέτρα σεμιδάλεως, καὶ ποίησον ἐγκρυφίας.
\vs{7}Καὶ εἰς τὰς βόας ἔδραμεν Ἁβραὰμ, καὶ ἔλαβεν ἁπαλὸν μοσχάριον καὶ καλὸν, καὶ ἔδωκε τῷ παιδὶ, καὶ ἐτάχυνε τοῦ ποιῆσαι αὐτό.
\vs{8}Ἔλαβε δὲ βούτυρον, καὶ γάλα, καὶ τὸ μοσχάριον ὃ ἐποίησε, καὶ παρέθηκεν αὐτοῖς, καὶ ἔφαγον· αὐτὸς δὲ παρειστήκει αὐτοῖς ὑπὸ τὸ δένδρον.

\vs{9}Εἶπε δὲ πρὸς αὐτὸν, ποῦ Σάῤῥα ἡ γυνή σου; ὁ δὲ ἀποκριθεὶς εἶπεν, ἰδοὺ ἐν τῇ σκηνῇ.
\vs{10}Εἶπε δὲ, ἐπαναστρέφων ἥξω πρὸς σὲ κατὰ τὸν καιρὸν τοῦτον εἰς ὥρας, καὶ ἕξει υἱὸν Σάῤῥα ἡ γυνή σου. Σάῤῥα δὲ ἤκουσε πρὸς τῇ θύρᾳ τῆς σκηνῆς οὖσα ὄπισθεν αὐτοῦ.
\vs{11}Ἁβραὰμ δὲ καὶ Σάῤῥα πρεσβύτεροι προβεβηκότες ἡμερῶν· ἐξέλιπε δὲ τῇ Σάῤῥᾳ γίνεσθαι τὰ γυναικεια.
\vs{12}Ἐγέλασε δὲ Σάῤῥα ἐν ἑαυτῇ λέγουσα, οὔπω μέν μοι γέγονεν ἕως τοῦ νῦν· ὁ δὲ κύριός μου πρεσβύτερος.
\vs{13}Καὶ εἶπε Κύριος πρὸς Ἁβραὰμ, τί ὅτι ἐγέλασε Σάῤῥα ἐν ἑαυτῇ, λέγουσα, ἆρά γε ἀληθῶς τέξομαι; ἐγὼ δὲ γεγήρακα.
\vs{14}Μὴ ἀδυνατήσει παρὰ τῷ Θεῷ ῥῆμα; εἰς τὸν καιρὸν τοῦτον ἀναστρέψω πρὸς σὲ εἰς ὥρας, καὶ ἔσται τῇ Σάῤῥᾳ υἱός.
\vs{15}Ἠρνήσατο δὲ Σάῤῥα, λέγουσα, οὐκ ἐγέλασα· ἐφοβήθη γάρ. Καὶ εἶπεν αὐτῇ, οὐχὶ, ἀλλὰ ἐγέλασας.

\vs{16}Ἐξαναστάντες δὲ ἐκεῖθεν οἱ ἄνδρες κατέβλεψαν ἐπὶ πρόσωπον Σοδόμων καὶ Γομόῤῥας. Ἁβραὰμ δὲ συνεπορεύετο μετʼ αὐτῶν, συμπροπέμπων αὐτούς.
\vs{17}Ὁ δὲ Κύριος εἶπε, οὐ μὴ κρύψω ἐγὼ ἀπὸ Ἁβραὰμ τοῦ παιδός μου ἃ ἐγὼ ποιῶ.
\vs{18}Ἁβραὰμ δὲ γινόμενος ἔσται εἰς ἔθνος μέγα καὶ πολὺ, καὶ ἐνευλογηθήσονται ἐν αὐτῷ πάντα τὰ ἔθνη τῆς γῆς.
\vs{19}Ἤδειν γὰρ ὅτι συντάξει τοῖς υἱοῖς αὐτοῦ, καὶ τῷ οἴκῳ αὐτοῦ μετʼ αὐτὸν, καὶ φυλάξουσι τὰς ὁδοὺς Κυρίου, ποιεῖν δικαιοσύνην καὶ κρίσιν, ὅπως ἂν ἐπαγάγῃ Κύριος ἐπὶ Ἁβραὰμ πάντα ὅσα ἐλάλησε πρὸς αὐτόν.
\vs{20}Εἶπε δὲ Κύριος, κραυγὴ Σοδόμων καὶ Γομόῤῥας πεπλήθυνται πρὸς μὲ, καὶ αἱ ἁμαρτίαι αὐτῶν μεγάλαι σφόδρα.
\vs{21}Καταβὰς οὖν ὄψομαι, εἰ κατὰ τὴν κραυγὴν αὐτῶν τὴν ἐρχομενεην πρὸς μὲ, συντελοῦνται· εἰ δὲ μὴ, ἵνα γνῶ.
\vs{22}Καὶ ἀποστρέψαντες ἐκεῖθεν οἱ ἄνδρες, ἦλθον εἰς Σόδομα· Ἁβραὰμ δὲ ἔτι ἦν ἑστηκὼς ἐναντίον Κυρίου.
\vs{23}Καὶ ἐγγίσας Ἁβραὰμ, εἶπε, μὴ συναπολέσῃς δίκαιον μετὰ ἀσεβοῦς· καὶ ἔσται ὁ δίκαιος ὡς ὁ ἀσεβής.
\vs{24}Ἐὰν ὦσι πεντήκοντα δίκαιοι ἐν τῇ πόλει, ἀπολεῖς αὐτούς; οὐκ ἀνήσεις πάντα τὸν τόπον ἕνεκεν τῶν πεντήκοντα δικαίων, ἐὰν ὦσιν ἐν αὐτῇ;
\vs{25}Μηδαμῶς σὺ ποιήσεις ὡς τὸ ῥῆμα τοῦτο, τοῦ ἀποκτεῖναι δίκαιον μετὰ ἀσεβοῦς, καὶ ἔσται ὁ δίκαιος ὡς ὁ ἀσεβής· μηδαμῶς· ὁ κρίνων πᾶσαν τὴν γῆν, οὐ ποιήσεις κρίσιν;
\vs{26}Εἶπε δὲ Κύριος, ἐὰν ὦσιν ἐν Σοδόμοις πεντήκοντα δίκαιοι ἐν τῇ πόλει, ἀφήσω ὅλην τὴν πόλιν, καὶ πάντα τὸν τόπον διʼ αὐτούς.
\vs{27}Καὶ ἀποκριθεὶς Ἁβραὰμ εἶπε, νῦν ἠρξάμην λαλῆσαι πρὸς τὸν Κύριόν μου· ἐγὼ δὲ εἰμὶ γῆ καὶ σποδός.
\vs{28}Ἐὰν δὲ ἐλαττονωθῶσιν οἱ πεντήκοντα δίκαιοι εἰς τεσσαρακονταπέντε, ἀπολεῖς ἕνεκεν τῶν πέντε πᾶσαν τὴν πόλιν; καὶ εἶπεν, οὐ μὴ ἀπολέσω, ἐὰν εὕρω ἐκεῖ τεσσαρακονταπέντε.
\vs{29}Καὶ προσέθηκεν ἔτι λαλῆσαι πρὸς αὐτὸν, καὶ εἶπεν, ἐὰν δὲ εὑρεθῶσιν ἐκεῖ τεσσαράκοντα· καὶ εἶπεν, οὐ μὴ ἀπολέσω ἕνεκεν τῶν τεσσαράκοντα.
\vs{30}Καὶ εἶπε, μή τι Κύριε ἐὰν λαλήσω; ἐὰν δὲ εὑρεθῶσιν ἐκεῖ τριάκοντα; καὶ εἶπεν, οὐ μὴ ἀπολέσω ἕνεκεν τῶν τριάκοντα.
\vs{31}Καὶ εἶπεν, ἐπειδὴ ἔχω λαλῆσαι πρὸς τὸν Κύριον, ἐὰν δὲ εὑρεθῶσιν ἐκεῖ εἴκοσι; καὶ εἶπεν, οὐ μὴ ἀπολέσω, ἐὰν εὕρω ἐκεῖ εἴκοσι.
\vs{32}Καὶ εἶπε, μή τι Κύριε ἐὰν λαλήσω ἔτι ἅπαξ; ἐὰν δὲ εὑρεθῶσιν ἐκεῖ δέκα; καὶ εἶπεν, οὐ μὴ ἀπολέσω ἕνεκεν τῶν δέκα.
\vs{33}Ἀπῆλθε δὲ ὁ Κύριος, ὡς ἐπαύσατο λαλῶν τῷ Ἁβραάμ· καὶ Ἁβραὰμ ἀπέστρεψεν εἰς τὸν τόπον αὐτοῦ.

\ch{19}
Ἦλθον δε οἱ δύο ἄγγελοι εἰς Σόδομα ἑσπέρας. Λὼτ δὲ ἐκάθητο παρὰ τὴν πύλην Σοδόμων· ἰδὸν δὲ Λὼτ, ἐξανέστη εἰς συνάντησιν αὐτοῖς, καὶ προσεκύνησε τῷ προσώπῳ ἐπὶ τὴν γῆν.
\vs{2}Καὶ εἶπεν, ἰδοὺ, Κύριοι, ἐκκλίνατε εἰς τὸν οἶκον τοῦ παιδὸς ὑμῶν, καὶ καταλύσατε, καὶ νίψασθε τοὺς πόδας ὑμῶν, καὶ ὀρθρίσαντες ἀπελεύσεσθε εἰς τὴν ὁδὸν ὑμῶν. Καὶ εἶπαν, οὐχὶ, ἀλλʼ ἐν τῇ πλατείᾳ καταλύσομεν.
\vs{3}Καὶ κατεβιάσατο αὐτοὺς, καὶ ἐξέκλιναν πρὸς αὐτὸν, καὶ εἰσῆλθον εἰς τὸν οἶκον αὐτοῦ· καὶ ἐποίησεν αὐτοῖς πότον, καὶ ἀζύμους ἔπεψεν αὐτοῖς, καὶ ἔφαγον.
\vs{4}Πρὸ τοῦ κοιμηθῆναι δὲ, οἱ ἄνδρες τῆς πόλεως, οἱ Σοδομῖται περιεκύκλωσαν τὴν οἰκίαν, ἀπὸ νεανίσκου ἕως πρεσβυτέρου, ἅπας ὁ λαὸς ἅμα.
\vs{5}Καὶ ἐξεκαλοῦντο τὸν Λὼτ, καὶ ἔλεγον πρὸς αὐτὸν, ποῦ εἰσιν οἱ ἄνδρες οἱ εἰσελθόντες πρὸς σὲ τὴν νύκτα; ἐξάγαγε αὐτοὺς πρὸς ἡμᾶς, ἵνα συγγενώμεθα αὐτοῖς.
\vs{6}Ἐξῆλθε δὲ Λὼτ πρὸς αὐτοὺς πρὸς τὸ πρόθυρον, τὴν δὲ θύραν προσέῳξεν ὀπίσω αὐτοῦ.
\vs{7}Εἶπε δὲ πρὸς αὐτοὺς, μηδαμῶς ἀδελφοὶ μὴ πονηρεύσησθε.
\vs{8}Εἰσὶ δέ μοι δύο θυγατέρες, αἳ οὐκ ἔγνωσαν ἄνδρα· ἐξάξω αὐτὰς πρὸς ὑμᾶς, καὶ χρᾶσθε αὐταῖς καθὰ ἂν ἀρέσκοι ὑμῖν· μόνον εἰς τοὺς ἄνδρας τούτους μὴ ποιήσητε ἄδικον, οὗ εἵνεκεν εἰσῆλθον ὑπὸ τὴν σκέπην τῶν δοκῶν μου.
\vs{9}Εἶπαν δὲ αὐτῷ, ἀπόστα ἐκεῖ· εἰσῆλθες παροικεῖν, μὴ καὶ κρίσιν κρίνειν; νῦν οὖν σε κακώσωμεν μᾶλλον ἢ ἐκείνους. Καὶ παρεβιάζοντο τὸν ἄνδρα τὸν Λὼτ σφόδρα, καὶ ἤγγισαν συντρίψαι τὴν θύραν.
\vs{10}Ἐκτείναντες δὲ οἱ ἄνδρες τὰς χεῖρας εἰσεσπάσαντο τὸν Λὼτ πρὸς ἑαυτοὺς εἰς τὸν οἶκον, καὶ τὴν θύραν τοῦ οἴκου ἀπέκλεισαν.
\vs{11}Τοὺς δὲ ἄνδρας τοὺς ὄντας ἐπὶ τῆς θύρας τοῦ οἴκου ἐπάταξαν ἐν ἀορασίᾳ ἀπὸ μικροῦ ἕως μεγάλου· καὶ παρελύθησαν ζητοῦντες τὴν θύραν.
\vs{12}Εἶπαν δὲ οἱ ἄνδρες πρὸς τὸν Λὼτ, εἰσί σοι ὧδε γαμβροὶ, ἢ υἱοὶ, ἢ θυγατέρες; ἢ εἴτις σοι ἄλλος ἐστὶν ἐν τῇ πόλει, ἐξάγαγε ἐκ τοῦ τόπου τούτου,
\vs{13}Ὅτι ἡμεῖς ἀπόλλυμεν τὸν τόπον τοῦτον· ὅτι ὑψώθη ἡ κραυγὴ αὐτῶν ἔναντι Κυρίου, καὶ ἀπέστειλεν ἡμᾶς Κύριος ἐκτρίψαι αὐτήν.
\vs{14}Ἐξῆλθε δὲ Λῶτ, καὶ ἐλάλησε πρὸς τοὺς γαμβροὺς αὐτοῦ τοὺς εἰληφότας τὰς θυγατέρας αὐτοῦ, καὶ εἶπεν, ἀνάστητε, καὶ ἐξέλθετε ἐκ τοῦ τόπου τούτου, ὅτι ἐκτρίβει Κύριος τὴν πόλιν· ἔδοξε δὲ γελοιάζειν ἐναντίον τῶν γαμβρῶν αὐτοῦ.
\vs{15}Ἡνίκα δὲ ὄρθρος ἐγένετο, ἐσπούδαζον οἱ ἄγγελοι τὸν Λὼτ, λέγοντες, ἀναστὰς λάβε τὴν γυναῖκά σου, καὶ τὰς δύο θυγατέρας σου, ἃς ἔχεις, καὶ ἔξελθε, ἵνα μὴ καὶ σὺ συναπόλῃ ταῖς ἀνομίαις τῆς πόλεως.
\vs{16}Καὶ ἐταράχθησαν, καὶ ἐκράτησαν οἱ ἄγγελοι τῆς χειρὸς αὐτοῦ, καὶ τῆς χειρὸς τῆς γυναικὸς αὐτοῦ, καὶ τῶν χειρῶν τῶν δύο θυγατέρων αὐτοῦ, ἐν τῷ φείσασθαι Κύριον αὐτοῦ.

\vs{17}Καὶ ἐγένετο ἡνίκα ἐξήγαγον αὐτοὺς ἔξω, καὶ εἶπαν, σώζων σῶζε τὴν σεαυτοῦ ψυχήν· μὴ περιβλέψῃ εἰς τὰ ὀπίσω, μηδὲ στῇς ἐν πάσῃ τῇ περιχώρῳ· εἰς τὸ ὄρος σώζου, μή ποτε συμπαραληφθῇς.
\vs{18}Εἶπε δὲ Λὼτ πρὸς αὐτοὺς, δέομαι
\vs{19}Κύριε, ἐπειδὴ εὗρεν ὁ παῖς σου ἔλεος ἐναντίον σου, καὶ ἐμεγάλυνας τὴν δικαιοσύνην σου, ὃ ποιεῖς ἐπʼ ἐμὲ, τοῦ ζῆν τὴν ψυχήν μου· ἐγὼ δὲ οὐ δυνήσομαι διασωθῆναι εἰς τὸ ὄρος, μή ποτε καταλάβῃ με τὰ κακὰ, καὶ ἀποθάνω.
\vs{20}Ἰδοὺ πόλις αὕτη ἐγγὺς τοῦ καταφυγεῖν με ἐκεῖ, ἥ ἐστι μικρά· καὶ ἐκεῖ διασωθήσομαι· οὐ μικρά ἐστι; καὶ ζήσεται ἡ ψυχή μου ἕνεκέν σου.
\vs{21}Καὶ εἶπεν αὐτῷ, ἰδοὺ ἐθαύμασά σου τὸ πρόσωπον καὶ ἐπὶ τῷ ῥήματι τούτῳ, τοῦ μὴ καταστρέψαι τὴν πόλιν περὶ ἧς ἐλάλησας.
\vs{22}Σπεῦσον οὖν τοῦ σωθῆναι ἐκεῖ, οὐ γὰρ δυνήσομαι ποιῆσαι πρᾶγμα, ἕως τοῦ ἐλθεῖν σε ἐκεῖ. διὰ τοῦτο ἐκάλεσε τὸ ὄνομα τῆς πόλεως ἐκείνης, Σηγώρ.
\vs{23}Ὁ ἥλιος ἐξῆλθεν ἐπὶ τὴν γῆν, καὶ Λὼτ εἰσῆλθεν εἰς Σηγώρ.
\vs{24}Καὶ Κύριος ἔβρεξεν ἐπὶ Σόδομα καὶ Γόμοῤῥα θεῖον καὶ πῦρ παρὰ Κυρίου ἐξ οὐρανοῦ.
\vs{25}Καὶ κατέστρεψε τὰς πόλεις ταύτας, καὶ πᾶσαν τὴν περίχωρον, καὶ πάντας τοὺς κατοικοῦντας ἐν ταῖς πόλεσι, καὶ τὰ ἀνατέλλοντα ἐκ τῆς γῆς.
\vs{26}Καὶ ἐπέβλεψεν ἡ γυνὴ αὐτοῦ εἰς τὰ ὀπίσω, καὶ ἐγένετο στήλη ἁλός.
\vs{27}Ὤρθρισε δὲ Ἁβραὰμ τῷ πρωῒ εἰς τὸν τόπον, οὗ εἱστήκει ἐναντίον Κυρίου.
\vs{28}Καὶ ἐπέβλεψεν ἐπὶ πρόσωπον Σοδόμων καὶ Γομόῤῥας, καὶ ἐπὶ πρόσωπον τῆς περιχώρου, καὶ εἶδε, καὶ ἰδοὺ ἀνέβαινεν φλὸξ ἐκ τῆς γῆς, ὡσεὶ ἀτμὶς καμίνου.
\vs{29}Καὶ ἐγένετο ἐν τῷ ἐκτρίψαι τὸν Θεὸν πάσας τὰς πόλεις τῆς περιοίκου, ἐμνήσθη ὁ Θεὸς τοῦ Ἁβραάμ· καὶ ἐξαπέστειλε τὸν Λὼτ ἐκ μέσου τῆς καταστροφῆς, ἐν τῷ καταστρέψαι Κύριον τὰς πόλεις, ἐν αἷς κατῴκει ἐν αὐταῖς Λώτ.

\vs{30}Ἀνέβη δὲ Λὼτ ἐκ Σηγὼρ, καὶ ἐκάθητο ἐν τῷ ὄρει αὐτὸς, καὶ αἱ δύο θυγατέρες αὐτοῦ μετʼ αὐτοῦ· ἐφοβήθη γὰρ κατοικῆσαι ἐν Σηγώρ· καὶ κατῴκησεν ἐν τῷ σπηλαίῳ αὐτὸς, καὶ αἱ δύο θυγατέρες αὐτοῦ μετʼ αὐτοῦ.
\vs{31}Εἶπε δὲ ἡ πρεσβυτέρα πρὸς τὴν νεωτέραν, ὁ πατὴρ ἡμῶν πρεσβύτερος, καὶ οὐδείς ἐστιν ἐπὶ τῆς γῆς, ὃς εἰσελεύσεται πρὸς ἡμᾶς, ὡς καθήκει πάσῃ τῇ γῇ.
\vs{32}Δεῦρο καὶ ποτίσωμεν τὸν πατέρα ἡμῶν οἶνον, καὶ κοιμηθῶμεν μετʼ αὐτοῦ, καὶ ἐξαναστήσωμεν ἐκ τοῦ πατρὸς ἡμῶν σπέρμα.
\vs{33}Ἐπότισαν δὲ τὸν πατέρα αὐτῶν οἶνον ἐν τῇ νυκτὶ ἐκείνῃ, καὶ εἰσελθοῦσα ἡ πρεσβυτέρα ἐκοιμήθη μετὰ τοῦ πατρὸς αὐτῆς ἐν τῇ νυκτὶ ἐκείνῃ· καὶ οὐκ ᾔδει ἐν τῷ κοιμηθῆναι αὐτὸν, καὶ ἐν τῷ ἀναστῆναι.
\vs{34}Ἐγένετο δὲ ἐν τῇ ἐπαύριον, καὶ εἶπεν ἡ πρεσβυτέρα πρὸς τὴν νεωτέραν, ἰδοὺ ἐκοιμήθην χθὲς μετὰ τοῦ πατρὸς ἡμῶν· ποτίσωμεν αὐτὸν οἶνον καὶ ἐν τῇ νυκτὶ ταύτῃ, καὶ εἰσελθοῦσα κοιμήθητι μετʼ αὐτοῦ, καὶ ἐξαναστήσωμεν ἐκ τοῦ πατρὸς ἡμῶν σπέρμα.
\vs{35}Ἐπότισαν δὲ καὶ ἐν τῇ νυκτὶ ἐκείνῃ τὸν πατέρα αὐτῶν οἶνον, καὶ εἰσελθοῦσα ἡ νεωτέρα ἐκοιμήθη μετὰ τοῦ πατρὸς αὐτῆς· καὶ οὐκ ᾔδει ἐν τῷ κοιμηθῆναι αὐτὸν, καὶ ἀναστῆναι.
\vs{36}Καὶ συνέλαβον αἱ δύο θυγατέρες Λὼτ ἐκ τοῦ πατρὸς αὐτῶν.
\vs{37}Καὶ ἔτεκεν ἡ πρεσβυτέρα υἱὸν, καὶ ἐκάλεσε τὸ ὄνομα αὐτοῦ Μωὰβ, λέγουσα, ἐκ τοῦ πατρός μου· οὗτος πατὴρ Μωαβιτῶν ἕως τῆς σήμερον ἡμέρας.
\vs{38}Ἔτεκε δὲ καὶ ἡ νεωτέρα υἱὸν, καὶ ἐκάλεσε τὸ ὄνομα αὐτοῦ Ἀμμὰν, λέγουσα, υἱὸς γένους μου· οὗτος πατὴρ Ἀμμανιτῶν ἕως τῆς σήμερον ἡμέρας.

\ch{20}
Καὶ ἐκίνησεν ἐκεῖθεν Ἁβραὰμ εἰς γῆν πρὸς Λίβα· καὶ ᾤκησεν ἀνὰ μέσον Κάδης, καὶ ἀνὰ μέσον Σούρ· καὶ παρῴκησεν ἐν Γεράροις.
\vs{2}Εἶπε δὲ Ἁβραὰμ περὶ Σάῤῥας τῆς γυναικὸς αὐτοῦ, ὅτι ἀδελφή μου ἐστίν· ἐφοβήθη γὰρ εἰπεῖν ὅτι γυνή μου ἐστὶ, μή ποτε ἀποκτείνωσιν αὐτὸν οἱ ἄνδρες τῆς πόλεως διʼ αὐτήν· ἀπέστειλε δὲ Ἀβιμέλεχ βασιλεὺς Γεράρων, καὶ ἔλαβε τὴν Σάῤῥαν.
\vs{3}Καὶ εἰσῆλθεν ὁ Θεὸς πρὸς Ἀβιμέλεχ ἐν ὕπνῳ τὴν νύκτα, καὶ εἶπεν, ἰδοὺ σὺ ἀποθνήσκεις περὶ τῆς γυναικὸς, ἧς ἔλαβες· αὕτη δέ ἐστι συνῳκηκυῖα ἀνδρί.
\vs{4}Ἀβιμέλεχ δὲ οὐχ ἥψατο αὐτῆς· καὶ εἶπε, Κύριε, ἔθνος ἀγνοοῦν καὶ δίκαιον ἀπολεῖς;
\vs{5}Οὐκ αὐτός μοι εἶπεν, ἀδφή μου ἐστί; καὶ αὕτη μοι εἶπεν, ἀδελφός μου ἐστίν; ἐν καθαρᾷ καρδίᾳ καὶ ἐν δικαιοσύνῃ χειρῶν ἐποίησα τοῦτο.
\vs{6}Εἶπε δὲ αὐτῷ ὁ Θεὸς καθʼ ὕπνον, κᾀγὼ ἔγνων ὅτι ἐν καθαρᾷ καρδίᾳ ἐποίησας τοῦτο, καὶ ἐφεισάμην σου τοῦ μὴ ἁμαρτεῖν σε εἰς ἐμέ· ἕνεκα τούτου οὐκ ἀφῆκά σε ἅψασθαι αὐτῆς.
\vs{7}Νῦν δὲ ἀπόδος τὴν γυναῖκα τῷ ἀνθρώπῳ, ὅτι προφήτης ἐστὶ, καὶ προσεύξεται περὶ σοῦ, καὶ ζήσῃ· εἰ δὲ μὴ ἀποδίδως, γνώσῃ ὅτι ἀποθανῇ σὺ καὶ πάντα τὰ σὰ.
\vs{8}Καὶ ὤρθρισεν Ἀβιμέλεχ τῷ πρωῒ, καὶ ἐκάλεσε πάντας τοὺς παῖδας αὐτοῦ, καὶ ἐλάλησε πάντα τὰ ῥήματα ταῦτα εἰς τὰ ὦτα αὐτῶν· ἐφοβήθησαν δὲ πάντες οἱ ἄνθρωποι σφόδρα.
\vs{9}Καὶ ἐκάλεσεν Ἀβιμέλεχ τὸν Ἁβραὰμ καὶ εἶπεν αὐτῷ, τί τοῦτο ἐποίησας ἡμῖν; μήτι ἡμάρτομεν εἰς σὲ, ὅτι ἐπήγαγες ἐπʼ ἐμὲ καὶ ἐπὶ τὴν βασιλείαν μου ἁμαρτίαν μεγάλην; ἔργον ὃ οὐδεὶς ποιήσει, πεποίηκάς μοι.
\vs{10}Εἶπε δὲ Ἀβιμέλεχ τῷ Ἁβραὰμ, τί ἐνιδὼν ἐποίησας τοῦτο;
\vs{11}Εἶπε δὲ Ἁβραὰμ, εἶπα γὰρ, ἄρα οὐκ ἔστι θεοσέβεια ἐν τῷ τόπῳ τούτῳ, ἐμέ τε ἀποκτενοῦσιν ἕνεκεν τῆς γυναικός μου.
\vs{12}Καὶ γὰρ ἀληθῶς, ἀδελφή μου ἐστὶν ἐκ πατρὸς, ἀλλʼ οὐκ ἐκ μητρός· ἐγενήθη δέ μοι εἰς γυναῖκα.
\vs{13}Ἐγένετο δὲ ἡνίκα ἐξήγαγέ με ὁ Θεὸς ἐκ τοῦ οἴκου τοῦ πατρός μου, καὶ εἶπα αὐτῇ, ταύτην τὴν δικαιοσύνην ποιήσεις εἰς ἐμὲ, εἰς πάντα τόπον οὗ ἐὰν εἰσέλθωμεν ἐκεῖ, εἶπον ἐμὲ, ὅτι ἀδελφός μου ἐστίν.
\vs{14}Ἔλαβε δὲ Ἀβιμέλεχ χίλια δίδραγμα, καὶ πρόβατα, καὶ μόσχους, καὶ παῖδας, καὶ παιδίσκας, καὶ ἔδωκε τῷ Ἁβραάμ· καὶ ἀπέδωκεν αὐτῷ Σάῤῥαν τὴν γυναῖκα αὐτοῦ.
\vs{15}Καὶ εἶπεν Ἀβιμέλεχ τῷ Ἁβραὰμ, ἰδοὺ ἡ γῆ μου ἐναντίον σου· οὗ ἄν σοι ἀρέσκῃ, κατοίκει.
\vs{16}Τῇ δὲ Σάῤῥᾳ εἶπεν, ἰδοὺ δέδωκα χίλια δίδραγμα τῷ ἀδελφῷ σου· ταῦτα ἔσται σοι εἰς τιμὴν τοῦ προσώπου σου, καὶ πάσαις ταῖς μετὰ σοῦ· καὶ πάντα ἀλήθευσον.
\vs{17}Προσηύξατο δὲ Ἁβραὰμ πρὸς τὸν Θεὸν, καὶ ἰάσατο ὁ Θεὸς τὸν Ἀβιμέλεχ, καὶ τὴν γυναῖκα αὐτοῦ, καὶ τὰς παιδίσκας αὐτοῦ· καὶ ἔτεκον.
\vs{18}Ὅτι συγκλείων συνέκλεισε Κύριος ἔξωθεν πᾶσαν μήτραν ἐν τῷ οἴκῳ Ἀβιμέλεχ, ἕνεκεν Σάῤῥας τῆς γυναικὸς Ἁβραάμ.

\ch{21}
Καὶ Κύριος ἐπεσκέψατο τὴν Σάῤῥαν, καθὰ εἶπε· καὶ ἐποίησε Κύριος τῇ Σάῥῥᾳ, καθὰ ἐλάλησε.
\vs{2}Καὶ συλλαβοῦσα ἔτεκε τῷ Ἁβραὰμ υἱὸν εἰς τὸ γῆρας, εἰς τὸν καιρὸν καθὰ ἐλάλησεν αὐτῷ Κύριος.
\vs{3}Καὶ ἐκάλεσεν Ἁβραὰμ τὸ ὄνομα τοῦ υἱοῦ αὐτοῦ τοῦ γενομένου αὐτῷ, ὃν ἔτεκεν αὐτῷ Σάῤῥα, Ἰσαάκ·
\vs{4}Περιέτεμε δὲ Ἁβραὰμ τὸν Ἰσαὰκ τῇ ἡμέρᾳ τῇ ὀγδόῃ, καθὰ ἐνετείλατο αὐτῷ ὁ Θεός.
\vs{5}Καὶ Ἁβραὰμ ἦν ἑκατὸν ἐτῶν, ηνίκα ἐγένετο αὐτῷ Ἰσαὰκ ὁ υἱὸς αὐτοῦ.
\vs{6}Εἶπε δὲ Σάῤῥα, γέλωτά μοι ἐποίησε Κύριος· ὃς γὰρ ἂν ἀκούσῃ συγχαρεῖταί μοι.
\vs{7}Καὶ εἶπε τίς ἀναγγελεῖ τῷ Ἁβραὰμ ὅτι θηλάζει παιδίον Σάῤῥα; ὅτι ἔτεκον υἱὸν ἐν τῷ γήρᾳ μου.
\vs{8}Καὶ ηὐξήθη τὸ παιδίον, καὶ ἀπεγαλακτίσθη· καὶ ἐποίησεν Ἁβραὰμ δοχὴν μεγάλην, ᾗ ἡμέρᾳ ἀπεκγαλακτίσθη Ἰσαὰκ ὁ υἱὸς αὐτοῦ.
\vs{9}Ἰδοῦσα δὲ Σάῥῥα τὸν υἱὸν Ἄγαρ τῆς Αἰγυπτίας, ὃς ἐγένετο τῷ Ἁβραὰμ, παίζοντα μετὰ Ἰσαὰκ τοῦ υἱοῦ αὐτῆς,
\vs{10}καὶ εἶπε τῷ Ἁβραὰμ, ἔκβαλε τὴν παιδίσκην ταύτην, καὶ τὸν υἱὸν αὐτῆς· οὐ γὰρ μὴ κληρονομήσει ὁ υἱὸς τῆς παιδίσκης ταύτης μετὰ τοῦ υἱοῦ μου Ἰσαάκ.
\vs{11}Σκληρὸν δὲ ἐφάνη τὸ ῥῆμα σφόδρα ἐνατίον Ἁβραὰμ περὶ τοῦ υἱοῦ αὐτοῦ.
\vs{12}Εἶπε δὲ ὁ Θεὸς τῷ Ἁβραὰμ, μὴ σκληρὸν ἔστω ἐναντίον σου περὶ τοῦ παιδίου, καὶ περὶ τῆς παιδίσκης· πάντα ὅσα ἂν εἴπῃ σοι Σάῤῥα, ἄκουε τῆς φωνῆς αὐτῆς· ὅτι ἐν Ἰσαὰκ κληθήσεταί σοι σπέρμα.
\vs{13}Καὶ τὸν υἱὸν δὲ τῆς παιδίσκης ταύτης εἰς ἔθνος μέγα ποιήσω αὐτὸν, ὅτι σπέρμα σόν ἐστιν.
\vs{14}Ἀνέστη δὲ Ἁβραὰμ τὸ πρωῒ, καὶ ἔλαβεν ἄρτους καὶ ἀσκὸν ὕδατος, καὶ ἔδωκεν τῇ Ἄγαρ· καὶ ἐπέθηκεν ἐπὶ τὸν ὦμον αὐτῆς τὸ παιδίον, καὶ ἀπέστειλεν αὐτήν· Ἀπελθοῦσα δὲ ἐπλανᾶτο κατὰ τὴν ἔρημον, κατὰ τὸ φρέαρ τοῦ ὅρκου.
\vs{15}Ἐξέλιπε δὲ τὸ ὕδωρ ἐκ τοῦ ἀσκου· καὶ ἔῤῥιψε τὸ παιδίον ὑποκάτω μιᾶς ἐλάτης·
\vs{16}Ἀπελθοῦσα δὲ ἐκάθητο ἀπέναντι αὐτοῦ μακρόθεν, ὡσεὶ τόξου βολήν· εἶπε γὰρ, οὐ μὴ ἴδω τὸν θάνατον τοῦ παιδίου μου. καὶ ἐκάθισεν ἀπέναντι αὐτοῦ· ἀναβοῆσαν δὲ τὸ παιδίον ἔκλαυσεν.
\vs{17}Εἰσήκουσε δὲ ὁ Θεὸς τῆς φωνῆς τοῦ παιδίου ἐκ τοῦ τόπου οὗ ἦν· καὶ ἐκάλεσεν ἄγγελος Θεοῦ τὴν Ἄγαρ ἐκ τοῦ οὐρανοῦ, καὶ εἶπεν αὐτῇ, τί ἐστιν Ἄγαρ; μὴ φοβοῦ· ἐπακήκοε γὰρ ὁ Θεὸς τῆς φωνῆς τοῦ παιδίου ἐκ τοῦ τόπου οὗ ἐστιν.
\vs{18}Ἀνάστηθι καὶ λάβε τὸ παιδίον, καὶ κράτησον τῇ χειρί σου αὐτό· εἰς γὰρ ἔθνος μέγα ποιήσω αὐτό.
\vs{19}Καὶ ἀνέῳξεν ὁ Θεὸς τοὺς ὀφθαλμοὺς αὐτῆς· καὶ εἶδε φρέαρ ὕδατος ζῶντος, καὶ ἐπορεύθη, καὶ ἔπλησε τὸν ἀσκὸν ὕδατος, καὶ ἐπότισε τὸ παιδίον.
\vs{20}Καὶ ἦν ὁ Θεὸς μετὰ τοῦ παιδίου· καὶ ηὐξήθη, καὶ κατῴκησεν ἐν τῇ ἐρήμῳ· ἐγένετο δὲ τοξότης.
\vs{21}Καὶ κατῴκησεν ἐν τῇ ἐρήμῳ· καὶ ἔλαβεν αὐτῷ ἡ μήτηρ γυναῖκα ἐκ Φαρὰν Αἰγύπτου.

\vs{22}Ἐγένετο δὲ ἐν τῷ καιρῷ ἐκείνῳ, καὶ εἶπεν Ἀβιμέλεχ, καὶ Ὁχοζὰθ ὁ νυμφαγωγὸς αὐτοῦ, καὶ Φιχὸλ ὁ ἀρχιστράτηγος τῆς δυνάμεως αὐτοῦ, πρὸς Ἁβραὰμ, λέγων, ὁ Θεὸς μετὰ σοῦ ἐν πᾶσιν, οἷς ἐὰν ποιῇς.
\vs{23}Νῦν οὖν ὄμοσόν μοι τὸν Θεὸν μὴ ἀδικήσειν με, μηδὲ τὸ σπέρμα μου, μηδὲ τὸ ὄνομά μου· ἀλλὰ κατὰ τὴν δικαιοσύνην ἣν ἐποίησα μετὰ σοῦ, ποιήσεις μετʼ ἐμοῦ, καὶ τῇ γᾗ, ᾗ σὺ παρῴκησας ἐν αὐτῇ.
\vs{24}Καὶ εἶπεν Ἁβραὰμ, ἐγὼ ὀμοῦμαι.
\vs{25}Καὶ ἤλεγξεν Ἁβραὰμ τὸν Ἀβιμέλεχ περὶ τῶν φρεάτων τοῦ ὕδατος, ὧν ἀφείλοντο οἱ παῖδες τοῦ Ἀβιμέλεχ.
\vs{26}Καὶ εἶπεν αὐτῷ Ἀβιμέλεχ, οὐκ ἔγνων τίς ἐποίησέ σοι τὸ ῥῆμα τοῦτο· οὐδὲ σύ μοι ἀπήγγειλας, οὐδὲ ἐγὼ ἤκουσα, ἀλλʼ ἢ σήμερον.
\vs{27}Καὶ ἔλαβεν Ἁβραὰμ πρόβατα καὶ μόσχους, καὶ ἔδωκε τῷ Ἀβιμέλεχ· καὶ διέθεντο ἀμφότεροι διαθήκην.
\vs{28}Καὶ ἔστησεν Ἁβραὰμ, ἑπτὰ ἀμνάδας προβάτων μόνας.
\vs{29}Καὶ εἶπεν Ἀβιμέλεχ τῷ Ἁβραὰμ, τί εἰσιν αἱ ἑπτὰ ἀμνάδες τῶν προβάτων τούτων, ἃς ἔστησας μόνας;
\vs{30}Καὶ εἶπεν Ἁβραὰμ, ὅτι τὰς ἑπτὰ ἀμνάδας λήψῃ παρʼ ἐμοῦ, ἵνα ὦσι μοι εἰς μαρτύριον, ὅτι ἐγὼ ὤρυξα τό φρέαρ τοῦτο.
\vs{31}Διὰ τοῦτο ἐπωνόμασε τὸ ὄνομα τοῦ τόπου ἐκείνου, Φρέαρ ὁρκισμοῦ· ὅτι ἐκεῖ ὤμοσαν ἀμφότεροι.
\vs{32}Καὶ διέθεντο διαθήκην ἐν τῷ φρέατι τοῦ ὁρκισμου· ἀνέστη δὲ Ἀβιμέλεχ, Ὁχοζὰθ ὁ νυμφαγωγὸς αὐτοῦ, καὶ Φίχολ ὁ ἀρχιστράτηγος τῆς δυνάμεως αὐτοῦ, καὶ ἐπέστρεψαν εἰς τὴν γῆν τῶν Φυλιστιείμ.
\vs{33}Καὶ ἐφύτευσεν Ἁβραὰμ ἄρουραν ἐπὶ τῷ φρέατι τοῦ ὅρκου· καὶ ἐπεκαλέσατο ἐκεῖ τὸ ὄνομα Κυρίου, Θεὸς αἰώνιος.
\vs{34}Παρῴκησε δὲ Ἁβραὰμ ἐν τῇ γῇ τῶν Φυλιστιεὶμ ἡμέρας πολλάς.

\ch{22}
Καὶ ἐγένετο μετὰ τὰ ῥήματα ταῦτα ὁ Θεὸς ἐπείρασε τὸν Ἁβραὰμ, καὶ εἶπεν αὐτῷ, Ἁβραὰμ, Ἁβραάμ· καὶ εἶπεν, ἰδοὺ ἐγώ.
\vs{2}Καὶ εἶπε, λάβε τὸν υἱόν σου τὸν ἀγαπητὸν, ὃν ἠγάπησας, τὸν Ἰσαὰκ, καὶ πορεύθητι εἰς τὴν γῆν τὴν ὑψηλὴν, καὶ ἀνένεγκε αὐτὸν ἐκεῖ εἰς ὁλοκάρπωσιν ἐφʼ ἓν τῶν ὀρέων ὧν ἄν σοι εἴπω.
\vs{3}Ἀναστὰς δὲ Ἁβραὰμ τὸ πρωῒ, ἐπέσαξε τὴν ὄνον αὐτοῦ· παρέλαβε δὲ μεθʼ ἑαυτοῦ δύο παῖδας, καὶ Ἰσαὰκ τὸν υἱὸν αὐτοῦ· καὶ σχίσας ξύλα εἰς ὁλοκάρπωσιν, ἀναστὰς ἐπορεύθη, καὶ ἦλθεν ἐπὶ τὸν τόπον, ὃν εἶπεν αὐτῷ ὁ Θεὸς, τῇ ἡμέρᾳ τῇ τρίτῃ.
\vs{4}Καὶ ἀναβλέψας Ἁβραὰμ τοῖς ὀφθαλμοῖς αὐτοῦ, εἶδε τὸν τόπον μακρόθεν.
\vs{5}Καὶ εἶπεν Ἁβραὰμ τοῖς παισὶν αὐτοῦ, καθίσατε αὐτοῦ μετὰ τῆς ὄνου· ἐγὼ δὲ καὶ τὸ παιδάριον διελευσόμεθα ἕως ὧδε· καὶ προσκυνήσαντες ἀναστρέψομεν πρὸς ὑμᾶς.
\vs{6}Ἔλαβε δὲ Ἁβραὰμ τὰ ξύλα τῆς ὁλοκαρπώσεως, καὶ ἐπέθηκεν Ἰσαὰκ τῷ υἱῷ αὐτοῦ· ἔλαβε δὲ μετὰ χεῖρας καὶ τὸ πῦρ καὶ τὴν μάχαιραν, καὶ ἐπορεύθησαν οἱ δύο ἅμα.
\vs{7}Εἶπε δὲ Ἰσαὰκ πρὸς Ἁβραὰμ τὸν πατέρα αὐτοῦ, πάτερ· ὁ δὲ εἶπε, τί ἐστι, τέκνον; εἶπε, δὲ, ἰδοὺ τὸ πῦρ καὶ τὰ ξύλα, ποῦ ἐστὶ τὸ πρόβατον τὸ εἰς ὁλοκάρπωσιν;
\vs{8}Εἶπε δὲ Ἁβραὰμ, ὁ Θεὸς ὄψεται ἑαυτῷ πρόβατον εἰς ὁλοκάρπωσιν, τέκνον. πορευθέντες δὲ ἀμφότεροι ἅμα,
\vs{9}ἦλθον ἐπὶ τὸν τόπον, ὃν εἶπεν αὐτῷ ὁ Θεός· καὶ ᾠκοδόμησεν ἐκεῖ Ἁβραὰμ τὸ θυσιαστήριον, καὶ ἐπέθηκε τὰ ξύλα· καὶ συμποδίσας Ἰσαὰκ τὸν υἱὸν αὐτοῦ, ἐπέθηκεν αὐτὸν ἐπὶ τὸ θυσιαστήριον ἐπάνω τῶν ξύλων.
\vs{10}Καὶ ἐξέτεινεν Ἁβραὰμ τὴν χεῖρα αὐτοῦ λαβεῖν τὴν μάχαιραν, σφάξαι τὸν υἱὸν αὐτοῦ.
\vs{11}Καὶ ἐκάλεσεν αὐτὸν Ἄγγελος Κυρίου ἐκ τοῦ οὐρανοῦ, καὶ εἶπεν, Ἁβραὰμ, Ἁβραάμ· ὁ δὲ εἶπεν, ἰδοὺ ἐγώ.
\vs{12}Καὶ εἶπε, μὴ ἐπιβάλῃς τὴν χεῖρά σου ἐπὶ τὸ παιδάριον, μηδὲ ποιήσῃς αὐτῷ μηδέν· νῦν γὰρ ἔγνων, ὅτι φοβῇ σὺ τὸν Θεόν· καὶ οὐκ ἐφείσω τοῦ υἱοῦ σου τοῦ ἀγαπητοῦ διʼ ἐμέ.
\vs{13}Καὶ ἀναβλέψας Ἁβραὰμ τοῖς ὀφθαλμοῖς αὐτοῦ εἶδε, καὶ ἰδοὺ κριὸς εἷς κατεχόμενος ἐν φυτῷ Σαβὲκ τῶν κεράτων. Καὶ ἐπορεύθη Ἁβραὰμ, καὶ ἔλαβε τὸν κριὸν, καὶ ἁνήνεγκεν αὐτὸν εἰς ὁλοκάρπωσιν ἀντὶ Ἰσαὰκ τοῦ υἱοῦ αὐτοῦ.

\vs{14}Καὶ ἐκάλεσεν Ἁβραὰμ τὸ ὄνομα τοῦ τόπου ἐκείνου, Κύριος εἶδεν· ἵνα εἴπωσιν σήμερον, ἐν τῷ ὄρει Κύριος ὤφθη.
\vs{15}Καὶ ἐκάλεσεν Ἄγγελος Κυρίου τὸν Ἁβραὰμ δεύτερον ἐκ τοῦ οὐρανοῦ,
\vs{16}λέγων, Κατʼ ἐμαυτοῦ ὤμοσα, λέγει Κύριος, οὗ εἵνεκεν ἐποίησας τὸ ῥῆμα τοῦτο, καὶ οὐκ ἐφείσω τοῦ υἱοῦ σου τοῦ ἀγαπτοῦ διʼ ἐμὲ,
\vs{17}Ἦ μὴν εὐλογῶν εὐλογήσω σε, καὶ πληθύνων πληθυνῶ τὸ σπέρμα σου, ὡς τοὺς ἀστέρας τοῦ οὐρανοῦ, καὶ ὡς τὴν ἄμμον τὴν παρὰ τὸ χεῖλος τῆς θαλάσσης· καὶ κληρονομήσει τὸ σπέρμα σου τὰς πόλεις τῶν ὑπεναντίων.
\vs{18}Καὶ ἐνευλογηθήσονται ἐν τῷ σπέρματί σου πάντα τὰ ἔθνη τῆς γῆς, ἀνθʼ ὧν ὑπήκουσας τῆς ἐμῆς φωνῆς.
\vs{19}Ἀπεστράφη δὲ Ἁβραὰμ πρὸς τοὺς παῖδας αὐτοῦ· καὶ ἀναστάντες ἐπορεύθησαν ἅμα ἐπὶ τὸ φρέαρ τοῦ ὅρκου. Καὶ κατῴκησεν Ἁβραὰμ ἐπὶ τὸ φρέαρ τοῦ ὅρκου.

\vs{20}Ἐγένετο δὲ μετὰ τὰ ῥήματα ταῦτα, καὶ ἀνηγγέλη τῷ Ἁβραὰμ, λέγοντες, ἰδοὺ τέτοκε Μελχὰ καὶ αὐτὴ υἱοὺς τῷ Ναχὼρ τῷ ἀδελφῷ σου,
\vs{21}τὸν Οὒζ πρωτότοκον, καὶ τὸν Βαὺξ ἀδελφὸν αὐτοῦ, καὶ τὸν Καμουὴλ πατέρα Σύρων,
\vs{22}καὶ τὸν Χαζὰδ, καὶ Ἀζαῦ, καὶ τὸν Φαλδὲς, καὶ τὸν Ἰελδὰφ, καὶ τὸν Βαθουήλ.
\vs{23}Βαθουὴλ δὲ ἐγέννησε τὴν Ῥεβέκκαν. ὀκτὼ οὗτοι υἱοὶ, οὓς ἔτεκε Μελχὰ τῷ Ναχὼρ τῷ ἀδελφῷ Ἁβραάμ.
\vs{24}Καὶ ἡ παλλακὴ αὐτοῦ, ᾗ ὄνομα Ῥεύμα, ἔτεκε καὶ αὐτὴ τὸν Ταβὲκ, καὶ τὸν Ταὰμ, καὶ τὸν Τοχός, καὶ τὸν Μοχά.

\ch{23}
Ἐγένετο δὲ ἡ ζωὴ Σάῤῥας, ἔτη ἑκατὸν εἰκοσιεπτά.
\vs{2}Καὶ ἀπέθανε Σάῤῥα ἐν πόλει Ἀρβὸκ, ἥ ἐστιν ἐν τῷ κοιλώματι· αὕτη ἔστι Χεβρὼν ἐν τῇ γῇ Χαναάν ἦλθε δὲ Ἁβραὰμ κόψασθαι Σάῤῥαν, καὶ πενθῆσαι.
\vs{3}Καὶ ἀνέστη Ἁβραὰμ ἀπὸ τοῦ νεκροῦ αὐτοῦ· καὶ εἶπεν Ἁβραὰμ τοῖς υἱοῖς τοῦ Χὲτ, λέγων,
\vs{4}Πάροικος καὶ παρεπίδημος ἐγώ εἰμι μεθʼ ὑμῶν· δότε μοι οὖν κτῆσιν τάφου μεθʼ ὑμῶν, καὶ θάψω τὸν νεκρόν μου ἀπʼ ἐμοῦ.
\vs{5}Ἀπεκρίθησαν δὲ οἱ υἱοὶ Χὲτ πρὸς Ἁβραὰμ, λέγοντες, μὴ, κύριε.
\vs{6}Ἄκουσον δὲ ἡμῶν· βασιλεὺς παρὰ Θεοῦ σὺ εἶ ἐν ἡμῖν· ἐν τοῖς ἐκλεκτοῖς μνημείοις ἡμῶν θάψον τὸν νεκρόν σου· οὐδεὶς γὰρ ἡμῶν οὐ μὴ κωλύσει τὸ μνημεῖον αὐτοῦ ἀπὸ σοῦ, τοῦ θάψαι τὸν νεκρόν σου ἐκεῖ.
\vs{7}Ἀναστὰς δὲ Ἁβραὰμ προσεκύνησε τῷ λαῷ τῆς γῆς, τοῖς υἱοῖς τοῦ Χέτ.
\vs{8}Καὶ ἐλάλησε πρὸς αὐτοὺς Ἁβραὰμ, λέγων, εἰ ἔχετε τῇ ψυχῇ ὑμῶν, ὥστε θάψαι τὸν νεκρόν μου ἀπὸ προσώπου μου, ἀκούσατέ μου, καὶ λαλήσατε περὶ ἐμοῦ Ἐφρὼν τῷ τοῦ Σαάρ.
\vs{9}Καὶ δότω μοι τὸ σπήλαιον τὸ διπλοῦν, ὅ ἐστιν αὐτῷ, τὸ ὂν ἐν μέρει τοῦ ἀγροῦ αὐτοῦ· ἀργυρίου τοῦ ἀξίου δότε μοι αὐτὸ ἐν ὑμῖν εἰς κτῆσιν μνημείου.
\vs{10}Ἐφρὼν δὲ ἐκάθητο ἐν μέσῳ τῶν υἱῶν Χέτ· ἀποκριθεὶς δὲ Ἐφρὼν ὁ Χετταῖος πρὸς Ἁβραὰμ εἶπεν, ἀκουόντων τῶν υἱῶν Χὲτ, καὶ τῶν εἰσπορευομένων εἰς τὴν πόλιν πάντων, λέγων,
\vs{11}Παρʼ ἐμοὶ γενοῦ, κύριε, καὶ ἄκουσόν μου· τὸν ἀγρὸν, καὶ τὸ σπήλαιον τὸ ἐν αὐτῷ, σοὶ δίδωμι· ἐναντίον πάντων τῶν πολιτῶν μου δέδωκά σοι· θάψον τὸν νεκρόν σου.
\vs{12}Καὶ προσεκύνησεν Ἁβραὰμ ἐναντίον τοῦ λαοῦ τῆς γῆς.
\vs{13}Καὶ εἶπε τῷ Ἐφρὼν εἰς τὰ ὦτα ἐναντίον τοῦ λαοῦ τῆς γῆς, ἐπειδὴ πρὸς ἐμοῦ εἶ, ἄκουσόν μου· τὸ ἀργύριον τοῦ ἀγροῦ λάβε παρʼ ἐμοῦ, καὶ θάψω τὸν νεκρόν μου ἐκεῖ.
\vs{14}Ἀπεκρίθη δὲ Ἐφρὼν τῷ Ἁβραὰμ, λέγων,
\vs{15}Οὐχὶ, κύριε· ἀκήκοα γὰρ, γῆ τετρακοσίων διδράχμων ἀργύριου· ἀλλὰ τί ἂν εἴη τοῦτο ἀνὰ μέσον ἐμοῦ καὶ σοῦ; σὺ δὲ τὸν νεκρόν σου θάψον.
\vs{16}καὶ ἤκουσεν Ἁβραὰμ τοῦ Ἐφρών· καὶ ἀπεκατέστησεν Ἁβραὰμ τῷ Ἐφρὼν τὸ ἀργύριον, ὃ ἐλάλησεν εἰς τὰ ὦτα τῶν υἱῶν Χὲτ, τετρακόσια δίδραχμα ἀργυρίου δοκίμου ἐμπόροις.
\vs{17}Καὶ ἔστη ὁ ἀγρὸς Ἐφρών, ὃς ἦν ἐν τῷ διπλῷ σπηλαίῳ, ὅς ἐστι κατὰ πρόσωπον Μαμβρῆ, ὁ ἀγρὸς καὶ τὸ σπήλαιον, ὃ ἦν ἐν αὐτῷ, καὶ πᾶν δένδρον, ὃ ἦν ἐν τῷ ἀγρῷ, καὶ πᾶν ὅ ἐστιν ἐν τοῖς ὁρίοις αὐτοῦ κύκλῳ,
\vs{18}τῷ Ἁβραὰμ, εἰς κτῆσιν ἐναντίον τῶν υἱῶν Χὲτ, καὶ πάντων τῶν εἰσπορευομένων εἰς τὴν πόλιν.
\vs{19}Μετὰ ταῦτα ἔθαψεν Ἁβραὰμ Σάῤῥαν τὴν γυναῖκα αὐτοῦ ἐν τῷ σπηλαίῳ τοῦ ἀγροῦ τῷ διπλῷ, ὅ ἐστιν ἀπέναντι Μαμβρῆ· αὕτη ἐστὶ Χεβρὼν ἐν τῇ γῇ Χαναάν.
\vs{20}Καὶ ἐκυρώθη ὁ ἀγρὸς καὶ τὸ σπήλαιον ὃ ἦν ἐν αὐτῷ τῷ Ἁβραὰμ εἰς κτῆσιν τάφου, παρὰ τῶν υἱῶν Χέτ.

\ch{24}Καὶ Ἁβραὰμ ἦν πρεσβύτερος προβεβηκὼς ἡμερῶν· καὶ Κύριος ηὐλόγησε τὸν Ἁβραὰμ κατὰ πάντα.

\vs{2}Καὶ εἶπεν Ἁβραὰμ τῷ παιδὶ αὐτοῦ τῷ πρεσβυτέρῳ τῆς οἰκίας αὐτοῦ, τῷ ἄρχοντι πάντων τῶν αὐτοῦ, θὲς τὴν χεῖρά σου ὑπὸ τὸν μηρόν μου.
\vs{3}Καὶ ἐξορκιῶ σε Κύριον τὸν Θεὸν τοῦ οὐρανοῦ καὶ τὸν Θεὸν τῆς γῆς, ἵνα μὴ λάβῃς γυναῖκα τῷ υἱῷ μου Ἰσαὰκ ἀπὸ τῶν θυγατέρων τῶν Χαναναίων, μεθʼ ὧν ἐγὼ οἰκῶ ἐν αὐτοις.
\vs{4}Ἀλλʼ ἢ εἰς τὴν γῆν μου, οὗ ἐγεννήθην, πορεύσῃ, καὶ εἰς τὴν φυλήν μου, καὶ λήψῃ γυναῖκα τῷ υἱῷ μου Ἰσαὰκ ἐκεῖθεν.
\vs{5}Εἶπε δὲ πρὸς αὐτὸν ὁ παῖς, μή ποτε οὐ βούληται ἡ γυνὴ πορευθῆναι μετʼ ἐμοῦ ὀπίσω εἰς τὴν γῆν ταύτην, ἀποστρέψω τὸν υἱόν σου εἰς τὴν γῆν, ὅθεν ἐξῆλθες ἐκεῖθεν;
\vs{6}Εἶπε δὲ πρὸς αὐτὸν Ἁβραάμ, πρόσεχε σεαυτῷ μὴ ἀποστρέψῃς τὸν υἱόν μου ἐκεῖ.
\vs{7}Κύριος ὁ Θεὸς τοῦ οὐρανοῦ καὶ ὁ Θεὸς τῆς γῆς, ὃς ἔλαβέ με ἐκ τοῦ οἴκου τοῦ πατρός μου, καἰ ἐκ τῆς γῆς ἧς ἐγεννήθην, ὃς ἐλάλησέ μοι, καὶ ὃς ὤμοσέ μοι, λέγων, σοὶ δώσω τὴν γῆν ταύτην καὶ τῷ σπέρματί σου, αὐτὸς ἀποστελεῖ τὸν Ἄγγελον αὐτοῦ ἔμπροσθέν σου, καὶ λήψῃ γυναῖκα τῷ υἱῷ μου ἐκεῖθεν.
\vs{8}Ἐὰν δὲ μὴ θέλῃ ἡ γυνὴ πορευθῆναι μετὰ σοῦ εἰς τὴν γῆν ταύτην, καθαρὸς ἔσῃ ἀπὸ τοῦ ὅρκου μου· μόνον τὸν υἱόν μου μὴ ἀποστρέψῃς ἐκεῖ.
\vs{9}Καὶ ἔθηκεν ὁ παῖς τὴν χεῖρα αὐτοῦ ὑπὸ τὸν μηρὸν Ἁβραὰμ τοῦ κυρίου αὐτοῦ, καὶ ὤμοσεν αὐτῷ περὶ τοῦ ῥήματος τούτου.
\vs{10}Καὶ ἔλαβεν ὁ παῖς δέκα καμήλους ἀπὸ τῶν καμήλων τοῦ κυρίου αὐτοῦ, καὶ ἀπὸ πάντων τῶν ἀγαθῶν τοῦ κυρίου αὐτοῦ μεθʼ ἑαυτοῦ· καὶ ἀναστὰς ἐπορεύθη εἰς τὴν Μεσοποταμίαν εἰς τὴν πόλιν Ναχώρ.
\vs{11}Καὶ ἐκοίμησε τὰς καμήλους ἔξω τῆς πόλεως παρὰ τὸ φρέαρ τοῦ ὕδατος τὸ πρὸς ὀψέ, ἡνίκα ἐκπορεύονται αἱ ὑδρευόμεναι.

\vs{12}Καὶ εἶπε, Κύριε ὁ Θεὸς τοῦ κυρίου μου Ἁβραάμ, εὐόδωσον ἐναντίον ἐμοῦ σήμερον, καὶ ποίησον ἔλεος μετὰ τοῦ κυρίου μου Ἁβραάμ.
\vs{13}Ἰδοὺ ἐγὼ ἕστηκα ἐπὶ τῆς πηγῆς τοῦ ὕδατος· αἱ δὲ θυγατέρες τῶν οἰκούντων τὴν πόλιν ἐκπορεύονται ἀντλῆσαι ὕδωρ.
\vs{14}Καὶ ἔσται ἡ παρθένος ᾗ ἂν ἐγὼ εἴπω, ἐπίκλινον τὴν ὑδρίαν σου, ἵνα πίω, καὶ εἴπῃ μοι, πίε σύ, καὶ τὰς καμήλους σου ποτιῶ, ἕως ἂν παύσωνται πίνουσαι, ταύτην ἡτοίμασας τῷ παιδί σου τῷ Ἰσαάκ· καὶ ἐν τούτῳ γνώσομαι, ὅτι ἐποίησας ἔλεος μετὰ τοῦ κυρίου μου Ἁβραάμ.

\vs{15}Καὶ ἐγένετο πρὸ τοῦ συντελέσαι αὐτὸν λαλοῦντα ἐν τῇ διανοίᾳ αὐτοῦ, καὶ ἰδοὺ Ῥεβέκκα ἐξεπορεύετο ἡ τεχθεῖσα Βαθουήλ, υἱῷ Μελχὰς τῆς γυναικὸς Ναχώρ, ἀδελφοῦ δὲ Ἁβραάμ, ἔχουσα τὴν ὑδρίαν ἐπὶ τῶν ὤμων αὐτῆς.
\vs{16}Ἡ δὲ παρθένος ἦν καλὴ τῇ ὄψει σφόδρα· παρθένος ἦν, ἀνὴρ οὐκ ἔγνω αὐτήν· καταβᾶσα δὲ ἐπὶ τὴν πηγὴν, ἔπλησε τὴν ὑδρίαν αὐτῆς, καὶ ἀνέβη.
\vs{17}Ἐπέδραμε δὲ ὁ παῖς εἰς συνάντησιν αὐτῆς, καὶ εἶπε, Πότισόν με μικρὸν ὕδωρ ἐκ τῆς ὑδρίας σου.
\vs{18}Ἡ δὲ εἶπε, πίε, κύριε· καὶ ἔσπευσε καὶ καθεῖλε τὴν ὑδρίαν ἐπὶ τὸν βραχίονα αὐτῆς, καὶ ἐπότισεν αὐτὸν, ἕων ἐπαύσατο πίνων.
\vs{19}Καὶ εἶπε, καὶ ταῖς καμήλοις σου ὑδρεύσομαι, ἕως ἂν πᾶσαι πίωσι.
\vs{20}Καὶ ἔσπευσε καὶ ἐξεκένωσε τὴν ὑδρίαν εἰς τὸ ποτιστήριον· καὶ ἔδραμεν ἐπὶ τὸ φρέαρ ἀντλῆσαι πάλιν· καὶ ὑδρεύσατο πάσαις ταῖς καμήλοις.
\vs{21}Ὁ δὲ ἄνθρωπος κατεμάνθανεν αὐτήν· καὶ παρεσιώπα τοῦ γνῶναι εἰ εὐώδωκε Κύριος τὴν ὁδὸν αὐτοῦ, ἢ οὔ.
\vs{22}Ἐγένετο δὲ ἡνίκα ἐπαύσαντο πᾶσαι αἱ κάμηλοι πίνουσαι, ἔλαβεν ὁ ἄνθρωπος ἐνώτια χρυσᾶ ἀνὰ δραχμὴν ὁλκῆς, καὶ δύο ψέλλια ἐπὶ τὰς χεῖρας αὐτῆς, δέκα χρυσῶν ὁλκὴ αὐτῶν.
\vs{23}Καὶ ἐπηρώτησεν αὐτὴν, καὶ εἶπε, θυγάτηρ τίνος εἶ; ἀνάγγειλόν μοι, εἰ ἔστι παρὰ τῷ πατρί σου τόπος ἡμῖν του καταλῦσαι.
\vs{24}Ἡ δὲ εἶπεν αὐτῷ, θυγάτηρ Βαθουήλ εἰμι τοῦ Μελχάς, ὃν ἔτεκε τῷ Ναχώρ.
\vs{25}Καὶ εἶπεν αὐτῷ, Καὶ ἄχυρα καὶ χορτάσματα πολλὰ παρʼ ἡμῖν, καὶ τόπος τοῦ καταλῦσαι.
\vs{26}Καὶ εὐδοκήσας ὁ ἄνθρωπος προσεκύνησε τῷ Κυρίῳ
\vs{27}Καὶ εἶπεν, εὐλογητὸς Κύριος ὁ Θεὸς τοῦ κυρίου μου Ἁβραάμ, ὃς οὐκ ἐγκατέλειπε τὴν δικαιοσύνην αὐτοῦ, καὶ τὴν ἀλήθειαν, ἀπὸ τοῦ κυρίου μου· ἐμὲ τʼ εὐώδωκε Κύριος εἰς οἶκον τοῦ ἀδελφοῦ τοῦ κυρίου μου.
\vs{28}Καὶ δραμοῦσα ἡ παῖς ἀνήγγειλεν εἰς τὸν οἶκον τῆς μητρὸς αὐτῆς, κατὰ τὰ ῥήματα ταῦτα.
\vs{29}Τῇ δὲ Ῥεβέκκᾷ ἀδελφὸς ἦν, ᾧ ὄνομα Λάβαν· καὶ ἔδραμε Λάβαν πρὸς τὸν ἄνθρωπον ἔξω ἐπὶ τὴν πηγήν.
\vs{30}Καὶ ἐγένετο ἡνίκα εἶδε τὰ ἐνώτια, καὶ τὰ ψέλλια ἐν ταῖς χερσὶ τῆς ἀδελφῆς αὐτοῦ, καὶ ὅτε ἤκουσε τὰ ῥήματα Ῥεβέκκας τῆς ἀδελφῆς αὐτοῦ, λεγούσης, οὕτω λελάληκέ μοι ὁ ἄνθρωπος, καὶ ἦλθε πρὸς τὸν ἄνθρωπον, ἑστηκότος αὐτοῦ ἐπὶ τῶν καμήλων ἐπὶ τῆς πηγῆς.
\vs{31}Καὶ εἶπεν αὐτῷ, δεῦρο εἴσελθε, εὐλογητὸς Κυροίυ· ἱνατί ἕστηκας ἔξω; ἐγὼ δὲ ἡτοίμασα τὴν οἰκίαν, καὶ τόπον ταῖς καμήλοις.
\vs{32}Εἰσῆλθε δὲ ὁ ἄνθρωπος εἰς τὴν οἰκίαν, καὶ ἀπέσαξε τὰς καμήλους· καὶ ἔδωκεν ἄχυρα καὶ χορτάσματα ταῖς καμήλοις, καὶ ὕδωρ νίψασθαι τοῖς ποσὶν αὐτοῦ, καὶ τοῖς ποσὶ τῶν ἀνδρῶν τῶν μετʼ αὐτοῦ.
\vs{33}Καὶ παρέθηκεν αὐτοῖς ἄρτους φαγεῖν· καὶ εἶπεν, οὐ μὴ φάγω, ἕως τοῦ λαλῆσαί με τὰ ῥήματά μου· καὶ εἶπεν, λάλησον.

\vs{34}Καὶ εἶπε, παῖς Ἁβραὰμ ἐγώ εἰμι.
\vs{35}Κύριος δὲ ηὐλόγησε τὸν κύριόν μου σφόδρα, καὶ ὑψώθη· καὶ ἔδωκεν αὐτῷ πρόβατα, καὶ μόσχους, καὶ ἀργύριον, καὶ χρυσίον, παῖδας, καὶ παιδίσκας, καμήλους, καὶ ὄνους.
\vs{36}Καὶ ἔτεκε Σάῤῥα ἡ γυνὴ τοῦ κυρίου μου υἱὸν ἕνα τῷ κυρίῳ μου μετὰ τὸ γηράσαι αὐτόν· καὶ ἔδωκεν αὐτῷ ὅσα ἦν αὐτῷ.
\vs{37}Καὶ ὥρκισέ με ὁ κύριός μου, λέγων, οὐ λήμψῃ γυναῖκα τῷ υἱῷ μου ἀπὸ τῶν θυγατέρων τῶν Χαναναίων, ἐν οἷς ἐγὼ παροικῶ ἐν τῇ γῇ αὐτῶν.
\vs{38}Ἀλλʼ εἰς τὸν οἶκον τοῦ πατρός μου πορεύσῃ, καὶ εἰς τὴν φυλήν μου, καὶ λήψῃ γυναῖκα τῷ υἱῷ μου ἐκεῖθεν.
\vs{39}Εἶπα δὲ τῷ κυρίῳ μου, μήποτε οὐ πορεύσεται ἡ γυνὴ μετʼ ἐμοῦ.
\vs{40}Καὶ εἶπέ μοι, Κύριος ὁ Θεὸς ᾧ εὐηρέστησα ἐναντίον αὐτοῦ, αὐτὸς ἐξαποστελεῖ τὸν Ἀγγελον αὐτοῦ μετὰ σοῦ, καὶ εὐοδώσει τὴν ὁδόν σου· καὶ λήψῃ γυναῖκα τῷ υἱῷ μου ἐκ τῆς φυλῆς μου, καὶ ἐκ τοῦ οἴκου τοῦ πατρός μου.
\vs{41}Τότε ἀθῷος ἔσῃ ἀπὸ τῆς ἀρᾶς μου· ἡνίκα γὰρ ἐὰν ἔλθῃς εἰς τὴν φυλήν μου, καὶ μή σοι δῶσι, καὶ ἔσῃ ἀθῷος ἀπὸ τοῦ ὁρκισμοῦ μου.
\vs{42}Καὶ ἐλθὼν σήμερον ἐπὶ τὴν πηγὴν εἶπα, Κύριε ὁ Θεὸς τοῦ κυρίου μου Ἁβραὰμ, εἰ σὺ εὐοδοῖς τὴν ὁδόν μου, ἐν ᾗ νῦν ἐγὼ πορεύομαι ἐν αὐτῇ,
\vs{43}ἰδοὺ ἐγὼ ἐφέστηκα ἐπὶ τῆς πηγῆς τοῦ ὕδατος, καὶ αἱ θυγατέρες τῶν ἀνθρώπων τῆς πόλεως ἐκπορεύονται ἀντλῆσαι ὕδωρ· καὶ ἔσται ἡ παρθένος, ᾗ ἂν ἐγὼ εἴπω, πότισόν με ἐκ τῆς ὑδρίας σου μικρὸν ὕδωρ,
\vs{44}καὶ εἴπῃ μοι, καὶ σὺ πίε, καὶ ταῖς καμήλοις σου ὑδρεύσομαι, αὕτη ἡ γυνὴ ἣν ἡτοίμασε Κύριος τῷ ἑαυτοῦ θεράποντι Ἰσαάκ· καὶ ἐν τούτῳ γνώσομαι, ὅτι πεποίηκας ἔλεος τῷ κυρίῳ μου Ἁβραάμ.
\vs{45}Καὶ ἐγένετο πρὸ τοῦ συντελέσαι με λαλοῦντα ἐν τῇ διανοίᾳ μου, εὐθὺς Ῥεβέκκα ἐξεπορεύετο, ἔχουσα τὴν ὑδρίαν ἐπὶ τῶν ὤμων· καὶ κατέβη ἐπὶ τὴν πηγὴν, καὶ ὑδρεύσατο· εἶπα δὲ αὐτῇ, πότισόν με.
\vs{46}Καὶ σπεύσασα καθεῖλε τὴν ὑδρίαν ἐπὶ τὸν βραχίονα αὐτῆς ἀφʼ ἑαυτῆς, καὶ εἶπε, πίε σὺ, καὶ τὰς καμήλους σου ποτιῶ· καὶ ἔπιον, καὶ τὰς καμήλους ἐπότισε.
\vs{47}Καὶ ἠρώτησα αὐτὴν, καὶ εἶπα, θυγάτηρ τίνος εἶ, ἀναγγειλόν μοι· ἡ δὲ ἔφη, θυγάτηρ Βαθουὴλ εἰμὶ υἱοῦ τοῦ Ναχὼρ, ὃν ἔτεκεν αὐτῷ Μελχά· καὶ περιέθηκα αὐτῇ τὰ ἐνώτια, καὶ τὰ ψέλλια περὶ τὰς χεῖρας αὐτῆς.
\vs{48}Καὶ εὐδοκήσας προσεκύνησα τῷ Κυρίῳ, καὶ εὐλόγησα Κύριον τὸν Θεὸν τοῦ κυρίου μου Ἁβραὰμ, ὃς εὐώδωσέ με ἐν ὁδῷ ἀληθείας λαβεῖν τὴν θυγατέρα τοῦ ἀδελφοῦ τοῦ κυρίου μου τῷ υἱῷ αὐτοῦ.
\vs{49}Εἰ οὖν ποιεῖτε ὑμεῖς ἔλεος καὶ δικαιοσύνην πρὸς τὸν κύριόν μου· εἰ δὲ μὴ, ἀπαγγείλατέ μοι, ἵνα ἐπιστρέψω εἰς δεξιὰν ἤ ἀριστεράν.

\vs{50}Ἀποκριθεὶς δὲ Λάβαν καὶ Βαθουὴλ εἶπαν, παρὰ κυρίου ἐξῆλθε τὸ πρᾶγμα τοῦτο· οὐ δυνησόμεθά σοι ἀντειπεῖν κακὸν ἢ καλόν.
\vs{51}Ἰδοὺ Ῥεβέκκα ἐνώπιόν σου· λαβὼν ἀπότρεχε· καὶ ἔστω γυνὴ τῷ υἱῷ τοῦ κυρίου σου, καθὰ ἐλάλησε Κύριος.
\vs{52}Ἐγένετο δὲ ἐν τῷ ἀκοῦσαι τὸν παῖδα τοῦ Ἁβραὰμ τῶν ῥημάτων αὐτῶν, προσεκύνησεν ἐπὶ τὴν γῆν τῷ κυρίῳ.
\vs{53}καὶ ἐξενέγκας ὁ παῖς σκεύη ἀργυρᾶ καὶ χρυσᾶ καὶ ἱματισμὸν, ἔδωκε τῇ Ῥεβέκκᾳ· καὶ δῶρα ἔδωκε τῷ ἀδελφῷ αὐτῆς, καὶ τῇ μητρὶ αὐτῆς.
\vs{54}Καὶ ἔφαγον καὶ ἔπιον καὶ αὐτὸς καὶ οἱ ἄνδρες οἱ μετʼ αὐτοῦ ὄντες, καὶ ἐκοιμήθησαν· καὶ ἀναστὰς τὸ πρωῒ εἶπεν, ἐκπέμψατέ με, ἵνα ἀπέλθω πρὸς τὸν κύριόν μου.
\vs{55}Εἶπαν δὲ οἱ ἀδελφοὶ αὐτῆς, καὶ ἡ μήτηρ, μεινάτω ἡ παρθένος μεθʼ ἡμῶν ἡμέρας ὡσεὶ δέκα, καὶ μετὰ ταῦτα ἀπελεύσεται.
\vs{56}Ὁ δὲ εἶπε πρὸς αὐτοὺς, μὴ κατέχετέ με· καὶ Κύριος εὐώδωσε τὴν ὁδόν μου ἐν ἐμοί· ἐκπέμψατέ με, ἵνα ἀπέλθω πρὸς τὸν κύριόν μου.
\vs{57}Οἱ δὲ εἶπαν, Καλέσωμεν τὴν παῖδα, καὶ ἐρωτήσωμεν τὸ στόμα αὐτῆς.
\vs{58}Καὶ ἐκάλεσαν τὴν Ῥεβέκκαν, καὶ εἶπαν αὐτῇ, πορεύσῃ μετὰ τοῦ ἀνθρώπου τούτου; ἡ δὲ εἶπε, πορεύσομαι.
\vs{59}Καὶ ἐξέπεμψαν Ῥεβέκκαν τὴν ἀδελφὴν αὐτῶν, καὶ τὰ ὑπάρχοντα αὐτῆς, καὶ τὸν παῖδα τοῦ Ἁβραὰμ, καὶ τοὺς μετʼ αὐτοῦ.
\vs{60}Καὶ εὐλόγησαν Ῥεβέκκαν, καὶ εἶπαν αὐτῇ, ἀδελφὴ ἡμῶν εἶ, γίνου εἰς χιλιάδας μυριάδων, καὶ κληρονομησάτω τὸ σπέρμα σου τὰς πόλεις τῶν ὑπεναντίων.
\vs{61}Ἀναστᾶσα δὲ Ῥεβέκκα καὶ αἱ ἅβραι αὐτῆς, ἐπέβησαν ἐπὶ τὰς καμήλους, καὶ ἐπορεύθησαν μετὰ τοῦ ἀνθρώπου· καὶ ἀναλαβὼν ὁ παῖς τὴν Ῥεβέκκαν ἀπῆλθεν.

\vs{62}Ἰσαὰκ δὲ διεπορεύετο διὰ τῆς ἐρήμου κατὰ τὸ φρέαρ τῆς ὁράσεως· αὐτὸς δὲ κατῴκει ἐν τῇ γῇ τῇ πρὸς Λίβα.
\vs{63}Καὶ ἐξῆλθεν Ἰσαὰκ ἀδολεσχῆσαι εἰς τὸ πεδίον τὸ πρὸς δείλης, καὶ ἀναβλέψας τοῖς ὀφθαλμοῖς αὐτοῦ εἶδε καμήλους ἐρχομένας.
\vs{64}Καὶ ἀναβλέψασα Ῥεβέκκα τοῖς ὀφθαλμοῖς εἶδε τὸν Ἰσαάκ· καὶ κατεπήδησεν ἀπὸ τῆς καμήλου.
\vs{65}Καὶ εἶπε τῷ παιδὶ, τίς ἐστιν ὁ ἄνθρωπος ἐκεῖνος ὁ πορευόμενος ἐν τῷ πεδίῳ εἰς συνάντησιν ἡμῖν; εἶπε δὲ ὁ παῖς, οὗτός ἐστιν ὁ κύριός μου· ἡ δὲ λαβοῦσα τὸ θέριστρον, περιεβάλετο.
\vs{66}Καὶ διηγήσατο ὁ παῖς τῷ Ἰσαὰκ πάντα τὰ ῥήματα, ἃ ἐποίησεν.
\vs{67}Εἰσῆλθε δὲ Ἰσαὰκ εἰς τὸν οἶκον τῆς μητρὸς αὐτοῦ, καὶ ἔλαβε τὴν Ῥεβέκκαν, καὶ ἐγένετο αὐτοῦ γυνὴ, καὶ ἠγάπησεν αὐτήν· καὶ παρεκλήθη Ἰσαὰκ περὶ Σάῤῥας τῆς μητρὸς αὐτοῦ.

\ch{25}
Προσθέμενος δὲ Ἁβραὰμ ἔλαβε γυναῖκα, ᾗ ὄνομα Χεττούρα.
\vs{2}Ἔτεκε δὲ αὐτῷ τὸν Ζομβρᾶν, καὶ τὸν Ἰεζὰν, καὶ τὸν Μαδὰλ, καὶ τὸν Μαδιὰμ, καὶ τὸν Ἰεσβὼκ, καὶ τὸν Σωίε.
\vs{3}Ἰεζὰν δὲ ἐγέννησε τὸν Σαβὰ, καὶ τὸν Δεδάν· υἱοὶ δὲ Δεδὰν Ἀσσουριεὶμ, καὶ Λατουσιεὶμ, καὶ Λαωμείμ.
\vs{4}Υἱοὶ δὲ Μαδιὰμ Γεφὰρ, καὶ Ἀφεὶρ, καὶ Ἐνὼχ, καὶ Ἀβειδὰ, καὶ Ἐλδαγά· πάντες οὗτοι ἦσαν υἱοὶ Χεττούρας.
\vs{5}Ἔδωκε δὲ Ἁβραὰμ πάντα τὰ ὑπάρχοντα αὐτοῦ Ἰσαὰκ τῷ υἱῷ αὐτοῦ.
\vs{6}Καὶ τοῖς υἱοῖς τῶν παλλακῶν αὐτοῦ ἔδωκεν Ἁβραὰμ δόματα, καὶ ἐξαπέστειλεν αὐτοὺς ἀπὸ Ἰσαὰκ τοῦ υἱοῦ αὐτοῦ, ἔτι ζῶντος αὐτοῦ, πρὸς ἀνατολὰς εἰς γῆν ἀνατολῶν.
\vs{7}Ταῦτα δὲ τὰ ἔτη ἡμερῶν τῆς ζωῆς Ἁβραὰμ ὅσα ἔζησεν, ἑκατὸν ἑβδομηκονταπεντε ἔτη.
\vs{8}Καὶ ἐκλείπων ἀπέθανεν Ἁβραὰμ ἐν γήρᾳ καλῷ πρεσβύτης, καὶ πλήρης ἡμερῶν, καὶ προσετέθη πρὸς τὸν λαὸν αὐτοῦ.
\vs{9}Καὶ ἔθαψαν αὐτὸν Ἰσαὰκ καὶ Ἰσμαὴλ οἱ υἱοὶ αὐτοῦ εἰς τὸ σπήλαιον τὸ διπλοῦν, εἰς τὸν ἀγρὸν Ἐφρων τοῦ Σαὰρ τοῦ Χετταίου, ὅς ἐστιν ἀπέναντι Μαμβρῆ,
\vs{10}τὸν ἀγρὸν καὶ τὸ σπήλαιον, ὃ ἐκτήσατο Ἁβραὰμ παρὰ τῶν υἱῶν τοῦ Χέτ· ἐκεῖ ἔθαψαν Ἁβραὰμ, καὶ Σάῤῥαν τὴν γυναῖκα αὐτοῦ.
\vs{11}Ἐγένετο δὲ μετὰ τὸ ἀποθανεῖν Ἁβραὰμ, εὐλόγησεν ὁ Θεὸς τὸν Ἰσαὰκ υἱὸν αὐτοῦ· καὶ κατῴκησεν Ἰσαὰκ παρὰ τὸ φρέαρ τῆς ὁράσεως.
\vs{12}Αὗται δὲ αἱ γενέσεις Ἰσμαὴλ τοῦ υἱοῦ Ἁβραὰμ, ὃν ἔτεκεν Ἄγαρ ἡ Αἰγυπτία, ἡ παιδίσκη Σάῤῥας, τῷ Ἁβραάμ.
\vs{13}Καὶ ταῦτα τὰ ὀνόματα τῶν υἱῶν Ἰσμαὴλ, κατʼ ὀνόματα τῶν γενεῶν αὐτοῦ· πρωτότοκος Ἰσμαὴλ, καὶ Ναβαϊὼθ, καὶ Κηδὰρ, καὶ Ναβδεὴλ, καὶ Μασσὰμ,
\vs{14}καὶ Μασμὰ, καὶ Δουμὰ, καὶ Μασσῆ,
\vs{15}καὶ Χοδδὰν, καὶ Θαιμὰν, καὶ Ἰετοὺρ, καὶ Ναφὲς, καὶ Κεδμά.
\vs{16}οὗτοί εἰσιν οἱ υἱοὶ Ἰσμαὴλ, καὶ ταῦτα τὰ ὀνόματα αὐτῶν ἐν ταῖς σκηναῖς αὐτῶν, καὶ ἐν ταῖς ἐπαύλεσιν αὐτῶν· δώδεκα ἄρχοντες κατὰ ἔθνη αὐτῶν.
\vs{17}Καὶ ταῦτα τὰ ἔτη τῆς ζωῆς Ἰσμαὴλ, ἑκατὸν τριακονταεπτὰ ἔτη· καὶ ἐκλείπων ἀπέθανε, καὶ προσετέθη πρὸς τὸ γένος αὐτοῦ.
\vs{18}Κατῴκησε δὲ ἀπὸ Εὐϊλὰτ ἕως Σοὺρ, ἥ ἐστι κατὰ πρόσωπον Αἰγύπτου ἕως ἐλθεῖν πρὸς Ἀσσυρίους· κατὰ πρόσωπον πάντων τῶν ἀδελφῶν αὐτοῦ κατῴκησε.

\vs{19}Καὶ αὗται αἱ γενέσεις Ἰσαὰκ τοῦ υἱοῦ Ἁβραάμ· Ἁβραάμ ἐγέννησε τὸν Ἰσαάκ.
\vs{20}Ἦν δὲ Ἰσαὰκ ἐτῶν τεσσαράκοντα ὅτε ἔλαβε τὴν Ῥεβέκκαν θυγατέρα Βαθουὴλ τοῦ Σύρου ἐκ τῆς Μεσοποταμίας Συρίας, ἀδελφὴν Λάβαν τοῦ Σύρου, ἑαυτῷ εἰς γυναῖκα.
\vs{21}Ἐδέετο δὲ Ἰσαὰκ Κυρίου περὶ Ῥεβέκκας τῆς γυναικὸς αὐτοῦ, ὅτι στεῖρα ἦν· ἐπήκουσε δὲ αὐτοῦ ὁ Θεὸς, καὶ συνέλαβεν ἐν γαστρὶ Ῥεβέκκα ἡ γυνὴ αὐτοῦ.
\vs{22}Ἐσκίρτων δὲ τὰ παιδία ἐν αὐτῇ· εἶπε δὲ, εἰ οὕτω μοι μέλλει γίνεσθαι, ἵνα τί μοι τοῦτο; ἐπορεύθη δὲ πυθέσθαι παρὰ Κυρίου.
\vs{23}Καὶ εἶπε Κύριος αὐτῇ, δύο ἔθνη ἐν γαστρί σου εἰσὶ, καὶ δύο λαοὶ ἐκ τῆς κοιλίας σου διασταλήσονται· καὶ λαὸς λαοῦ ὑπερέξει, καὶ ὁ μείζων δουλεύσει τῷ ἐλάσσονι.
\vs{24}Καὶ ἐπληρώθησαν αἱ ἡμέραι τοῦ τεκεῖν αὐτήν· καὶ τῇδε ἦν δίδυμα ἐν τῇ κοιλίᾳ αὐτῆς.
\vs{25}Ἐξῆλθε δὲ ὁ πρωτότοκος πυῤῥάκης· ὅλος, ὡσεὶ δορὰ, δασύς· ἐπωνόμασε δὲ τὸ ὄνομα αὐτοῦ, Ἡσαῦ.
\vs{26}Καὶ μετὰ τοῦτο ἐξῆλθεν ὁ ἀδελφὸς αὐτοῦ, καὶ ἡ χεὶρ αὐτοῦ ἐπειλημμένη τῆς πτέρνης Ἡσαῦ· καὶ ἐκάλεσε τὸ ὄνομα αὐτοῦ, Ἰακώβ. Ἰσαὰκ δὲ ἦν ἐτῶν ἑξήκοντα, ὅτε ἔτεκεν αὐτοὺς Ῥεβέκκα.
\vs{27}Ηὐξήθησαν δὲ οἱ νεανίσκοι· καὶ ἦν Ἡσαῦ ἄνθρωπος εἰδὼς κυνηγεῖν, ἄγροικος· Ἰακὼβ δὲ ἄνθρωπος ἄπλαστος, οἰκῶν οἰκίαν.
\vs{28}Ἠγάπησε δὲ Ἰσαὰκ τὸν Ἡσαῦ, ὅτι ἡ θήρα αὐτοῦ βρῶσις αὐτῷ· Ῥεβέκκα δὲ ἠγάπα τὸν Ἰακώβ.

\vs{29}Ἥψησε δὲ Ἰακὼβ ἕψημα· ἦλθε δὲ Ἡσαῦ ἐκ τοῦ πεδίου ἐκλείπων.
\vs{30}Καὶ εἶπεν Ἡσαῦ τῷ Ἰακὼβ, γεῦσόν με ἀπὸ τοῦ ἑψήματος πυῤῥου τούτου, ὅτι ἐκλείπω· διὰ τοῦτο ἐκλήθη τὸ ὄνομα αὐτοῦ, Ἐδώμ.
\vs{31}Εἶπε δὲ Ἰακὼβ τῷ Ἡσαῦ, ἀπόδου μοι σήμερον τὰ πρωτοτόκιά σου ἐμοί.
\vs{32}Καὶ εἶπεν Ἡσαῦ, ἰδοὺ ἐγὼ πορεύομαι τελευτᾷν· καὶ ἵνα τί μοι ταῦτα τὰ πρωτοτόκια;
\vs{33}Καὶ εἶπεν αὐτῷ Ἰακὼβ, ὄμοσόν μοι σήμερον· καὶ ὤμοσεν αὐτῷ· ἀπέδοτο δὲ Ἡσαῦ τὰ πρωτοτόκια τῷ Ἰακώβ.
\vs{34}Ἰακὼβ δὲ ἔδωκε τῷ Ἠσαῦ ἄρτον, καὶ ἕψημα φακοῦ· καὶ ἔφαγε καὶ ἔπιε, καὶ ἀναστὰς ᾤχετο· καὶ ἐφαύλισεν Ἡσαῦ τὰ πρωτοτόκια.

\ch{26}
Ἐγένετο δὲ λιμὸς ἐπὶ τῆς γῆς, χωρὶς τοῦ λιμοῦ τοῦ πρότερον, ὃς ἐγένετο ἐν τῷ καιρῷ τοῦ Ἁβραάμ· ἐπορεύθη δὲ Ἰσαὰκ πρὸς Ἀβιμέλεχ βασιλέα Φυλιστιεὶμ εἰς Γέραρα.
\vs{2}Ὤφθη δὲ αὐτῷ Κύριος, καὶ εἶπε, μὴ καταβῇς εἰς Αἴγυπτον· κατοίκησον δὲ ἐν τῇ γῇ, ᾗ ἄν σοι εἴπω.
\vs{3}Καὶ παροίκει ἐν τῇ γῇ ταύτῃ, καὶ ἔσομαι μετὰ σοῦ, καὶ εὐλογήσω σε· σοὶ γὰρ καὶ τῷ σπέρματί σου δώσω πᾶσαν τὴν γῆν ταύτην· καὶ στήσω τὸν ὅρκον μου, ὅν ὤμοσα τῷ Ἁβραὰμ τῷ πατρί σου.
\vs{4}Καὶ πληθυνῶ τὸ σπέρμα σου, ὡς τοὺς ἀστέρας τοῦ οὐρανοῦ· καὶ δώσω τῷ σπέρματί σου πᾶσαν τὴν γῆν ταύτην· καὶ εὐλογηθήσονται ἐν τῷ σπέρματί σου πάντα τὰ ἔθνη τῆς γῆς.
\vs{5}Ἀνθʼ ὧν ὑπήκουσεν Ἁβραὰμ ὁ πατήρ σου τῆς ἐμῆς φωνῆς, καὶ ἐφύλαξε τὰ προστάγματά μου, καὶ τὰς ἐντολάς μου, καὶ τὰ δικαιώματά μου, καὶ τὰ νόμιμά μου.
\vs{6}Κατῴκησε δὲ Ἰσαὰκ ἐν Γεράροις.
\vs{7}Ἐπηρώτησαν δὲ οἱ ἄνδρες τοῦ τόπου περὶ Ῥεβέκκας τῆς γυναικὸς αὐτοῦ, καὶ εἶπεν, ἀδελφή μου ἐστίν· ἐφοβήθη γὰρ εἰπεῖν, ὅτι γυνή μου ἐστὶ, μή ποτε ἀποκτείνωσιν αὐτὸν οἱ ἄνδρες τοῦ τόπου περὶ Ῥεβέκκας, ὅτι ὡραία τῇ ὄψει ἦν.
\vs{8}Ἐγένετο δὲ πολυχρόνιος ἐκεῖ· καὶ παρακύψας Ἀβιμέλεχ ὁ βασιλεὺς Γεράρων διὰ τῆς θυρίδος, εἶδε τὸν Ἰσαὰκ παίζοντα μετὰ Ῥεβέκκας τῆς γυναικὸς αὐτοῦ.
\vs{9}Ἐκάλεσε δὲ Ἀβιμέλεχ τὸν Ἰσαὰκ, καὶ εἶπεν αὐτῷ, ἆρά γε γυνή σου ἐστί; τί ὅτι εἶπας, ἀδελφή μου ἐστίν; εἶπε δὲ αὐτῷ Ἰσαὰκ, εἶπα γὰρ, μή ποτε ἀποθάνω διʼ αὐτήν.
\vs{10}Εἶπε δὲ αὐτῷ Ἀβιμέλεχ, τί τοῦτο ἐποίησας ἡμῖν; μικροῦ ἐκοιμήθη τις ἐκ τοῦ γένους μου μετὰ τῆς γυναικός σου, καὶ ἐπήγαγες ἂν ἐφʼ ἡμᾶς ἄγνοιαν.
\vs{11}Συνέταξε δὲ Ἀβιμέλεχ παντὶ τῷ λαῷ αὐτοῦ, λέγων, πᾶς ὁ ἁψάμενος τοῦ ἀνθρώπου τούτου καὶ τῆς γυναικὸς αὐτοῦ, θανάτῳ ἔνοχος ἔσται.
\vs{12}Ἔσπειρε δὲ Ἰσαὰκ ἐν τῇ γῇ ἐκείνῃ, καὶ εὗρεν ἐν τῷ ἐνιαυτῷ ἐκείνῳ ἑκατοστεύουσαν κριθήν· εὐλόγησε δὲ αὐτὸν Κύριος.
\vs{13}Καὶ ὑψώθη ὁ ἄνθρωπος, καὶ προβαίνων μείζων ἐγένετο, ἕως οὗ μέγας ἐγένετο σφόδρα.
\vs{14}Ἐγένετο δὲ αὐτῷ κτήνη προβάτων, καὶ κτήνη βοῶν, καὶ γεώργια πολλά. ἐζήλωσαν δὲ αὐτὸν οἱ Φυλιστιείμ.
\vs{15}Καὶ πάντα τὰ φρέατα, ἃ ὤρυξαν οἱ παῖδες τοῦ πατρὸς αὐτοῦ ἐν τῷ χρόνῳ τοῦ πατρὸς αὐτοῦ, ἐνέφραξαν αὐτὰ οἱ Φυλιστιεὶμ, καὶ ἔπλησαν αὐτὰ γῆς.
\vs{16}Εἶπε δὲ Ἀβιμέλεχ πρὸς Ἰσαὰκ, ἄπελθε ἀφʼ ἡμῶν, ὅτι δυνατώτερος ἡμῶν ἐγένου σφόδρα.
\vs{17}Καὶ ἀπῆλθεν ἐκεῖθεν Ἰσαάκ· καὶ κατέλυσεν ἐν τῇ φάραγγι Γεράρων, καὶ κατῴκησεν ἐκεῖ.

\vs{18}Καὶ πάλιν Ἰσαὰκ ὤρυξε τὰ φρέατα τοῦ ὕδατος, ἃ ὤρυξαν οἱ παῖδες Ἁβραὰμ τοῦ πατρὸς αὐτοῦ, καὶ ἐνέφραξαν αὐτὰ οἱ Φυλιστιεὶμ μετὰ τὸ ἀποθανεῖν Ἁβραὰμ τὸν πατέρα αὐτοῦ· καὶ ἐπωνόμασεν αὐτοῖς ὀνόματα κατὰ τὰ ὀνόματα, ἃ ὠνόμασεν ὁ πατὴρ αὐτοῦ.
\vs{19}Καὶ ὤρυξαν οἱ παῖδες Ἰσαὰκ ἐν τῇ φάραγγι Γεράρων· καὶ εὗρον ἐκεῖ φρέαρ ὕδατος ζῶντος.
\vs{20}Καὶ ἐμαχέσαντο οἱ ποιμένες Γεράρων μετὰ τῶν ποιμένων Ἰσαὰκ, φάσκοντες αὐτῶν εἶναι τὸ ὕδωρ· καὶ ἐκάλεσαν τὸ ὄνομα τοῦ φρέατος, Ἀδικία· ἠδίκησαν γὰρ αὐτόν.
\vs{21}Ἀπᾴρας δὲ ἐκεῖθεν ὤρυξε φρέαρ ἕτερον· ἐκρίνοντο δὲ καὶ περὶ ἐκείνου· καὶ ἐπωνόμασε τὸ ὄνομα αὐτοῦ, Ἐχθρία.
\vs{22}Ἀπᾴρας δὲ ἐκεῖθεν ὤρυξε φρέαρ ἕτερον· καὶ οὐκ ἐμαχέσαντο περὶ αὐτοῦ· καὶ ἐπωνόμασε τὸ ὄνομα αὐτοῦ, Εὐρυχωρία, λέγων, διότι νῦν ἐπλάτυνε Κύριος ἡμῖν, καὶ ηὔξησεν ἡμᾶς ἐπὶ τῆς γῆς.

\vs{23}Ἀνέβη δὲ ἐκεῖθεν ἐπὶ τὸ φρέαρ τοῦ ὅρκου.
\vs{24}Καὶ ὤφθη αὐτῷ Κύριος ἐν τῇ νυκτὶ ἐκείνῃ, καὶ εἶπεν, ἐγώ εἰμι ὁ Θεὸς Ἁβραὰμ τοῦ πατρός σου· μὴ φοβοῦ, μετὰ σοῦ γάρ εἰμι, καὶ εὐλογήσω σε, καὶ πληθυνῶ τὸ σπέρμα σου διʼ Ἁβραὰμ τὸν πατέρα σου.
\vs{25}Καὶ ᾠκοδόμησεν ἐκεῖ θυσιαστήριον, καὶ ἐπεκαλέσατο τὸ ὄνομα Κυρίου, καὶ ἔπηξεν ἐκεῖ τὴν σκηνὴν αὐτοῦ· ὤρυξαν δὲ ἐκεῖ οἱ παῖδες Ἰσαὰκ φρέαρ ἐν τῇ φάραγγι Γεράρων.
\vs{26}Καὶ Ἀβιμέλεχ ἐπορεύθη πρὸς αὐτὸν ἀπὸ Γεράρων, καὶ Ὁχοζὰθ ὁ νυμφαγωγὸς αὐτοῦ, καὶ Φιχὼλ ὁ ἀρχιστράτηγος τῆς δυνάμεως αὐτοῦ.
\vs{27}Καὶ εἶπεν αὐτοῖς Ἰσαὰκ, ἵνα τί ἤλθετε πρός με; ὑμεῖς δὲ ἐμισήσατέ με, καὶ ἐξαπεστείλατέ με ἀφʼ ὑμῶν.
\vs{28}Οἱ δὲ εἶπαν, ἰδόντες ἑωράκαμεν ὅτι ἦν Κύριος μετὰ σοῦ· καὶ εἴπαμεν, γενέσθω ἀρὰ ἀνὰ μέσον ἡμῶν καὶ ἀνὰ μέσον σου, καὶ διαθησόμεθα μετὰ σοῦ διαθήκην,
\vs{29}Μὴ ποιήσαι μεθʼ ἡμῶν κακὸν, καθότι οὐκ ἐβδελυξάμεθά σε ἡμεῖς, καὶ ὃν τρόπον ἐχρησάμεθά σοι καλῶς, καὶ ἐξαπεστείλαμέν σε μετʼ εἰρήνης· καὶ νῦν εὐλογημένος σὺ ὑπὸ Κυρίου.
\vs{30}Καὶ ἐποίησεν αὐτοῖς δοχὴν, καὶ ἔφαγον καὶ ἔπιον.
\vs{31}Καὶ ἀναστάντες τὸ πρωῒ, ὤμοσεν ἕκαστος τῷ πλησίον· καὶ ἐξαπέστειλεν αὐτοὺς Ἰσαάκ· καὶ ἀπῴχοντο ἀπʼ αὐτοῦ μετὰ σωτηρίας.
\vs{32}Ἐγένετο δὲ ἐν τῇ ἡμέρᾳ ἐκείνῃ, καὶ παραγενόμενοι οἱ παῖδες Ἰσαὰκ ἀπήγγειλαν αὐτῷ περὶ τοῦ φρέατος οὗ ὤρυξαν, καὶ εἶπαν, οὐχ εὕρομεν ὕδωρ.
\vs{33}Καὶ ἐκάλεσεν αὐτὸ, Ὅρκος· διὰ τοῦτο ἐκάλεσεν ὄνομα τῇ πόλει ἐκείνῃ, Φρέαρ Ὅρκου, ἕως τῆς σήμερον ἡμέρας.

\vs{34}Ἦν δὲ Ἡσαῦ ἐτῶν τεσσαράκοντα, καὶ ἔλαβε γυναῖκα Ἰουδὶθ, θυγατέρα Βεὼχ τοῦ Χετταίου, καὶ τὴν Βασεμὰθ, θυγατέρα Ἑλὼν Χετταίου.
\vs{35}Καὶ ἦσαν ἐρίζουσαι τῷ Ἰσαὰκ καὶ τῇ Ῥεβέκκᾳ.

\ch{27}
Ἐγένετο δὲ μετὰ τὸ γηράσαι τὸν Ἰσαὰκ, καὶ ἠμβλύνθησαν οἱ ὀφθαλμοὶ αὐτοῦ τοῦ ὁρᾷν, καὶ ἐκάλεσεν Ἡσαῦ τὸν υἱὸν αὐτοῦ τὸν πρεσβύτερον, καὶ εἶπεν αὐτῷ, υἱέ μου· καὶ εἶπεν, ἰδοὺ ἐγώ.
\vs{2}Καὶ εἶπεν, ἰδοὺ γεγήρακα, καὶ οὐ γινώσκω τὴν ἡμέραν τῆς τελευτῆς μου.
\vs{3}Νῦν οὖν λάβε τὸ σκεῦός σου, τήν τε φαρέτραν, καὶ τὸ τόξον, καὶ ἔξελθε εἰς τὸ πεδίον, καὶ θήρευσόν μοι θήραν.
\vs{4}Καὶ ποίησόν μοι ἐδέσματα, ὡς φιλῶ ἐγὼ, καὶ ἔνεγκέ μοι, ἵνα φάγω, ὅπως εὐλογήσῃ σε ἡ ψυχή μου πρὶν ἀποθανεῖν με.
\vs{5}Ῥεβέκκα δὲ ἤκουσε λαλοῦντος Ἰσαὰκ πρὸς Ἡσαῦ τὸν υἱὸν αὐτοῦ· ἐπορεύθη δὲ Ἡσαῦ εἰς τὸ πεδίον θηρεῦσαι θήραν τῷ πατρὶ αὐτοῦ.
\vs{6}Ῥεβέκκα δὲ εἶπε πρὸς τὸν Ἰακὼβ τὸν υἱὸν αὐτῆς τὸν ἐλάσσω, ἴδε, ἤκουσα τοῦ πατρός σου λαλοῦντος πρὸς Ἡσαῦ τὸν ἀδελφόν σου, λέγοντος,
\vs{7}Ἔνεγκόν μοι θήραν, καὶ ποίησόν μοι ἐδέσματα, ἵνα φαγὼν εὐλογήσω σε ἐναντίον Κυρίου πρὸ τοῦ ἀποθανεῖν με.
\vs{8}Νῦν οὖν, υἱέ μου, ἄκουσόν μου, καθὰ ἐγώ σοι ἐντέλλομαι.
\vs{9}Καὶ πορευθεὶς εἰς τὰ πρόβατα, λάβε μοι ἐκεῖθεν δύο ἐρίφους ἁπαλοὺς καὶ καλοὺς, καὶ ποιήσω αὐτοὺς ἐδέσματα τῷ πατρί σου, ὡς φιλεῖ.
\vs{10}Καὶ εἰσοίσεις τῷ πατρί σου, καὶ φάγεται, ὅπως εὐλογήσῃ σε ὁ πατήρ σου πρὸ τοῦ ἀποθανεῖν αὐτόν.
\vs{11}Εἶπε δὲ Ἰακὼβ πρὸς Ῥεβέκκαν τὴν μητέρα αὐτοῦ, ἔστιν Ἡσαῦ ὁ ἀδελφός μου ἀνὴρ δασὺς, ἐγὼ δὲ ἀνὴρ λεῖος.
\vs{12}Μή ποτε ψηλαφήσῃ με ὁ πατὴρ, καὶ ἔσομαι ἐναντίον αὐτοῦ ὡς καταφρονῶν, καὶ ἐπάξω ἐπʼ ἐμαυτὸν κατάραν, καὶ οὐκ εὐλογίαν.
\vs{13}Εἶπε δὲ αὐτῷ ἡ μήτηρ, ἐπʼ ἐμὲ ἡ κατάρα σου, τέκνον· μόνον ἐπάκουσόν μου τῆς φωνῆς, καὶ πορευθεὶς ἔνεγκέ μοι.
\vs{14}Πορευθεὶς δὲ ἔλαβε, καὶ ἤνεγκε τῇ μητρί· καὶ ἐποίησεν ἡ μήτηρ αὐτοῦ ἐδέσματα, καθὰ ἐφίλει ὁ πατὴρ αὐτοῦ.

\vs{15}Καὶ λαβοῦσα Ῥεβέκκα τὴν στολὴν Ἡσαῦ τοῦ υἱοῦ αὐτῆς τοῦ πρεσβυτέρου τὴν καλὴν, ἣ ἦν παρʼ αὐτῇ ἐν τῷ οἴκῳ, ἐνέδυσεν αὐτὴν Ἰακὼβ τὸν υἱὸν αὐτῆς τὸν νεώτερον.
\vs{16}Καὶ τὰ δέρματα τῶν ἐρίφων περιέθηκεν ἐπὶ τοὺς βραχίονας αὐτοῦ, καὶ ἐπὶ τὰ γυμνὰ τοῦ τραχήλου αὐτοῦ.
\vs{17}Καὶ ἔδωκε τὰ ἐδέσματα, καὶ τοὺς ἄρτους οὓς ἐποίησεν, εἰς τὰς χεῖρας Ἰακὼβ τοῦ υἱοῦ αὐτῆς.
\vs{18}Καὶ εἰσήνεγκε τῷ πατρὶ αὐτοῦ· εἶπε δὲ, πάτερ· ὁ δὲ εἶπεν, ἰδοὺ ἐγώ· τίς εἶ σὺ, τέκνον;
\vs{19}Καὶ εἶπεν Ἰακὼβ τῷ πατρὶ, ἐγὼ Ἡσαῦ ὁ πρωτότοκός σου πεποίηκα καθὰ ἐλάλησάς μοι· ἀναστὰς κάθισον, καὶ φάγε ἀπὸ τῆς θήρας μου, ὅπως εὐλογήσῃ με ἡ ψυχή σου.
\vs{20}Εἶπε δὲ Ἰσαὰκ τῷ υἱῷ αὐτοῦ, τί τοῦτο, ὃ ταχὺ εὗρες, ὦ τέκνον; ὁ δὲ εἶπεν, ὃ παρέδωκε Κύριος ὁ Θεός σου ἐναντίον μου.
\vs{21}Εἶπε δὲ Ἰσαὰκ τῷ Ἰακὼβ, ἔγγισόν μοι, καὶ ψηλαφήσω σε, τέκνον, εἰ σὺ εἶ ὁ υἱός μου Ἡσαῦ, ἢ οὔ.
\vs{22}Ἤγγισε δὲ Ἰακὼβ πρὸς Ἰσαὰκ τὸν πατέρα αὐτοῦ· καὶ ἐψηλάφησεν αὐτὸν, καὶ εἶπεν, ἡ μὲν φωνὴ, φωνὴ Ἰακὼβ, αἱ δὲ χεῖρες, χεῖρες Ἡσαῦ.
\vs{23}Καὶ οὐκ ἐπέγνω αὐτὸν, ἦσαν γὰρ αἱ χεῖρες αὐτοῦ, ὡς αἱ χεῖρες Ἡσαῦ τοῦ ἀδελφοῦ αὐτοῦ, δασεῖαι· καὶ εὐλόγησεν αὐτὸν,
\vs{24}καὶ εἶπε, σὺ εἶ ὁ υἱός μου Ἡσαῦ; ὁ δὲ εἶπεν, ἐγώ.
\vs{25}Καὶ εἶπε, προσάγαγέ μοι, καὶ φάγομαι ἀπὸ τῆς θήρας σου, τέκνον, ἵνα εὐλογήσῃ σε ἡ ψυχή μου· καὶ προσήνεγκεν αὐτῷ, καὶ ἔφαγε· καὶ εἰσήνεγκεν αὐτῷ οἶνον, καὶ ἔπιε.
\vs{26}Καὶ εἴπεν αὐτῷ Ἰσαὰκ ὁ πατὴρ αὐτοῦ, ἔγγισόν μοι, καὶ φίλησόν με, τέκνον.
\vs{27}Καὶ ἐγγίσας ἐφίλησεν αὐτόν· καὶ ὠσφράνθη τὴν ὀσμὴν τῶν ἱματίων αὐτοῦ, καὶ εὐλόγησεν αὐτὸν, καὶ εἶπεν, ἰδοὺ ὀσμὴ τοῦ υἱοῦ μου, ὡς ὀσμὴ ἀγροῦ πλήρους, ὃν εὐλόγησε Κύριος.
\vs{28}Καὶ δῴη σοι ὁ Θεὸς ἀπὸ τῆς δρόσου τοῦ οὐρανοῦ, καὶ ἀπὸ τῆς πιότητος τῆς γῆς, καὶ πλῆθος σίτου καὶ οἴνου.
\vs{29}Καὶ δουλευσάτωσάν σοι ἔθνη, καὶ προσκυνησάτωσάν σοι ἄρχοντες· καὶ γίνου κύριος τοῦ ἀδελφοῦ σου, καὶ προσκυνήσουσί σοι οἱ υἱοὶ τοῦ πατρός σου· ὁ καταρώμενός σε, ἐπικατάρατος· ὁ δὲ εὐλογῶν σε, εὐλογημένος.

\vs{30}Καὶ ἐγένετο μετὰ τὸ παύσασθαι Ἰσαὰκ εὐλογοῦντα Ἰακὼβ τὸν υἱὸν αὐτοῦ, καὶ ἐγένετο, ὡς ἂν ἐξῆλθεν Ἰακὼβ ἀπὸ προσώπου Ἰσαὰκ τοῦ πατρὸς αὐτοῦ, καὶ Ἡσαῦ ὁ ἀδελφὸς αὐτοῦ ἦλθεν ἀπὸ τῆς θήρας.
\vs{31}Καὶ ἐποίησε καὶ αὐτὸς ἐδέσματα, καὶ προσήνεγκε τῷ πατρὶ αὐτοῦ· καὶ εἶπε τῷ πατρὶ, ἀναστήτω ὁ πατήρ μου, καὶ φαγέτω ἀπὸ τῆς θήρας τοῦ υἱοῦ αὐτοῦ, ὅπως εὐλογήσῃ με ἡ ψυχή σου.
\vs{32}Καὶ εἶπεν αὐτῷ Ἰσαὰκ ὁ πατὴρ αὐτοῦ, τίς εἶ σύ; ὁ δὲ εἶπεν, ἐγώ εἰμι ὁ υἱός σου ὁ πρωτότοκος Ἡσαῦ.
\vs{33}Ἐξέστη δὲ Ἰσαὰκ ἔκστασιν μεγάλην σφόδρα, καὶ εἶπε, τίς οὖν ὁ θηρεύσας μοι θήραν καὶ εἰσενέγκας μοι, καὶ ἔφαγον ἀπὸ πάντων πρὸ τοῦ ἐλθεῖν σε; καὶ εὐλόγησα αὐτὸν, καὶ εὐλογημένος ἔσται.
\vs{34}Ἐγένετο δὲ ἡνίκα ἤκουσεν Ἡσαῦ τὰ ῥήματα τοῦ πατρὸς αὐτοῦ Ἰσαὰκ, ἀνεβόησε φωνὴν μεγάλην καὶ πικρὰν σφόδρα· καὶ εἶπεν, εὐλόγησον δὴ κᾀμὲ, πάτερ.
\vs{35}Εἶπε δὲ αὐτῷ, ἐλθὼν ὁ ἀδελφός σου μετὰ δόλου ἔλαβε τὴν εὐλογίαν σου.
\vs{36}Καὶ εἶπε, δικαίως ἐκλήθη τὸ ὄνομα αὐτοῦ Ἰακὼβ, ἐπτέρνικε γάρ με ἰδοὺ δεύτερον τοῦτο· τά τε πρωτοτόκιά μου εἴληφε, καὶ νῦν ἔλαβε τὴν εὐλογίαν μου· καὶ εἶπεν Ἡσαῦ τῷ πατρὶ αὐτοῦ, οὐχ ὑπελίπου μοι εὐλογίαν, πάτερ;
\vs{37}Ἀποκριθεὶς δὲ Ἰσαὰκ εἶπε τῷ Ἡσαῦ, εἰ κύριον αὐτὸν πεποίηκά σου, καὶ πάντας τοὺς ἀδελφοὺς αὐτοῦ πεποίηκα αὐτοῦ οἰκέτας· σίτῳ καὶ οἴνῳ ἐστήριξα αὐτόν· σοὶ δὲ τί ποιήσω, τέκνον;
\vs{38}Εἶπε δὲ Ἡσαῦ πρὸς τὸν πατέρα αὐτοῦ, μὴ εὐλογία μία σοι ἔστι, πάτερ; εὐλόγησον δὴ κᾀμὲ, πάτερ· κατανυχθέντος δὲ Ἰσαὰκ, ἀνεβόησε φωνῇ Ἡσαῦ, καὶ ἔκλαυσεν.
\vs{39}Ἀποκοιθεὶς δὲ Ἰσαὰκ ὁ πατὴρ αὐτοῦ εἶπεν αὐτῷ, ἰδοὺ ἀπὸ τῆς πιότητος τῆς γῆς ἔσται ἡ κατοίκησίς σου, καὶ ἀπὸ τῆς δρόσου τοῦ οὐρανοῦ ἄνωθεν.
\vs{40}Καὶ ἐπὶ τῇ μαχαίρᾳ σου ζήσῃ, καὶ τῷ ἀδελφῷ σου δουλεύσεις· ἔσται δὲ ἡνίκα ἐὰν καθέλῃς καὶ ἐκλύσῃς τὸν ζυγὸν αὐτοῦ ἀπὸ τοῦ τραχήλου σου.

\vs{41}Καὶ ἐνεκότει Ἡσαῦ τῷ Ἰακὼβ περὶ τῆς εὐλογίας, ἧς εὐλόγησεν αὐτὸν ὁ πατὴρ αὐτοῦ· εἶπε δὲ Ἡσαῦ ἐν τῇ διανοίᾳ αὐτοῦ, ἐγγισάτωσαν αἱ ἡμέραι τοῦ πένθους τοῦ πατρός μου, ἵνα ἀποκτείνω Ἰακὼβ τὸν ἀδελφόν μου.
\vs{42}Ἀπηγγέλη δὲ Ῥεβέκκᾳ τὰ ῥήματα Ἡσαῦ τοῦ υἱοῦ αὐτῆς τοῦ πρεσβυτέρου· καὶ πέμψασα ἐκάλεσεν Ἰακὼβ τὸν υἱὸν αὐτῆς τὸν νεώτερον, καὶ εἶπεν αὐτῷ, ἰδοὺ Ἡσαῦ ὁ ἀδελφός σου ἀπειλεῖ σοι τοῦ ἀποκτεῖναί σε.
\vs{43}Νῦν οὖν, τέκνον, ἄκουσόν μου τῆς φωνῆς, καὶ ἀναστὰς ἀπόδραθι εἰς τὴν Μεσοποταμίαν πρὸς Λάβαν τὸν ἀδελφόν μου εἰς Χαῤῥάν.
\vs{44}Καὶ οἴκησον μετʼ αὐτοῦ ἡμέρας τινὰς, ἕως τοῦ ἀποστρέψαι τὸν θυμὸν,
\vs{45}καὶ τὴν ὀργὴν τοῦ ἀδελφοῦ σου ἀπὸ σοῦ, καὶ ἐπιλάθηται ἃ πεποίηκας αὐτῷ· καὶ ἀποστείλασα μεταπέμψομαί σε ἐκεῖθεν, μή ποτε ἀποτεκνωθῶ ἀπὸ τῶν δύο ὑμῶν ἐν ἡμέρᾳ μιᾷ.
\vs{46}Εἶπε δὲ Ῥεβέκκα πρὸς Ἰσαὰκ, προσώχθικα τῇ ζωῇ μου διὰ τὰς θυγατέρας τῶν υἱῶν Χέτ· εἰ λήψεται Ἰακὼβ γυναῖκα ἀπὸ τῶν θυγατέρων τῆς γῆς ταύτης, ἵνα τί μοι τὸ ζῇν;

\ch{28}
Προσκαλεσάμενος δὲ Ἰσαὰκ τὸν Ἰακὼβ, εὐλόγησεν αὐτὸν, καὶ ἐνετείλατο αὐτῷ, λέγων, οὐ λήψῃ γυναῖκα ἐκ τῶν θυγατέρων τῶν Χαναναίων.
\vs{2}Ἀναστὰς ἀπόδραθι εἰς τὴν Μεσοποταμίαν, εἰς τὸν οἶκον Βαθουὴλ τοῦ πατρὸς τῆς μητρός σου, καὶ λάβε σεαυτῷ ἐκεῖθεν γυναῖκα ἐκ τῶν θυγατέρων Λάβαν τοῦ ἀδελφοῦ τῆς μητρός σου.
\vs{3}Ὁ δὲ Θεός μου εὐλογήσαι σε, καὶ αὐξήσαι σε, καὶ πληθύναι σε· καὶ ἔσῃ εἰς συναγωγὰς ἐθνῶν.
\vs{4}Καὶ δῴη σοι τὴν εὐλόγιαν Ἁβραὰμ τοῦ πατρός μου, σοὶ καὶ τῷ σπέρματί σου μετὰ σὲ, κληρονομῆσαι τὴν γῆν τῆς παροικήσεώς σου, ἣν ἔδωκεν ὁ Θεὸς τῷ Ἁβραάμ.
\vs{5}Καὶ ἀπέστειλεν Ἰσαὰκ τὸν Ἰακώβ· καὶ ἐπορεύθη εἰς τὴν Μεσοποταμίαν πρὸς Λάβαν τὸν υἱὸν Βαθουὴλ τοῦ Σύρου, ἀδελφὸν Ῥεβέκκας τῆς μητρὸς Ἰακὼβ καὶ Ἡσαῦ.

\vs{6}Ἴδε δὲ Ἡσαῦ ὅτι εὐλόγησεν Ἰσαὰκ τὸν Ἰακὼβ, καὶ ἀπέστειλεν εἰς τὴν Μεσοποταμίαν Συρίας, λαβεῖν ἑαυτῷ γυναῖκα ἐκεῖθεν, ἐν τῷ εὐλογεῖν αὐτόν· καὶ ἐνετείλατο αὐτῷ, λέγων, οὐ λήψῃ γυναῖκα ἐκ τῶν θυγατέρων τῶν Χαναναίων.
\vs{7}Καὶ ἤκουσεν Ἰακὼβ τοῦ πατρὸς καὶ τῆς μητρὸς αὐτοῦ· καὶ ἐπορεύθη εἰς τὴν Μεσοποταμίαν Συρίας.
\vs{8}Ἰδὼν δὲ καὶ Ἡσαῦ ὅτι πονηραί εἰσιν αἱ θυγατέρες Χαναὰν ἐναντίον Ἰσαὰκ τοῦ πατρὸς αὐτοῦ,
\vs{9}ἐπορεύθη Ἡσαῦ πρὸς Ἰσμαήλ· καὶ ἔλαβε τὴν Μαελὲθ, θυγατέρα Ἰσμαὴλ τοῦ υἱοῦ Ἁβραὰμ, ἀδελφὴν Ναβεὼθ, πρὸς ταῖς γυναιξὶν αὐτοῦ γυναῖκα.

\vs{10}Καὶ ἐξῆλθεν Ἰακὼβ ἀπὸ τοῦ φρέατος τοῦ ὅρκου, καὶ ἐπορεύθη εἰς Χαῤῥάν.
\vs{11}Καὶ ἀπήντησε τόπῳ, καὶ ἐκοιμήθη ἐκεῖ, ἔδυ γὰρ ὁ ἥλιος· καὶ ἔλαβεν ἀπὸ τῶν λίθων τοῦ τόπου, καὶ ἔθηκε πρὸς κεφαλῆς αὐτοῦ· καὶ ἐκοιμήθη ἐν τῷ τόπῳ ἐκείνῳ.
\vs{12}Καὶ ἐνυπνιάσθη· καὶ ἰδοὺ κλίμαξ ἐστηριγμένη ἐν τῇ γῇ, ἧς ἡ κεφαλὴ ἀφικνεῖτο εἰς τὸν οὐρανόν· καὶ οἱ ἄγγελοι τοῦ θεοῦ ἀνέβαινον καὶ κατέβαινον ἐπʼ αὐτῇ.
\vs{13}Ὁ δὲ Κύριος ἐπεστήρικτο ἐπʼ αὐτῆς· καὶ εἶπεν, ἐγώ εἰμι ὁ Θεὸς Ἁβραὰμ τοῦ πατρός σου, καὶ ὁ Θεὸς Ἰσαάκ· μὴ φοβοῦ· ἡ γῆ ἐφʼ ἧς σὺ καθεύδεις ἐπʼ αὐτῆς, σοὶ δώσω αὐτὴν, καὶ τῷ σπέρματί σου.
\vs{14}Καὶ ἔσται τὸ σπέρμα σου ὡς ἡ ἄμμος τῆς γῆς, καὶ πλατυνθήσεται ἐπὶ θάλασσαν, καὶ Λίβα, καὶ Βοῤῥὰν, καὶ ἐπὶ ἀνατολάς· καὶ ἐνευλογηθήσονται ἐν σοὶ πᾶσαι αἱ φυλαὶ τῆς γῆς, καὶ ἐν τῷ σπέρματί σου.
\vs{15}Καὶ ἰδοὺ ἐγώ εἰμι μετὰ σοῦ, διαφυλάσσων σε ἐν τῇ ὁδῷ πάσῃ, οὗ ἂν πορευθῇς· καὶ ἀποστρέψω σε εἰς τὴν γῆν ταύτην· ὅτι οὐ μή σε ἐγκαταλίπω, ἕως τοῦ ποιῆσαί με πάντα ὅσα ἐλάλησά σοι.
\vs{16}Καὶ ἐξηγέρθη Ἰακὼβ ἐκ τοῦ ὕπνου αὐτοῦ, καὶ εἶπεν, ὅτι ἔστι Κύριος ἐν τῷ τόπῳ τούτῳ, ἐγὼ δὲ οὐκ ᾔδειν.
\vs{17}Καὶ ἐφοβήθη, καὶ εἶπεν, ὡς φοβερὸς ὁ τόπος οὗτος· οὐκ ἔστι τοῦτο ἀλλʼ ἢ οἶκος Θεοῦ, καὶ αὕτη ἡ πύλη τοῦ οὐρανοῦ.
\vs{18}Καὶ ἀνέστη Ἰακὼβ τὸ πρωῒ, καὶ ἔλαβε τὸν λίθον, ὃν ὑπέθηκεν ἐκεῖ πρὸς κεφαλῆς αὐτοῦ, καὶ ἔστησεν αὐτὸν στήλην, καὶ ἐπέχεεν ἔλαιον ἐπὶ τὸ ἄκρον αὐτῆς.
\vs{19}Καὶ ἐκάλεσε τὸ ὄνομα τοῦ τόπου ἐκείνου, οἶκος Θεοῦ· καὶ Οὐλαμλοὺζ ἦν ὄνομα τῇ πόλει τὸ πρότερον.
\vs{20}Καὶ ηὔξατο Ἰακὼβ εὐχὴν, λέγων, ἐὰν ᾖ Κύριος ὁ Θεὸς μετʼ ἐμοῦ, καὶ διαφυλάξῃ με ἐν τῇ ὁδῷ ταύτῃ, ᾗ ἐγὼ πορεύομαι, καὶ δῷ μοι ἄρτον φαγεῖν, καὶ ἱμάτιον περιβαλέσθαι,
\vs{21}καὶ ἀποστρέψῃ με μετὰ σωτηρίας εἰς τὸν οἶκον τοῦ πατρός μου, καὶ ἔσται Κύριός μοι εἰς Θεόν.
\vs{22}Καὶ ὁ λίθος οὗτος, ὃν ἔστησα στήλην, ἔσται μοι οἶκος Θεοῦ· καὶ πάντων ὧν ἐάν μοι δῷς, δεκάτην ἀποδεκατώσω αὐτά σοι.

\ch{29}
Καὶ ἐξᾴρας Ἰακὼβ τοὺς πόδας ἐπορεύθη εἰς γῆν ἀνατολῶν, πρὸς Λάβαν τὸν υἱὸν Βαθουὴλ τοῦ Σύρου, ἀδελφὸν δὲ Ῥεβέκκας, μητρὸς Ἰακὼβ καὶ Ἡσαῦ.
\vs{2}Καὶ ὁρᾷ, καὶ ἰδοὺ φρέαρ ἐν τῷ πεδίῳ· ἦσαν δὲ ἐκεῖ τρία ποίμνια προβάτων ἀναπαυόμενα ἐπʼ αὐτοῦ· ἐκ γὰρ τοῦ φρέατος ἐκείνου ἐπότιζον τὰ ποίμνια· λίθος δὲ ἦν μέγας ἐπὶ τῷ στόματι τοῦ φρέατος.
\vs{3}Καὶ συνήγοντο ἐκεῖ πάντα τὰ ποίμνια· καὶ ἀπεκύλιον τὸν λίθον ἀπὸ τοῦ στόματος τοῦ φρέατος, καὶ ἐπότιζον τὰ πρόβατα, καὶ ἀπεκαθίστων τὸν λίθον ἐπὶ τὸ στόμα τοῦ φρέατος εἰς τὸν τόπον αὐτοῦ.
\vs{4}Εἶπε δὲ αὐτοῖς Ἰακὼβ, ἀδελφοὶ, πόθεν ἐστὲ ὑμεῖς; οἱ δὲ εἶπαν, ἐκ Χαῤῥὰν ἐσμέν.
\vs{5}Εἶπε δὲ αὐτοῖς, γινώσκετε Λάβαν τὸν υἱὸν Ναχώρ; οἱ δὲ εἶπαν, γινώσκομεν·
\vs{6}Εἶπε δὲ αὐτοῖς, ὑγιαίνει; οἱ δὲ εἶπαν, ὑγιαίνει· καὶ ἰδοὺ Ῥαχὴλ ἡ θυγάτηρ αὐτοῦ ἤρχετο μετὰ τῶν προβάτων.
\vs{7}Καὶ εἶπεν Ἰακὼβ, ἔτι ἐστὶν ἡμέρα πολλὴ· οὔπω ὥρα συναχθῆναι τὰ κτήνη· ποτίσαντες τὰ πρόβατα, ἀπελθόντες βόσκετε.
\vs{8}Οἱ δὲ εἶπαν, οὐ δυνησόμεθα, ἕως τοῦ συναχθῆναι πάντας τοὺς ποιμένας, καὶ ἀποκυλίσουσι τὸν λίθον ἀπὸ τοῦ στόματος τοῦ φρέατος, καὶ ποτιοῦμεν τὰ πρόβατα.
\vs{9}Ἔτι αὐτοῦ λαλοῦντος αὐτοῖς, καὶ ἰδοὺ Ῥαχὴλ ἡ θυγάτηρ Λάβαν ἤρχετο μετὰ τῶν προβάτων τοῦ πατρὸς αὐτῆς· αὐτὴ γὰρ ἔβοσκε τὰ πρόβατα τοῦ πατρὸς αὐτῆς.
\vs{10}Ἐγένετο δὲ ὡς εἶδεν Ἰακὼβ τὴν Ῥαχὴλ τὴν θυγατέρα Λάβαν, τοῦ ἀδελφοῦ τῆς μητρὸς αὐτοῦ, καὶ τὰ πρόβατα Λάβαν τοῦ ἀδελφοῦ τῆς μητρὸς αὐτοῦ, καὶ προσελθὼν Ἰακὼβ ἀπεκύλισε τὸν λίθον ἀπὸ τοῦ στόματος τοῦ φρέατος, καὶ ἐπότιζε τὰ πρόβατα Λάβαν τοῦ ἀδελφοῦ τῆς μητρὸς αὐτοῦ.
\vs{11}Καὶ ἐφίλησεν Ἰακὼβ τὴν Ῥαχὴλ, καὶ βοήσας τῇ φωνῇ αὐτοῦ ἔκλαυσε.
\vs{12}Καὶ ἀπήγγειλε τῇ Ῥαχὴλ, ὅτι ἀδελφὸς τοῦ πατρὸς αὐτῆς ἐστι, καὶ ὅτι υἱὸς Ῥεβέκκας ἐστί· καὶ δραμοῦσα ἀπήγγειλε τῷ πατρὶ αὐτῆς κατὰ τὰ ῥήματα ταῦτα.
\vs{13}Ἐγένετο δὲ ὡς ἤκουσε Λάβαν τὸ ὄνομα Ἰακὼβ τοῦ υἱοῦ τῆς ἀδελφῆς αὐτοῦ, ἔδραμεν εἰς συνάντησιν αὐτῷ, καὶ περιλαβὼν αὐτὸν ἐφίλησε, καὶ εἰσήγαγεν αὐτὸν εἰς τὸν οἶκον αὐτοῦ· καὶ διηγήσατο τῷ Λάβαν πάντας τοὺς λόγους τούτους.
\vs{14}Καὶ εἶπεν αὐτῷ Λάβαν, ἐκ τῶν ὀστῶν μου καὶ ἐκ τῆς σαρκός μου εἶ σύ· καὶ ἦν μετʼ αὐτοῦ μῆνα ἡμερῶν.

\vs{15}Εἶπε δὲ Λάβαν τῷ Ἰακὼβ, ὅτι γὰρ ἀδελφός μου εἶ, οὐ δουλεύσεις μοι δωρεάν· ἀπάγγειλόν μοι τίς ὁ μισθός σου ἐστί;
\vs{16}Τῷ δὲ Λάβαν ἦσαν δύο θυγατέρες· ὄνομα τῇ μείζονι, Λεία, καὶ ὄνομα τῇ νεωτέρᾳ, Ῥαχήλ.
\vs{17}Οἱ δὲ ὀφθάλμοὶ Λείας, ἀσθενεῖς· Ῥαχῆλ δὲ ἦν καλὴ τῷ εἴδει, καὶ ὡραία τῇ ὄψει σφάδρα.
\vs{18}Ἠγάπησε δὲ Ἰακὼβ τὴν Ῥαχήλ· καὶ εἶπε, δουλεύσω σοι ἑπτὰ ἔτη περὶ τῆς Ῥαχὴλ τῆς θυγατρός σου τῆς νεωτέρας.
\vs{19}Εἶπε δὲ αὐτῷ Λάβαν, βέλτιον δοῦναί με αὐτήν σοι, ἢ δοῦναί με αὐτὴν ἀνδρὶ ἑτέρῳ· οἴκησον μετʼ ἐμοῦ.
\vs{20}Καὶ ἐδούλευσεν Ἰακὼβ περὶ Ῥαχὴλ ἑπτὰ ἔτη· καὶ ἤσαν ἐναντίον αὐτοῦ ὡς ἡμέραι ὀλίγαι, παρὰ τὸ ἀγαπᾷν αὐτὸν αὐτήν.
\vs{21}Εἶπε δὲ Ἰακὼβ τῷ Λάβαν, δός μοι τὴν γυναῖκά μου, πεπλήρωνται γὰρ αἱ ἡμέραι ὅπως εἰσέλθω πρὸς αὐτήν.
\vs{22}Συνήγαγε δὲ Λάβαν πάντας τοὺς ἄνδρας τοῦ τόπου, καὶ ἐποίησε γάμον.
\vs{23}Καὶ ἐγένετο ἑσπέρα, καὶ λαβὼν Λείαν τὴν θυγατέρα αὐτοῦ, εἰσήγαγεν πρὸς Ἰακὼβ, καὶ εἰσῆλθε πρὸς αὐτὴν Ἰακώβ.
\vs{24}Ἔδωκε δὲ Λάβαν Λείᾳ τῇ θυγατρὶ αὐτοῦ Ζελφὰν τὴν παιδίσκην αὐτοῦ, αὐτῇ παιδίσκην.
\vs{25}Ἐγένετο δὲ πρωῒ, καὶ ἰδοὺ ἦν Λεία· εἶπε δὲ Ἰακὼβ τῷ Λάβαν, τί τοῦτο ἐποίησάς μοι; οὐ περὶ Ῥαχὴλ ἐδούλευσα παρὰ σοι; καὶ ἱνατί παρελογίσω με;
\vs{26}Ἀπεκρίθη δὲ Λάβαν, οὐκ ἔστιν οὕτως ἐν τῷ τόπῳ ἡμῶν, δοῦναι τὴν νεωτέραν πρινὴ τὴν πρεσβυτέραν.
\vs{27}Συντέλεσον οὖν τὰ ἕβδομα ταύτης, καὶ δώσω σοι καὶ ταύτην ἀντὶ τῆς ἐργασίας, ἧς ἐργᾷ παρʼ ἐμοὶ ἔτι ἑπτὰ ἔτη ἕτερα.
\vs{28}Ἐποίησε δὲ Ἰακὼβ οὕτως, καὶ ἀνεπλήρωσε τὰ ἕβδομα ταύτης· καὶ ἔδωκεν αὐτῷ Λάβαν Ῥαχὴλ τὴν θυγατέρα αὐτοῦ αὐτῷ γυναῖκα.
\vs{29}Ἔδωκε δὲ Λάβαν τῇ θυγατρὶ αὐτοῦ Βαλλὰν τὴν παιδίσκην αὐτοῦ, αὐτῇ παιδίσκην.
\vs{30}Καὶ εἰσῆλθε πρὸς Ῥαχήλ· ἠγάπησε δὲ Ῥαχὴλ μᾶλλον ἢ Λείαν· καὶ ἐδούλευσεν αὐτῷ ἑπτὰ ἔτη ἕτερα.

\vs{31}Ἰδὼν δὲ Κύριος ὁ Θεὸς ὅτι ἐμισεῖτο Λεία, ἤνοιξε τὴν μήτραν αὐτῆς· Ῥαχὴλ δὲ ἦν στεῖρα.
\vs{32}Καὶ συνέλαβε Λεία, καὶ ἔτεκεν υἱὸν τῷ Ἰακώβ· ἐκάλεσε δὲ τὸ ὄνομα αὐτοῦ Ῥουβὴν, λέγουσα, διότι εἶδέ μου Κύριος τὴν ταπείνωσιν, καὶ ἔδωκέ μοι υἱόν· νῦν οὖν ἀγαπήσει με ὁ ἀνήρ μου.
\vs{33}Καὶ συνέλαβε πάλιν, καὶ ἔτεκεν υἱὸν δεύτερον τῷ Ἰακὼβ, καὶ εἶπεν, ὅτι ἤκουσε Κύριος ὅτι μισοῦμαι, καὶ προσέδωκέ μοι καὶ τοῦτον· καὶ ἐκάλεσε τὸ ὄνομα αὐτοῦ, Συμεών.
\vs{34}Καὶ συνέλαβεν ἔτι, καὶ ἔτεκεν υἱὸν, καὶ εἶπεν, ἐν τῷ νῦν καιρῷ πρὸς ἐμοῦ ἔσται ὁ ἀνήρ μου, τέτοκα γὰρ αὐτῷ τρεῖς υἱούς· διὰ τοῦτο ἐκάλεσε τὸ ὄνομα αὐτοῦ, Λευεί.
\vs{35}Καὶ συλλαβοῦσα ἔτι ἔτεκεν υἱὸν, καὶ εἶπε, νῦν ἔτι τοῦτο ἐξομολογήσομαι Κυρίῳ· διὰ τοῦτο ἐκάλεσε τὸ ὄνομα αὐτοῦ, Ἰούδαν· καὶ ἔστη τοῦ τίκτειν.

\ch{30}
Ἰδοῦσα δὲ Ῥαχὴλ, ὅτι οὐ τέτοκε τῷ Ἱακώβ· καὶ ἐζήλωσε Ῥαχὴλ τὴν ἀδελφὴν αὐτῆς· καὶ εἶπε τῷ Ἰακὼβ, δός μοι τέκνα· εἰ δὲ μὴ, τελευτήσω ἐγώ.
\vs{2}Θυμωθεὶς δὲ Ἰακὼβ τῇ Ῥαχὴλ εἶπεν αὐτῇ, μὴ ἀντὶ Θεοῦ ἐγώ εἰμι, ὃς ἐστέρησέ σε καρπὸν κοιλίας;
\vs{3}Εἶπε δὲ Ῥαχὴλ τῷ Ἰακὼβ, ἰδοὺ ἡ παιδίσκη μου Βαλλά· εἴσελθε πρὸς αὐτήν· καὶ τέξεται ἐπὶ τῶν γονάτων μου, καὶ τεκνοποιήσομαι κᾀγὼ ἐξ αὐτῆς.
\vs{4}Καὶ ἔδωκεν αὐτῷ Βαλλὰν τὴν παιδίσκην αὐτῆς, αὐτῷ γυναῖκα· καὶ εἰσῆλθε πρὸς αὐτὴν Ἰακώβ.
\vs{5}Καὶ συνέλαβε Βαλλὰ ἡ παιδίσκη Ῥαχὴλ, καὶ ἔτεκε τῷ Ἰακὼβ υἱόν.
\vs{6}Καὶ εἶπε Ῥαχὴλ, ἔκρινέ μοι ὁ Θεὸς, καὶ ἐπήκουσε τῆς φωνῆς μου, καὶ ἔδωκε μοι υἱόν· διὰ τοῦτο ἐκάλεσε τὸ ὄνομα αὐτοῦ, Δάν.
\vs{7}Καὶ συνέλαβεν ἔτι Βαλλὰ ἡ παιδίσκη Ῥαχὴλ, καὶ ἔτεκεν υἱὸν δεύτερον τῷ Ἰακώβ.
\vs{8}Καὶ εἶπε Ῥαχὴλ, συναντελάβετό μου ὁ Θεὸς, καὶ συνανεστράφην τῇ ἀδελφῇ μου, καὶ ἠδυνάσθην· καὶ ἐκάλεσε τὸ ὄνομα αὐτοῦ, Νεφθαλεί.
\vs{9}Εἶδε δὲ Λεία ὅτι ἔστη τοῦ τίκτειν· καὶ ἔλαβε Ζελφὰν τὴν παιδίσκην αὐτῆς, καὶ ἔδωκεν αὐτὴν τῷ Ἰακὼβ γυναῖκα· καὶ εἰσῆλθε πρὸς αὐτήν.
\vs{10}Καὶ συνέλαβε Ζελφὰ ἡ παιδίσκη Λείας, καὶ ἔτεκε τῷ Ἰακὼβ υἱόν.
\vs{11}Καὶ εἶπε Λεία, ἐν τύχῃ· καὶ ἐπωνόμασε τὸ ὄνομα αὐτοῦ, Γάδ.
\vs{12}Καὶ συνέλαβεν ἔτι Ζελφὰ ἡ παιδίσκη Λείας, καὶ ἔτεκε τῷ Ἰακὼβ υἱὸν δεύτερον.
\vs{13}Καὶ εἶπε Λεία, μακαρία ἐγὼ, ὅτι μακαριοῦσί με αἱ γυναῖκες· καὶ ἐκάλεσε τὸ ὄνομα αὐτοῦ, Ἀσήρ.
\vs{14}Ἐπορεύθη δὲ Ῥουβὴν ἐν ἡμέρᾳ θερισμοῦ πυρῶν, καὶ εὗρε μῆλα μανδραγορῶν ἐν τῷ ἀγρῷ, καὶ ἤνεγκεν αὐτὰ πρὸς Λείαν τὴν μητέρα αὐτοῦ· εἶπε δὲ Ῥαχὴλ τῇ Λείᾳ τῇ ἀδελφῇ αὐτῆς, δός μοι τῶν μανδραγορῶν τοῦ υἱοῦ σου.
\vs{15}Εἶπε δὲ Λεία, οὐχ ἱκανόν σοι ὅτι ἔλαβες τὸν ἄνδρα μου; μὴ καὶ τοὺς μανδραγόρας τοῦ υἱοῦ μου λήψῃ; εἶπε δὲ Ῥαχὴλ, οὐχ οὕτως· κοιμηθήτω μετὰ σοῦ τὴν νύκτα ταύτην ἀντὶ τῶν μανδραγορῶν τοῦ υἱοῦ σου.
\vs{16}Εἰσῆλθεν δὲ Ἰακὼβ ἐξ ἀγροῦ ἑσπέρας· καὶ ἐξῆλθε Λεία εἰς συνάντησιν αὐτῷ, καὶ εἶπε, πρὸς ἐμὲ εἰσελεύσῃ σήμερον· μεμίσθωμαι γάρ σε ἀντὶ τῶν μανδραγορῶν τοῦ υἱοῦ μου· καὶ ἐκοιμήθη μετʼ αὐτῆς τὴν νύκτα ἐκείνην.
\vs{17}Καὶ ἐπήκουσεν ὁ Θεὸς Λείας· καὶ συλλαβοῦσα ἔτεκε τῷ Ἰακὼβ υἱὸν πέμπτον.
\vs{18}Καὶ εἶπε Λεία, δέδωκέ μοι ὁ Θεὸς τὸν μισθόν μου, ἀνθʼ οὗ ἔδωκα τὴν παιδίσκην μου τῷ ἀνδρί μου· καὶ ἐκάλεσε τὸ ὄνομα αὐτοῦ, Ἰσσάχαρ, ὅ ἐστι μισθός.
\vs{19}Καὶ συνέλαβεν ἔτι Λεία, καὶ ἔτεκεν υἱὸν ἕκτον τῷ Ἰακώβ.
\vs{20}Καὶ εἶπε Λεία, δεδώρηται ὁ Θεός μοι δῶρον καλὸν ἐν τῷ νῦν καιρῷ· αἱρετιεῖ με ὁ ἀνήρ μου, τέτοκα γὰρ αὐτῷ υἱοὺς ἕξ· καὶ ἐκάλεσε τὸ ὄνομα αὐτοῦ, Ζαβουλών.
\vs{21}Καὶ μετὰ τοῦτο ἔτεκε θυγατέρα, καὶ ἐκάλεσε τὸ ὄνομα αὐτῆς, Δεῖνα.
\vs{22}Ἐμνήσθη δὲ ὁ Θεὸς τῆς Ῥαχὴλ, καὶ ἔπήκουσεν αὐτῆς ὁ Θεός· καὶ ἀνέῳξεν αὐτῆς τὴν μήτραν.
\vs{23}Καὶ συλλαβοῦσα ἔτεκε τῷ Ἰακὼβ υἱόν· εἶπε δὲ Ῥαχὴλ, ἀφεῖλεν ὁ Θεός μου τὸ ὄνειδος.
\vs{24}Καὶ ἐκάλεσε τὸ ὄνομα αὐτοῦ Ἰωσὴφ, λέγουσα, προσθέτω ὁ Θεός μοι υἱὸν ἕτερον.

\vs{25}Ἐγένετο δὲ ὡς ἔτεκε Ῥαχὴλ τὸν Ἰωσὴφ, εἶπεν Ἰακὼβ τῷ Λάβαν, ἀπόστειλόν με, ἵνα ἀπέλθω εἰς τὸν τόπον μου, καὶ εἰς τὴν γῆν μου.
\vs{26}Ἀπόδος τὰς γυναῖκας μου, καὶ τὰ παιδία μου, περὶ ὧν δεδούλευκά σοι, ἵνα ἀπέλθω· σὺ γὰρ γινώσκεις τὴν δουλείαν, ἣν δεδούλευκά σοι.
\vs{27}Εἶπε δὲ αὐτῷ Λάβαν, εἰ εὗρον χάριν ἐναντίον σου, οἰωνισάμην ἄν· εὐλόγησε γάρ με ὁ Θεὸς ἐπὶ τῇ σῇ εἰσόδῳ.
\vs{28}Διάστειλον τὸν μισθόν σου πρός με, καὶ δώσω.
\vs{29}Εἶπε δὲ Ἰακὼβ, σὺ γινώσκεις ἃ δεδούλευκά σοι, καὶ ὅσα ἦν κτήνη σου μετʼ ἐμοῦ.
\vs{30}Μικρὰ γὰρ ἦν ὅσα σοι ἐναντίον ἐμοῦ, καὶ ηὐξήθη εἰς πλῆθος· καὶ εὐλόγησέ σε Κύριος ὁ Θεὸς ἐπὶ τῷ ποδί μου· νῦν οὖν πότε ποιήσω κᾀγὼ ἐμαυτῷ οἶκον;
\vs{31}Καὶ εἶπεν αὐτῷ Λάβαν, τί σοι δώσω; Εἶπε δὲ αὐτῷ Ἰακὼβ, οὐ δώσεις μοι οὐθὲν, ἐὰν ποιήσῃς μοι τὸ ῥῆμα τοῦτο, πάλιν ποιμανῶ τὰ πρόβατά σου, καὶ φυλάξω.
\vs{32}Παρελθέτω πάντα τὰ πρόβατά σου σήμερον, καὶ διαχώρισον ἐκεῖθεν πᾶν πρόβατον φαιὸν ἐν τοῖς ἄρνασι, καὶ πᾶν διάλευκον καὶ ῥαντὸν ἐν ταῖς αἰξὶν, ἔσται μοι μισθός.
\vs{33}Καὶ ἐπακούσεταί μοι ἡ δικαιοσύνη μου ἐν τῇ ἡμέρᾳ τῇ ἐπαύριον, ὅτι ἐστὶν ὁ μισθός μου ἐνώπιόν σου· πᾶν ὃ ἐὰν μὴ ᾖ ῥαντὸν καὶ διάλευκον ἐν ταῖς αἰξὶ, καὶ φαιὸν ἐν τοῖς ἄρνασι, κεκλεμμένον ἔσται παρʼ ἐμοί.
\vs{34}Εἶπε δὲ αὐτῷ Λάβαν, ἔστω κατὰ τὸ ῥῆμά σου.
\vs{35}Καὶ διέστειλεν ἐν τῇ ἡμέρᾳ ἐκείνῃ τοὺς τράγους τοὺς ῥαντοὺς καὶ τοὺς διαλεύκους, καὶ πάσας τὰς αἶγας τὰς ῥαντὰς καὶ τὰς διαλεύκους, καὶ πᾶν ὃ ἦν φαιὸν ἐν τοῖς ἄρνασι, καὶ πᾶν ὃ ἦν λευκὸν ἐν αὐτοῖς, καὶ ἔδωκε διὰ χειρὸς τῶν υἱῶν αὐτοῦ.
\vs{36}Καὶ ἀπέστησεν ὁδὸν τριῶν ἡμερῶν, καὶ ἀνὰ μέσον αὐτῶν καὶ ἀνὰ μέσον Ἰακώβ· Ἰακὼβ δὲ ἐποίμαινε τὰ πρόβατα Λάβαν τὰ ὑπολειφθέντα.
\vs{37}Ἔλαβε δὲ ἑαυτῷ Ἰακὼβ ῥάβδον στυρακίνην χλωρὰν καὶ καρυΐνην καὶ πλατάνου· καὶ ἐλέπισεν αὐτὰς Ἰακὼβ λεπίσματα λευκά· καὶ περισύρων τὸ χλωρὸν, ἐφαίνετο ἐπὶ ταῖς ῥάβδοις τὸ λευκὸν, ὃ ἐλέπισε, ποικίλον.
\vs{38}Καὶ παρέθηκε τὰς ῥάβδους, ἃς ἐλέπισεν, ἐν τοῖς ληνοῖς τῶν ποτιστηρίων τοῦ ὕδατος, ἵνα ὡς ἂν ἔλθωσι τὰ πρόβατα πιεῖν, ἐνώπιον τῶν ῥάβδων ἐλθόντων αὐτῶν εἰς τὸ πιεῖν, ἐγκισσήσωσι τὰ πρόβατα εἰς τὰς ῥάβδους.
\vs{39}Καὶ ἐνεκίσσων τὰ πρόβατα εἰς τὰς ῥάβδους· καὶ ἔτικτον τὰ πρόβατα διάλευκα καὶ ποικίλα καὶ σποδοειδῆ ῥαντά.
\vs{40}Τοὺς δὲ ἀμνοὺς διέστειλεν Ἰακὼβ, καὶ ἔστησεν ἐναντίον τῶν προβάτων κριὸν διάλευκον, καὶ πᾶν ποικίλον ἐν τοῖς ἀμνοῖς· καὶ διεχώρισεν ἑαυτῷ ποίμνια καθʼ ἑαυτὸν, καὶ οὐκ ἔμιξεν αὐτὰ εἰς τὰ πρόβατα Λάβαν.
\vs{41}Ἐγένετο δὲ ἐν τῷ καιρῷ ᾧ ἐνεκίσσων τὰ πρόβατα ἐν γαστρὶ λαμβάνοντα, ἔθηκεν Ἰακὼβ τὰς ῥάβδους ἐναντίον τῶν προβάτων ἐν τοῖς ληνοῖς, τοῦ ἐγκισσῆσαι αὐτὰ κατὰ τὰς ῥάβδους.
\vs{42}Ἡνίκα δʼ ἂν ἔτεκε τὰ πρόβατα, οὐκ ἐτίθει· ἐγένετο δὲ τὰ μὲν ἄσημα τοῦ Λάβαν, τὰ δὲ ἐπίσημα τοῦ Ἰακώβ.
\vs{43}Καὶ ἐπλούτησεν ὁ ἄνθρωπος σφόδρα σφόδρα· καὶ ἐγένετο αὐτῷ κτήνη πολλὰ, καὶ βόες, καὶ παῖδες, καὶ παιδίσκαι, καὶ κάμηλοι, καὶ ὄνοι.

\ch{31}
Ἤκουσε δὲ Ἰακὼβ τὰ ῥήματα τῶν υἱῶν Λάβαν, λεγόντων, εἴληφεν Ἰακὼβ πάντα τὰ τοῦ πατρὸς ἡμῶν, καὶ ἐκ τῶν τοῦ πατρὸς ἡμῶν πεποίηκε πᾶσαν τὴν δόξαν ταύτην.
\vs{2}Καὶ εἶδεν Ἰακὼβ τὸ πρόσωπον τοῦ Λάβαν, καὶ ἰδοὺ οὐκ ἦν πρὸς αὐτὸν ὡσεὶ χθὲς καὶ τρίτην ἡμέραν.
\vs{3}Εἶπε δὲ Κύριος πρὸς Ἰακὼβ, ἀποστρέφου εἰς τὴν γῆν τοῦ πατρός σου, καὶ εἰς τὴν γενεάν σου, καὶ ἔσομαι μετὰ σοῦ.
\vs{4}Ἀποστείλας δὲ Ἰακὼβ ἐκάλεσε Λείαν καὶ Ῥαχὴλ εἰς τὸ πεδίον, οὗ ἦν τὰ ποίμνια.
\vs{5}Καὶ εἶπεν αὐταῖς, ὁρῶ ἐγὼ τὸ πρόσωπον τοῦ πατρὸς ὑμῶν, ὅτι οὐκ ἔστι πρὸς ἐμοῦ, ὡς ἐχθὲς καὶ τρίτην ἡμέραν· ὁ δὲ Θεὸς τοῦ πατρός μου ἦν μετʼ ἐμοῦ.
\vs{6}Καὶ αὐταὶ δὲ οἴδατε, ὅτι ἐν πάσῃ τῇ ἰσχύϊ μου δεδούλευκα τῷ πατρὶ ὑμῶν.
\vs{7}Ὁ δὲ πατὴρ ὑμῶν παρεκρούσατό με, καὶ ἤλλαξε τὸν μισθόν μου τῶν δέκα ἀμνῶν· καὶ οὐκ ἔδωκεν αὐτῷ ὁ Θεὸς κακοποιῆσαί με.
\vs{8}Ἐὰν οὕτως εἴπῃ, τὰ ποικίλα ἔσται σου μισθὸς, καὶ τέξεται πάντα τὰ πρόβατα ποικίλα· ἐὰν δὲ εἴπῃ, τὰ λευκὰ ἔσται σου μισθὸς, καὶ τέξεται πάντα τὰ πρόβατα λευκά.
\vs{9}Καὶ ἀφείλετο ὁ Θεὸς πάντα τὰ κτήνη τοῦ πατρὸς ὑμῶν, καὶ ἔδωκέ μοι αὐτά.
\vs{10}Καὶ ἐγένετο ἡνίκα ἐνεκίσσων τὰ πρόβατα ἐν γαστρὶ λαμβάνοντα, καὶ εἶδον τοῖς ὀφθαλμοῖς μου ἐν τῷ ὕπνῳ· καὶ ἰδοὺ οἱ τράγοι καὶ οἱ κριοὶ ἀναβαίνοντες ἐπὶ τὰ πρόβατα καὶ τὰς αἶγας, διάλευκοι καὶ ποικίλοι καὶ σποδοειδεῖς ῥαντοί.
\vs{11}Καὶ εἶπέ μοι ὁ Ἄγγελος τοῦ Θεοῦ καθʼ ὕπνον, Ἰακώβ· ἐγὼ δὲ εἶπα, τί ἐστι;
\vs{12}Καὶ εἶπεν, ἀνάβλεψον τοῖς ὀφθαλμοῖς σου, καὶ ἴδε τοὺς τράγους καὶ τοὺς κριοὺς ἀναβαίνοντας ἐπὶ τὰ πρόβατα καὶ τὰς αἶγας διαλεύκους καὶ ποικίλους καὶ σποδοειδεῖς ῥαντούς· ἑώρακα γὰρ ὅσα σοι Λάβαν ποιεῖ.
\vs{13}Ἐγώ εἰμι ὁ Θεὸς ὁ ὀφθείς σοι ἐν τόπῳ Θεοῦ, οὗ ἤλειψάς μοι ἐκεῖ στήλην, καὶ ηὔξω μοι ἐκεῖ εὐχήν· νῦν οὖν ἀνάστηθι, καὶ ἔξελθε ἐκ τῆς γῆς ταύτης, καὶ ἄπελθε εἰς τὴν γῆν τῆς γενέσεώς σου, καὶ ἔσομαι μετὰ σοῦ.
\vs{14}Καὶ ἀποκριθεῖσαι Ῥαχὴλ καὶ Λεία εἶπαν αὐτῷ, μὴ ἔστιν ἡμῖν ἔτι μερὶς ἢ κληρονομία ἐν τῷ οἴκῳ τοῦ πατρὸς ἡμῶν;
\vs{15}Οὐχ ὡς αἱ ἀλλότριαι λελογίσμεθα αὐτῷ; πέπρακε γὰρ ἡμᾶς, καὶ καταβρώσει κατέφαγε τὸ ἀργύριον ἡμῶν.
\vs{16}Πάντα τὸν πλοῦτον καὶ τὴν δόξαν, ἣν ἀφείλετο ὁ Θεὸς τοῦ πατρὸς ἡμῶν, ἡμῖν ἔσται καὶ τοῖς τέκνοις ἡμῶν· νῦν οὖν ὅσα σοι εἴρηκεν ὁ Θεὸς, ποίει.
\vs{17}Ἀναστὰς δὲ Ἰακὼβ ἔλαβε τὰς γυναῖκας αὐτοῦ καὶ τὰ παιδία αὐτοῦ ἐπὶ τὰς καμήλους·
\vs{18}Καὶ ἀπήγαγε πάντα τὰ ὑπάρχοντα αὐτῷ, καὶ πᾶσαν τὴν ἀποσκευὴν αὐτοῦ, ἣν περιεποιήσατο ἐν τῇ Μεσοποταμίᾳ, καὶ πάντα τὰ αὐτοῦ, ἀπελθεῖν πρὸς Ἰσαὰκ τὸν πατέρα αὐτοῦ εἰς γῆν Χαναάν.
\vs{19}Λάβαν δὲ ᾤχετο κεῖραι τὰ πρόβατα αὐτοῦ· ἔκλεψε δὲ Ῥαχὴλ τὰ εἴδωλα τοῦ πατρὸς αὐτῆς.
\vs{20}Ἔκρυψε δὲ Ἰακὼβ Λάβαν τὸν Σύρον, τοῦ μὴ ἀναγγεῖλαι αὐτῷ, ὅτι ἀποδιδράσκει.
\vs{21}Καὶ ἀπέδρα αὐτὸς, καὶ τὰ αὐτοῦ πάντα, καὶ διέβη τὸν ποταμὸν, καὶ ὥρμησεν εἰς τὸ ὄρος Γαλαάδ.
\vs{22}Ἀνηγγέλη δὲ Λάβαν τῷ Σύρῳ τῇ ἡμέρᾳ τῇ τρίτῃ, ὅτι ἀπέδρα Ἰακώβ.
\vs{23}Καὶ παραλαβὼν τοὺς ἀδελφοὺς αὐτοῦ μεθʼ ἑαντοῦ, ἐδίωξεν ὀπίσω αὐτοῦ ὁδὸν ἡμερῶν ἑπτά· καὶ κατέλαβεν αὐτὸν ἐν τῷ ὄρει Γαλαάδ.
\vs{24}Ἦλθε δὲ ὁ Θεὸς πρὸς Λάβαν τὸν Σύρον καθʼ ὕπνον τὴν νύκτα, καὶ εἶπεν αὐτῷ, Φύλαξαι σεαυτὸν μή ποτε λαλήσῃς μετὰ Ἰακὼβ πονηρά.
\vs{25}Καὶ κατέλαβε Λάβαν τὸν Ἰακώβ· Ἰακὼβ δὲ ἔπηξεν τὴν σκηνὴν αὐτοῦ ἐν τῷ ὄρει· Λάβαν δὲ ἔστησε τοὺς ἀδελφοὺς αὐτοῦ ἐν τῷ ὄρει Γαλαάδ.
\vs{26}Εἶπε δὲ Λάβαν τῷ Ἰακὼβ, τί ἐποίησας; ἱνατί κρυφῇ ἀπέδρας, καὶ ἐκλοποφόρησάς με, καὶ ἀπήγαγες τὰς θυγατέρας μου, ὡς αἰχμαλώτιδας μαχαίρᾳ;
\vs{27}Καὶ εἰ ἀνήγγειλάς μοι, ἐξαπέστειλα ἄν σε μετʼ εὐφροσύνης, καὶ μετὰ μουσικῶν, καὶ τυμπάνων, καὶ κιθάρας.
\vs{28}Καὶ οὐκ ἠξιώθην καταφιλῆσαι τὰ παιδία μου, καὶ τὰς θυγατέρας μου· νῦν δὲ ἀφρόνως ἔπραξας.
\vs{29}Καὶ νῦν ἰσχύει ἡ χείρ μου κακοποιῆσαί σε· ὁ δὲ Θεὸς τοῦ πατρός σου χθὲς εἶπε πρός με, λέγων, Φύλαξαι σεαυτὸν μή ποτε λαλήσῃς μετὰ Ἰακὼβ πονηρά.
\vs{30}Νῦν οὖν πεπόρευσαι· ἐπιθυμίᾳ γὰρ ἐπεθύμησας ἀπελθεῖν εἰς τὸν οἶκον τοῦ πατρός σου· ἱνατί ἔκλεψας τοὺς θεούς μου;
\vs{31}Ἀποκριθεὶς δὲ Ἰακὼβ εἶπε τῷ Λάβαν, ὅτι ἐφοβήθην· εἶπα γὰρ, μή ποτε ἀφέλῃ τὰς θυγατέρας σου ἀπʼ ἐμοῦ, καὶ πάντα τὰ ἐμά.
\vs{32}Καὶ εἶπεν Ἰακὼβ, παρʼ ᾧ ἂν εὕρῃς τοὺς θεούς σου, οὐ ζήσεται ἐναντίον τῶν ἀδελφῶν ἡμῶν· ἐπίγνωθι τί ἐστι παρʼ ἐμοὶ τῶν σῶν, καὶ λάβε· καὶ οὐκ ἐπέγνω παρʼ αὐτῷ οὐθέν· οὐκ ᾔδει δὲ Ἰακὼβ, ὅτι Ῥαχὴλ ἡ γυνὴ αὐτοῦ ἔκλεψεν αὐτούς.
\vs{33}Εἰσελθὼν δὲ Λάβαν ἠρεύνησεν εἰς τὸν οἶκον Λείας, καὶ οὐχ εὗρεν· καὶ ἐξῆλθεν ἐκ τοῦ οἴκου Λείας, καὶ ἠρεύνησε τὸν οἶκον Ἰακὼβ, καὶ ἐν τῷ οἴκῳ τῶν δύο παιδισκῶν, καὶ οὐχ εὗρεν· εἰσῆλθε δὲ καὶ εἰς τὸν οἶκον Ῥαχήλ.
\vs{34}Ῥαχὴλ δὲ ἔλαβε τὰ εἴδωλα, καὶ ἐνέβαλεν αὐτὰ εἰς τὰ σάγματα τῆς καμήλου, καὶ ἐπεκάθισεν αὐτοῖς.
\vs{35}Καὶ εἶπε τῷ πατρὶ αὐτῆς, μὴ βαρέως φέρε, κύριε· οὐ δυνάμαι ἀναστῆναι ἐνώπιόν σου, ὅτι τὰ κατʼ ἐθισμὸν τῶν γυναικῶν μοι ἐστίν· ἠρεύνησε Λάβαν ἐν ὅλῳ τῷ οἴκῳ, καὶ οὐχ εὗρε τὰ εἴδωλα.
\vs{36}Ὠργίσθη δὲ Ἰακὼβ, καὶ ἐμαχέσατο τῷ Λάβαν· ἀποκριθεὶς δὲ Ἰακὼβ εἶπε τῷ Λάβαν, τί τὸ ἀδίκημά μου; καὶ τί τὸ ἁμάρτημά μου, ὅτι κατεδίωξας ὀπίσω μου,
\vs{37}καὶ ὅτι ἠρεύνησας πάντα τὰ σκεύη τοῦ οἴκου μου; τί εὗρες ἀπὸ πάντων τῶν σκευῶν τοῦ οἴκου σου; θὲς ὧδε ἐνώπιον τῶν ἀδελφῶν σου καὶ τῶν ἀδελφῶν μου, καὶ ἐλεγξάτωσαν ἀνὰ μέσον τῶν δύο ἡμῶν.
\vs{38}Ταῦτά μοι εἴκοσι ἔτη ἐγώ εἰμι μετὰ σοῦ· τὰ πρόβατά σου καὶ αἱ αἶγές σου οὐκ ἠτεκνώθησαν· κριοὺς τῶν προβάτων σου οὐ κατέφαγον.
\vs{39}Θηριάλωτον οὐκ ἐνήνοχά σοι· ἐγὼ ἀπετίννυον παρʼ ἐμαυτοῦ κλέμματα ἡμέρας, καὶ κλέμματα νυκτός.
\vs{40}Ἐγενόμην τῆς ἡμέρας συγκαιόμενος τῷ καύματι, καὶ τῷ παγετῷ τῆς νυκτός· καὶ ἀφίστατο ὁ ὕπνος μου ἀπὸ τῶν ὀφθαλμῶν μου.
\vs{41}Ταῦτά μοι εἴκοσι ἔτη ἐγώ εἰμι ἐν τῇ οἰκίᾳ σου· ἐδούλευσά σοι δεκατέσσαρα ἔτη ἀντὶ τῶν δύο θυγατέρων σου, καὶ ἓξ ἔτη ἐν τοῖς προβάτοις σου, καὶ παρελογίσω τὸν μισθόν μου δέκα ἀμνάσιν.
\vs{42}Εἰ μὴ ὁ Θεὸς τοῦ πατρός μου Ἁβραὰμ, καὶ ὁ φόβος Ἰσαὰκ, ἦν μοι, νῦν ἂν κενόν με ἐξαπέστειλας· τὴν ταπείνωσίν μου, καὶ τὸν κόπον τῶν χειρῶν μου, εἶδεν ὁ Θεός· καὶ ἤλεγξέ σε χθές.

\vs{43}Ἀποκριθεὶς δὲ Λάβαν εἶπε τῷ Ἰακὼβ, αἱ θυγατέρες, θυγατέρες μου, καὶ υἱοὶ, υἱοί μου, καὶ τὰ κτήνη, κτήνη μου· καὶ πάντα ὅσα σὺ ὁρᾷς, ἐμά ἐστι, καὶ τῶν θυγατέρων μου· τί ποιήσω ταύταις σήμερον ἢ τοῖς τέκνοις αὐτῶν, οἷς ἔτεκον;
\vs{44}Νῦν οὖν δεῦρο διαθῶμαι διαθήκην ἐγώ τε καὶ σύ· καὶ ἔσται εἰς μαρτύριον ἀνὰ μέσον ἐμοῦ καὶ σοῦ· εἶπε δὲ αὐτῷ, ἰδοὺ οὐθεὶς μεθʼ ἡμῶν ἐστιν· ἴδε ὁ Θεὸς μάρτυς ἀνὰ μέσον ἐμοῦ καὶ σοῦ.
\vs{45}Λαβὼν δὲ Ἰακὼβ λίθον, ἔστησεν αὐτὸν στήλην.
\vs{46}Εἶπε δὲ Ἰακὼβ τοῖς ἀδελφοῖς αὐτοῦ, συλλέγετε λίθους· καὶ συνέλεξαν λίθους, καὶ ἐποίησαν βουνόν· καὶ ἔφαγον ἐκεῖ ἐπὶ τοῦ βουνοῦ· καὶ εἶπεν αὐτῷ Λάβαν, ὁ βουνὸς οὗτος μαρτυρεῖ ἀνὰ μέσον ἐμοῦ καὶ σοῦ σήμερον.
\vs{47}Καὶ ἐκάλεσεν αὐτὸν Λάβαν, βουνὸς τῆς μαρτυρίας· Ἰακὼβ δὲ ἐκάλεσεν αὐτὸν, βουνὸς μάρτυς.
\vs{48}Εἶπε δὲ Λάβαν τῷ Ἰακὼβ, ἰδοὺ ὁ βουνὸς οὗτος καὶ ἡ στήλη, ἣν ἔστησα ἀνὰ μέσον ἐμοῦ καὶ σοῦ· μαρτυρεῖ ὁ βουνὸς οὗτος, καὶ μαρτυρεῖ ἡ στήλη αὕτη· διὰ τοῦτο ἐκλήθη τὸ ὄνομα, βουνὸς μαρτυρεῖ.
\vs{49}Καὶ ἡ ὅρασις, ἣν εἶπεν, ἐπίδοι ὁ Θεὸς ἀνὰ μέσον ἐμοῦ καὶ σοῦ· ὅτι ἀποστησόμεθα ἕτερος ἀφʼ ἑτέρου.
\vs{50}Εἰ ταπεινώσεις τὰς θυγατέρας μου, εἰ λάβῃς γυναῖκας πρὸς ταῖς θυγατράσι μου, ὅρα, οὐθεὶς μεθʼ ἡμῶν ἐστιν ὁρῶν· Θεὸς μάρτυς μεταξὺ ἐμοῦ καὶ μεταξὺ σοῦ.
\vs{50a}Καὶ εἶπε Λάβαν τῷ Ἰακὼβ, ἰδοὺ ὁ βουνὸς οὗτος καὶ μάρτυς ἡ στήλη αὕτη.
\vs{52}Ἐάν τε γὰρ ἐγὼ μὴ διαβῶ πρός σε, μήτε σὺ διαβῇς πρός με τὸν βουνὸν τοῦτον καὶ τὴν στήλην ταύτην ἐπὶ κακίᾳ.
\vs{53}Ὁ Θεὸς Ἁβραὰμ καὶ ὁ Θεὸς Ναχὼρ κρίναι ἀνὰ μέσον ἡμῶν· καὶ ὤμοσεν Ἰακὼβ κατὰ τοῦ φόβου τοῦ πατρὸς αὐτοῦ Ἰσαάκ.
\vs{54}Καὶ ἔθυσεν θυσίαν ἐν τῷ ὄρει· καὶ ἐκάλεσε τοὺς ἀδελφοὺς αὐτοῦ, καὶ ἔφαγον καὶ ἔπιον, καὶ ἐκοιμήθησαν ἐν τῷ ὄρει.

\ch{32}Ἀναστὰς δὲ Λάβαν τὸ πρωῒ, κατεφίλησε τοὺς υἱοὺς καὶ τὰς θυγατέρας αὐτοῦ, καὶ εὐλόγησεν αὐτούς· καὶ ἀποστραφεὶς Λάβαν ἀπῆλθεν εἰς τὸν τόπον αὐτοῦ.

\vs{2}Καὶ Ἰακὼβ ἀπῆλθεν εἰς τὴν ὁδὸν ἑαυτοῦ· καὶ ἀναβλέψας εἶδε παρεμβολὴν Θεοῦ παρεμβεβληκυῖαν· καὶ συνήντησαν αὐτῷ οἱ Ἄγγελοι τοῦ Θεοῦ.
\vs{3}Εἶπε δὲ Ἰακὼβ, ἡνίκα εἶδεν αὐτοὺς, παρεμβολὴ Θεοῦ αὕτη· καὶ ἐκάλεσε τὸ ὄνομα τοῦ τόπου ἐκείνου, Παρεμβολαί.

\vs{4}Ἀπέστειλε δὲ Ἰακὼβ ἀγγέλους ἔμπροσθεν αὐτοῦ πρὸς Ἡσαῦ τὸν ἀδελφὸν αὐτοῦ εἰς γῆν Σηεὶρ, εἰς χώραν Ἐδώμ.
\vs{5}Καὶ ἐνετείλατο αὐτοῖς, λέγων, οὕτως ἐρεῖτε τῷ κυρίῳ μου Ἡσαῦ· οὕτως λέγει ὁ παῖς σου Ἰακώβ· μετὰ Λάβαν παρῴκησα, καὶ ἐχρόνισα ἕως τοῦ νῦν.
\vs{6}Καὶ ἐγένοντό μοι βόες, καὶ ὄνοι, καὶ πρόβατα, καὶ παῖδες, καὶ παιδίσκαι· καὶ ἀπέστειλα ἀναγγεῖλαι τῷ κυρίῳ μου Ἡσαῦ, ἵνα εὕρῃ ὁ παῖς σου χάριν ἐναντίον σου.
\vs{7}Καὶ ἀνέστρεψαν οἱ ἄγγελοι πρὸς Ἰακὼβ, λέγοντες, ἤλθομεν πρὸς τὸν ἀδελφόν σου Ἡσαυ· καὶ ἰδοὺ αὐτὸς ἔρχεται εἰς συνάντησίν σου, καὶ τετρακόσιοι ἄνδρες μεθʼ αὐτοῦ.
\vs{8}Ἐφοβήθη δὲ Ἰακὼβ σφόδρα, καὶ ἠπορεῖτο· καὶ διεῖλε τὸν λαὸν τὸν μεθʼ ἑαυτοῦ, καὶ τοὺς βόας, καὶ τὰς καμήλους, καὶ τὰ πρόβατα, εἰς δύο παρεμβολάς.
\vs{9}Καὶ εἶπεν Ἰακὼβ, ἐὰν ἔλθῃ Ἡσαῦ εἰς παρεμβολὴν μίαν, καὶ κόψῃ αὐτὴν, ἔσται ἡ παρεμβολὴ ἡ δευτέρα εἰς τὸ σώζεσθαι.
\vs{10}Εἶπε δὲ Ἰακὼβ, ὁ Θεὸς τοῦ πατρός μου Ἁβραὰμ, καὶ ὁ Θεὸς τοῦ πατρός μου Ἰσαὰκ, Κύριε σὺ ὁ εἰπών μοι, ἀπότρεχε εἰς τὴν γῆν τῆς γενέσεώς σου, καὶ εὖ σε ποιήσω·
\vs{11}Ἱκανούσθω μοι ἀπὸ πάσης δικαιοσύνης, καὶ ἀπὸ πάσης ἀληθείας, ἧς ἐποίησας τῷ παιδί σου· ἐν γὰρ τῇ ῥάβδῳ μου ταύτῃ διέβην τὸν Ἰορδάνην τοῦτον· νυνὶ δὲ γέγονα εἰς δύο παρεμβολάς.
\vs{12}Ἐξελοῦ με ἐκ χειρὸς τοῦ ἀδελφοῦ μου, ἐκ χειρὸς Ἡσαῦ· ὅτι φοβοῦμαι ἐγὼ αὐτὸν, μή ποτε ἐλθὼν πατάξῃ με, καὶ μητέρα ἐπὶ τέκνοις.
\vs{13}Σὺ δὲ εἶπας, εὐ σε ποιήσω, καὶ θήσω τὸ σπέρμα σου ὡς τὴν ἄμμον τῆς θαλάσσης, ἣ οὐκ ἀριθμηθήσεται ὑπὸ τοῦ πλήθους.
\vs{14}Καὶ ἐκοιμήθη ἐκεῖ τὴν νύκτα ἐκείνην· καὶ ἔλαβεν ὧν ἔφερεν δῶρα· καὶ ἐξαπέστειλεν Ἡσαῦ τῷ ἀδελφῷ αὐτοῦ,
\vs{15}αἶγας διακοσίας, τράγους εἴκοσι, πρόβατα διακόσια, κριοὺς εἴκοσι,
\vs{16}καμήλους θηλαζούσας καὶ τὰ παιδία αὐτῶν τριάκοντα, βόας τεσσαράκοντα, ταύρους δέκα, ὄνους εἴκοσι, καὶ πώλους δέκα.
\vs{17}Καὶ ἔδωκεν αὐτὰ τοῖς παισὶν αὐτοῦ ποίμνιον κατὰ μόνας· εἶπε δὲ τοῖς παισὶν αὐτοῦ, προπορεύεσθε ἔμπροσθέν μου, καὶ διάστημα ποιεῖτε ἀνὰ μέσον ποίμνης καὶ ποίμνης.
\vs{18}Καὶ ἐνετείλατο τῷ πρώτῳ, λέγων, ἐάν σοι συναντήσῃ Ἡσαῦ ὁ ἀδελφός μου, καὶ ἐρωτᾷ σε, λέγων, τίνος εἶ; καὶ ποῦ πορεύῃ; καὶ τίνος ταῦτα τὰ προπορευόμενά σου;
\vs{19}Ἐρεῖς, τοῦ παιδός σου Ἰακώβ· δῶρα ἀπέσταλκε τῷ κυρίῳ μου Ἡσαῦ· καὶ ἰδοὺ αὐτὸς ὀπίσω ἡμῶν.
\vs{20}Καὶ ἐνετείλατο τῷ πρώτῳ, καὶ τῷ δευτέρῳ, καὶ τῷ τρίτῳ, καὶ πᾶσι τοῖς προπορευομένοις ὀπίσω τῶν ποιμνίων τούτων, λέγων, κατὰ τὸ ῥῆμα τοῦτο λαλήσατε Ἡσαῦ ἐν τῷ εὑρεῖν ὑμᾶς αὐτόν·
\vs{21}Καὶ ἐρεῖτε, ἰδοὺ ὁ παῖς σου Ἰακὼβ παραγίνεται ὀπίσω ἡμῶν· εἶπε γὰρ, ἐξιλάσομαι τὸ πρόσωπον αὐτοῦ ἐν τοῖς δώροις τοῖς προπορευομένοις αὐτοῦ, καὶ μετὰ τοῦτο ὄψομαι τὸ πρόσωπον αὐτοῦ· ἴσως γὰρ προσδέξεται τὸ πρόσωπόν μου.
\vs{22}Καὶ προεπορεύετο τὰ δῶρα κατὰ πρόσωπον αὐτοῦ· αὐτὸς δὲ ἐκοιμήθη τὴν νύκτα ἐκείνην ἐν τῇ παρεμβολῇ.
\vs{23}Ἀναστὰς δὲ τὴν νύκτα ἐκείνην, ἔλαβε τὰς δύο γυναῖκας, καὶ τὰς δύο παιδίσκας, καὶ τὰ ἕνδεκα παιδία αὐτοῦ, καὶ διέβη τὴν διάβασιν τοῦ Ἰαβώχ.
\vs{24}Καὶ ἔλαβεν αὐτοὺς, καὶ διέβη τὸν χειμάῤῥουν, καὶ διεβίβασε πάντα τὰ αὐτοῦ.

\vs{25}Ὑπελείφθη δὲ Ἰακὼβ μόνος· καὶ ἐπάλαιεν ἄνθρωπος μετʼ αὐτοῦ ἕως πρωΐ.
\vs{26}Εἶδε δὲ ὅτι οὐ δύναται πρὸς αὐτόν· καὶ ἥψατο τοῦ πλάτους τοῦ μηροῦ αὐτοῦ, καὶ ἐνάρκησε τὸ πλάτος τοῦ μηροῦ Ἰακὼβ ἐν τῷ παλαίειν αὐτὸν μετʼ αὐτοῦ.
\vs{27}Καὶ εἶπεν αὐτῷ, ἀπόστειλόν με, ἀνέβη γὰρ ὁ ὄρθρος. ὁ δὲ εἶπεν, οὐ μή σε ἀποστείλω, ἐὰν μή με εὐλογήσῃς.
\vs{28}Εἶπε δὲ αὐτῷ, τί τὸ ὄνομά σου ἐστίν; ὁ δὲ εἶπεν, Ἰακώβ.
\vs{29}Καὶ εἶπεν αὐτῷ, οὐ κληθήσεται ἔτι τὸ ὄνομά σου Ἰακὼβ, ἀλλʼ Ἰσραὴλ ἔσται τὸ ὄνομά σου· ὅτι ἐνίσχυσας μετὰ Θεοῦ, καὶ μετὰ ἀνθρώπων δυνατὸς ἔσῃ.
\vs{30}Ἠρώτησε δὲ Ἰακὼβ, καὶ εἶπεν, ἀνάγγειλόν μοι τὸ ὄνομά σου· καὶ εἶπεν, ἱνατί τοῦτο ἐρωτᾷς σὺ τὸ ὄνομά μου; καὶ εὐλόγησεν αὐτὸν ἐκεῖ.
\vs{31}Καὶ ἐκάλεσεν Ἰακὼβ τὸ ὄνομα τοῦ τόπου ἐκείνου, εἶδος Θεοῦ· εἶδον γὰρ Θεὸν πρόσωπον πρὸς πρὸσωπον, καὶ ἐσώθη μου ἡ ψυχή.
\vs{32}Ἀνέτειλεν δὲ αὐτῷ ὁ ἥλιος, ἡνίκα παρῆλθε τὸ εἶδος τοῦ Θεοῦ· αὐτὸς δὲ ἐπέσκαζε τῷ μηρῷ αὐτοῦ.
\vs{33}Ἕνεκεν τούτου οὐ μὴ φάγωσιν υἱοὶ Ἰσραὴλ τὸ νεῦρον, ὃ ἐνάρκησεν, ὅ ἐστιν ἐπὶ τοῦ πλάτους τοῦ μηροῦ, ἕως τῆς ἡμέρας ταύτης, ὅτι ἥψατο τοῦ πλάτους τοῦ μηροῦ Ἰακὼβ τοῦ νεύρου, ὃ ἐνάρκησεν.

\ch{33}
Ἀναβλέψας δὲ Ἰακὼβ τοῖς ὀφθαλμοῖς αὐτοῦ εἶδε· καὶ ἰδοὺ Ἡσαῦ ὁ ἀδελφὸς αὐτοῦ ἐρχόμενος, καὶ τετρακόσιοι ἄνδρες μετʼ αὐτοῦ· καὶ διεῖλεν Ἰακὼβ τὰ παιδία ἐπὶ Λείαν, καὶ ἐπὶ Ῥαχὴλ, καὶ τὰς δύο παιδίσκας.
\vs{2}Καὶ ἔθετο τὰς δύο παιδίσκας καὶ τοὺς υἱοὺς αὐτῶν ἐν πρώτοις, καὶ Λείαν καὶ τὰ παιδία αὐτῆς ὀπίσω, καὶ Ῥαχὴλ καὶ Ἰωσὴφ ἐσχάτους.
\vs{3}Αὐτὸς δὲ προῆλθεν ἔμπροσθεν αὐτῶν· καὶ προσεκύνησεν ἐπὶ τὴν γῆν ἑπτάκις, ἕως τοῦ ἐγγίσαι τῷ ἀδελφῷ αὐτοῦ.
\vs{4}Καὶ προσέδραμεν Ἡσαῦ εἰς συνάντησιν αὐτῷ· καὶ περιλαβὼν αὐτὸν προσέπεσεν ἐπὶ τὸν τράχηλον αὐτοῦ, καὶ κατεφίλησεν αὐτόν· καὶ ἔκλαυσαν ἀμφότεροι.
\vs{5}Καὶ ἀναβλέψας Ἡσαῦ εἶδε τὰς γυναῖκας καὶ τὰ παιδία· καὶ εἶπε, τί ταῦτά σοι ἐστίν; ὁ δὲ εἶπε, τὰ παιδία, οἷς ἠλέησεν ὁ Θεὸς τὸν παῖδά σου.
\vs{6}Καὶ προσήγγισαν αἱ παιδίσκαι καὶ τὰ τέκνα αὐτῶν, καὶ προσεκύνησαν.
\vs{7}Καὶ προσήγγισε Λεία καὶ τὰ τέκνα αὐτῆς, καὶ προσεκύνησαν· καὶ μετὰ ταῦτα προσήγγισε Ῥαχὴλ καὶ Ἰωσὴφ, καὶ προσεκύνησαν.
\vs{8}Καὶ εἶπε, τί ταῦτά σοι ἐστὶν, πᾶσαι αἱ παρεμβολαὶ αὗται, αἷς ἀπήντηκα; ὁ δὲ εἶπεν, ἵνα εὕρῃ ὁ παῖς σου χάριν ἐναντίον σου, κύριε.
\vs{9}Εἶπε δὲ Ἡσαῦ, ἔστι μοι πολλὰ, ἀδελφέ· ἔστω σοι τὰ σά.
\vs{10}Εἶπε δὲ Ἰακὼβ, εἰ εὓρον χάριν ἐναντίον σου, δέξαι τὰ δῶρα διὰ τῶν ἐμῶν χειρῶν· ἕνεκεν τούτου εἶδον τὸ πρόσωπόν σου, ὡς ἄν τις ἴδοι πρόσωπον Θεοῦ, καὶ εὐδοκήσεις με.
\vs{11}Λάβε τὰς εὐλογίας μου, ἃς ἤνεγκά σοι, ὅτι ἠλέησέ με ὁ Θεὸς, καὶ ἔστι μοι πάντα· καὶ ἐβιάσατο αὐτὸν, καὶ ἔλαβε.
\vs{12}Καὶ εἶπεν, ἀπάραντες πορευσώμεθα ἐπʼ εὐθεῖαν.
\vs{13}Εἶπε δὲ αὐτῷ, ὁ κύριός μου γινώσκει, ὅτι τὰ παιδία ἁπαλώτερα, καὶ τὰ πρόβατα καὶ αἱ βόες λοχεύονται ἐπʼ ἐμέ· ἐὰν οὖν καταδιώξω αὐτὰ ἡμέραν μίαν, ἀποθανοῦνται πάντα τὰ κτήνη.
\vs{14}Προελθέτω ὁ κύριός μου ἔμπροσθεν τοῦ παιδὸς αὐτοῦ· ἐγὼ δὲ ἐνισχύσω ἐν τῇ ὁδῷ κατὰ σχολὴν τῆς πορεύσεως τῆς ἐναντίον μου, καὶ κατὰ πόδα τῶν παιδαρίων, ἕως τοῦ ἐλθεῖν με πρὸς τὸν κύριόν μου εἰς Σηείρ.
\vs{15}Εἶπε δὲ Ἡσαῦ, καταλείψω μετὰ σοῦ ἀπὸ τοῦ λαοῦ τοῦ μετʼ ἐμοῦ· ὁ δὲ εἶπεν, ἱνατί τοῦτο; ἱκανὸν ὅτι εὗρον χάριν ἐναντίον σου, κύριε.
\vs{16}Ἀπέστρεψε δὲ Ἡσαῦ ἐν τῇ ἡμέρᾳ ἐκείνῃ εἰς τὴν ὁδὸν αὐτοῦ εἰς Σηείρ.
\vs{17}Καὶ Ἰακὼβ ἀπαίρει εἰς σκηνὰς, καὶ ἐποίησεν ἑαυτῷ ἐκεῖ οἰκίας, καὶ τοῖς κτήνεσιν αὐτοῦ ἐποίησε σκηνάς· διὰ τοῦτο ἐκάλεσε τὸ ὄνομα τοῦ τόπου ἐκείνου, Σκηναί.

\vs{18}Καὶ ἦλθεν Ἰακὼβ εἰς Σαλὴμ, πόλιν Σηκίμων, ἥ ἐστιν ἐν γῇ Χαναὰν, ὅτε ἐπανῆλθεν ἐκ τῆς Μεσοποταμίας Συρίας· καὶ παρενέλαβε κατὰ πρόσωπον τῆς πόλεως.
\vs{19}Καὶ ἐκτήσατο τὴν μερίδα τοῦ ἀγροῦ, οὗ ἔστησεν ἐκεῖ τὴν σκηνὴν αὐτοῦ, παρὰ Ἐμμὼρ πατρὸς Συχὲμ, ἑκατὸν ἀμνῶν.
\vs{20}Καὶ ἔστησεν ἐκεῖ θυσιαστήριον, καὶ ἐπεκαλέσατο τὸν Θεὸν Ἰσραήλ.

\ch{34}
Ἐξῆλθε δὲ Δείνα, ἡ θυγάτηρ Λείας, ἣν ἔτεκε τῷ Ἰακώβ, καταμαθεῖν τὰς θυγατέρας τῶν ἐγχωρίων.
\vs{2}Καὶ εἶδεν αὐτὴν Συχὲμ ὁ υἱὸς Ἐμμὼρ ὁ Εὐαῖος, ὁ ἄρχων τῆς γῆς· καὶ λαβὼν αὐτὴν, ἐκοιμήθη μετʼ αὐτῆς, καὶ ἐταπείνωσεν αὐτήν.
\vs{3}Καὶ προσέσχε τῇ ψυχῇ Δείνας τῆς θυγατρὸς Ἰακώβ· καὶ ἠγάπησε τὴν παρθένον· καὶ ἐλάλησε κατὰ τὴν διάνοιαν τῆς παρθένου αυτῇ.
\vs{4}Εἶπε Συχὲμ πρὸς Ἐμμὼρ τὸν πατέρα αὐτοῦ, λέγων, λάβε μοι τὴν παῖδα ταύτην εἰς γυναῖκα.
\vs{5}Ἰακὼβ δὲ ἤκουσεν, ὅτι ἐμίανεν ὁ υἱὸς Ἐμμὼρ Δείναν τὴν θυγατέρα αὐτοῦ· οἱ δὲ υἱοὶ αὐτοῦ ἦσαν μετὰ τῶν κτηνῶν αὐτοῦ ἐν τῷ πεδίῳ· παρεσιώπησε δὲ Ἰακὼβ, ἕως τοῦ ἐλθεῖν αὐτούς.
\vs{6}Ἐξῆλθε δὲ Ἐμμὼρ ὁ πατὴρ Συχὲμ πρὸς Ἰακὼβ, λαλῆσαι αὐτῷ.
\vs{7}Οἱ δὲ υἱοὶ Ἰακὼβ ἦλθον ἐκ τοῦ πεδίου· ὡς δὲ ἤκουσαν, κατενύγησαν οἱ ἄνδρες, καὶ λυπηρὸν ἦν αὐτοῖς σφόδρα· ὅτι ἄσχημον ἐποίησεν ἐν Ἰσραὴλ, κοιμηθεὶς μετὰ τῆς θυγατρὸς Ἰακώβ· καὶ οὐχ οὕτως ἔσται.
\vs{8}Καὶ ἐλάλησεν Ἐμμὼρ αὐτοῖς, λέγων, Συχὲμ ὁ υἱός μου προείλετο τῇ ψυχῇ τὴν θυγατέρα ὑμῶν· δότε οὖν αὐτὴν αὐτῷ γυναῖκα,
\vs{9}καὶ ἐπιγαμβρεύσασθε ἡμῖν· τὰς θυγατέρας ὑμῶν δότε ἡμῖν, καὶ τὰς θυγατέρας ἡμῶν λάβετε τοῖς υἱοῖς ὑμῶν.
\vs{10}Καὶ ἐν ἡμῖν κατοικεῖτε· καὶ ἡ γῆ ἰδοὺ πλατεῖα ἐναντίον ὑμῶν· κατοικεῖτε, καὶ ἐμπορεύεσθε ἐπʼ αὐτῆς, καὶ ἐγκτᾶσθε ἐν αὐτῇ.
\vs{11}Εἶπε δὲ Συχὲμ πρὸς τὸν πατέρα αὐτῆς, καὶ πρὸς τοὺς ἀδελφοὺς αὐτῆς, εὕροιμι χάριν ἐναντίον ὑμῶν· καὶ ὃ ἐὰν εἴπητε, δώσομεν.
\vs{12}Πληθύνατε τὴν φερνὴν σφόδρα, καὶ δώσω καθότι ἂν εἴπητέ μοι, καὶ δώσετέ μοι τὴν παῖδα ταύτην εἰς γυναῖκα.

\vs{13}Ἀπεκρίθησαν δὲ οἱ υἱοὶ Ἰακὼβ τῷ Συχὲμ, καὶ Ἐμμὼρ τῷ πατρὶ αὐτοῦ, μετὰ δόλου· καὶ ἐλάλησαν αὐτοῖς, ὅτι ἐμίαναν Δείναν τὴν ἀδελφὴν αὐτῶν.
\vs{14}Καὶ εἶπαν αὐτοῖς Συμεὼν καὶ Λευὶ οἱ ἀδελφοὶ Δείνας, οὐ δυνησόμεθα ποιῆσαι τὸ ῥῆμα τοῦτο, δοῦναι τὴν ἀδελφὴν ἡμῶν ἀνθρώπῳ, ὃς ἔχει ἀκροβυστίαν· ἔστι γὰρ ὄνειδος ἡμῖν.
\vs{15}Μόνον ἐν τούτῳ ὁμοιωθησόμεθα ὑμῖν, καὶ κατοικήσομεν ἐν ὑμῖν, ἐὰν γένησθε ὡς ἡμεῖς καὶ ὑμεῖς, ἐν τῷ περιτμηθῆναι ὑμῶν πᾶν ἀρσενικόν.
\vs{16}Καὶ δώσομεν τὰς θυγατέρας ἡμῶν ὑμῖν, καὶ ἀπὸ τῶν θυγατέρων ὑμῶν ληψόμεθα ἡμῖν γυναῖκας, καὶ οἰκήσομεν παρʼ ὑμῖν, καὶ ἐσόμεθα ὡς γένος ἕν.
\vs{17}Ἐὰν δὲ μὴ εἰσακούσητε ἡμῶν τοῦ περιτεμέσθαι, λαβόντες τὴν θυγατέρα ἡμῶν ἀπελευσόμεθα.
\vs{18}Καὶ ἤρεσαν οἱ λόγοι ἐναντίον Ἐμμὼρ, καὶ ἐναντίον Συχὲμ τοῦ υἱοῦ Ἐμμώρ.
\vs{19}Καὶ οὐκ ἐχρόνισεν ὁ νεανίσκος τοῦ ποιῆσαι τὸ ῥῆμα τοῦτο· ἐνέκειτο γὰρ τῇ θυγατρὶ Ἰακώβ· αὐτὸς δὲ ἦν ἐνδοξότατος πάντων τῶν ἐν τῷ οἴκῳ τοῦ πατρὸς αὐτοῦ.
\vs{20}Ἦλθε δὲ Ἐμμὼρ καὶ Συχὲμ ὁ υἱὸς αὐτοῦ πρὸς τὴν πύλην τῆς πόλεως αὐτῶν, καὶ ἐλάλησαν πρὸς τοὺς ἄνδρας τῆς πόλεως αὐτῶν, λέγοντες,
\vs{21}Οἱ ἄνθρωποι οὗτοι εἰρήνικοί εἰσι, μεθʼ ἡμῶν οἰκείτωσαν επὶ τῆς γῆς, καὶ ἐμπορευέσθωσαν αὐτήν· ἡ δὲ γῆ ἰδοὺ πλατεῖα ἐναντίον αὐτῶν· τὰς θυγατέρας αὐτῶν ληψόμεθα ἡμῖν γυναῖκας, καὶ τὰς θυγατέρας ἡμῶν δώσομεν αὐτοῖς.
\vs{22}Ἐν τούτῳ μόνον ὁμοιωθήσονται ἡμῖν οἱ ἄνθρωποι τοῦ κατοικεῖν μεθʼ ἡμῶν, ὥστε εἶναι λαὸν ἕνα, ἐν τῷ περιτεμέσθαι ἡμῶν πᾶν ἀρσενικὸν, καθὰ καὶ αὐτοὶ περιτέτμηνται.
\vs{23}Καὶ τὰ κτήνη αὐτῶν, καὶ τὰ τετράποδα, καὶ τὰ ὑπάρχοντα αὐτῶν, οὐχ ἡμῶν ἔσται; μόνον ἐν τούτῳ ὁμοιωθῶμεν αὐτοῖς, καὶ οἰκήσουσι μεθʼ ἡμῶν.
\vs{24}Καὶ εἰσήκουσαν Ἐμμὼρ καὶ Συχὲμ τοῦ υἱοῦ αὐτοῦ πάντες οἱ ἐμπορευόμενοι τὴν πύλην τῆς πόλεως αὐτῶν· καὶ περιετέμοντο τὴν σάρκα τῆς ἀκροβυστίας αὐτῶν πᾶς ἄρσην.

\vs{25}Ἐγένετο δὲ ἐν τῇ ἡμέρᾳ τῇ τρίτῃ, ὅτε ἦσαν ἐν τῷ πόνῳ, ἔλαβον οἱ δύο υἱοὶ Ἰακὼβ Συμεὼν καὶ Λευὶ, ἀδελφοὶ Δείνας, ἕκαστος τὴν μάχαιραν αὐτοῦ, καὶ εἰσῆλθον εἰς τὴν πόλιν ἀσφαλὼς, καὶ ἀπέκτειναν πᾶν ἀρσενικόν.
\vs{26}Τόν τε Ἐμμὼρ καὶ Συχὲμ τὸν υἱὸν αὐτοῦ ἀπέκτειναν ἐν στόματι μαχαίρας· καὶ ἔλαβον τὴν Δείναν ἐκ τοῦ οἴκου τοῦ Συχὲμ, καὶ ἐξῆλθον.
\vs{27}Οἱ δὲ υἱοὶ Ἰακὼβ εἰσῆλθον ἐπὶ τοὺς τραυματίας, καὶ διήρπασαν τὴν πόλιν, ἐν ᾗ ἐμίαναν Δείναν τὴν ἀδελφὴν αὐτῶν.
\vs{28}Καὶ τὰ πρόβατα αὐτῶν, καὶ τοὺς βόας αὐτῶν, καὶ τοὺς ὄνους αὐτῶν, ὅσα τε ἦν ἐν τῇ πόλει, καὶ ὅσα ἦν ἐν τῷ πεδίῳ, ἔλαβον.
\vs{29}Καὶ πάντα τὰ σώματα αὐτῶν, καὶ πᾶσαν τὴν ἀποσκευὴν αὐτῶν, καὶ τὰς γυναῖκας αὐτῶν ἠχμαλώτευσαν· καὶ διήρπασαν ὅσα τε ἦν ἐν τῇ πόλει, καὶ ὅσα ἦν ἐν ταῖς οἰκίαις.
\vs{30}Εἶπε δὲ Ἰακὼβ πρὸς Συμεὼν καὶ Λευὶ, μισητόν με πεποιήκατε, ὥστε πονηρόν με εἶναι πᾶσι τοῖς κατοικοῦσι τὴν γῆν, ἔν τε τοῖς Χαναναίοις, καὶ ἐν τοῖς Φερεζαίοις· ἐγὼ δὲ ὀλιγοστός εἰμι ἐν ἀριθμῷ· καὶ συναχθέντες ἐπʼ ἐμὲ συγκόψουσί με, καὶ ἐκτριβήσομαι ἐγὼ, καὶ ὁ οἶκός μου.
\vs{31}Οἱ δὲ εἶπαν, ἀλλʼ ὡσεὶ πόρνῃ χρήσονται τῇ ἀδελφῇ ἡμῶν;

\ch{35}
Εἶπε δὲ ὁ Θεὸς πρὸς Ἰακὼβ, ἀναστὰς ἀνάβηθι εἰς τὸν τόπον Βαιθὴλ, καὶ οἴκει ἐκεῖ· καὶ ποίησον ἐκεῖ θυσιαστήριον τῷ Θεῷ τῷ ὀφθέντι σοι, ἐν τῷ ἀποδιδράσκειν σε ἀπὸ προσώπου Ἡσαῦ τοῦ ἀδελφοῦ σου.
\vs{2}Εἶπε δὲ Ἰακὼβ τῷ οἴκῳ αὐτοῦ, καὶ πᾶσι τοῖς μετʼ αὐτοῦ, ἄρατε τοὺς θεοὺς τοὺς ἀλλοτρίους τοὺς μεθʼ ὑμῶν ἐκ μέσου ὑμῶν, καὶ καθαρίσθητε, καὶ ἀλλάξατε τὰς στολὰς ὑμῶν.
\vs{3}Καὶ ἀναστάντες ἀναβῶμεν εἰς Βαιθὴλ, καὶ ποιήσωμεν ἐκεῖ θυσιαστήριον τῷ Θεῷ τῷ ἐπακούσαντί μου ἐν ἡμέρᾳ θλίψεως, ὃς ἦν μετʼ ἐμοῦ, καὶ διέσωσέ με ἐν τῇ ὁδῷ, ᾗ ἐπορεύθην.
\vs{4}Καὶ ἔδωκαν τῷ Ἰακὼβ τοὺς θεοὺς τοὺς ἀλλοτρίους, οἳ ἦσαν ἐν ταῖς χερσὶν αὐτῶν, καὶ τὰ ἐνώτια τὰ ἐν τοῖς ὠσὶν αὐτῶν· καὶ κατέκρυψεν αὐτὰ Ἰακὼβ ὑπὸ τὴν τερέβινθον τὴν ἐν Σηκίμοις· καὶ ἀπώλεσεν αὐτὰ, ἕως τῆς σήμερον ἡμέρας.
\vs{5}Καὶ ἐξῇρεν Ἰσραὴλ ἐκ Σηκίμων· καὶ ἐγένετο φόβος Θεοῦ ἐπὶ τὰς πόλεις τὰς κύκλῳ αὐτῶν, καὶ οὐ κατεδίωξαν ὀπίσω τῶν υἱῶν Ἰσραήλ.
\vs{6}Ἦλθε δὲ Ἰακὼβ εἰς Λουζὰ ἥ ἐστιν ἐν γῇ Χαναὰν, ἥ ἐστι Βαιθὴλ, αὐτὸς, καὶ πᾶς ὁ λαὸς, ὃς ἦν μετʼ αὐτοῦ.
\vs{7}Καὶ ᾠκοδόμησεν ἐκεῖ θυσιαστήριον, καὶ ἐκάλεσε τὸ ὄνομα τοῦ τόπου, Βαιθήλ· ἐκεῖ γὰρ ἐφάνη αὐτῷ ὁ Θεὸς, ἐν τῷ ἀποδιδράσκειν αὐτὸν ἀπὸ προσώπου Ἡσαῦ τοῦ ἀδελφοῦ αὐτοῦ.

\vs{8}Ἀπέθανε δὲ Δεβόῤῥα, ἡ τρόφος Ῥεβέκκας, καὶ ἐτάφη κατώτερον Βαιθὴλ ὑπὸ τὴν βάλανον· καὶ ἐκάλεσεν Ἰακὼβ τὸ ὄνομα αὐτῆς, βάλανος πένθους.
\vs{9}Ὤφθη δὲ ὁ Θεὸς τῷ Ἰακὼβ ἔτι ἐν Λουζᾷ, ὅτε παρεγένετο ἐκ Μεσοποταμίας τῆς Συρίας· καὶ εὐλόγησεν αὐτὸν ὁ Θεὸς.
\vs{10}Καὶ εἶπεν αὐτῷ ὁ Θεὸς, τὸ ὄνομά σου οὐ κληθήσεται ἔτι Ἰακὼβ, ἀλλʼ Ἰσραὴλ ἔσται τὸ ὄνομά σου· καὶ ἐκάλεσε τὸ ὄνομα αὐτοῦ Ἰσραήλ.
\vs{11}Εἶπε δὲ αὐτῷ ὁ Θεὸς, ἐγὼ ὁ Θεός σου· αὐξάνου, καὶ πληθύνου· ἔθνη καὶ συναγωγαὶ ἐθνῶν ἔσονται ἐκ σοῦ, καὶ βασιλεῖς ἐκ τῆς ὀσφύος σου ἐξελεύσονται.
\vs{12}Καὶ τὴν γῆν, ἣν ἔδωκα Ἁβραὰμ καὶ Ἰσαὰκ, σοὶ δέδωκα αὐτήν· σοὶ ἔσται· καὶ τῷ σπέρματί σου μετὰ σὲ δώσω τῆν γῆν ταύτην.
\vs{13}Ἀνέβη δὲ ὁ Θεὸς ἀπʼ αὐτοῦ ἐκ τοῦ τόπου, οὗ ἐλάλησε μετʼ αὐτοῦ.
\vs{14}Καὶ ἔστησεν Ἰακὼβ στήλην ἐν τῷ τόπῳ, ᾧ ἐλάλησε μετʼ αὐτοῦ ὁ Θεὸς, στήλην λιθίνην· καὶ ἔσπεισεν ἐπʼ αὐτὴν σπονδὴν, καὶ ἐπέχεεν ἐπʼ αὐτὴν ἔλαιον.
\vs{15}Καὶ ἐκάλεσεν Ἰακὼβ τὸ ὄνομα τοῦ τόπου, ἐν ᾧ ἐλάλησε μετʼ αὐτοῦ ἐκεῖ ὁ Θεὸς, Βαιθήλ.
\vs{16}Ἀπάρας δὲ Ἰακὼβ ἐκ Βαιθὴλ, ἔπηξε τὴν σκηνὴν αὐτοῦ ἐπέκεινα τοῦ πύργου Γαδέρ· ἐγένετο δὲ ἡνίκα ἤγγισεν εἰς Χαβραθὰ τοῦ ἐλθεῖν εἰς τὴν Ἐφραθᾶ, ἔτεκε Ῥαχήλ· καὶ ἐδυστόκησεν ἐν τῷ τοκετῷ.
\vs{17}Ἐγένετο δὲ ἐν τῷ σκληρὼς αὐτὴν τίκτειν, εἶπεν αὐτῇ ἡ μαῖα, θάρσει, καὶ γὰρ οὗτός σοι ἐστὶν υἱός.
\vs{18}Ἐγένετο δὲ ἐν τῷ ἀφιέναι αὐτὴν τὴν ψυχὴν, ἀπέθνησκε γὰρ, ἐκάλεσε τὸ ὄνομα αὐτοῦ, υἱὸς ὀδύνης μου· ὁ δὲ πατὴρ ἐκάλεσεν τὸ ὄνομα αὐτοῦ, Βενιαμίν.
\vs{19}Ἀπέθανε δὲ Ῥαχὴλ, καὶ ἐτάφη ἐν τῇ ὁδῷ τοῦ ἱπποδρόμου Ἐφραθᾶ· αὕτη ἐστὶ Βηθλεέμ.
\vs{20}Καὶ ἔστησεν Ἰακὼβ στήλην ἐπὶ τοῦ μνημείου αὐτῆς· αὕτη ἐστὶν ἡ στήλη ἐπὶ τοῦ μνημείου Ῥαχὴλ ἕως τῆς ἡμέρας ταύτης.
\vs{22}Ἐγένετο δὲ ἡνίκα κατῴκησεν Ἰσραὴλ ἐν τῇ γῇ ἐκείνῃ, ἐπορεύθη Ῥουβὴν, καὶ ἐκοιμήθη μετὰ Βαλλὰς, τῆς παλλακῆς τοῦ πατρὸς αὐτοῦ Ἰακώβ· καὶ ἤκουσεν Ἰσραὴλ, καὶ πονηρὸν ἐφάνη ἐναντίον αὐτοῦ.

Ἦσαν δὲ οἱ υἱοὶ Ἰακὼβ, δώδεκα.
\vs{23}Υἱοὶ Λείας, πρωτότοκος Ἰακὼβ, Ῥουβὴν, Συμεὼν, Λευὶ, Ἰούδας, Ἰσσάχαρ, Ζαβουλών.
\vs{24}Υἱοὶ δὲ Ῥαχὴλ, Ἰωσὴφ, καὶ Βενιαμίν.
\vs{25}Υἱοὶ δὲ Βαλλᾶς παιδίσκης Ῥαχὴλ, Δαν, καὶ Νεφθαλείμ.
\vs{26}Υἱοὶ δὲ Ζελφᾶς παιδίσκης Λείας, Γὰδ, καὶ Ἀσήρ· οὗτοι υἱοὶ Ἰακὼβ, οἳ ἐγένοντο αὐτῷ ἐν Μεσοποταμίᾳ τῆς Συρίας.
\vs{27}Ἦλθε δὲ Ἰακὼβ πρὸς Ἰσαὰκ τὸν πατέρα αὐτοῦ εἰς Μαμβρῆ, εἰς πόλιν τοῦ πεδίου· αὕτη ἐστὶ Χεβρὼν ἐν γῇ Χαναὰν, οὗ παρῴκησεν Ἁβραὰμ καὶ Ἰσαάκ.
\vs{28}Ἐγένοντο δὲ αἱ ἡμέραι Ἰσαὰκ, ἃς ἔζησεν, ἔτη ἑκατὸν ὀγδοήκοντα.
\vs{29}Καὶ ἐκλείπων Ἰσαὰκ ἀπέθανε, καὶ προσετέθη πρὸς τὸ γένος αὐτοῦ πρεσβύτερος καὶ πλήρης ἡμερῶν· καὶ ἔθαψαν αὐτὸν Ἡσαῦ καὶ Ἰακὼβ οἱ υἱοὶ αὐτοῦ.

\ch{36}
Αὗται δὲ αἱ γενέσεις Ἡσαῦ· αὐτός ἐστιν Ἐδώμ.
\vs{2}Ἡσαῦ δὲ ἔλαβε τὰς γυναῖκας ἑαυτῷ ἀπὸ τῶν θυγατέρων τῶν Χαναναίων· τὴν Ἀδὰ, θυγατέρα Αἰλὼμ τοῦ Χετταίου· καὶ τὴν Ὀλιβεμὰ, θυγατέρα Ἀνὰ τοῦ υἱοῦ Σεβεγὼν τοῦ Εὐαίου.
\vs{3}Καὶ τὴν Βασεμὰθ, θυγατέρα Ἰσμαὴλ, ἀδελφὴν Ναβαιώθ.
\vs{4}Ἔτεκε δὲ αὐτῷ Ἀδὰ τὸν Ἑλιφάς· καὶ Βασεμὰθ ἔτεκε τὸν Ῥαγουήλ.
\vs{5}Καὶ Ὀλιβεμὰ ἔτεκε τὸν Ἰεοὺς, καὶ τὸν Ἰεγλὸμ, καὶ τὸν Κορέ· οὗτοι υἱοὶ Ἡσαῦ, οἳ ἐγένοντο αὐτῷ ἐν γῇ Χαναάν.
\vs{6}Ἔλαβε δὲ Ἡσαῦ τὰς γυναῖκας αὐτοῦ, καὶ τοὺς υἱοὺς αὐτοῦ, καὶ τὰς θυγατέρας αὐτοῦ, καὶ πάντα τὰ σώματα τοῦ οἴκου αὐτοῦ, καὶ πάντα τὰ ὑπάρχοντα αὐτοῦ, καὶ πάντα τὰ κτήνη, καὶ πάντα ὅσα ἐκτήσατο, καὶ πάντα ὅσα περιεποιήσατο ἐν γῇ Χαναάν· καὶ ἐπορεύθη Ἡσαῦ ἐκ τῆς γῆς Χαναὰν ἀπὸ προσώπου Ἰακὼβ τοῦ ἀδελφοῦ αὐτοῦ.
\vs{7}Ἦν γὰρ αὐτῶν τὰ ὑπάρχοντα πολλὰ, τοῦ οἰκεῖν ἅμα· καὶ οὐκ ἠδύνατο ἡ γῆ τῆς παροικήσεως αὐτῶν φέρειν αὐτοὺς, ἀπὸ τοῦ πλήθους τῶν ὑπαρχόντων αὐτῶν.
\vs{8}Κατῴκησε δὲ Ἡσαῦ ἐν τῷ ὄρει Σηείρ· Ἡσαῦ αὐτός ἐστιν Ἐδώμ.
\vs{9}Αὗται δὲ αἱ γενέσεις Ἡσαῦ πατρὸς Ἐδὼμ ἐν τῷ ὄρει Σηείρ.
\vs{10}Καὶ ταῦτα τὰ ὀνόματα τῶν υἱῶν Ἡσαῦ· Ἑλιφὰς υἱὸς Ἀδὰς γυναικὸς Ἡσαῦ· καὶ Ῥαγουὴλ υἱὸς Βασεμὰθ γυναικὸς Ἡσαῦ.
\vs{11}Ἐγένοντο δὲ Ἑλιφὰς υἱοὶ, Θαιμὰν, Ὠμὰρ, Σωφὰρ, Γοθὼμ, καὶ Κενέζ.
\vs{12}Θαμνὰ δὲ ἦν παλλακὴ Ἑλιφὰς τοῦ υἱοῦ Ἡσαῦ· καὶ ἔτεκε τῷ Ἑλιφὰς τὸν Ἀμαλήκ· οὗτοι υἱοὶ Ἀδὰς γυναικὸς Ἡσαῦ.
\vs{13}Οὗτοι δὲ υἱοὶ Ῥαγουὴλ, Ναχὼθ, Ζαρὲ, Σομὲ, καὶ Μοζέ· οὗτοι ἦσαν υἱοὶ Βασεμὰθ γυναικὸς Ἡσαῦ.
\vs{14}Οὗτοι δὲ υἱοὶ Ὀλιβεμὰς θυγατρὸς Ἀνὰ τοῦ υἱοῦ Σεβεγὼν, γυναικὸς Ἡσαῦ· ἔτεκε δὲ τῷ Ἡσαῦ τὸν Ἰεοὺς, καὶ τὸν Ἰεγλὸμ, καὶ τὸν Κορέ.
\vs{15}Οὗτοι ἡγεμόνες υἱοὶ Ἡσαῦ· υἱοὶ Ἑλιφὰς πρωτοτόκου Ἡσαῦ· ἡγεμὼν Θαιμὰν, ἡγεμὼν Ὠμὰρ, ἡγεμὼν Σωφὰρ, ἡγεμὼν Κενὲζ,
\vs{16}ἡγεμὼν Κορὲ, ἡγεμὼν Γοθὼμ, ἡγεμὼν Ἀμαλήκ· οὗτοι ἡγεμόνες Ἑλιφὰς ἐν γῇ Ἰδουμαίᾳ· οὗτοι υἱοὶ Ἀδάς.
\vs{17}Καὶ οὗτοι υἱοὶ Ῥαγουὴλ υἱοῦ Ἡσαῦ· ἡγεμὼν Ναχὼθ, ἡγεμὼν Ζαρὲ, ἡγεμὼν Σομὲ, ἡγεμὼν Μοζέ· οὗτοι ἡγεμόνες Ῥαγουὴλ ἐν γῇ Ἐδώμ· οὗτοι υἱοὶ Βασεμὰθ γυναικὸς Ἡσαῦ.
\vs{18}Οὗτοι δὲ υἱοὶ Ὀλιβεμὰς γυναικὸς Ἡσαῦ· ἡγεμὼν Ἰεοὺς, ἡγεμὼν Ἰεγλὸμ, ἡγεμὼν Κορέ· οὗτοι ἡγεμόνες Ὀλιβεμὰς θυγατρὸς Ἀνὰ γυναικὸς Ἡσαῦ.
\vs{19}Οὗτοι υἱοὶ Ἡσαῦ, καὶ οὗτοι ἡγεμόνες αὐτῶν· οὗτοί εἰσιν υἱοὶ Ἐδώμ.
\vs{20}Οὗτοι δὲ υἱοὶ Σηεὶρ τοῦ Χοῤῥαίου, τοῦ κατοικοῦντος τὴν γῆν· Λωτὰν, Σωβὰλ, Σεβεγὼν, Ἀνὰ,
\vs{21}καὶ Δησὼν, καὶ Ἀσὰρ, καὶ Ῥισών· οὗτοι ἡγεμόνες τοῦ Χοῤῥαίου, τοῦ υἱοῦ Σηεὶρ ἐν τῇ γῇ Ἐδώμ.
\vs{22}Ἐγένοντο δὲ υἱοὶ Λωτάν· Χοῤῥὶ, καὶ Αἱμάν· ἀδελφὴ δὲ Λωτὰν, Θαμνά.
\vs{23}Οὗτοι δὲ υἱοὶ Σωβάλ· Γωλὰμ, καὶ Μαναχὰθ, καὶ Γαιβὴλ, καὶ Σωφὰρ, καὶ Ὠμάρ.
\vs{24}Καὶ οὗτοι υἱοὶ Σεβεγὼν, Ἀϊὲ, καὶ Ἀνά· οὗτός ἐστιν Ἀνὰ, ὃς εὗρε τὸν Ἰαμεὶν ἐν τῇ ἐρήμῳ, ὅτε ἔνεμε τὰ ὑποζύγια Σεβεγὼν τοῦ πατρὸς αὐτοῦ·
\vs{25}Οὗτοι δὲ υἱοὶ Ἀνά· Δησὼν, καὶ Ὀλιβεμὰ θυγάτηρ Ἀνά.
\vs{26}Οὗτοι δὲ υἱοὶ Δησών· Ἀμαδὰ, καὶ Ἀσβὰν, καὶ Ἰθρὰν, καὶ Χαῤῥάν.
\vs{27}Οὗτοι δὲ υἱοὶ Ἀσάρ· Βαλαὰμ, καὶ Ζουκὰμ, καὶ Ἰουκάμ.
\vs{28}Οὗτοι δὲ υἱοὶ Ῥισὼν, Ὧς, καὶ Ἀράν.
\vs{29}Οὗτοι δὲ ἡγεμόνες Χοῤῥί· ἡγεμὼν Λωτὰν, ἡγεμὼν Σωβὰλ, ἡγεμὼν Σεβεγὼν, ἡγεμὼν Ἀνὰ,
\vs{30}ἡγεμὼν Δησὼν, ἡγεμὼν Ἀσὰρ, ἡγεμὼν Ῥισών· οὗτοι ἡγεμόνες Χοῤῥὶ ἐν ταῖς ἡγεμονίαις αὐτῶν ἐν γῇ Ἐδώμ.

\vs{31}Καὶ οὗτοι οἱ βασιλεῖς οἱ βασιλεύσαντες ἐν Ἐδὼμ, πρὸ τοῦ βασιλεῦσαι βασιλέα ἐν Ἰσραήλ.
\vs{32}Καὶ ἐβασίλευσεν ἐν Ἐδὼμ Βαλὰκ υἱὸς Βεώρ· καὶ ὄνομα τῇ πόλει αὐτοῦ, Δενναβά.
\vs{33}Ἀπέθανε δὲ Βαλὰκ, καὶ ἐβασίλευσεν ἀντʼ αὐτοῦ Ἰωβὰβ υἱὸς Ζαρὰ ἐκ Βοσόῤῥας.
\vs{34}Ἀπέθανε δὲ Ἰωβὰβ, καὶ ἐβασίλευσεν ἀντʼ αὐτοῦ Ἀσὼμ ἐκ τῆς γῆς Θαιμανών.
\vs{35}Ἀπέθανε δὲ Ἀσὼμ, καὶ ἐβασίλευσεν ἀντʼ αὐτοῦ Ἀδὰδ υἱὸς Βαρὰδ ὁ ἐκκόψας Μαδιὰμ ἐν τῷ πεδίῳ Μωάβ· καὶ ὄνομα τῇ πόλει αὐτοῦ Γετθαίμ.
\vs{36}Ἀπέθανε δὲ Ἀδὰδ, καὶ ἐβασίλευσεν ἀντʼ αὐτοῦ Σαμαδὰ ἐκ Μασσεκκάς.
\vs{37}Ἀπέθανε δὲ Σαμαδὰ, καὶ ἐβασίλευσεν ἀντʼ αὐτοῦ Σαοὺλ ἐκ Ῥοωβὼθ τῆς παρὰ ποταμόν.
\vs{38}Ἀπέθανε δὲ Σαοὺλ, καὶ ἐβασίλευσεν ἀντʼ αὐτοῦ Βαλλενὼν υἱὸς Ἀχοβώρ.
\vs{39}Ἀπέθανε δὲ Βαλλενὼν υἱὸς Ἀχοβὼρ, καὶ ἐβασίλευσεν ἀντʼ αὐτοῦ Ἀρὰδ υἱὸς Βαράδ· καὶ ὄνομα τῇ πόλει αὐτοῦ Φογώρ· ὄνομα δὲ τῇ γυναικὶ αὐτοῦ Μετεβεὴλ, θυγάτηρ Ματραῒθ, υἱοῦ Μαιζοώβ.
\vs{40}Ταῦτα τὰ ὀνόματα τῶν ἡγεμόνων Ἡσαῦ, ἐν ταῖς φυλαῖς αὐτῶν, κατὰ τόπον αὐτῶν, ἐν ταῖς χώραις αὐτῶν, καὶ ἐν τοῖς ἔθνεσιν αὐτῶν· ἡγεμὼν Θαμνὰ, ἡγεμὼν Γωλὰ, ἡγεμὼν Ἰεθὲρ,
\vs{41}ἡγεμὼν Ὁλιβεμὰς, ἡγεμὼν Ἡλὰς, ἡγεμὼν Φινὼν,
\vs{42}ἡγεμὼν Κενὲζ, ἡγεμὼν Θαιμὰν, ἡγεμὼν Μαζὰρ,
\vs{43}ἡγεμὼν Μαγεδιὴλ, ἡγεμὼν Ζαφωίν· οὗτοι ἡγεμόνες Ἐδὼμ, ἐν ταῖς κατῳκοδομημέναις ἐν τῇ γῇ τῆς κτήσεως αὐτῶν· οὗτος Ἡσαῦ πατὴρ Ἐδώμ.

\ch{37}
Κατῴκει δὲ Ἰακὼβ ἐν τῇ γῇ, οὗ παρῴκησεν ὁ πατὴρ αὐτοῦ ἐν γῇ Χαναάν· αὗται δὲ αἱ γενέσεις Ἰακώβ.
\vs{2}Ἰωσὴφ δὲ δέκα καὶ ἑπτὰ ἐτῶν ἦν, ποιμαίνων τὰ πρόβατα τοῦ πατρὸς αὐτοῦ μετὰ τῶν ἀδελφῶν αὐτοῦ, ὢν νέος, μετὰ τῶν υἱῶν Βαλλᾶς, καὶ μετὰ τῶν υἱῶν Ζελφᾶς, τῶν γυναικῶν τοῦ πατρὸς αὐτοῦ· κατήνεγκαν δὲ Ἰωσὴφ ψόγον πονηρὸν πρὸς Ἰσραὴλ τὸν πατέρα αὐτῶν.
\vs{3}Ἰακὼβ δὲ ἠγάπα τὸν Ἰωσὴφ παρὰ πάντας τοὺς υἱοὺς αὐτοῦ, ὅτι υἱὸς γήρως ἦν αὐτῷ· ἐποίησε δὲ αὐτῷ χιτῶνα ποικίλον.
\vs{4}Ἰδόντες δὲ οἱ ἀδελφοὶ αὐτοῦ, ὅτι αὐτὸν ὁ πατὴρ φιλεῖ ἐκ πάντων τῶν υἱῶν αὐτοῦ, ἐμίσησαν αὐτὸν, καὶ οὐκ ἠδύναντο λαλεῖν αὐτῷ οὐδὲν εἰρηνικόν.
\vs{5}Ἐνυπνιασθεὶς δὲ Ἰωσὴφ ἐνύπνιον, ἀπήγγειλεν αὐτὸ τοῖς ἀδελφοῖς αὐτοῦ.
\vs{6}Καὶ εἶπεν αὐτοῖς, ἀκούσατε τοῦ ἐνυπνίου τούτου, οὗ ἐνυπνιάσθην.
\vs{7}Ὤμην ὑμᾶς δεσμεύειν δράγματα ἐν μέσῳ τῷ πεδίῳ· καὶ ἀνέστη τὸ ἐμὸν δράγμα, καὶ ὠρθώθη· περιστραφέντα δὲ τὰ δράγματα ὑμῶν, προσεκύνησαν τὸ ἐμὸν δράγμα.
\vs{8}Εἶπαν δὲ αὐτῷ οἱ ἀδελφοὶ αὐτοῦ, μὴ βασιλεύων βασιλεύσεις ἐφʼ ἡμᾶς, ἢ κυριεύων κυριεύσεις ἡμῶν, καὶ προσέθεντο ἔτι μισεῖν αὐτὸν ἕνεκεν τῶν ἐνυπνίων αὐτοῦ, καὶ ἕνεκεν τῶν ῥημάτων αὐτοῦ.
\vs{9}Εἶδε δὲ ἐνύπνιον ἕτερον, καὶ διηγήσατο αὐτὸ τῷ πατρὶ αὐτοῦ, καὶ τοῖς ἀδελφοῖς αὐτοῦ· καὶ εἶπεν, ἰδοὺ ἐνυπνιασάμην ἐνύπνιον ἕτερον· ὥσπερ ὁ ἥλιος, καὶ ἡ σελήνη, καὶ ἕνδεκα ἀστέρες προσεκύνουν με.
\vs{10}Καὶ ἐπετίμησεν αὐτῷ ὁ πατὴρ αὐτοῦ, καὶ εἶπεν αὐτῷ, τί τὸ ἐνύπνιον τοῦτο, ὃ ἐνυπνιάσθης; ἆρά γε ἐλθόντες ἐλευσόμεθα ἐγώ τε καὶ ἡ μήτηρ σου καὶ οἱ ἀδελφοί σου προσκυνῆσαί σοι ἐπὶ τὴν γῆν;
\vs{11}Ἐζήλωσαν δὲ αὐτὸν οἱ ἀδελφοὶ αὐτοῦ· ὁ δὲ πατὴρ αὐτοῦ διετήρησε τὸ ῥῆμα.
\vs{12}Ἐπορεύθησαν δὲ οἱ ἀδελφοὶ αὐτοῦ βόσκειν τὰ πρόβατα τοῦ πατρὸς αὐτῶν εἰς Συχέμ.
\vs{13}Καὶ εἶπεν Ἰσραὴλ πρὸς Ἰωσὴφ, οὐχὶ οἱ ἀδελφοί σου ποιμαίνουσιν εἰς Συχέμ; δεῦρο ἀποστείλω σε πρὸς αὐτούς· εἶπε δὲ αὐτῷ, ἰδοὺ ἐγώ.
\vs{14}Εἶπε δὲ αὐτῷ Ἰσραὴλ, πορευθεὶς ἴδε, εἰ ὑγιαίνουσιν οἱ ἀδελφοί σου, καὶ τὰ πρόβατα, καὶ ἀνάγγειλόν μοι· καὶ ἀπέστειλεν αὐτὸν ἐκ τῆς κοιλάδος τῆς Χεβρών· καὶ ἦλθεν εἰς Συχέμ.
\vs{15}Καὶ εὗρεν αὐτὸν ἄνθρωπος πλανώμενον ἐν τῷ πεδίῳ· ἠρώτησε δὲ αὐτὸν ὁ ἄνθρωπος, λέγων, τί ζητεῖς;
\vs{16}Ὁ δὲ εἶπε, τοὺς ἀδελφούς μου ζητῶ· ἀπάγγειλόν μοι ποῦ βόσκουσιν.
\vs{17}Εἶπε δὲ αὐτῷ ὁ ἄνθρωπος, ἀπῄρκασιν ἐντεῦθεν· ἤκουσα γὰρ αὐτῶν λεγόντων, πορευθῶμεν εἰς Δωθαείμ· καὶ ἐπορεύθη Ἰωσὴφ κατόπισθε τῶν ἀδελφῶν αὐτοῦ, καὶ εὗρεν αὐτοὺς ἐν Δωθαείμ.

\vs{18}Προεῖδον δὲ αὐτὸν μακρόθεν πρὸ τοῦ ἐγγίσαι αὐτὸν πρὸς αὐτούς· καὶ ἐπονηρεύοντο τοῦ ἀποκτεῖναι αὐτόν.
\vs{19}Εἶπε δὲ ἕκαστος πρὸς τὸν ἀδελφὸν αὐτοῦ, ἰδοὺ ὁ ἐνυπνιαστὴς ἐκεῖνος ἔρχεται.
\vs{20}Νῦν οὖν δεῦτε ἀποκτείνωμεν αὐτὸν, καὶ ῥίψωμεν αὐτὸν εἰς ἕνα τῶν λάκκων· καὶ ἐροῦμεν, θηρίον πονηρὸν κατέφαγεν αὐτόν· καὶ ὀψόμεθα, τί ἔσται τὰ ἐνύπνια αὐτοῦ.
\vs{21}Ἀκούσας δὲ Ῥουβὴν, ἐξείλετο αὐτὸν ἐκ τῶν χειρῶν αὐτῶν· καὶ εἶπεν, οὐ πατάξωμεν αὐτὸν εἰς ψυχήν.
\vs{22}Εἶπε δὲ αὐτοῖς Ῥουβὴν, μὴ ἐκχέητε αἷμα· ἐμβάλλετε αὐτὸν εἰς ἕνα τῶν λάκκων τούτων τῶν ἐν τῇ ἐρήμῳ, χεῖρα δὲ μὴ ἐπενέγκητε αὐτῷ· ὅπως ἐξέληται αὐτὸν ἐκ τῶν χειρῶν αὐτῶν, καὶ ἀποδῷ αὐτὸν τῷ πατρὶ αὐτοῦ.
\vs{23}Ἐγένετο δὲ ἡνίκα ἦλθεν Ἰωσὴφ πρὸς τοὺς ἀδελφοὺς αὐτοῦ, ἐξέδυσαν Ἰωσὴφ τὸν χιτῶνα τὸν ποικίλον τὸν περὶ αὐτόν.
\vs{24}Καὶ λαβόντες αὐτὸν, ἔῤῥιψαν εἰς τὸν λάκκον· ὁ δὲ λάκκος κενὸς, ὕδωρ οὐκ εἶχε.
\vs{25}Ἐκάθισαν δὲ φαγεῖν ἄρτον· καὶ ἀναβλέψαντες τοῖς ὀφθαλμοῖς εἶδον, καὶ ἰδοὺ ὁδοιπόροι Ἰσμαηλῖται ἤρχοντο ἐκ Γαλαάδ· καὶ αἱ κάμηλοι αὐτῶν ἔγεμον θυμιαμάτων καὶ ῥητίνης καὶ στακτῆς. ἐπορεύοντο δὲ καταγαγεῖν εἰς Αἴγυπτον.

\vs{26}Εἶπε δὲ Ἰούδας πρὸς τοὺς ἀδελφοὺς αὐτοῦ, τί χρήσιμον, ἐὰν ἀποκτείνωμεν τὸν ἀδελφὸν ἡμῶν, καὶ κρύψωμεν τὸ αἷμα αὐτοῦ;
\vs{27}Δεῦτε ἀποδώμεθα αὐτὸν τοῖς Ἰσμαηλίταις τούτοις· αἱ δὲ χεῖρες ἡμῶν μὴ ἔστωσαν ἐπʼ αὐτὸν, ὅτι ἀδελφὸς ἡμῶν καὶ σὰρξ ἡμῶν ἐστίν. Ἤκουσαν δὲ οἱ ἀδελφοὶ αὐτοῦ.
\vs{28}Καὶ παρεπορεύοντο οἱ ἄνθρωποι οἱ Μαδιηναῖοι ἔμποροι, καὶ ἐξείλκυσαν καὶ ἀνεβίβασαν τὸν Ἰωσὴφ ἐκ τοῦ λάκκου· καὶ ἀπέδοντο τὸν Ἰωσὴφ τοῖς Ἰσμαηλίταις εἴκοσι χρυσῶν. Καὶ κατήγαγον τὸν Ἰωσὴφ εἰς Αἴγυπτον.
\vs{29}Ἀνέστρεψε δὲ Ῥουβὴν ἐπὶ τὸν λάκκον, καὶ οὐχ ὁρᾷ τὸν Ἰωσὴφ ἐν τῷ λάκκῳ· καὶ διέῤῥηξε τὰ ἱμάτια αὐτοῦ.
\vs{30}Καὶ ἐπέστρεψε πρὸς τοὺς ἀδελφοὺς αὐτοῦ, καὶ εἶπε, τὸ παιδάριον οὐκ ἔστιν· ἐγὼ δὲ ποῦ πορεύομαι ἔτι;
\vs{31}Λαβόντες δὲ τὸν χιτῶνα τοῦ Ἰωσὴφ, ἔσφαξαν ἔριφον αἰγῶν, καὶ ἐμόλυναν τὸν χιτῶνα τῷ αἵματι.
\vs{32}Καὶ ἀπέστειλαν τὸν χιτῶνα τὸν ποικίλον, καὶ εἰσήνεγκαν τῷ πατρὶ αὐτῶν· καὶ εἶπαν, τοῦτον εὕρομεν, ἐπίγνωθι εἰ χιτὼν τοῦ υἱοῦ σου ἐστὶν, ἢ οὔ.
\vs{33}Καὶ ἐπέγνω αὐτὸν, καὶ εἶπε, χιτὼν τοῦ υἱοῦ μου ἐστί· θηρίον πονηρὸν κατέφαγεν αὐτόν· θηρίον ἥρπασε τὸν Ἰωσήφ.
\vs{34}Διέῤῥηξε δὲ Ἰακὼβ τὰ ἱμάτια αὐτοῦ, καὶ ἐπέθετο σάκκον ἐπὶ τὴν ὀσφῦν αὐτοῦ, καὶ ἐπένθει τὸν υἱὸν αὐτοῦ ἡμέρας πολλάς.
\vs{35}Συνήχθησαν δὲ πάντες οἱ υἱοὶ αὐτοῦ καὶ αἱ θυγατέρες, καὶ ἦλθον παρακαλέσαι αὐτόν· καὶ οὐκ ἤθελε παρακαλεῖσθαι, λέγων, ὅτι καταβήσομαι πρὸς τὸν υἱόν μου πενθῶν εἰς ᾅδου· καὶ ἔκλαυσεν αὐτὸν ὁ πατὴρ αὐτοῦ.
\vs{36}Οἱ δὲ Μαδιηναῖοι ἀπέδοντο τὸν Ἰωσὴφ εἰς Αἴγυπτον τῷ Πετεφρῇ τῷ σπάδοντι Φαραὼ ἀρχιμαγείρῳ.

\ch{38}
Ἐγένετο δὲ ἐν τῷ καιρῷ ἐκείνῳ, κατέβη Ἰούδας ἀπὸ τῶν ἀδελφῶν αὐτοῦ, καὶ ἀφίκετο ἕως πρὸς ἄνθρωπον τινὰ Ὀδολλαμίτην, ᾧ ὄνομα Εἰράς.
\vs{2}Καὶ εἶδεν ἐκεῖ Ἰούδας θυγατέρα ἀνθρώπου Χαναναίου, ᾗ ὄνομα Σαυά· καὶ ἔλαβεν αὐτὴν, καὶ εἰσῆλθε πρὸς αὐτήν.
\vs{3}Καὶ συλλαβοῦσα ἔτεκεν υἱὸν, καὶ ἐκάλεσε τὸ ὄνομα αὐτοῦ, Ἤρ.
\vs{4}Καὶ συλλαβοῦσα ἔτεκεν υἱὸν ἔτι, καὶ ἐκάλεσε τὸ ὄνομα αὐτοῦ, Αὐνάν.
\vs{5}Καὶ προσθεῖσα ἔτεκεν υἱὸν, καὶ ἐκάλεσε τὸ ὄνομα αὐτοῦ, Σηλώμ· αὕτη δὲ ἦν ἐν Χασβὶ, ἡνίκα ἔτεκεν αὐτούς.
\vs{6}Καὶ ἔλαβεν Ἰούδας γυναῖκα Ἢρ τῷ πρωτοτόκῳ αὐτοῦ, ᾗ ὄνομα Θάμαρ.
\vs{7}Ἐγένετο δὲ Ἢρ πρωτότοκος Ἰούδα πονηρὸς ἔναντι Κυρίου· καὶ ἀπέκτεινεν αὐτὸν ὁ Θεός.
\vs{8}Εἶπε δὲ Ἰούδας τῷ Αὐνάν· εἴσελθε πρὸς τὴν γυναῖκα τοῦ ἀδελφοῦ σου, καὶ ἐπιγάμβρευσαι αὐτὴν, καὶ ἀνάστησον σπέρμα τῷ ἀδελφῷ σου.
\vs{9}Γνοὺς δὲ Αὐνὰν, ὅτι οὐκ αὐτῷ ἔσται τὸ σπέρμα, ἐγένετο ὅταν εἰσήρχετο πρὸς τὴν γυναῖκα τοῦ ἀδελφοῦ αὐτου, ἐξέχεεν ἐπὶ τὴν γῆν, τοῦ μὴ δοῦναι σπέρμα τῷ ἀδελφῷ αὐτοῦ.
\vs{10}Πονηρὸν δὲ ἐφάνη ἐναντίον τοῦ Θεοῦ, ὅτι ἐποίησε τοῦτο· καὶ ἐθανάτωσε καὶ τοῦτον.

\vs{11}Εἶπε δὲ Ἰούδας Θάμαρ τῇ νύμφῃ αὐτοῦ, κάθου χήρα ἐν τῷ οἴκῳ τοῦ πατρός σου, ἕως μέγας γένηται Σηλὼμ ὁ υἱός μου· εἶπε γάρ, μή ποτε ἀποθάνῃ καὶ οὗτος, ὥσπερ καὶ οἱ ἀδελφοὶ αὐτοῦ. Ἀπελθοῦσα δὲ Θάμαρ ἐκάθητο ἐν τῷ οἴκῳ τοῦ πατρὸς αὐτῆς.
\vs{12}Ἐπληθύνθησαν δὲ αἱ ἡμέραι, καὶ ἀπέθανε Σαυὰ ἡ γυνὴ Ἰούδα· καὶ παρακληθεὶς Ἰούδας ἀνέβη ἐπὶ τοὺς κείροντας τὰ πρόβατα αὐτοῦ, αὐτὸς καὶ Εἰρὰς ὁ ποιμὴν αὐτοῦ ὁ Ὀδολλαμίτης εἰς Θαμνά.
\vs{13}Καὶ ἀπηγγέλε Θάμαρ τῇ νύμφῃ αὐτοῦ, λέγοντες, ἰδοὺ ὁ πενθερός σου ἀναβαίνει εἰς Θαμνὰ, κεῖραι τὰ πρόβατα αὐτοῦ.
\vs{14}Καὶ περιελομένη τὰ ἱμάτια τῆς χηρεύσεως ἀφʼ ἑαυτῆς, περιέβαλε τὸ θέριστρον, καὶ ἐκαλλωπίσατο, καὶ ἐκάθισε πρὸς ταῖς πύλαις Αἰνὰν, ἥ ἐστιν ἐν παρόδῳ Θαμνά· ἴδε γὰρ ὅτι μέγας γέγονε Σηλὼμ, αὐτὸς δὲ οὐκ ἔδωκεν αὐτὴν αὐτῷ γυναῖκα.
\vs{15}Καὶ ἰδὼν αὐτὴν Ἰούδας ἔδοξεν αὐτὴν πόρνην εἶναι· κατεκαλύψατο γὰρ τὸ πρόσωπον αὐτῆς καὶ οὐκ ἐπέγνω αὐτήν.
\vs{16}Ἐξέκλινε δὲ πρὸς αὐτὴν τὴν ὁδόν· καὶ εἶπεν αὐτῇ, ἔασόν με εἰσελθεῖν πρός σε· οὐ γὰρ ἔγνω, ὅτι νύμφη αὐτοῦ ἐστίν· ἡ δὲ εἶπε, τί μοι δώσεις, ἐὰν εἰσέλθῃς πρός με;
\vs{17}Ὁ δὲ εἶπεν, ἐγώ σοι ἀποστελλῶ ἔριφον αἰγῶν ἐκ τῶν προβάτων μον· ἡ δὲ εἶπεν, ἐὰν δῷς μοι ἀῤῥαβῶνα, ἕως τοῦ ἀποστεῖλαί σε.
\vs{18}Ὁ δὲ εἶπε, τίνα τὸν ἀῤῥαβῶνά σοι δώσω; ἡ δὲ εἶπε, τὸν δακτύλιόν σου, καὶ τὸν ὁρμίσκον, καὶ τὴν ῥάβδον τὴν ἐν τῇ χειρίσου. Καὶ ἔδωκεν αὐτῇ, καὶ εἰσῆλθε πρὸς αὐτήν· καὶ ἐν γαστρὶ ἔλαβεν ἐξ αὐτοῦ.
\vs{19}Καὶ ἀναστᾶσα ἀπῆλθε, καὶ περιείλετο τὸ θέριστρον αὐτῆς ἀφʼ ἑαυτῆς, καὶ ἐνεδύσατο τὰ ἱμάτια τῆς χηρεύσεως αὐτῆς.
\vs{20}Ἀπέστειλε δὲ Ἰούδας τὸν ἔριφον ἐξ αἰγῶν ἐν χειρὶ τοῦ ποιμένος αὐτοῦ τοῦ Ὀδολλαμείτου, κομίσασθαι παρὰ τῆς γυναικὸς τὸν ἀῤῥαβῶνα· καὶ οὐχ εὗρεν αὐτήν.
\vs{21}Ἐπηρώτησε δὲ τοὺς ἄνδρας τοὺς ἐκ τοῦ τόπου, ποῦ ἐστιν ἡ πόρνη ἡ γενομένη ἐν Αἰνὰν ἐπὶ τῆς ὁδοῦ; καὶ εἶπαν, οὐκ ἦν ἐνταῦθα πόρνη.
\vs{22}Καὶ ἀπεστράφη πρὸς Ἰούδαν, καὶ εἶπεν, οὐχ εὗρον· καὶ οἱ ἄνθρωποι οἱ ἐκ τοῦ τόπου λέγουσι, μὴ εἶναι ὧδε πόρνην.
\vs{23}Εἶπε δὲ Ἰούδας, ἐχέτω αὐτά· ἀλλὰ μή ποτε καταγελασθῶμεν· ἐγὼ μὲν ἀπέσταλκα τὸν ἔριφον τοῦτον, σὺ δὲ οὐχ εὕρηκας.
\vs{24}Ἐγένετο δὲ μετὰ τρίμηνον ἀνηγγέλη τῷ Ἰούδα, λέγοντες, ἐκπεπόρνευκε Θάμαρ ἡ νύμφη σου, καὶ ἰδοὺ ἐν γαστρὶ ἔχει ἐκ πορνείας· Εἶπε δὲ Ἰούδας, ἐξαγάγετε αὐτὴν, καὶ κατακαυθήτω.
\vs{25}Αὐτὴ δὲ ἀγομένη ἀπέστειλε πρὸς τὸν πενθερὸν αὐτὴς, λέγουσα, ἐκ τοῦ ἀνθρώπου οὕτινος ταῦτά ἐστιν, ἐγὼ ἐν γαστρὶ ἔχω· καὶ εἶπεν, ἐπίγνωθι τίνος ὁ δακτύλιος, καὶ ὁ ὁρμίσκος καὶ ἡ ῥάβδος αὕτη.
\vs{26}Ἐπέγνω δὲ Ἰούδας, καὶ εἶπε, δεδικαίωται Θάμαρ ἢ ἐγώ· οὗ ἕνεκεν οὐκ ἔδωκα αὐτὴν Σηλὼμ τῷ υἱῷ μου· Καὶ οὐ προσέθετο ἔτι τοῦ γνῶναι αὐτήν.
\vs{27}Ἐγένετο δὲ ἡνίκα ἔτικτε, καὶ τῇδε ἦν δίδυμα ἐν τῇ γαστρὶ αὐτῆς.
\vs{28}Ἐγένετο δὲ ἐν τῷ τίκτειν αὐτὴν, ὁ εἷς προεξήνεγκεν τὴν χεῖρα· λαβοῦσα δὲ ἡ μαῖα, ἔδησεν ἐπὶ τὴν χεῖρα αὐτοῦ κόκκινον, λέγουσα, οὗτος ἐξελεύσεται πρότερος.
\vs{29}Ὡς δὲ ἐπισυνήγαγε τὴν χεῖρα, καὶ εὐθὺς ἐξῆλθεν ὁ ἀδελφὸς αὐτοῦ· ἡ δὲ εἶπε, τί διεκόπη διὰ σὲ φραγμός; καὶ ἐκάλεσε τὸ ὄνομα αὐτοῦ, Φαρές.
\vs{30}Καὶ μετὰ τοῦτο ἐξῆλθεν ὁ ἀδελφὸς αὐτοῦ, ἐφʼ ᾧ ἦν ἐπὶ τῇ χειρὶ αὐτοῦ τὸ κόκκινον· καὶ ἐκάλεσε τὸ ὄνομα αὐτοῦ, Ζαρά.

\ch{39}
Ἰωσὴφ δὲ κατήχθη εἰς Αἴγυπτον· καὶ ἐκτήσατο αὐτὸν Πετεφρὴς ὁ εὐνοῦχος Φαραὼ, ὁ ἀρχιμάγειρος, ἀνὴρ Αἰγύπτιος, ἐκ χειρῶν τῶν Ἰσμαηλιτῶν, οἳ κατήγαγον αὐτὸν ἐκεῖ.
\vs{2}Καὶ ἦν Κύριος μετὰ Ἰωσήφ· καὶ ἦν ἀνὴρ ἐπιτυγχάνων· καὶ ἐγένετο ἐν τῷ οἴκῳ παρὰ τῷ κυρίῳ αὐτοῦ τῷ Αἰγυπτίῳ.
\vs{3}Ἤδει δὲ ὁ κύριος αὐτοῦ, ὅτι ὁ Κύριος ἦν μετʼ αὐτοῦ, καὶ ὅσα ἐὰν ποιῇ, Κύριος εὐοδοῖ ἐν ταῖς χερσὶν αὐτοῦ.
\vs{4}Καὶ εὗρεν Ἰωσὴφ χάριν ἐναντίον τοῦ κυρίου αὐτοῦ, καὶ εὐηρέστησεν αὐτῷ. Καὶ κατέστησε αὐτὸν ἐπὶ τοῦ οἴκου αὐτοῦ· καὶ πάντα ὅσα ἦν αὐτῷ, ἔδωκε διὰ χειρὸς Ἰωσήφ.
\vs{5}Ἐγένετο δὲ μετὰ τὸ καταστῆναι αὐτὸν ἐπὶ τοῦ οἴκου αὐτοῦ, καὶ ἐπὶ πάντα ὅσα ἦν αὐτῷ, καὶ ηὐλόγησε Κύριος τὸν οἶκον τοῦ Αἰγυπτίου διὰ Ἰωσήφ· καὶ ἐγενήθη εὐλογία Κυρίου ἐν πᾶσι τοῖς ὑπάρχουσιν αὐτῷ ἐν τῷ οἴκῳ, καὶ ἐν τῷ ἀγρῷ αὐτοῦ.
\vs{6}Καὶ ἐπέτρεψε πάντα ὅσα ἦν αὐτῷ, εἰς χεῖρας Ἰωσήφ· καὶ οὐκ ᾔδει τῶν καθʼ αὑτὸν οὐδὲν, πλὴν τοῦ ἄρτου, οὗ ἤσθιεν αὐτός. Καὶ ἦν Ἰωσὴφ καλὸς τῷ εἴδει, καὶ ὡραῖος τῇ ὄψει σφόδρα.
\vs{7}Καὶ ἐγένετο μετὰ τὰ ῥήματα ταῦτα, καὶ ἐπέβαλεν ἡ γυνὴ τοῦ κυρίου αὐτοῦ τοὺς ὀφθαλμοὺς αὐτῆς ἐπὶ Ἰωσήφ· καὶ εἶπεν, κοιμήθητι μετʼ ἐμοῦ.
\vs{8}Ὁ δὲ οὐκ ἤθελεν· εἶπε δὲ τῇ γυναικὶ τοῦ κυρίου αὐτοῦ, εἰ ὁ κύριός μου οὐ γινώσκει διʼ ἐμὲ οὐδὲν ἐν τῷ οἴκῳ αὐτοῦ, καὶ πάντα ὅσα ἐστὶν αὐτῷ ἔδωκεν εἰς τὰς χεῖράς μου,
\vs{9}καὶ οὐχ ὑπερέχει ἐν τῇ οἰκίᾳ ταύτῆ οὐθὲν ἐμοῦ, οὐδὲ ὑπεξῄρηται ἀπʼ ἐμοῦ οὐδὲν, πλὴν σοῦ, διὰ τὸ σὲ γυναῖκα αὐτοῦ εἶναι, καὶ πῶς ποιήσω τὸ ῥῆμα τὸ πονηρὸν τοῦτο, καὶ ἁμαρτήσομαι ἐναντίον τοῦ Θεοῦ;
\vs{10}Ἡνίκα δὲ ἐλάλει τῷ Ἰωσὴφ ἡμέραν ἐξ ἡμέρας, καὶ οὐχ ὑπήκουεν αὐτῇ καθεύδειν μετʼ αὐτῆς, τοῦ συγγενέσθαι αὐτῇ.
\vs{11}Ἐγένετο δὲ τοιαύτη τις ἡμέρα, καὶ εἰσῆλθεν Ἰωσὴφ εἰς τὴν οἰκίαν ποιεῖν τὰ ἔργα αὐτοῦ, καὶ οὐθεὶς ἦν τῶν ἐν τῇ οἰκίᾳ ἔσω.
\vs{12}Καὶ ἐπεσπάσατο αὐτὸν τῶν ἱματίων, λέγουσα, κοιμήθητι μετʼ ἐμοῦ· καὶ καταλιπὼν τὰ ἱμάτια αὐτοῦ ἐν ταῖς χερσὶν αὐτῆς ἔφυγε, καὶ ἐξῆλθεν ἔξω.
\vs{13}Καὶ ἐγένετο ὡς εἶδεν ὅτι καταλιπὼν τὰ ἱμάτια αὐτοῦ ἐν ταῖς χερσὶν αὐτῆς ἔφυγε, καὶ ἐξῆλθεν ἔξω,
\vs{14}καὶ ἐκάλεσε τοὺς ὄντας ἐν τῇ οἰκίᾳ, καὶ εἶπεν αὐτοῖς, λέγουσα, ἴδετε, εἰσήγαγε ἡμῖν παῖδα Ἐβραῖον, ἐμπαίζειν ἡμῖν· εἰσῆλθε πρός με, λέγων, κοιμήθητι μετʼ ἐμοῦ· καὶ ἐβόησα φωνῇ μεγάλῃ.
\vs{15}Ἐν δὲ τῷ ἀκοῦσαι αὐτὸν, ὅτι ὕψωσα τὴν φωνήν μου καὶ ἐβόησα, καταλιπὼν τὰ ἱμάτια αὐτοῦ παρʼ ἐμοὶ ἔφυγε, καὶ ἐξῆλθεν ἔξω.
\vs{16}Καὶ καταλιμπάνει τὰ ἱμάτια παρʼ ἑαυτῇ, ἕως ἦλθεν ὁ κύριος εἰς τὸν οἶκον αὐτοῦ.
\vs{17}Καὶ ἐλάλησεν αὐτῷ κατὰ τὰ ῥήματα ταῦτα, λέγουσα, εἰσῆλθε πρός με ὁ παῖς ὁ Ἑβραῖος, ὃν εἰσήγαγες πρὸς ἡμᾶς, ἐμπαῖξαί μοι· καὶ εἶπέ μοι, κοιμηθήσομαι μετὰ σοῦ.
\vs{18}Ὡς δὲ ἤκοῦσεν, ὅτι ὕψωσα τὴν φωνήν μου καὶ ἐβόησα, καταλιπὼν τὰ ἱμάτια αὐτοῦ παρʼ ἐμοὶ ἔφυγε, καὶ ἐξῆλθεν ἔξω.
\vs{19}Ἐγένετο δὲ, ὡς ἤκουσεν ὁ κύριος τὰ ῥήματα τῆς γυναικὸς αὐτοῦ, ὅσα ἐλάλησε πρὸς αὐτὸν, λέγουσα, οὕτως ἐποίησέ μοι ὁ παῖς σου, καὶ ἐθυμώθη ὀργῇ.

\vs{20}Καὶ λαβὼν ὁ κύριος Ἰωσὴφ, ἐνέβαλε αὐτὸν εἰς τὸ ὀχύρωμα, εἰς τὸν τόπον ἐν ᾧ οἱ δεσμῶται τοῦ βασιλέως κατέχονται ἐκεῖ ἐν τῷ ὀχυρώματι.
\vs{21}Καὶ ἦν Κύριος μετὰ Ἰωσὴφ, καὶ κατέχεεν αὐτοῦ ἔλεος· καὶ ἔδωκεν αὐτῷ χάριν ἐναντίον τοῦ ἀρχιδεσμοφύλακος.
\vs{22}Καὶ ἔδωκεν ὁ ἀρχιδεσμοφύλαξ τὸ δεσμωτήριον διὰ χειρὸς Ἰωσὴφ, καὶ πάντας τοὺς ἀπηγμένους ὅσοι ἐν τῷ δεσμωτηρίῳ, καὶ πάντα ὅσα ποιοῦσιν ἐκεῖ, αὐτὸς ἦν ποιῶν.
\vs{23}Οὐκ ἦν ὁ ἀρχιδεσμοφύλαξ τοῦ δεσμωτηρίου γινώσκον διʼ αὐτὸν οὐθέν· πάντα γὰρ ἦν διὰ χειρὸς Ἰωσὴφ, διὰ τὸ τὸν Κύριον μετʼ αὐτοῦ εἶναι· καὶ ὅσα αὐτὸς ἐποίει, ὁ Κύριος εὐώδο ἐν ταῖς χερσὶν αὐτοῦ.

\ch{40}
Ἐγένετο δὲ μετὰ τὰ ῥήματα ταῦτα, ἥμαρτεν ὁ ἀρχιοινοχόος τοῦ βασιλέως Αἰγύπτου, καὶ ὁ ἀρχισιτοποιὸς, τῷ κυρίῳ αὐτῶν βασιλεῖ Αἰγύπτου.
\vs{2}Καὶ ὠργίσθη Φαραὼ ἐπὶ τοῖς δυσὶν εὐνούχοις αὐτοῦ, ἐπὶ τῷ ἀρχιοινοχόῳ, καὶ ἐπὶ τῷ ἀρχισιτοποιῷ·
\vs{3}Καὶ ἔθετο αὐτοὺς ἐν φυλακῇ εἰς τὸ δεσμωτήριον, εἰς τὸν τόπον, οὗ Ἰωσὴφ ἀπῆκτο ἐκεῖ.
\vs{4}Καὶ συνέστησεν ὁ ἀρχιδεσμώτης τῷ Ἰωσὴφ αὐτούς· καὶ παρέστη αὐτοῖς· ἦσαν δὲ ἡμέρας ἐν τῇ φυλακῇ.
\vs{5}Καὶ εἶδον ἀμφότεροι ἐνύπνιον ἐν μιᾷ νυκτί· ἡ δὲ ὅρασις τοῦ ἐνυπνίου τοῦ ἀρχιοινοχόου καὶ ἀρχισιτοποιοῦ, οἳ ἦσαν τῷ βασιλεῖ Αἰγύπτου, οἱ ὄντες ἐν τῷ δεσμωτηρίῳ, ἦν αὕτη.
\vs{6}Εἰσῆλθε πρὸς αὐτοὺς Ἰωσὴφ τὸ πρωῒ, καὶ εἶδεν αὐτοὺς, καὶ ἦσαν τεταραγμένοι.
\vs{7}Καὶ ἠρώτα τοὺς εὐνούχους Φαραὼ, οἳ ἦσαν μετʼ αὐτοῦ ἐν τῇ φυλακῇ παρὰ τῷ κυρίῳ αὐτοῦ, λέγων, τί ὅτι τὰ πρόσωπα ὑμῶν σκυθρωπὰ σήμερον;
\vs{8}Οἱ δὲ εἶπαν αὐτῷ, ἐνύπνιον εἴδομεν, καὶ ὁ συγκρίνων οὐκ ἔστιν αὐτό· εἶπε δὲ αὐτοῖς Ἰωσὴφ, οὐχὶ διὰ τοῦ Θεοῦ ἡ διασάφησις αὐτῶν ἐστι; διηγήσασθε οὖν μοὶ.
\vs{9}Καὶ διηγήσατο ὁ ἀρχιοινοχόος τὸ ἐνύπνιον αὐτοῦ τῷ Ἰωσήφ· καὶ εἶπεν, ἐν τῷ ὕπνῳ μου ἦν ἄμπελος ἐναντίον μου.
\vs{10}Ἐν δὲ τῇ ἀμπέλῳ τρεῖς πυθμένες, καὶ αὐτὴ θάλλουσα, ἀνενηνοχυῖα βλαστούς· πέπειροι οἱ βότρυες σταφυλῆς.
\vs{11}Καὶ τὸ ποτήριον Φαραὼ ἐν τῇ χειρί μου· καὶ ἔλαβον τὴν σταφυλὴν, καὶ ἐξέθλιψα αὐτὴν εἰς τὸ ποτήριον, καὶ ἔδωκα τὸ ποτήριον εἰς τὴν χεῖρα Φαραώ.
\vs{12}Καὶ εἶπεν αὐτῷ Ἰωσὴφ, τοῦτο ἡ σύγκρίσις αὐτοῦ· οἱ τρεῖς πυθμένες, τρεῖς ἡμέραι εἰσίν.
\vs{13}Ετι τρεῖς ἡμέραι, καὶ μνησθήσεται Φαραὼ τῆς ἀρχῆς σου, καὶ ἀποκαταστήσει σε ἐπὶ τὴν ἀρχιοινοχοΐαν σου, καὶ δώσεις τὸ ποτήριον Φαραὼ εἰς τὴν χεῖρα αὐτοῦ κατὰ τὴν ἀρχήν σου τὴν προτέραν, ὡς ἦσθα οἰνοχοῶν.
\vs{14}Ἀλλὰ μνήσθητί μου διὰ σεαυτοῦ, ὅταν εὖ γενηταί σοι· καὶ ποιήσεις ἐν ἐμοὶ ἔλεος· καὶ μνησθήσῃ περὶ ἐμοῦ πρὸς Φαραὼ, καὶ ἐξάξεις με ἐκ τοῦ ὀχυρώματος τούτου.
\vs{15}Ὅτι κλοπῇ ἐκλάπην ἐκ γῆς Ἑβραίων, καὶ ὧδε οὐκ ἐποίησα οὐδὲν, ἀλλʼ ἐνέβαλόν με εἰς τὸν λάκκον τοῦτον.
\vs{16}Καὶ εἶδεν ὁ ἀρχισιτοποιὸς ὅτι ὀρθῶς συνέκριεν· καὶ εἶπε τῷ Ἰωσὴφ, κᾀγὼ εἶδον ἐνύπνιον· καὶ ᾤμην τρία κανᾶ χονδριτῶν αἴρειν ἐπὶ τῆς κεφαλῆς μου·
\vs{17}Ἐν δὲ κανῷ τῷ ἐπάνω ἀπὸ πάντων τῶν γενῶν, ὧν Φαραὼ ἐσθίει, ἔργον σιτοποιοῦ, καὶ τὰ πετεινὰ τοῦ οὐρανου κατήσθιεν αὐτὰ ἀπὸ τοῦ κανοῦ τοῦ ἐπάνω τῆς κεφαλῆς μου.
\vs{18}Ἀποκριθεὶς δὲ Ἰωσὴφ εἶπεν αὐτῷ, αὕτη ἡ σύγκρισις αὐτοῦ· τὰ τρία κανᾶ, τρεῖς ἡμέραι εἰσίν·
\vs{19}Ἔτι τριῶν ἡμερῶν, καὶ ἀφελεῖ Φαραὼ τὴν κεφαλήν σου ἀπὸ σου· καὶ κρεμάσει σε ἐπὶ ξύλου, καὶ φάγεται τὰ ὄρνεα τοῦ οὐρανοῦ τὰς σάρκας σου ἀπὸ σοῦ.
\vs{20}Ἐγένετο δὲ ἐν τῇ ἡμέρᾳ τῇ τρίτῃ, ἡμέρα γενέσεως ἦν Φαραὼ, καὶ ἐποίει πότον πᾶσι τοῖς παισὶν αὐτοῦ· καὶ ἐμνήσθη τῆς ἀρχῆς τοῦ οἰνοχόου καὶ τῆς ἀρχῆς τοῦ σιτοποιοῦ ἐν μέσῳ τῶν παίδων αὐτοῦ.
\vs{21}Καὶ ἀποκατέστησε τὸν ἀρχιοινοχόον ἐπὶ τὴν ἀρχὴν αὐτοῦ· καὶ ἔδωκε τὸ ποτήριον εἰς τὴν χεῖρα Φαραώ.
\vs{22}Τὸν δὲ ἀρχισιτοποιὸν ἐκρέμασεν, καθὰ συνέκρινεν αὐτοῖς Ἰωσήφ.
\vs{23}Καὶ οὐκ ἐμνήσθη ὁ ἀρχιοινοχόος τοῦ Ἰωσὴφ, ἀλλαʼ ἐπελάθετο αὐτοῦ.

\ch{41}
Ἐγένετο δὲ μετὰ δύο ἔτη ἡμερῶν, Φαραὼ εἶδεν ἐνύπνιον· ᾤετο ἑστάναι ἐπὶ τοῦ ποταμοῦ.
\vs{2}Καὶ ἰδοὺ ὥσπερ ἐκ τοῦ ποταμοῦ ἀνέβαινον ἐπτὰ βόες, καλαὶ τῷ εἴδει, καὶ ἐκλεκταὶ ταῖς σαρξὶ, καὶ ἐβόσκοντο ἐν τῷ Ἄχει.
\vs{3}Ἄλλαι δὲ ἑπτὰ βόες ἀνέβαινον μετὰ ταύτας ἐκ τοῦ ποταμοῦ, αἰσχραὶ τῷ εἴδει, καὶ λεπταὶ ταῖς σαρξὶ, καὶ ἐνέμοντο παρὰ τὰς βόας ἐπὶ τὸ χεῖλος τοῦ ποταμοῦ.
\vs{4}Καὶ κατέφαγον αἱ ἑπτὰ βόες αἱ αἰσχραὶ καὶ λεπταὶ ταῖς σαρξὶ τὰς ἑπτὰ βόας τὰς καλὰς τῷ εἴδει καὶ τὰς ἐκλεκτὰς ταῖς σαρξί· ἠγέρθη δὲ Φαραώ.
\vs{5}Καὶ ἐνυπνιάσθη τὸ δεύτερον· καὶ ἰδοὺ ἑπτὰ στάχυες ἀνέβαινον ἐν τῷ πυθμένι ἑνὶ ἐκλεκτοὶ καὶ καλοί.
\vs{6}Καὶ ἰδοὺ ἑπτὰ στάχυες λεπτοὶ καὶ ἀνεμόφθοροι ἀνεφύοντο μετʼ αὐτούς.
\vs{7}Καὶ κατέπιον οἱ ἑπτὰ στάχυες οἱ λεπτοὶ καὶ ἀνεμόφθοροι τοὺς ἑπτὰ στάχυας τοὺς ἐκλεκτοὺς καὶ τοὺς πλήρεις· ἠγέρθη δὲ Φαραὼ, καὶ ἦν ἐνύπνιον.
\vs{8}Ἐγένετο δὲ πρωῒ, καὶ ἐταράχθη ἡ ψυχὴ αὐτοῦ, καὶ ἀποστείλας ἐκάλεσε πάντας τοὺς ἐξηγητὰς Αἰγύπτου, καὶ πάντας τοὺς σοφοὺς αὐτῆς· καὶ διηγήσατο αὐτοῖς Φαραὼ τὸ ἐνύπνιον αὐτοῦ, καὶ οὐκ ἦν ὁ ἀπαγγέλλων αὐτὸ τῷ Φαραώ.
\vs{9}Καὶ ἐλάλησεν ὁ ἀρχιοινοχόος πρὸς Φαραὼ, λέγων, τὴν ἁμαρτίαν μου ἀναμιμνήσκω σήμερον.
\vs{10}Φαραὼ ὠργίσθη τοῖς παισὶν αὐτοῦ, καὶ ἔθετο ἡμᾶς ἐν φυλακῇ, ἐν τῷ οἴκῳ τοῦ ἀρχιμαγείρου, ἐμέ τε καὶ τὸν ἀρχισιτοποιόν.
\vs{11}Καὶ εἴδομεν ἐνύπνιον ἀμφότεροι ἐν νυκτὶ μιᾷ ἐγὼ καὶ αὐτὸς, ἕκαστος κατὰ τὸ αὐτοῦ ἐνύπνιον εἴδομεν.
\vs{12}Ἦν δὲ ἐκεῖ μεθʼ ἡμῶν νεανίσκος παῖς Ἑβραῖος τοῦ ἀρχιμαγείρου, καὶ διηγησάμεθα αὐτῷ, καὶ συνέκρινεν ἡμῖν.
\vs{13}Ἐγενήθη δὲ, καθὼς συνέκρινεν ἡμῖν οὕτω καὶ συνέβη, ἐμέ τε ἀποκατασταθῆναι ἐπὶ τὴν ἀρχήν μου, ἐκεῖνον δὲ κρεμασθῆναι.
\vs{14}Ἀποστείλας δὲ Φαραὼ ἐκάλεσε τὸν Ἰωσήφ· καὶ ἐξήγαγον αὐτὸν ἀπὸ τοῦ ὀχυρώματος, καὶ ἐξύρησαν αὐτὸν, καὶ ἤλλαξαν τὴν στολὴν αὐτοῦ· καὶ ἦλθε πρὸς Φαραώ.
\vs{15}Εἶπε δὲ Φαραὼ πρὸς Ἰωσὴφ, ἐνύπνιον ἑώρακα, καὶ ὁ συγκρίνων οὐκ ἔστιν αὐτό· ἐγὼ δὲ ἀκήκοα περὶ σοῦ λεγόντων, ἀκούσαντά σε ἐνύπνια, συγκρῖναι αὐτά.
\vs{16}Ἀποκριθεὶς δὲ Ἰωσὴφ τῷ Φαραὼ εἶπεν, ἄνευ τοῦ Θεοῦ οὐκ ἀποκριθήσεται τὸ σωτήριον Φαραώ.
\vs{17}Ἐλάλησε δὲ Φαραὼ τῷ Ἰωσὴφ, λέγων, ἐν τῷ ὕπνῳ μου ᾤμην ἑστάναι παρὰ τὸ χεῖλος τοῦ ποταμοῦ.
\vs{18}Καὶ ὥσπερ ἐκ τοῦ ποταμοῦ ἀνέβαινον ἑπτὰ βόες καλαὶ τῷ εἴδει καὶ ἐκλεκταὶ ταῖς σαρξὶ, καὶ ἐνέμοντο ἐν τῷ Ἄχει.
\vs{19}Καὶ ἰδοὺ ἑπτὰ βόες ἕτεραι ἀνέβαινον ὀπίσω αὐτῶν ἐκ τοῦ ποταμοῦ, πονηραὶ καὶ αἰσχραὶ τῷ εἴδει, καὶ λεπταὶ ταῖς σαρξὶν, οἵας οὐκ εἶδον τοιαύτας ἐν ὅλῃ γῇ Αἰγύπτου αἰσχροτέρας.
\vs{20}Καὶ κατέφαγον αἱ ἑπτὰ βόες αἱ αἰσχραὶ καὶ λεπταὶ τὰς ἑπτὰ βόας τὰς πρώτας τὰς καλὰς καὶ τὰς ἐκλεκτάς.
\vs{21}Καὶ εἰσῆλθον εἰς τὰς κοιλίας αὐτῶν· καὶ οὑ διάδηλοι ἐγένοντο, ὅτι εἰσῆλθον εἰς τὰς κοιλίας αὐτῶν· καὶ αἱ ὄψεις αὐτῶν αἰσχραὶ, καθὰ καὶ τὴν ἀρχήν· ἐξεγερθεὶς δὲ ἐκοιμήθην.
\vs{22}Καὶ εἶδον πάλιν ἐν τῷ ὕπνῳ μου, καὶ ὥσπερ ἑπτὰ στάχυες ἀνέβαινον ἐν πυθμένι ἑνὶ πλήρεις καὶ καλοί·
\vs{23}Ἄλλοι δὲ ἑπτὰ στάχυες λεπτοὶ καὶ ἀνεμόφθοροι ἀνεφύοντο ἐχόμενοι αὐτῶν.
\vs{24}Καὶ κατέπιον οἱ ἑπτὰ στάχυες οἱ λεπτοὶ καὶ ἀνεμόφθοροι τοὺς ἑπτὰ στάχυας τοὺς καλοὺς καὶ τοὺς πλήρεις· εἶπα οὖν τοῖς ἐξῆγηταῖς, καὶ οὐκ ἦν ὁ ἀπαγγέλλων μοι αὐτό.

\vs{25}Καὶ εἶπεν Ἰωσὴφ τῷ Φαραὼ, τὸ ἐνύπνιον Φαραὼ ἕν ἐστιν· ὅσα ὁ Θεὸς ποιεῖ, ἔδειξε τῷ Φαραώ.
\vs{26}Αἱ ἑπτὰ βόες αἱ καλαὶ, ἑπτὰ ἔτη ἐστί· καὶ οἱ ἑπτὰ στάχυες οἱ καλοὶ, ἑπτὰ ἔτη ἐστί· τὸ ἐνύπνιον Φαραὼ ἕν ἐστι.
\vs{27}Καὶ αἱ ἑπτὰ βόες αἱ λεπταὶ, αἱ ἀναβαίνουσαι ὀπίσω αὐτῶν, ἑπτὰ ἔτη ἐστί· καὶ οἱ ἑπτὰ στάχυες οἱ λεπτοὶ καὶ ἀνεμόφθοροι, ἑπτὰ ἔτη ἐστί· ἔσονται ἑπτὰ ἔτη λιμοῦ.
\vs{28}Τὸ δὲ ῥῆμα ὃ εἴρηκα Φαραὼ, ὅσα ὁ Θεὸς ποιεῖ, ἔδειξε τῷ Φαραώ.
\vs{29}Ἰδοὺ ἑπτὰ ἔτη ἔρχεται εὐθηνία πολλὴ ἐν πάσῃ γῇ Αἰγύπτου.
\vs{30}Ἥξει δὲ ἑπτὰ ἔτη λιμοῦ μετὰ ταῦτα· καὶ ἐπιλήσονται τῆς πλησμονῆς τῆς ἐσομένης ἐν ὅλῃ Αἰγύπτῳ· καὶ ἀναλώσει ὁ λιμὸς τῆν γῆν.
\vs{31}Καὶ οὐκ ἐπιγνωσθήσεται ἡ εὐθηνία ἐπὶ τῆς γῆς ἀπὸ τοῦ λιμοῦ τοῦ ἐσομένου μετὰ ταῦτα· ἰσχυρὸς γὰρ ἔσται σφόδρα.
\vs{32}Περὶ δὲ τοῦ δευτερῶσαι τὸ ἐνύπνιον Φαραὼ δὶς, ὅτι ἀληθὲς ἔσται τὸ ῥῆμα τὸ παρὰ τοῦ Θεοῦ· καὶ ταχυνεῖ ὁ Θεὸς τοῦ ποιῆσαι αὐτό.
\vs{33}Νῦν οὖν σκέψαι ἄνθρωπον φρόνιμον καὶ συνετὸν, καὶ κατάστησον αὐτὸν ἐπὶ γῆς Αἰγύπτου.
\vs{34}Καὶ ποιησάτω Φαραὼ καὶ καταστησάτω τοπάρχας ἐπὶ τῆς γῆς· καὶ ἀποπεμπτωσάτωσαν πάντα τὰ γεννήματα τῆς γῆς Αἰγύπτου τῶν ἑπτὰ ἐτῶν τῆς εὐθηνίας,
\vs{35}καὶ συναγαγέτωσαν πάντα τὰ βρώματα τῶν ἑπτὰ ἐτῶν τῶν ἐρχομένων τῶν καλῶν τούτων· καὶ συναχθήτω ὁ σῖτος ὑπὸ χεῖρα Φαραώ· βρώματα ἐν ταῖς πόλεσι φυλαχθήτω.
\vs{36}Καὶ ἔσται τὰ βρώματα τὰ πεφυλαγμένα τῇ γῇ εἰς τὰ ἑπτὰ ἔτη τοῦ λιμοῦ, ἃ ἔσονται ἐν γῇ Αἰγύπτου, καὶ οὐκ ἐκτριβήσεται ἡ γῆ ἐν τῷ λιμῷ.
\vs{37}Ἤρεσε δὲ τὸ ῥῆμα ἐναντίον Φαραὼ, καὶ ἐναντίον πάντων τῶν παίδων αὐτοῦ.

\vs{38}Καὶ εἶπε Φαραὼ πᾶσι τοῖς παισὶν αὐτοῦ, μῆ εὑρήσομεν ἄνθρωπον τοιοῦτον, ὃς ἔχει πνεῦμα Θεοῦ ἐν αὐτῷ;
\vs{39}Εἶπε δὲ Φαραὼ τῷ Ἰωσὴφ, ἐπειδὴ ἔδειξεν ὁ Θεός σοι πάντα ταῦτα, οὐκ ἔστιν ἄνθρωπος φρονιμώτερος καὶ συνετώτερός σου.
\vs{40}Σὺ ἔσῃ ἐπὶ τῷ οἴκῳ μου, καὶ ἐπὶ τῷ στόματί σου ὑπακούσεται πᾶς ὁ λαός μου· πλὴν τὸν θρόνον ὑπερέξω σου ἐγώ.
\vs{41}Εἶπε δὲ Φαραὼ τῷ Ἰωσὴφ, ἰδοὺ καθίστημί σε σήμερον ἐπὶ πάσῃ γῇ Αἰγύπτου.
\vs{42}Καὶ περιελόμενος Φαραὼ τὸν δακτύλιον ἀπὸ τῆς χειρὸς αὐτοῦ, περίεθηκεν αὐτὸν ἐπὶ τὴν χεῖρα Ἰωσὴφ, καὶ ἐνέδυσεν αὐτὸν στολὴν βυσσίνην, καὶ περιέθηκε κλοιὸν χρυσοῦν περὶ τὸν τράχηλον αὐτοῦ.
\vs{43}Καὶ ἀνεβίβασεν αὐτὸν ἐπὶ τὸ ἅρμα τὸ δεύτερον τῶν αὐτοῦ· καὶ ἐκήρυξεν ἔμπροσθεν αὐτοῦ κήρυξ· καὶ κατέστησεν αὐτὸν ἐφʼ ὅλης γῆς Αἰγύπτου.
\vs{44}Εἶπε δὲ Φαραὼ τῷ Ἰωσὴφ, ἐγὼ Φαραώ· ἄνευ σοῦ οὐκ ἐξαρεῖ οὐθεὶς τὴν χεῖρα αὐτοῦ ἐπὶ πάσης γῆς Αἰγύπτου.
\vs{45}Καὶ ἐκάλεσε Φαραὼ τὸ ὄνομα Ἰωσὴφ, Ψονθομφανήχ· καὶ ἔδωκεν αὐτῷ τὴν Ἀσενὲθ θυγατέρα Πετεφρῆ ἱερέως Ἡλιουπόλεως αὐτῷ εἰς γυναῖκα.
\vs{46}Ἰωσὴφ δὲ ἦν ἐτῶν τριάκοντα, ὅτε ἔστη ἐναντίον Φαραὼ βασιλέως Αἰγύπτου· ἐξῆλθε δὲ Ἰωσὴφ ἀπὸ προσώπου Φαραὼ, καὶ διῆλθε πᾶσαν γῆν Αἰγύπτου.
\vs{47}Καὶ ἐποίησεν ἡ γῆ ἐν τοῖς ἑπτὰ ἔτεσι τῆς εὐθηνίας δράγματα.
\vs{48}Καὶ συνήγαγε πάντα τὰ βρώματα τῶν ἑπτὰ ἐτῶν, ἐν οἷς ἦν ἡ εὐθηνία ἐν τῇ γῇ Αἰγύπτου· καὶ ἔθηκε τὰ βρώματα ἐν ταῖς πόλεσι· βρώματα τῶν πεδίων τῆς πόλεως τῶν κύκλῳ αὐτῆς ἔθηκεν ἐν αὐτῇ.
\vs{49}Καὶ συνήγαγεν Ἰωσὴφ σῖτον ὡσεὶ τὴν ἄμμον τῆς θαλάσσης πολὺν σφόδρα, ἕως οὐκ ἠδύνατο ἀριθμηθῆναι, οὐ γὰρ ἦν ἀριθμός.

\vs{50}Τῷ δὲ Ἰωσὴφ ἐγένοντο υἱοὶ δύο πρὸ τοῦ ἐλθεῖν τὰ ἑπτὰ ἔτη τοῦ λιμοῦ, οὓς ἔτεκεν αὐτῷ Ἀσενὲθ ἡ θυγάτηρ Πετεφρῆ ἱερέως Ἡλιουπόλεως.
\vs{51}Ἐκάλεσε δὲ Ἰωσὴφ τὸ ὄνομα τοῦ πρωτοτόκου, Μανασσῆ· ὅτι ἐπιλαθέσθαι με ἐποίησεν ὁ Θεὸς πάντων τῶν πόνων μου, καὶ πάντων τῶν τοῦ πατρός μου·
\vs{52}Τὸ δὲ ὄνομα τοῦ δευτέρου ἐκάλεσεν, Ἐφραίμ· ὅτι ηὔξησέ με ὁ Θεὸς ἐν γῇ ταπεινώσεώς μου.
\vs{53}Παρῆλθον δὲ τὰ ἑπτὰ ἔτη τῆς εὐθηνίας, ἃ ἐγένοντο ἐν τῇ γῇ Αἰγύπτου.
\vs{54}Καὶ ἤρξατο τὰ ἑπτὰ ἔτη τοῦ λιμοῦ ἔρχεσθαι, καθὰ εἶπεν Ἰωσήφ· καὶ ἐγένετο λιμὸς ἐν πάσῃ τῇ γῇ· ἐν δὲ πάσῃ τῇ γῇ Αἰγύπτου ἦσαν ἄρτοι.
\vs{55}Καὶ ἐπείνασε πᾶσα ἡ γῆ Αἰγύπτου· ἔκραξε δὲ ὁ λαὸς πρὸς Φαραὼ περὶ ἄρτων· εἶπε δὲ Φαραὼ πᾶσι τοῖς Αἰγυπτίοις, πορεύεσθε πρὸς Ἰωσὴφ, καὶ ὃ ἐὰν εἴπῃ ὑμῖν, ποιήσατε.
\vs{56}Καὶ ὁ λιμὸς ἦν ἐπὶ προσώπου πάσης τῆς γῆς· ἀνέῳξε δὲ Ἰωσὴφ πάντας τοὺς σιτοβολῶνας, καὶ ἐπώλει πᾶσι τοῖς Αἰγυπτίοις.
\vs{57}Καὶ πᾶσαι αἱ χῶραι ἦλθον εἰς Αἴγυπτον, ἀγοράζειν πρὸς Ἰωσήφ· ἐπεκράτησε γὰρ ὁ λιμὸς ἐν πάσῃ τῇ γῇ·

\ch{42}
Ἰδὼν δὲ Ἰακὼβ ὅτι ἐστὶ πράσις ἐν Αἰγύπτῳ, εἶπε τοῖς υἱοῖς αὐτοῦ, ἱνατί ῥαθυμεῖτε;
\vs{2}Ἰδοὺ ἀκήκοα, ὅτι ἐστὶ σῖτος ἐν Αἰγύπτῳ· κατάβητε ἐκεὶ, καὶ πρίασθε ἡμῖν μικρὰ βρώματα, ἵνα ζήσωμεν καὶ μὴ ἀποθάνωμεν.

\vs{3}Κατέβησαν δὲ οἱ ἀδελφοὶ Ἰωσὴφ οἱ δέκα, πρίασθαι σῖτον ἐξ Αἰγύπτου·
\vs{4}Τὸν δὲ Βενιαμὶν, τὸν ἀδελφὸν Ἰωσὴφ, οὐκ ἀπέστειλε μετὰ τῶν ἀδελφῶν αὐτοῦ· εἶπε γὰρ, μή ποτε συμβῇ αὐτῷ μαλακία.
\vs{5}Ἦλθον δὲ οἱ υἱοὶ Ἰσραὴλ ἀγοράζειν μετὰ τῶν ἐρχομένων· ἦν γὰρ ὁ λιμὸς ἐν γῇ Χαναάν.
\vs{6}Ἰωσὴφ δὲ ἦν ὁ ἄρχων τῆς γῆς· οὗτος ἐπώλει παντὶ τῷ λαῷ τῆς γῆς· ἐλθόντες δὲ οἱ ἀδελφοὶ Ἰωσὴφ προσεκύνησαν αὐτῷ ἐπὶ πρόσωπον ἐπὶ τὴν γῆν.
\vs{7}Ἰδὼν δὲ Ἰωσὴφ τοὺς ἀδελφοὺς αὐτοῦ, ἐπέγνω· καὶ ἠλλοτριοῦτο ἀπʼ αὐτῶν, καὶ ἐλάλησεν αὐτοῖς σκληρά· καὶ εἶπεν αὐτοῖς, πόθεν ἥκατε; οἱ δὲ εἶπον, ἐκ γῆς Χαναὰν, ἀγοράσαι βρώματα.
\vs{8}Ἐπέγνω δὲ Ἰωσὴφ τοὺς ἀδελφοὺς αὐτοῦ· αὐτοὶ δὲ οὐκ ἐπέγνωσαν αὐτόν·
\vs{9}Καὶ ἐμνήσθη Ἰωσὴφ τῶν ἐνυπνίων αὐτοῦ, ὧν εἶδεν αὐτός· καὶ εἶπεν αὐτοῖς, κατάσκοποί ἐστε, κατανοῆσαι τὰ ἴχνη τῆς χώρας ἥκατε.
\vs{10}Οἱ δὲ εἶπαν, οὐχὶ, κύριε· οἱ παῖδές σου ἤλθομεν πρίασθαι βρώματα.
\vs{11}Πάντες ἐσμὲν υἱοὶ ἑνὸς ἀνθρώπου· εἰρηνικοί ἐσμεν, οὐκ εἰσιν οἱ παῖδές σου κατάσκοποι.
\vs{12}Εἶπε δὲ αὐτοῖς, οὐχί· ἀλλὰ τὰ ἴχνη τῆς γῆς ἤλθετε ἰδεῖν.
\vs{13}Οἱ δὲ εἶπαν, δώδεκά ἐσμεν οἱ παῖδές σου ἀδελφοὶ ἐν γῇ Χαναάν· καὶ ἰδοὺ ὁ νεώτερος μετὰ τοῦ πατρὸς ἡμῶν σήμερον· ὁ δὲ ἕτερος οὐχ ὑπάρχει.
\vs{14}Εἶπε δὲ αὐτοῖς Ἰωσὴφ, τοῦτό ἐστιν ὃ εἴρηκα ὑμῖν, λέγων, ὅτι κατάσκοποί ἐστε.
\vs{15}Ἐν τούτῳ φανεῖσθε· νὴ τὴν ὑγίειαν Φαραὼ, οὐ μὴ ἐξέλθητε ἐντεῦθεν, ἐὰν μὴ ὁ ἀδελφὸς ὑμῶν ὁ νεώτερος ἔλθῃ ὧδε.
\vs{16}Ἀποστείλατε ἐξ ὑμῶν ἕνα, καὶ λάβετε τὸν ἀδελφὸν ὑμῶν· ὑμεῖς δὲ ἀπάχθητε ἕως τοῦ φανερὰ γενέσθαι τὰ ῥήματα ὑμῶν, εἰ ἀληθεύετε ἢ οὔ· εἰ δὲ μὴ, νὴ τὴν ὑγίειαν Φαραὼ, ἦ μὴν κατάσκοποί ἐστε.
\vs{17}Καὶ ἔθετο αὐτοὺς ἐν φυλακῇ ἡμέρας τρεῖς.
\vs{18}Εἶπε δὲ αὐτοῖς τῇ ἡμέρᾳ τῇ τρίτῃ, τοῦτο ποιήσατε, καὶ ζήσεσθε· τὸν Θεὸν γὰρ ἐγὼ φοβοῦμαι.
\vs{19}Εἰ εἰρηνικοί ἐστε, ἀδελφὸς ὑμῶν κατασχεθήτω εἷς ἐν τῇ φυλακῇ· αὐτοὶ δὲ βαδίσατε, καὶ ἀπαγάγετε τὸν ἀγορασμὸν τῆς σιτοδοσίας ὑμῶν.
\vs{20}Καὶ τὸν ἀδελφὸν ὑμῶν τὸν νεώτερον ἀγάγετε πρός με, καὶ πιστευθήσονται τὰ ῥήματα ὑμῶν· εἰ δὲ μὴ, ἀποθανεῖσθε. Ἐποίησαν δὲ οὕτως.
\vs{21}Καὶ εἶπεν ἕκαστος πρὸς τὸν ἀδελφὸν αὐτοῦ, ναὶ, ἐν ἁμαρτίαις γάρ ἐσμεν περὶ τοῦ ἀδελφοῦ ἡμῶν, ὅτι ὑπερίδομεν τὴν θλίψιν τῆς ψυχῆς αὐτοῦ, ὅτε κατεδέετο ἡμῶν, καὶ οὐκ εἰσηκούσαμεν αὐτοῦ· καὶ ἕνεκεν τούτου ἐπῆλθεν ἐφʼ ἡμᾶς ἡ θλίψις αὕτη.
\vs{22}Ἀποκριθεὶς δὲ Ῥουβὴν εἶπεν αὐτοῖς, οὐκ ἐλάλησα ὑμῖν, λέγων, μὴ ἀδικήσητε τὸ παιδάριον, καὶ οὐκ εἰσηκούσατέ μου; καὶ ἰδοὺ τὸ αἷμα αὐτοῦ ἐκζητεῖται.
\vs{23}Αὐτοὶ δὲ οὐκ ᾔδεισαν, ὅτι ἀκούει Ἰωσήφ· ὁ γὰρ ἑρμηνευτὴς ἀνὰ μέσον αὐτῶν ἦν·
\vs{24}Ἀποστραφεὶς δὲ ἀπʼ αὐτῶν ἔκλαυσεν Ἰωσήφ· καὶ πάλιν προσῆλθε πρὸς αὐτοὺς, καὶ εἶπεν αὐτοῖς· καὶ ἔλαβε τὸν Συμεὼν ἀπʼ αὐτῶν, καὶ ἔδησεν αὐτὸν ἐναντίον αὐτῶν.

\vs{25}Ἐνετείλατο δὲ Ἰωσὴφ ἐμπλῆσαι τὰ ἀγγεῖα αὐτῶν σίτου, καὶ ἀποδοῦναι τὸ ἀργύριον αὐτῶν ἑκάστῳ εἰς τὸν σάκκον αὐτοῦ, καὶ δοῦναι αὐτοῖς ἐπισιτισμὸν εἰς τὴν ὁδόν· καὶ ἐγενήθη αὐτοῖς οὕτως.
\vs{26}Καὶ ἐπιθέντες τὸν σῖτον ἐπὶ τοῦς ὄνους αὐτῶν, ἀπῆλθον ἐκεῖθεν.
\vs{27}Λύσας δὲ εἷς τὸν μάρσιππον αὐτοῦ, δοῦναι χορτάσματα τοῖς ὄνοις αὐτοῦ, οὗ κατέλυσαν, καὶ εἶδε τὸν δεσμὸν τοῦ ἀργυρίου αὐτοῦ, καὶ ἦν ἐπάνω τοῦ στόματος τοῦ μαρσίππου.
\vs{28}Καὶ εἶπε τοῖς ἀδελφοῖς αὐτοῦ, ἀπεδόθη μοι τὸ ἀργύριον, καὶ ἰδοὺ τοῦτο ἐν τῷ μαρσίππῳ μου· καὶ ἐξέστη ἡ καρδία αὐτῶν, καὶ ἐταράχθησαν πρὸς ἀλλήλους, λέγοντες, τί τοῦτο ἐποίησεν ὁ Θεὸς ἡμῖν;
\vs{29}Ἦλθον δὲ πρὸς Ἰακὼβ τὸν πατέρα αὐτῶν εἰς γὴν Χαναὰν, καὶ ἀπήγγειλαν αὐτῷ πάντα τὰ συμβάντα αὐτοῖς, λέγοντες,
\vs{30}Λελάληκεν ὁ ἄνθρωπος ὁ κύριος τῆς γῆς πρὸς ἡμᾶς σκληρὰ, καὶ ἔθετο ἡμᾶς ἐν φυλακῇ, ὡς κατασκοπεύοντας τὴν γῆν.
\vs{31}Εἴπαμεν δὲ αὐτῷ, εἰρήνικοί ἐσμεν, οὐκ ἐσμὲν κατάσκοποι.
\vs{32}Δώδεκα ἀδελφοί ἐσμεν, υἱοὶ τοῦ πατρὸς ἡμῶν· ὁ εἷς οὐχ ὑπάρχει· ὁ δὲ μικρὸς μετὰ τοῦ πατρὸς ἡμῶν σήμερον ἐν γῇ Χαναάν.
\vs{33}Εἶπε δὲ ἡμῖν ὁ ἄνθρωπος ὁ κύριος τῆς γῆς, ἐν τούτῳ γνώσομαι, ὅτι εἰρηνικοί ἐστε· ἀδελφὸν ἕνα ἄφετε ὧδε μετʼ ἐμοῦ· τὸν δὲ ἀγορασμὸν τῆς σιτοδοσίας τοῦ οἴκου ὑμῶν λαβόντες ἀπέλθατε.
\vs{34}Καὶ ἀγάγετε πρός με τὸν ἀδελφὸν ὑμῶν τὸν νεώτερον· καὶ γνώσομαι ὅτι οὐ κατάσκοποί ἐστε, ἀλλʼ ὅτι εἰρηνικοί ἐστε· καὶ τὸν ἀδελφὸν ὑμῶν ἀποδώσω ὑμῖν, καὶ τῇ γῇ ἐμπορεύσεσθε.
\vs{35}Ἐγένετο δὲ ἐν τῷ κατακενοῦν αὐτοὺς τοὺς σάκκους αὐτῶν, καὶ ἦν ἑκάστου ὁ δεσμὸς τοῦ ἀργυρίου ἐν τῷ σάκκῳ αὐτῶν· καὶ εἶδον τοὺς δεσμοὺς τοῦ ἀργυρίου αὐτῶν αὐτοὶ, καὶ ὁ πατὴρ αὐτῶν, καὶ ἐφοβήθησαν.
\vs{36}Εἶπε δὲ αὐτοῖς Ἰακὼβ ὁ πατὴρ αὐτῶν, ἐμὲ ἠτεκνώσατε· Ἰωσὴφ οὐκ ἔστι, Συμεὼν οὐκ ἔστι, καὶ τὸν Βενιαμὶν λήψεσθε; ἐπʼ ἐμὲ ἐγένετο ταῦτα πάντα.
\vs{37}Εἶπε δὲ Ῥουβὴν τῷ πατρὶ αὐτῶν, λέγων, τοὺς δύο υἱούς μου ἀπόκτεινον, ἐὰν μὴ ἀγάγω αὐτὸν πρὸς σέ· δὸς αὐτὸν εἰς τὴν χεῖρά μου, κᾀγὼ ἀνάξω αὐτὸν πρὸς σέ.
\vs{38}Ὁ δὲ εἶπεν, οὐ καταβήσεται ὁ υἱός μου μεθʼ ὑμῶν, ὅτι ὁ ἀδελφὸς αὐτοῦ ἀπέθανε, καὶ αὐτὸς μόνος καταλέλειπται· καὶ συμβήσεται αὐτὸν μαλακισθῆναι ἐν τῇ ὁδῷ, ᾗ ἐὰν πορεύησθε, καὶ κατάξετέ μου τὸ γῆρας μετὰ λύπης εἰς ᾅδοῦ.

\ch{43}
Ὁ δὲ λιμὸς ἐνίσχυσεν ἐπὶ τῆς γῆς.
\vs{2}Ἐγένετο δὲ ἡνίκα συνετέλεσαν καταφαγεῖν τὸν σῖτον, ὃν ἤνεγκαν ἐξ Αἰγύπτου, καὶ εἶπεν αὐτοῖς ὁ πατὴρ αὐτῶν, πάλιν πορευθέντες πρίασθε ἡμῖν μικρὰ βρώματα.
\vs{3}Εἶπε δὲ αὐτῷ Ἰούδας, λέγων, διαμαρτυρίᾳ μεμαρτύρηται ἡμῖν ὁ ἄνθρωπος ὁ κύριος τῆς γῆς, λέγων, οὐκ ὄψεσθε τὸ πρόσωπόν μου, ἐὰν μὴ ὁ ἀδελφὸς ὑμῶν ὁ νεώτερος μεθʼ ὑμῶν ᾖ.
\vs{4}Εἰ μὲν οὖν ἀποστέλλῃς τὸν ἀδελφὸν ἡμῶν μεθʼ ἡμῶν, καταβησόμεθα, καὶ ἀγοράσομέν σοι βρώματα.
\vs{5}Εἰ δὲ μὴ ἀποστέλλῃς τὸν ἀδελφὸν ἡμῶν μεθʼ ἡμῶν, οὐ πορευσόμεθα· ὁ γὰρ ἄνθρωπος εἶπεν ἡμῖν, λέγων, οὐκ ὄψεσθέ μου τὸ πρόσωπον, ἐὰν μὴ ὁ ἀδελφὸς ὑμῶν ὁ νεώτερος μεθʼ ὑμῶν ᾖ.
\vs{6}Εἶπε δὲ Ἰσραὴλ, τί ἐκακοποιήσατέ με, ἀναγγείλαντες τῷ ἀνθρώπῳ ὅτι ἐστὶν ὑμῖν ἀδελφός;
\vs{7}Οἱ δὲ εἶπαν, ἐρωτῶν ἐπηρώτησεν ἡμᾶς ὁ ἄνθρωπος καὶ τὴν γενεὰν ἡμῶν, λέγων, εἰ ἔτι ὁ πατὴρ ὑμῶν ζῇ, καὶ εἰ ἔστιν ὑμῖν ἀδελφός· καὶ ἀπηγγείλαμεν αὐτῷ κατὰ τὴν ἐπερώτησιν ταύτην· μὴ ᾔδειμεν ὅτι ἐρεῖ ἡμῖν, ἀγάγετε τὸν ἀδελφὸν ὑμῶν;
\vs{8}Εἶπε δὲ Ἰούδας πρὸς Ἰσραὴλ τὸν πατέρα αὐτοῦ, ἀπόστειλον τὸ παιδάριον μετʼ ἐμοῦ· καὶ ἀναστάντες πορευσόμεθα, ἵνα ζῶμεν καὶ μὴ ἀποθάνωμεν καὶ ἡμεῖς, καὶ σὺ, καὶ ἡ ἀποσκευὴ ἡμῶν.
\vs{9}Ἐγὼ δὲ ἐκδέχομαι αὐτόν· ἐκ χειρός μου ζήτησον αὐτόν· ἐὰν μὴ ἀγάγω αὐτὸν πρός σε, καὶ στήσω αὐτὸν ἐναντίον σου, ἡμαρτηκὼς ἔσομαι εἰς σὲ πάσας τὰς ἡμέραν.
\vs{10}Εἰ μὴ γὰρ ἐβραδύναμεν, ἤδη ἂν ὑπεστρέψαμεν δίς.
\vs{11}Εἶπε δὲ αὐτοῖς Ἰσραὴλ ὁ πατὴρ αὐτῶν, εἰ οὕτως ἐστὶ, τοῦτο ποιήσατε· λάβετε ἀπὸ τῶν καρπῶν τῆς γῆς ἐν τοῖς ἀγγείοις ὑμῶν, καὶ καταγάγετε τῷ ἀνθρώπῳ δῶρα τῆς ῥητίνης, καὶ τοῦ μέλιτος, θυμίαμά τε καὶ στακτὴν, καὶ τερέινθον, καὶ κάρυα.
\vs{12}Καὶ τὸ ἀργύριον δισσὸν λάβετε ἐν ταῖς χερσὶν ὑμῶν· τὸ ἀργύριον τὸ ἀποστραφὲν ἐν τοῖς μαρσίπποις ὑμῶν ἀποστρέψατε μεθʼ ὑμῶν· μή ποτε ἀγνόημά ἐστι.
\vs{13}Καὶ τὸν ἀδελφὸν ὑμῶν λάβετε· καὶ ἀναστάντες κατάβητε πρὸς τὸν ἄνθρωπον.
\vs{14}Ὁ δὲ Θεός μου δώῃ ὑμῖν χάριν ἐναντίον τοῦ ἀνθρώπου καὶ ἀποστείλαι τὸν ἀδελφὸν ὑμῶν τὸν ἕνα, καὶ τὸν Βενιαμίν· ἐγὼ μὲν γὰρ καθάπερ ἠτέκνωμαι, ἠτέκνωμαι.

\vs{15}Λαβόντες δὲ οἱ ἄνδρες τὰ δῶρα ταῦτα καὶ τὸ ἀργύριον διπλοῦν, ἔλαβον ἐν ταῖς χερσὶν αὐτῶν καὶ τὸν Βενιαμείν· καὶ ἀναστάντες κατέβησαν εἰς Αἴγυπτον· καὶ ἔστησαν ἐναντίον Ἰωσήφ.
\vs{16}Εἶδε δὲ Ἰωσὴφ αὐτοὺς, καὶ τὸν Βενιαμὶν τὸν ἀδελφὸν αὐτοῦ τὸν ὁμομήτριον· καὶ εἶπε τῷ ἐπὶ τῆς οἰκίας αὐτοῦ, εἰσάγαγε τοὺς ἀνθρώπους εἰς τὴν οἰκίαν, καὶ σφάξον θύματα, καὶ ἑτοίμασον· μετʼ ἐμοῦ γὰρ φάγονται οἱ ἄνθρωποι ἄρτους τὴν μεσημβρίαν.
\vs{17}Ἐποίησε δὲ ὁ ἄνθρωπος καθὰ εἶπεν Ἰωσήφ· καὶ εἰσήγαγε τοὺς ἀνθρώπους εἰς τὸν οἶκον Ἰωσήφ.
\vs{18}Ἰδόντες δὲ οἱ ἄνδρες ὅτι εἰσήχθησαν εἰς τὸν οἶκον τοῦ Ἰωσήφ, εἶπαν, διὰ τὸ ἀργύριον τὸ ἀποστραφὲν ἐν τοῖς μαρσίπποις ἡμῶν τὴν ἀρχὴν, ἡμεῖς εἰσαγόμεθα, τοῦ συκοφαντῆσαι ἡμᾶς καὶ ἐπιθέσθαι ἡμῖν, τοῦ λαβεῖν ἡμᾶς εἰς παῖδας, καὶ τοὺς ὄνους ἡμῶν.
\vs{19}Προσελθόντες δὲ πρὸς τὸν ἄνθρωπον τὸν ἐπὶ τοῦ οἴκου τοῦ Ἰωσὴφ, ἐλάλησαν αὐτῷ ἐν τῷ πυλῶνι τοῦ οἴκου,
\vs{20}λέγοντες, δεόμεθα, κύριε· κατέβηεν τὴν ἀρχὴν πρίασθαι βρώματα.
\vs{21}Ἐγένετο δὲ ἡνίκα ἤλθομεν εἰς τὸ καταλῦσαι, καὶ ἠνοίξαμεν τοὺς μαρσίππους ἡμῶν, καὶ τόδε τὸ ἀργύριον ἑκάστου ἐν τῷ μαρσίππῳ αὐτοῦ· τὸ ἀργύριον ἡμῶν ἐν σταθμῷ ἀπεστρέψαμεν νῦν ἐν ταῖς χερσὶν ἡμῶν.
\vs{22}Καὶ ἀργύριον ἕτερον ἠνέγκαμεν μεθʼ ἑαυτῶν, ἀγοράσαι βρώματα· οὐκ οἴδαμεν τίς ἐνέβαλεν τὸ ἀργύριον εἰς τοὺς μαρσίππους ἡμῶν.
\vs{23}Εἶπε δὲ αὐτοῖς, ἵλεως ὑμῖν, μὴ φοβεῖσθε· ὁ Θεὸς ὑμῶν, καὶ ὁ Θεὸς τῶν πατέρων ὑμῶν, ἔδωκεν ὑμῖν θησαυροὺς ἐν τοῖς μαρσίπποις ὑμῶν· καὶ τὸ ἀργύριον ὑμῶν εὐδοκιμοῦν ἀπέχω· καὶ ἐξήγαγε πρὸς αὐτοὺς τὸν Συμεών.
\vs{24}Καὶ ἤνεγκεν ὕδωρ νίψαι τοὺς πόδας αὐτῶν· καὶ ἔδωκε χορτάσματα τοῖς ὄνοις αὐτῶν.
\vs{25}Ἡτοίμασαν δὲ τὰ δῶρα, ἕως τοῦ ἐλθεῖν τὸν Ἰωσὴφ μεσημβρίας· ἤκουσαν γὰρ ὅτι ἐκεῖ μέλλει ἀριστᾷν.
\vs{26}Εἰσῆλθε δὲ Ἰωσὴφ εἰς τὴν οἰκίαν, καὶ προσήνεγκαν αὐτῷ τὰ δῶρα, ἃ εἶχον ἐν ταῖς χερσὶν αὐτῶν, εἰς τὸν οἶκον· καὶ προσεκύνησαν αὐτῷ ἐπὶ πρόσωπον ἐπὶ τὴν γῆν.
\vs{27}Ἠρώτησε δὲ αὐτοὺς, πῶς ἔχετε; καὶ εἶπεν αὐτοῖς, εἰ ὑγιαίνει ὁ πατὴρ ὑμῶν ὁ πρεσβύτης, ὃν εἴπατε; ἔτι ζῇ;
\vs{28}Οἱ δὲ εἶπαν, ὑγιαίνει ὁ παῖς σου ὁ πατὴρ ἡμῶν, ἔτι ζῇ. Καὶ εἶπεν, εὐλογημένος ὁ ἄνθρωπος ἐκεῖνος τῷ Θεῷ· καὶ κύψαντες προσεκύνησαν αὐτῷ.
\vs{29}Ἀναβλέψας δὲ τοῖς ὀφθαλμοῖς αὐτοῦ Ἰωσὴφ, εἶδε Βενιαμὶν τὸν ἀδελφὸν αὐτοῦ τὸν ὁμομήτριον· καὶ εἶπεν, οὗτος ὁ ἀδελφὸς ὑμῶν ὁ νεώτερος, ὃν εἴπατε πρός με ἀγαγεῖν; καὶ εἶπεν, ὁ Θεὸς ἐλεήσαι σε, τέκνον.
\vs{30}Ἐταράχθη δὲ Ἰωσήφ· συνεστρέφετο γὰρ τὰ ἔγκατα αὐτοῦ ἐπὶ τῷ ἀδελφῷ αὐτοῦ, καὶ ἐζήτει κλαῦσαι· εἰσελθὼν δὲ εἰς τὸ ταμεῖον, ἔκλαυσεν ἐκεῖ.

\vs{31}Καὶ νιψάμενος τὸ πρόσωπον, ἐξελθὼν ἐνεκρατεύσατο· καὶ εἶπε, παράθετε ἄρτους.
\vs{32}Καὶ παρέθηκαν αὐτῷ μόνῳ, καὶ αὐτοῖς καθʼ ἑαυτούς, καὶ τοῖς Αἰγυπτίοις τοῖς συνδειπνοῦσι μετʼ αὐτοῦ καθʼ ἑαυτούς· οὐ γὰρ ἐδύναντο οἱ Αἰγύπτιοι συνεσθίειν μετὰ τῶν Ἐβραίων ἄρτους· βδέλυγμα γάρ ἐστι τοῖς Αἰγυπτίοις.
\vs{33}Ἐκάθισαν δὲ ἐναντίον αὐτοῦ, ὁ πρωτότοκος κατὰ τὰ πρεσβεῖα αὐτοῦ, καὶ ὁ νεώτερος κατὰ τὴν νεότητα αὐτοῦ· ἐξίσταντο δὲ οἱ ἄνθρωποι ἕκαστος πρὸς τὸν ἀδελφὸν αὐτοῦ.
\vs{34}Ἦραν δὲ μερίδα παρʼ αὐτοῦ πρὸς ἑαυτούς· ἐμεγαλύνθη δὲ ἡ μερὶς Βενιαμεὶν παρὰ τὰς μερίδας πάντων πενταπλασίως πρὸς τὰς ἐκείνων· ἔπιον δὲ καὶ ἐμεθύσθησαν μετʼ αὐτοῦ.

\ch{44}Καὶ ἐνετείλατο ὁ Ἰωσὴφ τῷ ὄντι ἐπὶ τῆς οἰκίας αὐτοῦ, λέγων, πλήσατε τοὺς μαρσίππους τῶν ἀνθρώπων βρωμάτων, ὅσα ἐὰν δύνωνται ἆραι· καὶ ἐμβάλετε ἑκάστου τὸ ἀργύριον ἐπὶ τοῦ στόματος τοῦ μαρσίππου.
\vs{2}Καὶ τὸ κόνδυ μου τὸ ἀργυροῦν ἐμβάλετε εἰς τὸν μάρσιππον τοῦ νεωτέρου, καὶ τὴν τιμὴν τοῦ σίτου αὐτοῦ· ἐγενήθη δὲ κατὰ τὸ ῥῆμα Ἰωσὴφ, καθὼς εἶπε.

\vs{3}Τὸ πρωῒ διέφαυσε· καὶ οἱ ἄνθρωποι ἀπεστάλησαν, αὐτοὶ καὶ οἱ ὄνοι αὐτῶν.
\vs{4}Ἐξελθόντων δὲ αὐτῶν τὴν πόλιν, οὐκ ἀπέσχον μακράν· καὶ Ἰωσὴφ εἶπε τῷ ἐπὶ τῆς οἰκίας αὐτοῦ, ἀναστὰς ἐπιδίωξον ὀπίσω τῶν ἀνθρώπων, καὶ καταλήμψῃ αὐτοὺς, καὶ ἐρεῖς αὐτοῖς τί ὅτι ἀνταπεδώκατε πονηρὰ ἀντὶ καλῶν;
\vs{5}Ἱνατί ἐκλέψατέ μου τὸ κόνδυ τὸ ἀργυροῦν; οὐ τοῦτό ἐστιν, ἐν ᾧ πίνει ὁ κύριός μου; αὐτὸς δὲ οἰωνισμῷ οἰωνίζεται ἐν αὐτῷ. πονηρὰ συντετελέκατε ἃ πεποιήκατε.
\vs{6}Εὑρὼν δὲ αὐτοὺς, εἶπεν αὐτοῖς κατὰ τὰ ῥήματα ταῦτα.
\vs{7}Οἱ δὲ εἶπαν αὐτῷ, ἱνατί λαλεῖ ὁ κύριος κατὰ τὰ ῥήματα ταῦτα; μὴ γένοιτο τοῖς παισίν σου ποιῆσαι κατὰ τὸ ῥῆμα τοῦτο.
\vs{8}Εἰ τὸ μὲν ἀργύριον, ὃ εὕρομεν ἐν τοῖς μαρσίπποις ἡμῶν, ἀπεστρέψαμεν πρὸς σὲ ἐκ γῆς Χαναὰν, πῶς ἂν κλέψαιμεν ἐκ τοῦ οἴκου τοῦ κυρίου σου ἀργύριον ἢ χρυσίον;
\vs{9}Παρʼ ᾧ ἂν εὕρῃς τὸ κόνδυ τῶν παιδων σου, ἀποθνησκέτω· καὶ ἡμεῖς δὲ ἐσόμεθα παῖδες τῷ κυρίῳ ἡμῶν.
\vs{10}Ὁ δὲ εἶπε, καὶ νῦν, ὡς λέγετε, οὕτως ἔσται· παρʼ ᾧ ἂν εὑρεθῇ τὸ κόνδυ, ἔσται μου παῖς ὑμεῖς δὲ ἔσεσθε καθαροί.
\vs{11}Καὶ ἔσπευσαν, καὶ καθεῖλαν ἕκαστος τὸν μάρσιππον αὐτοῦ ἐπὶ τὴν γῆν· καὶ ἤνοιξαν ἕκαστος τὸν μάρσιππον αὐτοῦ.
\vs{12}Ἠρεύνησε δὲ ἀπὸ τοῦ πρεσβυτέρου ἀρξάμενος, ἕως ἦλθεν ἐπὶ τὸν νεώτερον. καὶ εὗρε τὸ κόνδυ ἐν τῷ μαρσίππῳ τοῦ Βενιαμίν.
\vs{13}Καὶ διέῤῥηξαν τὰ ἱμάτια αὐτῶν, καὶ ἐπέθηκαν ἕκαστος τὸν μαρσίππον αὐτοῦ ἐπὶ τὸν ὄνον αὐτοῦ, καὶ ἐπέστρεψαν εἰς τὴν πόλιν.

\vs{14}Εἰσῆλθε δὲ Ἰούδας καὶ οἱ ἀδελφοὶ αὐτοῦ πρὸς Ἰωσὴφ ἔτι αὐτοῦ ὄντος ἐκεῖ, καὶ ἔπεσον ἐναντίον αὐτοῦ ἐπὶ τὴν γῆν.
\vs{15}Εἶπε δὲ αὐτοῖς Ἰωσὴφ, τί τὸ πρᾶγμα τοῦτο ἐποιήσατε; οὐκ οἴδατε ὅτι οἰωνισμῷ οἰωνιεῖται ὁ ἄνθρωπος, οἷος ἐγώ;
\vs{16}Εἶπε δὲ Ἰούδας, τί ἀντεροῦμεν τῷ κυρίῳ, ἢ τί λαλήσομεν, ἢ τί δικαιωθῶμεν; ὁ Θεὸς δὲ εὗρε τὴν ἀδικίαν τῶν παίδων σου· ἰδού ἐσμεν οἰκέται τῷ κυρίῳ ἡμῶν, καὶ ἡμεῖς, καὶ παρʼ ᾧ εὑρέθη τὸ κόνδυ.
\vs{17}Εἶπε δὲ Ἰωσὴφ, μή μοι γένοιτο ποιῆσαι τὸ ῥῆμα τοῦτο· ὁ ἄνθρωπος παρʼ ᾧ εὑρέθη τὸ κόνδυ, αὐτὸς ἔσται μου παῖς· ὑμεῖς δὲ ἀνάβητε μετὰ σωτηρίας πρὸς τὸν πατέρα ὑμῶν.
\vs{18}Ἐγγίσας δὲ αὐτῷ Ἰούδας εἶπε, δέομαι, κύριε· λαλησάτω ὁ παῖς σου ῥῆμα ἐναντίον σου, καὶ μὴ θυμωθῇς τῷ παιδί σου, ὅτι σὺ εἶ μετὰ Φαραώ.
\vs{19}Κύριε, σὺ ἠρώτησας τοὺς παῖδάς σου, λέγων, εἰ ἔχετε πατέρα ἢ ἀδελφόν.
\vs{20}Καὶ εἴπαμεν τῷ κυρίῳ, ἔστιν ἡμῖν πατὴρ πρεσβύτερος, καὶ παιδίον γήρως νεώτερον αὐτῷ, καὶ ὁ ἀδελφὸς αὐτοῦ ἀπέθανεν, αὐτὸς δὲ μόνος ὑπελείφθη τῇ μητρὶ αὐτοῦ, ὁ δὲ πατὴρ αὐτὸν ἠγάπησεν·
\vs{21}Εἶπας δὲ τοῖς παισί σου, καταγάγετε αὐτὸν πρὸς μέ, καὶ ἐπιμελοῦμαι αὐτοῦ.
\vs{22}Καὶ εἴπαμεν τῷ κυρίῳ, οὐ δυνήσεται τὸ παιδίον καταλιπεῖν τὸν πατέρα αὐτοῦ· ἐὰν δὲ καταλείπῃ τὸν πατέρα, ἀποθανεῖται.
\vs{23}Σὺ δὲ εἶπας τοῖς παισί σου, ἐὰν μὴ καταβῇ ὁ ἀδελφὸς ὑμῶν ὁ νεώτερος μεθʼ ὑμῶν, οὐ προσθήσεσθε ἰδεῖν τὸ πρόσωπόν μου.
\vs{24}Ἐγένετο δὲ ἡνίκα ἀνέβημεν πρὸς τὸν παῖδά σου πατέρα ἡμῶν, ἀπηγγείλαμεν αὐτῷ τὰ ῥήματα τοῦ κυρίου ἡμῶν.
\vs{25}Εἶπε δὲ ὁ πατὴρ ἡμῶν, βαδίσατε πάλιν καὶ ἀγοράσατε ἡμῖν μικρὰ βρώματα.
\vs{26}Ἡμεῖς δὲ εἴπομεν, οὐ δυνησόμεθα καταβῆναι· ἀλλʼ εἰ μὲν ὁ ἀδελφὸς ἡμῶν ὁ νεώτερος καταβαίνει μεθʼ ἡμῶν, καταβησόμεθα· οὐ γὰρ δυνησόμεθα ἰδεῖν τὸ πρόσωπον τοῦ ἀνθρώπου, τοῦ ἀδελφοῦ ἡμῶν τοῦ νεωτέρου μὴ ὄντος μεθʼ ἡμῶν.
\vs{27}Εἶπε δὲ ὁ παῖς σου πατὴρ ἡμῶν πρὸς ἡμᾶς, ὑμεῖς γινώσκετε ὅτι δύο ἔτεκέ μοι ἡ γυνὴ,
\vs{28}καὶ ἐξῆλθεν ὁ εἷς ἀπʼ ἐμοῦ· καὶ εἴπατε ὅτι θηριόβρωτος γέγονεν, καὶ οὐκ ἴδον αὐτὸν ἄχρι νῦν.
\vs{29}Ἐὰν οὖν λάβητε καὶ τοῦτον ἐκ τοῦ προσώπου μου, καὶ συμβῇ αὐτῷ μαλακία ἐν τῇ ὁδῷ, καὶ κατάξετέ μου τὸ γῆρας μετὰ λύπης εἰς ᾅδου.
\vs{30}Νῦν οὖν ἐὰν εἰσπορεύωμαι πρὸς τὸν παῖδά σου, πατέρα δὲ ἡμῶν, καὶ τὸ παιδίον μὴ ᾖ μεθʼ ἡμῶν, ἡ δὲ ψυχὴ αὐτοῦ ἐκκρέμαται ἐκ τῆς τούτου ψυχῆς,
\vs{31}καὶ ἔσται ἐν τῷ ἰδεῖν αὐτὸν μὴ ὂν τὸ παιδίον μεθʼ ἡμῶν, τελευτήσει, καὶ κατάξουσιν οἱ παῖδές σου τὸ γῆρας τοῦ παιδός σου, πατρὸς δὲ ἡμῶν, μετὰ λύπης εἰς ᾅδου.
\vs{32}Ὁ γὰρ παῖς σου παρὰ τοῦ πατρὸς ἐκδέδεκται τὸ παιδίον, λέγων, ἐὰν μὴ ἀγάγω αὐτὸν πρὸς σὲ, καὶ στήσω αὐτὸν ἐνώπιόν σου, ἡμαρτηκὼς ἔσομαι εἰς τὸν πατέρα πάσας τὰς ἡμέρας.
\vs{33}Νῦν οὖν παραμενῶ σοι παῖς ἀντὶ τοῦ παιδίου, οἰκέτης τοῦ κυρίου· τὸ δὲ παιδίον ἀναβήτω μετὰ τῶν ἀδελφῶν αὐτοῦ.
\vs{34}Πῶς γὰρ ἀναβήσομαι πρὸς τὸν πατέρα, τοῦ παιδίου μὴ ὄντος μεθʼ ἡμῶν; ἵνα μὴ ἴδω τὰ κακὰ, ἃ εὑρήσει τὸν πατέρα μου.

\ch{45}
Καὶ οὐκ ἠδύνατο Ἰωσὴφ ἀνέχεσθαι πάντων τῶν παρεστηκότων αὐτῷ, ἀλλʼ εἶπεν, ἐξαποστείλατε πάντας ἀπʼ ἐμοῦ· καὶ οὐ παρειστήκει οὐδεὶς τῷ Ἰωσὴφ, ἡνίκα ἀνεγνωρίζετο τοῖς ἀδελφοῖς αὐτοῦ.
\vs{2}Καὶ ἀφῆκε φωνὴν μετὰ κλαυθμοῦ· ἤκουσαν δὲ πάντες οἱ Αἰγύπτιοι, καὶ ἀκουστὸν ἐγένετο εἰς τὸν οἶκον Φαραώ.
\vs{3}Εἶπε δὲ Ἰωσὴφ πρὸς τοὺς ἀδελφοὺς αὐτοῦ, ἐγώ εἰμι Ἰωσήφ· ἔτι ὁ πατήρ μου ζῇ; καὶ οὐκ ἠδύναντο οἱ ἀδελφοὶ ἀποκριθῆναι αὐτῷ· ἐταράχθησαν γάρ.
\vs{4}Εἶπε δὲ Ἰωσὴφ πρὸς τοὺς ἀδελφοὺς αὐτοῦ, ἐγγίσατε πρὸς μέ· καὶ ἤγγισαν· καὶ εἶπεν, ἐγώ εἰμι Ἰωσὴφ ὁ ἀδελφὸς ὑμῶν, ὃν ἀπέδοσθε εἰς Αἴγυπτον.
\vs{5}Νῦν οὖν μὴ λυπεῖσθε, μηδὲ σκληρὸν ὑμῖν φανήτω, ὅτι ἀπέδοσθέ με ὧδε· εἰς γὰρ ζωὴν ἀπέστειλέ με ὁ Θεὸς ἔμπροσθεν ὑμῶν.
\vs{6}Τοῦτο γὰρ δεύτερον ἔτος λιμὸς ἐπὶ τῆς γῆς, καὶ ἔτι λοιπὰ πέντε ἔτη, ἐν οἷς οὐκ ἔσται ἀροτρίασις, οὐδὲ ἀμητός·
\vs{7}Ἀπέστειλε γάρ με ὁ Θεὸς ἔμπροσθεν ὑμῶν, ὑπολείπεσθαι ὑμῶν κατάλειμμα ἐπὶ τῆς γῆς, καὶ ἐκθρέψαι ὑμῶν κατάλειψιν μεγάλην.
\vs{8}Νῦν οὖν οὐχ ὑμεῖς με ἀπεστάλκατε ὧδε, ἀλλὰ ὁ Θεός· καὶ ἐποίησέ με ὡς πατέρα Φαραὼ, καὶ κύριον παντὸς τοῦ οἴκου αὐτοῦ, καὶ ἄρχοντα πάσης γῆς Αἰγύπτου.
\vs{9}Σπεύσαντες οὖν ἀνάβητε πρὸς τὸν πατέρα μου, καὶ εἴπατε αὐτῷ, τάδε λέγει ὁ υἱός σου Ἰωσήφ· ἐποίησέ με ὁ Θεὸς κύριον πάσης γῆς Αἰγύπτου· κατάβηθι οὖν πρός με, καὶ μὴ μείνῃς·
\vs{10}Καὶ κατοικήσεις ἐν γῇ Γεσὲμ Ἀραβίας· καὶ ἔσῃ ἐγγύς μου σὺ, καὶ οἱ υἱοί σου, καὶ οἱ υἱοὶ τῶν υἱῶν σου, τὰ πρόβατά σου, καὶ αἱ βόες σου, καὶ ὅσα σοι ἐστί.
\vs{11}Καὶ ἐκθρέψω σε ἐκεῖ· ἔτι γὰρ πέντε ἔτη λιμός· ἵνα μὴ ἐκτριβῇς σὺ, καὶ οἱ υἱοί σου, καὶ πάντα τὰ ὑπάρχοντά σου.
\vs{12}Ἰδοὺ οἱ ὀφθαλμοὶ ὑμῶν βλέπουσι, καὶ οἱ ὀφθαλμοὶ Βενιαμεὶν τοῦ ἀδελφοῦ μου, ὅτι τὸ στόμα μου τὸ λαλοῦν πρὸς ὑμᾶς.
\vs{13}Ἀπαγγείλατε οὖν τῷ πατρί μου πᾶσαν τὴν δόξαν μου τὴν ἐν Αἰγύπτῳ, καὶ ὅσα ἴδετε· καὶ ταχύναντες καταγάγετε τὸν πατέρα μου ὧδε.
\vs{14}Καὶ ἐπιπεσὼν ἐπὶ τὸν τράχηλον Βενιαμὶν τοῦ ἀδελφοῦ αὐτοῦ, ἔκλαυσεν ἐπʼ αὐτῷ· καὶ Βενιαμὶν ἔκλαυσεν ἐπὶ τῷ τραχήλῳ αὐτοῦ.
\vs{15}Καὶ καταφιλήσας πάντας τοὺς ἀδελφοὺς αὐτοῦ, ἔκλαυσεν ἐπʼ αὐτοῖς· καὶ μετὰ ταῦτα ἐλάλησαν οἱ ἀδελφοὶ αὐτοῦ πρὸς αὐτόν.

\vs{16}Καὶ διεβοήθη ἡ φωνὴ εἰς τὸν οἶκον Φαραὼ, λέγοντες, ἥκασιν οἱ ἀδελφοὶ Ἰωσήφ· ἐχάρη δὲ Φαραὼ καὶ ἡ θεραπεία αὐτοῦ.
\vs{17}Εἶπε δὲ Φαραὼ πρὸς Ἰωσὴφ, εἰπον τοῖς ἀδελφοῖς σου, τοῦτο ποιήσατε, γεμίσατε τὰ φορεῖα ὑμῶν, καὶ ἀπέλθετε εἰς γῆν Χαναάν.
\vs{18}Καὶ ἀναλαβόντες τὸν πατέρα ὑμῶν, καὶ τὰ ὑπάρχοντα ὑμῶν, ἥκετε πρός με· καὶ δώσω ὑμῖν πάντων τῶν ἀγαθῶν Αἰγύπτου, καὶ φάγεσθε τὸν μυελὸν τῆς γῆς.
\vs{19}Σὺ δὲ ἔντειλαι ταῦτα· λαβεῖν αὐτοῖς ἁμάξας ἐκ γῆς Αἰγύπτου τοῖς παιδίοις ὑμῶν, καὶ ταῖς γυναιξὶν ὑμῶν· καὶ ἀναλαβόντες τὸν πατέρα ὑμῶν παραγίνεσθε.
\vs{20}Καὶ μὴ φείσησθε τοῖς ὀφθαλμοῖς τῶν σκευῶν ὑμῶν· τὰ γὰρ πάντα ἀγαθὰ Αἰγύπτου ὑμῖν ἔσται.
\vs{21}Ἐποίησαν δὲ οὕτως οἱ υἱοὶ Ἰσραήλ· ἔδωκε δὲ Ἰωσὴφ αὐτοῖς ἁμάξας κατὰ τὰ εἰρημένα ὑπὸ Φαραὼ τοῦ βασιλέως· καὶ ἔδωκεν αὐτοῖς ἐπισιτισμὸν εἰς τὴν ὁδόν·
\vs{22}Καὶ πᾶσιν ἔδωκε δισσὰς στολάς· τῷ δὲ Βενιαμὶν ἔδωκε τριακοσίους χρυσοὺς, καὶ πέντε ἐξαλλασσούσας στολάς.
\vs{23}Καὶ τῷ πατρὶ αὐτοῦ ἀπέστειλε κατὰ τὰ αὐτά· καὶ δέκα ὄνους, αἴροντας ἀπὸ πάντων τῶν ἀγαθῶν Αἰγύπτου, καὶ δέκα ἡμιόνους, αἰρούσας ἄρτους τῷ πατρὶ αὐτοῦ εἰς ὁδόν.
\vs{24}Ἐξαπέστειλε δὲ τοὺς ἀδελφοὺς αὐτοῦ, καὶ ἐπορεύθησαν· καὶ εἶπεν αὐτοῖς, μὴ ὀργίζεσθε ἐν τῇ ὁδῷ.
\vs{25}Καὶ ἀνέβησαν ἐξ Αἰγυπτου, καὶ ἦλθον εἰς γῆν Χαναὰν πρὸς Ἰακὼβ τὸν πατέρα αὐτῶν.
\vs{26}Καὶ ἀνήγγειλαν αὐτῷ λέγοντες, ὅτι ὁ υἱός σου Ἰωσὴφ ζῇ, καὶ αὐτὸς ἄρχει πάσης γῆς Αἰγύπτου· καὶ ἐξέστη τῇ διανοίᾳ Ἰακὼβ, οὐ γὰρ ἐπίστευσεν αὐτοῖς.
\vs{27}Ἐλάλησαν δὲ αὐτῷ πάντα τὰ ῥηθέντα ὑπὸ Ἰωσὴφ, ὅσα εἶπεν αὐτοῖς. Ἰδὼν δὲ τὰς ἁμάξας, ἃς ἀπέστειλεν Ἰωσὴφ ὥστε ἀναλαβεῖν αὐτὸν, ἀνεζωπύρησε τὸ πνεῦμα Ἰακὼβ τοῦ πατρὸς αὐτῶν.
\vs{28}Εἶπε δὲ Ἰσραὴλ, μέγα μοι ἐστὶν, εἰ ἔτι Ἰωσὴφ ὁ υἱός μου ζῇ· πορευθεὶς ὄψομαι αὐτὸν πρὸ τοῦ ἀποθανεῖν με.

\ch{46}
Ἀπᾴρας δὲ Ἰσραὴλ, αὐτὸς καὶ πάντα τὰ αὐτοῦ, ἦλθεν ἐπὶ τὸ φρέαρ τοῦ ὅρκου· καὶ ἔθυσε θυσίαν τῷ Θεῷ τοῦ πατρὸς αὐτοῦ Ἰσαάκ.
\vs{2}Εἶπε δὲ ὁ Θεὸς τῷ Ἰσραὴλ ἐν ὁράματι τῆς νυκτὸς, εἰπὼν, Ἰακὼβ, Ἰακώβ· ὁ δὲ εἶπε, τί ἐστιν;
\vs{3}Ὁ δὲ λέγει αὐτῷ, ἐγώ εἰμι ὁ Θεὸς τῶν πατέρων σου· μὴ φοβοῦ καταβῆναι εἰς Αἴγυπτον· εἰς γὰρ ἔθνος μέγα ποιήσω σε ἐκεῖ.
\vs{4}Καὶ ἐγὼ καταβήσομαι μετὰ σοῦ εἰς Αἴγυπτον, καὶ ἐγὼ ἀναβιβάσω σε εἰς τέλος· καὶ Ἰωσὴφ ἐπιβαλεῖ τὰς χεῖρας ἐπὶ τοὺς ὀφθαλμούς σου.
\vs{5}Ἀνέστη δὲ Ἰακὼβ ἀπὸ τοῦ φρέατος τοῦ ὅρκου· καὶ ἀνέλαβον οἱ υἱοὶ Ἰσραὴλ τὸν πατέρα αὐτῶν, καὶ τὴν ἀποσκευὴν, καὶ τὰς γυναῖκας αὐτῶν, ἐπὶ τὰς ἁμάξας, ἃς ἀπέστειλεν Ἰωσὴφ ἆραι αὐτόν.
\vs{6}Καὶ ἀναλαβόντες τὰ ὑπάρχοντα αὐτῶν, καὶ πᾶσαν τὴν κτῆσιν, ἣν ἐκτήσαντο ἐκ γῇ Χαναὰν, εἰσῆλθον εἰς Αἴγυπτον, Ἰακὼβ, καὶ πᾶν τὸ σπέρμα αὐτοῦ μετʼ αὐτοῦ.
\vs{7}Υἱοὶ, καὶ υἱοὶ τῶν υἱῶν αὐτοῦ μετʼ αὐτοῦ· θυγατέρες, καὶ θυγατέρες τῶν θυγατέρων αὐτοῦ· καὶ πᾶν τὸ σπέρμα αὐτοῦ ἤγαγεν εἰς Αἴγυπτον·
\vs{8}Ταῦτα δὲ τὰ ὀνόματα τῶν υἱῶν Ἰσραὴλ τῶν εἰσελθόντων εἰς Αἴγυπτον ἅμα Ἰακὼβ τῷ πατρὶ αὐτῶν. Ἰακὼβ καὶ οἱ υἱοὶ αὐτοῦ· πρωτότοκος Ἰακὼβ, Ῥουβήν.
\vs{9}γἱοὶ δὲ Ῥουβὴν, Ἑνὼχ, καὶ Φαλλὸς, Ἀσρὼν, καὶ Χαρμί.
\vs{10}Υἱοὶ δὲ Συμεὼν, Ἰεμουὴλ, καὶ Ἰαμεὶν, καὶ Ἀὼδ, καὶ Ἀχεὶν, καὶ Σαὰρ, καὶ Σαοὺλ υἱὸς τῆς Χανανίτιδος.
\vs{11}Υἱοὶ δὲ Λευὶ, Γηρσὼν, Κὰθ, καὶ Μεραρί.
\vs{12}Υἱοὶ δὲ Ἰούδα, Ἢρ, καὶ Αὐνὰν, καὶ Σηλὼμ, καὶ Φαρὲς, καὶ Ζαρά· ἀπέθανε δὲ Ἢρ καὶ Αὐνὰν ἐν γῇ Χαναάν· ἐγένοντο δὲ υἱοὶ Φαρὲς, Ἑσρὼν, καὶ Ἰεμουήλ.
\vs{13}Υἱοὶ δὲ Ἰσσάχαρ, Θωλὰ, καὶ Φουὰ, καὶ Ἀσοὺμ, καὶ Σαμβράν.
\vs{14}Υἱοὶ δὲ Ζαβουλὼν, Σερὲδ, καὶ Ἀλλὼν, καὶ Ἀχοήλ.
\vs{15}Οὗτοι υἱοὶ Λείας, οὓς ἔτεκε τῷ Ἰακὼβ ἐν Μεσοποταμίᾳ τῆς Συρίας, καὶ Δείναν τὴν θυγατέρα αὐτοῦ· πᾶσαι αἱ ψυχαί, υἱοὶ καὶ θυγατέρες, τριάκοντα τρεῖς.
\vs{16}Υἱοὶ δὲ Γάδ, Σαφὼν, καὶ Ἀγγὶς, καὶ Σαννὶς, καὶ Θασοβὰν, καὶ Ἀηδεὶς, καὶ Ἀροηδεὶς, καὶ Ἀρεηλείς.
\vs{17}Υἱοὶ δὲ Ἀσὴρ, Ἰεμνα, Ἰεσσουὰ, καὶ Ἰεοὺλ, καὶ βαριὰ, καὶ Σάρα ἀδελφὴ αὐτῶν. Υἱοὶ δὲ βαριὰ, Χοβὸρ, καὶ Μελχιΐλ.
\vs{18}Οὗτοι υἱοὶ Ζελφᾶς, ἣν ἔδωκε Λάβαν Λείᾳ τῇ θυγατρὶ αὐτοῦ, ἣ ἔτεκε τούτους τῷ Ἰακὼβ, δεκαὲξ ψυχάς.
\vs{19}Υἱοὶ δὲ Ῥαχὴλ γυναικὸς Ἰακὼβ, Ἰωσὴφ, καὶ Βενιαμείν.
\vs{20}Ἐγένοντο δὲ υἱοὶ Ἰωσὴφ ἐν γῇ Αἰγύπτου, οὓς ἔτεκεν αὐτῷ Ἀσενὲθ θυγάτηρ Πετεφρῆ ἱερέως Ἡλιουπόλεως, τὸν Μανασσῆ, καὶ τὸν Ἐφραίμ· ἐγένοντο δὲ υἱοὶ Μανασσῆ, οὓς ἔτεκεν αὐτῷ ἡ παλλακὴ ἡ Σύρα, τὸν Μαχίρ· Μαχὶρ δὲ ἐγέννησε τὸν Γαλαάδ· υἱοὶ δὲ Ἐφραὶμ ἀδελφοῦ Μανασσῆ, Σουταλαὰμ, καὶ Ταάμ· υἱοὶ δὲ Σουταλαὰμ, Ἐδώμ.
\vs{21}Υἱοὶ δὲ Βενιαμεὶν, Βαλὰ καὶ Βοχὸρ, καὶ Ἀσβήλ. Ἐγένοντο δὲ υἱοὶ Βαλὰ, Γηρὰ, καὶ Νοεμὰν, καὶ Ἀγχὶς, καὶ Ῥὼς, καὶ Μαμφίμ· Γηρὰ δὲ ἐγέννησε τὸν Ἀράδ.
\vs{22}Οὗτοι υἱοὶ Ῥαχὴλ, οὓς ἔτεκε τῷ Ἰακώβ· πᾶσαι αἱ ψυχαὶ δεκαοκτώ.
\vs{23}Υἱοὶ δὲ Δὰν, Ἀσόμ.
\vs{24}Καὶ υἱοὶ Νεφθαλὶ, Ἀσιὴλ, καὶ Γωνὶ, καὶ Ἰσσάαρ, καὶ Σολλήμ.
\vs{25}Οὗτοι υἱοὶ Βαλλὰς, ἣν ἔδωκε Λάβαν Ῥαχὴλ τῇ θυγατρὶ αὐτοῦ, ἣ ἔτεκε τούτους τῷ Ἰακὼβ, πᾶσαι αἱ ψυχαὶ ἑπτά.
\vs{26}Πᾶσαι δὲ ψυχαὶ αἱ εἰσελθοῦσαι μετὰ Ἰακὼβ εἰς Αἴγυπτον, οἱ ἐξελθόντες ἐκ τῶν μηρῶν αὐτοῦ, χωρὶς τῶν γυναικῶν υἱῶν Ἰακὼβ, πᾶσαι αἱ ψυχαὶ, ἑξηκονταέξ·
\vs{27}Υἱοὶ δὲ Ἰωσὴφ οἱ γενόμενοι αὐτῷ ἐν γῇ Αἰγύπτῳ, ψυχαὶ ἐννέα. Πᾶσαι ψυχαὶ οἴκου Ἰακὼβ, αἱ εἰσελθοῦσαι μετὰ Ἰακὼβ εἰς Αἴγυπτον, ψυχαὶ ἑβδομηκονταπέντε.

\vs{28}Τὸν δὲ Ἰούδαν ἀπέστειλεν ἔμπροσθεν αὐτοῦ πρὸς Ἰωσὴφ, συναντῆσαι αὐτῷ καθʼ Ἡρώων πόλιν, εἰς γῆν Ῥαμεσσῆ.
\vs{29}Ζεύξας δὲ Ἰωσὴφ τὰ ἅρματα αὐτοῦ, ἀνέβη εἰς συνάντησιν Ἰσραὴλ τῷ πατρὶ αὐτοῦ, καθʼ Ἡρώων πόλιν· καὶ ὀφθεὶς αὐτῷ ἐπέπεσεν ἐπὶ τὸν τράχηλον αὐτοῦ, καὶ ἔκλαυσε κλαυθμῷ πίονι.
\vs{30}Καὶ εἶπεν Ἰσραήλ πρὸς Ἰωσὴφ, ἀποθανοῦμαι ἀπὸ τοῦ νῦν, ἐπεὶ ἑώρακα τὸ πρόσωπόν σου· ἔτι γὰρ σὺ ζῇς.
\vs{31}Εἶπε δὲ Ἰωσὴφ πρὸς τοὺς ἀδελφοὺς αὐτοῦ, ἀναβὰς ἀπαγγελῶ τῷ Φαραῷ, καὶ ἐρῶ αὐτῷ, οἱ ἀδελφοί μου, καὶ ὁ οἶκος τοῦ πατρός μου, οἳ ἦσαν ἐν γῇ χαναὰν, ἥκασι πρός με.
\vs{32}Οἱ δὲ ἄνδρες εἰσὶ ποιμένες· ἄνδρες γὰρ κτηνοτρόφοι ἦσαν· καὶ τὰ κτήνη, καὶ τοὺς βόας, καὶ πάντα τὰ αὐτῶν ἀγηόχασιν.
\vs{33}Ἐὰν οὖν καλέσῃ ὑμᾶς Φαραὼ, καὶ εἴπῃ ὑμῖν, τί τὸ ἔργον ὑμῶν ἐστι;
\vs{34}Ἐρεῖτε, ἄνδρες κτηνοτρόφοι ἐσμὲν οἱ παῖδές σου ἐκ παιδὸς ἕως τοῦ νῦν, καὶ ἡμεῖς, καὶ οἱ πατέρες ἡμῶν· ἵνα κατοικήσητε ἐν γῇ Γεσὲμ Ἀραβίας· βδέλυγμα γάρ ἐστιν Αἰγυπτίοις πᾶς ποιμὴν προβάτων.

\ch{47}
Ἐλθῶν δὲ Ἰωσὴφ ἀπήγγειλε τῷ Φαραῶ, λέγων, ὁ πατὴρ μου, καὶ οἱ ἀδελφοί μου, καὶ τὰ κτήνη, καὶ οἱ βόες αὐτῶν, καὶ πάντα τὰ αὐτῶν, ἦλθον ἐκ γῆς Χαναάν· καὶ ἰδού εἰσιν ἐν γῇ Γεσέμ.
\vs{2}Ἀπὸ δὲ τῶν ἀδελφῶν αὐτοῦ παρέλαβε πέντε ἄνδρας, καὶ ἔστησεν αὐτοὺς ἐναντίον Φαραώ.
\vs{3}Καὶ εἶπε Φαραὼ τοῖς ἀδελφοῖς Ἰωσὴφ, Τί τὸ ἔργον ὑμῶν; οἱ δὲ εἶπαν τῷ Φαραῷ, ποιμένες προβάτων οἱ παῖδές σου, καὶ ἡμεῖς καὶ οἱ πατέρες ἡμῶν.
\vs{4}Εἶπαν δὲ τῷ Φαραῷ, παροικεῖν ἐν τῇ γῇ ἥκαμεν, οὐ γάρ ἐστι νομὴ τοῖς κτήνεσι τῶν παιδων σου, ἐνίσχυσε γὰρ ὁ λιμὸς ἐν γῇ Χανάαν· νῦν οὖν κατοικήσομεν ἐν γῇ Γεσέμ. Εἶπε δὲ Φαραὼ τῷ Ἰωσὴφ, Κατοικείτωσαν ἐν γῇ Γεσέμ· εἰ δὲ ἐπίστῃ, ὅτι εἰσὶν ἐν αὐτοῖς ἄνδρες δυνατοὶ, κατάστησον αὐτοὺς ἄρχοντας τῶν ἐμῶν κτηνῶν. Ἦλθον δὲ εἰς Αἴγυπτον πρὸς Ἰωσὴφ Ἰακὼβ, καὶ οἱ υἱοὶ αὐτοῦ· καὶ ἤκουσε Φαραὼ βασιλεὺς Αἰγύπτου.
\vs{5}Καὶ εἶπε Φαραὼ πρὸς Ἰωσὴφ, λέγων, ὁ πατήρ σου, καὶ οἱ ἀδελφοί σου, ἥκασι πρὸς σέ.
\vs{6}Ἰδοὺ ἡ γῆ Αἰγύπτου ἐναντίον σου ἐστίν· ἐν τῇ βελτίστῃ γῇ κατοίκισον τὸν πατέρα σου, καὶ τοὺς ἀδελφούς σου.
\vs{7}Εἰσήγαγε δὲ Ἰωσὴφ Ἰακὼβ τὸν πατέρα αὐτοῦ, καὶ ἔστησεν αὐτὸν ἐναντίον Φαραώ· καὶ ηὐλόγησεν Ἰακὼβ τὸν Φαραώ.
\vs{8}Εἶπε δὲ Φαραὼ τῷ Ἰακὼβ, πόσα ἔτη ἡμερῶν τῆς ζωῆς σου;
\vs{9}Καὶ εἶπεν Ἰακὼβ τῷ Φαραῷ, αἱ ἡμέραι τῶν ἐτῶν τῆς ζωῆς μου, ἃς παροικῶ, ἑκατὸν τριάκοντα ἔτη· μικραὶ καὶ πονηραὶ γεγόνασιν αἱ ἡμέραι τῶν ἐτῶν τῆς ζωῆς μου· οὐκ ἀφίκοντο εἰς τὰς ἡμέρας τῶν ἐτῶν τῆς ζῶης τῶν πατέρων μου, ἃς ἡμέρας παρῴκησαν.
\vs{10}Καὶ εὐλογήσας Ἰακὼβ τὸν Φαραὼ, ἐξῆλθεν ἀπʼ αὐτοῦ.

\vs{11}Καὶ κατῴκισεν Ἰωσὴφ τὸν πατέρα αὐτοῦ, καὶ τοὺς ἀδελφοὺς αὐτοῦ, καὶ ἔδωκεν αὐτοῖς κατάσχεσιν ἐν γῇ Αἰγύπτῳ, ἐν τῇ βελτίστῃ γῇ, ἐν γῇ Ῥαμεσσῆ, καθὰ προσέταξε Φαραώ.
\vs{12}Καὶ ἐσιτομέτρει Ἰωσὴφ τῷ πατρὶ αὐτοῦ, καὶ τοῖς ἀδελφοῖς, καὶ παντὶ τῷ οἴκῳ τοῦ πατρὸς αὐτοῦ, σῖτον κατὰ σῶμα.

\vs{13}Σῖτος δὲ οὐκ ἦν ἐν πάσῃ τῇ γῇ, ἐνίσχυσε γὰρ ὁ λιμὸς σφόδρα· ἐξέλιπε δὲ ἡ γῆ Αἰγύπτου καὶ ἡ γῆ Χαναὰν ἀπὸ τοῦ λιμοῦ.
\vs{14}Συνήγαγε δὲ Ἰωσὴφ πᾶν τὸ ἀργύριον τὸ εὑρεθὲν ἐν γῇ Αἰγύπτου καὶ ἐν γῇ Χαναὰν, τοῦ σίτου, οὗ ἠγόραζον, καὶ ἐσιτομέτρει αὐτοῖς, καὶ εἰσήνεγκεν Ἰωσὴφ πᾶν τὸ ἀργύριον εἰς τὸν οἶκον Φαραώ.
\vs{15}Καὶ ἐξέλιπε πᾶν τὸ ἀργύριον ἐκ γῆς Αἰγύπτου καὶ ἐκ γῆς Χαναάν· ἦλθον δὲ πάντες οἱ Αἰγύπτιοι πρὸς Ἰωσὴφ, λέγοντες, δὸς ἡμῖν ἄρτους, καὶ ἱνατί ἀποθνήσκομεν ἐναντίον σου; ἐκλέλοιπε γὰρ τὸ ἀργύριον ἡμῶν.
\vs{16}Εἶπε δὲ αὐτοῖς Ἰωσὴφ, φέρετε τὰ κτήνη ὑμῶν, καὶ δώσω ὑμῖν ἄρτους, ἀντὶ τῶν κτηνῶν ὑμῶν, εἰ ἐκλέλοιπε τὸ ἀργύριον ὑμῶν.
\vs{17}Ἤγαγον δὲ τὰ κτήνη αὐτῶν πρὸς Ἰωσήφ· καὶ ἔδωκεν αὐτοῖς Ἰωσὴφ ἄρτους ἀντὶ τῶν ἵππων, καὶ ἀντὶ τῶν προβάτων, καὶ ἀντὶ τῶν βοῶν, καὶ ἀντὶ τῶν ὄνων· καὶ ἐξέθρεψεν αὐτοὺς ἐν ἄρτοις ἀντὶ πάντων τῶν κτηνῶν αὐτῶν ἐν τῷ ἐνιαυτῷ ἐκείνῳ.
\vs{18}Ἐξῆλθε δὲ τὸ ἔτος ἐκεῖνο, καὶ ἦλθον πρὸς αὐτὸν ἐν τῷ ἔτει τῷ δευτέρῳ, καὶ εἶπαν αὐτῷ, μή ποτε ἐκτριβῶμεν ἀπὸ τοῦ κυρίου ἡμῶν; εἰ γὰρ ἐκλέλοιπε τὸ ἀργύριον ἡμῶν, καὶ τὰ ὑπάρχοντα καὶ τὰ κτήνη πρός σε τὸν κύριον, καὶ οὐχ ὑπολέλειπται ἡμῖν ἐναντίον τοῦ κυρίου ἡμῶν, ἀλλʼ ἢ τὸ ἴδιον σῶμα καὶ ἡ γῆ ἡμῶν,
\vs{19}ἵνα οὖν μὴ ἀποθάνωμεν ἐναντίον σου, καὶ ἡ γῆ ἐρημωθῇ, κτῆσαι ἡμᾶς καὶ τὴν γῆν ἡμῶν ἀντὶ ἄρτων, καὶ ἐσόμεθα ἡμεῖς καὶ ἡ γῆ ἡμῶν παῖδες τῷ Φαραώ· δὸς σπέρμα, ἵνα σπείρωμεν, καὶ ζῶμεν καὶ μὴ ἀποθάνωμεν, καὶ ἡ γῆ οὐκ ἐρημωθήσεται.
\vs{20}Καὶ ἐκτήσατο Ἰωσὴφ πᾶσαν τὴν γῆν τῶν Αἰγυπτίων τῷ Φαραώ· ἀπέδοντο γὰρ οἱ Αἰγύπτιοι τὴν γῆν αὐτῶν τῷ Φαραώ· ἐπεκράτησε γὰρ αὐτῶν ὁ λιμός· καὶ ἐγένετο ἡ γῆ τῷ Φαραώ.
\vs{21}Καὶ τὸν λαὸν κατεδουλώσατο αὐτῷ εἰς παῖδας, ἀπʼ ἄκρων ὁρίων Αἰγύπτου ἕως τῶν ἄκρων,
\vs{22}χωρὶς τῆς γῆς τῶν ἱερέων μόνον· οὐκ ἐκτήσατο ταύτην Ἰωσήφ· ἐν δόσει γὰρ ἔδωκε δόμα τοῖς ἱερεῦσι Φαραὼ, καὶ ἤσθιον τὴν δόσιν, ἣν ἔδωκεν αὐτοῖς Φαραώ· διὰ τοῦτο οὐκ ἀπέδοντο τὴν γῆν αὐτῶν.
\vs{23}Εἶπε δὲ Ἰωσὴφ πᾶσι τοῖς Αἰγυπτίοις, ἰδοὺ κέκτημαι ὑμᾶς καὶ τὴν γῆν ὑμῶν σήμερον τῷ Φαραῷ· λάβετε ἑαυτοῖς σπέρμα, καὶ σπείρατε τὴν γῆν.
\vs{24}Καὶ ἔσται τὰ γεννήματα αὐτῆς· καὶ δώσετε τὸ πεμπτὸν μέρος τῷ Φαραώ· τὰ δὲ τέσσαρα μέρη ἔσται ὑμῖν αὐτοῖς εἰς σπέρμα τῇ γῇ, καὶ εἰς βρῶσιν ὑμῖν, καὶ πᾶσι τοῖς ἐν τοῖς οἴκοις ὑμῶν.
\vs{25}Καὶ εἶπαν, σέσωκας ἡμᾶς· εὕρομεν χάριν ἐναντίον τοῦ κυρίου ἡμῶν, καὶ ἐσόμεθα παῖδες τῷ Φαραώ.
\vs{26}Καὶ ἔθετο αὐτοῖς Ἰωσὴφ εἰς πρόσταγμα ἕως τῆς ἡμέρας ταύτης, ἐπὶ γῆς Αἰγύπτου τῷ Φαραὼ ἀποπεμπτοῦν, χωρὶς τῆς γῆς τῶν ἱερέων μόνον· οὐκ ἧν τῷ Φαραώ.

\vs{27}Κατῶκησε δὲ Ἰσραὴλ ἐν γῇ Αἰγύπτῳ ἐπὶ γῆς Γεσὲμ, καὶ ἐκληρονόμησαν ἐπʼ αὐτῆς· καὶ ηὐξήθησαν καὶ ἐπληθύνθησαν σφόδρα.
\vs{28}Ἐπέζησε δὲ Ἰακὼβ ἐν γῇ Αἰγύπτῳ δεκαεπτὰ ἔτη· καὶ ἐγένοντο αἱ ἡμέραι Ἰακὼβ ἐνιαυτῶν τῆς ζωῆς αὐτοῦ ἑκατὸν τεσσαρακονταεπτὰ ἔτη.
\vs{29}Ἤγγισαν δὲ αἱ ἡμέραι Ἰσραὴλ τοῦ ἀποθανεῖν· καὶ ἐκάλεσε τὸν υἱὸν αὐτοῦ Ἰωσὴφ, καὶ εἶπεν αὐτῷ, εἰ εὕρηκα χάριν ἐναντίον σου, ὑπόθες τὴν χεῖρά σου ὑπὸ τὸν μηρόν μου, καὶ ποιήσεις ἐπʼ ἐμὲ ἐλεημοσύνην, καὶ ἀλήθειαν, τοῦ μή με θάψαι ἐν Αἰγύπτῳ·
\vs{30}Ἀλλὰ κοιμηθήσομαι μετὰ τῶν πατέρων μου· καὶ ἀρεῖς με ἐξ Αἰγύπτου, καὶ θάψεις με ἐν τῷ τάφῳ αὐτῶν· ὁ δὲ εἶπεν, ἐγὼ ποιήσω κατὰ τὸ ῥῆμά σου.
\vs{31}Εἶπε δὲ, ὄμοσόν μοι· καὶ ὤμοσεν αὐτῷ· καὶ προσεκύνησεν Ἰσραὴλ ἐπὶ τὸ ἄκρον τῆς ῥάβδου αὐτοῦ.

\ch{48}
Ἐγένετο δὲ μετὰ τὰ ῥήματα ταῦτα, καὶ ἀπηγγέλη τῷ Ἰωσὴφ, ὅτι ὁ πατήρ σου ἐνοχλεῖται· καὶ ἀναλαβὼν τοὺς δύο υἱοὺς αὐτοῦ τὸν Μανασσῆ καὶ τὸν Ἐφραὶμ, ἦλθε πρὸς Ἰακώβ.
\vs{2}Ἀπηγγέλη δὲ τῷ Ἰακὼβ, λέγοντες, ἰδοὺ ὁ υἱός σου Ἰωσὴφ ἔρχεται πρὸς σέ· καὶ ἐνισχύσας Ἰσραὴλ ἐκάθισεν ἐπὶ τὴν κλίνην.
\vs{3}Καὶ εἶπεν Ἰακὼβ τῷ Ἰωσὴφ, ὁ Θεός μου ὤφθη μοι ἐν Λουζᾷ ἐν γῇ Χαναὰν, καὶ εὐλόγησέ με,
\vs{4}καὶ εἶπέ μοι, ἰδοὺ ἐγὼ αὐξανῶ σε, καὶ πληθυνῶ σε, καὶ ποιήσω σε εἰς συναγωγὰς ἐθνῶν· καὶ δώσω σοι τὴν γῆν ταύτην, καὶ τῷ σπέρματί σου μετὰ σὲ, εἰς κατάσχεσιν αἰώνιον.
\vs{5}Νῦν οὖν οἱ δύο υἱοί σου, οἱ γενόμενοί σοι ἐν γῇ Αἰγύπτῳ πρὸ τοῦ με ἐλθεῖν πρός σε εἰς Αἴγυπτον, ἐμοί εἰσιν, Ἐφραὶμ καὶ Μανασσῆ· ὡς Ῥουβὴν καὶ Συμεὼν ἔσονταί μοι.
\vs{6}Τὰ δὲ ἔκγονα, ἃ ἐὰν γεννήσῃς μετὰ ταῦτα, ἔσονται ἐπὶ τῷ ὀνόματι τῶν ἀδελφῶν αὐτῶν· κληθήσονται ἐπὶ τοῖς ἐκείνων κλήροις.
\vs{7}Ἐγὼ δὲ ἡνίκα ἠρχόμην ἐκ Μεσοποταμίας τῆς Συρίας, ἀπέθανε Ῥαχὴλ ἡ μήτηρ σου ἐν γῇ Χαναὰν, ἐγγίζοντός μου κατὰ τὸν ἱππόδρομον Χαβραθὰ τῆς γῆς, τοῦ ἐλθεῖν Ἐφραθά· καὶ κατώρυξα αὐτὴν ἐν τῇ ὁδῷ τοῦ ἱπποδρόμου· αὕτη ἐστὶ Βηθλεέμ.

\vs{8}Ἰδὼν δὲ Ἰσραὴλ τοὺς υἱοὺς Ἰωσὴφ, εἶπε, τίνες σοι οὗτοι;
\vs{9}Εἶπε δὲ Ἰωσὴφ τῷ πατρὶ αὐτοῦ, υἱοί μου εἰσὶν, οὓς ἔδωκε μοι ὁ Θεὸς ἐνταῦθα. Καὶ εἶπεν Ἰακὼβ, προσάγαγέ μοι αὐτοὺς, ἵνα εὐλογήσω αὐτούς.
\vs{10}Οἱ ὀφθαλμοὶ δὲ Ἰσραὴλ ἐβαρυώπησαν ἀπὸ τοῦ γήρως, καὶ οὐκ ἠδύνατο βλέπειν· καὶ ἤγγισεν αὐτοὺς πρὸς αὐτὸν, καὶ ἐφίλησεν αὐτοὺς, καὶ περιέλαβεν αὐτους.
\vs{11}Καὶ εἶπεν Ἰσραὴλ πρὸς Ἰωσὴφ, ἰδοὺ τοῦ προσώπου σου οὐκ ἐστερήθην, καὶ ἰδοὺ ἔδειξέ μοι ὁ Θεὸς καὶ τὸ σπέρμα σου.
\vs{12}Καὶ ἐξήγαγεν αὐτοὺς Ἰωσὴφ ἀπὸ τῶν γονάτων αὐτοῦ· καὶ προσεκύνησαν αὐτῷ ἐπὶ πρόσωπον ἐπὶ τῆς γῆς.
\vs{13}Λαβὼν δὲ Ἰωσὴφ τοὺς δύο υἱοὺς αὐτοῦ, τόν τε Ἐφραὶμ ἐν τῇ δεξιᾷ, ἐξ ἀριστερῶν δὲ Ἰσραὴλ, τὸν δὲ Μανασσῆ ἐξ ἀριστερῶν, ἐκ δεξιῶν δὲ Ἰσραὴλ, ἤγγισεν αὐτοὺς αὐτῷ.
\vs{14}Ἐκτείνας δὲ Ἰσραὴλ τὴν χεῖρα τὴν δεξιὰν, ἐπέβαλεν ἐπὶ τὴν κεφαλὴν Ἐφραὶμ, οὗτος δὲ ἦν ὁ νεώτερος, καὶ τὴν ἀριστερὰν ἐπὶ τὴν κεφαλὴν Μανασσῆ, ἐναλλὰξ τὰς χεῖρας.

\vs{15}Καὶ εὐλόγησεν αὐτοὺς, καὶ εἶπεν, ὁ Θεὸς, ᾧ εὐηρέστησαν οἱ πατέρες μου ἐνώπιον αὐτοῦ, Ἁβραὰμ καὶ Ἰσαὰκ, ὁ Θεὸς ὁ τρέφων με ἐκ νεότητος ἕως τῆς ἡμέρας ταύτης,
\vs{16}ὁ Ἄγγελος ὁ ῥυόμενός με ἐκ πάντων τῶν κακῶν, εὐλογήσαι τὰ παιδία ταῦτα· καὶ ἐπικληθήσεται ἐν αὐτοις τὸ ὄνομά μου, καὶ τὸ ὄνομα τῶν πατέρων μου Ἁβραὰμ καὶ Ἰσαάκ· καὶ πληθυνθείησαν εἰς πλῆθος πολὺ ἐπὶ τῆς γῆς.
\vs{17}Ἰδὼν δὲ Ἰωσὴφ ὅτι ἐπέβαλεν ὁ πατὴρ αὐτοῦ τὴν χεῖρα τὴν δεξιὰν αὐτοῦ ἐπὶ τὴν κεφαλὴν Ἐφραὶμ, βαρὺ αὐτῷ κατεφάνη· καὶ ἀντελάβετο Ἰωσὴφ τῆς χειρὸς τοῦ πατρὸς αὐτοῦ, ἀφελεῖν αὐτὴν ἀπὸ τῆς κεφαλῆς Ἐφραὶμ ἐπὶ τὴν κεφαλὴν Μανασσῆ.
\vs{18}Εἶπε δὲ Ἰωσὴφ τῷ πατρὶ αὐτοῦ, οὐχ οὕτως, πατὴρ, οὗτος γὰρ ὁ πρωτότοκος· ἐπίθες τὴν δεξιάν σου ἐπὶ τὴν κεφαλὴν αὐτοῦ.
\vs{19}Καὶ οὐκ ἠθέλησεν, ἀλλʼ εἶπεν, οἶδα, τέκνον, οἶδα· καὶ οὗτος ἔσται εἰς λαὸν, καὶ οὗτος ὑψωθήσεται· ἀλλʼ ὁ ἀδελφὸς αὐτοῦ ὁ νεώτερος μείζον αὐτοῦ ἔσται, καὶ τὸ σπέρμα αὐτοῦ ἔσται εἰς πλῆθος ἐθνῶν.
\vs{20}Καὶ εὐλόγησεν αὐτοὺς ἐν τῇ ἡμέρᾳ ἐκείνῃ, λέγων, ἐν ὑμῖν εὐλογηθήσεται Ἰσραὴλ, λέγοντες, ποιήσαι σε ὁ Θεὸς ὡς Ἐφραὶμ καὶ ὡς Μανασσῆ· καὶ ἔθηκε τὸν Ἐφραὶμ ἔμπροσθεν τοῦ Μανασσῆ.
\vs{21}Εἶπε δὲ Ἰσραὴλ τῷ Ἰωσὴφ, ἰδοὺ ἐγὼ ἀποθνήσκω· καὶ ἔσται ὁ Θεὸς μεθʼ ὑμῶν, καὶ ἀποστρέψει ὑμᾶς εἰς τὴν γῆν τῶν πατέρων ὑμῶν.
\vs{22}Ἐγὼ δὲ δίδωμί σοι Σίκιμα ἐξαίρετον ὑπὲρ τοὺς ἀδελφούς σου, ἣν ἔλαβον ἐκ χειρὸς Ἀμοῤῥαίων ἐν μαχαίρᾳ μου καὶ τόξῳ.

\ch{49}
Ἐκάλεσε δὲ Ἰακὼβ τοὺς υἱοὺς αὐτοῦ, καὶ εἶπεν αὐτοῖς, συνάχθητε, ἵνα ἀναγγείλω ὑμῖν, τί ἀπαντήσει ὑμῖν ἐπʼ ἐσχάτων τῶν ἡμέρων.
\vs{2}Συνάχθητε, καὶ ἀκούσατέ μου, υἱοὶ Ἰακώβ· ἀκούσατε Ἰσραὴλ, ἀκούσατε τοῦ πατρὸς ὑμῶν.
\vs{3}Ῥουβὴν πρωτότοκός μου, σὺ ἰσχύς μου, καὶ ἀρχὴ τέκνων μου, σκληρὸς φέρεσθαι, καὶ σκληρὸς αὐθάδης.
\vs{4}Ἐξύβρισας ὡς ὕδωρ, μὴ ἐκζέσῃς, ἀνέβης γὰρ ἐπὶ τὴν κοίτην τοῦ πατρός σου· τότε ἐμίανας τὴν στρωμνὴν, οὗ ἀνέβης.
\vs{5}Συμεὼν καὶ Λευὶ ἀδελφοὶ συνετέλεσαν ἀδικίαν ἐξαιρέσεως αὐτῶν·
\vs{6}Εἰς βουλὴν αὐτῶν μὴ ἔλθοι ἡ ψυχή μου, καὶ ἐπὶ τῇ συστάσει αὐτῶν μὴ ἐρίσαι τὰ ἥπατά μου· ὅτι ἐν τῷ θυμῷ αὐτῶν ἀπέκτειναν ἀνθρώπους, καὶ ἐν τῇ ἐπιθυμίᾳ αὐτῶν ἐνευροκόπησαν ταῦρον.
\vs{7}Ἐπικατάρατος ὁ θυμὸς αὐτὼν, ὅτι αὐθάδης· καὶ ἡ μῆνις αὐτῶν, ὅτι ἐσκληρύνθη· διαμεριῷ αὐτοὺς ἐν Ἰακὼβ, καὶ διασπερῷ αὐτοὺς ἐν Ἰσραήλ.
\vs{8}Ἰούδα, σὲ αἰνέσαισαν οἱ ἀδελφοί σου· αἱ χεῖρές σου ἐπὶ νώτου τῶν ἐχθρῶν σου· προσκυνήσουσί σοι οἱ υἱοὶ τοῦ πατρός σου.
\vs{9}Σκύμνος λέοντος Ἰούδα· ἐκ βλαστοῦ, υἱέ μου, ἀνέβης· ἀναπεσὼν ἐκοιμήθης ὡς λέων καὶ ὡς σκύμνος· τίς ἐγερεῖ αὐτόν;
\vs{10}Οὐκ ἐκλείψει ἄρχων ἐξ Ἰούδα, καὶ ἡγούμενος ἐκ τῶν μηρῶν αὐτοῦ, ἕως ἐὰν ἔλθῃ τὰ ἀποκείμενα αὐτῷ· καὶ αὐτὸς προσδοκία ἐθνῶν.
\vs{11}Δεσμεύων πρὸς ἄμπελον τὸν πῶλον αὐτοῦ, καὶ τῇ ἕλικι τὸν πῶλον τῆς ὄνου αὐτοῦ, πλυνεῖ ἐν οἴνῳ τὴν στολὴν αὐτοῦ, καὶ ἐν αἵματι σταφυλῆς τὴν περιβολὴν αὐτοῦ.
\vs{12}Χαροποιοὶ οἱ ὀφθαλμοὶ αὐτοῦ ὑπὲρ οἶνον· καὶ λευκοὶ οἱ ὀδόντες αὐτοῦ ἢ γάλα.
\vs{13}Ζαβουλὼν παράλιος κατοικήσει καὶ αὐτὸς παρʼ ὅρμον πλοίων, καὶ παρατενεῖ ἕως Σιδῶνος.
\vs{14}Ἰσσάχαρ τὸ καλὸν ἐπεθύμησεν, ἀναπαυόμενος ἀνὰ μέσον τῶν κλήρων.
\vs{15}Καὶ ἰδὼν τὴν ἀνάπαυσιν ὅτι καλὴ, καὶ τὴν γῆν ὅτι πίων, ὑπέθηκε τὸν ὦμον αὐτοῦ εἰς τὸ πονεῖν, καὶ ἐγενήθη ἀνὴρ γεωργός.
\vs{16}Δὰν κρινεῖ τὸν λαὸν αὐτοῦ, ὡσεὶ καὶ μία φυλὴ ἐν Ἰσραήλ.
\vs{17}Καὶ γενηθητω Δὰν ὄφις ἐφʼ ὁδοῦ, ἐγκαθήμενος ἐπὶ τρίβου, δάκνων πτέρναν ἵππου· καὶ πεσεῖται ὁ ἱππεὺς εἰς τὰ ὀπίσω,
\vs{18}τὴν σωτηρίαν περιμένων Κυρίου.
\vs{19}Γὰδ, πειρατήριον πειρατεύσει αὐτόν· αὐτὸς δὲ πειράτεύσει αὐτὸν κατὰ πόδας.
\vs{20}Ἀσὴρ, πίων αὐτοῦ ὁ ἄρτος· καὶ αὐτὸς δώσει τρυφὴν ἄρχουσι.
\vs{21}Νεφθαλὶ στέλεχος ἀνειμένον, ἐπιδιδοὺς ἐν τῷ γεννήματι κάλλος.
\vs{22}Υἱὸς ηὐξημένος Ἰωσὴφ, υἱὸς ηὐξημένος μου ζηλωτὸς, υἱός μου νεώτατος· πρός με ἀνάστρεψον.
\vs{23}Εἰς ὃν διαβουλευόμενοι ἐλοιδόρουν, καὶ ἐνεῖχον αὐτῷ κύριοι τοξευμάτων.
\vs{24}Καὶ συνετρίβη μετὰ κράτους τὰ τόξα αὐτῶν· καὶ ἐξελύθη τὰ νεῦρα βραχιόνων χειρὸς αὐτῶν, διὰ χεῖρα δυνάστου Ἰακώβ· ἐκεῖθεν ὁ κατισχύσας Ἰσραὴλ παρὰ Θεοῦ τοῦ πατρός σου.
\vs{25}Καὶ ἐβοήθησέ σοι ὁ Θεὸς ὁ ἐμὸς, καὶ εὐλόγησέ σε εὐλογίαν οὐρανοῦ ἄνωθεν, καὶ εὐλογίαν γῆς ἐχούσης πάντα, εἵνεκεν εὐλογίας μαστῶν καὶ μήτρας,
\vs{26}εὐλογίας πατρός σου καὶ μητρός σου· ὑπερίσχυσεν ὑπὲρ εὐλογίας ὀρέων μονίμων, καὶ ἐπʼ εὐλογίαις θινῶν ἀενάων· ἔσονται ἐπὶ κεφαλὴν Ἰωσὴφ, καὶ ἐπὶ κορυφῆς ὧν ἡγήσατο ἀδελφῶν.
\vs{27}Βενιαμὶν λύκος ἅρπαξ, τὸ πρωϊνὸν ἔδεται ἔτι, καὶ εἰς τὸ ἑσπέρας δίδωσι τροφήν.
\vs{28}Πάντες οὕτοι υἱοὶ Ἰακὼβ δώδεκα· καὶ ταῦτα ἐλάλησεν αὐτοῖς ὁ πατὴρ αὐτῶν· καὶ εὐλόγησεν αὐτούς· ἕκαστον κατὰ τὴν εὐλογίαν αὐτοῦ εὐλόγησεν αὐτούς.
\vs{29}Καὶ εἶπεν αὐτοῖς, ἐγὼ προστίθεμαι πρὸς τὸν ἐμὸν λαόν· θάψτέ με μετὰ τῶν πατέρων μου ἐν τῷ σπηλαίῳ, ὅ ἐστιν ἐν τῷ ἀγρῷ Ἐφρὼν τοῦ Χετταίου,
\vs{30}ἐν τῷ σπηλαίῳ τῷ διπλῷ, τῷ ἀπέναντι Μαμβρῆ, ἐν γῇ Χαναὰν, ὃ ἐκτήσατο Ἁβραὰμ τὸ σπήλαιον παρὰ Ἐφρὼν τοῦ Χετταίου ἐν κτήσει μνημείου.
\vs{31}Ἐκεῖ ἔθαψαν Ἁβραὰμ καὶ Σάῤῥαν τὴν γυναῖκα αὐτοῦ· ἐκεῖ ἔθαψαν Ἰσαὰκ καὶ Ῥεβέκκαν τὴν γυναῖκα αὐτοῦ· ἐκεῖ ἔθαψαν Λείαν·
\vs{32}Ἐν κτήσει τοῦ ἀγροῦ καὶ τοῦ σπηλαίου τοῦ ὄντος ἐν αὐτῷ, παρὰ τῶν υἱῶν Χέτ.
\vs{33}Καὶ κατέπαυσεν Ἰακὼβ ἐπιτάσσων τοῖς υἱοῖς αὐτοῦ· καὶ ἐξᾴρας τοὺς πόδας αὐτοῦ ἐπὶ τὴν κλίνην, ἐξέλιπε· καὶ προσετέθη πρὸς τὸν λαὸν αὐτοῦ.

\ch{50}
Καὶ ἐπιπεσὼν Ἰωσὴφ ἐπὶ πρόσωπον τοῦ πατρὸς αὐτοῦ ἔκλαυσεν αὐτὸν, καὶ ἐφίλησεν αὐτόν.
\vs{2}Καὶ προσέταξεν Ἰωσὴφ τοῖς παισὶν αὐτοῦ τοῖς ἐνταφιασταῖς, ἐνταφιάσαι τὸν πατέρα αὐτοῦ· καὶ ἐνεταφίασαν οἱ ἐνταφιασταὶ τὸν Ἰσραήλ.
\vs{3}Καὶ ἐπλήρωσαν αὐτοῦ τεσσαράκοντα ἡμέρας· οὕτω γὰρ καταριθμοῦνται αἱ ἡμέραι τῆς ταφῆς· καὶ ἐπένθησεν αὐτὸν Αἴγυπτος ἑβδομήκοντα ἡμέρας.
\vs{4}Ἐπεὶ δὲ παρῆλθον αἱ ἡμέραι τοῦ πένθους, ἐλάλησεν Ἰωσὴφ πρὸς τοὺς δυνάστας Φαραὼ, λέγων, εἰ εὗρον χάριν ἐναντίον ὑμῶν, λαλήσατε περὶ ἐμοῦ εἰς τὰ ὦτα Φαραὼ, λέγοντες,
\vs{5}ὁ πατήρ μου ὥρκισέ με, λέγων, ἐν τῷ μνημείῳ, ᾧ ὤρυξα ἐμαυτῷ ἐν γῇ Χαναὰν, ἐκεῖ με θάψεις· νῦν οὖν ἀναβὰς· θάψω τὸν πατέρα μου, καὶ ἐπανελεύσομαι·
\vs{6}Καὶ εἶπε Φαραὼ τῷ Ἰωσὴφ, ἀνάβηθι, θάψον τὸν πατέρα σου, καθάπερ ὥρκισέ σε.
\vs{7}Καὶ ἀνέβη Ἰωσὴφ θάψαι τὸν πατέρα αὐτοῦ· καὶ συνανέβησαν μετʼ αὐτοῦ πάντες οἱ παῖδες Φαραὼ, καὶ οἱ πρεσβύτεροι τοῦ οἴκου αὐτοῦ, καὶ πάντες οἱ πρεσβύτεροι τῆς γῆς Αἰγύπτου,
\vs{8}καὶ πᾶσα ἡ πανοικία Ἰωσὴφ, καὶ οἱ ἀδελφοὶ αὐτοῦ, καὶ πᾶσα ἡ οἰκία ἡ πατρικὴ αὐτοῦ, καὶ ἡ συγγένεια αὐτοῦ· καὶ τὰ πρόβατα, καὶ τοὺς βόας ὑπελίποντο ἐν γῇ Γεσέμ.
\vs{9}Καὶ συνανέβησαν μετʼ αὐτοῦ καὶ ἅρματα καὶ ἱππεῖς, καὶ ἐγένετο ἡ παρεμβολὴ μεγάλη σφόδρα.
\vs{10}Καὶ παρεγένοντο εἰς ἅλωνα Ἀτὰδ, ὅ ἐστι πέραν τοῦ Ἰορδάνου· καὶ ἐκόψαντο αὐτὸν κοπετὸν μέγαν καὶ ἰσχυρὸν σφόδρα· καὶ ἐποίησε τὸ πένθος τῷ πατρὶ αὐτοῦ ἑπτὰ ἡμέρας.
\vs{11}Καὶ εἶδον οἱ κάτοικοι τῆς γῆς Χαναὰν τὸ πένθος ἐπὶ ἅλωνι Ἀτὰδ, καὶ εἶπαν, πένθος μέγα τοῦτό ἐστι τοῖς Αἰγυπτίοις· διὰ τοῦτο ἐκάλεσε τὸ ὄνομα αὐτοῦ, Πένθος Αἰγύπτου, ὅ ἐστι πέραν τοῦ Ἰορδάνου.
\vs{12}Καὶ ἐποίησαν αὐτῷ οὕτως οἱ υἱοὶ αὐτοῦ.
\vs{13}Καὶ ἀνέλαβον αὐτὸν οἱ υἱοὶ αὐτοῦ εἰς γῆν Χαναάν· καὶ ἔθαψαν αὐτὸν εἰς τὸ σπήλαιον τὸ διπλοῦν, ὃ ἐκτήσατο Ἁβραὰμ τὸ σπήλαιον ἐν κτήσει μνημείου παρὰ Ἐφρὼν τοῦ Χετταίου, κατέναντι Μαμβρή.
\vs{14}Καὶ ὑπέστρεψεν Ἰωσὴφ εἰς Αἴγυπτον, αὐτὸς καὶ οἱ ἀδελφοὶ αὐτοῦ, καὶ οἱ συναναβάντες θάψαι τὸν πατέρα αὐτοῦ.

\vs{15}Ἰδόντες δὲ οἱ ἀδελφοὶ Ἰωσὴφ, ὅτι τέθνηκεν ὁ πατὴρ αὐτῶν, εἶπαν, μή ποτε μνησικακήσῃ ἡμῖν Ἰωσὴφ, καὶ ἀνταπόδομα ἀνταποδῷ ἡμῖν πάντα τὰ κακὰ, ἃ ἐνεδειξάμεθα εἰς αὐτὸν.
\vs{16}Καὶ παραγενόμενοι πρὸς Ἰωσὴφ εἶπαν, ὁ πατήρ σου ὥρκισε πρὸ τοῦ τελευτῆσαι αὐτὸν, λέγων,
\vs{17}οὕτως εἴπατε Ἰωσήφ· ἄφες αὐτοῖς τὴν ἀδικίαν καὶ τὴν ἁμαρτίαν αὐτῶν, ὅτι πονηρά σοι ἐνεδείξαντο· καὶ νῦν δέξαι τὴν ἀδικίαν τῶν θεραπόντων τοῦ Θεοῦ τοῦ πατρός σου· καὶ ἔκλαυσεν Ἰωσὴφ λαλούντων αὐτῶν πρὸς αὐτόν.
\vs{18}Καὶ ἐλθόντες πρὸς αὐτὸν εἶπαν, οἵδε ἡμεῖς σοὶ οἰκέται.
\vs{19}Καὶ εἶπεν αὐτοῖς Ἰωσὴφ, μὴ φοβεῖσθε, τοῦ γὰρ Θεοῦ εἰμι ἐγώ.
\vs{20}Ὑμεῖς ἐβουλεύσασθε κατʼ ἐμοῦ εἰς πονηρὰ, ὁ δὲ Θεὸς ἐβουλεύσατο περὶ ἐμοῦ εἰς ἀγαθὰ, ὅπως ἂν γενηθῇ ὡς σήμερον, καὶ τραφῇ λαὸς πολύς.
\vs{21}Καὶ εἶπεν αὐτοῖς, μὴ φοβεῖσθε· ἐγὼ διαθρέψω ὑμᾶς, καὶ τὰς οἰκίας ὑμῶν· καὶ παρεκάλεσεν αὐτοὺς, καὶ ἐλάλησεν αὐτῶν εἰς τὴν καρδίαν.
\vs{22}Καὶ κατῴκησεν Ἰωσὴφ ἐν Αἰγύπτῳ, αὐτὸς καὶ οἱ ἀδελφοὶ αὐτοῦ, καὶ πᾶσα ἡ πανοικία τοῦ πατρὸς αὐτοῦ· καὶ ἔζησεν Ἰωσὴφ ἔτη ἑκατὸν δέκα.
\vs{23}Καὶ εἶδεν Ἰωσὴφ Ἐφραὶμ παιδία, ἕως τρίτης γενεᾶς· καὶ οἱ υἱοὶ Μαχεὶρ τοῦ υἱοῦ Μανασσῆ ἐτέχθησαν ἐπὶ μηρῶν Ἰωσήφ.
\vs{24}Καὶ εἶπεν Ἰωσὴφ τοῖς ἀδελφοῖς αὐτοῦ, λέγων, ἐγὼ ἀποθνήσκω· ἐπισκοπῇ δὲ ἐπισκέψεται ὁ Θεὸς ὑμᾶς, καὶ ἀνάξει ὑμᾶς ἐκ τῆς γῆς ταύτης εἰς τὴν γῆν, ἣν ὤμοσεν ὁ Θεὸς τοῖς πατράσιν ἡμῶν, Ἁβραὰμ, Ἰσαὰκ, καὶ Ἰακώβ.
\vs{25}Καὶ ὥρκισεν Ἰωσὴφ τοὺς υἱοὺς Ἰσραὴλ, λέγων, ἐν τῇ ἐπισκοπῇ ᾗ ἐπισκέψηται ὁ Θεὸς ὑμᾶς, καὶ συνανοίσετε τὰ ὀστᾶ μου ἐντεῦθεν μεθʼ ὑμῶν.
\vs{26}Καὶ ἐτελεύτησεν Ἰωσὴφ ἐτῶν ἑκατὸν δέκα· καὶ ἔθαψαν αὐτὸν, καὶ ἔθηκαν ἐν τῇ σορῷ ἐν Αἰγύπτῳ.


\def\book{ΕΞΟΔΟΣ}
\biblebook{ΕΞΟΔΟΣ}


\lettrine[lines=2, loversize=0.2, nindent=0em, findent=.25em]{\textcolor{bookheadingcolor}{Τ}}{ΑΥΤΑ} τὰ ὀνόματα τῶν υἱῶν Ἰσραὴλ τῶν εἰσπεπορευμένων εἰς Αἴγυπτον ἅμα Ἰακὼβ τῷ πατρὶ αὐτῶν, ἕκαστος πανοικὶ αὐτῶν εἰσήλθοσαν.
\vs{2}Ῥουβὴν, Συμεών, Λευὶ, Ἰούδας,
\vs{3}Ἰσσάχαρ, Ζαβουλὼν, Βενιαμὶν,
\vs{4}Δὰν, καὶ Νεφθαλὶ, Γὰδ, καὶ Ἀσήρ.
\vs{5}Ἰωσὴφ δὲ ἦν ἐν Αἰγύπτῳ· ἦσαν δὲ πᾶσαι ψυχαὶ ἐξ Ἰακὼβ, πέντε καὶ ἑβδομήκοντα.
\vs{6}Ἐτελεύτησε δὲ Ἰωσὴφ, καὶ πάντες οἱ ἀδελφοὶ αὐτοῦ, καὶ πᾶσα ἡ γενεὰ ἐκείνη.
\vs{7}Οἱ δὲ υἱοὶ Ἰσραὴλ ηὐξήθησαν, καὶ ἐπληθύνθησαν, καὶ χυδαῖοι ἐγένοντο, καὶ κατίσχυον σφόδρα σφόδρα· ἐπλήθυνε δὲ ἡ γῆ αὐτούς.
\vs{8}Ἀνέστη δὲ βασιλεὺς ἕτερος ἐπʼ Αἴγυπτον, ὃς οὐκ ᾔδει τὸν Ἰωσήφ.
\vs{9}Εἶπε δὲ τῷ ἔθνει αὐτοῦ, ἰδοὺ τὸ γένος τῶν υἱῶν Ἰσραὴλ μέγα πλῆθος, καὶ ἰσχύει ὑπὲρ ἡμᾶς.
\vs{10}Δεῦτε οὖν κατασοφισώμεθα αὐτοὺς, μήποτε πληθυνθῇ, καὶ ἡνίκα ἂν συμβῇ ἡμῖν πόλεμος, προστεθήσονται καὶ οὗτοι πρὸς τοὺς ὑπεναντίους, καὶ ἐκπολεμήσαντες ἡμᾶς, ἐξελεύσονται ἐκ τῆς γῆς.
\vs{11}Καὶ ἐπέστησεν αὐτοῖς ἐπιστάτας τῶν ἔργων, ἵνα κακώσωσιν αὐτοὺς ἐν τοῖς ἔργοις. Καὶ ᾠκοδόμησαν πόλεις ὀχυρὰς τῷ Φαραῷ, τήν τε Πειθὼ, καὶ Ῥαμεσσῆ, καὶ Ὢν, ἥ ἐστιν Ἡλιού πολις.
\vs{12}Καθότι δὲ αὐτοὺς ἐταπείνουν, τοσούτῳ πλείους ἐγίνοντο, καὶ ἴσχυον σφόδρα σφόδρα· καὶ ἐβδελύσσοντο οἱ Αἰγύπτιοι ἀπὸ τῶν υἱῶν Ἰσραήλ.
\vs{13}Καὶ κατεδυνάστευον οἱ Αἰγύπτιοι τοὺς υἱοὺς Ἰσραὴλ βίᾳ.
\vs{14}Καὶ κατωδύνων αὐτῶν τὴν ζωὴν ἐν τοῖς ἔργοις τοῖς σκληροῖς, τῷ πηλῷ καὶ τῇ πλινθείᾳ, καὶ πᾶσι τοῖς ἔργοις τοῖς ἐν τοῖς πεδίοις, κατὰ πάντα τὰ ἔργα, ὧν κατεδουλοῦντο αὐτοὺς μετὰ βίας.

\vs{15}Καὶ εἶπεν ὁ βασιλεὺς τῶν Αἰγυπτίων ταῖς μαίαις τῶν Ἐβραίων, τῇ μιᾷ αὐτῶν ὄνομα Σεπφώρα, καὶ τὸ ὄνομα τῆς δευτέρας Φουά·
\vs{16}Καὶ εἶπεν, ὅταν μαιοῦσθε τὰς Ἐβραίας, καὶ ὦσι πρὸς τῷ τίκτειν, ἐὰν μὲν ἄρσεν ᾖ, ἀποκτείνατε αὐτό· ἐὰν δὲ θῆλυ, περιποιεῖσθε αὐτό.
\vs{17}Ἐφοβήθησαν δὲ αἱ μαῖαι τὸν Θεὸν, καὶ οὐκ ἐποίησαν καθότι συνέταξεν αὐταῖς ὁ βασιλεὺς Αἰγύπτου, καὶ ἐζωογόνουν τὰ ἄρσενα.
\vs{18}Ἐκάλεσε δὲ ὁ βασιλεὺς Αἰγύπτου τὰς μαίας, καὶ εἶπεν αὐταῖς, τί ὅτι ἐποιήσατε τὸ πρᾶγμα τοῦτο, καὶ ἐζωογονεῖτε τὰ ἄρσενα;
\vs{19}Εἶπαν δὲ αἱ μαῖαι τῷ Φαραῷ, οὐχ ὡς γυναῖκες Αἰγύπτου αἱ Ἐβραῖαι· τίκτουσι γὰρ πρὶν ἢ εἰσελθεῖν πρὸς αὐτὰς τὰς μαίας· καὶ ἔτικτον.
\vs{20}Εὖ δὲ ἐποίει ὁ Θεὸς ταῖς μαίαις· καὶ ἐπλήθυνεν ὁ λαὸς, καὶ ἴσχυε σφόδπα.
\vs{21}Ἐπεὶ δὲ ἐφοβοῦντο αἱ μαῖαι τὸν Θεὸν, ἐποίησαν ἑαυταῖς οἰκίας.
\vs{22}Συνέταξε δὲ Φαραὼ παντὶ τῷ λαῷ αὐτοῦ, λέγων, πᾶν ἄρσεν, ὃ ἐὰν τεχθῇ τοῖς Ἑβραίοις, εἰς τὸν ποταμὸν ῥίψατε, καὶ πᾶν θῆλυ, ζωογονεῖτε αὐτό.

\ch{2}
Ἦν δέ τις ἐκ τῆς φυλῆς Λευὶ, ὃς ἔλαβεν τῶν θυγατέρων Λευί.
\vs{2}Καὶ ἐν γαστρὶ ἔλαβε, καὶ ἔτεκεν ἄρσεν· ἰδόντες δὲ αὐτὸ ἀστεῖον, ἐσπέπασαν αὐτὸ μῆνας τρεῖς.
\vs{3}Ἐπεὶ δὲ οὐκ ἐδύναντο αὐτὸ ἔτι κρύπτειν, ἔλαβεν αὐτῷ ἡ μήτηρ αὐτοῦ θῖβιν, καὶ κατέχρισεν αὐτὴν ἀσφαλτοπίσσῃ, καὶ ἐνέβαλε τὸ παιδίον εἰς αὐτήν, καὶ ἔθηκεν αὐτὴν εἰς τὸ ἕλος παρὰ τὸν ποταμόν.
\vs{4}Καὶ κατεσκόπευεν ἡ ἀδελφὴ αὐτοῦ μακρόθεν, μαθεῖν τί τὸ ἀποβησόμενον αὐτῷ.

\vs{5}Κατέβη δὲ ἡ θυγάτηρ Φαραὼ λούσασθαι ἐπὶ τὸν ποταμὸν, καὶ αἱ ἅβραι αὐτῆς παρεπορεύοντο παρὰ τὸν ποταμόν· καὶ ἰδοῦσα τὴν θίβιν ἐν τῷ ἕλει, ἀποστείλασα τὴν ἅβραν, ἀνείλατο αὐτήν.
\vs{6}Ἀνοίξασα δὲ ὁρᾷ παιδίον κλαῖον ἐν τῇ θίβει· καὶ ἐφείσατο αὐτοῦ ἡ θυγάτηρ Φαραὼ, καὶ ἔφη, ἀπὸ τῶν παιδίων τῶν Ἐβραίων τοῦτο.
\vs{7}Καὶ εἶπεν ἡ ἀδελφὴ αὐτοῦ τῇ θυγατρὶ Φαραὼ, θέλεις καλέσω σοι γυναῖκα τροφεύουσαν ἐκ τῶν Ἐβραίων, καὶ θηλάσει σαι τὸ παιδὶον σοι τὸ παιδίον;
\vs{8}Ἡ δὲ εἶπεν ἡ θυγάτηρ Φαραὼ, πορεύου· ἐλθοῦσα δὲ νεᾶνις ἐκάλεσε τὴν μητέρα τοῦ παιδίου.
\vs{9}Εἶπεν δὲ πρὸς αὐτὴν ἡ θυγάτηρ Φαραὼ, διατήρησόν μοι τὸ παιδίον τοῦτο, καὶ θήλασόν μοι αὐτὸ, ἐγὼ δὲ δώσω σοι τὸν μισθόν· ἔλαβε δὲ ἡ γυνὴ τὸ παιδίον, καὶ ἐθήλαζεν αὐτό.
\vs{10}Ἁδρυνθέντος δὲ τοῦ παιδίου, εἰσήγαγεν αὐτὸ πρὸς τὴν θυγατέρα Φαραὼ, καὶ ἐγενήθη αὐτῇ εἰς υἱόν· ἐπωνόμασε δὲ τὸ ὄνομα αὐτοῦ Μωυσῆν, λέγουσα, ἐκ τοῦ ὕδατος αὐτὸν ἀνειλόμην.

\vs{11}Ἐγένετο δὲ ἐν ταῖς ἡμέραις ταῖς πολλαῖς ἐκείναις μέγας γενόμενος Μωυσῆς, ἐξῆλθε πρὸς τοὺς ἀδελφοὺς αὐτοῦ τοὺς υἱοὺς Ἰσραήλ· κατανοήσας δὲ τὸν πόνον αὐτῶν, ὁρᾷ ἄνθρωπον Αἰγύπτιον τύπτοντα τινὰ Ἐβραῖον, τῶν ἑαυτοῦ ἀδελφῶν τῶν υἱῶν Ἰσραήλ.
\vs{12}Περιβλεψάμενος δὲ ὧδε καὶ ὧδε οὐχ ὁρᾷ οὐδένα, καὶ πατάξας τὸν Αἰγύπτιον, ἔκρυψεν αὐτὸν ἐν τῇ ἄμμῳ.
\vs{13}Ἐξελθὼν δὲ τῇ ἡμέρᾳ τῇ δευτέρᾳ, ὁρᾷ δύο ἄνδρας Ἐβραίους διαπληκτιζομένους· καὶ λέγει τῷ ἀδικοῦντι, διὰ τί σὺ τύπτεις τὸν πλησίον;
\vs{14}Ὁ δὲ εἶπε, τίς σε κατέστησεν ἄρχοντα καὶ δικαστὴν ἐφʼ ἡμῶν; μὴ ἀνελεῖν με σὺ θέλεις, ὃν τρόπον ἀνεῖλες χθὲς τὸν Αἰγύπτιον; ἐφοβήθη δὲ Μωυσῆς, καὶ εἶπεν, εἰ οὕτως ἐμφανὲς γέγονε τὸ ῥῆμα τοῦτο.
\vs{15}Ἤκουσε δὲ Φαραὼ τὸ ῥῆμα τοῦτο, καὶ ἐζήτει ἀνελεῖν Μωυσῆν. Ἀνεχώρησε δὲ Μωυσῆς ἀπὸ προσώπου Φαραὼ, καὶ ᾤκησεν ἐν γῇ Μαδιάμ· ἐλθὼν δὲ εἰς γῆν Μαδιὰμ, ἐκάθισεν ἐπὶ τοῦ φρέατος.
\vs{16}Τῷ δὲ ἱερεῖ Μαδιὰμ ἦσαν ἑπτὰ θυγατέρες, ποιμαίνουσαι τὰ πρόβατα τοῦ πατρὸς αὐτῶν Ἰοθόρ· παραγενόμεναι δὲ ἤντλουν, ἕως ἔπλησαν τὰς δεξαμενάς, ποτίσαι τὰ πρόβατα τοῦ πατρὸς αὐτῶν Ἰοθόρ.
\vs{17}Παραγενόμενοι δὲ οἱ ποιμένες ἐξέβαλλον αὐτάς· ἀναστὰς δὲ Μωυσῆς ἐῤῥύσατο αὐτὰς, καὶ ἤντλησεν αὐταῖς, καὶ ἐπότισε τὰ πρόβατα αὐτῶν.
\vs{18}Παρεγένοντο δὲ πρὸς Ῥαγουὴλ τὸν πατέρα αὐτῶν· ὁ δὲ εἶπεν αὐταῖς, διατί ἐταχύνατε τοῦ παραγενέσθαι σήμερον;
\vs{19}Αἱ δὲ εἶπαν, ἄνθρωπος Αἰγύπτιος ἐῤῥύσατο ἡμᾶς ἀπὸ τῶν ποιμένων, καὶ ἤντλησεν ἡμῖν, καὶ ἐπότισε τὰ πρόβατα ἡμῶν.
\vs{20}Ὁ δὲ εἶπε ταῖς θυγατράσιν αὐτοῦ, καὶ ποῦ ἐστιν; καὶ ἱνατί καταλελοίπατε τὸν ἄνθρωπον; καλέσατε οὖν αὐτὸν, ὅπως φάγῃ ἄρτον.
\vs{21}Κατῳκίσθη δὲ Μωυσῆς παρὰ τῷ ἀνθρώπῳ· καὶ ἐξέδοτο Σεπφώραν τὴν θυγατέρα αὐτοῦ Μωυσῇ γυναῖκα.
\vs{22}Ἐν γαστρὶ δὲ λαβοῦσα ἡ γυνὴ ἔτεκεν υἱόν· καὶ ἐπωνόμασε Μωυσῆς τὸ ὄνομα αὐτοῦ Γηρσάμ, λέγων, ὅτι παροικός εἰμι ἐν γῇ ἀλλοτρίᾳ.
\vs{23}Μετὰ δὲ τὰς ἡμέρας τὰς πολλὰς ἐκείνας, ἐτελεύτησεν ὁ βασιλεὺς Αἰγύπτου, καὶ κατεστέναξαν οἱ υἱοὶ Ἰσραὴλ ἀπὸ τῶν ἔργων, καὶ ἀνεβόησαν· καὶ ἀνέβη ἡ βοὴ αὐτῶν πρὸς τὸν Θεὸν ἀπὸ τῶν ἔργων.
\vs{24}Καὶ εἰσήκουσεν ὁ Θεὸς τὸν στεναγμὸν αὐτῶν· καὶ ἐμνήσθη ὁ Θεὸς τῆς διαθήκης αὐτοῦ τῆς πρὸς Ἀβραὰμ, καὶ Ἰσαὰκ, καὶ Ἰακώβ.
\vs{25}Καὶ ἐπεῖδεν ὁ Θεὸς τοὺς υἱοὺς Ἰσραὴλ, καὶ ἐγνώσθη αὐτοῖς.

\ch{3}
Καὶ Μωυσῆς ἦν ποιμαίνων τὰ πρόβατα Ἰοθὸρ τοῦ γαμβροῦ αὐτοῦ, τοῦ ἱερέως Μαδιὰμ, καὶ ἤγαγεν τὰ πρόβατα ὑπὸ τὴν ἔρημον, καὶ ἦλθεν εἰς τὸ ὄρος Χωρήβ.
\vs{2}Ὤφθη δὲ αὐτῷ Ἄγγελος Κυρίου ἐν πυρὶ φλογὸς ἐκ τοῦ βάτου· καὶ ὁρᾷ ὅτι ὁ βάτος καίεται πυρί, ὁ δὲ βάτος οὐ κατεκαίετο.
\vs{3}Εἶπε δὲ Μωυσῆς, παρελθὼν ὄψομαι τὸ ὅραμα τὸ μέγα τοῦτο, ὅτι οὐ κατακαίεται ὁ βάτος.
\vs{4}Ὡς δὲ εἶδεν Κύριος ὅτι προσάγει ἰδεῖν, ἐκάλεσεν αὐτὸν Κύριος ἐκ τοῦ βάτου, λέγων, Μωυσῆ, Μωυσῆ· ὁ δὲ εἶπε, τί ἐστιν;
\vs{5}Ὁ δὲ εἶπε, μὴ ἐγγίσῃς ὧδε· λύσαι τὸ ὑπόδημα ἐκ τῶν ποδῶν σου, ὁ γὰρ τόπος. ἐν ᾧ σὺ ἕστηκας, γῆ ἁγία ἐστί.
\vs{6}Καὶ εἶπεν, ἐγώ εἰμι ὁ Θεὸς τοῦ πατρός σου, Θεὸς Ἁβραὰμ, καὶ Θεὸς Ἰσαὰκ, καὶ Θεὸς Ἰακώβ· ἀπέστρεψε δὲ Μωυσῆς τὸ πρόσωπον αὐτοῦ, εὐλαβεῖτο γὰρ κατεμβλέψαι ἐνώπιον τοῦ Θεοῦ.
\vs{7}Εἶπε δὲ Κύριος πρὸς Μωυσῆν, ἰδὼν εἶδον τὴν κάκωσιν τοῦ λαοῦ μου τοῦ ἐν Αἰγύπτῳ, καὶ τῆς κραυγῆς αὐτῶν ἀκήκοα ἀπὸ τῶν ἐργοδιωκτῶν· οἶδα γὰρ τὴν ὀδύνην αὐτων,
\vs{8}καὶ κατέβην ἐξελέσθαι αὐτοὺς ἐκ χειρὸς τῶν Αἰγυπτίων, καὶ ἐξαγαγεῖν αὐτοὺς ἐκ τῆς γῆς ἐκείνης, καὶ εἰσαγαγεῖν αὐτοὺς εἰς γῆν ἀγαθὴν καὶ πολλήν, εἰς γῆν ῥέουσαν γάλα καὶ μέλι, εἰς τὸν τόπον τῶν Χαναναίων, καὶ Χετταίων, καὶ Ἀμοῤῥαίων, καὶ Φερεζαίων, καὶ Γεργεσαὶων, καὶ Εὐαίων, καὶ Ἰεβουσαίων.
\vs{9}Καὶ νῦν ἰδοὺ κραυγὴ τῶν υἱῶν Ἰσραὴλ ἥκει πρὸς με· κᾀγὼ ἑώρακα τὸν θλιμμὸν, ὃν οἱ Αἰγύπτιοι θλίβουσιν αὐτούς.
\vs{10}Καὶ νῦν δεῦρο, ἀποστείλω σε πρὸς Φαραὼ βασιλέα Αἰγύπτου, καὶ ἐξάξεις τὸν λαόν μου τοὺς υἱοὺς Ἰσραὴλ ἐκ γῆς Αἰγύπτου.

\vs{11}Καὶ εἶπε Μωυσῆς πρὸς τὸν Θεὸν, τίς εἰμι ἐγὼ, ὅτι πορεύσομαι πρὸς Φαραὼ βασιλέα Αἰγύπτου, καὶ ὅτι ἐξάξω τοὺς υἱοὺς Ἰσραὴλ ἐκ γῆς Αἰγύπτου;
\vs{12}Εἶπε δὲ ὁ Θεὸς Μωυσῇ, λέγων, ὅτι ἔσομαι μετὰ σοῦ· καὶ τοῦτό σοι τὸ σημεῖον ὅτι ἐγώ σε ἐξαποστελῶ, ἐν τῷ ἐξαγαγεῖν σε τὸν λαόν μου ἐξ Αἰγύπτου, καὶ λατρεύσετε τῷ Θεῷ ἐν τῷ ὄρει τοῦτῳ.
\vs{13}Καὶ εἶπε Μωυσῆς πρὸς τὸν Θεὸν, ἰδοὺ ἐγὼ ἐξελεύσομαι πρὸς τοὺς υἱοὺς Ἰσραὴλ, καὶ ἐρῶ πρὸς αὐτοὺς, ὁ Θεὸς τῶν πατέρων ἡμῶν ἀπέσταλκέ με πρὸς ὑμᾶς· ἐρωτήσουσί με, τί ὄνομα αὐτῷ; τί ἐρῶ πρὸς αὐτούς;
\vs{14}Καὶ εἶπεν ὁ Θεὸς πρὸς Μωυσῆν, λέγων, ἐγώ εἰμι ὁ Ὤν· καὶ εἶπεν, οὕτως ἐρεῖς τοῖς υἱοῖς Ἰσραὴλ, ὁ Ὢν ἀπέσταλκέ με πρὸς ὑμᾶς.
\vs{15}Καὶ εἶπεν ὁ Θεὸς πάλιν πρὸς Μωυσῆν, οὕτως ἐρεῖς τοῖς υἱοῖς Ἰσραήλ, Κύριος ὁ Θεὸς τῶν πατέρων ἡμῶν, Θεὸς Ἀβραὰμ, καὶ Θεὸς Ἰσαὰκ, καὶ Θεὸς Ἰακὼβ, ἀπέσταλκέ με πρὸς ὑμᾶς· τοῦτό μου ἐστὶν ὄνομα αἰώνιον, καὶ μνημόσυνον γενεῶν γενεαῖς.
\vs{16}Ἐλθὼν οὐν συνάγαγε τὴν γερουσίαν τῶν υἱῶν Ἰσραὴλ, καὶ ἐρεῖς πρὸς αὐτοὺς, Κύριος ὁ Θεὸς τῶν πατέρων ἡμων ὦπταί μοι, Θεὸς Ἀβραὰμ, καὶ Θεὸς Ἰσαὰκ, καὶ Θεὸς Ἰακὼβ, λέγων, ἐπισκοπῇ ἐπέσκεμμαι ὑμᾶς, καὶ ὅσα συμβέβηκεν ὑμῖν ἐν Αἰγύπτῳ.
\vs{17}Καὶ εἶπεν, ἀναβιβάσω ὑμᾶς ἐκ τῆς κακώσεως τῶν Αἰγυπτίων, εἰς τὴν γῆν τῶν Χαναναίων, καὶ Χετταίων, καὶ Ἀμοῤῥαίων, καὶ Φερεζαίων, καὶ Γεργεσαίων, καὶ Εὑαίων, καὶ Ἰεβουσαίων, εἰς γῆν ῥέουσαν γάλα καὶ μέλι.
\vs{18}Καὶ εἰσακούσονταί σου τῆς φωνῆς· καὶ εἰσελεύσῃ σὺ, καὶ ἡ γερουσία Ἰσραὴλ, πρὸς Φαραὼ βασιλέα Αἰγύπτου, καὶ ἐρεῖς πρὸς αὐτὸν ὁ Θεὸς τῶν Ἑβραίων προσκέκληται ἡμᾶς· πορευσόμεθα οὖν ὁδὸν τριῶν ἡμερῶν εἰς τὴν ἔρημον, ἵνα θύσωμεν τῷ Θεῷ ἡμῶν.
\vs{19}Ἐγὼ δὲ οἶδα ὅτι οὐ προήσεται ὑμᾶς Φαραὼ βασιλεὺς Αἰγύπτου πορευθῆναι, ἐὰν μὴ μετὰ χειρὸς κραταιᾶς.
\vs{20}Καὶ ἐκτείνας τὴν χεῖρα, πατάξω τοὺς Αἰγυπτίους ἐν πᾶσι τοῖς θαυμασίοις μου, οἷς ποιήσω ἐν αὐτοῖς· καὶ μετὰ ταῦτα ἐξαποστελεῖ ὑμᾶς.
\vs{21}Καὶ δώσω χάριν τῷ λαῷ τούτῳ ἐναντίον τῶν Αἰγυπτίων· ὅταν δὲ ἀποτρέχητε, οὐκ ἀπελεύσεσθε κενοί·
\vs{22}Ἀλλὰ αἰτήσει γυνὴ παρὰ γείτονος καὶ συσκήνου αὐτῆς σκεύη ἀργυρᾶ, καὶ χρυσᾶ, καὶ ἱματισμόν· καὶ ἐπιθήσετε ἐπὶ τοὺς υἱοὺς ὑμῶν, καὶ ἐπὶ τὰς θυγατέρας ὑμῶν, καὶ σκυλεύσατε τοὺς Αἰγυπτίους.

\ch{4}
Ἀπεκρίθη δὲ Μωυσῆς, καὶ εἶπεν, ἐὰν μὴ πιστεύσωσί μοι, μηδὲ εἰσακούσωσι τῆς φωνῆς μου, ἐροῦσι γὰρ, ὅτι οὐκ ὦπταί σοι ὁ Θεὸς, τί ἐρῶ πρὸς αὐτούς;
\vs{2}Εἶπε δὲ αὐτῳ Κύριος, τί τοῦτό ἐστι τὸ ἐν τῇ χειρί σου; ὁ δὲ εἶπε, ῥάβδος.
\vs{3}Καὶ εἶπε, ῥίψον αὐτὴν ἐπὶ τὴν γῆν· καὶ ἔῤῥιψεν αὐτὴν ἐπὶ τὴν γῆν, καὶ ἐγένετο ὄφις· καὶ ἔφυγε Μωυσῆς ἀπʼ αὐτοῦ.
\vs{4}Καὶ εἶπε Κύριος πρὸς Μωυσῆν, ἔκτεινον τῆν χεῖρα, καὶ ἐπιλαβοῦ τῆς κέρκου· ἐκτείνας οὖν τὴν χεῖρα ἐπελάβετο τῆς κέρκου· καὶ ἐγένετο ῥάβδος ἐν τῇ χειρὶ αὐτοῦ.
\vs{5}Ἵνα πιστεύσωσί σοι, ὅτι ὦπταί σοι ὁ Θεὸς τῶν πατέρων αὐτῶν, Θεὸς Ἀβραὰμ, καὶ Θεὸς Ἰσαὰκ, καὶ Θεὸς Ἰακώβ.
\vs{6}Εἶπε δὲ αὐτῷ Κύριος πάλιν, εἰσένεγκον τὴν χεῖρά σου εἰς τὸν κόλπον σου· καὶ εἰσήνεγκε τὴν χεῖρα αὐτοῦ εἰς τὸν κόλπον αὐτοῦ· καὶ ἐξήνεγκεν τὴν χεῖρα αὐτοῦ ἐκ τοῦ κόλπου αὐτοῦ, καὶ ἐγενήθη ἡ χεὶρ αὐτοῦ ὡσεὶ χιών.
\vs{7}Καὶ εἶπεν πάλιν, εἰσένεγκον τὴν χεῖρά σου εἰς τὸν κόλπον σου· καὶ εἰσήνεγκε τὴν χεῖρα εἰς τὸν κόλπον αὐτοῦ· καὶ ἐξήνεγκεν αὐτὴν ἐκ τοῦ κόλπου αὐτοῦ, καὶ πάλιν ἀπεκατέστη εἰς τὴν χρόαν τῆς σαρκὸς αὐτῆς.
\vs{8}Ἐὰν δὲ μὴ πιστεύσωσί σοι, μηδὲ εἰσακούσωσι τῆς φωνῆς τοῦ σημείου τοῦ πρώτου, πιστεύσουσί σοι τῆς φωνῆς τοῦ σημείου τοῦ δετέρου.
\vs{9}Καὶ ἔσται ἐὰν μὴ πιστεύσωσί σοι τοῖς δυσὶ σημείοις τούτοις, μηδὲ εἰσακούσωσι τῆς φωνῆς σου, λήψῃ ἀπὸ τοῦ ὕδατος τοῦ ποταμοῦ, καὶ ἐκχεεῖς ἐπὶ τὸ ξηρόν· καὶ ἔσται τὸ ὕδωρ, ὃ ἐὰν λάβῃς ἀπὸ τοῦ ποταμοῦ, αἷμα ἐπὶ τοῦ ξηροῦ.
\vs{10}Εἶπε δὲ Μωυσῆς πρὸς Κύριον, δέομαι, Κύριε· οὐχ ἱκανός εἰμι πρὸ τῆς χθὲς οὐδὲ πρὸ τῆς τρίτης ἡμέρας, οὐδὲ ἀφʼ οὗ ἤρξω λαλεῖν τῷ θεράποντί σου· ἰσχνόφωνος καὶ βραδύγλωσσος ἐγώ εἰμι.
\vs{11}Εἶπε δὲ Κύριος πρὸς Μωυσῆν, τίς ἔδωκε στόμα ἀνθρώπῳ; καὶ τίς ἐποιήσε δύσκωφον καὶ κωφὸν, βλέποντα καὶ τυφλόν; οὐκ ἐγὼ ὁ Θεός;
\vs{12}Καὶ νῦν πορεύου, καὶ ἐγὼ ἀνοίξω τὸ στόμα σου, καὶ συμβιβάσω σε ὃ μέλλεις λαλῆσαι.
\vs{13}Καὶ εἶπε Μωυσῆς, δέομαι, Κύριε· προχείρισαι δυνάμενον ἄλλον, ὃν ἀποστελεῖς.
\vs{14}Καὶ θυμωθεὶς ὀργῇ Κύριος ἐπὶ Μωυσῆν, εἶπεν, οὐκ ἰδοὺ Ἀαρὼν ὁ ἄδελφός σου ὁ Λευίτης; ἐπίσταμαι ὅτι λαλῶν λαλήσει αὐτός σοι· καὶ ἰδοὺ αὐτὸς ἐξελεύσεται εἰς συνάντησίν σοι, καὶ ἰδών σε χαρήσεται ἐν ἑαυτῷ.
\vs{15}Καὶ ἐρεῖς πρὸς αὐτὸν, καὶ δώσεις τὰ ῥήματά μου εἰς τὸ στόμα αὐτοῦ, καὶ ἐγὼ ἀνοίξω τὸ στόμα σου καὶ τὸ στόμα αὐτοῦ, καί συμβιβάσω ὑμᾶς ἃ ποιήσετε.
\vs{16}Καὶ αὐτός σοι λαλήσει πρὸς τὸν λαὸν, καὶ αὐτὸς ἔσται σου στόμα· σὺ δὲ αὐτῷ ἔσῃ τὰ πρὸς τὸν Θεόν.
\vs{17}Καὶ τὴν ῥάβδον ταύτην, τὴν στραφεῖσαν εἰς ὄφιν, λήψῃ ἐν τῇ χειρί σου, ἐν ᾗ ποιήσεις ἐν αὐτῇ τὰ σημεῖα.

\vs{18}Ἐπορεύθη δὲ Μωυσῆς, καὶ ἀπέστρεψε πρὸς Ἰοθὸρ τὸν γαμβρὸν αὐτοῦ, καὶ λέγει, πορεύσομαι καὶ ἀποστρέψω πρὸς τοὺς ἀδελφούς μου τοὺς ἐν Αἰγύπτῳ, καὶ ὄψομαι εἰ ἔτι ζῶσι· καὶ εἶπεν Ἰοθὸρ Μωυσῇ, βάδιζε ὑγιαίνων· μετὰ δὲ τὰς ἡμέρας τὰς πολλὰς ἐκείνας ἐτελεύτησεν ὁ βασιλεὺς Αἰγύπτου.
\vs{19}Εἶπε δὲ Κύριος πρὸς Μωυσῆν ἐν Μαδιὰμ, βάδιζε, ἄπελθε εἰς Αἴγυπτον, τεθνήκασι γὰρ πάντες οἱ ζητοῦντες σου τὴν ψυχήν.
\vs{20}Ἀναλαβὼν δὲ Μωυσῆς τὴν γυναῖκα καὶ τὰ παιδία, ἀνεβίβασεν αὐτὰ ἐπὶ τὰ ὑποζύγια, καὶ ἐπέστρεψεν εἰς Αἴγυπτον· ἔλαβε δὲ Μωυσῆς τὴν ῥάβδον τὴν παρὰ τοῦ Θεοῦ ἐν τῇ χειρὶ αὐτοῦ.
\vs{21}Εἶπε δὲ Κύριος πρὸς Μωυσῆν, πορευομένου σου καὶ ἀποστρέφοντος εἰς Αἴγυπτον, ὅρα πάντα τὰ τέρατα ἃ δέδωκα ἐν ταῖς χερσί σου, ποιήσεις αὐτὰ ἐναντίον Φαραώ· ἐγὼ δὲ σκληρυνῶ τὴν καρδίαν αὐτοῦ, καὶ οὐ μὴ ἐξαποστείλῃ τὸν λαόν.
\vs{22}Σὺ δὲ ἐρεῖς τῷ Φαραῴ, τάδε λέγει Κύριος, υἱὸς πρωτότοκός μου Ἰσραήλ.
\vs{23}Εἶπα δέ σοι, ἐξαπόστειλον τὸν λαόν μου, ἵνα μοι λατρεύσῃ· εἰ μὲν οὖν μὴ βούλει ἐξαποστεῖλαι αὐτούς, ὅρα οὖν, ἐγὼ ἀποκτένῶ τὸν υἱόν σου τὸν πρωτότοκον.
\vs{24}Ἐγένετο δὲ ἐν τῇ ὁδῷ ἐν τῷ καταλύματι συνήντησεν αὐτῷ Ἄγγελος Κυρίου, καὶ ἐζήτει αὐτὸν ἀποκτεῖναι.
\vs{25}Καὶ λαβοῦσα Σεπφώρα ψῆφον, περιέτεμε τὴν ἀκροβυστίαν τοῦ υἱοῦ αὐτῆς· καὶ προσέπεσε πρὸς τοὺς πόδας αὐτοῦ, καὶ εἶπεν, ἔστη τὸ αἷμα τῆς περιτομῆς τοῦ παιδίου μου.
\vs{26}Καὶ ἀπῆλθεν ἀπʼ αὐτοῦ, διότι εἶπεν, ἔστη τὸ αἷμα τῆς περιτομῆς τοῦ παιδίου μου.
\vs{27}Εἶπε δὲ Κύριος πρὸς Ἀαρὼν, πορεύθητι εἰς συνάντησιν Μωυσῇ εἰς τὴν ἔρημον· καὶ ἐπορεύθη, καὶ συνήντησεν αὐτῷ ἐν τῷ ὄρει τοῦ Θεοῦ, καὶ κατεφίλησαν ἀλλήλους.
\vs{28}Καὶ ἀνήγγειλε Μωυσῆς τῷ Ἀαρὼν πάντας τοὺς λόγους Κυρίου, οὓς ἀπέστειλε, καὶ πάντα τὰ ῥήματα, ἃ ἐνετείλατο αὐτῷ.
\vs{29}Ἐπορεύθη δὲ Μωυσῆς καὶ Ἀαρὼν, καὶ συνήγαγον τὴν γερουσίαν τῶν υἱῶν Ἰσραήλ.
\vs{30}Καὶ ἐλάλησεν Ἀαρὼν πάντα τὰ ῥήματα ταῦτα, ἃ ἐλάλησεν ὁ Θεὸς πρὸς Μωυσῆν, καὶ ἐποίησε τὰ σημεῖα ἐναντίον τοῦ λαοῦ.
\vs{31}Καὶ ἐπίστευσεν ὁ λαὸς καὶ ἐχάρη, ὅτι ἐπεσκέψατο ὁ Θεὸς τοὺς υἱοὺς Ἰσραὴλ, καὶ ὅτι εἶδεν αὐτῶν τὴν θλίψιν· κύψας δὲ ὁ λαὸς προσεκύνησε.

\ch{5}
Καὶ μετὰ ταῦτα εἰσῆλθε Μωυσῆς καὶ Ἀαρὼν πρὸς Φαραὼ, καὶ εἶπαν αὐτῷ, τάδε λέγει Κύριος ὁ Θεὸς Ἰσραὴλ, ἐξαπόστειλον τὸν λαόν μου, ἵνα μοι ἑορτάσωσιν ἐν τῇ ἐρήμῳ.
\vs{2}Καὶ εἶπε Φαραὼ, τίς ἐστιν οὗ εἰσακούσομαι τῆς φωνῆς αὐτοῦ, ὥστε ἐξαποστεῖλαι τοὺς υἱοὺς Ἰσραήλ; οὐκ οἶδα τὸν Κύριον, καὶ τὸν Ἰσραὴλ οὐκ ἐξαποστέλλω.
\vs{3}Καὶ λέγουσιν αὐτῷ, ὁ Θεὸς τῶν Ἐβραίων προσκέκληται ἡμᾶς· πορευσόμεθα οὖν ὁδὸν τριῶν ἡμερῶν εἰς τὴν ἔρημον, ὅπως θύσωμεν Κυρίῳ τῷ Θεῷ ἡμῶν, μή ποτε συναντήσῃ ἡμῖν θάνατος ἢ φόνος.
\vs{4}Καὶ εἶπεν αὐτοῖς ὁ βασιλεὺς Αἰγύπτου, ἱνατί Μωυσῆς καὶ Ἀαρών διαστρέφετε τὸν λαὸν ἀπὸ τῶν ἔργων; ἀπέλθατε ἕκαστος ὑμῶν πρὸς τὰ ἔργα αὐτοῦ.
\vs{5}Καὶ εἶπεν Φαραὼ, ἰδοὺ νῦν πολυπληθεῖ ὁ λαὸς, μὴ οὖν καταπαύσωμεν αὐτοὺς ἀπὸ τῶν ἔργων.
\vs{6}Συνέταξε δὲ Φαραὼ τοῖς ἐργοδιώκταις τοῦ λαοῦ, καὶ τοῖς γραμματεῦσι, λέγων,
\vs{7}οὐκέτι προστεθήσεσθε διδόναι ἄχυρον τῷ λαῷ εἰς τὴν πλινθουργίαν, καθάπερ χθὲς καὶ τρίτην ἡμέραν· ἀλλ αὐτοὶ πορευέσθωσαν καὶ συναγαγέτωσαν ἑαυτοῖς ἄχυρα.
\vs{8}Καὶ τὴν σύνταξιν τῆς πλινθείας, ἧς αὐτοὶ ποιοῦσι, καθʼ ἑκάστην ἡμέραν ἐπιβαλεῖς αὐτοῖς· οὐκ ἀφελεῖς οὐδέν· σχολάζουσι γάρ· διὰ τοῦτο κεκράγασι, λέγοντες, ἐγερθῶμεν, καὶ θύσωμεν τῷ Θεῷ ἡμῶν.
\vs{9}Βαρυνέσθω τὰ ἔργα τῶν ἀνθρώπων τούτων, καὶ μεριμνάτωσαν ταῦτα, καὶ μὴ μεριμνάτωσαν ἐν λόγοις κενοῖς.

\vs{10}Κατέσπευδον δὲ αὐτοὺς οἱ ἐργοδιῶκται καὶ οἱ γραμματεῖς, καὶ ἔλεγον πρὸς τὸν λαὸν, λέγοντες, τάδε λέγει Φαραὼ, οὐκέτι δίδωμι ὑμῖν ἄχυρα.
\vs{11}Αὐτοὶ ὑμεῖς πορευόμενοι συλλέγετε ἑαυτοῖς ἄχυρα, ὅθεν ἐὰν εὕρητε· οὐ γὰρ ἀφαιρεῖται ἀπὸ τῆς συντάξεως ὑμῶν οὐθέν.
\vs{12}Καὶ διεσπάρη ὁ λαὸς ἐν ὅλῃ γῇ Αἰγύπτῳ συναγαγεῖν καλάμην εἰς ἄχυρα.
\vs{13}Οἱ δὲ ἐργοδιῶκται κατέσπευδον αὐτοὺς, λέγοντες, συντελεῖτε τὰ ἔργα τὰ καθήκοντα καθʼ ἡμέραν, καθάπερ καὶ ὅτε τὸ ἄχυρον ἐδίδοτο ὑμῖν.
\vs{14}Καὶ ἐμαστιγώθησαν οἱ γραμματεῖς τοῦ γένους τῶν υἱῶν Ἰσραὴλ, οἱ κατασταθέντες ἐπʼ αὐτοὺς, ὑπὸ τῶν ἐπιστατῶν τοῦ Φαραὼ, λέγοντες, διατί οὐ συνετελέσατε τὰς συντάξεις ὑμῶν τῆς πλινθείας καθάπερ χθὲς καὶ τρίτην ἡμέραν, καὶ τὸ τῆς σήμερον;
\vs{15}Εἰσελθόντες δὲ οἱ γραμματεῖς τῶν υἱῶν Ἰσραὴλ κατεβόησαν πρὸς Φαραὼ, λέγοντες, ἱνατί σὺ οὕτως ποιεῖς τοῖς σοῖς οἰκέταις;
\vs{16}Ἄχυρον οὐ δίδοται τοῖς οἰκέταις σου, καὶ τὴν πλίνθον ἡμῖν λέγουσι ποιεῖν· καὶ ἰδοὺ οἱ παῖδές σου μεμαστίγωνται, ἀδικήσεις οὖν τὸν λαόν σου.
\vs{17}Καὶ εἶπεν αὐτοῖς, σχολάζετε, σχολασταί ἐστε· διὰ τοῦτο λέγετε, πορευθῶμεν, θύσωμεν τῷ Θεῷ ἡμῶν.
\vs{18}Νῦν οὖν πορευθέντες, ἐργάζεσθε· τὸ γὰρ ἄχυρον οὐ δοθήσεται ὑμῖν, καὶ τὴν σύνταξιν τῆς πλινθείας ἀποδώσετε.
\vs{19}Ἑώρων δὲ οἱ γραμματεῖς τῶν υἱῶν Ἰσραὴλ ἑαυτοὺς ἐν κακοῖς, λέγοντες, οὐκ ἀπολείψετε τῆς πλινθείας τὸ καθῆκον τῇ ἡμέρᾳ.
\vs{20}Συνήντησαν δὲ Μωυσῇ καὶ Ἀαρὼν ἐρχομένοις εἰς συνάντησιν αὐτοῖς, ἐκπορευομένων αὐτῶν ἀπὸ Φαραώ,
\vs{21}Καὶ εἶπαν αὐτοῖς, ἴδοι ὁ Θεὸς ὑμᾶς καὶ κρίναι, ὅτι ἐβδελύξατε τὴν ὀσμὴν ἡμῶν ἐναντίον Φαραὼ, καὶ ἐναντίον τῶν θεραπόντων αὐτοῦ, δοῦναι ῥομφαίαν εἰς τὰς χεῖρας αὐτοῦ, ἀποκτεῖναι ἡμᾶς.
\vs{22}Ἐπέστρεψε δὲ Μωυσῆς πρὸς Κύριον, καὶ εἶπε, δέομαι, Κύριε· τί ἐκάκωσας τὸν λαὸν τοῦτον; καὶ ἱνατί ἀπέσταλκάς με;
\vs{23}Καὶ ἀφʼ οὗ πεπόρευμαι πρὸς Φαραὼ, λαλῆσαι ἐπὶ τῷ σῷ ὀνόματι, ἐκάκωσε τὸν λαὸν τοῦτον· καὶ οὐκ ἐῤῥύσω τὸν λαόν σου.

\ch{6}
Καὶ εἶπε Κύριος πρὸς Μωυσῆν, ἤδη ὄψει ἃ ποιήσω τῷ Φαραῷ· ἐν γὰρ χειρὶ κραταίᾳ ἐξαποστελεῖ αὐτούς, καὶ ἐν βραχίονι ὑψηλῷ ἐκβαλεῖ αὐτοὺς ἐκ τῆς γῆς αὐτοῦ.
\vs{2}Ἐλάλησε δὲ ὁ Θεὸς πρὸς Μωυσῆν, καὶ εἶπε πρὸς αὐτὸν, ἐγὼ Κύριος.
\vs{3}Καὶ ὤφθην πρὸς Ἀβραὰμ καὶ Ἰσαὰκ καὶ Ἰακὼβ, Θεὸς ὢν αὐτῶν· καὶ τὸ ὄνομά μου Κύριος οὐκ ἐδήλωσα αὐτοῖς.
\vs{4}Καὶ ἔστησα τὴν διαθήκην μου πρὸς αὐτοὺς, ὥστε δοῦναι αὐτοῖς τὴν γῆν τῶν Χαναναίων, τὴν γῆν ἣν παρῳκήκασιν, ἐν ᾗ καὶ παρῴκησαν ἐπʼ αὐτῆς.
\vs{5}Καὶ ἐγὼ εἰσήκουσα τὸν στεναγμὸν τῶν υἱῶν Ἰσραήλ, ὃν οἱ Αἰγύπτιοι καταδουλοῦνται αὐτούς, καὶ ἐμνήσθην τῆς διαθήκης ὑμῶν.
\vs{6}Βάδιζε, εἶπον τοῖς υἱοῖς Ἰσραὴλ, λέγων, ἐγὼ Κύριος· καὶ ἐξάξω ὑμᾶς ἀπὸ τῆς δυναστείας τῶν Αἰγυπτίων, καὶ ῥύσομαι ὑμᾶς ἐκ τῆς δουλείας, καὶ λυτρώσομαι ὑμᾶς ἐν βραχίονι ὑψηλῷ καὶ κρίσει μεγάλῃ·
\vs{7}Καὶ λὴψομαι ἐμαυτῷ ὑμᾶς λαὸν ἐμοὶ, καὶ ἔσομαι ὑμῶν Θεός· καὶ γνώσεσθε ὅτι ἐγὼ Κύριος ὁ Θεὸς ὑμῶν, ὁ ἐξαγαγὼν ὑμᾶς ἐκ τῆς καταδυναστείας τῶν Αἰγυπτίων.
\vs{8}Καὶ εἰσάξω ὑμᾶς εἰς τὴν γῆν, εἰς ἣν ἐξέτεινα τὴν χεῖρά μου, δοῦναι αὐτὴν τῷ Ἀβραὰμ, καὶ Ἰσαὰκ, καὶ Ἰακὼβ, καὶ δώσω ὑμῖν αὐτὴν ἐν κληρῷ· ἐγὼ Κύριος.
\vs{9}Ἐλάλησε δὲ Μωυσῆς οὕτω τοῖς υἱοῖς Ἰσραήλ· καὶ οὐκ εἰσήκουσαν Μωυσῇ ἀπὸ τῆς ὀλιγοψυχίας, καὶ ἀπὸ τῶν ἔργων τῶν σκληρῶν.
\vs{10}Εἶπε δὲ Κύριος πρὸς Μωυσῆν λέγων,
\vs{11}εἴσελθε, λάλησον Φαραῷ βασιλεῖ Αἰγύπτου, ἵνα ἐξαποστείλῃ τοὺς υἱοὺς Ἰσραὴλ ἐκ τῆς γῆς αὐτοῦ.
\vs{12}Ἐλάλησε δὲ Μωυσῆς ἔναντι Κυρίου, λέγων, ἰδοὺ οἱ υἱοὶ Ἰσραὴλ οὐκ εἰσήκουσάν μου, καὶ πῶς εἰσακούσεταί μου Φαραώ; ἐγὼ δὲ ἄλογός εἰμι.
\vs{13}Εἶπε δὲ Κύριος πρὸς Μωυσῆν καὶ Ἀαρὼν, καὶ συνέταξεν αὐτοῖς πρὸς Φαραὼ βασιλέα Αἰγύπτου, ὥστε ἐξαποστεῖλαι τοὺς υἱοὺς Ἰσραὴλ ἐκ γῆς Αἰγύπτου.

\vs{14}Καὶ οὗτοι ἀρχηγοὶ οἴκων πατριῶν αὐτῶν· υἱοὶ Ῥουβὴν, πρωτοτόκου Ἰσραήλ· Ἑνὼχ, καὶ Φαλλοὺς, Ἀσρὼν, καὶ Χαρμεί· αὕτη ἡ συγγένεια Ῥουβήν.
\vs{15}Καὶ υἱοὶ Συμεών· Ἰεμουὴλ, καὶ Ἰαμεὶμ, καὶ Ἀὼδ, καὶ Ἰαχεὶν, καὶ Σαὰρ, καὶ Σαοὺλ ὁ ἐκ τῆς Φοινίσσης· αὗται αἱ πατριαὶ τῶν υἱῶν Συμεών.
\vs{16}Καὶ ταῦτα τὰ ὀνόματα τῶν υἱῶν Λευὶ κατὰ συγγενείας αὐτῶν· Γεδσὼν, Καὰθ, καὶ Μεραρεί· καὶ τὰ ἔτη τῆς ζωῆς Λευὶ ἑκατὸν τριάκοντα ἑπτά.
\vs{17}Καὶ οὗτοι υἱοὶ Γεδσών· Λοβενεὶ, καὶ Σεμεεί· οἶκοι πατριᾶς αὐτῶν.
\vs{18}Καὶ υἱοὶ Καάθ· Ἀμβρὰμ, καὶ Ἰσσαάρ, Χεβρὼν, καὶ Ὀζειήλ· καὶ τὰ ἔτη τῆς ζωῆς Καὰθ ἑκατὸν τριάκοντα τρία ἔτη.
\vs{19}Καὶ υἱοὶ Μεραρεί· Μοολεὶ, καὶ Ὀμουσεί. οὗτοι οἱ οἶκοι πατριῶν Λευὶ κατὰ συγγενείας αὐτῶν.
\vs{20}Καὶ ἔλαβεν Ἀμβρὰν τὴν Ἰωχαβὲδ, θυγατέρα τοῦ ἀδελφοῦ τοῦ πατρὸς αὐτοῦ, ἑαυτῷ εἰς γυναῖκα· καὶ ἐγέννησεν αὐτῷ τόν τε Ἀαρὼν καὶ τὸν Μωυσῆν, καὶ Μαριὰμ τὴν ἀδελφὴν αὐτῶν· τὰ δὲ ἔτη τῆς ζωῆς Ἀμβρὰμ, ἑκατὸν τριάκοντα δύο ἔτη.
\vs{21}Καὶ υἱοὶ Ἰσσαάρ· Κορὲ, καὶ Ναφὲκ, καὶ Ζεχρεί.
\vs{22}Καὶ υἱοὶ Ὀζειήλ· Μισαὴλ, καὶ Ἐλισαφὰν, καὶ Σεγρεί.
\vs{23}Ἔλαβε δὲ Ἀαρὼν τὴν Ἐλισαβὲθ θυγατέρα Ἀμιναδὰβ, ἀδελφὴν Ναασσὼν, αὐτῷ γυναῖκα· καὶ ἔτεκεν αὐτῷ τόν τε Ναδὰβ, καὶ Ἀβιοὺδ, καὶ τὸν Ἐλεάζαρ, καὶ Ἰθάμαρ.
\vs{24}Υἱοὶ δὲ Κορέ· Ἀσεὶρ, καὶ Ἑλκανὰ, καὶ Ἀβιασάρ· αὗται αἱ γενέσεις Κορέ.
\vs{25}Καὶ Ἐλεάζαρ ὁ τοῦ Ἀαρὼν ἔλαβε τῶν θυγατέρων Φουτιὴλ αὐτῷ γυναῖκα· καὶ ἔτεκεν αὐτῷ τὸν Φινεές· αὗται αἱ ἀρχαὶ πατριᾶς Λευιτῶν, κατὰ γενέσεις αὐτῶν.
\vs{26}Οὗτος Ἀαρὼν καὶ Μωυσῆς, οἷς εἶπεν αὐτοῖς ὁ Θεὸς ἐξαγαγεῖν τοὺς υἱοὺς Ἰσραὴλ ἐκ γῆς Αἰγύπτου σὺν δυνάμει αὐτῶν.
\vs{27}Οὗτοί εἰσιν οἱ διαλεγόμενοι πρὸς Φαραὼ βασιλέα Αἰγύπτου· καὶ ἐξήγαγον τοὺς υἱοὺς Ἰσραὴλ ἐκ γῆς Αἰγύπτου αὐτὸς Ἀαρὼν καὶ Μωυσὴς,
\vs{28}ᾗ ἡμέρᾳ ἐλάλησε Κύριος Μωυσῇ ἐν γῇ Αἰγύπτῳ.
\vs{29}Καὶ ἐλάλησε Κύριος πρὸς Μωυσῆν, λέγων, ἐγὼ Κύριος· λάλησον πρὸς Φαραὼ βασιλέα Αἰγύπτου ὅσα ἐγὼ λέγω πρὸς σέ.
\vs{30}Καὶ εἶπε Μωυσῆς ἐναντίον Κυρίου, ἰδοὺ ἐγὼ ἰσχνόφωνός εἰμι, καὶ πῶς εἰσακούσεταί μου Φαραώ,

\ch{7}
Καὶ εἶπε Κύριος πρὸς Μωυσῆν, λέγων, ἰδοὺ δέδωκά σε θεὸν Φαραὼ, καὶ Ἀαρὼν ὁ ἀδελφός σου ἔσται σου προφήτης.
\vs{2}Σὺ δὲ λαλήσεις αὐτῷ πάντα ὅσα σοι ἐντέλλομαι· ὁ δὲ Ἀαρὼν ὁ ἀδελφός σου λαλήσει πρὸς Φαραὼ, ὥστε ἐξαποστεῖλαι τοὺς υἱοὺς Ἰσραὴλ ἐκ τῆς γῆς αὐτοῦ.
\vs{3}Ἐγὼ δὲ σκληρυνῶ τὴν καρδίαν Φαραὼ, καὶ πληθυνῶ τὰ σημεῖά μου καὶ τὰ τέρατα ἐν γῇ Αἰγύπτῳ.
\vs{4}Καὶ οὐκ εἰσακούσεται ὑμῶν Φαραώ· καὶ ἐπιβαλῶ τὴν χεῖρά μου ἐπʼ Αἴγυπτον, καὶ ἐξάξω σὺν δυνάμει μου τὸν λαόν μου τοὺς υἱοὺς Ἰσραὴλ ἐκ γῆς Αἰγύπτου σὺν ἐκδικήσει μεγάλῃ.
\vs{5}Καὶ γνώσονται πάντες οἱ Αἰγύπτιοι ὅτι ἐγώ εἰμι Κύριος, ἐκτείνων τὴν χεῖρά μου ἐπʼ Αἴγυπτον, καὶ ἐξάξω τοὺς υἱοὺς Ἰσραὴλ ἐκ μέσον αὐτῶν.
\vs{6}Ἐποίησε δὲ Μωυσῆς καὶ Ἀαρὼν καθάπερ ἐνετείλατο αὐτοῖς Κύριος, οὕτως ἐποίησαν.
\vs{7}Μωυσῆς δὲ ἦν ἐτῶν ὀγδοήκοντα, Ἀαρὼν δὲ ὁ ἀδελφὸς αὐτοῦ ἐτῶν ὀγδοήκοντατριῶν, ἡνίκα ἐλάλησεν πρὸς Φαραώ.
\vs{8}Καὶ εἶπε Κύριος πρὸς Μωυσῆν καὶ Ἀαρὼν, λέγων,
\vs{9}καὶ ἐὰν λαλήσῃ πρὸς ὑμᾶς Φαραὼ, λέγων, δότε ἡμῖν σημεῖον ἢ τέρας, καὶ ἐρεῖς Ἀαρὼν τῷ ἀδελφῷ σου, λάβε τὴν ῥάβδον, καὶ ῥίψον ἐπὶ τὴν γῆν ἐναντίον Φαραὼ, καὶ ἐναντίον τῶν θεραπόντων αὐτοῦ, καὶ ἔσται δράκων.
\vs{10}Εἰσῆλθε δὲ Μωυσῆς καὶ Ἀαρὼν ἐναντίον Φαραὼ, καὶ τῶν θεραπόντων αὐτοῦ· καὶ ἐποίησαν οὕτως, καθάπερ ἐνετείλατο αὐτοῖς Κύριος· καὶ ἔῤῥιψεν Ἀαρὼν τὴν ῥάβδον ἐναντίον Φαραὼ, καὶ ἐναντίον τῶν θεραποντων αὐτοῦ, καὶ ἐγένετο δράκων.
\vs{11}Συνεκάλεσε δὲ Φαραὼ τοὺς σοφιστὰς Αἰγύπτου, καὶ τοὺς φαρμακούς· καὶ ἐποίησαν καὶ οἱ ἐπαοιδοὶ τῶν Αἰγυπτίων ταῖς φαρμακίαις αὐτῶν ὡσαύτως.
\vs{12}Καὶ ἔῤῥιψαν ἔκαστος τὴν ῥάβδον αὐτῶν, καὶ ἐγένοντο δράκοντες· καὶ κατέπιεν ἡ ῥάβδος ἡ Ἀαρὼν τὰς ἐκείνων ῥάβδους.
\vs{13}Καὶ κατίσχυσεν ἡ καρδία Φαραὼ, καὶ οὐκ εἰσήκουσεν αὐτῶν, καθάπερ ἐνετείλατο αὐτοῖς Κύριος.

\vs{14}Εἶπε δὲ Κύριος πρὸς Μωυσῆν, βεβάρηται ἡ καρδία Φαραὼ, τοῦ μὴ ἐξαποστεῖλαι τὸν λαόν.
\vs{15}Βάδισον πρὸς Φαραὼ τὸ πρωΐ· ἰδοὺ αὐτὸς ἐκπορεύεται ἐπὶ τὸ ὕδωρ, καὶ ἔσῃ συναντῶν αὐτῷ ἐπὶ τὸ χεῖλος τοῦ ποταμοῦ· καὶ τὴν ῥάβδον τὴν στραφεῖσαν εἰς ὄφιν λήψῃ ἐν τῇ χειρί σου.
\vs{16}Καὶ ἐρεῖς πρὸς αὐτὸν, Κύριος ὁ Θεὸς τῶν Ἐβραίων ἀπέσταλκέ με πρὸς σὲ, λέγων, ἐξαπόστειλον τὸν λαόν μου, ἵνα μοι λατρεύσῃ ἐν τῇ ἐρήμῳ· καὶ ἰδοὺ οὐκ εἰσήκουσας ἕως τούτου.
\vs{17}Τάδε λέγει Κύριος, ἐν τούτῳ γνώσῃ ὅτι ἐγὼ Κύριος· ἰδοὺ ἑγὼ τύπτω τῇ ῥάβδῳ τῇ ἐν τῇ χειρί μου ἐπὶ τὸ ὕδωρ τὸ ἐν τῷ ποταμῷ, καὶ μεταβαλεῖ εἰς αἷμα.
\vs{18}Καὶ οἱ ἰχθύες οἱ ἐν τῷ ποταμῷ τελευτήσουσι· καὶ ἐποζέσει ὁ ποταμὸς, καὶ οὐ δυνήσονται οἱ Αἰγύπτιοι πιεῖν ὕδωρ ἀπὸ τοῦ ποταμοῦ.
\vs{19}Εἶπε δὲ Κύριος πρὸς Μωυσῆν, εἶπὸν Ἀαρὼν τῷ ἀδελφῷ σου, λάβε τὴν ῥάβδον σου ἐν τῇ χειρί σου, καὶ ἔκτεινον τὴν χεῖρά σου ἐπὶ τὰ ὕδατα Αἰγύπτου, καὶ ἐπὶ τοὺς ποταμοὺς αὐτῶν, καὶ ἐπὶ τὰς διώρυγας αὐτῶν, καὶ ἐπὶ τὰ ἕλη αὐτῶν, καὶ ἐπὶ πᾶν συνεστηκὸς ὕδωρ αὐτῶν, καὶ ἔσται αἷμα· καὶ ἐγένετο αἷμα ἐν πάσῃ γῇ Αἰγύπτου, ἔν τε τοῖς ξύλοις καὶ ἐν τοῖς λίθοις.
\vs{20}Καὶ ἐποίησαν οὕτως Μωυσῆς καὶ Ἀαρὼν, καθάπερ ἐνετείλατο αὐτοῖς Κύριος· καὶ ἐπάρας τῇ ῥάβδῳ αὐτοῦ ἐπάταξε τὸ ὕδωρ τὸ ἐν τῷ ποταμῷ ἐναντίον Φαραὼ, καὶ ἐναντίον τῶν θεραπόντων αὐτοῦ· καὶ μετέβαλε πᾶν τὸ ὕδωρ τὸ ἐν τῷ ποταμῷ εἰς αἷμα.
\vs{21}Καὶ οἱ ἰχθύες οἱ ἐν τῷ ποταμῷ ἐτελεύτησαν· καὶ ἐπώζεσεν ὁ ποταμὸς, καὶ οὐκ ἠδύναντο οἱ Αἰγύπτιοι πιεῖν ὕδωρ ἐκ τοῦ ποταμοῦ· καὶ ἦν τὸ αἷμα ἐν πάσῃ γῇ Αἰγύπτου.
\vs{22}Ἐποίησαν δὲ ὡσαύτως καὶ οἱ ἐπαοιδοὶ τῶν Αἰγυπτίων ταῖς φαρμακίαις αὐτῶν· καὶ ἐσκληρύνθη ἡ καρδία Φαραὼ, καὶ οὐκ εἰσήκουσεν αὐτῶν, καθάπερ εἶπε Κύριος.
\vs{23}Ἐπιστραφεὶς δὲ Φαραὼ εἰσῆλθεν εἰς τὸν οἶκον αὐτοῦ· καὶ οὐκ ἐπέστησε τὸν νοῦν αὐτοῦ οὐδὲ ἐπὶ τούτῳ.
\vs{24}Ὤρυξαν δὲ πάντες οἱ Αἰγύπτιοι κύκλῳ τοῦ ποταμοῦ, ὥστε πιεῖν ὕδωρ· καὶ οὐκ ἠδύναντο πιεῖν ὕδωρ ἀπὸ τοῦ ποταμοῦ.
\vs{25}Καὶ ἀνεπληρώθησαν ἑπτὰ ἡμέραι, μετὰ τὸ πατάξαι Κύριον τὸν ποταμόν.

\vs{26}Εἶπε δὲ Κύριος πρὸς Μωυσὴν, εἴσελθε πρὸς Φαραὼ, καὶ ἐρεῖς πρὸς αὐτὸν, τάδε λεγέι Κύριος, ἐξαπόστειλον τὸν λαόν μου, ἵνα μοι λατρεύσωσιν.
\vs{27}Εἰ δὲ μὴ βούλει σὺ ἐξαποστεῖλαι, ἰδοὺ ἐγὼ τύπτω πάντα τὰ ὅριά σου τοῖς βατράχοις.
\vs{28}Καὶ ἐξερεύξεται ὁ ποταμὸς βατράχους· καὶ ἀναβάντες εἰσελεύσονται εἰς τοὺς οἴκους σου, καὶ εἰς τὰ ταμιεῖα τῶν κοιτώνων σου, καὶ ἐπὶ τῶν κλινῶν σου, καὶ ἐπὶ τοὺς οἴκους τῶν θεραπόντων σου, καὶ τοῦ λαοῦ σου, καὶ ἐν τοῖς φυράμασί σου, καὶ ἐν τοῖς κλιβάνοις σου.
\vs{29}Καὶ ἐπὶ σὲ, καὶ ἐπὶ τοὺς θεράποντάς σου, καὶ ἐπὶ τὸν λαόν σου, ἀναβήσονται οἱ βάτραχοι.

\ch{8}Εἶπε δὲ Κύριος πρὸς Μωυσῆν, εἶπον Ἀαρὼν τῷ ἀδελφῷ σου, ἔκτεινον τῇ χειρὶ τὴν ῥάβδον σου ἐπὶ τοὺς ποταμοὺς, καὶ ἐπὶ τὰς διώρυγας, καὶ ἐπὶ τὰ ἕλη, καὶ ἀνάγαγε τοὺς βατράχους.
\vs{2}Καὶ ἐξέτεινεν Ἀαρὼν τὴν χεῖρα ἐπὶ τὰ ὕδατα Αἰγύπτου, καὶ ἀνήγαγε τοὺς βατράχους· καὶ ἀνεβιβάσθη ὁ βάτραχος, καὶ ἐκάλυψε τὴν γῆν Αἰγύπτου.
\vs{3}Ἐποίησαν δὲ ὡσαύτως καὶ οἱ ἐπαοιδοὶ τῶν Αἰγυπτίων ταῖς φαρμακίαις αὐτῶν, καὶ ἀνήγαγον τοὺς βατράχους ἐπὶ γῆν Αἰγύπτου.
\vs{4}Καὶ ἐκάλεσε Φαραὼ Μωυσῆν καὶ Ἀαρὼν, καὶ εἶπεν, εὔξασθε περὶ ἐμοῦ πρὸς Κύριον, καὶ περιελέτω τοὺς βατράχους ἀπʼ ἐμοῦ, καὶ ἀπὸ τοῦ ἐμοῦ λαοῦ· καὶ ἐξαποστελῶ αὐτοὺς, καὶ θύσωσι τῷ Κυρίῳ.
\vs{5}Εἶπε δὲ Μωυσῆς πρὸς Φαραὼ, τάξαι πρὸς με πότε εὔξομαι περὶ σοῦ, καὶ περὶ τῶν θεραπόντων σου, καὶ τοῦ λαοῦ σου, ἀφανίσαι τοὺς βατράχους ἀπὸ σοῦ, καὶ ἀπὸ τοῦ λαοῦ σου, καὶ ἐκ τῶν οἰκιῶν ὑμῶν, πλὴν ἐν τῷ ποταμῷ ὑπολειφθήσονται.
\vs{6}Ὁ δὲ εἶπεν, εἰς αὔριον· εἶπεν οὖν, ὡς εἴρηκας· ἵνα εἰδῇς ὅτι οὐκ ἔστιν ἄλλος πλὴν Κυρίου·
\vs{7}Καὶ περιαιρεθήσονται οἱ βάτραχοι ἀπὸ σοῦ, καὶ ἀπὸ τῶν οἰκιῶν ὑμῶν, καὶ ἀπὸ τῶν ἐπαύλεων, καὶ ἀπὸ τῶν θεραπόντων σου, καὶ ἀπὸ τοῦ λαοῦ σου, πλὴν ἐν τῷ ποταμῷ ὑπολειφθήσονται.
\vs{8}Ἐξῆλθε δὲ Μωυσῆς καὶ Ἀαρὼν ἀπὸ Φαραώ· καὶ ἐβόησε Μωυσῆς πρὸς Κύριον περὶ τοῦ ὁρισμοῦ τῶν βατράχων, ὡς ἐτάξατο Φαραώ.
\vs{9}Ἐποιήσε δὲ Κύριος καθάπερ εἶπε Μωυσῆς· καὶ ἐτελεύτησαν οἱ βάτραχοι ἐκ τῶν οἰκιῶν, καὶ ἐκ τῶν ἐπαύλεων, καὶ ἐκ τῶν ἀγρῶν.
\vs{10}Καὶ συνήγαγον αὐτοὺς, θημωνίας θημωνίας· καὶ ὤζεσεν ἡ γῆ.
\vs{11}Ἰδὼν δὲ Φαραὼ ὅτι γέγονεν ἀνάψυξις, ἐβαρύνθη ἡ καρδία αὐτοῦ, καὶ οὐκ εἰσήκουσεν αὐτῶν, καθάπερ ἐλάλησε Κύριος.
\vs{12}Εἶπε δὲ Κύριος πρὸς Μωυσῆν, εἶπον Ἀαρὼν, ἔκτεινον τῇ χειρὶ τὴν ῥάβδον σου, καὶ πάταξον τὸ χῶμα τῆς γῆς· καὶ ἔσονται σκνίφες ἔν τε τοῖς ἀνθρώποις, καὶ ἐν τοῖς τετράποσι, καὶ ἐν πάσῃ γῇ Αἰγύπτου.
\vs{13}Ἐξέτεινεν οὖν Ἀαρὼν τῇ χειρὶ τὴν ῥάβδον, καὶ ἐπάταξε τὸ χῶμα τῆς γῆς· καὶ ἐγένοντο οἱ σκνίφες ἐν τοῖς ἀνθρώποις, ἔν τε τοῖς τετράποσι, καὶ ἐν παντὶ χώματι τῆς γῆς ἐγένοντο οἱ σκνίφες.
\vs{14}Ἐποίησαν δὲ ὡσαύτως καὶ οἱ ἐπαοιδοὶ ταῖς φαρμακίαις αὐτῶν, ἐξαγαγεῖν τὸν σκνῖφα, καὶ οὐκ ἠδύναντο· καὶ ἐγένοντο οἱ σκνίφες ἔν τε τοῖς ἀνθρώποις, καὶ ἐν τοῖς τετράποσιν.
\vs{15}Εἶπαν οὖν οἱ ἐπαοιδοὶ τῷ Φαραῷ, δάκτυλος Θεοῦ ἐστι τοῦτο· καὶ ἐσκληρύνθη ἡ καρδία Φαραὼ, καὶ οὐκ εἰσήκουσεν αὐτῶν, καθάπερ ἐλάλησε Κύριος.
\vs{16}Εἶπε δὲ Κύριος πρὸς Μωυσῆν, ὄρθρισον τὸ πρωΐ, καὶ στῆθι ἐναντίον Φαραώ· καὶ ἰδοὺ αὐτὸς ἐξελεύσεται ἐπὶ τὸ ὕδωρ· καὶ ἐρεῖς πρὸς αὐτὸν, τάδε λέγει Κύριος, ἐξαπόστειλον τὸν λαόν μου, ἵνα μοι λατρεύσωσιν ἐν τῇ ἐρήμῳ.
\vs{17}Ἐὰν δὲ μὴ βούλει ἐξαποστεῖλαι τὸν λαόν μου, ἰδοὺ ἐγὼ ἐξαποστέλλω ἐπὶ σὲ, καὶ ἐπὶ τοὺς θεράποντάς σου, καὶ ἐπὶ τὸν λαόν σου, καὶ ἐπὶ τοὺς οἴκους ὑμῶν, κυνόμυιαν· καὶ πλησθήσονται αἱ οἰκίαι τῶν Αἰγυπτίων τῆς κυνομυίης, καὶ εἰς τὴν γῆν ἐφʼ ἧς εἰσιν ἐπʼ αὐτῆς.
\vs{18}Καὶ παραδοξάσω ἐν τῇ ἡμέρᾳ ἐκείνῃ τὴν γῆν Γεσὲμ, ἐφʼ ἧς ὁ λαός μου ἔπεστιν ἐπʼ αὐτῆς, ἐφʼ ἧς οὐκ ἔσται ἐκεῖ ἡ κυνόμυια· ἵνα εἰδῇς ὅτι ἐγώ εἰμι Κύριος ὁ Θεὸς πάσης τῆς γῆς.
\vs{19}Καὶ δώσω διαστολὴν ἀνὰ μέσον τοῦ ἐμοῦ λαοῦ, καὶ ἀνὰ μέσον τοῦ σου λαοῦ· ἐν δὲ τῇ αὔριον ἔσται τοῦτο ἐπὶ τῆς γῆς.
\vs{20}Ἐποίησε δὲ Κύριος οὕτως· καὶ παρεγένετο ἡ κυνόμυια πλῆθος εἰς τοὺς οἴκους Φαραὼ, καὶ εἰς τοὺς οἴκους τῶν θεραπόντων αὐτοῦ, καὶ εἰς πᾶσαν τὴν γῆν Αἰγύπτου· καὶ ἐξωλοθρεύθη ἡ γῆ ἀπὸ τῆς κυνομυίης.

\vs{21}Ἐκάλεσε δὲ Φαραὼ Μωυσῆν καὶ Ἀαρὼν, λέγων, ἐλθόντες θύσατε Κυρίῳ τῷ Θεῷ ὑμῶν ἐν τῇ γῇ.
\vs{22}Καὶ εἶπε Μωυσῆς, οὐ δυνατὸν γενέσθαι οὕτως· τὰ γὰρ βδελύγματα τῶν Αἰγυπτίων θύσομεν Κυρίῳ τῷ Θεῷ ἡμῶν· ἐὰν γὰρ θύσωμεν τὰ βδελύγματα τῶν Αἰγυπτίων ἐναντίον αὐτῶν, λιθοβοληθησόμεθα.
\vs{23}Ὁδὸν τριῶν ἡμερῶν πορευσόμεθα εἰς τὴν ἔρημον· καὶ θύσομεν τῷ Θεῷ ἡμῶν, καθάπερ εἶπεν Κύριος ἡμῖν.
\vs{24}Καὶ εἶπε Φαραὼ, ἐγὼ ἀποστέλλω ὑμᾶς, καὶ θύσατε τῷ Θεῷ ὑμῶν ἐν τῇ ἐρήμῳ· ἀλλʼ οὐ μακρὰν ἀποτενεῖτε πορευθῆναι· εὔξασθε οὖν περὶ ἐμοῦ πρὸς Κύριον.
\vs{25}Εἶπε δὲ Μωυσῆς, ὁ δὲ ἐγὼ ἐξελεύσομαι ἀπὸ σοῦ, καὶ εὔξομαι πρὸς τὸν Θεὸν, καὶ ἀπελεύσεται ἡ κυνόμυια καὶ ἀπὸ τῶν θεραπόντων σου, καὶ ἀπὸ τοῦ λαοῦ σου αὔριον· μὴ προσθῇς ἔτι Φαραὼ ἐξαπατῆσαι, τοῦ μὴ ἐξαποστεῖλαι τὸν λαὸν θῦσαι Κυρίῳ.
\vs{26}Ἐξῆλθε δὲ Μωυσῆς ἀπὸ Φαραὼ, καὶ ηὔξατο πρὸς τὸν Θεόν.
\vs{27}Ἐποίησε δὲ Κύριος καθάπερ εἶπε Μωυσῆς· καὶ περιεῖλε τὴν κυνόμυιαν ἀπὸ Φαραὼ, καὶ τῶν θεραπόντων αὐτοῦ, καὶ τοῦ λαοῦ αὐτοῦ, καὶ οὐ κατελείφθη οὐδεμία.
\vs{28}Καὶ ἐβάρυνε Φαραὼ τὴν καρδίαν αὐτοῦ καὶ ἐπὶ τοῦ καιροῦ τούτου, καὶ οὐκ ἠθέλησεν ἐξαποστεῖλαι τὸν λαόν.

\ch{9}
Εἶπε δὲ Κύριος πρὸς Μωυσῆν, εἴσελθε πρὸς Φαραὼ, καὶ ἐρεῖς αὐτῷ, τάδε λέγει Κύριος ὁ Θεὸς τῶν Ἑβραίων, ἐξαπόστειλον τὸν λαόν μου, ἵνα μοι λατρεύσωσι.
\vs{2}Εἰ μὲν οὖν μὴ βούλει ἐξαποστεῖλαι τὸν λαόν μου, ἀλλὰ ἔτι ἐγκρατεῖς αὐτοῦ,
\vs{3}Ἰδοὺ, χεὶρ Κυρίου ἐπέσται ἐν τοῖς κτήνεσί σου τοῖς ἐν τοῖς πεδίοις, ἔν τε τοῖς ἵπποις, καὶ ἐν τοῖς ὑποζυγίοις, καὶ ταῖς καμήλοις, καὶ βουσὶ, καὶ προβάτοις, θάνατος μέγας σφόδρα.
\vs{4}Καὶ παραδοξάσω ἐγὼ ἐν τῷ καιρῷ ἐκείνῳ ἀνὰ μέσον τῶν κτηνῶν τῶν Αἰγυπτίων, καὶ ἀνὰ μέσον τῶν κτηνῶν τῶν υἱῶν Ἰσραήλ· οὐ τελευτήσει ἀπὸ πάντων τῶν τοῦ Ἰσραὴλ υἱῶν ῥητόν.
\vs{5}Καὶ ἔδωκεν ὁ Θεὸς ὅρον, λέγων, ἐν τῇ αὔριον ποιήσει Κύριος τὸ ῥῆμα τοῦτο ἐπὶ τῆς γῆς.
\vs{6}Καὶ ἐποίησε Κύριος τὸ ῥῆμα τοῦτο τῇ ἐπαύριον· καὶ ἐτελεύτησε πάντα τὰ κτήνη τῶν Αἰγυπτίων· ἀπὸ δὲ τῶν κτηνῶν τῶν υἱῶν Ἰσραὴλ οὐκ ἐτελεύτησεν οὐδέν.
\vs{7}Ἰδὼν δὲ Φαραὼ ὅτι οὐκ ἐτελεύτησεν ἀπὸ πάντων τῶν κτηνῶν τῶν υἱῶν Ἰσραὴλ οὐδὲν, ἐβαρύνθη ἡ καρδία Φαραὼ, καὶ οὐκ ἐξαπέστειλε τὸν λαόν.
\vs{8}Εἶπε δὲ Κύριος πρὸς Μωυσῆν καὶ Ἀαρὼν, λέγων, λάβετε ὑμεῖς πληρεῖς τὰς χεῖρας αἰθάλης καμιναίας, καὶ πασάτω Μωυσῆς εἰς τὸν οὐρανὸν ἐναντίον Φαραὼ, καὶ ἐναντίον τῶν θεραπόντων αὐτοῦ.
\vs{9}Καὶ γενηθήτω κονιορτὸς ἐπὶ πᾶσαν τὴν γῆν Αἰγύπτου· καὶ ἔσται ἐπὶ τοὺς ἀνθρώπους, καὶ ἐπὶ τὰ τετράποδα, ἕλκη, φλυκτίδες ἀναζέουσαι ἔν τε τοῖς ἀνθρώποις, καὶ ἐν τοῖς τετράποσιν, ἐν πάσῃ γῇ Αἰγύπτου.
\vs{10}Καὶ ἔλαβεν τὴν αἰθάλην τῆς καμιναίας ἐναντίον Φαραὼ, καὶ ἔπασεν αὐτὴν Μωυσῆς εἰς τὸν οὐρανόν· καὶ ἐγένετο ἕλκη, φλυκτίδες ἀναζέουσαι, ἔν τε τοῖς ἀνθρώποις, καὶ ἐν τοῖς τετράποσι.
\vs{11}Καὶ οὐκ ἠδύναντο οἱ φαρμακοὶ στῆναι ἐναντίον Μωυσῆ διὰ τὰ ἕλκη· ἐγένετο γὰρ τὰ ἕλκη ἐν τοῖς φαρμακοῖς, καὶ ἐν πάσῃ γῇ Αἰγύπτου.
\vs{12}Ἐσκλήρυνε δὲ Κύριος τὴν καρδίαν Φαραὼ, καὶ οὐκ εἰσήκουσεν αὐτῶν, καθὰ συνέταξε Κύριος.
\vs{13}Εἶπε δὲ Κύριος πρὸς Μωυσῆν, ὄρθρισον τὸ πρωῒ, καὶ στῆθι ἐναντίον Φαραὼ, καὶ ἐρεῖς πρὸς αὐτὸν, τάδε λέγει Κύριος ὁ Θεὸς τῶν Ἑβραίων, ἐξαπόστειλον τὸν λαόν μου, ἵνα λατρεύσωσί μοι.
\vs{14}Ἐν τῷ γὰρ νῦν καιρῷ ἐγὼ ἐξαποστέλλω πάντα τὰ συναντήματά μου εἰς τὴν καρδίαν σου, καὶ τῶν θεραπόντων σου, καὶ τοῦ λαοῦ σου, ἵνα εἴδῇς ὅτι οὐκ ἔστιν, ὡς ἐγὼ, ἄλλος ἐν πάσῃ τῇ γῇ.
\vs{15}Νῦν γὰρ ἀποστείλας τὴν χεῖρα πατάξω σε, καὶ τὸν λαόν σου θανατώσω, καὶ ἐκτριβήσῃ ἀπὸ τῆς γῆς.
\vs{16}Καὶ ἕνεκεν τούτου διετηρήθης, ἵνα ἐνδείξωμαι ἐν σοὶ τὴν ἰσχύν μου, καὶ ὅπως διαγγελῇ τὸ ὄνομά μου ἐν πάσῃ τῇ γῇ.
\vs{17}Ἔτι οὖν σὺ ἐνποιῇ τοῦ λαοῦ μου, τοῦ μὴ ἐξαποστεῖλαι αὐτούς;
\vs{18}Ἰδοὺ ἐγὼ ὕω ταύτην τὴν ὥραν αὔριον χάλαζαν πολλὴν σφόδρα, ἥτις τοιαύτη οὐ γέγονεν ἐν Αἰγύπτῳ, ἀφʼ ἧς ἡμέρας ἔκτισται, ἕως τῆς ἡμέρας ταύτης.
\vs{19}Νῦν οὖν κατάσπευσον συναγαγεῖν τὰ κτήνη σου, καὶ ὅσα σοι ἐστὶν ἐν τῷ πεδίῳ· πάντες γὰρ οἱ ἄνθρωποι, καὶ τὰ κτήνη, ὅσα σοί ἐστιν ἐν τῷ πεδίῳ· πὰντες γὰρ οἱ ἄνθρωποι, καὶ τὰ κτήνη, ὅσα ἐὰν εὑρεθῇ ἐν τοῖς πεδίοις, καὶ μὴ εἰσέλθῃ εἰς οἰκίαν, πεσῇ δὲ ἐπʼ αὐτὰ ἡ χάλαζα, τελευτήσει.
\vs{20}Ὁ φοβούμενος τὸ ῥῆμα Κυρίου τῶν θεραπόντων Φαραὼ, συνήγαγε τὰ κτήνη αὐτοῦ εἰς τοὺς οἴκους.
\vs{21}Ὃς δὲ μὴ πρόσεσχεν τῇ διανοίᾳ εἰς τὸ ῥῆμα Κυρίου, ἀφῆκε τὰ κτήνη ἐν τοῖς πεδίοις.

\vs{22}Εἶπε δὲ Κύριος πρὸς Μωυσῆν, ἔκτεινον τὴν χεῖρά σου εἰς τὸν οὐρανὸν, καὶ ἔσται χάλαζα ἐπὶ πᾶσαν γῆν Αἰγύπτου, ἐπί τε τοὺς ἀνθρώπους, καὶ τὰ κτήνη, καὶ ἐπὶ πᾶσαν βοτάνην τὴν ἐπὶ τῆς γῆς.
\vs{23}Ἐξέτεινε δὲ Μωυσῆς τὴν χεῖρα εἰς τὸν οὐρανὸν, καὶ Κύριος ἔδωκε φωνὰς καὶ χάλαζαν· καὶ διέτρεχε τὸ πῦρ ἐπὶ τῆς γῆς· καὶ ἔβρεξε Κύριος χάλαζαν ἐπὶ πᾶσαν γῆν Αἰγύπτου.
\vs{24}Ἦν δὲ ἡ χάλαζα καὶ τὸ πῦρ φλογίζον ἐν τῇ χαλάζῃ· ἡ δὲ χάλαζα πολλὴ σφόδρα, ἥτις τοιαύτη οὐ γέγονεν ἐν Αἰγύπτῳ, ἀφʼ ἧς ἡμέρας γεγένηται ἐπʼ αὐτῆς ἔθνος.
\vs{25}Ἐπάταξε δὲ ἡ χάλαζα ἐν πάσῃ γῇ Αἰγύπτου, ἀπὸ ἀνθρώπου ἕως κτήνους· καὶ πᾶσαν βοτάνην τὴν ἐν τῷ πεδίῳ ἐπάταξεν ἡ χάλαζα· καὶ πάντα τὰ ξύλα τὰ ἐν τοῖς πεδίοις συνέτριψεν ἡ χάλαζα.
\vs{26}Πλὴν ἐν γῇ Γεσὲμ, οὗ ἦσαν οἱ υἱοὶ Ἰσραὴλ, οὐκ ἐγένετο ἡ χάλαζα.
\vs{27}Ἀποστείλας δὲ Φαραὼ ἐκάλεσε Μωυσῆν καὶ Ἀαρὼν, καὶ εἶπεν αὐτοῖς, ἡμάρτηκα τὸ νῦν· ὁ Κύριος δίκαιος, ἐγὼ δὲ καὶ ὁ λαός μου ἀσεβεῖς.
\vs{28}Εὔξασθε οὖν περὶ ἐμοῦ πρὸς Κύριον, καὶ παυσάσθω τοῦ γενηθῆναι φωνὰς Θεοῦ, καὶ χάλαζαν, καὶ πῦρ· καὶ ἐξαποστελῶ ὑμᾶς, καὶ οὐκέτι προστεθήσεσθε μένειν.
\vs{29}Εἶπε δὲ αὐτῷ Μωυσῆς, ὡς ἂν ἐξέλθω τὴν πόλιν, ἐκπετάσω τὰς χεῖράς μου πρὸς τὸν Κύριον, καὶ αἱ φωναὶ παύσονται, καὶ ἡ χὰλαζα καὶ ὁ ὑετὸς οὐκ ἔσται ἔτι, ἵνα γνῷς ὅτι τοῦ Κυρίου ἡ γῆ.
\vs{30}Καὶ σὺ καὶ οἱ θεράποντές σου, ἐπίσταμαι ὅτι οὐδέπω πεφόβησθε τὸν Κύριον.
\vs{31}Τὸ δὲ λίνον καὶ ἡ κριθὴ ἐπλήγη· ἡ γὰρ κριθὴ παρεστηκυῖα, τὸ δὲ λίνον σπερματίζον.
\vs{32}Ὁ δὲ πυρὸς καὶ ἡ ὀλύρα οὐκ ἐπληγησαν, ὄψιμα γὰρ ἦν.
\vs{33}Ἐξῆλθε δὲ Μωυσῆς ἀπὸ Φαραὼ ἐκτὸς τῆς πόλεως, καὶ ἐξέτεινε τὰς χεῖρας πρὸς Κύριον· καὶ αἱ φωναὶ ἐπαύσαντο, καὶ ἡ χάλαζα καὶ ὁ ὑετὸς οὐκ ἔσταξεν ἔτι ἐπὶ τὴν γῆν.
\vs{34}Ἰδὼν δὲ Φαραὼ ὅτι πέπαυται ὁ ὑετὸς καὶ ἡ χάλαζα καὶ αἱ φωναὶ, προσέθετο τοῦ ἁμαρτάνειν· καὶ ἐβάρυνεν αὐτοῦ τὴν καρδίαν, καὶ τῶν θεραπόντων αὐτοῦ.
\vs{35}Καὶ ἐσκληρύνθη ἡ καρδία Φαραὼ, καὶ οὐκ ἐξαπέστειλε τοὺς υἱοὺς Ἰσραὴλ, καθάπερ ἐλάλησε Κύριος τῷ Μωυσῇ.

\ch{10}
Εἶπε δὲ Κύριος πρὸς Μωυσῆν, λέγων, εἴσελθε πρὸς Φαραὼ, ἐγὼ γὰρ ἐσκλήρυνα αὐτοῦ τὴν καρδίαν καὶ τῶν θεραπόντων αὐτοῦ, ἵνα ἑξῆς ἐπέλθῃ τὰ σημεῖα ταῦτα ἐπʼ αὐτούς·
\vs{2}ὅπως διηγήσησθε εἰς τὰ ὦτα τῶν τέκνων ὑμῶν, καὶ τοῖς τέκνοις τῶν τέκνων ὑμῶν, ὅσα ἐμπέπαιχα τοῖς Αἰγυπτίοις, καὶ τὰ σημεῖά μου, ἃ ἐποίησα ἐν αὐτοῖς· καὶ γνώσεσθε ὅτι ἐγὼ Κύριος.
\vs{3}Εἰσῆλθε δὲ Μωυσῆς καὶ Ἀαρὼν ἐναντίον Φαραὼ, καὶ εἶπαν αὐτῷ, τάδε λέγει Κύριος ὁ Θεὸς τῶν Ἐβραίων, ἕως τίνος οὐ βούλει ἐντραπῆναί με; ἔξαπόστειλον τὸν λαόν μου, ἵνα λατρεύσωσί μοι.
\vs{4}Ἐὰν δὲ μὴ θέλῃς σὺ ἐξαποστεῖλαι τὸν λαόν μου, ἰδοὺ ἐγὼ ἐπάγω ταύτην τὴν ὥραν αὔριον ἀκρίδα πολλὴν ἐπὶ πάντα τὰ ὅριά σου.
\vs{5}Καὶ καλύψει τὴν ὄψιν τῆς γῆς, καὶ οὐ δυνήσῃ κατιδεῖν τὴν γῆν· καὶ κατέδεται πᾶν τὸ περισσὸν τῆς γῆς τὸ καταλειφθὲν, ὃ κατέλιπεν ὑμῖν ἡ χάλαζα, καὶ κατέδεται πᾶν ξύλον τὸ φυόμενον ὑμῖν ἐπὶ τῆς γῆς.
\vs{6}Καὶ πλησθήσονταί σου αἱ οἰκίαι, καὶ αἱ οἰκίαι τῶν θεραπόντων σου, καὶ πᾶσαι αἱ οἰκίαι ἐν πάσῃ γῇ τῶν Αἰγυπτίων· ἃ οὐδέποτε ἑωράκασιν οἱ πατέρες σου, οὐδʼ οἱ πρόπαπποι αὐτῶν, ἀφʼ ἧς ἡμέρας γεγόνασιν ἐπὶ τῆς γῆς, ἔως τῆς ἡμέρας ταύτης· καὶ ἐκκλίνας Μωυσῆς ἐξῆλθεν ἀπὸ Φαραώ.
\vs{7}Καὶ λέγουσιν οἱ θεράποντες Φαραὼ πρὸς αὐτὸν, ἕως τίνος ἔσται τοῦτο ἡμῖν σκῶλον; ἐξαπόστειλον τοὺς ἀνθρώπους, ὅπως λατρεύσωσι τῷ Θεῷ αὐτῶν· ἢ εἰδέναι βούλει ὅτι ἀπόλωλεν Αἴγυπτος;
\vs{8}Καὶ ἀπέστρεψαν τόν τε Μωυσῆν καὶ Ἀαρὼν πρὸς Φαραὼ, καὶ εἶπεν αὐτοῖς, πορεύεσθε καὶ λατρεύσατε Κυρίῳ τῷ Θεῷ ὑμῶν· τίνες δὲ καὶ τίνες εἰσιν οἱ πορευόμενοι;
\vs{9}Καὶ λέγει Μωυσῆς, σὺν τοῖς νεανίσκοις καὶ πρεσβυτέροις πορευσόμεθα, σὺν τοῖς υἱοῖς καὶ θυγατράσι, καὶ προβάτοις, καὶ βουσὶν ἡμῶν· ἔστι γὰρ ἑορτὴ Κυρίου.
\vs{10}Καὶ εἶπε πρὸς αὐτοὺς, ἔστω οὕτω Κύριος μεθʼ ὑμῶν· καθότι ἀποστέλλω ὑμᾶς, μὴ καὶ τὴν ἀποσκευὴν ὑμῶν; ἴδετε ὅτι πονηρία πρόσκειται ὑμῖν.
\vs{11}Μὴ οὕτως· πορευέσθωσαν δὲ οἱ ἄνδρες, καὶ λατρευσάτωσαν τῷ Θεῷ· τοῦτο γὰρ αὐτοὶ ἐκζητεῖτε· ἐξέβαλον δὲ αὐτοὺς ἀπὸ προσώπου Φαραώ.
\vs{12}Εἶπε δὲ Κύριος πρὸς Μωυσῆν, ἔκτεινον τὴν χεῖρα ἐπὶ γῆν Αἰγύπτου· καὶ ἀναβήτω ἀκρὶς ἐπὶ τὴν γῆν, καὶ κατέδεται πᾶσαν βοτάνην τῆς γῆς, καὶ πάντα τὸν καρπὸν τῶν ξύλων, ὃν ὑπελίπετο ἡ χάλαζα.
\vs{13}Καὶ ἐπῇρε Μωυσῆς τὴν ῥάβδον εἰς τὸν οὐρανὸν, καὶ Κύριος ἐπήγαγεν ἄνεμον νότον ἐπὶ τὴν γῆν, ὅλην τὴν ἡμέραν ἐκείνην, καὶ ὅλην τὴν νύκτα· τὸ πρωῒ ἐγενήθη, καὶ ὁ ἄνεμος ὁ νότος ἀνέλαβεν τὴν ἀκρίδα,
\vs{14}καὶ ἀνήγαγεν αὐτὴν ἐπὶ πᾶσαν γῆν Αἰγύπτου· καὶ κατέπαυσεν ἐπὶ πάντα τὰ ὅρια Αἰγύπτου πολλὴ σφόδρα· προτέρα αὐτῆς οὐ γέγονε τοιαύτη ἀκρὶς, καὶ μετὰ ταῦτα οὐκ ἔσται οὕτως.
\vs{15}Καὶ ἐκάλυψε τὴν ὄψιν τῆς γῆς, καὶ ἐφθάρη ἡ γῆ· καὶ κατέφαγε πᾶσαν βοτάνην τῆς γῆς, καὶ πάντα τόν καρπὸν τῶν ξύλων, ὃς ὑπελείφθη ἀπὸ τῆς χαλάζης· οὐχ ὑπελείφθη χλωρὸν οὐδὲν ἐν τοῖς ξύλοις, καὶ ἐν πάσῃ βοτάνῃ τοῦ πεδίου, ἐν πάσηῃ γῇ Αἰγύπτου.

\vs{16}Κατέσπευδε δὲ Φαραὼ καλέσαι Μωυσῆν καὶ Ἀαρὼν, λέγων, ἡμάρτηκα ἐναντίον Κυρίου τοῦ Θεοῦ ὑμῶν, καὶ εἰς ὑμᾶς.
\vs{17}Προσδέξασθε οὖν μου τὴν ἁμαρτίαν ἔτι νῦν, καὶ προσεύξασθε πρὸς Κύριον τὸν Θεὸν ὑμῶν, καὶ περιελέτω ἀπʼ ἐμοῦ τὸν θάνατον τοῦτον.
\vs{18}Ἐξῆλθε δὲ Μωυσῆς ἀπὸ Φαραὼ, καὶ ηὔξατο πρὸς τὸν Θεόν.
\vs{19}Καὶ μετέβαλε Κύριος ἄνεμον ἀπὸ θαλάσσης σφοδρὸν, καὶ ἀνέλαβε τὴν ἀκρίδα, καὶ ἔβαλεν αὐτὴν εἰς τὴν ἐρυθρὰν θαλάσσαν· καὶ οὐχ ὑπελείφη ἀκρὶς μία ἐν πάσῃ γῇ Αἰγύπτου.
\vs{20}Καὶ ἐσκλήρυνε Κύριος τὴν καρδίαν Φαραὼ, καὶ οὐκ ἐξαπέστειλε τοὺς υἱοὺς Ἰσραήλ.
\vs{21}Εἶπε δὲ Κύριος πρὸς Μωυσῆν, ἔκτεινον τὴν χεῖρά σου εἰς τὸν οὐρανὸν, καὶ γενηθήτω σκότος ἐπὶ γῆς Αἰγύπτου, ψηλαφητὸν σκότος.
\vs{22}Ἐξέτεινε δὲ Μωυσῆς τὴν χεῖρα εἰς τὸν οὐρανόν· καὶ ἐγένετο σκότος γνόφος, θύελλα ἐπὶ πᾶσαν γῆν Αἰγύπτου τρεῖς ἡμέρας·
\vs{23}Καὶ οὐκ εἶδεν οὐδεὶς τὸν ἀδελφὸν αὐτοῦ τρεῖς ἡμέρας· καὶ οὐκ ἐξανέστη οὐδεὶς ἐκ τῆς κοίτης αὐτοῦ τρεῖς ἡμέρας· πᾶσι δὲ τοῖς υἱοῖς Ἰσραὴλ φῶς ἦν ἐν πᾶσιν οἷς κατεγίνοντο.
\vs{24}Καὶ ἐκάλεσε Φαραὼ Μωυσῆν καὶ Ἀαρὼν, λέγων, Βαδίζετε, λατρεύσατε Κυρίῳ τῷ Θεῷ ὑμῶν, πλὴν τῶν προβάτων καὶ τῶν βοῶν ὑπολείπεσθε· καὶ ἡ ἀποσκευὴ ὑμῶν ἀποτρεχέτω μεθʼ ὑμῶν.
\vs{25}Καὶ εἶπε Μωυσῆς, ἀλλὰ καὶ σὺ δώσεις ἡμῖν ὁλοκαυτώματα καὶ θυσίας, ἂ ποιήσομεν Κυρίῳ τῷ Θεῷ ἡμῶν.
\vs{26}Καὶ τὰ κτήνη ἡμῶν πορεύσεται μεθʼ ἡμῶν, καὶ οὐχ ὑπολειψόμεθα ὁπλήν· ἀπʼ αὐτῶν γὰρ ληψόμεθα λατρεῦσαι Κυρίῳ τῷ Θεῷ ἡμῶν· ἡμεῖς δὲ οὐκ οἴδαμεν τί λατρεύσομεν Κυρίῳ τῷ Θεῷ ἡμῶν, ἕως τοῦ ἐλθεῖν ἡμᾶς ἐκεῖ.
\vs{27}Ἐσκλήρυνε δὲ Κύριος τὴν καρδίαν Φαραὼ, καὶ οὐκ ἐβουλήθη ἐξαποστεῖλαι αὐτούς.
\vs{28}Καὶ λέγει Φαραὼ, ἄπελθε ἀπʼ ἐμοῦ· πρόσεχε σεαυτῷ ἔτι προσθεῖναι ἰδεῖν μου τὸ πρόσωπον· ᾗ δʼ ἂν ἡμέρᾳ ὀφθῇς μοι, ἀποθανῇ.
\vs{29}Λέγει δὲ Μωυσῆς, εἴρηκας· οὐκ ἔτι ὀφθήσομαί σοι εἰς πρόσωπον.

\ch{11}
Εἶπε δὲ Κύριος πρὸς Μωυσῆν, ἔτι μίαν πληγὴν ἐγὼ ἐπάξω ἐπὶ Φαραὼ, καὶ ἐπʼ Αἴγυπτον, καὶ μετὰ ταῦτα ἐξαποστελεῖ ὑμᾶς ἐντεῦθεν· ὅταν δὲ ἐξαποστέλλῃ ὑμᾶς σὺν παντὶ, ἐκβαλεῖ ὑμᾶς ἐκβολῇ.
\vs{2}Λάλησον οὖν κρυφῇ εἰς τὰ ὦτα τοῦ λαοῦ, καὶ αἰτησάτω ἕκαστος παρὰ τοῦ πλησίον σκεύη ἀργυρᾶ καὶ χρυσὰ καὶ ἱματισμόν.
\vs{3}Κύριος δὲ ἔδωκε τὴν χάριν τῷ λαῷ αὐτοῦ ἐναντίον τῶν Αἰγυπτίων, καὶ ἔχρησαν αὐτοῖς· καὶ ὁ ἄνθρωπος Μωυσῆς μέγας ἐγενήθη σφόδρα ἐναντίον τῶν Αἰγυπτίων, καὶ ἐναντίον Φαραὼ, καὶ ἐναντίον τῶν θεραπόντων ἀτοῦ.
\vs{4}Καὶ εἶπε Μωυσῆς, τάδε λέγει Κύριος, περὶ μέσας νύκτας ἐγὼ εἰσπορεύομαι εἰς μέσον Αἰγύπτου·
\vs{5}Καὶ τελευτήσει πᾶν πρωτότοκον ἐν γῇ Αἰγύπτῳ, ἀπὸ πρωτοτόκου Φαραὼ, ὃς κάθηται ἐπὶ τοῦ θρόνου, καὶ ἕως πρωτοτόκου τῆς θεραπαίνης τῆς παρὰ τὸν μύλον, καὶ ἕως πρωτοτοκου παντος κτήνους.
\vs{6}Καὶ ἔσται κραυγὴ μεγάλη κατὰ πᾶσαν γῆν Αἰγύπτου, ἥτις τοιαύτη οὐ γέγονε, καὶ τοιαύτη οὐκ ἔτι προστεθήσεται.
\vs{7}Καὶ ἐν πᾶσι τοῖς υἱοῖς Ἰσραὴλ οὐ γρύξει κύων τῇ γλώσσῃ αὐτοῦ, ἀπὸ ἀνθρώπου ἕως κτήνους· ὅπως εἰδῇς ὅσα παραδοξάσει Κύριος ἀνὰ μέσον τῶν Αἰγυπτίων καὶ τοῦ Ἰσραήλ.
\vs{8}Καὶ καταβήσονται πάντες οἱ παῖδές σου οὗτοι πρός με, καὶ προσκυνήσουσί με, λέγοντες, ἔξελθε σὺ, καὶ πᾶς ὁ λαός σου, οὗ σὺ ἀφηγῇ· καὶ μετὰ ταῦτα ἐξελεύσομαι· ἐξῆλθε δὲ Μωυσῆς ἀπὸ Φαραὼ μετὰ θυμοῦ.
\vs{9}Εἶπε δὲ Κύριος πρὸς Μωυσῆν, οὐκ εἰσακούσεται ὑμῶν Φαραὼ, ἵνα πληθύνων πληθυνῶ μου τὰ σημεῖα, καὶ τὰ τέρατα ἐν γῇ Αἰγύπτῳ.
\vs{10}Μωσῆς δὲ καὶ Ἀαρὼν ἐποίησαν πάντα τὰ σημεῖα καὶ τὰ τέρατα ταῦτα ἐν γῇ Αἰγύπτῳ ἐναντίον Φαραώ· ἐσκλήρυνε δὲ Κύριος τὴν καρδίαν Φαραὼ, καὶ οὐκ εἰσήκουσεν ἐξαποστεῖλαι τοὺς υἱοὺς Ἰσραὴλ ἐκ γῆς Αἰγύπτου.

\ch{12}
Εἶπε δὲ Κύριος πρὸς Μευσῆν καὶ Ἀαρὼν ἐν γῇ Αἰγύπτου, λέγων,
\vs{2}ὁ μὴν οὗτος ὑμῖν ἀρχὴ μηνῶν· πρῶτός ἐστιν ὑμῖν ἐν τοῖς μησὶ τοῦ ἐνιαυτοῦ.
\vs{3}Λάλησον πρὸς πᾶσαν συναγωγὴν υἱῶν Ἰσραὴλ, λέγων, τῇ δεκάτῃ τοῦ μηνὸς τούτου λαβέτωσαν ἕκαστος πρόβατον κατʼ οἴκους πατριῶν, ἕκαστος πρόβατον κατʼ οἰκίαν.
\vs{4}Ἐὰν δὲ ὀλιγοστοὶ ὦσιν ἐν τῇ οἰκίᾳ, ὥστε μὴ εἶναι ἱκανοὺς εἰς πρόβατον, συλλήψεται μεθʼ ἑαυτοῦ τὸν γείτονα τὸν πλησίον αὐτοῦ· κατὰ ἀριθμὸν ψυχῶν, ἕκαστος τὸ ἀρκοῦν αὐτῷ συναριθμήσεται εἰς πρόβατον.
\vs{5}Πρόβατον τέλειον, ἄρσεν, ἐνιαύσιον ἔσται ὑμῖν· ἀπὸ τῶν ἀρνῶν καὶ τῶν ἐρίφων λὴψεσθε.
\vs{6}Καὶ ἔσται ὑμῖν διατετηρημένον ἕως τῆς τεσσαρεσκαιδεκάτης τοῦ μηνὸς τούτου· καὶ σφάξουσιν αὐτὸ πᾶν τὸ πλῆθος συναγωγῆς υἱῶν Ἰσραὴλ πρὸς ἑσπέραν.
\vs{7}Καὶ λήψονται ἀπὸ τοῦ αἵματος, καὶ θήσουσιν ἐπὶ τῶν δύο σταθμῶν καὶ ἐπὶ τὴν φλιὰν, ἐν τοῖς οἴκοις ἐν οἷς ἐὰν φάγωσιν αὐτὰ ἐν αὐτοῖς.
\vs{8}Καὶ φάγονται τὰ κρέα τῇ νυκτὶ ταύτῃ ὀπτὰ πυρὶ, καὶ ἄζυμα ἐπὶ πικρίδων ἔδονται.
\vs{9}Οὐκ ἔδεσθε ἀπʼ αὐτῶν ὠμὸν, οὐδὲ ἡψημένον ἐν ὕδατι, ἀλλʼ ἢ ὀπτὰ πυρὶ, κεφαλὴν σὺν τοῖς ποσὶ καὶ τοῖς ἐνδοσθίοις.
\vs{10}Οὐκ ἀπολείψεται ἀπʼ αὐτοῦ ἕως πρωΐ· καὶ ὀστοῦν οὐ συντρίψετε ἀπʼ αὐτοῦ· τὰ δὲ καταλειπόμενα ἀπʼ αὐτοῦ ἕως πρωῒ ἐν πυρὶ κατακαύσετε.
\vs{11}Οὕτω δὲ φάγεσθε αὐτό· αἱ ὀσφύες ὑμῶν περιεζωσμέναι, καὶ τὰ ὑποδήματα ἐν τοῖς ποσὶν ὑμῶν, καὶ αἱ βακτηρίαι ἐν ταῖς χερσὶν ὑμῶν· καὶ ἔδεσθε αὐτὸ μετὰ σπουδῆς· Πάσχα ἐστὶ Κυρίῳ.
\vs{12}Καὶ διελεύσομαι ἐν γῇ Αἰγύπτῳ ἐν τῇ νυκτὶ ταύτῃ, καὶ πατάξω πᾶν πρωτότοκον ἐν γῇ Αἰγύπτῳ ἀπὸ ἀνθρώπου ἕως κτήνους· καὶ ἐν πᾶσι τοῖς θεοῖς τῶν Αἰγυπτίων ποιήσω τὴν ἐκδίκησιν· ἐγὼ Κύριος.
\vs{13}Καὶ ἔσται τὸ αἷμα ὑμῖν ἐν σημείῳ ἐπὶ τῶν οἰκιῶν, ἐν αἷς ὑμεῖς ἔστε ἐκεῖ· καὶ ὄψομαι τὸ αἷμα, καὶ σκεπάσω ὑμᾶς, καὶ οὐκ ἔσται ἐν ὑμῖν πληγὴ τοῦ ἐκτριβῆναι ὅταν παίω ἐν γῇ Αἰγύπτῳ.

\vs{14}Καὶ ἔσται ἡ ἡμέρα ὑμῖν αὕτη μνημόσυνον, καὶ ἑορτάσετε αὐτὴν ἑορτὴν Κυρίῳ εἰς πάσας τὰς γενεὰς ὑμῶν· νόμιμον αἰώνιον ἑορτάσετε αὐτήν.
\vs{15}Ἑπτὰ ἡμέρας ἄζυμα ἔδεσθε· ἀπὸ δὲ τῆς ἡμέρας τῆς πρώτης, ἀφανιεῖτε ζύμην ἐκ τῶν οἰκιῶν ὑμῶν· πᾶς ὃς ἂν φάγῃ ζύμην, ἐξολοθρευθήσεται ἡ ψυχὴ ἐκείνη ἐξ Ἰσραήλ, ἀπὸ τῆς ἡμέρας τῆς πρώτης ἕως τῆς ἡμέρας τῆς ἑβδόμης.
\vs{16}Καὶ ἡ ἡμέρα ἡ πρώτη, κληθήσεται ἁγία· καὶ ἡ ἡμέρα ἡ ἑβδόμη, κλητὴ ἁγία ἔσται ὑμῖν· πᾶν ἔργον λατρευτὸν οὐ ποιήσετε ἐν αὐταῖς, πλὴν ὅσα ποιηθήσεται πάσῃ ψυχῇ, τοῦτο μόνον ποιηθήσεται ὑμῖν.
\vs{17}Καὶ φυλάξετε τὴν ἐντολὴν ταύτην· ἐν γὰρ τῇ ἡμέρᾳ ταύτῃ ἐξάξω τὴν δύναμιν ὑμῶν ἐκ γῆς Αἰγύπτου, καὶ ποιήσετε τὴν ἡμέραν ταύτην εἰς γενεὰς ὑμῶν νόμιμον αἰώνιον,
\vs{18}ἐναρχόμενοι τῇ τεσσαρεσκαιδεκάτῃ ἡμέρᾳ τοῦ μηνὸς τοῦ πρώτου, ἀφʼ ἑσπέρας ἔδεσθε ἄζυμα, ἕως ἡμέρας μίας καὶ εἰκάδος τοῦ μηνὸς, ἕως ἑσπέρας.
\vs{19}Ἑπτὰ ἡμέρας ζύμη οὐχ εὑρεθήσεται ἐν ταῖς οἰκιαῖς ὑμῶν· πᾶς ὃς ἂν φάγῃ ζυμωτὸν, ἐξολοθρευθήσεται ἡ ψυχὴ ἐκείνη ἐκ συναγωγῆς Ἰσραήλ· ἔν τε τοῖς γειώραις, καὶ αὐτόχθοσι τῆς γῆς.
\vs{20}Πᾶν ζυμωτὸν οὐκ ἔδεσθε, ἐν παντὶ δὲ κατοικητηρίῳ ὑμῶν ἔδεσθε ἄζυμα.

\vs{21}Ἐκάλεσε δὲ Μωυσῆς πᾶσαν γερουσίαν υἱῶν Ἰσραὴλ, καὶ εἶπε πρὸς αὐτοὺς, ἀπελθόντες λάβετε ὑμῖν αὐτοῖς πρόβατον κατὰ συγγενείας ὑμῶν, καὶ θύσατε τὸ πάσχα.
\vs{22}Λήψεσθε δὲ δέσμην ὑσσώπου, καὶ βάψαντες ἀπὸ τοῦ αἵματος τοῦ παρὰ τὴν θύραν, καθίξετε τῆς φλιᾶς, καὶ ἐπʼ ἀμφοτέρων τῶν σταθμῶν, ἀπὸ τοῦ αἵματος ὅ ἐστι παρὰ τὴν θύραν· ὑμεῖς δὲ οὐκ ἐξελεύσεσθε ἕκαστος τὴν θύραν τοῦ οἴκου αὐτοῦ ἕως πρωΐ.
\vs{23}Καὶ παρελεύσεται Κύριος πατάξαι τοὺς Αἰγυπτίους, καὶ ὄψεται τὸ αἷμα ἐπὶ τῆς φλιᾶς, καὶ ἐπʼ ἀμφοτέρων τῶν σταθμῶν· καὶ παρελεύσεται Κύριος τὴν θύραν, καὶ οὐκ ἀφήσει τὸν ὀλοθρεύοντα εἰσελθεῖν εἰς τὰς οἰκίας ὑμῶν πατάξαι.
\vs{24}Καὶ φυλάξασθε τὸ ῥῆμα τοῦτο νόμιμον σεαυτῷ, καὶ τοῖς υἱοῖς σου, ἕως αἰῶνος.
\vs{25}Ἐὰν δὲ εἰσέλθητε εἰς τὴν γῆν, ἣν ἂν δῷ Κύριος ὑμῖν, καθότι ἐλάλησε, φυλάξασθε τὴν λατρείαν ταύτην.
\vs{26}Καὶ ἐσται ἐὰν λέγωσι πρὸς ὑμᾶς οἱ υἱοὶ ὑμῶν, τίς ἡ λατρεία αὕτη;
\vs{27}Καὶ ἐρεῖτε αὐτοῖς, θυσία τὸ πάσχα τοῦτο Κυρίῳ, ὡς ἐσκέπασε τοὺς οἴκους τῶν υἱῶν Ἰσραὴλ ἐν Αἰγύπτῳ, ἡνίκα ἐπάταξε τοὺς Αἰγυπτίους, τοὺς δὲ οἴκους ἡμῶν ἐῤῥύσατο· καὶ κύψας ὁ λαὸς προσεκύνησε.
\vs{28}Καὶ ἀπελθόντες ἐποίησαν οἱ υἱοὶ Ἰσραὴλ, καθὰ ἐνετείλατο Κύριος τῷ Μωυσῇ καὶ Ἀαρῶν, οὕτως ἐποίησαν.

\vs{29}Ἐγενήθη δὲ μεσούσης τῆς νυκτὸς, καὶ Κύριος ἐπάταξε πᾶν πρωτότοκον ἐν γῇ Αἰγύπτῳ, ἀπὸ πρωτοτόκου Φαραὼ τοῦ καθημένου ἐπὶ τοῦ θρόνου, ἕως πρωτοτόκου τῆς αἰχμαλωτίδος τῆς ἐν τῷ λάκκῳ, καὶ ἕως πρωτοτόκου παντὸς κτήνους.
\vs{30}Καὶ ἀναστὰς Φαραὼ νυκτὸς, καὶ οἱ θεράποντες αὐτοῦ, καὶ πάντες οἱ Αἰγύπτιοι, καὶ ἔγενήθη κραυγὴ μεγάλη ἐν πάσῃ γῇ Αἰγύπτῳ· οὐ γὰρ ἦν οἰκία, ἐν ᾗ οὐκ ἦν ἐν αὐτῇ τεθνηκώς.
\vs{31}Καὶ ἐκάλεσε Φαραὼ Μωυσῆν καὶ Ἀαρὼν νυκτὸς, καὶ εἶπεν αὐτοῖς, ἀνάστητε, καὶ ἐξέλθατε ἐκ τοῦ λαοῦ μου, καὶ ὑμεῖς, καὶ οἱ υἱοὶ Ἰσραήλ· βαδίζετε καὶ λατρεύσατε Κυρίῳ τῷ Θεῷ ὑμῶν, καθὰ λέγετε.
\vs{32}Καὶ τὰ πρόβατα καὶ τοὺς βόας ὑμῶν ἀναλαβόντες πορεύεσθε· εὐλογήσατε δὴ κᾀμέ.
\vs{33}Καὶ κατεβιάζοντο οἱ Αἰγύπτιοι τὸν λαὸν σπουδῇ ἐκβαλεῖν αὐτοὺς ἐν τῆς γῆς· εἶπαν γὰρ, ὅτι πάντες ἡμεῖς ἀποθνήσκομεν.
\vs{34}Ἀνέλαβε δὲ ὁ λαὸς τὸ σταῖς αὐτῶν, πρὸ τοῦ ζυμωθῆναι τὰ φυράματα αὐτῶν, ἐνδεδεμένα ἐν τοῖς ἱματίοις αὐτῶν ἐπὶ τῶν ὤμαν.
\vs{35}Οἱ δὲ υἱοὶ Ἰσραὴλ ἐποίησαν, καθὰ συνέταξεν αὐτοῖς Μωυσῆς, καὶ ᾔτησαν παρὰ τῶν Αἰγυπτίων σκεύη ἀργυρᾶ καὶ χρυσᾶ καὶ ἱματισμόν.
\vs{36}Καὶ ἔδωκε Κύριος τὴν χάριν τῷ λαῷ αὐτοῦ ἐναντίον τῶν Αἰγυπτίων, καὶ ἔχρησαν αὐτοῖς· καὶ ἐσκύλευσαν τοὺς Αἰγυπτίους.

\vs{37}Ἀπάραντες δὲ υἱοὶ Ἰσραὴλ ἐκ Ῥαμεσσῆ εἰς Σοκχὼθ εἰς ἑξακοσίας χιλιάδας πεζῶν, οἱ ἄνδρες, πλὴν τῆς ἀποσκευῆς.
\vs{38}Καὶ ἐπίμικτος πολὺς συνανέβη αὐτοῖς, καὶ πρόβατα, καὶ βόες, καὶ κτήνη πολλὰ σφόδρα.
\vs{39}Καὶ ἔπεψαν τὸ σταῖς ὃ ἐξήνεγκαν ἐξ Αἰγύπτου, ἐγκρυφίας ἀζύμους, οὐ γὰρ ἐζυμώθη· ἐξέβαλον γὰρ αὐτοὺς οἱ Αἰγύπτιοι, καὶ οὐκ ἠδυνήθησαν ἐπιμεῖναι, οὐδὲ ἐπισιτισμὸν ἐποίησαν ἑαυτοῖς εἰς τὴν ὁδόν.
\vs{40}Ἡ δὲ κατοίκησις τῶν υἱῶν Ἰσραὴλ, ἣν κατῴκησαν ἐν γῇ Αἰγύπτῳ καὶ ἐν γῇ Χαναὰν, ἔτη τετρακόσια τριάκοντα.
\vs{41}Καὶ ἐγένετο μετὰ τὰ τετρακόσια τριάκοντα ἔτη, ἐξῆλθε πᾶσα ἡ δύναμις Κυρίου ἐκ γῆς Αἰγύπτου νυκτός.
\vs{42}Προφυλακή ἐστι τῷ Κυρίῳ, ὥστε ἐξαγαγεῖν αὐτοὺς ἐκ γῆς Αἰγύπτου· ἐκείνη ἡ νὺξ αὕτη, προφυλακὴ Κυρίῳ, ὥστε πᾶσι τοῖς υἱοῖς Ἰσραὴλ εἶναι εἰς γενεὰς αὐτῶν.
\vs{43}Εἶπε δὲ Κύριος πρὸς Μωυσῆν καὶ Ἀαρὼν, οὗτος ὁ νόμος τοῦ πάσχα· πᾶς ἀλλογενὴς οὐκ ἔδεται ἀπʼ αὐτοῦ·
\vs{44}Καὶ πάντα οἰκέτην ἢ ἀργυρώνητον περιτεμεῖς αὐτόν· καὶ τότε φάγεται ἀπʼ αὐτοῦ.
\vs{45}Πάροικος ἢ μισθωτὸς οὐκ ἔδεται ἀπʼ αὐτοῦ.
\vs{46}Ἐν οἰκίᾳ μιᾷ βρωθήσεται, καὶ οὐκ ἐξοίσετε ἐκ τῆς οἰκίας τῶν κρεῶν ἔξω· καὶ ὀστοῦν οὐ συντρίψετε ἀπʼ αὐτοῦ.
\vs{47}Πᾶσα συναγωγὴ υἱῶν Ἰσραὴλ ποιήσει αὐτό.
\vs{48}Ἐὰν δέ τις προσέλθῃ πρὸς ὑμᾶς προσήλυτος ποιῆσαι τὸ πάσχα Κυρίῳ, περιτεμεῖς αὐτοῦ πᾶν ἀρσενικόν, καὶ τότε προσελεύσεται ποιῆσαι αὐτό· καὶ ἔσται ὥσπερ καὶ ὁ αὐτόχθων τῆς γῆς· πᾶς ἀπερίτμητος οὐκ ἔδεται ἀπʼ αὐτοῦ.
\vs{49}Νόμος εἷς ἔσται τῷ ἐγχωρίῳ, καὶ τῷ προσελθόντι προσηλύτῳ ἐν ὑμῖν.
\vs{50}Καὶ ἐποίησαν οἱ υἱοὶ Ἰσραὴλ καθὰ ἐνετείλατο Κύριος τῷ Μωυσῇ καὶ Ἀαρὼν πρὸς αὐτούς, οὕτως ἐποίησαν.
\vs{51}Καὶ ἐγένετο ἐν τῇ ἡμέρᾳ ἐκείνῃ, ἐξήγαγε Κύριος τοὺς υἱοὺς Ἰσραὴλ ἐκ γῆς Αἰγύπτου σὺν δυνάμει αὐτῶν.

\ch{13}
Εἶπε δὲ Κύριος πρὸς Μωυσῆν, λέγων,
\vs{2}ἁγίασόν μοι πᾶν πρωτότοκον πρωτογενὲς διανοῖγον πᾶσαν μήτραν ἐν τοῖς υἱοῖς Ἰσραὴλ ἀπὸ ἀνθρώπου ἕως κτήνους, ἐμοί ἐστιν.
\vs{3}Εἶπε δὲ Μωυσῆς πρὸς τὸν λαὸν, μνημονεύετε τὴν ἡμέραν ταύτην, ἐν ᾗ ἐξήλθατε ἐκ γῆς Αἰγύπτου, ἐξ οἴκου δουλείας· ἐν γὰρ χειρὶ κραταιᾷ ἐξήγαγεν ὑμᾶς Κύριος ἐντεῦθεν· καὶ οὐ βρωθήσεται ζύμη.
\vs{4}Ἐν γὰρ τῇ σήμερον ὑμεῖς ἐκπορεύεσθε ἐν μηνὶ τῶν νέων.
\vs{5}Καὶ ἔσται ἡνίκα ἐὰν εἰσαγάγῃ σε Κύριος ὁ Θεός σου εἰς τὴν γῆν τῶν Χαναναίων, καὶ Χετταίων, καὶ Ἀμοῤῥαίων, καὶ Εὐαίων, καὶ Ἰεβουσαίων, καὶ Γεργεσαίων, καὶ Φερεζαίων, ἣν ὤμοσε τοῖς πατράσι σου, δοῦναί σοι γῆν ῥέουσαν γάλα καὶ μέλι· καὶ ποιήσεις τὴν λατρείαν ταύτην ἐν τῷ μηνὶ τούτῳ.
\vs{6}Ἓξ ἡμέρας ἔδεσθε ἄζυμα, τῇ δὲ ἡμέρᾳ τῇ ἑβδόμῃ ἑορτὴ Κυρίου.
\vs{7}Ἄζυμα ἔδεσθε ἑπτὰ ἡμέρας· οὐκ ὀφθήσεταί σοι ζυμωτὸν, οὐδὲ ἔσται σοι ζύμη ἐν πᾶσι τοῖς ὁρίοις σου.
\vs{8}Καὶ ἀναγγελεῖς τῷ υἱῷ σου ἐν τῇ ἡμέρᾳ ἐκείνῃ, λέγων, διὰ τοῦτο ἐποίησε Κύριος ὁ Θεός μοι, ὡς ἐξεπορευόμην ἐξ Αἰγύπτου.
\vs{9}Καὶ ἔσται σοι σημεῖον ἐπὶ τῆς χειρός σου, καὶ μνημόσυνον πρὸ ὀφθαλμῶν σου, ὅπως ἂν γένηται ὁ νόμος Κυρίου ἐν τῷ στόματί σου· ἐν γὰρ χειρὶ κραταιᾷ ἐξήγαγέ σε Κύριος ὁ Θεὸς ἐξ Αἰγύπτου.
\vs{10}Καὶ φυλάξασθε τὸν νόμον τοῦτον κατὰ καιροὺς ὡρῶν, ἀφʼ ἡμερῶν εἰς ἡμέρας.

\vs{11}Καὶ ἔσται ὡς ἂν εἰσαγάγῃ σε Κύριος ὁ Θεός σου εἰς τὴν γῆν τῶν Χαναναίων, ὃν τρόπον ὤμοσε τοῖς πατράσι σου, καὶ δώσει σοι αὐτήν.
\vs{12}Καὶ ἀφελεῖς πᾶν διανοῖγον μήτραν, τὰ ἀρσενικὰ τῷ Κυρίῳ· πᾶν διανοῖγον μήτραν ἐκ βουκολίων ἢ ἐν τοῖς κτήνεσί σου, ὅσα ἐὰν γένηταί σοι, τὰ ἀρσενικὰ ἁγιάσεις τῷ Κυρίῳ.
\vs{13}Πᾶν διανοῖγον μήτραν ὄνου, ἀλλάξεις προβάτῳ· ἐὰν δὲ μὴ ἀλλάξῃς, λυτρώσῃ αὐτό· πᾶν πρωτότοκον ἀνθρώπου τῶν υἱῶν σου λυτρώσῃ.
\vs{14}Ἐὰν δὲ ἐρωτήσῃ σε ὁ υἱός σου μετὰ ταῦτα, λέγων, τί τοῦτο; καὶ ἐρεῖς αὐτῷ, ὅτι ἐν χειρὶ κραταιᾷ ἐξήγαγεν Κύριος ἡμᾶς ἐκ γῆς Αἰγύπτου, ἐξ οἴκου δουλείας.
\vs{15}Ἡνίκα δὲ ἐσκλήρυνε Φαραὼ ἐξαποστεῖλαι ἡμᾶς, ἀπέκτεινε πᾶν πρωτότοκον ἐν γῇ Αἰγύπτῳ, ἀπὸ πρωτοτόκων ἀνθρώπων ἕως πρωτοτόκων κτηνῶν· διὰ τοῦτο ἐγὼ θύω πᾶν διανοῖγον μήτραν, τὰ ἀρσενικὰ τῷ Κυρίῳ, καὶ πᾶν πρωτότοκον τῶν υἱῶν μου λυτρώσομαι.
\vs{16}Καὶ ἔσται εἰς σημεῖον ἐπὶ τῆς χειρός σου, καὶ ἀσαλευτον πρὸ ὀφθαλμων σου· ἐν γὰρ χειρὶ κραταιᾷ ἐξήγαγέ σε Κύριος ἐξ Αἰγύπτου.

\vs{17}Ὡς δὲ ἐξαπέστειλε Φαραὼ τὸν λαὸν, οὐχ ὡδήγησεν αὐτοὺς ὁ Θεὸς ὁδὸν γῆς· Φυλιστιεὶμ, ὅτι ἐγγὺς ἦν· εἶπε γὰρ ὁ Θεὸς, μήποτε μεταμελήσῃ τῷ λαῷ ἰδόντι πόλεμον, καὶ ἀποστρέψῃ εἰς Αἴγυπτον.
\vs{18}Καὶ ἐκύκλωσεν ὁ Θεὸς τὸν λαὸν ὁδὸν τὴν εἰς τὴν ἔρημον, εἰς τὴν ἐρυθρὰν θάλασσαν· πέμπτῃ δὲ γενεᾷ ἀνέβησαν οἱ υἱοὶ Ἰσραὴλ ἐκ γῆς Αἰγύπτου.
\vs{19}Καὶ ἔλαβε Μωυσῆς τὰ ὀστᾶ Ἰωσὴφ μεθʼ ἑαυτοῦ· ὅρκῳ γὰρ ὥρκισεν τοὺς υἱοὺς Ἰσραὴλ, λέγων, ἐπισκοπῇ ἐπισκέψεται ὑμᾶς Κύριος, καὶ συνανοίσετε μου τὰ ὀστᾶ ἐντεῦθεν μεθʼ ὑμῶν.

\vs{20}Ἐξάραντες δὲ οἱ υἱοὶ Ἰσραὴλ ἐκ Σοκχὼθ, ἐστρατοπέδευσαν ἐν Ὀθὼμ παρὰ τὴν ἔρημον.
\vs{21}Ὁ δὲ Θεὸς ἡγεῖτο αὐτῶν, ἡμέρας μὲν ἐν στύλῳ νεφέλης, δεῖξαι αὐτοῖς τὴν ὁδόν· τὴν δὲ νύκτα ἐν στύλῳ πυρός.
\vs{22}Οὐκ ἐξέλιπεν δὲ ὁ στύλος τῆς νεφέλης ἡμέρας, καὶ ὁ στύλος τοῦ πυρὸς νυκτὸς, ἐναντίον τοῦ λαοῦ παντός.

\ch{14}
Καὶ ἐλάλησε Κύριος πρὸς Μωυσῆν, λέγων,
\vs{2}Λάλησον τοῖς υἱοῖς Ἰσραὴλ, καὶ ἀποστρέψαντες στρατοπεδευσάτωσαν ἀπέναντι τῆς ἐπαύλεως, ἀνὰ μέσον Μαγδώλου καὶ ἀνὰ μέσον τῆς θαλάσσης, ἐξεναντίας Βεελσεπφῶν· ἐνώπιον αὐτῶν στρατοπεδεύσεις ἐπὶ τῆς θαλάσσης.
\vs{3}Καὶ ἐρεῖ Φαραὼ τῷ λαῷ αὐτοῦ, οἱ υἱοὶ Ἰσραὴλ πλανῶνται οὗτοι ἐν τῇ γῇ, συγκέκλεικε γὰρ αὐτοὺς ἡ ἔρημος.
\vs{4}Ἐγὼ δὲ σκληρυνῶ τὴν καρδίαν Φαραὼ, καὶ καταδιώξεται ὀπίσω αὐτῶν· καὶ ἐνδοξασθήσομαι ἐν Φαραῷ, καὶ ἐν πάσῃ τῇ στρατίᾳ αὐτοῦ· καὶ γνώσονται πάντες οἱ Αἰγύπτιοι ὅτι ἐγώ εἰμι Κύριος· καὶ ἐποίησαν οὕτως.
\vs{5}Καὶ ἀνηγγέλη τῷ βασιλεῖ τῶν Αἰγυπτίων ὅτι πέφευγεν ὁ λαός· καὶ μετεστράφη ἡ καρδία Φαραὼ, καὶ τῶν θεραπόντων αὐτοῦ, ἐπὶ τὸν λαὸν, καὶ εἶπαν, τί τοῦτο ἐποιήσαμεν, τοῦ ἐξαποστεῖλαι τοὺς υἱοὺς Ἰσραὴλ, τοῦ μὴ δουλεύειν ἡμῖν;
\vs{6}Ἔζευξεν οὖν Φαραὼ τὰ ἅρματα αὐτοῦ, καὶ πάντα τὸν λαὸν αὐτοῦ συναπήγαγε μεθʼ ἑαυτοῦ,
\vs{7}καὶ λαβὼν ἑξακόσια ἅρματα ἐκλεκτὰ, καὶ πᾶσαν τὴν ἵππον τῶν Αἰγυπτίων, καὶ τριστάτας ἐπὶ πάντων.
\vs{8}Καὶ ἐσκλήρυνε Κύριος τὴν καρδίαν Φαραὼ βασιλέως Αἰγύπτου, καὶ τῶν θεραπόντων αὐτοῦ, καὶ κατεδίωξεν ὀπίσω τῶν υἱῶν Ἰσραήλ· οἱ δὲ υἱοὶ Ἰσραὴλ ἐξεπορεύοντο ἐν χειρὶ ὑψηλῇ.
\vs{9}Καὶ κατεδίωξαν οἱ Αἰγύπτιοι ὀπίσω αὐτῶν, καὶ εὕροσαν αὐτοὺς παρεμβεβληκότας παρὰ τὴν θάλασσαν· καὶ πᾶσα ἡ ἵππος καὶ τὰ ἅρματα Φαραὼ, καὶ οἱ ἱππεῖς, καὶ ἡ στρατιὰ αὐτοῦ ἀπέναντι τῆς ἐπαύλεως, ἐξεναντίας Βεελσεπφῶν.
\vs{10}Καὶ Φαραὼ προσῆγε· καὶ ἀναβλέψαντες οἱ υἱοὶ Ἰσραὴλ τοῖς ὀφθαλμοῖς ὁρῶσι, καὶ οἱ Αἰγύπτιοι ἐστρατοπέδευσαν ὀπίσω αὐτῶν· καὶ ἐφοβήθησαν σφόδρα· ἀνεβόησαν δὲ οἱ υἱοὶ Ἰσραὴλ πρὸς Κύριον.
\vs{11}Καὶ εἶπαν πρὸς Μωυσῆν, παρὰ τὸ μὴ ὑπάρχειν μνήματα ἐν γῇ Αἰγύπτῳ, ἐξήγαγες ἡμᾶς θανατῶσαι ἐν τῇ ἐρήμῳ· τί τοῦτο ἐποίησας ἡμῖν, ἐξαγαγὼν ἐξ Αἰγύπτου;
\vs{12}Οὐ τοῦτο ἦν τὸ ῥῆμα, ὃ ἐλαλήσαμεν πρὸς σὲ ἐν Αἰγύπτῳ, λέγοντες, πάρες ἡμᾶς, ὅπως δουλεύσωμεν τοῖς Αἰγυπτίοις; κρεῖσσον γὰρ ἡμᾶς δουλεύειν τοῖς Αἰγυπτίοις, ἢ ἀποθανεῖν ἐν τῇ ἐρήμῳ ταύτῃ.

\vs{13}Εἶπε δὲ Μωυσῆς πρὸς τὸν λαὸν, θαρσεῖτε, στῆτε καὶ ὁρᾶτε τὴν σωτηρίαν τὴν παρὰ τοῦ Κυρίου, ἣν ποιήσει ἡμῖν σήμερον· ὃν τρόπον γὰρ ἑωράκατε τοὺς Αἰγυπτίους σήμερον, οὐ προσθήσεσθε ἔτι ἰδεῖν αὐτοὺς εἰς τὸν αἰῶνα χρόνον·
\vs{14}Κύριος πολεμήσει περὶ ὑμῶν, καὶ ὑμεῖς σιγήσετε.
\vs{15}Εἶπε δὲ Κύριος πρὸς Μωυσῆν, τί βοᾷς πρός με; λάλησον τοῖς υἱοῖς Ἰσραὴλ, καὶ ἀναζευξάτωσαν.
\vs{16}Καὶ σὺ ἔπαρον τῇ ῥάβδῳ σου, καὶ ἔκτεινον τὴν χεῖρά σου ἐπὶ τὴν θάλασσαν, καὶ ῥῆξον αὐτήν· καὶ εἰσελθάτωσαν οἱ υἱοὶ Ἰσραὴλ εἰς μέσον τῆς θαλάσσης κατὰ τὸ ξηρόν.
\vs{17}Καὶ ἰδοὺ ἐγὼ σκληρυνῶ τὴν καρδίαν Φαραὼ, καὶ τῶν Αἰγυπτίων πάντων, καὶ εἰσελεύσονται ὀπίσω αὐτῶν· καὶ ἐνδοξασθήσομαι ἐν Φαραῷ, καὶ ἐν πάσῃ τῇ στρατιᾷ αὐτοῦ, καὶ ἐν τοῖς ἅρμασι, καὶ ἐν τοῖς ἵπποις αὐτοῦ.
\vs{18}Καὶ γνώσονται πάντες οἱ Αἰγύπτιοι ὅτι ἐγώ εἰμὶ Κύριος, ἐνδοξαζομένου μου ἐν Φαραῷ, καὶ ἐν τοῖς ἅρμασι, καὶ ἵπποις αὐτοῦ.
\vs{19}Ἐξῇρε δὲ ὁ Ἄγγελος τοῦ Θεοῦ ὁ προπορευόμενος τῆς παρεμβολῆς τῶν υἱῶν Ἰσραὴλ, καὶ ἐπορεύθη ἐκ τῶν ὄπισθεν· ἐξῇρε δὲ καὶ ὁ στύλος τὴς νεφέλης ἀπὸ προσώπου αὐτῶν, καὶ ἔστη ἐκ τῶν ὀπίσω αὐτῶν.
\vs{20}Καὶ εἰσῆλθεν ἀνὰ μέσον τῆς παρεμβολῆς τῶν Αἰγυπτίων, καὶ ἀνὰ μέσον τῆς παρεμβολῆς τῶν Αἰγυπίων, καὶ ἀνὰ μέσον τῆς παρεμβολῆς Ἰσραὴλ, καὶ ἔστη· καὶ ἐγένετο σκότος καὶ γνόφος· καὶ διῆλθεν ἡ νύξ· καὶ οὐ συνέμιξαν ἀλλήλοις ὅλην τὴν νύκτα.
\vs{21}Ἐξέτεινε δὲ Μωυσῆς τὴν χεῖρα ἐπὶ τὴν θάλασσαν· καὶ ὑπήγαγε Κύριος τὴν θάλασσαν ἐν ἀνέμῳ νότῳ βιαίῳ ὅλην τὴν νύκτα, καὶ ἐποίησε τὴν θάλασσαν ξηράν· καὶ ἐσχίσθη τὸ ὕδωρ.
\vs{22}Καὶ εἰσῆλθον οἱ υἱοὶ Ἰσραὴλ εἰς μέσον τῆς θαλάσσης κατὰ τὸ ξηρόν· καὶ τὸ ὕδωρ αὐτῆς τεῖχος ἐκ δεξιῶν, καὶ τεῖχος ἐξ εὐωνύμων.

\vs{23}Καὶ κάτεδίωξαν οἱ Αἰγύπτιοι, καὶ εἰσῆλθον ὀπίσω αὐτῶν καὶ πᾶς ἵππος Φαραὼ, καὶ τὰ ἅρματα, καὶ οἱ ἀναβάται, εἰς μέσον τῆς θαλάσσης.
\vs{24}Ἐγενήθη δὲ ἐν τῇ φυλακῇ τῇ ἑωθινῇ, καὶ ἐπίβλεψε Κύριος ἐπὶ τὴν παρεμβολὴν τῶν Αἰγυπτίων ἐν στύλῳ πυρὸς καὶ νεφέλης, καὶ συνετάραξε τὴν παρεμβολὴν τῶν Αἰγυπτίων,
\vs{25}καὶ συνέδησε τοὺς ἄξονας τῶν ἁρμάτων αὐτῶν, καὶ ἤγαγεν αὐτοὺς μετὰ βίας· καὶ εἶπαν οἱ Αἰγύπτιοι, φυγωμεν ἀπὸ προσώπου Ἰσραήλ· ὁ γὰρ Κύριος πολεμεῖ περὶ αὐτῶν τοὺς Αἰγυπτίους.
\vs{26}Εἶπε δὲ Κύριος πρὸς Μωυσῆν, ἔκτεινον τὴν χεῖρά σου ἐπὶ τὴν θάλασσαν, καὶ ἀποκαταστήτω τὸ ὕδωρ, καὶ ἐπικαλυψάτω τοὺς Αἰγυπτίους, ἐπί τε τὰ ἅρματα καὶ τοὺς ἀναβάτας.
\vs{27}Ἐξέτεινε δὲ Μωυσῆς τὴν χεῖρα ἐπὶ τὴν θάλασσαν, καὶ ἀπεκατέστη τὸ ὕδωρ πρὸς ἡμέραν ἐπὶ χώρας· οἱ δὲ Αἰγύπτιοι ἔφυγον ὑπὸ τὸ ὕδωρ· καὶ ἐξετίναξε Κύριος τοὺς Αἰγυπτίους μέσον τῆς θαλάσοης.
\vs{28}Καὶ ἐπαναστραφὲν τὸ ὕδωρ ἐκάλυψε τὰ ἅρματα καὶ τοὺς ἀναβάτας, καὶ πᾶσαν τὴν δύναμιν Φαραὼ, τοὺς εἰσπεπορευμένους ὀπίσω αὐτῶν εἰς τὴν θάλασσαν· καὶ οὐ κατελείφθη ἐξ αὐτῶν οὐδὲ εἷς.
\vs{29}Οἱ δὲ υἱοὶ Ἰσραὴλ ἐπορεύθησαν διὰ ξηρᾶς ἐν μέσῳ τῆς θάλασσης· τὸ δὲ ὕδωρ αὐτοῖς τεῖχος ἐκ δεξιῶν, καὶ τεῖχος ἐξ εὐωνύμων.
\vs{30}Καὶ ἐῤῥύσατο Κύριος τὸν Ἰσραὴλ ἐν τῇ ἡμέρᾳ ἐκείνῃ ἐκ χειρὸς τῶν Αἰγυπτίων· καὶ εἶδεν Ἰσραὴλ τοὺς Αἰγυπτίους τεθνηκότας παρὰ τὸ χεῖλος τῆς θαλάσσης.
\vs{31}Εἶδε δὲ Ἰσραὴλ τὴν χεῖρα τὴν μεγάλην, ἃ ἐποίησε Κύριος τοῖς Αἰγυπτίοις· ἐφοβήθη δὲ ὁ λαὸς τὸν Κύριον, καὶ ἐπίστευσαν τῷ Θεῷ, καὶ Μωυσῇ τῷ θεράποντι αὐτοῦ.

\ch{15}
Τότε ᾖσε Μωυσῆς καὶ οἱ υἱοὶ Ἰσραὴλ τὴν ᾠδὴν ταύτην τῷ Θεῷ, καὶ εἶπαν, λέγοντες, ᾄσωμεν τῷ Κυρίῳ, ἐνδόξως γὰρ δεδόξασται· ἵππον καὶ ἀναβάτην ἔῤῥιψεν εἰς θάλασσαν.
\vs{2}Βοηθὸς καὶ σκεπαστὴς ἐγένετό μοι εἰς σωτηρίαν· οὗτός μου Θεὸς, καὶ δοξάσω αὐτόν· Θεὸς τοῦ πατρός μου, καὶ ὑψώσω αὐτόν.
\vs{3}Κύριος συντρίβων πολέμους, Κύριος ὄνομα αὐτῷ.
\vs{4}Ἅρματα Φαραὼ, καὶ τὴν δύναμιν αὐτοῦ, ἔῤῥιψεν εἰς θάλασσαν, ἐπιλέκτους ἀναβάτας τριστάτας· κατεπόθησαν ἐν ἐρυθρᾷ θαλάσσῃ·
\vs{5}Πόντῳ ἐκάλυψεν αὐτούς· κατέδυσαν εἰς βυθὸν ὡσεὶ λίθος.
\vs{6}Ἡ δεξιά σου, Κύριε, δεδόξασται ἐν ἰσχύϊ· ἡ δεξιά σου χεὶρ, Κύριε, ἔθραυσεν ἐχθρούς.
\vs{7}Καὶ τῷ πλήθει τῆς δόξης σου συνέτριψας τοὺς ὑπεναντίους· ἀπέστειλας τὴν ὀργήν σου κατέφαγεν αὐτοὺς ὡς καλάμην.
\vs{8}Καὶ διὰ πνεύματος τοῦ θυμοῦ σου διέστη τὸ ὕδωρ· ἐπάγη ὡσεὶ τεῖχος τὰ ὕδατα· ἐπάγη τὰ κύματα ἐν μέσῳ τῆς θαλάσσης.
\vs{9}Εἶπεν ὁ ἐχθρὸς, διώξας καταλήψομαι, μεριῶ σκῦλα· ἐμπλήσω ψυχήν μου, ἀνελῶ τῇ μαχαίρῃ μου, κυριεύσει ἡ χείρ μου.
\vs{10}Ἀπέστειλας τὸ πνεῦμά σου· ἐκάλυψεν αὐτοὺς θάλασσα· ἔδυσαν ὡσεὶ μόλιβος ἐν ὕδατι σφοδρῷ.
\vs{11}Τίς ὅμοιός σοι ἐν θεοῖς, Κύριε; τίς ὅμοιός σοι; δεδοξασμένος ἐν ἁγίοις, θαυμαστὸς ἐν δόξαις, ποιῶν τέρατα.
\vs{12}Ἐξέτεινας τὴν δεξιάν σου· κατέπιεν αὐτοὺς γῆ.
\vs{13}Ὡδήγησας τῇ δικαιοσύνῃ σου τὸν λαόν σου τοῦτον, ὃν ἐλυτρώσω· παρεκάλεσας τῇ ἰσχύϊ σου εἰς κατάλυμα ἅγιόν σου.
\vs{14}Ἤκουσαν ἔθνη, καὶ ὠργίσθησαν· ὠδῖνες ἔλαβον κατοικοῦντας Φυλιστιείμ.
\vs{15}Τότε ἔσπευσαν ἡγεμόνες Ἐδὼμ, καὶ ἄρχοντες Μωαβιτῶν· ἔλαβεν αὐτοὺς τρόμος· ἐτάκησαν πάντες οἱ κατοικοῦντες Χαναάν.
\vs{16}Ἐπιπέσοι ἐπʼ αὐτοὺς τρόμος καὶ φόβος· μεγέθει βραχίονός σου ἀπολιθωθήτωσαν, ἕως ἂν παρέλθῃ ὁ λαός σου, Κύριε· ἕως ἂν παρέλθῃ ὁ λαός σου οὗτος, ὃν ἐκτήσω.
\vs{17}Εἰσαγαγὼν καταφύτευσον αὐτοὺς εἰς ὄρος κληρονομίας σου, εἰς ἕτοιμον κατοικητήριόν σου, ὃ κατηρτίσω, Κύριε, ἁγίασμα, Κύριε, ὃ ἡτοίμασαν αἱ χεῖρές σου.
\vs{18}Κύριος βασιλεύων τὸν αἰῶνα, καὶ ἐπʼ αἰῶνα, καὶ ἔτι.
\vs{19}Ὅτι εἰσῆλθεν ἵππος Φαραὼ σὺν ἅρμασιν καὶ ἀναβάταις εἰς θάλασσαν, καὶ ἐπήγαγεν ἐπʼ αὐτοὺς Κύριος τὸ ὕδωρ τῆς θαλάσσης· οἱ δὲ υἱοὶ Ἰσραὴλ ἐπορεύθησαν διὰ ξηρᾶς ἐν μέσῳ τῆς θαλάσσης.

\vs{20}Λαβοῦσα δὲ Μαριὰμ ἡ προφῆτις ἡ ἀδελφὴ Ἀαρὼν τὸ τύμπανον ἐν τῇ χειρὶ αὐτῆς, καὶ ἐξήλθοσαν πᾶσαι αἱ γυναῖκες ὀπίσω αὐτῆς μετὰ τυμπάνων καὶ χορῶν.
\vs{21}Ἐξῆρχε δὲ αὐτῶν Μαριὰμ, λέγουσα, ᾄσωμεν τῷ Κυρίῳ, ἐνδόξως γὰρ δεδόξασται· ἵππον καὶ ἀναβάτην ἔῤῥιψεν εἰς θάλασσαν.
\vs{22}Ἐξῆρε δὲ Μωυσῆς τοὺς υἱοὺς Ἰσραὴλ ἀπὸ θαλάσσης ἐρυθρᾶς, καὶ ἤγαγεν αὐτοὺς εἰς τὴν ἔρημον Σούρ· καὶ ἐπορεύοντο τρεῖς ἡμέρας ἐν τῇ ἐρήμῳ, καὶ οὐχ ηὕρισκον ὕδωρ, ὥστε πιεῖν.
\vs{23}Ἦλθον δὲ εἰς Μεῤῥᾶ, καὶ οὐκ ἠδύναντο πιεῖν ἐκ Μεῤῥᾶς· πικρὸν γὰρ ἦν· διὰ τοῦτο ἐπωνόμασε τὸ ὄνομα τοῦ τόπου ἐκείνου, Πικρία.
\vs{24}Καὶ διεγόγγυζεν ὁ λαὸς ἐπὶ Μωυσῇ, λέγοντες, τί πιόμεθα;
\vs{25}Ἐβόησε δὲ Μωυσῆς πρὸς Κύριον· καὶ ἔδειξεν αὐτῷ Κύριος ξύλον, καὶ ἐνέβαλεν αὐτὸ εἰς τὸ ὕδωρ, καὶ ἐγλυκάνθη τὸ ὕδωρ· ἐκεῖ ἔθετο αὐτῷ δικαιώματα καὶ κρίσεις· καὶ ἐκεῖ αὐτὸν ἐπείρασε,
\vs{26}καὶ εἶπεν, ἐὰν ἀκοῇ ἀκούσῃς τῆς φωνῆς Κυρίου τοῦ Θεοῦ σου, καὶ τὰ ἀρεστὰ ἐναντίον αὐτοῦ ποιήσῃς, καὶ ἐνωτίσῃ ταῖς ἐντολαῖς αὐτοῦ, καὶ φυλάξῃς πάντα τὰ δικαιώματα αὐτοῦ, πᾶσαν νόσον, ἣν ἐπήγαγον τοῖς Αἰγυπτίοις, οὐκ ἐπάξω ἐπὶ σέ· ἐγὼ γάρ εἰμι Κύριος ὁ Θεός σου ὁ ἰώμενός σε.
\vs{27}Καὶ ἤλθοσαν εἰς Αἰλείμ· καὶ ἦσαν ἐκεῖ δώδεκα πηγαὶ ὑδάτων, καὶ ἑβδομήκοντα στελέχη φοινίκων· παρενέβαλον δὲ ἐκεῖ παρὰ τὰ ὕδατα.

\ch{16}
Ἀπῄραν δὲ ἐξ Αἰλεὶμ, καὶ ἤλθοσαν πᾶσα συναγωγὴ υἱῶν Ἰσραὴλ εἰς τὴν ἔρημον Σὶν, ὅ ἐστιν ἀνὰ μέσον Αἰλεὶμ, καὶ ἀνὰ μέσον Σινά. τῇ δὲ πεντεκαιδεκάτῃ ἡμέρᾳ, τῷ μηνὶ τῷ δευτέρῳ ἐξεληλυθότων αὐτῶν ἐκ γῆς Αἰγύπτου,
\vs{2}διεγόγγυζε πᾶσα συναγωγὴ υἱῶν Ἰσραὴλ ἐπὶ Μωυσὴν καὶ Ἀαρών.
\vs{3}Καὶ εἶπεν πρὸς αὐτοὺς οἱ υἱοὶ Ἰσραήλ, ὄφελον ἀπεθάνομεν πληγέντες ὑπὸ Κυρίου ἐν γῇ Αἰγύπτῳ, ὅταν ἐκαθίσαμεν ἐπὶ τῶν λεβήτων τῶν κρεῶν, καὶ ἠσθίομεν ἄρτους εἰς πλησμονήν· ὅτι ἐξηγάγετε ἡμᾶς εἰς τὴν ἔρημον ταύτην, ἀποκτεῖναι πᾶσαν τὴν συναγωγὴν ταύτην ἐν λιμῷ.
\vs{4}Εἶπε δὲ Κύριος πρὸς Μωυσῆν, ἰδοὺ ἐγὼ ὕω ὑμῖν ἄρτους ἐκ τοῦ οὐρανοῦ· καὶ ἐξελεύσεται ὁ λαὸς, καὶ συλλέξουσι τὸ τῆς ἡμέρας εἰς ἡμέραν, ὅπως πειράσω αὐτοὺς εἰ πορεύσονται τῷ νόμῳ μου, ἢ οὔ·
\vs{5}Καὶ ἔσται ἐν τῇ ἡμέρᾳ τῇ ἕκτῃ, καὶ ἑτοιμάσουσιν ὃ ἐὰν εἰσενέγκωσι· καὶ ἔσται διπλοῦν ὃ ἐὰν συναγάγωσι τὸ καθʼ ἡμέραν εἰς ἡμέραν.
\vs{6}Καὶ εἶπε Μωυσῆς καὶ Ἀαρὼν πρὸς πάσαν συναγωγὴν υἱῶν Ἰσραὴλ, ἑσπέρας γνώσεσθε, ὅτι Κύριος ἐξήγαγεν ὑμᾶς ἐκ γῆς Αἰγύπτου,
\vs{7}καὶ πρωῒ ὄψεσθε τὴν δόξαν Κυρίου ἐν τῷ εἰσακοῦσαι τὸν γογγυσμὸν ὑμῶν ἐπὶ τῷ Θεῷ· ἡμεῖς δὲ τί ἐσμεν, ὅτι διαγογγύζετε καθʼ ἡμῶν;
\vs{8}Καὶ εἶπε Μωυσῆς, ἐν τῷ διδόναι Κύριον ὑμῖν ἑσπέρας κρέα φαγεῖν, καὶ ἄρτους τὸ πρωῒ εἰς πλησμονὴν, διὰ τὸ εἰσακοῦσαι Κύριον τὸν γογγυσμὸν ὑμῶν, ὃν ὑμεῖς διαγογγύζετε καθʼ ἡμῶν· ἡμεῖς δὲ τί ἐσμεν; οὐ γὰρ καθʼ ἡμῶν ἐστιν ὁ γογγυσμὸς ὑμῶν, ἀλλʼ ἢ κατὰ τοῦ Θεοῦ.

\vs{9}Εἶπε δὲ Μωυσῆς πρὸς Ἀαρὼν, εἶπον πάσῃ συναγωγῇ υἱῶν Ἰσραὴλ, προσέλθετε ἐναντίον τοῦ Θεοῦ· εἰσακήκοε γὰρ τὸν γογγυσμὸν ὑμῶν.
\vs{10}Ἡνίκα δὲ ἐλάλει Ἀαρὼν πάσῃ συναγωγῇ υἱῶν Ἰσραὴλ, καὶ ἐπεστράφησαν εἰς τὴν ἔρημον, καὶ ἡ δόξα Κυρίου ὤφθη ἐν νεφέλῃ.
\vs{11}καὶ ἐλάλησε Κύριος πρὸς Μωυσῆν, λέγων,
\vs{12}εἰσακήκοα τὸν γογγυσμὸν τῶν υἱῶν Ἰσραήλ· λάλησον πρὸς αὐτοὺς, λέγων, τὸ πρὸς ἑσπέραν ἔδεσθε κρέα, καὶ τὸ πρωῒ πλησθήσεσθε ἄρτων· καὶ γνώσεσθε, ὅτι ἐγὼ Κύριος ὁ Θεὸς ὑμῶν.
\vs{13}Ἐγένετο δὲ ἑσπέρα· καὶ ἀνέβη ὀρτυγομήτρα, καὶ ἐκάλυψε τὴν παρεμβολήν· τὸ πρωῒ ἐγένετο καταπαυομένης τῆς δρόσου κύκλῳ τῆς παρεμβολῆς.
\vs{14}Καὶ ἰδοὺ ἐπὶ πρόσωπον τῆς ἐρήμου λεπτὸν ὡσεὶ κόριον λευκὸν, ὡσεὶ πάγος ἐπὶ τῆς γῆς.
\vs{15}Ἰδόντες δὲ αὐτὸ οἱ υἱοὶ Ἰσραὴλ, εἶπαν ἕτερος τῷ ἑτέρῳ, τί ἐστι τοῦτο; οὐ γὰρ ᾔδεισαν τί ἦν· εἶπε δὲ Μωυσῆς αὐτοῖς, οὗτος ὁ ἄρτος, ὃν ἔδωκε Κύριος ὑμῖν φαγεῖν.
\vs{16}Τοῦτο τὸ ῥῆμα ὃ συνέταξε Κύριος· συναγάγετε ἀπʼ αὐτοῦ ἕκαστος εἰς τοὺς καθήκοντας γομὸρ, κατὰ κεφαλὴν κατὰ ἀριθμὸν ψυχῶν ὑμῶν, ἕκαστος σὺν τοῖς συσκηνίοις ὑμῶν συλλέξατε.
\vs{17}Ἐποίησαν δὲ οὕτως οἱ υἱοὶ Ἰσραήλ· καὶ συνέλεξαν ὁ τὸ πολὺ καὶ ὁ τὸ ἔλαττον.
\vs{18}Καὶ μετρήσαντες γομὸρ, οὐκ ἐπλεόνασεν ὁ τὸ πόλυ, καὶ ὁ τὸ ἔλαττον οὐκ ἠλαττόνησεν· ἕκαστος εἰς τοὺς καθήκοντας παρʼ ἑαυτῷ συνέλεξαν.
\vs{19}Εἶπε δὲ Μωυσῆς πρὸς αὐτοὺς, μηδεὶς καταλειπέτω ἀπʼ αὐτοῦ εἰς τὸ πρωΐ.

\vs{20}Καὶ οὐκ εἰσήκουσαν Μωυσῆ, ἀλλὰ κατέλιπόν τινες ἀπʼ αὐτοῦ εἰς τὸ πρωΐ· καὶ ἐξέζεσε σκώληκας, καὶ ἐπώζεσε· καὶ ἐπικράνθη ἐπʼ αὐτοῖς Μωυσῆς.
\vs{21}Καὶ συνέλεξαν αὐτὸ πρωῒ πρωῒ, ἕκαστος τὸ καθῆκον αὐτῷ· ἡνίκα δὲ διεθέρμαινεν ὁ ἥλιος, ἐτήκετο.
\vs{22}Ἐγένετο δὲ τῇ ἡμέρᾳ τῇ ἕκτῃ, συνέλεξαν τὰ δέοντα διπλᾶ, δύο γομὸρ τῷ ἑνί· εἰσήλθοσαν δὲ πάντες οἱ ἄρχοντες τῆς συναγωγῆς, καὶ ἀνήγγειλαν Μωυσῇ.
\vs{23}Εἶπε δὲ Μωυσῆς πρὸς αὐτούς, οὐ τοῦτο τὸ ῥῆμά ἐστιν ὃ ἐλάλησε Κύριος; σάββατα ἀνάπαυσις ἁγία τῷ Κυρίῳ αὔριον· ὅσα ἐὰν πέσσητε, πέσσετε· καὶ ὅσα ἐὰν ἕψητε, ἕψετε· καὶ πᾶν τὸ πλεονάζον καταλείπετε αὐτὸ εἰς ἀποθήκην εἰς τὸ πρωΐ.
\vs{24}Καὶ κατελίποσαν ἀπʼ αὐτοῦ εἰς ἕως πρωῒ, καθὼς συνέταξεν αὐτοῖς Μωυσῆς· καὶ οὐκ ἐπώζεσεν, οὐδὲ σκώληξ ἐγένετο ἐν αὐτῷ.
\vs{25}Εἶπε δὲ Μωυσῆς, φάγετε σήμερον· ἔστι γὰρ σάββατα αήμερον τῷ Κυρίῳ· οὐχ εὑρεθήσεται ἐν τῷ πεδίῳ.
\vs{26}Ἓξ ἡμέρας συλλέξετε· τῇ δὲ ἡμέρᾳ τῇ ἑβδόμῃ σάββατα, ὅτι οὐκ ἔσται ἐν αὐτῇ.
\vs{27}Ἐγένετο δὲ ἐν τῇ ἡμέρᾳ τῇ ἑβδόμῃ ἐξήθλοσάν τινες ἐκ τοῦ λαοῦ συλλέξαι, καὶ οὐχ εὗρον.
\vs{28}Εἶπε δὲ Κύριος πρὸς Μωυσῆν, ἕως τίνος οὐ βούλεσθε εἰσακούειν τὰς ἐντολάς μου, καὶ τὸν νόμον μου;
\vs{29}Ἴδετε, ὁ γὰρ Κύριος ἔδωκεν ὑμῖν σάββατα τὴν ἡμέραν ταύτην· διὰ τοῦτο αὐτὸς ἔδωκεν ὑμῖν τῇ ἡμέρᾳ τῇ ἕκτῃ ἄρτους δύο ἡμερῶν· καθίσεσθε ἕκαστος εἰς τοὺς οἴκους ὑμῶν· μηδεὶς ἐκπορευέσθω ἐκ τοῦ τόπου αὐτοῦ τῇ ἡμέρᾳ τῇ ἑβδόμῃ.
\vs{30}Καὶ ἐσαββάτισεν ὁ λαὸς τῇ ἡμέρᾳ τῇ ἑβδόμῃ.
\vs{31}Καὶ ἐπωνόμασαν αὐτὸ οἱ υἱοὶ Ἰσραὴλ τὸ ὄνομα αὐτοῦ, Μάν· ἦν δὲ ὡσεὶ σπέρμα κορίου λευκόν· τὸ δὲ γεῦμα αὐτοῦ ὡς ἐγκρὶς ἐν μέλιτι.
\vs{32}Εἶπε δὲ Μωυσῆς, τοῦτο τὸ ῥῆμα, ὃ συνέταξε Κύριος, πλήσατε τὸ γομὸρ τοῦ μὰν, εἰς ἀποθήκην εἰς τὰς γενεὰς ὑμῶν· ἵνα ἴδωσι τὸν ἄρτον, ὃν ἐφάγετε ὑμεῖς ἐν τῇ ἐρήμῳ, ὡς ἐξήγαγεν ὑμᾶς Κύριος ἐκ γῆς Αἰγύπτου.
\vs{33}Καὶ εἶπε Μωυσῆς πρὸς Ἀαρὼν, λάβε στάμνον χρυσοῦν ἕνα, καὶ ἔμβαλε εἰς αὐτὸν πλῆρες τὸ γομὸρ τοῦ μὰν, καὶ ἀποθήσεις αὐτὸ ἐναντίον τοῦ Θεοῦ, εἰς διατήρησιν εἰς τὰς γενεὰς ὑμῶν,
\vs{34}ὃν τρόπον συνέταξε Κύριος τῷ Μωυσῇ· καὶ ἀπέθηκεν Ἀαρὼν ἐναντίον τοῦ μαρτυρίου εἰς διατήρησιν.
\vs{35}Οἱ δὲ υἱοὶ Ἰσραὴλ ἔφαγον τὸ μὰν ἔτη τεσσαράκοντα, ἕως ἦλθον εἰς τὴν οἰκουμένην ἐφάγοσαν τὸ μὰν, ἕως παρεγένοντο εἰς μέρος τῆς Φοινίκης.
\vs{36}Τὸ δὲ γομὸρ τὸ δέκατον τῶν τριῶν μέτρων ἦν.

\ch{17}
Καὶ ἀπῇρε πᾶσα συναγωγὴ υἱῶν Ἰσραὴλ ἐκ τῆς ἐρήμου Σὶν κατὰ παρεμβολὰς αὐτῶν, διὰ ῥήματος Κυρίου· καὶ παρενεβάλοσαν ἐν Ῥαφιδείν· οὐκ ἦν δὲ ὕδωρ τῷ λαῷ πιεῖν.
\vs{2}Καὶ ἐλοιδορεῖτο ὁ λαὸς πρὸς Μωυσῆν, λέγοντες, δὸς ἡμῖν ὕδωρ, ἵνα πίωμεν· καὶ εἶπεν αὐτοῖς Μωυσῆς, τί λοιδορεῖσθέ μοι, καὶ τί πειράζετε Κύριον;
\vs{3}Ἐδίψησε δὲ ἐκεῖ ὁ λαὸς ὕδατι· καὶ διεγόγγυσεν ἐκεῖ ὁ λαὸς πρὸς Μωυσῆν, λέγοντες, ἱνατί τοῦτο; ἀνεβίβασας ἡμᾶς ἐξ Αἰγύπτου ἀποκτεῖναι ἡμᾶς καὶ τὰ τέκνα ἡμῶν καὶ τὰ κτήνη τῷ δίψει;
\vs{4}Ἐβόησε δὲ Μωυσῆς πρὸς Κύριον, λέγων, τί ποιήσω τῷ λαῷ τούτῳ; ἔτι μικρὸν, καὶ καταλιθοβολὴσουσί με.
\vs{5}Καὶ εἶπε Κύριος πρὸς Μωυσῆν, προπορεύου τοῦ λαοῦ τούτου· λάβε δὲ σεαυτῷ ἀπὸ τῶν πρεσβυτέρων τοῦ λαοῦ· καὶ τὴν ῥάβδον, ἐν ᾗ ἐπάταξας τὸν ποταμὸν, λάβε ἐν τῇ χειρί σου, καὶ πορεύσῃ.
\vs{6}Ὅδε ἐγὼ ἕστηκα ἐκεῖ πρὸ τοῦ σὲ ἐπὶ τῆς πέτρας ἐν Χωρήβ· καὶ πατάξεις τὴν πέτραν, καὶ ἐξελεύσεται ἐξ αὐτῆς ὕδωρ, καὶ πίεται ὁ λαός. Ἐποίησε δὲ Μωυσῆς οὕτως ἐναντίον τῶν υἱῶν Ἰσραήλ.
\vs{7}Καὶ ἐπωνόμασε τὸ ὄνομα τοῦ τόπου ἐκείνου, Πειρασμὸς, καὶ Λοιδόρησις, διὰ τὴν λοιδορίαν τῶν υἱῶν Ἰσραὴλ, καὶ διὰ τὸ πειράζειν Κύριον, λέγοντας, εἰ ἔστι Κύριος ἐν ἡμῖν, ἢ οὔ;

\vs{8}Ἦλθε δὲ Ἀμαλὴκ καὶ ἐπολέμει Ἰσραὴλ ἐν Ῥαφιδείν.
\vs{9}Εἶπε δὲ Μωυσῆς τῷ Ἰησοῖ, Ἐπίλεξον σεαυτῷ ἄνδρας δυνατοὺς, καὶ ἐξελθὼν παράταξαι τῷ Ἀμαλὴκ αὔριον· καὶ ἰδοὺ ἐγὼ ἕστηκα ἐπὶ τῆς κορυφῆς τοῦ βουνοῦ, καὶ ἡ ῥάβδος τοῦ Θεοῦ ἐν τῇ χειρί μου.
\vs{10}Καὶ ἐποίησεν Ἰησοῦς καθάπερ εἶπεν αὐτῷ Μωυσῆς, καὶ ἐξελθὼν παρετάξατο τῷ Ἀμαλήκ· καὶ Μωυσῆς καὶ Ἀαρὼν καὶ Ὢρ ἀνέβησαν ἐπὶ τὴν κορυφὴν τοῦ βουνοῦ.
\vs{11}Καὶ ἐγένετο ὅταν ἐπῇρε Μωυσῆς τὰς χεῖρας, κατίσχυεν Ἰσραήλ· ὅταν δὲ καθῆκε τὰς χεῖρας, κατίσχυεν Ἀμαλήκ.
\vs{12}Αἱ δὲ χεῖρες Μωυσῆ βαρεῖαι· καὶ λαβόντες λίθον ὑπέθηκαν ὑπʼ αὐτὸν, καὶ ἐκάθητο ἐπʼ αὐτοῦ· καὶ Ἀαρὼν καὶ Ὢρ ἐστήριζον τὰς χεῖρας αὐτοῦ ἐντεῦθεν εἷς, καὶ ἐντεῦθεν εἷς· καὶ ἐγένοντο αἱ χεῖρες Μωυσῆ ἐστηριγμέναι ἕως δυσμῶν ἡλίου.
\vs{13}Καὶ ἐτρέψατο Ἰησοῦς τὸν Ἀμαλὴκ, καὶ πάντα τὸν λαὸν αὐτοῦ ἐν φόνῳ μαχαίρας.
\vs{14}Εἶπε δὲ Κύριος πρὸς Μωυσῆν, Κατάγραψον τοῦτο εἰς μνημόσυνον εἰς βιβλίον, καὶ δὸς εἰς τὰ ὦτα Ἰησοῖ· ὅτι ἀλοιφῇ ἐξαλείψω τὸ μνημόσυνον Ἀμαλὴκ ἐκ τῆς ὑπὸ τὸν οὐρανόν.
\vs{15}Καὶ ᾠκοδόμησε Μωυσῆς θυσιαστήριον Κυρίῳ· καὶ ἐπωνόμασε τὸ ὄνομα αὐτοῦ, Κύριος καταφυγή μου.
\vs{16}Ὅτι ἐν χειρὶ κρυφαίᾳ πολεμεῖ Κύριος ἐπὶ Ἀμαλὴκ ἀπὸ γενεῶν εἰς γενεάς.

\ch{18}
Ἤκουσε δὲ Ἰοθὸρ ἱερεὺς Μαδιὰμ ὁ γαμβρὸς Μωυσῆ πάντα ὅσα ἐποίησε Κύριος Ἰσραὴλ τῷ ἑαυτοῦ λαῷ· ἐξήγαγε γὰρ Κύριος τὸν Ἰσραὴλ ἐξ Αἰγύπτου.
\vs{2}Ἔλαβε δὲ Ἰοθὸρ ὁ γαμβρὸς Μωυσῆ Σεπφώραν τὴν γυναῖκα Μωυσῆ μετὰ τὴν ἄφεσιν αὐτῆς,
\vs{3}καὶ τοὺς δύο υἱοὺς αὐτῆς· ὄνομα τῷ ἑνὶ αὐτῶν Γηρσάμ, λέγων, πάροικος ἤμην ἐν γῇ ἀλλοτρίᾳ·
\vs{4}καὶ τὸ ὄνομα τοῦ δευτέρου Ἐλίεζερ, λέγων, ὁ γὰρ Θεὸς τοῦ πατρός μου βοηθός μου, καὶ ἐξείλατό με ἐκ χειρὸς Φαραώ.
\vs{5}Καὶ ἐξῆλθεν Ἰοθὸρ ὁ γαμβρὸς Μωυσῆ καὶ οἱ υἱοὶ καὶ ἡ γυνὴ πρὸς Μωυσῆν εἰς τὴν ἔρημον, οὗ παρενέβαλεν ἐπʼ ὄρους τοῦ Θεοῦ.
\vs{6}Ἀνηγγέλη δὲ Μωυσῇ, λέγοντες, ἰδοὺ ὁ γαμβρός σου Ἰοθὸρ παραγίνεται πρὸς σέ, καὶ ἡ γυνὴ καὶ οἱ δύο υἱοί σου μετʼ αὐτοῦ.
\vs{7}Ἐξῆλθε δὲ Μωυσῆς εἰς συνάντησιν τῷ γαμβρῷ, καὶ προσεκύνησεν αὐτῷ, καὶ ἐφίλησεν αυτὸν, καὶ ἠσπάσαντο ἀλλήλους, καὶ εἰσήγαγεν αὐτοὺς εἰς τὴν σκηνήν.
\vs{8}Καὶ διηγήσατο Μωυσῆς τῷ γαμβρῷ πάντα ὅσα ἐποίησε Κύριος τῷ Φαραῷ καὶ πᾶσι τοῖς Αἰγυπτίοις ἕνεκεν τοῦ Ἰσραήλ, καὶ πάντα τὸν μόχθον τὸν γενόμενον αὐτοῖς ἐν τῇ ὁδῷ, καὶ ὅτι ἐξείλατο αὐτοὺς Κύριος ἐκ χειρὸς Φαραὼ, καὶ ἐκ χειρὸς τῶν Αἰγυπτίων.
\vs{9}Ἐξέστη δὲ Ἰοθὸρ ἐπὶ πᾶσι τοῖς ἀγαθοῖς οἷς ἐποίησεν αὐτοῖς Κύριος, ὅτι ἐξείλατο αὐτοὺς ἐκ χειρὸς Αἰγυπτίων καὶ ἐκ χειρὸς Φαραώ.
\vs{10}Καὶ εἶπεν Ἰοθὸρ, εὐλογητὸς Κύριος, ὅτι ἐξείλατο αὐτοὺς ἐκ χειρὸς Αἰγυπτίων καὶ ἐκ χειρὸς Φαραώ.
\vs{11}Νῦν ἔγνων ὅτι μέγας Κύριος παρὰ πάντας τοὺς θεούς ἕνεκεν τούτου, ὅτι ἐπέθεντο αὐτοῖς.
\vs{12}Καὶ ἔλαβεν Ἰοθὸρ ὁ γαμβρὸς Μωυσῆ ὁλοκαυτώματα καὶ θυσίας τῷ Θεῷ· παρεγένετο δὲ Ἀαρὼν καὶ πάντες οἱ πρεσβύτεροι Ἰσραὴλ συμφαγεῖν ἄρτον μετὰ τοῦ γαμβροῦ Μωυσῆ, ἐναντίον τοῦ Θεοῦ.

\vs{13}Καὶ ἐγένετο μετὰ τὴν ἐπαύριον συνεκάθισε Μωυσῆς κρίνειν τὸν λαόν· παρειστήκει δὲ πᾶς ὁ λαὸς Μωυσῇ ἀπὸ πρωΐθεν ἕως δείλης.
\vs{14}Καὶ ἰδὼν Ἰοθὸρ πάντα ὅσα ποιεῖ τῷ λαῷ, λέγει, τί τοῦτο ὃ σὺ ποιεῖς τῷ λαῷ; διατί σὺ κάθησαι μόνος, πᾶς δὲ ὁ λαὸς παρέστηκέ σοι ἀπὸ πρωΐθεν ἕως δείλης;
\vs{15}Καὶ λέγει Μωυσῆς τῷ γαμβρῶ, Ὅτι παραγίνεται πρός με ὁ λαὸς ἐκζητῆσαι κρίσιν παρὰ τοῦ Θεοῦ.
\vs{16}Ὅταν γὰρ γένηται αὐτοῖς ἀντιλογία, καὶ ἔλθωσι πρός με, διακρίνω ἕκαστον, καὶ συμβιβάζω αὐτοὺς τὰ προστάγματα τοῦ Θεοῦ καὶ τὸν νόμον αὐτοῦ.
\vs{17}Εἶπε δὲ ὁ γαμβρὸς Μωυσῆ πρὸς αὐτὸν, οὐκ ὀρθῶς σὺ ποιεῖς τὸ ῥῆμα τοῦτο.
\vs{18}Φθορᾷ καταφθαρήσῃ ἀνυπομονήτῳ καὶ σὺ, καὶ πᾶς ὁ λαὸς οὗτος, ὅς ἐστι μετὰ σοῦ· βαρύ σοι τὸ ῥῆμα τοῦτο· οὐ δυνήσῃ ποιεῖν σὺ μόνος.
\vs{19}Νῦν οὖν ἄκουσόν μου, καὶ συμβουλεύσω σοι, καὶ ἔσται ὁ Θεὸς μετὰ σοῦ· γίνου σὺ τῷ λαῷ τὰ πρὸς τὸν Θεὸν, καὶ ἀνοίσεις τοὺς λόγους αὐτῶν πρὸς τὸν Θεόν.
\vs{20}Καὶ διαμαρτύρῇ αὐτοῖς τὰ προστάγματα τοῦ Θεοῦ καὶ τὸν νόμον αὐτοῦ, καὶ σημανεῖς αὐτοῖς τὰς ὁδοὺς ἐν αἷς πορεύσονται ἐν αὐταῖς, καὶ τὰ ἔργα ἃ ποιήσουσι.
\vs{21}Καὶ σὺ σεαυτῷ σκέψαι ἀπὸ παντὸς τοῦ λαοῦ ἄνδρας δυνατοὺς, θεοσεβεῖς, ἄνδρας δικαίους, μισοῦντας ὑπερηφανίαν, καὶ καταστήσεις ἐπʼ αὐτῶν χιλιάρχους καὶ ἑκατοντάρχους καὶ πεντηκοντάρχους καὶ δεκαδάρχους.
\vs{22}Καὶ κρινοῦσι τὸν λαὸν πᾶσαν ὥραν· τὸ δὲ ῥῆμα τὸ ὑπέρογκον ἀνοίσουσιν ἐπὶ σὲ· τὰ δὲ βραχέα τῶν κριμάτων κρινοῦσιν αὐτοί· καὶ κουφιοῦσιν ἀπὸ σοῦ, καὶ συναντιλήψονταί σοι.
\vs{23}Ἐὰν τὸ ῥῆμα τοῦτο ποιήσῃς, κατισχύσει σε ὁ Θεὸς, καὶ δυνήσῃ παραστῆναι, καὶ πᾶς ὁ λαὸς οὗτος εἰς τὸν ἑαυτοῦ τόπον μετʼ εἰρήνης ἥξει.
\vs{24}Ἤκουσε δὲ Μωυσῆς τῆς φωνῆς τοῦ γαμβροῦ, καὶ ἐποίησεν ὅσα εἶπεν αὐτῷ.
\vs{25}Καὶ ἐπέλεξε Μωυσῆς ἄνδρας δυνατοὺς ἀπὸ παντὸς Ἰσραὴλ, καὶ ἐποίησεν αὐτοὺς ἐπʼ αὐτῶν χιλιάρχους καὶ ἑκατοντάρχους καὶ πεντηκοντάρχους καὶ δεκαδάρχους.
\vs{26}Καὶ ἐκρίνοσαν τὸν λαὸν πᾶσαν ὥραν· πᾶν δὲ ῥῆμα ὑπέρογκον ἀνεφέροσαν ἐπὶ Μωυσῆν· πᾶν δὲ ῥῆμα ἐλαφρὸν ἐκρίνοσαν αὐτοί.
\vs{27}Ἐξαπέστειλε δὲ Μωυσῆς τὸν ἑαυτοῦ γαμβρὸν, καὶ ἀπῆλθεν εἰς τὴν γῆν αὐτοῦ.

\ch{19}
Τοῦ δὲ μηνὸς τοῦ τρίτου τῆς ἐξόδου τῶν υἱῶν Ἰσραὴλ ἐκ γῆς Αἰγύπτου τῇ ἡμέρᾳ ταύτῃ, ἤλθοσαν εἰς τὴν ἔρημον τοῦ Σινά.
\vs{2}Καὶ ἀπῆραν ἐκ Ῥαφιδεὶν, καὶ ἤλθοσαν εἰς τὴν ἔρημον τοῦ Σινὰ, καὶ παρενέβαλεν ἐκεῖ Ἰσραὴλ κατέναντι τοῦ ὄρους.
\vs{3}Καὶ Μωυσῆς ἀνέβη εἰς τὸ ὄρος τοῦ Θεοῦ· καὶ ἐκάλεσεν αὐτὸν ὁ Θεὸς ἐκ τοῦ ὄρους, λέγων, τάδε ἐρεῖς τῷ οἴκῳ Ἰακὼβ, καὶ ἀναγγελεῖς τοῖς υἱοῖς Ἰσραήλ.
\vs{4}Αὐτοὶ ἑωράκατε ὅσα πεποίηκα τοῖς Αἱγυπτίοις, καὶ ἀνέλαβον ὑμᾶς ὡσεὶ ἐπὶ πτερύγων ἀετῶν, καὶ προσηγαγόμην ὑμᾶς πρὸς ἐμαυτόν.
\vs{5}Καὶ νῦν ἐὰν ἀκοῇ ἀκούσητε τῆς ἐμῆς φωνῆς, καὶ φυλάξητε τὴν διαθήκην μου, ἔσεσθέ μοι λαὸς περιούσιος ἀπὸ πάντων τῶν ἐθνῶν· ἐμὴ γάρ ἐστι πᾶσα ἡ γῆ.
\vs{6}Ὑμεῖς δὲ ἔσεσθέ μοι βασίλειον ἱεράτευμα καὶ ἔθνος ἅγιον· ταῦτα τὰ ῥήματα ἐρεῖς τοῖς υἱοῖς Ἰσραήλ.
\vs{7}Ἦλθε δὲ Μωυσῆς, καὶ ἐκάλεσεν τοὺς πρεσβυτέρους τοῦ λαοῦ· καὶ παρέθηκεν αὐτοῖς πάντας τοὺς λόγους τούτους, οὓς συνέταξεν αὐτοῖς ὁ Θεός.
\vs{8}Ἀπεκρίθη δὲ πᾶς ὁ λαὸς ὁμοθυμαδὸν, καὶ εἶπαν, πάντα ὅσα εἶπεν ὁ Θεὸς, ποιήσομεν καὶ ἀκουσόμεθα· ἀνήνεγκε δὲ Μωυσῆς τοὺς λόγους τούτους πρὸς τὸν Θεόν.
\vs{9}Εἶπε δὲ Κύριος πρὸς Μωυσῆν, ἰδοὺ ἐγὼ παραγίνομαι πρὸς σὲ ἐν στύλῳ νεφέλης, ἵνα ἀκούσῃ ὁ λαὸς λαλοῦντός μου πρὸς σὲ, καὶ σοὶ πιστεύσωσιν εἰς τὸν αἰῶνα· ἀνήγγειλε δὲ Μωυσῆς τὰ ῥήματα τοῦ λαοῦ πρὸς Κύριον.
\vs{10}Εἶπε δὲ Κύριος πρὸς Μωυσῆν, Καταβὰς διαμάρτυραι τῷ λαῷ, καὶ ἅγνισον αὐτοὺς σήμερον καὶ αὔριον, καὶ πλυνάτωσαν τὰ ἱμάτια,
\vs{11}καὶ ἔστωσαν ἕτοιμοι εἰς τὴν ἡμέραν τὴν τρίτην· τῇ γὰρ ἡμέρᾳ τῇ τρίτῃ καταβήσεται Κύριος ἐπὶ τὸ ὄρος τὸ Σινὰ, ἐναντίον παντὸς τοῦ λαοῦ.
\vs{12}Καὶ ἀφοριεῖς τὸν λαὸν κύκλῳ, λέγων, προσέχετε ἑαυτοῖς τοῦ ἀναβῆναι εἰς τὸ ὄρος, καὶ θίγειν τι αὐτοῦ· πᾶς ὁ ἁψάμενος τοῦ ὄρους, θανάτῳ τελευτήσει.
\vs{13}Οὐχ ἅψεται αὐτοῦ χείρ· ἐν γὰρ λίθοις λιθοβοληθήσεται, ἢ βολίδι κατατοξευθήσεται· ἐάν τε κτῆνος ἐάν τε ἄνθρωπος, οὐ ζήσεται· ὅταν αἱ φωναὶ καὶ αἱ σάλπιγγες καὶ ἡ νεφέλη ἀπέλθῃ ἀπὸ τοῦ ὄρους, ἐκεῖνοι ἀναβήσονται ἐπὶ τὸ ὄρος.

\vs{14}Κατέβη δὲ Μωυσῆς ἐκ τοῦ ὄρους πρὸς τὸν λαὸν, καὶ ἡγίασεν αὐτούς· καὶ ἔπλυναν τὰ ἱμάτια.
\vs{15}Καὶ εἶπε τῷ λαῷ, γίνεσθε ἕτοιμοι, τρεῖς ἡμέρας μὴ προσέλθητε γυναικί.
\vs{16}Ἐγένετο δὲ τῇ ἡμέρᾳ τῇ τρίτῃ γενηθέντος πρὸς ὄρθρον, καὶ ἐγένοντο φωναὶ καὶ ἀστραπαὶ καὶ νεφέλη γνοφώδης ἐπʼ ὄρους Σινά· φωνὴ τῆς σάλπιγγος ἤχει μέγα· καὶ ἐπτοήθη πᾶς ὁ λαὸς ὁ ἐν τῇ παρεμβολῇ.
\vs{17}Καὶ ἐξήγαγε Μωυσῆς τὸν λαὸν εἰς συνάντησιν τοῦ Θεοῦ ἐκ τῆς παρεμβολῆς· καὶ παρέστησαν ὑπὸ τὸ ὄρος.
\vs{18}Τὸ ὄρος τὸ Σινὰ ἐκαπνίζετο ὅλον, διὰ τὸ καταβεβηκέναι ἐπʼ αὐτὸ τὸν Θεὸν ἐν πυρί· καὶ ἀνέβαινεν ὁ καπνὸς, ὡσεὶ καπνὸς καμίνου· καὶ ἐξέστη πᾶς ὁ λαὸς σφόδρα.
\vs{19}Ἐγίνοντο δὲ αἱ φωναὶ τῆς σάλπιγγος προβαίνουσαι ἰσχυρότεραι σφόδρα. Μωυσῆς ἐλάλησεν, ὁ δὲ Θεὸς ἀπεκρίνατο αὐτῷ φωνῇ.
\vs{20}Κατέβη δὲ Κύριος ἐπὶ τὸ ὄρος τὸ Σινὰ ἐπὶ τὴν κορυφὴν τοῦ ὄρους· καὶ ἐκάλεσε Κύριος Μωυσῆν ἐπὶ τὴν κορυφὴν τοῦ ὄρους· καὶ ἀνέβη Μωυσῆς.
\vs{21}Καὶ εἶπεν ὁ Θεὸς πρὸς Μωυσῆν, λέγων, καταβὰς διαμάρτυραι τῷ λαῷ, μή ποτε ἐγγίσωσι πρὸς τὸν Θεὸν κατανοῆσαι, καὶ πέσωσιν ἐξ αὐτῶν πλῆθος·
\vs{22}Καὶ οἱ ἱερεῖς οἱ ἐγγίζοντες Κυρίῳ τῷ Θεῷ ἁγιασθήτωσαν, μήποτε ἀπαλλάξῃ ἀπʼ αὐτῶν Κύριος.

\vs{23}Καὶ εἶπε Μωυσῆς πρὸς τὸν Θεὸν, οὐ δυνήσεται ὁ λαὸς προσαναβῆναι πρὸς τὸ ὄρος τὸ Σινά· σὺ γὰρ διαμεμαρτύρησαι ἡμῖν, λέγων, ἀφόρισαι τὸ ὄρος, καὶ ἁγίασαι αὐτό.
\vs{24}Εἴπε δὲ αὐτῷ Κύριος, βάδιζε, κατάβηθι, καὶ ἀνάβηθι σὺ καὶ Ἀαρὼν μετὰ σοῦ· οἱ δὲ ἱερεῖς καὶ ὁ λαὸς μὴ βιαζέσθωσαν ἀναβῆναι πρὸς τὸν Θεὸν, μὴ ποτε ἀπολέσῃ ἀπʼ αὐτῶν Κύριος.
\vs{25}Κατέβη δὲ Μωυσῆς πρὸς τὸν λαὸν, καὶ εἶπεν αὐτοῖς.

\ch{20}
Καὶ ἐλάλησε Κύριος πάντας τοὺς λόγους τούτους, λέγων,
\vs{2}ἐγώ εἰμι Κύριος ὁ Θεός σου, ὅστις ἐξήγαγόν σε ἐκ γῆς Αἰγύπτου, ἐξ οἴκου δουλείας.
\vs{3}Οὐκ ἔσονταί σοι θεοὶ ἕτεροι πλὴν ἐμοῦ.
\vs{4}Οὐ ποιήσεις σεαυτῷ εἴδωλον, οὐδὲ παντὸς ὁμοίωμα, ὅσα ἐν τῷ οὐρανῷ ἄνω, καὶ ὅσα ἐν τῇ γῇ κάτω, καὶ ὅσα ἐν τοῖς ὕδασιν ὑποκάτω τῆς γῆς.
\vs{5}Οὐ προσκυνήσεις αὐτοῖς, οὐδὲ μὴ λατρεύσεις αὐτοῖς· ἐγὼ γάρ εἰμι Κύριος ὁ Θεός σου, Θεὸς ζηλωτὴς, ἀποδιδοὺς ἁμαρτίας πατέρων ἐπὶ τέκνα, ἕως τρίτης καὶ τετάρτης γενεᾶς τοῖς μισοῦσί με,
\vs{6}καὶ ποιῶν ἔλεος εἰς χιλιάδας τοῖς ἀγαπῶσί με, καὶ τοῖς φυλάσσουσι τὰ προστάγματά μου.
\vs{7}Οὐ λήψῃ τὸ ὄνομα Κυρίου τοῦ Θεοῦ σου ἐπὶ ματαίῳ· οὐ γὰρ μὴ καθαρίσῃ Κύριος ὁ Θεός σου τὸν λαμβάνοντα τὸ ὄνομα αὐτοῦ ἐπὶ ματαίῳ.
\vs{8}Μνήσθητι τὴν ἡμέραν τῶν σαββάτων ἁγιάζειν αὐτήν.
\vs{9}Ἓξ ἡμέρας ἐργᾷ, καὶ ποιήσεις πάντα τὰ ἔργα σου.
\vs{10}Τῇ δὲ ἡμέρᾳ τῇ ἑβδόμῃ, σάββατα Κυρίῳ τῷ Θεῷ σου· οὐ ποιήσεις ἐν αὐτῇ πᾶν ἔργον σὺ, καὶ ὁ υἱός σου, καὶ ἡ θυγάτηρ σου, ὁ παῖς σου, καὶ ἡ παιδίσκη σου, ὁ βοῦς σου, καὶ τὸ ὑποζύγιόν σου, καὶ πᾶν κτῆνός σου, καὶ ὁ προσήλυτος ὁ παροικῶν ἐν σοί.
\vs{11}Ἐν γὰρ ἓξ ἡμέραις ἐποίησε Κύριος τὸν οὐρανὸν καὶ τὴν γῆν καὶ τὴν θὰλασσαν καὶ πάντα τὰ ἐν αὐτοῖς, καὶ κατέπαυσε τῇ ἡμέρᾳ τῇ ἑβδόμῃ· διὰ τοῦτο εὐλόγησε Κύριος τὴν ἡμέραν τὴν ἑβδόμην, καὶ ἡγίασεν αὐτήν.
\vs{12}Τίμα τὸν πατέρα σου, καὶ τὴν μητέρα σου, ἵνα εὖ σοι γένηται, καὶ ἵνα μακροχρόνιος γένῃ ἐπὶ τῆς γῆς τῆς ἀγαθῆς, ἧς Κύριος ὁ Θεός σου δίδωσί σοι.
\vs{13}Οὐ μοιχεύσεις.
\vs{14}Οὐ κλέψεις.
\vs{15}Οὐ φονεύσεις.
\vs{16}Οὐ ψευδομαρτυρήσεις κατὰ τοῦ πλησίον σου μαρτυρίαν ψευδῆ.
\vs{17}Οὐκ ἐπιθυμήσεις τὴν γυναῖκα τοῦ πλησίον σου· οὐκ ἐπιθυμήσεις τὴν οἰκίαν τοῦ πλησίον σου, οὔτε τὸν ἀγρὸν αὐτοῦ, οὔτε τὸν παῖδα αὐτοῦ, οὔτε τὴν παιδίσκην αὐτοῦ, οὔτε τοῦ βοὸς αὐτοῦ, οὔτε τοῦ ὑποζυγίου αὐτοῦ, οὔτε παντὸς κτήνους αὐτοῦ, οὔτε ὅσα τῷ πλησίον σου ἐστί.

\vs{18}Καὶ πᾶς ὁ λαὸς ἑώρα τὴν φωνὴν, καὶ τὰς λαμπάδας, καὶ τὴν φωνὴν τῆς σάλπιγγος, καὶ τὸ ὄρος τὸ καπνίζον· φοβηθέντες δὲ πᾶς ὁ λαὸς ἔστησαν μακρόθεν.
\vs{19}Καὶ εἶπαν πρὸς Μωυσῆν, λάλησον σὺ ἡμῖν, καὶ μὴ λαλείτω πρὸς ἡμᾶς ὁ Θεὸς, μὴ ἀποθάνωμεν.
\vs{20}Καὶ λέγει αὐτοῖς Μωυσῆς, θαρσεῖτε· ἕνεκεν γὰρ τοῦ πειράσαι ὑμᾶς παρεγενήθη ὁ Θεὸς πρὸς ὑμᾶς, ὅπως ἂν γένηται ὁ φόβος αὐτοῦ ἐν ὑμῖν, ἵνα μὴ ἁμαρτάνητε.
\vs{21}Εἱστήκει δὲ ὁ λαὸς μακρόθεν, Μωυσῆς δὲ εἰσῆλθεν εἰς τὸν γνόφον, οὗ ἦν ὁ Θεός.
\vs{22}Εἶπε δὲ Κύριος πρὸς Μωυσῆν, τάδε ἐρεῖς τῷ οἴκῳ Ἰακὼβ, καὶ ἀναγγελεῖς τοῖς υἱοῖς Ἰσραήλ· ὑμεῖς ἑωράκατε, ὅτι ἐκ τοῦ οὐρανοῦ λελάληκα πρὸς ὑμᾶς.
\vs{23}Οὐ ποιήσετε ὑμῖν αὐτοῖς θεοὺς ἀργυροῦς, καὶ θεοὺς χρυσοῦς οὐ ποιήσετε ὑμῖν αὑτοῖς.
\vs{24}Θυσιαστήριον ἐκ γῆς ποιήσετέ μοι, καὶ θύσετε ἐπʼ αὐτοῦ τὰ ὁλοκαυτώματα ὑμῶν, καὶ τὰ σωτήρια ὑμῶν, καὶ τὰ πρόβατα, καὶ τοὺς μόσχους ὑμῶν ἐν παντὶ τόπῳ, οὗ ἐὰν ἐπονομάσω τὸ ὄνομά μου ἐκεῖ, καὶ ἥξω πρὸς σὲ, καὶ εὐλογήσω σε.
\vs{25}Ἐὰν δὲ θυσιαστήριον ἐκ λίθων ποιῇς μοι, οὐκ οἰκοδομήσεις αὐτοὺς τμητούς· τὸ γὰρ ἐγχειρίδιόν σου ἐπιβέβληκας ἐπʼ αὐτοὺς, καὶ μεμίανται.
\vs{26}Οὐκ ἀναβήσῃ ἐν ἀναβαθμίσιν ἐπὶ τὸ θυσιαστήριόν μου, ὅπως ἂν μὴ ἀποκαλύψῃς τὴν ἀσχημοσύνην σου ἐπʼ αὐτοῦ.

\ch{21}
Καὶ ταῦτα τὰ δικαιώματα, ἃ παραθήσῃ ἐνώπιον αὐτῶν.
\vs{2}Ἐὰν κτήσῃ παῖδα Ἐβραῖον, ἓξ ἔτη δουλεύσει σοι· τῷ δὲ ἑβδόμῳ ἔτει ἀπελεύσεται ἐλεύθερος δωρεάν.
\vs{3}Ἐὰν αὐτὸς μόνος εἰσέλθῃ, καὶ μόνος ἐξελεύσεται· ἐὰν δὲ γυνὴ συνεισέλθῃ μετʼ αὐτοῦ, ἐξελεύσεται καὶ ἡ γυνὴ αὐτοῦ.
\vs{4}Καὶ ἐὰν δὲ ὁ κύριος δῷ αὐτῷ γυναῖκα, καὶ τέκῃ αὐτῷ υἱοὺς ἢ θυγατέρας, ἡ γυνὴ καὶ τὰ παιδία ἔσται τῷ κυρίῳ αὐτοῦ, αὐτὸς δὲ μόνος ἐξελεύσεται.
\vs{5}Ἐὰν δὲ ἀποκριθεὶς εἴπῃ ὁ παῖς, ἠγάπηκα τὸν κύριόν μου, καὶ τὴν γυναῖκα, καὶ τὰ παιδία, οὐκ ἀποτρέχω ἐλεύθερος·
\vs{6}προσάξει αὐτὸν ὁ κύριος αὐτοῦ πρὸς τὸ κριτήριον τοῦ Θεοῦ, καὶ τότε προσάξει αὐτὸν ἐπὶ τὴν θύραν ἐπὶ τὸν σταθμὸν, καὶ τρυπήσει ὁ κύριος αὐτοῦ τὸ οὖς τῷ ὀπητίῳ, καὶ δουλεύσει αὐτῷ εἰς τὸν αἰῶνα.

\vs{7}Ἐὰν δέ τις ἀποδῶται τὴν ἑαυτοῦ θυγατέρα οἰκέτιν, οὐκ ἀπελεύσεται, ὥσπερ ἀποτρέχουσιν αἱ δοῦλαι.
\vs{8}Ἐὰν μὴ εὐαρεστήσῃ τῷ κυρίῳ αὐτῆς, ἣ αὐτῷ καθωμολογήσατο, ἀπολυτρώσει αὐτήν· ἔθνει δὲ ἀλλοτρίῳ οὐ κύριός ἐστι πωλεῖν αὐτὴν, ὅτι ἠθέτησεν ἐν αὐτῇ.
\vs{9}Ἐὰν δὲ τῷ υἱῷ καθομολογήσηται αὐτὴν, κατὰ τὸ δικαίωμα τῶν θυγατέρων ποιήσει αὐτῇ.
\vs{10}Ἐὰν δὲ ἄλλην λάβῃ ἑαυτῷ, τὰ δέοντα καὶ τὸν ἱματισμὸν καὶ τὴν ὁμιλίαν αὐτῆς οὐκ ἀποστερήσει.
\vs{11}Ἐὰν δὲ τὰ τρία ταῦτα μὴ ποιήσῃ αὐτῇ, ἐξελεύσεται δωρεὰν ἄνευ ἀργυρίου.
\vs{12}Ἐὰν δὲ πατάξῃ τις τινὰ, καὶ ἀποθάνῃ, θανάτῳ θανατούσθω.
\vs{13}Ὁ δὲ οὐχ ἑκὼν, ἀλλὰ ὁ Θεὸς παρέδωκεν εἰς τὰς χεῖρας αὐτοῦ, δώσω σοι τόπον οὗ φεύξεται ἐκεῖ ὁ φονεύσας.
\vs{14}Ἐὰν δέ τις ἐπιθῆται τῷ πλησίον ἀποκτεῖναι αὐτὸν δόλῳ, καὶ καταφύγῃ, ἀπὸ τοῦ θυσιαστηρίου μου λήψῃ αὐτὸν θανατῶσαι.
\vs{15}Ὃς τύπτει πατέρα αὐτοῦ ἢ μητέρα αὐτοῦ, θανάτῳ θανατούσθω.
\vs{16}Ὁ κακολογῶν πατέρα αὐτοῦ ἢ μητέρα αὐτοῦ, τελευτήσει θανάτῳ.
\vs{17}Ὃς ἐὰν κλέψῃ τις τινὰ τῶν υἱῶν Ἰσραὴλ, καὶ καταδυναστεύσας αὐτὸν ἀποδῶται, καὶ εὑρεθῇ ἐν αὐτῷ, θανάτῳ τελευτάτω.
\vs{18}Ἐὰν δὲ λοιδορῶνται δύο ἄνδρες, καὶ πατάξωσι τὸν πλησίον λίθῳ ἢ πυγμῇ, καὶ μὴ ἀποθάνῃ, κατακλιθῇ δὲ ἐπὶ τὴν κοίτην,
\vs{19}ἐὰν ἐξαναστὰς ὁ ἄνθρωπος περιπατήσῃ ἔξω ἐπὶ ῥάβδου, ἀθῶος ἔσται ὁ πατάξας· πλὴν τῆς ἀργείας αὐτοῦ ἀποτίσει, καὶ τὰ ἰατρεῖα.
\vs{20}Ἐὰν δέ τις πατάξῃ τὸν παῖδα αὐτοῦ ἢ τὴν παιδίσκην αὐτοῦ ἐν ῥάβδῳ, καὶ ἀποθάνῃ ὑπὸ τὰς χεῖρας αὐτοῦ, δίκῃ ἐκδικηθήσεται.
\vs{21}Ἐὰν δὲ διαβιώσῃ ἡμέραν μίαν ἢ δύο, οὐκ ἐκδικηθήτω· τὸ γὰρ ἀργύριον αὐτοῦ ἐστιν.
\vs{22}Ἐὰν δὲ μάχωνται δύο ἄνδρες, καὶ πατάξωσι γυναῖκα ἐν γαστρὶ ἔχουσαν, καὶ ἐξέλθῃ τὸ παιδίον αὐτῆς μὴ ἐξεικονισμένον, ἐπιζήμιον ζημιωθήσεται· καθότι ἂν ἐπιβάλῃ ὁ ἀνὴρ τῆς γυναικὸς, δώσει μετὰ ἀξιώματος.
\vs{23}Ἐὰν δὲ ἐξεικονισμένον ᾖ, δώσει ψυχὴν ἀντὶ ψυχῆς,
\vs{24}Ὀφθαλμὸν ἀντὶ ὀφθαλμοῦ, ὀδόντα ἀντὶ ὀδόντος, χεῖρα ἀντὶ χειρὸς, πόδα ἀντὶ ποδὸς,
\vs{25}κατάκαυμα ἀντὶ κατακαύματος, τραῦμα ἀντὶ τραύματος, μώλωπα ἀντὶ μώλωπος.
\vs{26}Ἐὰν δέ τις πατάξῃ τὸν ὀφθαλμὸν τοῦ οἰκέτου αὐτοῦ, ἢ τὸν ὀφθαλμὸν τῆς θεραπαίνης αὐτοῦ, καὶ ἐκτυφλώσῃ, ἐλευθέρους ἐξαποστελεῖ αὐτοὺς ἀντὶ τοῦ ὀφθαλμοῦ αὐτῶν.
\vs{27}Ἐὰν δὲ τὸν ὀδόντα τοῦ οἰκέτου, ἢ τὸν ὀδόντα τῆς θεραπαίνης αὐτοῦ ἐκκόψῃ, ἐλευθέρους ἐξαποστελεῖ αὐτοὺς ἀντὶ τοῦ ὀδόντος αὐτῶν.
\vs{28}Ἐὰν δὲ κερατίσῃ ταῦρος ἄνδρα ἢ γυναῖκα καὶ ἀποθάνῃ, λίθοις λιθοβοληθήσεται ὁ ταῦρος, καὶ οὐ βρωθήσεται τὰ κρέα αὐτοῦ· ὁ δὲ κύριος τοῦ ταύρου ἀθῶος ἔσται.
\vs{29}Ἐὰν δὲ ὁ ταῦρος κερατιστὴς ᾖ πρὸ τῆς χθὲς καὶ πρὸ τῆς τρίτης, καὶ διαμαρτύρωνται τῷ κυρίῳ αὐτοῦ, καὶ μὴ ἀφανίσῃ αὐτὸν, ἀνέλῃ δὲ ἄνδρα ἢ γυναῖκα, ὁ ταῦρος λιθοβοληθήσεται, καὶ ὁ κύριος αὐτοῦ προσαποθανεῖται.
\vs{30}Ἐὰν δὲ λύτρα ἐπιβληθῇ αὐτῷ, δώσει λύτρα τῆς ψυχῆς αὐτοῦ ὅσα ἐὰν ἐπιβάλωσιν αὐτῶ.
\vs{31}Ἐὰν δὲ υἱὸν ἢ θυγατέρα κερατίσῃ, κατὰ τὸ δικαίωμα τοῦτο ποιήσωσιν αὐτῷ.
\vs{32}Ἐὰν δὲ παῖδα κερατίσῃ ὁ ταῦρος ἢ παιδίσκην, ἀργυρίου τριάκοντα δίδραχμα δώσει τῷ κυρίῳ αὐτῶν, καὶ ὁ ταῦρος λιθοβοληθήσεται.
\vs{33}Ἐὰν δέ τις ἀνοίξῃ λάκκον ἢ λατομήσῃ λάκκον, καὶ μὴ καλύψῃ αὐτὸν, καὶ ἐμπέσῃ ἐκεῖ μόσχος ἢ ὄνος,
\vs{34}ὁ κύριος τοῦ λάκκου ἀποτίσει, ἀργύριον δώσει τῷ κυρίῳ αὐτῶν· τὸ δὲ τετελευτηκὸς αὐτῷ ἔσται.
\vs{35}Ἐὰν δὲ κερατίσῃ τινὸς ταῦρος τόν ταῦρον τοῦ πλησίον, καὶ τελευτήσῃ, ἀποδώσονται τὸν ταῦρον τὸν ζῶντα, καὶ διελοῦνται τὸ ἀργύριον αὐτοῦ, καὶ τὸν ταῦρον τὸν τεθνηκότα διελοῦνται.
\vs{36}Ἐὰν δὲ γνωρίζηται ὁ ταῦρος ὅτι κερατιστής ἐστι πρὸ τῆς χθὲς καὶ πρὸ τῆς τρίτης ἡμέρας, καὶ διαμεμαρτυρημένοι ὦσι τῷ κυρίῳ αὐτοῦ· καὶ μὴ ἀφανίσῃ αὐτὸν, ἀποτίσει ταῦρον ἀντὶ ταύρου, ὁ δὲ τετελευτηκὼς αὐτῷ ἔσται.

\vs{37}Ἐὰν δέ τις κλέψῃ μόσχον ἢ πρόβατον, καὶ σφάξῃ ἢ ἀποδῶται, πέντε μόσχους ἀποτίσει ἀντὶ τοῦ μόσχου, καὶ τέσσερα πρόβατα ἀντὶ τοῦ προβάτου.

\ch{22}Ἐὰν δὲ ἐν τῷ διορύγματι εὑρεθῇ ὁ κλέπτης, καὶ πληγεὶς ἀποθάνῃ, οὐκ ἔστιν αὐτῷ φόνος.
\vs{2}Ἐὰν δὲ ἀνατείλῃ ὁ ἥλιος ἐπʼ αὐτῷ, ἔνοχός ἐστιν, ἀνταποθανεῖται· ἐὰν δὲ μὴ ὑπάρχῃ αὐτῷ, πραθήτω ἀντὶ τοῦ κλέμματος.
\vs{3}Ἐὰν δὲ καταλειφθῇ καὶ εὑρεθῇ ἐν τῇ χειρὶ αὐτοῦ τὸ κλέμμα ἀπό τε ὄνου ἕως προβάτου ζῶντα, διπλᾶ αὐτὰ ἀποτίσει.
\vs{4}Ἐὰν δὲ καταβοσκήσῃ τις ἀγρὸν ἢ ἀμπελῶνα, καὶ ἀφῇ τὸ κτῆνος αὐτοῦ καταβοσκῆσαι ἀγρὸν ἕτερον, ἀποτίσει ἐκ τοῦ ἀγροῦ αὐτοῦ κατὰ τὸ γέννημα αὐτοῦ· ἐὰν δὲ πάντα τὸν ἀγρὸν καταβοσκήσῃ, τὰ βέλτιστα τοῦ ἀγροῦ αὐτοῦ καὶ τὰ βέλτιστα τοῦ ἀμπελῶνος αὐτοῦ ἀποτίσει.
\vs{5}Ἐὰν δὲ ἐξελθὸν πῦρ εὕρῃ ἀκάνθας, καὶ προσεμπρήσῃ ἅλωνας ἢ στάχυς ἢ πεδίον, ἀποτίσει ὁ τὸ πῦρ ἐκκαύσας.

\vs{6}Ἐὰν δέ τις δῷ τῷ πλησίον ἀργύριον ἢ σκεύη φυλάξαι, καὶ κλαπῇ ἐκ τῆς οἰκίας τοῦ ἀνθρώπου, ἐὰν εὑρεθῇ ὁ κλέψας, ἀποτίσει τὸ διπλοῦν.
\vs{7}Ἐὰν δὲ μὴ εὑρεθῇ ὁ κλέψας, προσελεύσεται ὁ κύριος τῆς οἰκίας ἐνώπιον τοῦ Θεοῦ, καὶ ὀμεῖται ἦ μὴν μὴ αὐτὸν πεπονηρεῦσθαι ἐφʼ ὅλης τῆς παρακαταθήκης τοῦ πλησίον,
\vs{8}κατὰ πᾶν ῥητὸν ἀδίκημα, περί τε μόσχου, καὶ ὑποζυγίου, καὶ προβάτου, καὶ ἱματίου, καὶ πάσης ἀπωλίας τῆς ἐνκαλουμένης· ὅ, τι οὖν ἂν ᾖ, ἐνώπιον τοῦ Θεοῦ ἐλεύσεται ἡ κρίσις ἀμφοτέρων, καὶ ὁ ἁλοὺς διὰ τοῦ Θεοῦ, ἀποτίσει διπλοῦν τῷ πλησίον.
\vs{9}Ἐὰν δέ τις δῷ τῷ πλησίον ὑποζύγιον ἢ μόσχον ἢ πρόβατον ἢ πᾶν κτῆνος φυλάξαι, καὶ συντριβῇ ἢ τελευτήσῃ ἢ αἰχμάλωτον γένηται, καὶ μηδεὶς γνῷ,
\vs{10}ὅρκος ἔσται τοῦ Θεοῦ ἀνὰ μέσον ἀμφοτέρων, ἦ μὴν μὴ αὐτὸν πεπονηρεῦσθαι καθόλου τῆς παρακαταθήκης τοῦ πλησίον· καὶ οὕτως προσδέξεται ὁ κύριος αὐτοῦ, καὶ οὐκ ἀποτίσει.
\vs{11}Ἐὰν δὲ κλαπῇ παρʼ αὐτοῦ, ἀποτίσει τῷ κυρίῳ·
\vs{12}Ἐὰν δὲ θηριάλωτον γένηται, ἄξει αὐτὸν ἐπὶ τὴν θήραν, καὶ οὐκ ἀποτίσει.
\vs{13}Ἐὰν δὲ αἰτήσῃ τις παρὰ τοῦ πλησίον, καὶ συντριβῇ ἢ ἀποθάνῃ ἢ αἰχμάλωτον γένηται, ὁ δὲ κύριος μὴ ᾖ μετʼ αὐτοῦ, ἀποτίσει.
\vs{14}Ἐὰν δὲ ὁ κύριος ᾖ μετʼ αὐτοῦ, οὐκ ἀποτίσει· ἐὰν δὲ μισθωτὸς ᾖ, ἔσται αὐτῷ ἀντὶ τοῦ μισθοῦ αὐτοῦ.

\vs{15}Ἐὰν δὲ ἀπατήσῃ τις παρθένον ἀμνήστευτον, καὶ κοιμηθῇ μετʼ αὐτῆς, φερνῇ φερνιεῖ αὐτὴν αὐτῷ γυναῖκα.
\vs{16}Ἐὰν δὲ ἀνανεύων ἀνανεύσῃ, καὶ μὴ βούληται ὁ πατὴρ αὐτῆς δοῦναι αὐτὴν αὐτῷ γυναῖκα, ἀργύριον ἀποτίσει τῷ πατρὶ καθʼ ὅσον ἐστὶν ἡ φερνὴ τῶν παρθένων.
\vs{17}Φαρμακοὺς οὐ περιποιήσετε.
\vs{18}Πᾶν κοιμώμενον μετὰ κτήνους θανάτῳ ἀποκτενεῖτε αὐτούς.
\vs{19}Ὁ θυσιάζων θεοῖς θανάτῳ ἐξολοθρευθήσεται, πλὴν Κυρίῳ μόνῳ.

\vs{20}Καὶ προσήλυτον οὐ κακώσετε, οὐδὲ μὴ θλίψητε αὐτόν· ἦτε γάρ προσήλυτοι ἐν γῇ Αἰγύπτῳ.
\vs{21}Πᾶσαν χήραν καὶ ὀρφανὸν οὐ κακώσετε.
\vs{22}Ἐὰν δὲ κακίᾳ κακώσητε αὐτοὺς, καὶ κεκράξαντες καταβοήσωσι πρός με, ἀκοῇ εἰσακούσομαι τῆς φωνῆς αὐτῶν,
\vs{23}καὶ ὀργισθήσομαι θυμῷ, καὶ ἀποκτενῶ ὑμᾶς μαχαίρᾳ, καὶ ἔσονται αἱ γυναῖκες ὑμῶν χῆραι, καὶ τὰ παιδία ὑμῶν ὀρφανά.
\vs{24}Ἐὰν δὲ ἀργύριον ἐκδανείσῃς τῷ ἀδελφῷ τῷ πενιχρῷ παρὰ σοὶ, οὐκ ἔσῃ αὐτὸν κατεπείγων, οὐκ ἐπιθήσεις αὐτῷ τόκον.
\vs{25}Ἐὰν δὲ ἐνεχύρασμα ἐνεχυράσῃς τὸ ἱμάτιον τοῦ πλησίον, πρὸ δυσμῶν ἡλίου ἀποδώσεις αὐτῷ·
\vs{26}Ἔστι γὰρ τοῦτο περιβόλαιον αὐτοῦ, μόνον τοῦτο τὸ ἱμάτιον ἀσχημοσύνης αὐτοῦ· ἐν τίνι κοιμηθήσεται; Ἐὰν οὖν καταβοήσῃ πρός μέ, εἰσακούσομαι αὐτοῦ· ἐλεήμων γάρ εἰμι.
\vs{27}Θεοὺς οὐ κακολογήσεις, καὶ ἄρχοντα τοῦ λαοῦ σου οὐ κακῶς ἐρεῖς.
\vs{28}Ἀπαρχὰς ἅλωνος καὶ ληνοῦ σου οὐ καθυστερήσεις· τὰ πρωτότοκα τῶν υἱῶν σου δώσεις ἐμοί.
\vs{29}Οὕτω ποιήσεις τὸν μόσχον σου καὶ τὸ πρόβατόν σου καὶ τὸ ὑποζύγιόν σου· ἑπτὰ ἡμέρας ἔσται ὑπὸ τὴν μητέρα, τῇ δὲ ὀγδόῃ ἡμέρᾳ ἀποδώσεις μοι αὐτό.
\vs{30}Καὶ ἄνδρες ἅγιοι ἔσεσθέ μοι· καὶ κρέας θηριάλωτον οὐκ ἔδεσθε, τῷ κυνὶ ἀποῤῥίψατε αὐτό.

\ch{23}
Οὐ παραδέξῃ ἀκοὴν ματαίαν· οὐ συγκαταθήσῃ μετὰ τοῦ ἀδίκου γενέσθαι μάρτυς ἄδικος.
\vs{2}Οὐκ ἔσῃ μετὰ πλειόνων ἐπὶ κακίᾳ· οὐ προστεθήσῃ μετὰ πλήθους ἐκκλῖναι μετὰ τῶν πλειόνων, ὥστε ἑκκλεῖσαι κρίσιν.
\vs{3}Καὶ πένητα οὐκ ἐλεήσεις ἐν κρίσει.
\vs{4}Ἐὰν δὲ συναντήσῃς τῷ βοῒ τοῦ ἐχθροῦ σου, ἢ τῷ ὑποζυγίῳ αὐτοῦ πλανωμένοις, ἀποστρέψας ἀποδώσεις αὐτῷ.
\vs{5}Ἐὰν δὲ ἴδῃς τὸ ὑποζύγιον τοῦ ἐχθροῦ σου πεπτωκὸς ὑπὸ τὸν γόμον αὐτοῦ, οὐ παρελεύσῃ αὐτὸ, ἀλλὰ συναρεῖς αὐτὸ μετʼ αὐτοῦ.

\vs{6}Οὐ διαστρέψεις κρίμα πένητος ἐν κρίσει αὐτοῦ.
\vs{7}Ἀπὸ παντὸς ῥήματος ἀδίκου ἀποστήσῃ· ἀθῷον καὶ δίκαιον οὐκ ἀποκτενεῖς· καὶ οὐ δικαιώσεις τὸν ἀσεβῆ ἕνεκεν δώρων.
\vs{8}Καὶ δῶρα οὐ λήψῃ· τὰ γὰρ δῶρα ἐκτυφλοῖ ὀφθαλμοὺς βλεπόντων, καὶ λυμαῖνεται ῥήματα δίκαια.
\vs{9}Καὶ προσήλυτον οὐ θλίψετε· ὑμεῖς γὰρ οἴδατε τὴν ψυχὴν τοῦ προσηλύτου· αὐτοὶ γὰρ προσήλυτοι ἦτε ἐν γῇ Αἰγύπτῳ.
\vs{10}Ἓξ ἔτη σπερεῖς τὴν γῆν σου, καὶ συνάξεις τὰ γεννήματα αὐτῆς.
\vs{11}Τῷ δὲ ἑβδόμῳ ἄφεσιν ποιήσεις, καὶ ἀνήσεις αὐτὴν, καὶ ἔδονται οἱ πτωχοὶ τοῦ ἔθνους σου· τὰ δὲ ὑπολειπόμενα ἔδεται τὰ ἄγρια θηρία· οὕτως ποιήσεις τὸν ἀμπελῶνά σου, καὶ τὸν ἐλαιῶνά σου.
\vs{12}Ἓξ ἡμέρας ποιήσεις τὰ ἔργα σου, τῇ δὲ ἡμέρᾳ τῇ ἑβδόμῃ, ἀνάπαυσις· ἵνα ἀναπαύσηται ὁ βοῦς σου, καὶ τὸ ὑποζύγιόν σου, καὶ ἵνα ἀναψύξῃ ὁ υἱὸς τῆς παιδίσκης σου καὶ ὁ προσήλυτος.
\vs{13}Πάντα ὅσα εἴρηκα πρὸς ὑμᾶς, φυλάξασθε· καὶ ὄνομα θεῶν ἑτέρων οὐκ ἀναμνησθήσεσθε, οὐδὲ μὴ ἀκουσθῇ ἐκ τοῦ στόματος ὑμῶν.

\vs{14}Τρεῖς καιροὺς τοῦ ἐνιαυτοῦ ἑορτάσατέ μοι.
\vs{15}Τὴν ἑορτὴν τῶν ἀζύμων φυλάξασθε ποιεῖν· ἑπτὰ ἡμέρας ἔδεσθε ἄζυμα, καθάπερ ἐνετειλάμην σοι κατὰ τὸν καιρὸν τοῦ μηνὸς τῶν νέων· ἐν γὰρ αὐτῷ ἐξῆλθες ἐξ Αἰγύπτου· οὐκ ὀφθήσῃ ἐνώπίον μου κενός.
\vs{16}Καὶ ἑορτὴν θερισμοῦ πρωτογεννημάτων ποιήσεις τῶν ἔργων σου, ὧν ἐὰν σπείρῃς ἐν τῷ ἀγρῷ σου, καὶ ἑορτὴν συντελείας ἐπʼ ἐξόδου τοῦ ἐνιαυτοῦ ἐν τῇ συναγωγῇ τῶν ἔργων σου τῶν ἐκ τοῦ ἀγροῦ σου.
\vs{17}Τρεῖς καιροὺς τοῦ ἐνιαυτοῦ ὀφθήσεται πᾶν ἀρσενικόν σου ἐνώπιον Κυρίου τοῦ Θεοῦ σου.
\vs{18}Ὅταν γὰρ ἐκβάλω τὰ ἔθνη ἀπὸ προσώπου σου, καὶ ἐμπλατύνω τὰ ὅριά σου, οὐ θύσεις ἐπὶ ζύμῃ αἷμα θυμιάματός μου, οὐδὲ μὴ κοιμηθῇ στέαρ τῆς ἑορτῆς μου ἕως πρωΐ.
\vs{19}Τὰς ἀπαρχὰς τῶν πρωτογενημάτων τῆς γνς σου εἰσοίσεις εἰς τὸν οἶκον Κυρίου τοῦ Θεοῦ σου· οὐχ ἑψήσεις ἄρνα ἐν γάλακτι μητρὸς αὐτοῦ.
\vs{20}Καὶ ἰδοὺ ἐγὼ ἀποστέλλω τὸν ἄγγελόν μου πρὸ προσώπου σου, ἵνα φυλάξῃ σε ἐν τῇ ὁδῷ, ὅπως εἰσαγάγῃ σε εἰς τὴν γῆν, ἣν ἡτοίμασά σοι.
\vs{21}Πρόσεχε σεαυτῷ, καὶ εἰσάκουε αὐτοῦ, καὶ μὴ ἀπείθει αὐτῷ, οὐ γὰρ μὴ ὑποστείληταί σε· τὸ γὰρ ὄνομά μου ἐστὶν ἐπʼ αὐτῷ.
\vs{22}Ἐὰν ἀκοῇ ἀκούσητε τῆς ἐμῆς φωνῆς, καὶ ποιήσῃς πάντα ὅσα ἂν ἐντείλωμαί σοι, καὶ φυλάξητε τὴν διαθήκην μου, ἔσεσθέ μοι λαὸς περιούσιος ἀπὸ πάντων τῶν ἐθνῶν· ἐμὴ γάρ ἐστι πᾶσα ἡ γῆ· ὑμεῖς δὲ ἔσεσθέ μοι βασίλειον ἱεράτευμα, καὶ ἔθνος ἅγιον· ταῦτα τὰ ῥήματα ἐρεῖς τοῖς υἱοῖς Ἰσραὴλ, ἐὰν ἀκοῇ ἀκούσητε τῆς φωνῆς μου, καὶ ποιήσητε πάντα ὅσα ἂν εἴπω σοι, ἐχθρεύσω τοῖς ἐχθροῖς σου, καὶ ἀντικείσομαι τοῖς ἀντικειμένοις σοι.
\vs{23}Πορεύσεται γὰρ ὁ ἄγγελός μου ἡγούμενός σου, καὶ εἰσάξει σε πρὸς τὸν Ἀμοῤῥαῖον, καὶ Χετταῖον, καὶ Φερεζαῖον, καὶ Χαναναῖον, καὶ Γεργεσαῖον, καὶ Εὑαῖον, καὶ Ἰεβουσαῖον, καὶ ἐκτρίψω αὐτούς.
\vs{24}Οὐ προσκυνήσεις τοῖς θεοῖς αὐτῶν, οὐδὲ μὴ λατρεύσῃς αὐτοῖς· οὐ ποιήσεις κατὰ τὰ ἔργα αὐτῶν· ἀλλὰ καθαιρέσει καθελεῖς, καὶ συντρίβων συντρίψεις τὰς στήλας αὐτῶν.
\vs{25}Καὶ λατρεύσεις Κυρίῳ τῷ Θεῷ σου· καὶ εὐλογήσω τὸν ἄρτον σου καὶ τὸν οἶνόν σου καὶ τὸ ὕδωρ σου, καὶ ἀποστρέψω μαλακίαν ἀφʼ ὑμῶν.
\vs{26}Οὐκ ἔσται ἄγονος, οὐδὲ στεῖρα ἐπὶ τῆς γῆς σου· τὸν ἀριθμὸν τῶν ἡμερῶν σου ἀναπληρῶν ἀναπληρώσω.
\vs{27}Καὶ τὸν φόβον ἀποστελῶ ἡγούμενόν σου, καὶ ἐκστήσω πάντα τὰ ἔθνη, εἰς οὓς σὺ εἰσπορεύῃ εἰς αὐτούς· καὶ δώσω πάντας τοὺς ὑπεναντίους σου φυγάδας.
\vs{28}Καὶ ἀποστελῶ τὰς σφηκίας προτέρας σου· καὶ ἐκβαλεῖς τοὺς Ἀμοῤῥαίους, καὶ τοὺς Εὑαίους, καὶ τοὺς Χαναναίους, καὶ τοὺς Χετταίους ἀπὸ σοῦ.
\vs{29}Οὐκ ἐκβαλῶ αὐτοὺς ἐν ἐνιαυτῷ ἑνὶ, ἵνα μὴ γένηται ἡ γῆ ἔρημος, καὶ πολλὰ γένηται ἐπὶ σὲ τὰ θηρία τῆς γῆς.
\vs{30}Κατὰ μικρὸν ἐκβαλῶ αὐτοὺς ἀπὸ σοῦ, ἕως ἂν αὐξηθῇς καὶ κληρονομήσῃς τὴν γῆν.
\vs{31}Καὶ θήσω τὰ ὅριά σου ἀπὸ τῆς ἐρυθρᾶς θαλάσσης, ἕως τῆς θαλάσσης τῆς Φυλιστιείμ· καὶ ἀπὸ τῆς ἐρήμου, ἕως τοῦ μεγάλου ποταμοῦ Εὐφράτου· καὶ παραδώσω εἰς τὰς χεῖρας ὑμῶν τοὺς ἐγκαθημένους ἐν τῇ γῇ, καὶ ἐκβαλῶ αὐτοὺς ἀπὸ σοῦ.
\vs{32}Οὐ συγκαταθήσῃ αὐτοῖς καὶ τοῖς θεοῖς αὐτῶν διαθήκην.
\vs{33}Καὶ οὐκ ἐνκαθήσονται ἐν τῇ γῇ σου, ἵνα μὴ ἁμαρτεῖν σε ποιήσωσι πρὸς μέ· ἐὰν γὰρ δουλεύσῃς τοῖς θεοῖς αὐτῶν, οὗτοι ἔσονταί σοι πρόσκομμα.

\ch{24}
Καὶ Μωυσῇ εἶπεν, ἀνάβηθι πρὸς τὸν Κύριον σὺ καὶ Ἀαρὼν, καὶ Ναδὰβ, καὶ Ἀβιοὺδ, καὶ ἑβδομήκοντα τῶν πρεσβυτέρων Ἰσραήλ· καὶ προσκυνήσουσι μακρόθεν τῷ Κυρίῳ.
\vs{2}Καὶ ἐγγιεῖ Μωσῆς μόνος πρὸς τὸν Θεὸν, αὐτοὶ δὲ οὐκ ἐγγιοῦσιν, ὁ δὲ λαὸς οὐ συναναβήσεται μετʼ αὐτῶν.
\vs{3}Εἰσῆλθε δὲ Μωυσῆς, καὶ διηγήσατο τῷ λαῷ πάντα τὰ ῥήματα τοῦ Θεοῦ καὶ τὰ δικαιώματα· ἀπεκρίθη δὲ πᾶς ὁ λαὸς φωνῇ μιᾷ, λέγοντες, πάντας τοὺς λόγους, οὓς ἐλάλησε Κύριος, ποιήσομεν, καὶ ἀκουσόμεθα.
\vs{4}Καὶ ἔγραψε Μωυσῆς πάντα τὰ ῥήματα Κυρίου· ὀρθρίσας δὲ Μωυσῆς τὸ πρωῒ ᾠκοδόμησε θυσιαστήριον ὑπὸ τὸ ὄρος, καὶ δώδεκα λίθους εἰς τὰς δώδεκα φυλὰς τοῦ Ἰσραήλ.
\vs{5}Καὶ ἐξαπέστειλε τοὺς νεανίσκους τῶν υἱῶν Ἰσραήλ, καὶ ἀνήνεγκαν ὁλοκαυτώματα· καὶ ἔθυσαν θυσίαν σωτηρίου τῷ Θεῷ μοσχάρια.
\vs{6}Λαβὼν δὲ Μωυσῆς τὸ ἥμισυ τοῦ αἵματος, ἐνέχεεν εἰς κρατῆρας, τὸ δὲ ἥμισυ τοῦ αἵματος προσέχεε πρὸς τὸ θυσιαστήριον.
\vs{7}Καὶ λαβὼν τὸ βιβλίον τῆς διαθήκης, ἀνέγνω εἰς τὰ ὦτα τοῦ λαοῦ· καὶ εἶπαν, πάντα ὅσα ἐλάλησε Κύριος, ποιήσομεν καὶ ἀκουσόμεθα.
\vs{8}Λαβὼν δὲ Μωυσῆς τὸ αἷμα, κατεσκέδασε τοῦ λαοῦ, καὶ εἶπεν, ἰδοὺ τὸ αἷμα τῆς διαθήκης, ἧς διέθετο Κύριος πρὸς ὑμᾶς περὶ πάντων τῶν λόγων τούτων.

\vs{9}Καὶ ἀνέβη Μωυσῆς καὶ Ἀαρὼν, καὶ Ναδὰβ, καὶ Ἀβιοῦδ, καὶ ἑβδομήκοντα τῆς γερουσίας Ἰσραήλ.
\vs{10}Καὶ εἶδον τὸν τόπον οὗ εἱστήκει ὁ Θεὸς τοῦ Ἰσραήλ· καὶ τὰ ὑπὸ τοὺς πόδας αὐτοῦ, ὡσεὶ ἔργον πλίνθου σαπφείρου, καὶ ὥσπερ εἶδος στερεώματος τοῦ οὐρανοῦ τῇ καθαριότητι.
\vs{11}Καὶ τῶν ἐπιλέκτων τοῦ Ἰσραὴλ οὐ διεφώνησεν οὐδὲ εἷς· καὶ ὤφθησαν ἐν τῷ τόπῳ τοῦ Θεοῦ, καὶ ἔφαγον καὶ ἔπιον.
\vs{12}Καὶ εἶπε Κύριος πρὸς Μωυσῆν, ἀνάβηθι πρὸς με εἰς τὸ ὄρος, καὶ ἴσθι ἐκεῖ· καὶ δώσω σοι τὰ πυξία τὰ λίθινα, τὸν νόμον καὶ τὰς ἐντολας, ἃς ἔγραψα νομοθετῆσαι αὐτοῖς.
\vs{13}Καὶ ἀναστὰς Μωυσῆς καὶ Ἰησοῦς ὁ παρεστηκὼς αὐτῷ, ἀνέβησαν εἰς τὸ ὄρος τοῦ Θεοῦ.
\vs{14}Καὶ τοῖς πρεσβυτέροις εἶπαν, ἡσυχάζετε αὐτοῦ, ἕως ἀναστρέψωμεν πρὸς ὑμᾶς· καὶ ἰδοὺ Ἀαρὼν καὶ Ὢρ μεθʼ ὑμῶν· ἐάν τινι συμβῇ κρίσις, προσπορευέσθωσαν αὐτοῖς.
\vs{15}Καὶ ἀνέβη Μωυσῆς καὶ Ἰησοῦς εἰς τὸ ὄρος· καὶ ἐκάλυψεν ἡ νεφέλη τὸ ὄρος.
\vs{16}Καὶ κατέβη ἡ δόξα τοῦ Θεοῦ ἐπὶ τὸ ὄρος τὸ Σινὰ, καὶ ἐκάλυψεν αὐτὸ ἡ νεφέλη ἓξ ἡμέρας· καὶ ἐκάλεσε Κύριος τὸν Μωυσῆν τῇ ἡμέρᾳ τῇ ἑβδόμῃ ἐκ μέσου τῆς νεφέλης.
\vs{17}Τὸ δὲ εἶδος τῆς δόξης Κυρίου, ὡσεὶ πῦρ φλέγον ἐπὶ τῆς κορυφῆς τοῦ ὄρους, ἐναντίον τῶν υἱῶν Ἰσραήλ.
\vs{18}Καὶ εἰσῆλθε Μωυσῆς εἰς τὸ μέσον τῆς νεφέλης, καὶ ἀνέβη εἰς τὸ ὄρος· καὶ ἦν ἐκεῖ ἐν τῷ ὄρει τεσσεράκοντα ἡμέρας καὶ τεσσαράκοντα νύκτας.

\ch{25}
Καὶ ἐλάλησε Κύριος πρὸς Μωυσῆν, λέγων,
\vs{2}εἶπον τοῖς υἱοῖς Ἰσραὴλ, καὶ λάβετε ἀπαρχὰς παρὰ πάντων, οἷς ἂν δόξῃ τῇ καρδίᾳ, καὶ λήψεσθε τὰς ἀπαρχάς μου.
\vs{3}Καὶ αὕτη ἐστὶν ἡ ἀπαρχὴ, ἣν λήψεσθε παρʼ αὐτῶν· χρυσίον, καὶ ἀργύριον, καὶ χαλκὸν,
\vs{4}καὶ ὑάκινθον, καὶ πορφύραν, καὶ κόκκινον διπλοῦν, καὶ βύσσον κεκλωσμένην, καὶ τρίχας αἰγείας,
\vs{5}καὶ δέρματα κριῶν ἠρυθροδανωμένα, καὶ δέρματα ὑακίνθινα, καὶ ξύλα ἄσηπτα,
\vs{5a}καὶ ἔλαιον εἰς τὴν φαῦσιν, θυμιάματα εἰς τὸ ἔλαιον τῆς χρίσεως, καὶ εἰς τὴν σύνθεσιν τοῦ θυμιάματος,
\vs{7}καὶ λίθους Σαρδίου, καὶ λίθους εἰς τὴν γλυφὴν εἰς τὴν ἐπωμίδα, καὶ τὸν ποδήρη.
\vs{8}Καὶ ποιήεις μοι ἁγίασμα, καὶ ὀφθήσομαι ἐν ὑμῖν.
\vs{9}Καὶ ποιήσεις μοι κατὰ πάντα ὅσα σοι δεικνύω ἐν τῷ ὄρει, τὸ παράδειγμα τῆς σκηνῆς, καὶ τὸ παράδειγμα πάντων τῶν σκευῶν αὐτῆς· οὕτω ποιήσεις.
\vs{10}Καὶ ποιήσεις κιβωτὸν μαρτυρίου ἐκ ξύλων ἀσήπτων, δύο πήχεων καὶ ἡμίσους τὸ μῆκος, καὶ πήχεος καὶ ἡμίσους τὸ πλάτος, καὶ πήχεως καὶ ἡμίσους τὸ ὕψος.
\vs{11}Καὶ καταχρυσώσεις αὐτὴν χρυσίῳ καθαρῷ, ἔσωθεν καὶ ἔξωθεν χρυσώσεις αὐτήν· καὶ ποιήσεις αὐτῇ κυμάτια χρυσᾶ στρεπτὰ κύκλῳ.
\vs{12}Καὶ ἐλάσεις αὐτῇ τέσσαρας δακτυλίους χρυσοῦς, καὶ ἐπιθήσεις ἐπὶ τὰ τέσσαρα κλίτη· δύο δακτυλίους ἐπὶ τὸ κλίτος τὸ ἓν, καὶ δύο δακτυλίους ἐπὶ τὸ κλίτος τὸ δεύτερον.
\vs{13}Ποιήσεις δὲ ἀναφορεῖς ξύλα ἄσηπτα, καὶ καταχρυσώσεις αὐτὰ χρυσίῳ·
\vs{14}Καὶ εἰσάξεις τοὺς ἀναφορεῖς εἰς τοὺς δακτυλίους τοὺς ἐν τοῖς κλίτεσι τῆς κιβωτοῦ, αἴρειν τὴν κιβωτὸν ἐν αὐτοῖς.
\vs{15}Ἐν τοῖς δακτυλίοις τῆς κιβωτοῦ ἔσονται οἱ ἀναφορεῖς ἀκίνητοι.
\vs{16}Καὶ ἐμβαλεῖς εἰς τὴν κιβωτὸν τὰ μαρτύρια, ἃ ἂν δῶ σοι.
\vs{17}Καὶ ποιήσεις ἱλαστήριον ἐπίθεμα χρυσίου καθαροῦ, δύο πήχεων καὶ ἡμίσους τὸ μῆκος, καὶ πήχεως καὶ ἡμίσους τὸ πλάτος.
\vs{18}Καὶ ποιήσεις δύο χερουβὶμ χρυσοτορευτὰ, καὶ ἐπιθήσεις αὐτὰ ἐξ ἀμφοτέρων τῶν κλιτῶν τοῦ ἱλαστηρίου.
\vs{19}Ποιηθήσονται χεροὺβ εἷς ἐκ τοῦ κλίτους τούτου, καὶ χεροὺβ εἷς ἐκ τοῦ κλίτους τοῦ δευτέρου τοῦ ἱλαστηρίου· καὶ ποιήσεις τοὺς δύο χερουβὶμ ἐπὶ τὰ δύο κλίτη.
\vs{20}Ἔσονται οἱ χερουβὶμ ἐκτείνοντες τὰς πτέρυγας ἐπάνωθεν, συσκιάζοντες ἐν ταῖς πτέρυξιν αὐτῶν ἐπὶ τοῦ ἱλαστηρίου, καὶ τὰ πρόσωπα αὐτῶν εἰς ἄλληλα, εἰς τὸ ἱλαστήριον ἔσονται τὰ πρόσωπα τῶν χερουβίμ.
\vs{21}Καὶ ἐπιθήσεις τὸ ἱλαστήριον ἐπὶ τὴν κιβωτὸν ἄνωθεν, καὶ εἰς τὴν κιβωτὸν ἐμβαλεῖς τὰ μαρτύρια, ἃ ἂν δῶ σοι.
\vs{22}Καὶ γνωσθήσομαί σοι ἐκεῖθεν, καὶ λαλήσω σοι ἄνωθεν τοῦ ἱλαστηρίου ἀνὰ μέσον τῶν δύο χερουβὶμ, τῶν ὄντων ἐπὶ τῆς κιβωτοῦ τοῦ μαρτυρίου, καὶ κατὰ πάντα ὅσα ἐὰν ἐντείλωμαί σοι πρὸς τοὺς υἱοὺς Ἰσραήλ.
\vs{23}Καὶ ποιήσεις τράπεζαν χρυσῆν χρυσίου καθαροῦ, δύο πήχεων τὸ μῆκος, καὶ πήχεως τὸ εὖρος, καὶ πήχεως καὶ ἡμίσους τὸ ὕψος.
\vs{24}Καὶ ποιήσεις αὐτῇ στρεπτὰ κυμάτια χρυσᾶ κύκλῳ· καὶ ποιήσεις αὐτῇ στεφάνην παλαιστοῦ κύκλῳ·

\vs{25}Καὶ ποιήσεις στρεπτὸν κυμάτιον τῇ στεφάνῃ κύκλῳ.
\vs{26}Καὶ ποιήσεις τέσσαρας δακτυλίους χρυσοῦς, καὶ ἐπιθήσεις τοὺς τέσσαρας δακτυλίους ἐπὶ τὰ τέσσαρα μέρη τῶν ποδῶν αὐτῆς ὑπὸ τὴν στεφάνην.
\vs{27}Καὶ ἔσονται οἱ δακτύλιοι εἰς θήκας τοῖς ἀναφορεῦσιν, ὥστε αἴρειν ἐν αὐτοῖς τὴν τράπεζαν.
\vs{28}Καὶ ποιήσεις τοὺς ἀναφορεῖς ἐκ ξύλων ἀσήπτων, καὶ καταχρυσώσεις αὐτοὺς χρυσίῳ καθαρῷ, καὶ ἀρθήσεται ἐν αὐτοῖς ἡ τράπεζα.
\vs{29}Καὶ ποιήσεις τὰ τρυβλία αὐτῆς, καὶ τὰς θυΐσκας, καὶ τὰ σπονδεῖα, καὶ τοὺς κυάθους, ἐν οἷς σπείσεις ἐν αὐτοῖς, ἐκ χρυσίου καθαροῦ ποιήσεις αὐτά.
\vs{30}Καὶ ἐπιθήσεις ἐπὶ τὴν τράπεζαν ἄρτους ἐνωπίους ἐναντίον μου διαπαντός.

\vs{31}Καὶ ποιήσεις λυχνίαν ἐκ χρυσίου καθαροῦ, τορευτὴν ποιήσεις τὴν λυχνίαν· ὁ καυλὸς αὐτῆς, καὶ ὁ καλαμίσκοι, καὶ οἱ κρατῆρες, καὶ οἱ σφαιρωτῆρες, καὶ τὰ κρίνα ἐξ αὐτῆς ἔσται.
\vs{32}Ἓξ δὲ καλαμίσκοι ἐκπορευόμενοι ἐκ πλαγίων, τρεῖς καλαμίσκοι τῆς λυχνίας ἐκ τοῦ κλίτους τοῦ ἑνὸς αὐτῆς, καὶ τρεῖς καλαμίσκοι τῆς λυχνίας ἐκ τοῦ κλίτους τοῦ δευτέρου.
\vs{33}Καὶ τρεῖς κρατῆρες ἐκτετυπωμένοι καρυΐσκους· ἐν τῷ ἑνὶ καλαμίσκῳ σφαιρωτὴρ καὶ κρίνον· οὕτω τοῖς ἓξ καλαμίσκοις τοῖς ἐκπορευομένοις ἐκ τῆς λυχνίας.
\vs{34}Καὶ ἐν τῇ λυχνίᾳ τέσσαρες κρατῆρες ἐκτετυπωμένοι καρυΐσκους· ἐν τῷ ἑνὶ καλαμίσκῳ σφαιρωτῆρες, καὶ τὰ κρίνα αὐτῆς.
\vs{35}Ὁ σφαιρωτὴρ ὑπὸ τοὺς δύο καλαμίσκους ἐξ αὐτῆς· καὶ σφαιρωτὴρ ὑπὸ τοὺς τέσσαρας καλαμίσκους ἐξ αὐτῆς· οὕτω τοῖς ἓξ καλαμίσκοις τοῖς ἐκπορευομένοις ἐκ τῆς λυχνίας· καὶ ἐν τῇ λυχνίᾳ τέσσαρες κρατῆρες ἐκτετυπωμένοι καρυΐσκους.
\vs{36}Οἱ σφαιρωτῆρες καὶ οἱ καλαμίσκοι ἐξ αὐτῆς ἔστωσαν· ὅλη τορευτὴ ἐξ ἑνὸς χρυσίου καθαροῦ.
\vs{37}Καὶ ποιήσεις τοὺς λύχνους αὐτῆς ἑπτά· καὶ ἐπιθήσεις τοὺς λύχνους, καὶ φανοῦσιν ἐκ τοῦ ἑνὸς προσώπου.
\vs{38}Καὶ τὸν ἐπαρυστῆρα αὐτῆς, καὶ τὰ ὑποθέματα αὐτῆς ἐκ χρυσίου καθαροῦ ποιήσεις.
\vs{39}Πάντα τὰ σκεύη ταῦτα τάλαντον χρυσίου καθαροῦ.
\vs{40}Ὅρα, ποιήσεις κατὰ τὸν τύπον τὸν δεδειγμένον σοι ἐν τῷ ὄρει.

\ch{26}
Καὶ τὴν σκηνὴν ποιήσεις, δέκα αὐλαίας ἐκ βύσσου κεκλωσμένης, καὶ ὑακίνθου, καὶ πορφύρας, καὶ κοκκίνου κεκλωσμένου χερουβὶμ· ἐργασίᾳ ὑφάντου ποιήσεις αὐτάς.
\vs{2}Μῆκος τῆς αὐλαίας τῆς μιᾶς ὀκτὼ καὶ εἴκοσι πήχεων, καὶ εὖρος τεσσάρων πήχεων ἡ αὐλαία ἡ μία ἔσται· μέτρον τὸ αὐτὸ ἔσται πάσαις ταῖς αὐλαίαις.
\vs{3}Πέντε δὲ αὐλαῖαι ἔσονται ἐξ ἀλλήλων ἐχόμεναι ἡ ἑτέρα ἐκ τῆς ἑτέρας· καὶ πέντε αὐλαῖαι ἔσονται συνεχόμεναι ἑτέρα τῇ ἑτέρᾳ.
\vs{4}Καὶ ποιήσεις αὐταῖς ἀγκύλας ὑακινθίνας ἐπὶ τοῦ χείλους τῆς αὐλαίας τῆς μιᾶς, ἐκ τοῦ ἑνὸς μέρους εἰς τὴν συμβολήν· καὶ οὕτω ποιήσεις ἐπὶ τοῦ χείλους τῆς αὐλαίας τῆς ἐξωτέρας πρὸς τῇ συμβολῇ τῇ δευτέρᾳ.
\vs{5}Πεντήκοντα ἀγκύλας ποιήσεις τῇ αὐλαίᾳ τῇ μιᾷ, καὶ πεντήκοντα ἀγκύλας ποιήσεις ἐκ τοῦ μέρους τῆς αὐλαίας κατὰ τὴν συμβολὴν τῆς δευτέρας, ἀντιπρόσωποι ἀντιπίπτουσαι ἀλλήλαις εἰς ἑκάστην.
\vs{6}Καὶ ποιήσεις κρίκους πεντήκοντα χρυσοῦς· καὶ συνάψεις τὰς αὐλαίας ἑτέραν τῇ ἑτέρα τοῖς κρίκοις· καὶ ἔσται ἡ σκηνὴ μία.
\vs{7}Καὶ ποιήσεις δέῤῥεις τριχίνας σκέπην ἐπὶ τῆς σκηνῆς, ἕνδεκα δέῤῥεις ποιήσεις αὐτάς.
\vs{8}Τὸ μῆκος τῆς δέῤῥεως τῆς μιᾶς, τριάκοντα πήχεων, καὶ τεσσάρων πήχεων τὸ εὖρος τῆς δέῤῥεως τῆς μιᾶς· τὸ αὐτὸ μέτρον ἔσται ταῖς ἕνδεκα δέῤῥεσι.
\vs{9}Καὶ συνάψεις τὰς πέντε δέῤῥεις ἐπὶ τὸ αὐτὸ, καὶ τὰς ἓξ δέῤῥεις ἐπὶ τὸ αὐτό· καὶ ἐπιδιπλώσεις τὴν δέῤῥιν τὴν ἕκτην κατὰ πρόσωπον τῆς σκηνῆς.
\vs{10}Καὶ ποιήσεις ἀγκύλας πεντήκοντα ἐπὶ τοῦ χείλους τῆς δέῤῥεως τῆς μιᾶς, τῆς ἀναμέσον κατὰ συμβολήν· καὶ πεντήκοντα ἀγκύλας ποιήσεις ἐπὶ τοῦ χείλους τῆς δέῤῥεως, τῆς συναπτούσης τῆς δευτέρας.

\vs{11}Καὶ ποιήσεις κρίκους χαλκοῦς πεντήκοντα· καὶ συνάψεις τοὺς κρίκους ἐκ τῶν ἀγκυλῶν, καὶ συνάψεις τὰς δέῤῥεις, καὶ ἔσται ἕν.
\vs{12}Καὶ ὑποθήσεις τὸ πλεονάζον ἐν ταῖς δέῤῥεσι τῆς σκηνῆς· τὸ ἥμισυ τῆς δέῤῥεως τὸ ὑπολελειμμένον ὑποκαλύψεις εἰς τὸ πλεονάζον τῶν δέῤῥεων τῆς σκηνῆς, ὑποκαλύψεις ὀπίσω τῆς σκηνῆς.
\vs{13}Πῆχυν ἐκ τούτου, καὶ πῆχυν ἐκ τούτου, ἐκ τοῦ ὑπερέχοντος τῶν δέῤῥεων, ἐκ τοῦ μήκους τῶν δέῤῥεων τῆς σκηνῆς· ἔσται συγκαλύπτον ἐπὶ τὰ πλάγια τῆς σκηνῆς ἔνθεν καὶ ἔνθεν, ἵνα καλύπτῃ.
\vs{14}Καὶ ποιήσεις κατακάλυμμα τῇ σκηνῇ δέρματα κριῶν ἠρυθροδανωμένα, καὶ ἐπικαλύμματα δέρματα ὑακίνθινα ἐπάνωθεν.

\vs{15}Καὶ ποιήσεις στύλους τῆς σκηνῆς ἐκ ξύλων ἀσήπτων.
\vs{16}Δέκα πήχεων ποιήσεις τὸν στύλον τὸν ἕνα, καὶ πήχεως ἑνὸς καὶ ἡμίσους τὸ πλάτος τοῦ στύλου τοῦ ἑνός.
\vs{17}Δύο ἀγκωνίσκους τῷ στύλῳ τῷ ἑνὶ, ἀντιπίπτοντας ἕτερον τῷ ἑτέρῳ· οὕτω ποιήσεις πᾶσι τοῖς στύλοις τῆς σκηνῆς.
\vs{18}Καὶ ποιήσεις στύλους τῇ σκηνῇ, εἴκοσι στύλους ἐκ τοῦ κλίτους τοῦ πρὸς Βοῤῥᾶν.
\vs{19}Καὶ τεσσαράκοντα βάσεις ἀργυρᾶς ποιήσεις τοῖς εἴκοσι στύλοις· δύο βάσεις τῷ στύλῳ τῷ ἑνὶ εἰς ἀμφότερα τὰ μέρη αὐτοῦ· και δύο βάσεις τῷ στύλῳ τῷ ἑνὶ εἰς ἀμφοτέρα τὰ μέρη αὐτοῦ.
\vs{20}Καὶ τὸ κλίτος τὸ δεύτερον τὸ πρὸς Νότον, εἴκοσι στύλους,
\vs{21}καὶ τεσσαράκοντα βάσεις αὐτῶν ἀργυρᾶς· δύο βάσεις τῷ στύλῳ τῷ ἑνὶ εἰς ἀμφότερα τὰ μέρη αὐτοῦ, καὶ δύο βάσεις τῷ στύλῳ τῷ ἑνὶ εἰς ἀμφότερα τὰ μέρη αὐτοῦ.
\vs{22}Καὶ ἐκ τῶν ὀπίσω τῆς σκηνῆς κατὰ τὸ μέρος τὸ πρὸς θάλασσαν ποιήσεις ἓξ στύλους.
\vs{23}Καὶ δύο στύλους ποιήσεις ἐπὶ τῶν γωνιῶν τῆς σκηνῆς ἐκ τῶν ὀπισθίων.
\vs{24}Καὶ ἔσται ἐξ ἴσου κάτωθεν· κατὰ τὸ αὐτὸ ἔσονται ἴσοι ἐκ τῶν κεφαλῶν εἰς σύμβλησιν μίαν· οὕτω ποιήσεῖς ἀμφοτέραις ταῖς δυσὶ γωνίαις· ἴσαι ἔστωσαν.
\vs{25}Καὶ ἔσονται ὀκτὼ στύλοι, καὶ αἱ βάσεις αὐτῶν ἀργυραῖ δεκαέξ· δύο βάσεις τῷ ἑνὶ στύλῳ εἰς ἀμφότερα τὰ μέρη αὐτοῦ, καὶ δύο βάσεις τῷ στύλῳ τῷ ἑνί.
\vs{26}Καὶ ποιήσεις μοχλοὺς ἐκ ξύλων ἀσήπτων· πέντε τῷ ἑνὶ στύλῳ ἐκ τοῦ ἑνὸς μέρους τῆς σκηνῆς,
\vs{27}καὶ πέντε μοχλοὺς τῷ στύλῳ τῷ ἑνὶ κλίτει τῆς σκηνῆς τῷ δευτέρῳ, καὶ πέντε μοχλοὺς τῷ στύλῳ τῷ ὀπισθίῳ τῷ κλίτει τῆς σκηνῆς τῷ πρὸς θάλασσαν.
\vs{28}Καὶ ὁ μοχλὸς ὁ μέσος ἀναμέσον τῶν στύλων διϊκνείσθω ἀπὸ τοῦ ἑνὸς κλίτους εἰς τὸ ἕτερον κλίτος.
\vs{29}Καὶ τοὺς στύλους καταχρυσώσεις χρυσίῳ· καὶ τοὺς δακτυλίους ποιήσεις χρυσοῦς, εἰς οὓς εἰσάξεις τούς μοχλούς· καὶ καταχρυσώσεις τοὺς μοχλοὺς χρυσίῳ.
\vs{30}Καὶ ἀναστήσεις τὴν σκηνὴν κατὰ τὸ εἶδος τὸ δεδειγμένον σοι ἐν τῷ ὄρει.

\vs{31}Καὶ ποιήσεις καταπέτασμα ἐξ ὑακίνθου, καὶ πορφύρας, καὶ κοκκίνου κεκλωσμένου, καὶ βύσσου νενησμένης· ἔργον ὑφαντὸν ποιήσεις αὐτὸ χερουβίμ.
\vs{32}Καὶ ἐπιθήσεις αὐτὸ ἐπὶ τεσσάρων στύλων ἀσήπτων κεχρυσωμένων χρυσίῳ· καὶ αἱ κεφαλίδες αὐτῶν χρυσαῖ, καὶ αἱ βάσεις αὐτῶν τέσσαρες ἀργυραῖ.
\vs{33}Καὶ θήσεις τὸ καταπέτασμα ἐπὶ τῶν στύλων· καὶ εἰσοίσεις ἐκεῖ ἐσώτερον τοῦ καταπετάσματος τὴν κιβωτὸν τοῦ μαρτυρίου· καὶ διοριεῖ τὸ καταπέτασμα ὑμῖν ἀναμέσον τοῦ ἁγίου καὶ ἀναμέσον τοῦ ἁγίου τῶν ἁγίων.
\vs{34}Καὶ κατακαλύψεις τῷ καταπετάσματι τὴν κιβωτὸν τοῦ μαρτυρίου ἐν τῷ ἁγίῳ τῶν ἁγίων.
\vs{35}Καὶ ἐπιθήσεις τὴν τράπεζαν ἔξωθεν τοῦ καταπετάσματος, καὶ τὴν λυχνίαν ἀπέναντι τῆς τραπέζης ἐπὶ μέρους τῆς σκηνῆς τὸ πρὸς Νότον· καὶ τὴν τράπεζαν θήσεις ἐπὶ μέρους τῆς σκηνῆς τὸ πρὸς Βοῤῥᾶν.
\vs{36}Καὶ ποιήσεις ἐπίσπαστρον τῇ θύρᾳ τῆς σκηνῆς ἐξ ὑακίνθου, καὶ πορφύρας, καὶ κοκκίνου κεκλωσμένου, καὶ βύσσου κεκλωσμένης, ἔργον ποικιλτοῦ.
\vs{37}Καὶ ποιήσεις τῷ καταπετάσματι πέντε στύλους, καὶ χρυσώσεις αὐτοὺς χρυσίῳ· καὶ αἱ κεφαλίδες αὐτῶν χρυσαῖ· καὶ χωνεύσεις αὐτοῖς πέντε βάσεις χαλκᾶς.

\ch{27}
Καὶ ποιήσεις θυσιαστήριον ἐκ ξύλων ἀσήπτων, πέντε πήχεων τὸ μῆκος, καὶ πέντε πήχεων τὸ εὖρος· τετράγωνον ἔσται τὸ θυσιαστήριον, καὶ τριῶν πήχεων τὸ ὕψος αὐτοῦ.
\vs{2}Καὶ ποιήσεις τὰ κέρατα ἐπὶ τῶν τεσσάρων γωνιῶν· ἐξ αὐτοῦ ἔσται τὰ κέρατα, καὶ καλύψεις αὐτὰ χαλκῷ.
\vs{3}Καὶ ποιήσεις στεφάνην τῷ θυσιαστηρίῳ· καὶ τὸν καλυπτῆρα αὐτοῦ, καὶ τὰς φιάλας αὐτοῦ, καὶ τὰς κρεάγρας αὐτοῦ, καὶ τὸ πυρεῖον αὐτοῦ, καὶ πάντα τὰ σκεύη αὐτοῦ ποιήσεις χαλκᾶ.
\vs{4}Καὶ ποιήσεις αὐτῷ ἐσχάραν ἔργῳ δικτυωτῷ χαλκῆν· καὶ ποιήσεις τῇ ἐσχάρᾳ τέσσαρες δακτυλίους χαλκοῦς ὑπὸ τὰ τέσσαρα κλίτη.
\vs{5}Καὶ ὑποθήσεις αὐτοὺς ὑπὸ τὴν ἐσχάραν τοῦ θυσιαστήριου κάτωθεν· ἔσται δὲ ἡ ἐσχάρα ἕως τοῦ ἡμίσους τοῦ θυσιαστηρίου.
\vs{6}Καὶ ποιήσεις τῷ θυσιαστηρίῳ ἀναφορεῖς ἐκ ξύλων ἀσήπτων, καὶ περιχαλκώσεις αὐτοὺς χαλκῷ.
\vs{7}Καὶ εἰσάξεις τοὺς ἀναφορεῖς εἰς τοὺς δακτυλίους· καὶ ἔστωσαν ἀναφορεῖς κατὰ πλευρὰ τοῦ θυσιαστηρίου ἐν τῷ αἴρειν αὐτό.
\vs{8}Κοῖλον συνιδωτὸν ποιήσεις αὐτό· κατὰ τὸ παραδειχθέν σοι ἐν τῷ ὄρει, οὕτω ποιήσεις αὐτό.
\vs{9}Καὶ ποιήσεις αὐλὴν τῇ σκηνῇ· εἰς τὸ κλίτος τὸ πρὸς Λίβα ἱστία τῆς αὐλῆς ἐκ βύσσου κεκλωσμένης· μῆκος ἑκατὸν πήχεων τῷ ἑνὶ κλίτει.
\vs{10}Καὶ οἱ στύλοι αὐτῶν εἴκοσι, καὶ αἱ βάσεις αὐτῶν εἴκοσι χαλκαῖ, καὶ οἱ κρίκοι αὐτῶν καὶ αἱ ψαλίδες ἀργυραῖ.
\vs{11}Οὕτως τῷ κλίτει τῷ πρὸς ἀπηλιώτην ἱστία ἑκατὸν πήχεων μῆκος· καὶ οἱ στύλοι αὐτῶν εἴκοσι, καὶ αἱ βάσεις αὐτῶν εἴκοσι χαλκαῖ· καὶ οἱ κρίκοι καὶ αἱ ψαλίδες τῶν στύλων, καὶ αἱ βάσεις αὐτῶν περιηργυρωμέναι ἀργυρίῳ.
\vs{12}Τὸ δὲ εὖρος τῆς αὐλῆς τὸ κατὰ θάλασσαν ἱστία πεντήκοντα πήχεων· στύλοι αὐτῶν δέκα, καὶ βάσεις αὐτῶν δέκα.
\vs{13}Καὶ εὖρος τῆς αὐλῆς τῆς πρὸς Νότον ἱστία πεντήκοντα πήχεων· στύλοι αὐτῶν δέκα, καὶ βάσεις αὐτῶν δέκα.
\vs{14}Καὶ πεντεκαίδεκα πήχεων τὸ ὕψος τῶν ἱστίων τῷ κλίτει τῷ ἑνί· στύλοι αὐτῶν τρεῖς, καὶ αἱ βάσεις αὐτῶν τρεῖς.
\vs{15}Καὶ τὸ κλίτος τὸ δεύτερον δεκαπέντε πήχεων τῶν ἱστίων τὸ ὕψος· στύλοι αὐτῶν τρεῖς, καὶ αἱ βάσεις αὐτῶν τρεῖς.
\vs{16}Καὶ τῇ πύλῃ τῆς αὐλῆς κάλυμμα· εἴκοσι πήχεων τὸ ὕψος ἐξ ὑακίνθου, καὶ πορφύρας, καὶ κοκκίνου κεκλωσμένου, καὶ βύσσου κεκλωσμένης τῇ ποικιλίᾳ τοῦ ῥαφιδευτοῦ· στύλοι αὐτῶν τέσσαρες, καὶ αἱ βάσεις αὐτῶν τέσσαρες.
\vs{17}Πάντες οἱ στύλοι τῆς αὐλῆς κύκλῳ κατηργυρωμένοι ἀργυρίῳ, καὶ αἱ κεφαλίδες αὐτῶν ἀργυραῖ, καὶ αἱ βάσεις αὐτῶν χαλκαῖ.
\vs{18}Τὸ δὲ μῆκος τῆς αὐλῆς ἑκατὸν ἐφʼ ἑκατόν· καὶ εὖρος πεντήκοντα ἐπὶ πεντήκοντα· καὶ ὕψος πέντε πήχεῶν ἐκ βύσσου κεκλωσμένης, καὶ βάσεις αὐτῶν χαλκαῖ.
\vs{19}Καὶ πᾶσα ἡ κατασκευὴ καὶ πάντα τὰ ἐργαλεῖα καὶ οἱ πάσσαλοι τῆς αὐλῆς χαλκοῖ.

\vs{20}Καὶ σὺ σύνταξον τοῖς υἱοῖς Ἰσραὴλ, καὶ λαβέτωσάν σοι ἔλαιον ἐξ ἐλαιῶν ἀτρυγον καθαρὸν κεκομμένον εἰς φῶς καῦσαι, ἵνα καίηται λύχνος διαπαντός
\vs{21}ἐν τῇ σκηνῇ τοῦ μαρτυρίου· ἔξωθεν τοῦ καταπετάσματος τοῦ ἐπὶ τῆς διαθήκης καύσει αὐτὸ Ἀαρὼν καὶ οἱ υἱοὶ αὐτοῦ ἀφʼ ἑσπέρας ἕως πρωῒ, ἐναντίον Κυρίου, νόμιμον αἰώνιον εἰς τὰς γενεὰς ὑμῶν παρὰ τῶν υἱῶν Ἰσραήλ.

\ch{28}
Καὶ σὺ προσαγάγου πρὸς σεαυτὸν τόν τε Ἀαρὼν τὸν ἀδελφόν σου, καὶ τοὺς υἱοὺς αὐτοῦ, καὶ ἐκ τῶν υἱῶν Ἰσραὴλ, ἱερατεύειν μοι Ἀαρὼν, καὶ Ναδὰβ, καὶ Ἀβιοὺδ, καὶ Ἐλεάζαρ, καὶ Ἰθάμαρ, υἱοὺς Ἀαρών.
\vs{2}Καὶ ποιήσεις στολὴν ἁγίαν Ἀαρὼν τῷ ἀδελφῷ σου εἰς τιμὴν καὶ δόξαν.
\vs{3}Καὶ σύ λάλησον πᾶσι τοῖς σοφοῖς τῇ διανοίᾳ, οὓς ἐνέπλησα πνεύματος σοφίας καὶ αἰσθήσεως· καὶ ποιήσουσι τὴν στολὴν τὴν ἁγίαν Ἀαρὼν εἰς τὸ ἅγιον, ἐν ᾗ ἱερατεύσει μοι.
\vs{4}Καὶ αὗται αἱ στολαὶ, ἃς ποιησουσι· τὸ περιστήθιον, καὶ τὴν ἐπωμίδα, καὶ τὸν ποδήρη, καὶ χιτῶνα κοσυμβωτὸν, καὶ κίδαριν, καὶ ζώνην· καὶ ποιήσουσι στολὰς ἁγίας Ἀαρὼν καὶ τοῖς υἱοῖς αὐτοῦ εἰς τὸ ἱερατεύειν μοι.
\vs{5}Καὶ αὐτοὶ λήψονται τὸ χρυσίον, καὶ τὸν ὑάκινθον, καὶ τὴν πορφύραν, καὶ τὸ κόκκινον, καὶ τὴν βύσσον.
\vs{6}Καὶ ποιήσουσι τὴν ἐπωμίδα ἐκ βύσσου κεκλωσμένης, ἔργον ὑφαντὸν ποικιλτοῦ.
\vs{7}Δύο ἐπωμίδες συνέχουσαι ἔσονται αὐτῷ ἑτέρα τὴν ἑτέραν, ἐπὶ τοῖς δυσὶ μέρεσιν ἐξηρτισμέναι.
\vs{8}Καὶ τὸ ὕφασμα τῶν ἐπωμίδων ὅ ἐστιν ἐπʼ αὐτῷ, κατὰ τὴν ποίησιν ἐξ αὐτοῦ ἔσται ἐκ χρυσίου καθαροῦ, καὶ ὑακίνθου, καὶ πορφύρας, καὶ κοκκίνου διανενησμένου, καὶ βύσσου κεκλωσμένης.
\vs{9}Καὶ λήψῃ τοὺς δύο λίθους, λίθους σμαράγδου, καὶ γλύψεις ἐν αὐτοῖς τὰ ὀνόματα τῶν υἱῶν Ἰσραήλ.
\vs{10}Ἓξ ὀνόματα ἐπὶ τὸν λίθον τὸν ἕνα, καὶ τὰ ἓξ ὀνόματα τὰ λοιπὰ ἐπὶ τὸν λίθον τὸν δεύτερον κατὰ τὰς γενέσεις αὐτῶν.
\vs{11}Ἔργον λιθουργικῆς τέχνης· γλύμμα σφραγίδος διαγλύψεις τοὺς δύο λίθους ἐπὶ τοῖς ὀνόμασι τῶς υἱῶν Ἰσρσήλ.
\vs{12}Καὶ θήσεις τοὺς δύο λίθους ἐπὶ τῶς ὤμων τῆς ἐπωμίδος· λίθοι μνημοσύνου εἰσὶ τοῖς υἱοῖς Ἰσραήλ· καὶ ἀναλήψεται Ἀαρὼν τὰ ὀνόματα τῶν υἱῶν Ἰσραὴλ ἔναντι Κυρίου ἐπὶ τῶν δύο ὤμων αὐτοῦ, μνημόσυνον πεπὶ αὐτῶν.
\vs{13}Καὶ ποιήσεις ἀσπιδίσκας ἐκ χρυσίου καθαροῦ.
\vs{14}Καὶ ποιήσεις δύο κροσωτὰ ἐκ χρυσίου καθαροῦ, καταμεμιγμένα ἐν ἄνθεσιν, ἔργον πλοκῆς· καὶ ἐπιθήσεις τὰ κροσσωτὰ τὰ πεπλεγμένα ἐπὶ τὰς ἀσπιδίσκας, κατὰ τὰς παρωμίδας αὐτῶν ἐκ τῶν ἐμπροσθίων.

\vs{15}Καὶ ποιήσεις λογεῖον τῶν κρίσεων, ἔργον ποικιλτοῦ· κατὰ τὸν ῥυθμὸν τῆς ἐπωμίδος ποιήσεις αὐτὸ ἐκ χρυσίου, καὶ ὑακίνθου, καὶ πορφύρας, καὶ κοκκίνου κεκλωσμένου, καὶ βύσσου κεκλωσμένης.
\vs{16}Ποιήσεις αὐτό τετράγωνον· ἔσται διπλοῦν, σπιθαμῆς τὸ μῆκος αὐτοῦ, καὶ σπιθαμῆς τὸ εὖρος.
\vs{17}Καὶ καθυφανεῖς ἐν αὐτῷ ὕφασμα κατάλιθον τετράστιχον· στίχος λίθων ἔσται, σάρδιον, τοπάζιον, καὶ σμαράγδος, ὁ στίχος ὁ εἷς.
\vs{18}Καὶ ὁ στίχος ὁ δεύτερος, ἄνθραξ, καὶ σάπφειρος, καὶ ἴασπις.
\vs{19}Καὶ ὁ στίχος ὁ τρίτος, λιγύριον, ἀχάτης, ἀμέθυστος.
\vs{20}Καὶ ὁ στίχος ὁ τέταρτος, χρυσόλιθος, καὶ βηρύλλιον, καὶ ὀνύχιον, περικεκαλυμμένα χρυσίῳ, συνδεδεμένα ἐν χρυσίῳ· ἔστωσαν κατὰ στίχον αὐτῶν.
\vs{21}Καὶ οἱ λίθοι ἔστωσαν ἐκ τῶν ὀνομάτων τῶν υἱῶν Ἰσραὴλ δεκαδύο κατὰ τὰ ὀνόματα αὐτῶν· γλυφαὶ σφραγίδων, ἕκαστος κατὰ τὸ ὄνομα ἔστωσαν εἰς δεκαδύο φυλάς.
\vs{22}Καὶ ποιήσεις ἐπὶ τὸ λογιον κρωσσοὺς συμπεπλεγμένους, ἔργον ἁλυσιδωτὸν ἐκ χρυσίου καθαροῦ.
\vs{29}Καὶ λήψεται Ἀαρὼν τὰ ὀνόματα τῶν υἱῶν Ἰσραὴλ ἐπὶ τοῦ λογείου τῆς κρίσεως ἐπὶ τοῦ στήθους, εἰσιόντι εἰς τὸ ἅγιον μνημόσυνου ἐναντίον τοῦ Θεοῦ.
\vs{29a}Καὶ θήσεις ἐπὶ τὸ λογεῖον τῆς κρίσεως τοὺς κρωσσούς· τὰ ἁλυσιδωτὰ ἐπʼ ἀμφοτέρων τῶν κλιτῶν τοῦ λογείου ἐπιθήσεις. Καὶ τὰς δύο ἀσπιδίσκας ἐπιθήσεις ἐπʼ ἀμφοτέρους τοὺς ὤμους τῆς ἐπωμίδος κατὰ πρόσωπον.
\vs{30}Καὶ ἐπιθήσεις ἐπὶ τὸ λογεῖον τῆς κρίσεως τὴν δήλωσιν καὶ τὴν ἀλήθειαν· καὶ ἔσται ἐπὶ τοῦ στήθους Ἀαρὼν, ὃταν εἰσπορεύεται εἰς τὸ ἅγιον ἔναντὶ Κυρίου· καὶ οἴσει Ἀαρὼν τὰς κρίσεις τῶν υἱῶν Ἰσραὴλ ἐπὶ τοῦ στήθους ἔναντι Κυρίου διαπαντός.
\vs{31}Καὶ ποιήσεις ὑποδύτην ποδήρη ὅλον ὑακίνθινον.
\vs{32}Καὶ ἔσται τὸ περιστόμιον ἐξ αὐτοῦ μέσον, ὤαν ἔχον κύκλῳ τοῦ περιστομίου, ἔργον ὑφαντου, τὴν συμβολὴν συνυφασμένην ἐξ αὐτοῦ, ἵνα μὴ ῥαγῇ.
\vs{33}Καὶ ποιήσεις ὑπὸ τὸ λῶμα τοῦ ὑποδύτου κάτωθεν, ὡσεὶ ἐξανθούσης ῥόας ῥοΐσκους ἐξ ὑακίνθου, καὶ πορφύρας, καὶ κοκκίνου διανενησμένου, καὶ βύσσου κεκλωσμένης, ὑπὸ τοῦ λώματος τοῦ ὑποδύτου κύκλῳ· τὸ αὐτὸ εἶδος ῥοΐσκους χρυσοῦς, καὶ κώδωνας ἀναμέσον τούτων περικύκλῳ.
\vs{34}Παρὰ ῥοΐσκον χρυσοῦν δώδωνα, καὶ ἄνθινον ἐπὶ τοῦ λώματος τοῦ ὑποδύτου κύκλῳ·
\vs{35}Καὶ ἔσται Ἀαρὼν ἐν τῷ λειτουργεῖν ἀκουστὴ ἡ φωνὴ αὐτοῦ, εἰσιόντι εἰς τὸ ἅγιον ἔναντι Κυρίου, καὶ ἐξιόντι, ἵνα μὴ ἀποθάνῃ.
\vs{36}Καὶ ποιήσεις πέταλον χρυσοῦν καθαρόν· καὶ ἐκτυπώσεις ἐν αὐτῷ ἐκτύπωμα σφραγίδος, Ἁγίασμα Κυρίου.
\vs{37}Καὶ ἐπιθήσεις αὐτὸ ἐπὶ ὑακίνθου κεκλωσμένης· καὶ ἔσται ἐπὶ τῆς μίτρας, κατὰ πρόσωπον τῆς μίτρας ἔσται.
\vs{38}Καὶ ἔσται ἐπὶ τοῦ μετώπου Ἀαρών· καὶ ἐξαρεῖ Ἀαρὼν τὰ ἁμαρτήματα τῶν ἁγίων, ὅσα ἂν ἁγιάσωσιν οἱ υἱοὶ Ἰσραὴλ παντὸς δόματος τῶν ἁγίων αὐτῶν· καὶ ἔσται ἐπὶ τοῦ μετώπου Ἀαρὼν διαπαντὸς δεκτὸν αὐτοῖς ἔναντι Κυρίου.

\vs{39}Καὶ οἱ κοσυμβωτοὶ τῶν χιτώνων ἐκ βύσσου· καὶ ποιήσεις κίδαριν βυσσίνην· καὶ ζώνην ποιήσεις, ἔργον ποικιλτοῦ.
\vs{40}Καὶ τοῖς υἱοῖς Ἀαρὼν ποιήσεις χιτῶνας καὶ ζώνας, καὶ κιδάρεις ποιήσεις αὐτοῖς εἰς τιμὴν καὶ δόξαν.
\vs{41}Καὶ ἐνδύσεις αὐτὰ Ἀαρὼν τὸν ἀδελφόν σου, καὶ τοὺς υἱοὺς αὐτοῦ μετʼ αὐτοῦ· καὶ χρίσεις αὐτοὺς, καὶ ἐμπλήσεις αὐτῶν τὰς χεῖρας· καὶ ἁγιάσεις αὐτοὺς, ἵνα ἱερατεύωσί μοι.
\vs{42}Καὶ ποιήσεις αὐτοῖς περισκελῆ λινᾶ καλύψαι ἀσχημοσύνην χρωτὸς αὐτῶν, ἀπὸ ὀσφύος ἕως μηρῶν ἔσται.
\vs{43}Καὶ ἕξει Ἀαρὼν αὐτὰ καὶ οἱ υἱοὶ αὐτοῦ, ὅταν εἰσπορεύωνται εἰς τὴν σκηνὴν τοῦ μαρτυρίου, ἢ ὅταν προσπορεύωνται λειτουργεῖν πρὸς τὸ θυσιαστήριον τοῦ ἁγίου· καὶ οὐκ ἐπάξονται πρὸς ἑαυτοὺς ἁμαρτίαν, ἵνα μὴ ἀποθάνωσι· νόμιμον αἰώνιον αὐτῷ, καὶ τῷ σπέρματι αὐτοῦ μετʼ αὐτόν.

\ch{29}
Καὶ ταῦτά ἐστιν, ἃ ποιήσεις αὐτοῖς· ἁγιάσεις αὐτοὺς, ὥστε ἱερατεύειν μοι αὐτούς· λήψῃ δὲ μοσχάριον ἐκ βοῶν ἓν, καὶ κριοὺς ἀμώμους δύο,
\vs{2}καὶ ἄρτους ἀζύμους πεφυραμένους ἑν ἐλαίῳ, καὶ λάγανα ἄζυμα κεχρισμένα ἐν ἐλαίῳ· σεμίδαλιν ἐκ πυρῶν ποιήσεις αὐτά.
\vs{3}Καὶ ἐπιθήσεις αὐτὰ ἐπὶ κανοῦν ἕν· καὶ προσοίσεις αὐτὰ ἐπὶ τῷ κανῷ· καὶ τὸ μοσχάριον, καὶ τοὺς δύο κριούς.
\vs{4}Καὶ Ἀαρὼν καὶ τοὺς υἱοὺς αὐτοῦ προσάξεις ἐπὶ τὰς θύρας τῆς σκηνῆς τοῦ μαρτυρίου, καὶ λούσεις αὐτοὺς ἐν ὕδατι.
\vs{5}Καὶ λαβὼν τὰς στολὰς, ἐνδύσεις Ἀαρὼν τὸν ἀδελφόν σου καὶ τὸν χιτῶνα τὸν ποδήρη, καὶ τὴν ἐπωμίδα, καὶ τὸ λογεῖον· καὶ συνάψεις αὐτῷ τὸ λογεῖον πρὸς τὴν ἐπωμίδα.
\vs{6}Καὶ ἐπιθήσεις τὴν μίτραν ἐπὶ τὴν κεφαλὴν αὐτοῦ, καὶ ἐπιθήσεις τὸ πέταλον τὸ ἁγίασμα ἐπὶ τὴν μίτραν.
\vs{7}Καὶ λήψῃ τοῦ ἐλαίου τοῦ χρίσματος· καὶ ἐπιχεεῖς αὐτὸ ἐπὶ τὴν κεφαλὴν αὐτοῦ, καὶ χρίσεις αὐτόν.
\vs{8}Καὶ τοὺς υἱοὺς αὐτοῦ προσάξεις, καὶ ἐνδύσεις αὐτοὺς χιτῶνας.
\vs{9}Καὶ ζώσεις αὐτοὺς ταῖς ζωναῖς, καὶ περιθήσεις αὐτοῖς τὰς κιδάρεις· καὶ ἔσται αὐτοῖς ἱερατῖα μοι εἰς τὸν αἰῶνα· καὶ τελειώσεις Ἀαρὼν τὰς χεῖρας αὐτοῦ, καὶ τὰς χεῖρας τῶν υἱῶν αὐτοῦ.
\vs{10}Καὶ προσάξεις τὸν μόσχον ἐπὶ τὰς θύρας τῆς σκηνῆς τοῦ μαρτυρίου· καὶ ἐπιθήσουσιν Ἀαρὼν καὶ οἱ υἱοὶ αὐτοῦ τὰς χεῖρας αὐτῶν ἐπὶ τὴν κεφαλὴν τοῦ μόσχου, ἔναντι Κυρίου, παρὰ τὰς θύρας τῆς σκηνῆς τοῦ μαρτυρίου.
\vs{11}Καὶ σφάξεις τὸν μόσχον ἔναντι Κυρίου, παρὰ τὰς θύρας τῆς σκηνῆς τοῦ μαρτυρίου.
\vs{12}Καὶ λήψῃ ἀπὸ τοῦ αἵματος τοῦ μόσχου, καὶ θήσεις ἐπὶ τῶν κεράτων τοῦ θυσιαστηρίου τῷ δακτύλῳ σου· τὸ δὲ λοιπὸν πᾶν αἷμα ἐκχεεῖς παρὰ τὴν βάσιν τοῦ θυσιαστηρίου.
\vs{13}Καὶ λήψῃ πᾶν τὸ στέαρ τὸ ἐπὶ τῆς κοιλίας, καὶ τὸν λοβὸν τοῦ ἥπατος, καὶ τοὺς δύο νεφροὺς, καὶ τὸ στέαρ τὸ ἐπʼ αὐτῶν, καὶ ἐπιθήσεις ἐπὶ τὸ θυσιαστήριον.
\vs{14}Τὰ δὲ κρέατα τοῦ μόσχου, καὶ τὸ δέρμα, καὶ τὴν κόπρον κατακαύσεις πυρὶ ἔξω τῆς παρεμβολῆς· ἁμαρτίας γάρ ἐστι.

\vs{15}Καὶ τὸν κριὸν λήψῃ τὸν ἕνα, καὶ ἐπιθήσουσιν Ἀαρὼν καὶ οἱ υἱοὶ αὐτοῦ τὰς χεῖρας αὐτῶν ἐπὶ τὴν κεφαλὴν τοῦ κριοῦ.
\vs{16}Καὶ σφάξεις αὐτὸν, καὶ λαβὼν τὸ αἷμα προσχεεῖς πρὸς τὸ θυσιαστήριον κύκλῳ.
\vs{17}Καὶ τὸν κριὸν διχοτομήσεις κατὰ μέλη· καὶ πλυνεῖς τὰ ἐνδόσθια καὶ τοὺς πόδας ὕδατι, καὶ ἐπιθήσεις ἐπὶ τὰ διχοτομήματα σὺν τῇ κεφαλῇ.
\vs{18}Καὶ ἀνοίσεις ὅλον τὸν κριὸν ἐπὶ τὸ θυσιαστήριον, ὁλοκαύτωμα τῷ Κυρίῳ εἰς ὀσμὴν εὐωδίας· θυμίαμα Κυρίῳ ἐστί.
\vs{19}Καὶ λήψῃ τὸν κριὸν τὸν δεύτερον, καὶ ἐπιθήσει Ἀαρὼν καὶ οἱ υἱοὶ αὐτοῦ τὰς χεῖρας αὐτῶν ἐπὶ τὴν κεφαλὴν τοῦ κριοῦ.
\vs{20}Καὶ σφάξεις αὐτὸν, καὶ λήψῃ τοῦ αἵματος αὐτοῦ, καὶ ἐπιθήσεις ἐπὶ τὸν λοβὸν τοῦ ὠτὸς Ἀαρὼν τοῦ δεξιοῦ, καὶ ἐπὶ τὸ ἄκρον τῆς δεξιᾶς χειρὸς, καὶ ἐπὶ τὸ ἄκρον τοῦ ποδὸς τοῦ δεξιοῦ, καὶ ἐπὶ τοὺς λοβοὺς τῶν ὤτων τῶν υἱῶν αὐτοῦ τῶν δεξιῶν, καὶ ἐπὶ τὰ ἄκρα τῶν χειρῶν αὐτῶν τῶν δεξιῶν, καὶ ἐπὶ τὰ ἄκρα τῶν ποδῶν αὐτῶν τῶν δεξιῶν.
\vs{21}Καὶ λήψῃ ἀπὸ τοῦ αἵματος τοῦ ἀπὸ τοῦ θυσιαστηρίου, καὶ ἀπὸ τοῦ ἐλαίου τῆς χρίσεως, καὶ ῥανεῖς ἐπὶ Ἀαρὼν καὶ ἐπὶ τὴν στολὴν αὐτοῦ, καὶ ἐπὶ τοὺς υἱοὺς αὐτοῦ καὶ ἐπὶ τὰς στολὰς τῶν υἱῶν αὐτοῦ μετʼ αὐτοῦ· καὶ ἁγιασθήσεται αὐτὸς καὶ ἡ στολὴ αὐτοῦ, καὶ οἱ υἱοὶ αὐτοῦ καὶ αἱ στολαὶ τῶν υἱῶν αὐτοῦ μετʼ αὐτοῦ· τὸ δὲ αἷμα τοῦ κριοῦ προσχεεῖς πρὸς τὸ θυσιαστήριον κύκλῳ.
\vs{22}Καὶ λήψῃ ἀπὸ τοῦ κριοῦ τὸ στέαρ αὐτοῦ, καὶ τὸ στέαρ τὸ κατακαλύπτον τὴν κοιλίαν, καὶ τὸν λοβὸν τοῦ ἥπατος, καὶ τοὺς δύο νεφροὺς, καὶ τὸ στέαρ τὸ ἐπʼ αὐτῶν, καὶ τὸν βραχίονα τὸν δεξιόν· ἔστι γὰρ τελείωσις αὕτη.
\vs{23}Καὶ ἄρτον ἕνα ἐξ ἐλαίου, καὶ λάγανον ἓν ἀπὸ τοῦ κανοῦ τῶν ἀζύμων τῶν προτεθειμένων ἔναντι Κυρίου.
\vs{24}Καὶ ἐπιθήσεις τὰ πάντα ἐπὶ τὰς χεῖρας Ἀαρὼν, καὶ ἐπὶ τὰς χεῖρας τῶν υἱῶν αὐτοῦ· καὶ ἀφοριεῖς αὐτὰ ἀφόρισμα ἔναντι Κυρίου.
\vs{25}Καὶ λήψῃ αὐτὰ ἐκ τῶν χειρῶν αὐτῶν, καὶ ἀνοίσεις ἐπὶ τὸ θυσιαστήριον τῆς ὁλοκαυτώσεως εἰς ὀσμὴν εὐωδίας ἔναντι Κύριου· κάρπωμά ἐστι Κυρίῳ.
\vs{26}Καὶ λήψῃ τὸ στηθύνιον ἀπὸ τοῦ κριοῦ τῆς τελειώσεως, ὅ ἐστιν Ἀαρών· καὶ ἀφοριεῖς αὐτὸ ἀφόρισμα ἔναντι Κυρίου· καὶ ἔσται σοι ἐν μερίδι.
\vs{27}Καὶ ἁγιάσεις τὸ στηθύνιον ἀφόρισμα, καὶ τὸν βραχίονα τοῦ ἀφαιρέματος, ὃς ἀφώρισται, καὶ ὃς ἀφῄρηται ἀπὸ τοῦ κριοῦ τῆς τελειώσεως ἀπὸ τοῦ Ἀαρὼν, καὶ ἀπὸ τῶν υἱῶν αὐτοῦ.
\vs{28}Καὶ ἔσται Ἀαρὼν καὶ τοῖς υἱοῖς αὐτοῦ νόμιμον αἰώνιον παρὰ τῶν υἱῶν Ἰσραήλ· ἔστι γὰρ ἀφόρισμα τοῦτο· καὶ ἀφαίρεμα ἕσται παρὰ τῶν υἱῶν Ἰσραὴλ ἀπὸ τῶν θυμάτων τῶν σωτηρίων τῶν υἱῶν Ἰσραὴλ, ἀφαίρεμα Κυρίῳ.

\vs{29}Καὶ ἡ στολὴ τοῦ ἁγίου, ἥ ἐστιν Ἀαρὼν, ἔσται τοῖς υἱοῖς αὐτοῦ μετʼ αὐτὸν, χρισθῆναι αὐτοὺς ἐν αὐτοῖς, καὶ τελειῶσαι τὰς χεῖρας αὐτῶν.
\vs{30}Ἑπτὰ ἡμέρας ἐνδύσεται αὐτὰ ὁ ἱερεὺς ὁ ἀντʼ αὐτοῦ ἐκ τῶν υἱῶν αὐτοῦ, ὃς εἰσελεύσεται εἰς τὴν σκηνὴν τοῦ μαρτυρίου λειτουργεῖν ἐν τοῖς ἁγίοις.
\vs{31}Καὶ τὸν κριὸν τῆς τελειώσεως λήψῃ· καὶ ἑψήσεις τὰ κρέα ἐν τόπῳ ἁγίῳ.
\vs{32}Καὶ ἔδονται Ἀαρὼν καὶ οἱ υἱοὶ αὐτοῦ τὰ κρέα τοῦ κριοῦ, καὶ τοὺς ἄρτους τοὺς ἐν τῷ κανῷ, παρὰ τὰς θύρας τῆς σκηνῆς τοῦ μαρτυρίου.
\vs{33}Ἔδονται αὐτὰ ἐν οἷς ἡγιάσθησαν ἐν αὐτοῖς τελειῶσαι τὰς χεῖρας αὐτῶν, ἁγιάσαι αὐτούς· καὶ ἀλλογενὴς οὐκ ἔδεται ἀπʼ αὐτῶν· ἔστι γὰρ ἅγια.
\vs{34}Ἐὰν δὲ καταλειφθῇ ἀπὸ τῶν κρεῶν τῆς θυσίας τῆς τελειώσεως καὶ τῶν ἄρτων ἕως πρωῒ, κατακαύσεις τὰ λοιπὰ πυρί· οὐ βρωθήσεται· ἁγίασμα γάρ ἐστι.

\vs{35}Καὶ ποιήσεις Ἀαρὼν καὶ τοῖς υἱοῖς αὐτοῦ οὕτω κατὰ πάντα ὅσα ἐνετειλάμην σοι· ἑπτὰ ἡμέρας τελειώσεις τὰς χεῖρας αὐτῶν.
\vs{36}Καὶ τὸ μοσχάριον τῆς ἁμαρτίας ποιήσεις τῇ ἡμέρᾳ τοῦ καθαρισμοῦ· καὶ καθαριεῖς τὸ θυσιαστήριον ἐν τῷ ἁγιάζειν σε ἐπʼ αὐτῷ· καὶ χρίσεις αὐτὸ ὥστε ἁγιάσαι αὐτό.
\vs{37}Ἑπτὰ ἡμέρας καθαριεῖς τὸ θυσιαστήριον, καὶ ἁγιάσεις αὐτό· καὶ ἔσται τὸ θυσιαστήριον, ἅγιον τοῦ ἁγίου· πᾶς ὁ ἁπτόμενος τοῦ θυσιαστηρίου, ἁγιασθήσεται.
\vs{38}Καὶ ταῦτά ἐστιν, ἃ ποιήσεις ἐπὶ τοῦ θυσιαστηρίου· ἀμνοὺς ἐνιαυσίους ἀμώμους δύο τὴν ἡμέραν ἐπὶ τὸ θυσιαστήριον ἐνδελεχῶς, κάρπωμα ἐνδελεχισμοῦ.

\vs{39}Τὸν ἀμνὸν τὸν ἕνα ποιήσεις τὸ πρωῒ, καὶ τὸν ἀμνὸν τὸν δεύτερον ποιήσεις τὸ δειλινόν.
\vs{40}Καὶ δέκατον σεμιδάλεως πεφυραμένης ἐν ἐλαίῳ κεκομμένῳ τῷ τετάρτῳ τοῦ εἴν· καὶ σπονδὴν τὸ τέταρτον τοῦ εἲν οἴνου τῷ ἀμνῷ τῷ ἑνί.
\vs{41}Καὶ τὸν ἀμνὸν τὸν δεύτερον ποιήσεις τὸ δειλινὸν, κατὰ τὴν θυσίαν τὴν πρωϊνὴν, καὶ κατὰ τὴν σπονδὴν αὐτοῦ· ποιήσεις εἰς ὀσμὴν εὐωδίας κάρπωμα Κυρίῳ,
\vs{42}θυσίαν ἐνδελεχισμοῦ εἰς γενεὰς ὑμῶν, ἐπὶ θύρας τῆς σκηνῆς τοῦ μαρτυρίου ἔναντι Κυρίου, ἐν οἷς γνωσθήσομαί σοι ἐκεῖθεν, ὥστε λαλῆσαί σοι.
\vs{43}Καὶ τάξομαι ἐκεῖ τοῖς υἱοῖς Ἰσραὴλ, καὶ ἁγιασθήσομαι ἐν δόξῃ μου.
\vs{44}Καὶ ἁγιάσω τὴν σκηνὴν τοῦ μαρτυρίου, καὶ τὸ θυσιαστήριον· καὶ Ἀαρὼν καὶ τοὺς υἱοὺς αὐτοῦ ἁγιάσω, ἱερατεύειν μοι.
\vs{45}Καὶ ἐπικληθήσομαι ἐν τοῖς υἱοῖς Ἰσραὴλ, καὶ ἔσομαι αὐτῶν Θεός.
\vs{46}Καὶ γνώσονται, ὅτι ἐγώ εἰμι Κύριος ὁ Θεὸς αὐτῶν, ὁ ἐξαγαγὼν αὐτοὺς ἐκ γῆς Αἰγύπτου, ἐπικληθῆναι αὐτοῖς, καὶ εἶναι αὐτῶν Θεός.

\ch{30}
Καὶ ποιήσεις θυσιαστήριον θυμιάματος ἐκ ξύλων ἀσήπτων.
\vs{2}Καὶ ποιήσεις αὐτὸ πήχεος τὸ μῆκος, καὶ πήχεος τὸ εὖρος· τετράγωνον ἔσται· καὶ δύο πήχεων τὸ ὕψος· ἐξ αὐτοῦ ἔσται τὰ κέρατα αὐτοῦ.
\vs{3}Καὶ καταχρυσώσεις χρυσίῳ καθαρῷ τὴν ἐσχάραν αὐτοῦ, καὶ τοὺς τοίχους αὐτοῦ κύκλῳ, καὶ τὰ κέρατα αὐτοῦ· καὶ ποιήσεις αὐτῷ στρεπτὴν στεφάνην χρυσῆν κύκλῳ.
\vs{4}Καὶ δύο δακτυλίους χρυσοῦς καθαροὺς ποιήσεις ὑπὸ τὴν στρεπτὴν στεφάνην αὐτοῦ, εἰς τὰ δύο κλίτη ποιήσεις ἐν τοῖς δυσὶ πλευροῖς· καὶ ἔσονται ψαλίδες ταῖς σκυτάλαις, ὥστε αἴρειν αὐτὸ ἐν αὐταῖς.
\vs{5}Καὶ ποιήσεις σκυτάλας ἐκ ξύλων ἀσήπτων, καὶ καταχρυσώσεις αὐτὰς χρυσίῳ.
\vs{6}Καὶ θήσεις αὐτὸ ἀπέναντι τοῦ καταπετάσματος, τοῦ ὄντος ἐπὶ τῆς κιβωτοῦ τῶν μαρτυρίων, ἐν οἷς γνωσθήσομαί σοι ἐκεῖθεν.
\vs{7}Καὶ θυμιάσει ἐπʼ αὐτοῦ Ἀαρὼν θυμίαμα σύνθετον λεπτὸν τὸ πρωῒ πρωΐ· ὅταν ἐπισκευάζῃ τοὺς λύχνους, θυμιάσει ἐπʼ αὐτοῦ.
\vs{8}Καὶ ὅταν ἐξάπτῃ Ἀαρὼν τοὺς λύχνους ὀψὲ, θυμιάσει ἐπʼ αὐτοῦ. θυμίαμα ἐνδελεχισμοῦ διαπαντὸς ἔναντι Κυρίου εἰς γενεὰς αὐτῶν.
\vs{9}Καὶ οὐκ ἀνοίσει ἐπʼ αὐτοῦ θυμίαμα ἕτερον· κάρπωμα, θυσίαν, και σπονδὴν οὐ σπείσεις ἐπʼ αὐτοῦ.
\vs{10}Καὶ ἐξιλάσεται ἐπʼ αὐτοῦ Ἀαρὼν ἐπὶ τῶν κεράτων αὐτοῦ ἅπαξ τοῦ ἐνιαυτοῦ· ἀπὸ τοῦ αἵματος τοῦ καθαρισμοῦ καθαριεῖ αὐτὸ εἰς γενεὰς αὐτῶν· ἅγιον τῶν ἁγίων ἐστὶ Κυρίῳ.

\vs{11}Καὶ ἑλάλησε Κύριος πρὸς Μωυσῆν, λέγων,
\vs{12}ἐὰν λάβῃς τὸν συλλογισμὸν τῶν υἱῶν Ἰσραὴλ ἐν τῇ ἐπισκοπῇ αὐτῶν, καὶ δώσουσιν ἕκαστος λύτρα τῆς ψυχῆς αὐτοῦ Κυρίῳ, καὶ οὐκ ἔσται ἐν αὐτοῖς πτῶσις ἐν τῇ ἐπισκοπῇ αὐτῶν.
\vs{13}Καὶ τοῦτό ἐστιν ὅ δώσουσιν ὅσοι ἂν παραπορεύωνται τὴν ἐπίσκεψιν· τὸ ἥμισυ τοῦ διδράχμου ὅ ἐστιν κατὰ τὸ δίδραχμον τὸ ἅγιον, εἴκοσι ὀβολοὶ τὸ δίδραχμον, τὸ δὲ ἥμισυ τοῦ διδράχμου εἰσφορὰ Κυρίῳ.
\vs{14}Πᾶς ὁ παραπορευόμενος εἰς τὴν ἐπίσκεψιν ἀπὸ εἰκοσαετοῦς καὶ ἐπάνω, δώσουσι τὴν εἰσφορὰν Κυρίῳ.
\vs{15}Ὁ πλουτῶν οὐ προσθήσει, καὶ ὁ πενόμενος οὐκ ἐλαττονήσει ἀπὸ τοῦ ἡμίσεως τοῦ διδράχμου ἐν τῷ διδόναι τὴν εἰσφορὰν Κυρίῳ, ἐξιλάσασθαι περὶ τῶν ψυχῶν ὑμῶν.
\vs{16}Καὶ λήψῃ τὸ ἀργύριον τῆς εἰσφορᾶς παρὰ τῶν υἱῶν Ἰσραήλ, καὶ δώσεις αὐτὸ εἰς τὸ κάτεργον τῆς σκηνῆς τοῦ μαρτυρίου· καὶ ἔσται τοῖς υἱοῖς Ἰσραὴλ μνημόσυνον ἔναντι Κυρίου, ἐξιλάσασθαι περὶ τῶν ψυχῶν ὑμῶν.
\vs{17}Καὶ ἐλάλησε Κύριος πρὸς Μωυσῆν, λέγων,
\vs{18}ποίησον λουτῆρα χαλκοῦν, καὶ βάσιν αὐτῷ χαλκῆν, ὥστε νίπτεσθαι· καὶ θήσεις αὐτὸν ἀνὰ μέσον τῆς σκηνῆς τοῦ μαρτυρίου καὶ ἀνὰ μέσον τοῦ θυσιαστηρίου· καὶ ἐκχεεῖς εἰς αὐτὸν ὕδωρ.
\vs{19}Καὶ νίψεται Ἀαρὼν καὶ οἱ υἱοὶ αὐτοῦ ἑξ αὐτοῦ τὰς χεῖρας, καὶ τοὺς πόδας ὕδατι.
\vs{20}Ὅταν εἰσπορεύωνται εἰς τὴν σκηνὴν τοῦ μαρτυρίου, νίψονται ὕδατι, καὶ οὐ μὴ ἀποθάνωσιν, ὅταν προσπορεύωνται πρὸς τὸ θυσιαστήριον λειτουργεῖν καὶ ἀναφέρειν τὰ ὁλοκαυτώματα Κυρίῳ.
\vs{21}Νίψονται τὰς χεῖρας καὶ τοὺς πόδας ὕδατι, ὅταν εἰσπορεύωνται εἰς τὴν σκηνὴν τοῦ μαρτυρίου, νίψονται ὕδατι, ἵνα μὴ ἀποθάνωσι· καὶ ἔσται αὐτοῖς νόμιμον αἰώνιον, αὐτῷ καὶ ταῖς γενεαῖς αὐτοῦ μετʼ αὐτόν.
\vs{22}Καὶ ἐλάλησε Κύριος πρὸς Μωυσῆν, λέγων,
\vs{23}καὶ σὺ λάβε ἡδύσματα, τὸ ἄνθος σμύρνης ἐκλεκτῆς πεντακοσίους σίκλους, καὶ κινναμώμου εὐώδους τὸ ἥμισυ τούτου διακοσίους πεντήκοντα, καὶ καλάμου εὐώδους διακοσίους πεντήκοντα,
\vs{24}καὶ ἴρεως πεντακοσίους σίκλους τοῦ ἁγίου, καὶ ἔλαιον ἐξ ἐλαιῶν εἵν.
\vs{25}Καὶ ποιήσεις αὐτὸ ἔλαιον χρίσμα ἅγιον, μύρον μυρεψικὸν τέχνῃ μυρεψοῦ· ἔλαιον χρίσμα ἅγιον ἔσται.
\vs{26}Καὶ χρίσεις ἐξ αὐτοῦ τὴν σκηνὴν τοῦ μαρτυρίου, καὶ τὴν κιβωτὸν τῆς σκηνῆς τοῦ μαρτυρίου, καὶ πάντα τὰ σκεύη αὐτῆς,
\vs{27}καὶ τὴν λυχνίαν καὶ πάντα τὰ σκεύη αὐτῆς, καὶ τὸ θυσιαστήριον τοῦ θυμιάματος,
\vs{28}καὶ τὸ θυσιαστήριον τῶν ὁλοκαυτωμάτων καὶ πάντα αὐτοῦ τὰ σκεύη, καὶ τὴν τράπεζαν καὶ πάντα τὰ σκεύη αὐτῆς, καὶ τὸν λουτῆρα.
\vs{29}Καὶ ἁγιάσεις αὐτά· καὶ ἔσται ἅγια τῶν ἁγίων· πᾶς ὁ ἁπτόμενος αὐτῶν, ἁγιασθήσεται.
\vs{30}Καὶ Ἀαρὼν καὶ τοὺς υἱοὺς αὐτοῦ χρίσεις, καὶ ἁγιάσεις αὐτοὺς ἱερατεύειν μοι.
\vs{31}Καὶ τοῖς υἱοῖς Ἰσραὴλ λαλήσεις, λέγων, ἔλαιον ἄλειμμα χρίσεως ἅγιον ἔσται τοῦτο ὑμῖν εἰς τὰς γενεὰς ὑμῶν.
\vs{32}Ἐπὶ σάρκα ἀνθρώπου οὐ χρισθήσεται· καὶ κατὰ τὴν σύνθεσιν ταύτην οὐ ποιήσετε ὑμῖν ἑαυτοῖς ὡσαύτως· ἅγιόν ἐστιν, καὶ ἁγίασμα ἔσται ὑμῖν.
\vs{33}Ὃς ἂν ποιήσῃ ὡσαύτως, καὶ ὃς ἂν δῷ ἀπʼ αὐτοῦ ἀλλογενεῖ, ἐξολοθρευθήσεται ἐκ τοῦ λαοῦ αὐτοῦ.

\vs{34}Καὶ εἶπε Κύριος πρὸς Μωυσῆν, λάβε σεαυτῷ ἡδύσματα, στακτήν, ὄνυχα, χαλβάνην ἡδυσμοῦ καὶ λίβανον διαφανῆ· ἴσον ἴσῳ ἔσται.
\vs{35}Καὶ ποιήσουσιν ἐν αὐτῷ θυμίαμα μυρεψικὸν ἔργον μυρεψοῦ μεμιγμένον, καθαρὸν ἔργον ἅγιον.
\vs{36}Καὶ συνκόψεις ἐκ τούτων λεπτόν, καὶ θήσεις ἀπέναντι τῶν μαρτυρίων ἐν τῇ σκηνῇ τοῦ μαρτυρίου, ὅθεν γνωσθήσομαί σοι ἐκεῖθεν· ἅγιον τῶν ἁγίων ἔσται ὑμῖν θυμίαμα.
\vs{37}Κατὰ τὴν σύνθεσιν ταύτην οὐ ποιήσετε ὑμῖν ἐαυτοῖς· ἁγίασμα ἔσται ὑμῖν Κυρίῳ·
\vs{38}Ὃς ἂν ποιήσῃ ὡσαύτως ὥστε ὀσφραίνεσθαι ἐν αὐτῷ, ἀπολεῖται ἐκ τοῦ λαοῦ αὐτοῦ.

\ch{31}
Καὶ ἐλάλησε Κύριος πρὸς Μωυσῆν, λέγων,
\vs{2}ἰδοὺ ἀνακέκλημαι ἐξ ὀνόματος τὸν Βεσελεὴλ τὸν τοῦ Οὐρείου τὸν Ὣρ, ἐκ τῆς φυλῆς Ἰούδα.
\vs{3}Καὶ ἐνέπλησα αὐτὸν πνεῦμα θεῖον σοφίας καὶ συνέσεως καὶ ἐπιστήμης, ἐν παντὶ ἔργῳ διανοεῖσθαι,
\vs{4}καὶ ἀρχιτεκτονῆσαι, ἐργάζεσθαι τὸ χρυσίον, καὶ τὸ ἀργύριον, καὶ τὸν χαλκὸν, καὶ τὴν ὑάκινθον, καὶ τὴν πορφύραν, καὶ τὸ κόκκινον τὸ νηστὸν,
\vs{5}καὶ τὰ λιθουργικὰ, καὶ εἰς τὰ ἔργα τὰ τεκτονικὰ τῶν ξύλων, ἐργάζεσθαι κατὰ πάντα τὰ ἔργα.
\vs{6}Καὶ ἐγὼ ἔδωκα αὐτὸν καὶ τὸν Ἐλιὰβ τὸν τοῦ Ἀχισαμὰχ ἐκ φυλῆς Δάν· καὶ παντὶ συνετῷ καρδίᾳ δέδωκα σύνεσιν·
\vs{7}καὶ πονήσουσι πάντα ὅσα συνέταξά σοι, τὴν σκηνὴν τοῦ μαρτυρίου, καὶ τὴν κιβωτὸν τῆς διαθήκης, καὶ τὸ ἱλαστήριον τὸ ἐπʼ αὐτῆς, καὶ τὴν διασκευὴν τῆς σκηνῆς,
\vs{8}καὶ τὰ θυσιαστήρια, καὶ τὴν τράπεζαν καὶ πάντα τὰ σκεύη αὐτῆς, καὶ τὴν λυχνίαν τὴν καθαρὰν καὶ πάντα τὰ σκεύη αὐτῆς
\vs{9}καὶ τὸν λουτῆρα καὶ τὴν βάσιν αὐτοῦ,
\vs{10}καὶ τὰς στολὰς τὰς λειτουργικὰς Ἀαρὼν, καὶ τὰς στολὰς τῶν υἱῶν αὐτοῦ ἱερατεύειν μοι,
\vs{11}καὶ τὸ ἔλαιον τῆς χρίσεως, καὶ τὸ θυμίαμα τῆς συνθέσεως τοῦ ἁγίου· κατὰ πάντα ὅσα ἐγὼ ἐνετειλάμην σοι, ποιήσουσι.

\vs{12}Καὶ ἐλάλησε Κύριος πρὸς Μωυσῆν, λέγων,
\vs{13}Καὶ σὺ σύνταξον τοῖς υἱοῖς Ἰσραὴλ, λέγων, Ὁρᾶτε, καὶ τὰ σάββατά μου φυλάξεσθε· σημεῖόν ἐστι παρʼ ἐμοὶ καὶ ἐν ὑμῖν εἰς τὰς γενεὰς ὑμῶν, ἵνα γνῶτε ὅτι ἐγὼ Κύριος ὁ ἁγιάξων ὑμᾶς.
\vs{14}καὶ φυλάξεσθε τὰ σάββατα, ὅτι ἅγιον τοῦτό ἐστι Κυρίῳ ὑμῖν· ὁ βεβηλῶν αὐτὸ, θανάτῳ θανατωθήσεται· πᾶς ὃς ποιήσει ἐν αὐτῷ ἔργον, ἐξολοθρευθήσεται ἡ ψυχὴ ἐκείνη ἐκ μέσου τοῦ λαοῦ αὐτοῦ.
\vs{15}ἓξ ἡμέρας ποιήσεις ἔργα, τῇ δὲ ἡμέρᾳ τῇ ἑβδόμῃ σάββατα, ἀνάπαυσις ἁγία τῷ κυρίῳ· πᾶς ὃς ποιήσει ἔργον τῇ ἡμέρᾳ τῇ ἑβδόμῃ θανατωθήσεται.
\vs{16}Καὶ φυλάξουσιν οἱ υἱοὶ Ἰσραὴλ τὰ σάββατα, ποιεῖν αὐτὰ εἰς τὰς γενεὰς αὐτῶν·
\vs{17}Διαθήκη αἰώνιος ἐν ἐμοὶ καὶ τοῖς υἱοῖς Ἰσραὴλ, σημεῖόν ἐστιν ἐν ἐμοὶ αἰώνιον· ὅτι ἓξ ἡμεραις ἐποίησε Κύριος τὸν οὐρανὸν καὶ τὴν γῆν, καὶ τῇ ἡμέρᾳ τῇ ἑβδόμῃ κατέπαυσε, καὶ ἐπαύσατο.
\vs{18}Καὶ ἔδωκε Μωυσῇ ἡνίκα κατέπαυσε λαλῶν αὐτῷ ἐν τῷ ὄρει τῷ Σινὰ, τὰς δύο πλάκας τοῦ μαρτυρίου, πλάκας λιθίνας γεγραμμένας τῷ δακτύλῳ τοῦ Θεοῦ.

\ch{32}
Καὶ ἰδὼν ὁ λαὸς, ὅτι κεχρόνικε Μωυσῆς καταβῆναι ἐκ τοῦ ὄρους, συνέστη ὁ λαὸς ἐπὶ Ἀαρὼν, καὶ λέγουσιν αὐτῷ ἀνάστηθι, καὶ ποίησον ἡμῖν θεοὺς, οἳ προπορεύσονται ἡμῶν· ὁ γὰρ Μωυσῆς οὗτος ὁ ἄνθρωπος ὃς ἐξήγαγεν ἡμᾶς ἐκ γῆς Αἰγύπτου, οὐκ οἴδαμεν τί γέγονεν αὐτῷ.
\vs{2}Καὶ λέγει αὐτοῖς Ἀαρὼν Περιέλεσθε τὰ ἐνώτια τὰ χρυσᾶ τὰ ἐν τοῖς ὠσὶ τῶν γυναικῶν ὑμῶν καὶ θυγατέρων, καὶ ἐνέγκατε πρός με.
\vs{3}Καὶ περιείλαντο πᾶς ὁ λαὸς τὰ ἐνώτια τὰ χρυσᾶ τὰ ἐν τοῖς ὠσὶν αὐτῶν, καὶ ἤνεγκαν πρὸς Ἀαρών.
\vs{4}Καὶ ἐδέξατο ἐκ τῶν χειρῶν αὐτῶν, καὶ ἔπλασεν αὐτὰ ἐν τῇ γραφίδι· καὶ ἐποίησεν αὐτὰ μόσχον χωνευτὸν καὶ εἶπεν, Οὗτοι οἱ θεοί σου Ἰσραὴλ, οἵτινες ἀνεβίβασάν σε ἐκ γῆς Αἰγύπτου.
\vs{5}Καὶ ἰδὼν Ἀαρὼν ᾠκοδόμησε θυσιαστήριον κατέναντι αὐτοῦ· καὶ ἐκήρυξεν Ἀαρὼν λέγων, ἑορτὴ τοῦ κυρίου αὔριον.
\vs{6}Καὶ ὀρθρίσας τῇ ἐπαύριον ἀνεβίβασεν ὁλοκαυτώματα, καὶ προσήνεγκε θυσίαν σωτηρίου· καὶ ἐκάθισεν ὁ λαὸς φαγεῖν καὶ πιεῖν, καὶ ἀνέστησαν παίζειν.

\vs{7}Καὶ ἐλάλησε Κύριος πρὸς Μωυσῆν, λέγων, βάδιζε τὸ τάχος, κατάβηθι ἐντεύθεν· ἠνόμησε γὰρ ὁ λαός σου ὃν ἐξήγαγες ἐκ γῆς Αἰγύπτου.
\vs{8}Παρέβησαν ταχὺ ἐκ τῆς ὁδοῦ, ἧς ἐνετείλω αὐτοῖς· ἐποίησαν ἑαυτοῖς μόσχον, καὶ προσκεκυνήκασιν αὐτῷ, καὶ τεθύκασιν αὐτῷ, καὶ εἶπαν, Οὗτοι οἱ θεοί σου Ἰσραὴλ, οἵτινες ἀνεβίβασάν σε ἐκ γῆς Αἰγύπτου.
\vs{10}καὶ νῦν ἔασόν με, καὶ θυμωθεὶς ὀργῇ εἰς αὐτοὺς, ἐκτρίψω αὐτούς· καὶ ποιήσω σὲ εἰς ἔθνος μέγα.
\vs{11}καὶ ἐδεήθη Μωυσῆς ἔναντι Κυρίου τοῦ Θεοῦ, καὶ εἶπεν, ἱνατί, Κύριε, θυμοῖ ὀργῇ εἰς τὸν λαόν σου, οὓς ἐξήγαγες ἐκ γῆς Αἰγύπτου ἐν ἰσχύϊ μεγάλῃ, καὶ ἐν τῷ βραχίονί σου τῷ ὑψηλῷ;
\vs{12}Μή ποτε εἴπωσιν οἱ Αἰγύπτιοι λέγοντες Μετὰ πονηρίας ἐξήγαγεν αὐτοὺς ἀποκτεῖναι ἐν τοῖς ὄρεσιν καὶ ἐξαναλῶσαι αὐτοὺς ἀπὸ τῆς γῆς. παῦσαι τῆς ὀργῆς τοῦ θυμοῦ σου, καὶ ἵλεως γενοῦ ἐπὶ τῇ κακίᾳ τοῦ λαοῦ σου,
\vs{13}μνησθεὶς Ἀβραὰμ καὶ Ἰσαὰκ καὶ Ἰακὼβ τῶν σῶν οἰκετῶν, οἷς ὤμοσας κατὰ σεαυτοῦ, καὶ ἐλάλησας πρὸς αὐτοὺς, λέγων, πολυπληθυνῶ τὸ σπέρμα ὑμῶν ὡσεὶ τὰ ἄστρα τοῦ οὐρανοῦ τῷ πλήθει· καὶ πᾶσαν τὴν γῆν ταύτην ἣν εἶπας δοῦναι αὐτοῖς, καὶ καθέξουσιν αὐτὴν εἰς τὸν αἰῶνα.
\vs{14}Καὶ ἱλάσθη Κύριος περιποιῆσαι τὸν λαὸν αὐτοῦ.

\vs{15}Καὶ ἀποστρέψας Μωυσῆς, κατέβη ἀπὸ τοῦ ὄρους· καὶ αἱ δύο πλάκες τοῦ μαρτυρίου ἐν ταῖς χερσὶν αὐτοῦ, πλάκες λίθιναι καταγεγραμμέναι ἐξ ἀμφοτέρων τῶν μερῶν αὐτῶν, ἔνθεν καὶ ἔνθεν ἦσανς γεγραμμέναι·
\vs{16}καὶ αἱ πλάκες ἔργον Θεοῦ ἦσαν, καὶ ἡ γραφὴ γραφὴ Θεοῦ κεκολαμμένη ἐν ταῖς πλαξί.
\vs{17}καὶ ἀκούσας Ἰησοῦς τῆς φωνῆς τοῦ λαοῦ κραζόντων, λέγει πρὸς Μωυσὴν, Φωνὴ πολέμου ἐν τῇ παρεμβολῇ.
\vs{18}καὶ λέγει Οὐκ ἔστι φωνὴ ἐξαρχόντων κατʼ ἰσχὺν, οὐδὲ φωνὴ ἐξαρχόντων τροπῆς, ἀλλὰ φωνὴν ἐξαρχόντων οἴνου ἐγὼ ἀκούω.

\vs{19}καὶ ἡνίκα ἤγγιζε τῇ παρεμβολῇ, ὁρᾷ τὸν μόσχον καὶ τοὺς χορούς· καὶ ὀργισθεὶς θυμῷ Μωυσῆς ἔῤῥιψεν ἀπὸ τῶν χειρῶν αὐτοῦ τὰς δύο πλάκας, καὶ συνέτριψεν αὐτὰς ὑπὸ τὸ ὄρος·
\vs{20}καὶ λαβὼν τὸν μόσχον ὃν ἐποίησαν, κατέκαυσεν αὐτὸν ἐν πυρὶ, καὶ κατήλεσεν αὐτὸν λεπτὸν, καὶ ἔσπειρεν αὐτὸν ὑπὸ τὸ ὕδωρ, καὶ ἐπότισεν αὐτὸ τοὺς υἱοὺς Ἰσραήλ.
\vs{21}καὶ εἶπε Μωυσῆς τῷ Ἀαρὼν, Τί ἐποίησέ σοι ὁ λαὸς οὗτος, ὅτι ἐπήγαγες ἐπʼ αὐτοὺς ἁμαρτίαν μεγάλην;
\vs{22}καὶ εἶπεν Ἀαρὼν πρὸς Μωυσῆν, μὴ ὀργίζου, κύριε· σὺ γὰρ οἶδας τὸ ὅρμημα τοῦ λαοῦ τούτου.
\vs{23}Λέγουσι γάρ μοι, ποιήσον ἡμῖν θεοὺς, οἳ προπορεύσονται ἡμῶν· ὁ γὰρ Μωυσῆς οὗτος ὁ ἄνθρωπος, ὃς ἐξήγαγεν ἡμᾶς ἐξ Αἰγύπτου, οὐκ οἴδαμεν τί γέγονεν αὐτῷ.
\vs{24}καὶ εἶπα αὐτοῖς, εἴ τινι ὑπάρχει χρυσία, περιέλεσθε· καὶ ἔδωκάν μοι· καὶ ἔῤῥιψα εἰς τὸ πῦρ· καὶ ἐξῆλθεν ὁ μόσχος οὗτος.
\vs{25}Καὶ ἰδὼν Μωυσῆς τὸν λαὸν ὅτι διεσκέδασται· (διεσκέδασε γὰρ αὐτοὺς Ἀαρών ἐπίχαρμα τοῖς ὑπεναντίοις αὐτῶν)
\vs{26}ἔστη δὲ Μωυσῆς ἑπὶ τῆς πύλης τῆς παρεμβολῆς, καὶ εἶπε, τίς πρὸς Κύριον; ἴτω πρός με. Συνῆλθον οὖν πρὸς αὐτὸν πάντες οἱ υἱοὶ Λευί.
\vs{27}Καὶ λέγει αὐτοῖς τάδε λέγει Κύριος ὁ Θεὸς Ἰσραήλ θέσθε ἕκαστος τὴν ἑαυτοῦ ῥομφαίανν ἐπὶ τὸν μηρὸν, καὶ διέλθατε καὶ ἀνακάμψατε ἀπὸ πύλης ἐπὶ πύλην διὰ τῆς παρεμβολῆς, καὶ ἀποκτείνατε ἕκαστος τὸν ἀδελφὸν αὐτοῦ, καὶ ἕκαστος τὸν πλησίον αὐτοῦ, καὶ ἓκαστος τὸν ἔγγιστα αὐτοῦ.
\vs{28}Καὶ ἐποίησαν οἱ υἱοὶ Λευεὶ καθὰ ἐλάλησεν αὐτοῖς Μωυσῆς· καὶ ἔπεσαν ἐκ τοῦ λαοῦ ἐν ἐκείνῃ τῇ ἡμέρᾳ εἰς τρισχιλίους ἄνδρας.
\vs{29}Καὶ εἶπεν αὐτοῖς Μωυσῆς, ἐπληρώσατε τὰς χεῖρας ὑμῶν σήμερον Κυρίῳ ἕκαστος ἐν τῷ υἱῷ ἢ ἐν τῷ ἀδελφῷ αὐτοῦ, δοθῆναι ἐφʼ ὑμᾶς εὐλογίαν.

\vs{30}Καὶ ἐγένετο μετὰ τὴν αὔριον εἶπε Μωυσῆς πρὸς τὸν λαὸν, ὑμεῖς ἡμαρτήκατε ἁμαρτίαν μεγάλη· καὶ νῦν ἀναβήσομαι πρὸς τὸν Θεὸν, ἵνα ἐξιλάσωμαι περὶ τῆς ἁμαρτίας ὑμῶν.
\vs{31}Ὑπέστρεψε δὲ Μωυσῆς πρὸς Κύριον, καὶ εἶπε, δέομαι, κύριε· ἡμάρτηκεν ὁ λαὸς οὗτος ἁμαρτίαν μεγάλην, καὶ ἐποίησαν ἑαυτοῖς θεοὺς χρυσοῦς.
\vs{32}Καὶ νῦν εἰ μὲν ἀφεῖς αὐτοῖς τὴν ἁμαρτίαν αὐτῶν, ἄφες· εἰ δὲ μή, ἐξάλειψόν με ἐκ τῆς βίβλου σου, ἧς ἔγραψας.
\vs{33}Καὶ εἶπε Κύριος πρὸς Μωυσῆν, εἴ τις ἡμάρτηκεν ἐνώπιόν μου, ἐξαλείψω αὐτοὺς ἐκ τῆς βίβλου μου.
\vs{34}Νυνὶ δὲ βάδιζε, κατάβηθι, καὶ ὁδήγησον τὸν λαὸν τοῦτον εἰς τὸν τόπον, ὃν εἶπά σοι· ἰδοὺ ὁ ἄγγελός μου προπορεύσεται πρὸ προσώπου σου· ᾖ δʼ ἂν ἡμέρᾳ ἐπισκέπτωμαι, ἐπάξω ἐπʼ αὐτοὺς τὴν ἁμαρτίαν αὐτῶν
\vs{35}Καὶ ἐπάταξε Κύριος τὸν λαὸν περὶ τῆς ποιήσεως τοῦ μόσχου, οὗ ἐποίησεν Ἀαρών.

\ch{33}
Καὶ εἶπε Κύριος πρὸς Μωυσῆς, προπορεύου, ἀνάβηθι ἐντεῦθεν σὺ καὶ ὁ λαός σου, οὓς ἐξήγαγες ἐκ γῆς Αἰγύπτου, εἰς τὴν γῆν, ἣν ὤμοσα τῷ Ἀβραὰμ, καὶ Ἰσαὰκ, καὶ Ἰακὼβ, λέγων, Τῷ σπέρματι ὑμῶν δώσω αὐτήν.
\vs{2}Καὶ συναποστελῶ τὸν ἄγγελόν μου πρὸ προσώπου σου· καὶ ἐκβαλεῖ τὸν Ἀμοῤῥαῖον, καὶ Χετταῖον, καὶ Φερεζαῖον, καὶ Γεργεσαῖον, καὶ Εὐαῖον, καὶ Ἰεβουσαῖον, καὶ Χαναναῖον.
\vs{3}Καὶ εἰσάξω σε εἰς γῆν ῥέουσαν γάλα καὶ μέλι· οὐ γὰρ μὴ συναναβῶ μετὰ σου, διὰ τὸ λαὸν σκληροτράχηλόν σε εἶναι, ἵνα μὴ ἐξαναλώσω σεε ἐν τῇ ὁδῷ.
\vs{4}Καὶ ἀκούσας ὁ λαὸς τὸ ῥῆμα τὸ πονηρὸν τοῦτο, κατεπένθησεν ἐν πενθικοῖς.
\vs{5}Καὶ εἶπε Κύριος τοῖς υἱοῖς Ἰσραὴλ, ὑμεῖς λαὸς σκληροτράχηλος· ὁρᾶτε, μὴ πληγὴν ἄλλην ἐπάξω ἐγὼ ἐφʼ ὑμᾶς, καὶ ἐξαναλώσω ὑμᾶς· νῦν οὖν ἀφέλεσθε τὰς στολὰς τῶν δοξῶν ὑμῶν, καὶ τὸν κόσμον, καὶ δείξω σοι ἃ ποιήσω σοι.
\vs{6}Καὶ περιέλαντο οἱ υἱοὶ Ἰσραὴλ τὸν κόσμον αὐτῶν, καὶ τὴν περιστολὴν ἀπὸ τοῦ ὄρους τοῦ Χωρήβ.
\vs{7}Καὶ λαβὼν Μωυσῆς τὴν σκηνὴν αὐτοῦ, ἔπηξεν ἔξω τῆς παρεμβολῆς, μακρὰν ἀπὸ τῆς παρεμβολῆς· καὶ ἐκλήθν Σκηνὴ μαρτυρίου· καὶ ἐγένετο, πᾶς ὁ ζητῶν Κύριον ἐξεπορεύετο εἰς τὴν σκηνὴν τὴν ἔξω τῆς παρεμβολῆς.
\vs{8}Ἡνίκα δʼ ἂν εἰσεπορεύετο Μωυσῆς εἰς τὴν σκηνὴν ἔξω τῆς παρεμβολῆς, εἱστήκει πᾶς ὁ λαὸς σκοπεύοντες ἕκαστος παρὰ τὰς θύρας τῆς σκηνῆς αὐτοῦ· καὶ κατενοοῦσαν ἀπιόντος Μωυσῆ ἕως τοῦ εἰσελθεῖν αὐτὸν εἰς τὴν σκηνὴν.
\vs{9}Ὡς δʼ ἂν εἰσῆλθε Μωσῆς εἰς τὴν σκηνήν, κατέβαινεν ὁ στύλος τῆς νεφέλης, καὶ ἵστατο ἐπὶ τὴν θύραν τῆς σκηνῆς, καὶ ἐλάλει Μωσῇ·
\vs{10}καὶ ἐλάλει Μωυσῇ. Καὶ ἑώρα πᾶς ὁ λαὸς τὸν στύλον τῆς νεφέλης ἑστῶτα ἐπὶ τῆς θύρας τῆς σκηνῆς· καὶ στάντες πᾶς ὁ λαὸς, προσεκύνησαν ἕκαστος ἀπὸ τῆς θύρας τῆς σκηνῆς αὐτοῦ.
\vs{11}Καὶ ἐλάλησε Κύριος πρὸς Μωυσῆν, ἐνώπιος ἐνωπίῳ, ὡς εἴτις λαλήσει πρὸς τὸν ἑαυτοῦ φίλον· καὶ ἀπελύετο εἰς τὴν παρεμβολήν· ὁ δὲ θεράπων Ἰησοῦς υἱὸς Ναυὴ νέος οὐκ ἐξεπορεύετο ἐκ τῆς σκηνῆς.

\vs{12}Καὶ εἶπε Μωυσῆς, πρὸς Κύριον, Ἰδοὺ σύ μοι λέγεις, ἀνάγαγε τὸν λαὸν τοῦτον, σὺ δὲ οὐκ ἐδήλωσάς μοι, ὃν συναποστελεῖς μετʼ ἐμοῦ· σὺ δέ μοι εἶπας, Οἶδά σε παρὰ πάντας, καὶ χάριν ἔχεις παρʼ ἐμοί.
\vs{13}Εἰ οὖν εὕρηκα χάριν ἐναντίον σου, ἐμφάνισόν μοι σεαυτόν· γνωστῶς ἴδω ἴδω σε, ὅπως ἂν ὦ εὑρηκὼς χάριν ἐναντίον σου, καὶ ἵνα γνῶ, ὅτι λαός σου τὸ ἔθνος τὸ μέγα τοῦτο.
\vs{14}Καὶ λέγει, αὐτὸς προπορεύσομαί σου, καὶ καταπαύσω σε.
\vs{15}Καὶ λέγει πρὸς αὐτόν, εἰ μὴ αὐτὸς σὺ σνμπορεύῃ, μή με ἀναγάγῃς ἐντεῦθεν.
\vs{16}Καὶ πῶς γνωστὸν ἔσται ἀληθῶς, ὅτι εὕρηκα χάριν παρὰ σοί ἐγώ τε καὶ ὁ λαός σου, ἀλλʼ ἢ συμπορευομένου σου μεθʼ ἡμῶν; καὶ ἐνδοξασθήσομαι ἐγώ τε καὶ ὁ λαός σου παρὰ πάντα τὰ ἔθνη, ὅσα ἐπὶ τῆς γῆς ἐστί.
\vs{17}Καὶ εἶπε Κύριος πρὸς Μωυσῆν, Καὶ τοῦτόν σοι τὸν λόγον, ὃν εἴρηκας ποιήσω· εὕρηκας, ποιήσω· εὕρηκας γὰρ χάριν ἐνώπιον ἐμοῦ, καὶ οἶδά σε παρὰ πάντας.
\vs{18}Καὶ λέγει, ἐμφάνισόν μοι σεαυτόν.
\vs{19}Καὶ εἶπεν, ἐγὼ παρελεύσομαι πρότερός σου τῇ δόξῃ μου, καὶ καλέσω τῷ ὀνόματί μου, Κύριος ἐναντίον σου· καὶ ἐλεήσω, ὃν ἂν ἐλεῶ, καὶ οἰκτειρήσω, ὃν ἂν οἰκτείρῶ.
\vs{20}Καὶ εἶπε, οὐ δυνήσῃ ἰδεῖν τὸ πρόσωπόν μου· οὐ γὰρ μὴ ἴδῃ ἄνθρωπος τὸ πρόσωπόν μου, καὶ ζήσεται.
\vs{21}Καὶ εἶπεν Κύριος, Ἰδοὺ τόπος παρʼ ἐμοί, στήσῃ ἐπὶ τῆς πέτρας·
\vs{22}Ἡνίκα δʼ ἂν παρέλθηνᾑ ᾑ δόξα μου, καί θήσω σε εἰς ὀπὴν τῆς πέτρας, καὶ σκεπάσω τῇ χειρί μου ἐπὶ σὲ, ἕως ἂν παρέλθω.
\vs{23}Καὶ ἀφελῶ τὴν χεῖρα, καὶ τότε ὄψει τὰ ὀπίσω μου· τὸ δὲ πρόσωπόν μου οὐκ ὀφθήσεταί σοι.

\ch{34}
Καὶ εἶπε Κύριος πρὸς Μωυσῆν λαξευσον σεαυτῷ δύο πλάκας λιθίνας, καθὼς καὶ αἱ πρῶται, καὶ ἀνάβηθι πρὸς μὲ εἰς τὸ ὄρος, καὶ γράψω ἐπὶ τῶν πλακῶν τὰ ῥήματα ἃ ἦν ἐν ταῖς πλαξὶ ταῖς πρώταις, αἷς συνέτριψας.
\vs{2}Καὶ γίνου ἕτοιμος εἰς τὸ πρωί, καὶ ἀναβήσῃ ἐπὶ τὸ ὄρος τὸ Σινά, καὶ στήσῃ μοι ἐκεῖ ἐπʼ ἄκρου τοῦ ὄρους.
\vs{3}Καὶ μηδεὶς ἀναβήτω μετὰ σοῦ μηδὲ ὀφθήτω ἐν παντὶ τῷ ὄρει· καὶ τὰ πρόβατα καὶ βόες μὴ νεμέσθωσαν πλησίον τοῦ ὄρους ἐκείνου.
\vs{4}Καὶ ἐλάξευσε δύο πλάκας λιθίνας, καθάπερ καὶ αἱ πρῶται· καὶ ὀρθρίσας Μωυσῆς, ἀνέβη εἰς τὸ ὄρος τὸ Σινὰ, καθότι συνέταξεν αὐτῷ Κύριος· καὶ ἔλαβε Μωυσῆς τὰς δύο πλάκας τὰς λιθίνας.
\vs{5}Καὶ κατέβη Κύριος ἐν νεφέλῃ, καὶ παρέστη αὐτῷ ἐκεῖ, καὶ ἐκάλεσε τῷ ὀνόματι Κυρίου.
\vs{6}Καὶ παρῆλθε Κύριος πρὸ προσώπου αὐτοῦ, καὶ ἐκάλεσε κύριος ὁ Θεὸς οἰκτείρμων καὶ ἐλεήμων, μακρόθυμος καὶ πολυέλεος καὶ ἀληθινός,
\vs{7}καὶ δικαιοσύνην διατηρῶν καὶ ἔλεος εἰς χιλιάδας, ἀφαιρῶν ἀνομίας καὶ ἀδικίας καὶ ἁμαρτίας, καὶ οὐ καθαριεῖ τὸν ἔνοχον, ἐπάγων ἀνομίας πατέρων ἐπὶ τέκνα καὶ ἐπὶ τέκνα τέκνων ἐπὶ τρίτην καὶ τετάρτην γενεάν.
\vs{8}Καὶ σπεύσας Μωσῆς κύψας ἐπὶ τὴν γῆν προσεκύνησε·
\vs{9}καὶ εἶπεν, εἰ εὕρηκα χάριν ἐνώπιόν σου, συμπορευθήτω ὁ Κύριός μου μεθʼ ἡμῶν· ὁ λαὸς γὰρ σκληροτράχηλός ἐστι, καὶ ἀφελεῖς σὺ τὰς ἁμαρτίας ἡμῶν, καὶ τὰς ἀνομίας ἡμῶν, καὶ ἐσόμεθά σοι.

\vs{10}Καὶ εἶπε Κύριος πρὸς Μωυσῆν, Ἰδοὺ, ἐγὼ τίθημί σοι διαθήκην ἐνώπιον παντὸς τοῦ λαοῦ σοῦ, ποιήσω ἔνδοξα, ἃ οὐ γέγονεν ἐν πάσῃ τῇ γῇ, καὶ ἐν παντὶ ἔθνει· καὶ ὄψεται πᾶς ὁ λαὸς, ἐν οἷς εἶ σὺ, τὰ ἔργα Κυρίου, ὅτι θαυμαστά ἐστιν, ἃ ἐγὼ ποιήσω σοι.
\vs{11}Πρόσεχε σὺ πάντα ὅσα ἐγὼ ἐντέλλομαί σοι· ἰδοὺ ἐγὼ ἐκβάλλω πρὸ προσώπου ὑμῶν τὸν Ἀμοῤῥαῖον, καὶ Χαναναῖον, καὶ Φερεζαῖον, καὶ Χετταῖον, καὶ Εὑαῖον, καὶ Γεργεσαῖον, καὶ Ἰεβουσαῖον.
\vs{12}Πρόσεχε σεαυτῷ, μή ποτε θῇς διαθήκην τοῖς ἐγκαθημένοις ἐπὶ τῆς γῆς, εἰς ἣν εἰσπορεύῃ εἰς αὐτὴν, μή σοι γένηται πρόσκομμα ἐν ὑμῖν.
\vs{13}Τοὺς βωμοὺς αὐτῶν καθελεῖτε, καὶ τὰς στήλας αὐτῶν συντρίψετε, καὶ τὰ ἄλση αὐτῶν ἐκκόψετε, καὶ τὰ γλυπτὰ τῶν θεῶν αὐτῶν κατακαύσετε ἐν πυρί.
\vs{14}Οὐ γὰρ μὴ προσκυνήσητε θεοῖς ἑτέροις· ὁ γὰρ Κύριος ὁ Θεὸς, ζηλωτὸν ὄνομα, Θεὸς ζηλωτής ἐστι.
\vs{15}Μή ποτε θῇς διαθήκην τοῖς ἔγκαθημένοις ἐπὶ τῆς γῆς, καὶ ἐκπορνεύσωσιν ὀπίσω τῶν θεῶν αὐτῶν, καὶ θύσωσι τοῖς θεοῖς αὐτῶν, καὶ καλέσωσίν σε, καὶ φάγῃς τῶν αὐτῶν,
\vs{16}καὶ λάβῃς τῶν θυγατέρων αὐτῶν τοῖς υἱοῖς σου, καὶ τῶν θυγατέρων σου δῷς τοῖς υἱοῖς αὐτῶν, καὶ ἐκπορνεύσωσιν αἱ θυγατέρες σου ὀπίσω τῶν θεῶν αὐτῶν, καὶ ἐκπορνεύσωσιν οἱ υἱοί σου ὀπίσω τῶν θεῶν αὐτῶν.
\vs{17}Καὶ θεοὺς χωνευτοὺς οὐ ποιήσεις σεαυτῷ.
\vs{18}Καὶ τὴν ἑσρτὴν τῶν ἀζύμων φυλάξῃ· ἑπτὰ ἡμέρας φαγῃ ἄζυμα, καθάπερ ἐντέταλμαί σοι, εἰς τὸν καιρὸν ἐν μηνὶ τῶν νέων· ἐν γὰρ μηνὶ τῶν νέων ἐξῆλθες ἐξ Αἰγύπτου.
\vs{19}Πᾶν διανοῖγον μήτραν, ἐμοὶ τὰ ἀρσενικὰ, πᾶν πρωτότοκον μόσχου, καὶ πρωτότοκον προβάτου.
\vs{20}Καὶ πρωτότοκον ὑποζυγίου λυτρώσῃ προβάτῳ· ἐὰν δὲ μὴ λυτρώσῃ αὐτὸ, τιμὴν δώσεις. πᾶν πρωτότοκον τῶν υἱῶν σου λυτρώσῃ· οὐκ ὀφθήσῃ ἐνώπιόν μου κενός.

\vs{21}Ἓξ ἡμέρας ἐργᾷ, τῇ δὲ ἑβδόμῃ καταπαύσεις· τῷ σπόρῳ καὶ τῷ ἀμητῷ κατάπαυσις.
\vs{22}Καὶ ἑορτὴν ἑβδομάδων ποιήσεις μοι, ἀρχὴν θερισμοῦ πυροῦ· καὶ ἐορτὴν συναγωγῆς μεσοῦντος τοῦ ἐνιαυτοῦ.
\vs{23}Τρεῖς καιροὺς τοῦ ἐνιαυτοῦ ὀφθήσεται πᾶν ἀρσενικόν σου ἐνώπιον Κυρίου τοῦ Θεοῦ Ἰσραήλ.
\vs{24}Ὅταν γὰρ ἐκβάλω τὰ ἔθνη πρὸ προσώπου σου, καὶ πλατυνῶ τὰ ὅριά σου, οὐκ ἐπιθυμήσει οὐδεὶς τῆς γῆς σου, ἡνίκα ἂν ἀναβαίνῃς ὀφθῆναι ἐναντίον Κυρίου τοῦ Θεοῦ σου, τρεῖς καιροὺς τοῦ ἐνιαυτοῦ.
\vs{25}Οὐ σφάξεις ἐπὶ ζύμῃ αἷμα θυμιαμάτων μου, καὶ οὐ κοιμηθήσεται εἰς τὸ πρωῒ θύματα ἑορτῆς τοῦ πάσχα.
\vs{26}Τὰ πρωτογεννήματα τῆς γῆς σου θήσεις εἰς τὸν οἶκον Κυρίου τοῦ Θεοῦ σου· οὐχ ἑψήσεις ἄρνα ἐν γάλακτι μητρὸς αὐτοῦ.
\vs{27}Καὶ εἶπε Κύριος πρὸς Μωυσῆν, γράψον σεαυτῷ τὰ ῥήματα ταῦτα· ἐπὶ γὰρ τῶν λόγων τούτων τέθειμαί σοι διαθήκην, καὶ τῷ Ἰσραήλ.
\vs{28}Καὶ ἦν ἐκεῖ Μωυσῆς ἐναντίον Κυρίου τεσσεράκοντα ἡμέρας, καὶ τεσσεράκοντα νύκτας· ἄρτον οὐκ ἔφαγε, καὶ ὕδωρ οὐκ ἔπιε· καὶ ἔγραψεν ἐπὶ τῶν πλακῶν τὰ ῥήματα ταῦτα τῆς διαθήκης, τοὺς δέκα λόγους.

\vs{29}Ὡς δὲ κατέβαινε Μωυσῆς ἐκ τοῦ ὄρους, καὶ αἱ δύο πλάκες ἐπὶ τῶν χειρῶν Μωυσῆ· καταβαίνοντος δὲ αὐτοῦ ἐκ τοῦ ὄρους, Μωυσῆς οὐκ ᾔδει ὅτι δεδόξασται ἡ ὄψις τοῦ χρώματος τοῦ προσώπου αὐτοῦ ἐν τῷ λαλεῖν αὐτὸν αὐτῷ.
\vs{30}Καὶ εἶδεν Ἀαρὼν καὶ πάντες οἱ πρεσβύτεροι Ἰσραὴλ τὸν Μωυσῆν, καὶ ἦν δεδοξασμένη ἡ ὄψις τοῦ χρώματος τοῦ προσώπου αὐτοῦ. καὶ ἐφοβήθησαν ἐγγίσαι αὐτῷ.
\vs{31}Καὶ ἐκάλεσεν αὐτοὺς Μωυσῆς, καὶ ἐπεστράφησαν πρὸς αὐτὸν Ἀαρὼν καὶ πάντες οἱ ἄρχοντες τῆς συναγωγῆς· καὶ ἐλάλησεν αὐτοῖς Μωυσῆς.

\vs{32}Καὶ μετὰ ταῦτα προσῆλθον πρὸς αὐτὸν πάντες οἱ υἱοὶ Ἰσραήλ. καὶ ἐνετείλατο αὐτοῖς πάντα, ὅσα ἐνετείλατο Κύριος πρὸς αὐτὸν ἐν τῷ ὄρει Σινά.
\vs{33}Καὶ ἐπειδὴ κατέπαυσε λαλῶν πρὸς αὐτοὺς, ἐπέθηκεν ἐπὶ τὸ πρόσωπον αὐτοῦ κάλυμμα.
\vs{34}Ἡνίκα δʼ ἂν εἰσεπορεύετο Μωυσῆς, ἔναντι Κυρίου λαλεῖν αὐτῷ, περιῃρεῖτο τὸ κάλυμμα ἕως τοῦ ἐκπορεύεσθαι· καὶ ἐξελθὼν ἐλάλει πᾶσι τοῖς υἱοῖς Ἰσραὴλ ὅσα ἐνετείλατο αὐτῷ Κύριος.
\vs{35}Καὶ εἶδον οἱ υἱοὶ Ἰσραὴλ τὸ πρόσωπον Μωυσέως, ὅτι δεδόξασται· καὶ περιέθηκε Μωυσῆς κάλυμμα ἐπὶ τὸ πρόσωπον ἑαυτοῦ, ἕως ἂν εἰσέλθῃ συλλαλεῖν αὐτῷ.

\ch{35}
Καὶ συνήθροισε Μωυσῆς πᾶσαν συναγωγὴν υἱῶν Ἰσραὴλ, καὶ εἶπεν, οὗτοι οἱ λόγοι, οὓς εἶπε Κύριος ποιῆσαι αὐτούς.
\vs{2}Ἓξ ἡμέρας ποιήσεις ἔργα, τῇ δὲ ἡμέρᾳ τῇ ἑβδόμῃ κατάπαυσις· ἅγια, σάββατα· ἀνάπαυσις Κυρίῳ· πᾶς ὁ ποιῶν ἔργον ἐν αὐτῇ, τελευτάτω.
\vs{3}Οὐ καύσετε πῦρ ἐν πάσῃ κατοικίᾳ ὑμῶν τῇ ἡμέρᾳ τῶν σαββάτων· ἐγὼ Κύριος.
\vs{4}Καὶ εἶπε Μωυσῆς πρὸς πᾶσαν συναγωγὴν υἱῶν Ἰσραὴλ, λέγων, τοῦτο τὸ ῥῆμα, ὃ συνέταξε Κύριος, λέγων,
\vs{5}λάβετε παρʼ ὑμῶν αὐτῶν ἀφαίρεμα Κυρίῳ· πᾶς ὁ καταδεχόμενος τῇ καρδίᾳ, οἴσουσι τὰς ἀπαρχὰς Κυρίῳ, χρυσίον, ἀργύριον, χαλκὸν,
\vs{6}ὑάκινθον, πορφύραν, κόκκινον διπλοῦν διανενησμένον, καὶ βύσσον κεκλωσμένην, καὶ τρίχας αἰγείας,
\vs{7}καὶ δέρματα κριῶν ἠρυθροδανωμένα, καὶ δέρματα ὑακινθινα, καὶ ξύλα ἄσηπτα,
\vs{9}καὶ λίθους σαρδίου, καὶ λίθους εἰς τὴν γλυφὴν εἰς τὴν ἐπωμίδα καὶ τὸν ποδήρη.
\vs{10}Καὶ πᾶς σοφὸς τῇ καρδίᾳ ἐν ὑμῖν, ἐλθὼν ἐργαζέσθω πάντα ὅσα συνέταξε Κύριος·
\vs{11}Τὴν σκηνὴν, καὶ τὰ παραρύματα, καὶ τὰ κατακαλύμματα, καὶ τὰ διατόνια, καὶ τοὺς μοχλοὺς, καὶ τοὺς στύλους,
\vs{12}καὶ τὴν κιβωτὸν τοῦ μαρτυρίου, καὶ τοὺς ἀναφορεῖς αὐτῆς, καὶ τὸ ἱλαστήριον αὐτῆς, καὶ τὸ καταπέτασμα,
\vs{12a}καὶ τὰ ἱστία τῆς αὐλῆς, καὶ τοὺς στύλους αὐτῆς, καὶ τοὺς λίθους τοὺς τῆς σμαράγδου, καὶ τὸ θυμίαμα, καὶ τὸ ἔλαιον τοῦ χρίσματος,
\vs{13}καὶ τὴν τράπεζαν καὶ πάντα τὰ σκεύη αὐτῆς,
\vs{14}καὶ τὴν λυχνίαν τοῦ φωτὸς καὶ πάντα τὰ σκεύη αὐτῆς,
\vs{16}καὶ τὸ θυσιαστήριον καὶ πάντα τὰ σκεύη αὐτοῦ,
\vs{19}καὶ τὰς στολὰς τὰς ἁγίας Ἀαρὼν τοῦ ἱερέως, καὶ τὰς στολὰς ἐν αἷς λειτουργήσουσιν ἐν αὐταῖς, καὶ τοὺς χιτῶνας τοῖς υἱοῖς Ἀαρὼν τῆς ἱερατίας, καὶ τὸ ἔλαιον τοῦ χρίσματος, καὶ τὸ θυμίαμα τῆς συνθέσεως.

\vs{20}Καὶ ἐξῆλθε πᾶσα συναγωγὴ υἱῶν Ἰσραὴλ ἀπὸ Μωυσῆ.
\vs{21}Καὶ ἤνεγκαν ἕκαστος, ὧν ἔφερεν ἡ καρδία αὐτῶν, καὶ ὅσοις ἔδοξε τῇ ψυχῇ αὐτῶν, ἀφαίρεμα· καὶ ἤνεγκαν ἀφαίρεμα Κυρίῳ εἰς πάντα τὰ ἔργα τῆς σκηνῆς τοῦ μαρτυρίου, καὶ εἰς πάντα τὰ κάτεργα αὐτῆς καὶ εἰς πάσας τὰς στολὰς τοῦ ἁγίου.
\vs{22}Καὶ ἤνεγκαν οἱ ἄνδρες παρὰ τῶν γυναικῶν, πᾶς ᾧ ἔδοξε τῇ διανοίᾳ, ἤνεγκαν σφραγίδας, καὶ ἐνώτια, καὶ δακτυλίους, καὶ ἐμπλόκια, καὶ περιδέξια, πᾶν σκεῦος χρυσοῦν.
\vs{23}Καὶ πάντες ὅσοι ἤνεγκαν ἀφαιρέματα χρυσίου Κυρίῳ, καὶ παρʼ ᾧ εὑρέθη βύσσος· καὶ δέρματα ὑακίνθινα καὶ δέρματα κριῶν ἠρυθροδανωμένα ἤνεγκαν.
\vs{24}Καὶ πᾶς ὁ ἀφαιρῶν ἀφαίρεμα, ἤνεγκαν ἀργύριον καὶ χαλκὸν, τὰ ἀφαιρέματα Κυρίῳ· καὶ παρʼ οἷς εὑρέθη ξύλα ἄσηπτα· καὶ εἰς πάντα τὰ ἔργα τῆς παρασκευῆς ἤνεγκαν.
\vs{25}Καὶ πᾶσα γυνὴ σοφὴ τῇ διανοίᾳ ταῖς χερσὶ νήθειν, ἤνεγκαν νενησμένα, τὴν ὑάκινθον, καὶ τὴν πορφύραν, καὶ τὸ κόκκινον, καὶ τὴν βύσσον.
\vs{26}Καὶ πᾶσαι αἱ γυναῖκες, αἷς ἔδοξε τῇ διανοίᾳ αὐτῶν ἐν σοφίᾳ, ἔνησαν τὰς τρίχας τὰς αἰγείας.
\vs{27}Καὶ οἱ ἄρχοντες ἤνεγκαν τοὺς λίθους τῆς σμαράγδου, καὶ τοὺς λίθους τῆς πληρώσεως εἰς τὴν ἐπωμίδα, καὶ τὸ λογεῖον,
\vs{28}καὶ τὰς συνθέσεις, καὶ εἰς τὸ ἔλαιον τῆς χρίσεως, καὶ τὴν συνθεσιν τοῦ θυμιάματος.
\vs{29}Καὶ πᾶς ἀνὴρ καὶ γυνὴ, ὧν ἔφερεν ἡ διάνοια αὐτῶν εἰσελθόντας ποιεῖν πάντα τὰ ἔργα, ὅσα συνέταξε Κύριος ποιῆσαι αὐτὰ διὰ Μωυσῆ, ἤνεγκαν οἱ υἱοὶ Ἰσραὴλ, ἀφαίρεμα Κυρίῳ.
\vs{30}Καὶ εἶπε Μωυσῆς τοῖς υἱοῖς Ἰσραὴλ, Ἰδοὺ ἀνακέκληκεν ὁ Θεὸς ἐξ ὀνόματος τὸν Βεσελεὴλ τὸν τοῦ Οὐρείου τὸν Ὢρ, ἐκ τῆς φυλῆς Ἰούδα,
\vs{31}καὶ ἐνέπλησεν αὐτὸν πνεῦμα θεῖον σοφίας καὶ συνέσεως, καὶ ἐπιστήμης πάντων,
\vs{32}ἀρχιτεκτονεῖν κατὰ πάντα τὰ ἔργα τῆς ἀρχιτεκτονίας, ποιεῖν τὸ χρυσίον καὶ τὸ ἀργύριον καὶ τὸν χαλκόν,
\vs{33}καὶ λιθουργῆσαι τὸν λίθον, καὶ κατεργάζεσθαι τὰ ξύλα, καὶ ποιεῖν ἐν παντὶ ἔργῳ σοφίας.
\vs{34}Καὶ προβιβάσαι γε ἔδωκεν ἐν τῇ διανοίᾳ αὐτῷ τε, καὶ τῷ Ἐλιὰβ τῷ τοῦ Ἀχισαμὰκ, ἐκ φυλῆς Δάν·
\vs{35}Καὶ ἐνέπλησεν αὐτοὺς σοφίας, συνέσεως, διανοίας, πάντα συνιέναι ποιῆσαι τὰ ἔργα τοῦ ἁγίου, καὶ τὰ ὑφαντὰ καὶ ποικιλτὰ ὑφάναι τῷ κοκκίνῳ, καὶ τῇ βύσσῳ, ποιεῖν πᾶν ἔργον ἀρχιτεκτονίας, ποικιλίας.

\ch{36}
Καὶ ἐποίησε Βεσελεὴλ καὶ Ἐλιὰβ, καὶ πᾶς σοφὸς τῇ διανοὶᾳ, ᾧ ἐδόθη σοφία καὶ ἐπιστήμη ἐν αὑτοῖς, συνιέναι ποιεῖν πάντα τὰ ἔργα, κατὰ τὰ ἅγια καθήκοντα, κατὰ πάντα ὅσα συνέταξε Κύριος.
\vs{2}Καὶ ἐκάλεσε Μωυσῆς Βεσελεὴλ καὶ Ἐλιὰβ, καὶ πάντας τοὺς ἔχοντας τὴν σοφίαν, ᾧ ἔδωκεν ὁ Θεὸς ἐπιστήμην ἐν τῇ καρδίᾳ, καὶ πάντας τοὺς ἑκουσίως βουλομένους προσπορεύεσθαι πρὸς τὰ ἔργα, ὥστε συντελεῖν αὐτά.
\vs{3}Καὶ ἔλαβον παρὰ Μωσῆ πάντα τὰ ἀφαιρέματα, ἃ ἤνεγκαν οἱ υἱοὶ Ἰσραὴλ εἰς πάντα τὰ ἔργα τοῦ ἁγίου ποιεῖν αὐτά· καὶ αὐτοὶ προσεδέχοντο ἔτι τὰ προσφερόμενα παρὰ τῶν φερόντων τὸ πρωΐ.
\vs{4}Καὶ παρεγίνοντο πάντες οἱ σοφοὶ οἱ ποιοῦντες τὰ ἔργα τοῦ ἁγίου, ἕκαστος κατὰ τὸ αὐτοῦ ἔργον, ὃ εἰργάζοντο αὐτοί.
\vs{5}Καὶ εἶπε πρὸς Μωυσῆν, ὅτι πλῆθος φέρει ὁ λαὸς κατὰ τὰ ἔργα ὅσα συνέταξε Κύριος ποιῆσαι.
\vs{6}Καὶ προσέταξε Μωυσῆς, καὶ ἐκήρυξεν ἐν τῇ παρεμβολῇ, λέγων, ἀνὴρ καὶ γυνὴ μηκέτι ἐργαζέσθωσαν εἰς τάς ἀπαρχὰς τοῦ ἁγίου· καὶ ἐκωλύθη ὁ λαὸς ἔτι προσφέρειν.
\vs{7}Καὶ τὰ ἔργα ἦν αὐτοῖς ἱκανὰ εἰς τὴν κατασκευὴν ποιῆσαι, καὶ προσκατέλιπον.
\vs{8}Καὶ ἐποίησε πᾶς σοφὸς ἐν τοῖς ἐργαζομένοις τὰς στολὰς τῶν ἁγίων, αἵ εἰσιν Ἀαρὼν τῷ ἱερεῖ, καθὰ συνέταξε Κύριος τῷ Μωυσῇ.
\vs{9}Καὶ ἐποίησε τὴν ἐπωμίδα ἐκ χρυσίου, καὶ ὑακίνθου, καὶ πορφύρας, καὶ κοκκίνου νενησμένου, καὶ βύσσου κεκλωσμένης·
\vs{10}καὶ ἐτμήθη τὰ πέταλα τοῦ χρυσίου τρίχες, ὥστε συνυφάναι σὺν τῇ ὑακίνθῳ, καὶ τῇ πορφύρᾳ, καὶ σὺν τῷ κοκκίνῳ τῷ διανενησμένῳ, καὶ τῇ βύσσῳ τῇ κεκλωσμένῃ· ἔργον ὑφαντὸν ἐποίησαν αὐτό·
\vs{11}ἐπωμίδας συνεχούσας ἐξ ἀμφοτέρων τῶν μερῶν, ἔργον ὑφαντὸν εἰς ἄλληλα συμπεπλεγμένα καθʼ ἑαυτό.
\vs{12}Ἐξ αὐτοῦ ἐποίησαν αὐτὸ κατὰ τὴν αὐτοῦ ποίησιν, ἐκ χρυσίου, καὶ ὑακίνθου, καὶ πορφύρας, καὶ κοκκίνου διανενησμένου, καὶ βύσσου κεκλωσμένης, καθὰ συνέταξε Κύριος τῷ Μωυσῇ·
\vs{13}καὶ ἐποίησαν ἀμφοτέρους τοὺς λίθους τῆς σμαράγδου συνπεπορπημένους καὶ περισεσιαλωμένους χρυσίῳ, γεγλυμμένους καὶ ἐκκεκολαμμένους ἐγκόλαμμα σφραγίδος ἐκ τῶν ὀνομάτων τῶν υἱῶν Ἰσραήλ·
\vs{14}καὶ ἐπέθηκεν αὐτοὺς ἐπὶ τοὺς ὤμους τῆς ἐπωμίδος, λίθους μνημοσύνου τῶν υἱῶν Ἰσραὴλ, καθὰ συνέταξε Κύριος τῷ Μωυσῇ.

\vs{15}Καὶ ἐποίησαν λογεῖον, ἔργον ὑφαντὸν ποικιλίᾳ κατὰ τὸ ἔργον τῆς ἐπωμίδος, ἐκ χρυσίου, καὶ ὑακίνθου, καὶ πορφύρας, καὶ κοκκίνου διανενησμένου, καὶ βύσσου κεκλωσμένης·
\vs{16}τετράγωνον διπλοῦν ἐπόησαν τὸ λογεῖον· σπιθαμῆς τὸ μῆκος, καὶ σπιθαμῆς τὸ εὖρος διπλοῦν.
\vs{17}Καὶ συνυφάνθη ἐν αὐτῷ ὕφασμα κατάλιθον τετράστιχον· στίχος λίθων, σάρδιον καὶ τοπάζιον καὶ σμάραγδος, ὁ στίχος ὁ εἷς·
\vs{18}καὶ ὁ στίχος ὁ δεύτερος, ἄνθραξ καὶ σάπφειρος καὶ ἴασπις·
\vs{19}καὶ ὁ στίχος ὁ τρίτος, λιγύριον καὶ ἀχάτης καὶ ἀμέθυστος·
\vs{20}καὶ ὁ στίχος ὁ τέταρτος, χρυσόλιθος καὶ βηρύλλιον καὶ ὀνύχιον περικεκυκλωμένα χρυσίῳ, καὶ συνδεδεμένα χρυσίῳ.
\vs{21}Καὶ οἱ λίθοι ἦσαν ἐκ τῶν ὀνομάτων τῶν υἱῶν Ἰσραὴλ δώδεκα, ἐκ τῶν ὀνομάτων αὐτῶν ἐγγεγλυμμένα εἰς σφραγίδας, ἕκαστος ἐκ τοῦ ἑαυτοῦ ὀνόματος εἰς τὰς δώδεκα φυλάς.
\vs{22}Καὶ ἐποίησαν ἐπὶ τὸ λογεῖον κρωσσοὺς συμπεπλεγμένους, ἔργον ἐμπλοκίου, ἐκ χρυσίου καθαροῦ.
\vs{23}Καὶ ἐποίησαν δύο ἀσπιδίσκας χρυσᾶς, καὶ δύο δακτυλίους χρυσοῦς· καὶ ἐπέθηκαν τοὺς δύο δακτυλίους τοὺς χρυσοῦς ἐπʼ ἀμφοτέρας τὰς ἀρχὰς τοῦ λογείου.
\vs{24}Καὶ ἐπέθηκαν τὰ ἐμπλόκια ἐκ χρυσίου ἐπὶ τοὺς δακτυλίους ἐπʼ ἀμφοτέρων τῶν μερῶν τοῦ λογείου·
\vs{25}καὶ εἰς τὰς δύο συμβολὰς τὰ δύο ἐμπλόκια. Καὶ ἐπέθηκαν ἐπὶ τὰς δύο ἀσπιδίσκας· καὶ ἐπέθηκαν ἐπὶ τοὺς ὤμους τῆς ἐπωμίδος ἐξεναντίας κατὰ πρόσωπον.
\vs{26}Καὶ ἐποίησαν δύο δακτυλίους χρυσοῦς, καὶ ἐπέθηκαν ἐπὶ τὰ δύο πτερύγια ἐπʼ ἄκρου τοῦ λογείου, καὶ ἐπὶ τὸ ἄκρον τοῦ ὀπισθίου τῆς ἐπωμίδος ἔσωθεν·
\vs{27}Καὶ ἐποίησαν δύο δακτυλίους χρυσοῦς, καὶ ἐπέθηκαν ἐπʼ ἀμφοτέρους τοὺς ὤμους τῆς ἐπωμίδος κάτωθεν αὐτοῦ, κατὰ πρόσωπον κατὰ τὴν συμβολὴν ἄνωθεν τῆς συνυφῆς τῆς ἐπωμίδος·
\vs{28}καὶ συνέσφιγξε τὸ λογεῖον ἀπὸ τῶν δακτυλίων τῶν ἐπʼ αὐτοῦ εἰς τοὺς δακτυλίους τῆς ἐπωμίδος, συνεχομένους ἐκ τῆς ὑακίνθου, συμπεπλεγμένους εἰς τὸ ὕφασμα τῆς ἐπωμίδος, ἵνα μὴ χαλᾶται τὸ λογεῖον ἀπὸ τῆς ἐπωμίδος, καθὰ συνέταξε Κύριος τῷ Μωυσῇ.
\vs{29}Καὶ ἐποίησαν τὸν ὑποδύτην ὑπὸ τὴν ἐπωμίδα, ἔργον ὑφαντὸν, ὅλον ὑακίνθινον·
\vs{30}τὸ δὲ περιστόμιον τοῦ ὑποδύτου ἐν τῷ μέσῳ διυφασμένον συμπλεκτὸν, ὤαν ἔχον κύκλῳ τὸ περιστόμιονν ἀδιάλυτον·
\vs{31}Καὶ ἐποίησαν ἐπὶ τοῦ λώματος τοῦ ὑποδύτου κάτωθεν ὡς ἐξανθούσης ῥόας ῥοΐσκους, ἐξ ὑακίνθου, καὶ πορφύρας, καὶ κοκκίνου νενησμένου, καὶ βύσσου κεκλωσμένης.
\vs{32}Καὶ ἐποίησαν κώδωνας χρυσοῦς, καὶ ἐπέθηκαν τοὺς κώδωνας ἐπὶ τὸ λῶμα τοῦ ὑποδύτου κύκλῳ ἀνὰ μέσον τῶν ῥοΐσκων·
\vs{33}κώδων χρυσοῦς καὶ ῥοΐσκος ἐπὶ τοῦ λώματος τοῦ ὑποδύτου κύκλῳ, εἰς τὸ λειτουργεῖν, καθὰ συνέταξε Κύριος τῷ Μωυσῇ.
\vs{34}Καὶ ἐποίησαν χιτῶνας βυσσίνους, ἔργον ὑφαντὸν, Ἀαρὼν καὶ τοῖς υἱοῖς αὐτοῦ,
\vs{35}καὶ τὰς κιδάρεις ἐκ βύσσου, καὶ τὴν μίτραν ἐκ βύσσου, καὶ τὰ περισκελῆ ἐκ βύσσου κεκλωσμένης,
\vs{36}καὶ τὰς ζώνας αὐτῶν ἐκ βύσσου, καὶ ὑακίνθου, καὶ πορφύρας, καὶ κοκκίνου νενησμένου, ἔργον ποικιλτοῦ, ὃν τρόπον συνέταξε Κύριος τῷ Μωυσῇ.
\vs{37}Καὶ ἐποίησαν τὸ πέταλον τὸ χρυσοῦν, ἀφόρισμα τοῦ ἁγίου, χρυσίου καθαροῦ· καὶ ἔγραψεν ἐπʼ αὐτοῦ γράμματα ἐκτετυπωμένα, σφραγίδος, Ἁγίασμα Κυρίῳ·
\vs{38}Καὶ ἐπέθηκαν ἐπὶ τὸ λῶμα ὑακίνθινον, ὥστε ἐπικεῖσθαι ἐπὶ τὴν μίτραν ἄνωθεν, ὅν τρόπον συνέταξε Κύριος τῷ Μωυσῇ.

\ch{37}
Καὶ ἐποίησαν τῇ σκηνῇ δέκα αὐλαίας·
\vs{2}ὀκτὼ καὶ εἴκοσι πήχεων μῆκος τῆς αὐλαίας τῆς μιᾶς· τὸ αὐτὸ ἦν πάσαις· καὶ τεσσάρων πήχεων τὸ εὖρος τῆς αὐλαίας τῆς μιᾶς.
\vs{3}καὶ ἐποίησαν τὸ καταπέτασμα ἐξ ὑακίνθου, καὶ πορφύρας, καὶ κοκκίνου νενησμένου, καὶ βύσσου κεκλωσμένης, ἔργον ὑφαντὸυ χερουβείμ·
\vs{4}καὶ ἐπέθηκαν αὐτὸ ἐπὶ τέσσαρας στύλους ἀσήπτους κατακεχρυσωμένους ἐν χρυσίῳ· καὶ αἱ κεφαλίδες αὐτῶν χρυσαῖ, καὶ αἱ βάσεις αὐτῶν τέσσαρες ἀργυραῖ.
\vs{5}καὶ ἐποίησαν τὸ καταπέτασμα τῆς θύρας τῆς σκηνῆς τοῦ μαρτυρίου ἐξ ὑακίνθου, καὶ πορφύρας, καὶ κοκκίνου νενησμένου, καὶ βύσσου κεκλωσμένης, ἔργον ὑφαντὸντοῦ χερουβείμ·
\vs{6}καὶ τοὺς στύλους αὐτῶν πέντε, καὶ τοὺς κρίκους· καὶ τὰς κεφαλίδας αὐτῶν, καὶ τὰς ψαλίδας αὐτῶν κατεχρύσωσαν χρυσίῳ· καὶ αἱ βάσεις αὐτῶν πέντε χαλκαῖ.

\vs{7}Καὶ ἐποίησαν τὴν αὐλῆν τὰ πρὸς Λίβα, ἱστία τῆς αὐλῆς ἐκ βύσσου κεκλωσμένης ἑκατὸν ἐφʼ ἑκατόν·
\vs{8}καὶ οἱ στύλοι αὐτῶν εἴκοσι, καὶ αἱ βάσεις αὐτῶν εἴκοσι.
\vs{9}καὶ τὸ κλίτος τὸ πρὸς Βοῤῥᾶν, ἑκατὸν ἐφʼ ἑκατόν· καὶ τὸ κλίτος τὸ πρὸς Νότον, ἑκατὸν ἐφʼ ἑκατόν· καὶ οἱ στύλοι αὐτῶν εἴκοσι, καὶ αἱ βάσεις αὐτῶν εἴκοσι·
\vs{10}Καὶ τὸ κλίτος τὸ πρὸς θάλασσαν αὐλαῖαι πεντήκοντα πήχεων· στύλοι αὐτῶν δέκα, καὶ αἱ βάσεις αὐτῶν δέκα·
\vs{11}Καὶ τὸ κλίτος τὸ πρὸς ἀνατολὰς πεντήκοντα πήχεων ἱστία, πεντεαίδεκα πήχεων τὸ κατὰ νώτου·
\vs{12}καὶ οἱ στύλοι αὐτῶν τρεῖς, καὶ αἱ βάσεις αὐτῶν τρεῖς·
\vs{13}Καὶ ἐπὶ τοῦ νώτου τοῦ δευτέρου ἔνθεν καὶ ἔνθεν κατὰ τὴν πύλην τῆς αὐλῆς, αὐλαῖαι πεντεκαίδεκα πήχεων· στύλοι αὐτῶν τρεῖς, καὶ αἱ βάσεις αὐτῶν τρεῖς·
\vs{14}πᾶσαι αἱ αὐλαῖαι τῆς σκηνῆς ἐκ βύσσου κεκλωσμένης.
\vs{15}Καὶ αἱ βάσεις τῶν στύλων αὐτῶν χαλκαῖ, καὶ αἱ ἀγκύλαι αὐτῶν ἀργυραῖ, καὶ αἱ κεφαλίδες αὐτῶν περιηργυρωμέναι ἀργυρίῳ, καὶ οἱ στύλοι περιηργυρωμένοι ἀργυρίῳ πάντες οἱ στύλοι τῆς αὐλῆς·
\vs{16}καὶ τὸ καταπέτασμα τῆς πύλης τῆς αὐλῆς ἔργον ποικιλτοῦ ἐξ ὑακίνθου, καὶ πορφύρας, καὶ κοκκίνου νενησμένου, καὶ βύσσου κεκλωσμένης· εἴκοσι πήχεων τὸ μῆκος, καὶ τὸ ὕψος καὶ τὸ εὖρος πέντε πήχεων ἐξισούμενον τοῖς ἱστίοις τῆς αὐλῆς·
\vs{17}καὶ οἱ στύλοι αὐτῶν τέσσαρες, καὶ αἱ βάσεις αὐτῶν τέσσαρες χαλκαῖ, καὶ αἱ ἀγκύλαι αὐτῶν ἀργυραῖ, καὶ αἱ κεφαλίδες αὐτῶν περιηργυρωμέναι ἀργυρίῳ.
\vs{18}Καὶ πάντες οἱ πάσσαλοι τῆς αὐλῆς κύκλῳ χαλκοῖ, καὶ αὐτοὶ περιηργυρωμένοι ἀργυρίῳ.
\vs{19}Καὶ αὕτη ἡ σύνταξις τῆς σκηνῆς τοῦ μαρτυρίου, καθὰ συνετάγη Μωυσῇ, τὴν λειτουργίαν εἶναι τῶν Λευιτῶν διὰ Ἰθάμαρ τοῦ υἱοῦ Ἀαρὼν τοῦ ἱερέως.

\vs{20}Καὶ Βεσελεὴλ ὁ τοῦ Οὐρείου, ἐκ φυλῆς Ἰούδα, ἐποίησε καθὰ συνέταξε Κύριος τῷ Μωυσῇ,
\vs{21}καὶ Ἐλιὰβ ὁ τοῦ Ἀχισαμὰχ ἐκ φυλῆς Δὰν, ὅς ἠρχιτεκτόνησε τὰ ὑφαντὰ καὶ τὰ ῥαφιδευτὰ καὶ ποικιλτικά, ὑφάναι τῷ κοκκίνῳ καὶ τῇ βύσσῳ.

\ch{38}
Καὶ ἐποίησε Βεσελεὴλ τὴν κιβωτόν,
\vs{2}καὶ κατεχρύσωσεν αὐτὴν χρωσίῳ καθαρῷ ἔσωθεν καὶ ἔξωθεν·
\vs{3}καὶ ἐχώνευσεν αὐτῇ τέσσαρας δακτυλίους χρυσοῦς· δύο ἐπὶ τὸ κλίτος τὸ ἓν, καὶ δύο ἐπὶ τὸ κλίτος τὸ δεύτερον,
\vs{4}εὐρεῖς τοῖς διωστῆρσιν, ὥστε αἴρειν αὐτὴν ἐν αὐτοῖς.
\vs{5}Καὶ ἐποίησε τὸ ἱλαστήριον ἐπάνωθεν τῆς κιβωτοῦ ἐκ χρυσίου καθαροῦ,
\vs{6}καὶ τοὺς δύο χερουβεὶμ χρυσοῦς·
\vs{7}χεροὺβ ἕνα ἐπὶ τὸ ἄκρον τοῦ ἱλαστηρίου τὸ ἓν, καὶ χεροὺβ ἕνα ἐπὶ τὸ ἄκρον τοῦ ἱλαστηρίου τὸ δεύτερον,
\vs{8}σκιάζοντα ταῖς πτέρυξιν αὐτῶν ἐπὶ τὸ ἱλαστήριον.
\vs{9}Καὶ ἐποίησε τὴν τράπεζαν τὴν προκειμένην ἐκ χρυσίου καθαροῦ,
\vs{10}καὶ ἐχώνευσεν αὐτῇ τέσσαρας δακτυλίους, δύο ἐπὶ τοῦ κλίτους τοῦ ἑνὸς, καὶ δύο ἐπὶ τοῦ κλίτους τοῦ δευτέρου, εὐρεῖς, ὥστε αἴρειν τοῖς διωστῆρσιν ἐν αὐτοῖς.
\vs{11}Καὶ τοὺς διωστῆρας τῆς κιβωτοῦ καὶ τῆς τραπέζης ἐποίησε, καὶ κατεχρύσωσεν αὐτοὺς χρυσίῳ.
\vs{12}Καὶ ἐποίησε τὰ σκεύη τῆς τραπέζης, τά τε τρυβλία, καὶ τὰς θυίσκας, καὶ τοὺς κυάθους, καὶ τὰ σπονδεῖα, ἐν οἷς σπείσει ἐν αὐτοῖς, χρυσᾶ.
\vs{13}Καὶ ἐποίησε τὴν λυχνίαν ἣ φωτίζει, χρυσῆν,
\vs{14}στερεὰν τὸν καυλόν, καὶ τοὺς καλαμίσκους ἐξ ἀμφοτέρων τῶν μερῶν αὐτῆς·
\vs{15}ἐκ τῶν καλαμίσκων αὐτῆς οἱ βλαστοὶ ἐκ τῶν καλαμίσκων αὐτῆς οἱ βλαστοὶ ἐξέχοντες· τρεῖς ἐκ τούτου, καὶ τρεῖς ἐκ τούτου, ἐξισούμενοι ἀλλήλοις.
\vs{16}Καὶ τὰ λαμπάδια αὐτῶν, ἅ ἐστιν ἐπὶ τῶν ἄκρων, καρυωτὰ ἐξ αὐτῶν· καὶ τὰ ἐνθέμια ἐξ αὐτῶν, ἵνα ὦσιν οἱ λύχνοι ἐπʼ αὐτῶν· καὶ τὸ ἐνθέμιον τὸ ἕβδομον, τὸ ἐπʼ ἄκρου τοῦ λαμπαδίου, ἐπὶ τῆς κορυφῆς ἄνωθεν, στερεὸν ὅλον χρυσοῦν·
\vs{17}Καὶ ἑπτὰ λύχνους ἐπʼ αὐτῆς χρυσοῦς, καὶ τὰς λαβίδας αὐτῆς χρυσᾶς, καὶ τὰς ἐπαρυστρίδας αὐτῶν χρυσᾶς.
\vs{18}Οὗτος περιηργύρωσε τοὺς στύλους, καὶ ἐχώνευσε τῷ στύλῳ δακτυλίους χρυσοῦς, καὶ ἐχρύσωσε τοὺς μοχλοὺς χρυσίῳ, καὶ κατεχρύσωσε τοὺς στύλους τοῦ καταπετάσματος χρυσίῳ, καὶ ἐποίησε τὰς ἀγκύλας χρυσᾶς.
\vs{19}Οὗτος ἐποίησε καὶ τοὺς κρίκους τῆς σκηνῆς χρυσοῦς, καὶ τοὺς κρίκους τῆς αὐλῆς, καὶ κρίκους εἰς τὸ ἐκτείνειν τὸ κατακάλυμμα ἄνωθεν χαλκοῦς·
\vs{20}Οὗτος ἐχώνευσε τὰς κεφαλίδας τὰς ἀργυρᾶς τῆς σκηνῆς, καὶ τὰς κεφαλίδας τὰς χαλκᾶς τῆς θύρας τῆς σκηνῆς, καὶ τὴν πύλην τῆς αὐλῆς· καὶ ἀγκύλας ἐποίησε τοῖς στύλοις ἀργυρᾶς, ἐπὶ τῶν στύλων οὗτος περιηργύρωσεν αὐτάς·
\vs{21}Οὗτος ἐποίησε τοὺς πασσάλους τῆς σκηνῆς, καὶ τοὺς πασσάλους τῆς αὐλῆς χαλκοῦς·
\vs{22}Οὗτος ἐποίησε τὸ θυσιαστήριον τὸ χαλκοῦν ἐκ τῶν πυρείων τῶν χαλκῶν, ἃ ἦσαν τοῖς ἀνδράσιν τοῖς καταστασιάσασι μετὰ τῆς Κορὲ συναγωγῆς·
\vs{23}Οὗτος ἐποίησε πάντα τὰ σκεύη τοῦ θυσιαστηρίου, καὶ τὸ πυρεῖον αὐτοῦ, καὶ τὴν βάσιν, καὶ τὰς φιάλας, καὶ τὰς κρεάγρας τὰς χαλκᾶς·
\vs{24}Οὗτος ἐποίησε θυσιαστηρίῳ παράθεμα, ἔργον δικτυωτὸν κάτωθεν τοῦ πυρείου ὑπὸ αὐτὸ ἕως τοῦ ἡμίσους αὐτοῦ· καὶ ἐπέθηκεν αὐτῷ τέσσαρας δακτυλίους ἐκ τῶν τεσσάρων μερῶν τοῦ παραθέματος τοῦ θυσιαστηρίου χαλκοῦς, εὐρεῖς τοῖς μοχλοῖς, ὥστε αἴρειν ἐν αὐτοῖς τὸ θυσιαστήριον·
\vs{25}Οὗτος ἐποίησε τὸ ἔλαιον τῆς χρίσεως τὸ ἅγιον, καὶ τὴν σύνθεσιν τοῦ θυμιάματος καθαρὸν ἔργον μυρεψοῦ·
\vs{26}Οὗτος ἐποίησε τὸν λουτῆρα τὸν χαλκοῦν, καὶ τὴν βάσιν αὐτοῦ χαλκῆν ἐκ τῶν κατόπτρων τῶν νηστευσασῶν, αἳ ἐνήστευσαν παρὰ τὰς θύρας τῆς σκηνῆς τοῦ μαρτυρίου, ἐν ᾗ ἡμέρᾳ ἔπηξεν αὐτήν.

\vs{27}Καὶ ἐποίησε τὸν λουτῆρα, ἵνα νίπτωνται ἐξ αὐτοῦ Μωυσῆς καὶ Ἀαρὼν καὶ οἱ υἱοὶ αὐτοῦ τὰς χεῖρας αὐτῶν καὶ τοὺς πόδας, εἰσπορευομένων αὐτῶν εἰς τὴν σκηνὴν τοῦ μαρτυρίου, ἢ ὅταν προσπορεύωνται πρὸς τὸ θυσιαστήριον λειτουργεῖν, ἐνίπτοντο ἐξ αὐτοῦ, καθάπερ συνέταξε Κύριος τῷ Μωυσῇ.

\ch{39}
Πᾶν τὸ χρυσίον, ὃ κατειργάσθη εἰς τὰ ἔργα κατὰ πᾶσαν τὴν ἐργασίαν τῶν ἁγίων, ἐγένετο χρυσίου τοῦ τῆς ἀπαρχῆς, ἐννέα καὶ εἴκοσι τάλαντα, καὶ ἑπτακόσιοι εἴκοσι σίκλοι κατὰ τὸν σίκλον τὸν ἅγιον·
\vs{2}Καὶ ἀργυρίου ἀφαίρεμα παρὰ τῶν ἐπεσκεμμένων ἀνδρῶν τῆς συναγωγῆς ἑκατὸν τάλαντα, καὶ χίλιοι ἑπτακόσιοι ἑβδομηκονταπέντε σίκλοι· δραχμὴ μία τῇ κεφαλῇ τὸ ἥμισυ τοῦ σίκλου, κατὰ τὸν σίκλον τὸν ἅγιον·
\vs{3}Πᾶς ὁ παραπορευόμενος τὴν ἐπίσκεψιν ἀπὸ εἰκοσαετοῦς καὶ ἐπάνω εἰς τὰς ἑξήκοντα μυριάδας, καὶ τρισχίλιοι πεντακόσιοι καὶ πεντήκοντα.
\vs{4}Καὶ ἐγενήθη τὰ ἑκατὸν τάλαντα τοῦ ἀργυρίου εἰς τὴν χώνευσιν τῶν ἑκατὸν κεφαλίδων τῆς σκηνῆς, καὶ εἰς τὰς κεφαλίδας τοῦ καταπετάσματος,
\vs{5}ἑκατὸν κεφαλίδες εἰς τὰ ἑκατὸν τάλαντα, τάλαντον τῇ κεφαλίδι·
\vs{6}Καὶ τοὺς χιλίους ἑπτακοσίους ἑβδομηκοντα πέντε σίκλους ἐποίησεν εἰς τὰς ἀγκύλας τοῖς στύλοις· καὶ κατεχρύσωσε τὰς κεφαλίδας αὐτῶν, καὶ κατεκόσμησεν αὐτούς.

\vs{7}Καὶ ὁ χαλκὸς τοῦ ἀφαιρέματος ἑβδομήκοντα τάλαντα, καὶ χίλιοι πεντακόσιοι σίκλοι·
\vs{8}Καὶ ἐποίησαν ἐξ αὐτον τὰς βάσεις τῆς θύρας τῆς σκηνῆς τοῦ μαρτυρίου,
\vs{9}καὶ τὰς βάσεις τῆς αὐλῆς κύκλῳ, καὶ τὰς βάσεις τῆς πύλης τῆς αὐλῆς, καὶ τοὺς πασσάλους τῆς σκηνῆς, καὶ τοὺς πασσάλους τῆς αὐλῆς κύκλῳ,
\vs{10}καὶ τὸ παράθεμα τὸ χαλκοῦν τοῦ θυσιαστηρίου, καὶ πάντα τὰ σκεύη τοῦ θυσιαστηρίου, καὶ πάντα τὰ ἐργαλεῖα τῆς σκηνῆς τοῦ μαρτυρίου·
\vs{11}Καὶ ἐποίησαν οἱ υἱοὶ Ἰσραὴλ, καθὰ συνέταξε Κύριος τῷ Μωυσῇ, οὕτως ἐποίησαν·
\vs{12}Τὸ δὲ λοιπὸν χρυσίον τοῦ ἀφαιρέματος ἐποίησαν σκεύη εἰς τὸ λειτουργεῖν ἐν αὐτοῖς ἔναντι Κυρίου·
\vs{13}Καὶ τὴν καταλειφθεῖσαν ὑάκινθον, καὶ πορφύραν, καὶ τὸ κόκκινον ἐποίησαν στολὰς λειτουργικὰς Ἀαρών, ὥστε λειτουργεῖν ἐν αὐταῖς ἐν τῷ ἁγίῳ·
\vs{14}Καὶ ἤνεγκαν τὰς στολὰς πρὸς Μωυσῆν, καὶ τὴν σκηνὴν, καὶ τὰ σκεύη αὐτῆς, τὰς βάσεις καὶ τοὺς μοχλοὺς αὐτῆς, καὶ τοὺς στύλους·
\vs{15}καὶ τὸ θυσιαστήριον, καὶ πάντα τὰ σκεύη αὐτοῦ.

\vs{16}Καὶ τὸ ἔλαιον τῆς χρίσεως, καὶ τὸ θυμίαμα τῆς συνθέσεως, καὶ τὴν λυχνίαν τὴν καθαρὰν,
\vs{17}καὶ τοὺς λύχνους αὐτῆς, λύχνους τῆς καύσεως, καὶ τὸ ἔλαιον τοῦ φωτός·
\vs{18}Καὶ τὴν τράπεζαν τῆς προθέσεως, καὶ πάντα τὰ σκεύη αὐτῆς· καὶ τοὺς ἄρτους τοὺς προκειμένους·
\vs{19}Καὶ τὰς στολὰς τοῦ ἁγίου, αἵ εἰσιν Ἀαρών, καὶ τὰς στολὰς τῶν υἱῶν αὐτοῦ, εἰς τὴν ἱερατείαν·
\vs{20}Καὶ τὰ ἱστία τῆς αὐλῆς, καὶ τοὺς στύλους· καὶ τὸ καταπέτασμα τῆς θύρας τῆς σκηνῆς, καὶ τῆς πύλης τῆς αὐλῆς·
\vs{21}Καὶ πάντα τὰ σκεύη τῆς σκηνῆς, καὶ πάντα τὰ ἐργαλεῖα αὐτῆς· καὶ τὰς διφθέρας δέρματα κριῶν ἠρυθροδανωμένα, καὶ τὰ καλύμματα ὑακίνθινα, καὶ τῶν λοιπῶν τὰ ἐπικαλύμματα· καὶ τοὺς πασσάλους, καὶ πάντα τὰ ἐργαλεῖα τὰ εἰς τὰ ἔργα τῆς σκηνῆς τοῦ μαρτυρίου·
\vs{22}Ὃσα συνέταξε Κύριος τῷ Μωυσῇ, οὕτως ἐποίησαν οἱ υἱοὶ Ἰσραὴλ πᾶσαν τὴν ἀποσκευήν·
\vs{23}Καὶ εἶδε Μωυσῆς πάντα τὰ ἔργα, καὶ ἦσαν πεποιηκότες αὐτὰ ὃν τρόπον συνέταξε Κύριος τῷ Μωυσῇ, οὕτως ἐποίησαν αὐτὰ, καὶ εὐλόγησεν αὐτοὺς Μωυσῆς.

\ch{40}
Καὶ ἐλάλησε Κύριος πρὸς Μωυσῆν, λέγων,
\vs{2}ἐν ἡμέρᾳ μιᾷ τοῦ μηνὸς τοῦ πρώτου νουμηνίᾳ, στήσεις τὴν σκηνὴν τοῦ μαρτυρίου.
\vs{3}Καὶ θήσεις τὴν κιβωτὸν τοῦ μαρτυρίου, καὶ σκεπάσεις τὴν κιβωτὸν τῷ καταπετάσματι.
\vs{4}Καὶ εἰσοίσεις τὴν τράπεζαν, καὶ προθήσεις τὴν πρόθεσιν αὐτῆς· καὶ εἰσοίσεις τὴν λυχνίαν, καὶ ἐπιθήσεις τοὺς λύχνους αὐτῆς.
\vs{5}Καὶ θήσεις τὸ θυσιαστήριον τὸ χρυσοῦν, εἰς τὸ θυμιᾷν ἐναντίον τῆς κιβωτοῦ· καὶ ἐπιθήσεις κάλυμμα καταπετάσματος ἐπὶ τὴν θύραν τῆς σκηνῆς τοῦ μαρτυρίου.
\vs{6}Καὶ τὸ θυσιαστήριον τῶν καρπωμάτων θήσεις παρὰ τὰς θύρας τῆς σκηνῆς τοῦ μαρτυρίου·
\vs{8}καὶ περιθήσεις τὴν σκηνήν, καὶ πάντα τὰ αὐτῆς ἁγιάσεις κύκλῳ.
\vs{9}Καὶ λήψῃ τὸ ἔλαιον τοῦ χρίσματος, καὶ χρίσεις τὴν σκηνὴν, καὶ πάντα τὰ ἐν αὐτῇ, καὶ ἁγιάσεις αὐτὴν, καὶ πάντα τὰ σκεύη αὐτῆς, καὶ ἔσται ἁγία.
\vs{10}Καὶ χρίσεις τὸ θυσιαστήριον τῶν καρπωμάτων, καὶ πάντα τὰ σκεύη αὐτοῦ· καὶ ἁγιάσεις τὸ θυσιαστήριον, καὶ ἔσται τὸ θυσιαστήριον ἅγιον τῶν ἁγίων.
\vs{12}Καὶ προσάξεις Ἀαρὼν καὶ τοὺς υἱοὺς αὐτοῦ ἐπὶ τὰς θύρας τῆς σκηνῆς τοῦ μαρτυρίου, καὶ λούσεις αὐτοὺς ὕδατι.
\vs{13}Καὶ ἐνδύσεις Ἀαρὼν τὰς στολὰς τὰς ἁγίας, καὶ χρίσεις αὐτὸν, καὶ ἁγιάσεις αὐτὸν, καὶ ἱερατεύει μοι.
\vs{14}Καὶ τοὺς υἱοὺς αὐτοῦ προσάξεις, καὶ ἐνδύσεις αὐτοὺς χιτῶνας.
\vs{15}Καὶ ἀλείψεις αὐτοὺς ὃν τρόπον ἤλειψας τὸν πατέρα αὐτῶν, καὶ ἱερατεύσουσί μοι· καὶ ἔσται, ὥστε εἶναι αὐτοῖς χρίσμα ἱερατείας εἰς τὸν αἰῶνα, εἰς τὰς γενεὰς αὐτῶν.
\vs{16}Καὶ ἐποίησε Μωυσῆς πάντα, ὅσα ἐνετείλατο αὐτῷ Κύριος, οὕτως ἐποίησε.

\vs{17}Καὶ ἐγένετο ἐν τῷ μηνὶ τῷ πρώτῳ, τῷ δευτέρῳ ἔτει, ἐκπορευομένων αὐτῶν ἐξ Αἰγύπτου, νουμηνίᾳ ἐστάθη ἡ σκηνή.
\vs{18}Καὶ ἔστησε Μωυσῆς τὴν σκηνὴν, καὶ ἐπέθηκε τὰς κεφαλιδας, καὶ διενέβαλε τοὺς μοχλοὺς, καὶ ἔστησε τοὺς στύλους.
\vs{19}Καὶ ἐξέτεινε τὰς αὐλαίας ἐπὶ τὴν σκηνὴν, καὶ ἐπέθηκε τὸ κατακάλυμμα τῆς σκηνῆς ἐπʼ αὐτὴν ἄνωθεν, καθὰ συνέταξε Κύριος τῷ Μωυσῇ.
\vs{20}Καὶ λαβὼν τὰ μαρτύρια ἐνέβαλεν εἰς τὴν κιβωτόν· καὶ ὑπέθηκε τοὺς διωστῆρας ὑπὸ τὴν κιβωτὸν,
\vs{21}καὶ εἰσήνεγκε τὴν κιβωτὸν εἰς τὴν σκηνὴν, καὶ ἐπέθηκε τὸ κατακάλυμμα τοῦ καταπετάσματος, καὶ ἐσκέπασε τὴν κιβωτὸν τοῦ μαρτυρίου, ὃν τρόπον συνέταξε Κύριος τῷ Μωυσῇ·
\vs{22}Καὶ ἐπέθηκε τὴν τράπεζαν εἰς τὴν σκηνὴν τοῦ μαρτυρίου, τὸ πρὸς Βοῤῥᾶν ἔξωθεν τοῦ καταπετάσματος τῆς σκηνῆς.
\vs{23}Καὶ προσέθηκεν ἐπʼ αὐτῆς ἄρτους τῆς προθέσεως ἔναντι Κυρίου, ὃν τρόπον συνέταξε Κύριος τῷ Μωυσῇ.
\vs{24}Καὶ ἔθηκε τὴν λυχνίαν εἰς τὴν σκηνὴν τοῦ μαρτυρίου, εἰς τὸ κλίτος τῆς σκηνῆς τὸ πρὸς Νότον.
\vs{25}Καὶ ἐπέθηκε τοὺς λύχνους αὐτῆς ἔναντι Κυρίου, ὃν τρόπον συνέταξε Κύριος τῷ Μωυσῇ.
\vs{26}Καὶ ἔθηκε τὸ θυσιαστήριον τὸ χρυσοῦν ἐν τῇ σκηνῇ τοῦ μαρτυρίου ἀπέναντι τοῦ καταπετάσματος,
\vs{27}καὶ ἐθυμίασεν ἐνʼ αὐτοῦ θυμίαμα τῆς συνθέσεως, καθάπερ συνέταξε Κύριος τῷ Μωυσῇ.
\vs{29}Καὶ τὸ θυσιαστήριον τῶν καρπωμάτων ἔθηκε παρὰ τὰς θύρας τῆς σκηνῆς.
\vs{33}Καὶ ἔστησε τὴν αὐλὴν κύκλῳ τῆς σκηνῆς, και τοῦ θυσιαστηρίου· καὶ συνετέλεσε Μωυσῆς πάντα τὰ ἔργα.

\vs{34}Καὶ ἐκάλυψεν ἡ νεφέλη τὴν σκηνὴν τοῦ μαρτυρίου· καὶ δόξης Κυρίου ἐπλήσθη ἡ σκηνή.
\vs{35}Καὶ οὐκ ἠδυνάσθη Μωυσῆς εἰσελθεῖν εἰς τὴν σκηνὴν τοῦ μαρτυρίου, ὅτι ἐπεσκίαζεν ἐπʼ αὐτὴν ἡ νεφέλη, καὶ δόξης Κυρίου ἐνεπλήσθη ἡ σκηνή.
\vs{36}Ἡνίκα δʼ ἂν ἀνέβη ἡ νεφέλη ἀπὸ τῆς σκηνῆς, ἀνεζεύγνυσαν οἱ υἱοὶ Ἰσραὴλ σὺν τῇ ἀπαρτίᾳ αὐτῶν.
\vs{37}Εἰ δὲ μὴ ἀνέβη ἡ νεφέλη, οὐκ ἀνεζεύγνυσαν ἕως ἡμέρας, ἧς ἀνέβη ἡ νεφέλη.
\vs{38}Νεφέλη γὰρ ἦν ἐπὶ τῆς σκηνῆς ἡμέρας, καὶ πῦρ ἦν ἐπʼ αὐτῆς νυκτὸς ἐναντίον παντὸς Ἰσραὴλ, ἐν πάσαις ταῖς ἀναζυγαῖς αὐτῶν.


\def\book{ΨΑΛΜΟΙ}
\biblebook{ΨΑΛΜΟΙ}

\begin{psalms}
\ch{1}{1}ΜΑΚΑΡΙΟΣ ἀνὴρ, ὃς οὐκ ἐπορεύθη ἐν βουλῇ ἀσεβῶν, καὶ ἐν ὁδῷ ἁμαρτωλῶν· οὐκ ἔστη, καὶ ἐπὶ καθέδρᾳ λοιμῶν οὐκ ἐκάθισεν.
\vs{2}Ἀλλ’ ἢ ἐν τῷ νόμῳ Κυρίου τὸ θέλημα αὐτοῦ, καὶ ἐν τῷ νόμῳ αὐτοῦ μελετήσει ἡμέρας καὶ νυκτός.
\vs{3}Καὶ ἔσται ὡς τὸ ξύλον τὸ πεφυτευμένον παρὰ τὰς διεξόδους τῶν ὑδάτων, ὃ τὸν καρπὸν αὐτοῦ δώσει ἐν καιρῷ αὐτοῦ· καὶ τὸ φύλλον αὐτοῦ οὐκ ἀποῤῥυήσεται, καὶ πάντα ὅσα ἂν ποιῇ κατευοδωθήσεται.

\vs{4}Οὐχ οὕτως οἱ ἀσεβεῖς, οὐχ οὕτως, ἀλλʼ ἢ ὡς ὁ χνοῦς ὃν ἐκρίπτει ὁ ἄνεμος ἀπὸ προσώπου τῆς γῆς.
\vs{5}Διὰ τοῦτο οὐκ ἀναστήσονται οἱ ἀσεβεῖς ἐν κρίσει, οὐδὲ ἁμαρτωλοὶ ἐν βουλῇ δικαίων.
\vs{6}Ὅτι γινώσκει Κύριος ὁδὸν δικαίων, καὶ ὁδὸς ἀσεβῶν ἀπολεῖται.

\ch{2}{2}Ἱνατί ἐφρύαξαν ἔθνη, καὶ λαοὶ ἐμελέτησαν κενά;
\vs{2}Παρέστησαν οἱ βασιλεῖς τῆς γῆς, καὶ οἱ ἄρχοντες συνήχθησαν ἐπιτοαυτὸ κατὰ τοῦ Κυρίου, καὶ κατὰ τοῦ Χριστοῦ αὐτοῦ.
\vs{3}Διαῤῥήξωμεν τοὺς δεσμοὺς αὐτῶν, καὶ ἀποῤῥίψωμεν ἀφʼ ἡμῶν τὸν ζυγὸν αὐτῶν.

\vs{4}Ὁ κατοικῶν ἐν οὐρανοῖς ἐκγελάσεται αὐτοὺς, καὶ ὁ Κύριος ἐκμυκτηριεῖ αὐτούς.
\vs{5}Τότε λαλήσει πρὸς αὐτοὺς ἐν ὀργῇ αὐτοῦ, καὶ ἐν τῷ θυμῷ αὐτοῦ ταράξει αὐτούς.
\vs{6}Ἐγὼ δὲ κατεστάθην βασιλεὺς ὑπʼ αὐτοῦ ἐπὶ Σιὼν ὄρος τὸ ἅγιον αὐτοῦ,
\vs{7}διαγγέλλων τὸ πρόσταγμα Κυρίου· Κύριος εἶπε πρὸς μὲ, υἱός μου εἶ σὺ, ἐγὼ σήμερον γεγέννηκά σε.
\vs{8}Αἴτησαι παρʼ ἐμοῦ, καὶ δώσω σοι ἔθνη τὴν κληρονομίαν σου, καὶ τὴν κατάσχεσίν σου τὰ πέρατα τῆς γῆς.
\vs{9}Ποιμανεῖς αὐτοὺς ἐν ῥάβδῳ σιδηρᾷ, ὡς σκεῦος κεραμέως συντρίψεις αὐτούς.

\vs{10}Καὶ νῦν βασιλεῖς σύνετε, παιδεύθητε πάντες οἱ κρίνοντες τὴν γῆν.
\vs{11}Δουλεύσατε τῷ Κυρίῳ ἐν φόβῳ, καὶ ἀγαλλιᾶσθε αὐτῷ ἐν τρόμῳ.
\vs{12}Δράξασθε παιδείας, μή ποτε ὀργισθῇ Κύριος, καὶ ἀπολεῖσθε ἐξ ὁδοῦ δικαίας· ὅταν ἐκκαυθῇ ἐν τάχει ὁ θυμὸς αὐτοῦ, μακάριοι πάντες οἱ πεποιθότες ἐπʼ αὐτῷ.

\begin{psalmheading}{\ch{3}{3} Ψαλμὸς τῷ Δαυὶδ, ὁπότε ἀπεδίδρασκεν ἀπὸ προσώπου Ἀβεσσαλὼμ τοῦ υἱοῦ αὐτοῦ.}
\end{psalmheading}
\vs{2}Κύριε τί ἐπληθύνθησαν οἱ θλίβοντές με; πολλοὶ ἐπανίστανται ἐπʼ ἐμέ.
\vs{3}Πολλοὶ λέγουσι τῇ ψυχῇ μου, οὐκ ἔστι σωτηρία αὐτῷ ἐν τῷ Θεῷ αὐτοῦ· διάψαλμα.

\vs{4}Σὺ δὲ Κύριε, ἀντιλήπτωρ μου εἶ, δόξα μου, καὶ ὑψῶν τὴν κεφαλήν μου.
\vs{5}Φωνῇ μου πρὸς Κύριον ἐκέκραξα, καὶ ἐπήκουσέ μου ἐξ ὄρους ἁγίου αὐτοῦ· διάψαλμα.
\vs{6}Ἐγὼ ἐκοιμήθην καὶ ὕπνωσα, ἐξηγέρθην, ὅτι Κύριος ἀντιλήψεταί μου.
\vs{7}Οὐ φοβηθήσομαι ἀπὸ μυριάδων λαοῦ, τῶν κύκλῳ ἐπιτιθεμένων μοι.
\vs{8}Ἀνάστα Κύριε, σῶσόν με ὁ Θεός μου· ὅτι σὺ ἐπάταξας πάντας τοὺς ἐχθραίνοντάς μοι ματαίως, ὀδόντας ἁμαρτωλῶν συνέτριψας.
\vs{9}Τοῦ Κυρίου ἡ σωτηρία, καὶ ἐπὶ τὸν λαόν σου ἡ εὐλογία σου.

\begin{psalmheading}{\ch{4}{4} Εἰς τὸ τέλος, ἐν ψαλμοῖς ᾠδὴ τᾠ Δαυίδ.}
\end{psalmheading}
\vs{2}Ἐν τῷ ἐπικαλεῖσθαί με, εἰσήκουσέ μου ὁ Θεὸς τῆς δικαιοσύνης μου· ἐν θλίψει ἐπλάτυνάς μοι· οἰκτείρησόν με, καὶ εἰσάκουσον τῆς προσευχῆς μου.

\vs{3}Υἱοὶ ἀνθρώπων, ἕως πότε βαρυκάρδιοι; ἱνατί ἀγαπᾶτε ματαιότητα, καὶ ζητεῖτε ψεῦδος; διάψαλμα.
\vs{4}Καὶ γνῶτε ὅτι ἐθαυμάστωσε Κύριος τὸν ὅσιον αὐτοῦ, Κύριος εἰσακούσεταί μου ἐν τῷ κεκραγέναι με πρὸς αὐτόν.
\vs{5}Ὀργίζεσθε καὶ μὴ ἁμαρτάνετε· ἃ λέγετε ἐν ταῖς καρδίαις ὑμῶν, ἐπὶ ταῖς κοίταις ὑμῶν κατανύγητε· διάψαλμα.
\vs{6}Θύσατε θυσίαν δικαιοσύνης, καὶ ἐλπίσατε ἐπὶ Κύριον.

\vs{7}Πολλοὶ λέγουσι, τίς δείξει ἡμῖν τὰ ἀγαθά; ἐσημειώθη ἐφʼ ἡμᾶς τὸ φῶς τοῦ προσώπου σου, Κύριε.
\vs{8}Ἔδωκας εὐφροσύνην εἰς τὴν καρδίαν μου· ἀπὸ καρποῦ σίτου καὶ οἴνου καὶ ἐλαίου αὐτῶν ἐπληθύνθησαν.
\vs{9}Ἐν εἰρήνῃ ἐπὶ τὸ αὐτὸ κοιμηθήσομαι, καὶ ὑπνώσω· ὅτι σὺ Κύριε κατὰ μόνας ἐπʼ ἐλπίδι κατῴκισάς με.

\begin{psalmheading}{\ch{5}{5} Εἰς τὸ τέλος, ὑπὲρ τῆς κληρονομούσης, ψαλμὸς τῷ Δαυίδ.}
\end{psalmheading}
\vs{2}Τὰ ῥήματά μου ἐνώτισαι Κύριε, σύνες τῆς κραυγῆς μου,
\vs{3}πρόσχες τῇ φωνῇ τῆς δεήσεώς μου, ὁ βασιλεύς μου καὶ ὁ Θεός μου· ὅτι πρὸς σὲ προσεύξομαι Κύριε,
\vs{4}τοπρωῒ εἰσακούσῃ τῆς φωνῆς μου· τοπρωῒ παραστήσομαί σοι, καὶ ἐπόψομαι.
\vs{5}Ὅτι οὐχὶ Θεὸς θέλων ἀνομίαν σὺ εἶ· οὐδὲ παροικησει σοι πονηρευόμενος,
\vs{6}οὐδὲ διαμενοῦσι παράνομοι κατέναντι τῶν ὀφθαλμῶν σου· ἐμίσησας Κύριε πάντας τοὺς ἐργαζομένους τὴν ἀνομίαν,
\vs{7}ἀπολεῖς πάντας τοὺς λαλοῦντας τὸ ψεῦδος· ἄνδρα αἱμάτων καὶ δόλιον βδελύσσεται Κύριος.
\vs{8}Ἐγὼ δὲ ἐν τῷ πλήθει τοῦ ἐλέου σου εἰσελεύσομαι εἰς τὸν οἶκόν σου, προσκυνήσω πρὸς ναὸν ἅγιόν σου ἐν φόβῳ σου.

\vs{9}Κύριε ὁδήγησόν με ἐν τῇ δικαιοσύνῃ σου ἕνεκα τῶν ἐχθρῶν μου, κατεύθυνον ἐνώπιόν σου τὴν ὁδόν μου.
\vs{10}Ὅτι οὐκ ἔστιν ἐν τῷ στόματι αὐτῶν ἀλήθεια· ἡ καρδία αὐτῶν ματαία· τάφος ἀνεῳγμένος ὁ λάρυγξ αὐτῶν· ταῖς γλώσσαις αὐτῶν ἐδολιοῦσαν.
\vs{11}Κρίνον αὐτοὺς ὁ Θεός· ἀποπεσάτωσαν ἀπὸ τῶν διαβουλιῶν αὐτῶν· κατὰ τὸ πλῆθος τῶν ἀσεβειῶν αὐτῶν ἔξωσον αὐτοὺς, ὅτι παρεπίκρανάν σε Κύριε.

\vs{12}Καὶ εὐφρανθήτωσαν ἐπὶ σοὶ πάντες οἱ ἐλπίζοντες ἐπὶ σὲ, εἰς αἰῶνα ἀγαλλιάσονται, καὶ κατασκηνώσεις ἐν αὐτοῖς· καὶ καυχήσονται ἐπὶ σοὶ πάντες οἱ ἀγαπῶντες τὸ ὄνομά σου,
\vs{13}ὅτι σὺ εὐλογήσεις δίκαιον Κύριε, ὡς ὅπλῳ εὐδοκίας ἐστεφάνωσας ἡμᾶς.

\begin{psalmheading}{\ch{6}{6} Εἰς τὸ τέλος, ἐν ὕμνοις ὑπὲρ τῆς ὀγδόης, ψαλμὸς τῷ Δαυίδ.}
\end{psalmheading}
\vs{2}Κύριε, μὴ τῷ θυμῷ σου ἐλέγξῃς με, μηδὲ τῇ ὀργῇ σου παιδεύσῃς με.
\vs{3}Ἐλέησόν με Κύριε, ὅτι ἀσθενής εἰμι· ἴασαί με Κύριε, ὅτι ἐταράχθη τὰ ὀστᾶ μου.
\vs{4}Καὶ ἡ ψυχή μου ἐταράχθη σφόδρα· καὶ σὺ Κύριε ἕως πότε;
\vs{5}Ἐπίστρεψον Κύριε, ῥῦσαι τὴν ψυχήν μου· σῶσόν με ἕνεκεν τοῦ ἐλέους σου,
\vs{6}ὅτι οὐκ ἔστιν ἐν τῷ θανάτῳ ὁ μνημονεύων σου, ἐν δὲ τῷ ᾅδῃ τίς ἐξομολογήσεταί σοι;
\vs{7}Ἐκοπίασα ἐν στεναγμῷ μου, λούσω καθʼ ἑκάστην νύκτα τὴν κλίνην μου, ἐν δάκρυσί μου τὴν στρωμνήν μου βρέξω.
\vs{8}Ἐταράχθη ἀπὸ θυμοῦ ὁ ὀφθαλμός μου, ἐπαλαιώθην ἐν πᾶσι τοῖς ἐχθροῖς μου.

\vs{9}Ἀπόστητε ἀπʼ ἐμοῦ πάντες οἱ ἐργαζόμενοι τὴν ἀνομίαν, ὅτι εἰσήκουσε Κύριος τῆς φωνῆς τοῦ κλαυθμοῦ μου.
\vs{10}Εἰσήκουσε Κύριος τῆς δεήσεώς μου, Κύριος τὴν προσευχήν μου προσεδέξατο.
\vs{11}Αἰσχυνθείησαν καὶ ταραχθείησαν σφόδρα πάντες οἱ ἐχθροί μου, ἐπιστραφείησαν καὶ αἰσχυνθείησαν σφόδρα διὰ τάχους.

\begin{psalmheading}{\ch{7}{7} Ψαλμὸς τῷ Δαυὶδ, ὃν ᾖσε τῷ Κυρίῳ ὑπὲρ τῶν λόγων Χουσὶ υἱοῦ Ἰεμενεί.}
\end{psalmheading}
\vs{2}Κύριε ὁ Θεός μου, ἐπὶ σοὶ ἤλπισα, σῶσόν με ἐκ πάντων τῶν διωκόντων με, καὶ ῥῦσαί με,
\vs{3}μή ποτε ἁρπάσῃ ὡς λέων τὴν ψυχήν μου, μὴ ὄντος λυτρουμένου, μηδὲ σώζοντος.

\vs{4}Κύριε ὁ Θεός μου, εἰ ἐποίησα τοῦτο, εἰ ἔστιν ἀδικία ἐν χερσί μου,
\vs{5}εἰ ἀνταπέδωκα τοῖς ἀνταποδιδοῦσί μοι κακὰ, ἀποπέσοιμι ἄρα ἀπὸ τῶν ἐχθρῶν μου κενός·
\vs{6}Καταδιώξαι ὁ ἐχθρὸς τὴν ψυχήν μου καὶ καταλάβοι, καὶ καταπατήσαι εἰς γῆν τὴν ζωήν μου, καὶ τὴν δόξαν μου εἰς χοῦν κατασκηνώσαι· διάψαλμα.

\vs{7}Ἀνάστηθι Κύριε ἐν ὀργῇ σου, ὑψώθητι ἐν τοῖς πέρασι τῶν ἐχθρῶν μου· ἐξεγέρθητι Κύριε ὁ Θεός μου ἐν προστάγματι ᾧ ἐνετείλω,
\vs{8}καὶ συναγωγὴ λαῶν κυκλώσει σε· καὶ ὑπὲρ ταύτης εἰς ὕψος ἐπίστρεψον.
\vs{9}Κύριος κρινεῖ λαούς· κρίνον με Κύριε κατὰ τὴν δικαιοσύνην μου, καὶ κατὰ τὴν ἀκακίαν μου ἐπʼ ἐμοί.
\vs{10}Συντελεσθήτω δὴ πονηρία ἁμαρτωλῶν, καὶ κατευθυνεῖς δίκαιον, ἐτάζων καρδίας καὶ νεφροὺς ὁ Θεός.

\vs{11}Δικαία ἡ βοήθειά μου παρὰ τοῦ Θεοῦ τοῦ σώζοντος τοὺς εὐθεῖς τῇ καρδίᾳ.
\vs{12}Ὁ Θεὸς κριτὴς δίκαιος, καὶ ἰσχυρὸς, καὶ μακρόθυμος, μὴ ὀργὴν ἐπάγων καθʼ ἑκάστην ἡμέραν.
\vs{13}Ἐὰν μὴ ἐπιστραφῆτε, τὴν ῥομφαίαν αὐτοῦ στιλβώσει, τὸ τόξον αὐτοῦ ἐνέτεινε, καὶ ἡτοίμασεν αὐτό.
\vs{14}Καὶ ἐν αὐτῷ ἡτοίμασε σκεύη θανάτου, τὰ βέλη αὐτοῦ τοῖς καιομένοις ἐξειργάσατο.

\vs{15}Ἰδοὺ ὠδίνησεν ἀδικίαν, συνέλαβε πόνον, καὶ ἔτεκεν ἀνομίαν.
\vs{16}Λάκκον ὤρυξε καὶ ἀνέσκαψεν αὐτὸν, καὶ ἐμπεσεῖται εἰς βόθρον ὃν εἰργάσατο.
\vs{17}Ἐπιστρέψει ὁ πόνος αὐτοῦ εἰς κεφαλὴν αὐτοῦ, καὶ ἐπὶ κορυφὴν αὐτοῦ ἡ ἀδικία αὐτοῦ καταβήσεται.
\vs{18}Ἐξομολογήσομαι Κυρίῳ κατὰ τὴν δικαιοσύνην αὐτοῦ, ψαλῶ τῷ ὀνόματι Κυρίου τοῦ ὑψίστου.

\begin{psalmheading}{\ch{8}{8} Εἰς τὸ τέλος, ὑπὲρ τῶν ληνῶν, ψαλμὸς τῷ Δαυίδ.}
\end{psalmheading}
\vs{2}Κύριε ὁ Κύριος ἡμῶν, ὡς θαυμαστὸν τὸ ὄνομά σου ἐν πάσῃ τῇ γῇ; ὅτι ἐπῄρθη ἡ μεγαλοπρέπειά σου ὑπεράνω τῶν οὐρανῶν.
\vs{3}Ἐκ στόματος νηπίων καὶ θηλαζόντων κατηρτίσω αἶνον· ἕνεκα τῶν ἐχθρῶν σου, τοῦ καταλῦσαι ἐχθρὸν καὶ ἐκδικητήν.

\vs{4}Ὅτι ὄψομαι τοὺς οὐρανοὺς ἔργα τῶν δακτύλων σου, σελήνην καὶ ἀστέρας, ἃ σὺ ἐθεμελίωσας·
\vs{5}Τί ἐστιν ἄνθρωπος, ὅτι μιμνήσκῃ αὐτοῦ; ἢ υἱὸς ἀνθρώπου, ὅτι ἐπισκέπτῃ αὐτόν;
\vs{6}Ἠλάττωσας αὐτὸν βραχύ τι παρʼ ἀγγέλους, δόξῃ καὶ τιμῇ ἐστεφάνωσας αὐτὸν,
\vs{7}καὶ κατέστησας αὐτὸν ἐπὶ τὰ ἔργα τῶν χειρῶν σου· πάντα ὑπέταξας ὑποκάτω τῶν ποδῶν αὐτοῦ,
\vs{8}πρόβατα καὶ βόας πάσας, ἔτι δὲ καὶ τὰ κτήνη τοῦ πεδίου,
\vs{9}τὰ πετεινὰ τοῦ οὐρανοῦ, καὶ τοὺς ἰχθύας τῆς θαλάσσης, τὰ διαπορευόμενα τρίβους θαλασσῶν.
\vs{10}Κύριε ὁ Κύριος ἡμῶν, ὡς θαυμαστὸν ὄνομά σου ἐν πάσῃ τῇ γῇ;

\begin{psalmheading}{\ch{9}{9-10} Εἰς τὸ τέλος, ὑπὲρ τῶν κρυφίων τοῦ υἱοῦ, ψαλμὸς τῷ Δαυίδ.}
\end{psalmheading}
\vs{2}Ἐξομολογήσομαι σοι Κύριε ἐν ὅλῃ καρδίᾳ μου, διηγήσομαι πάντα τὰ θαυμάσιά σου.
\vs{3}Εὐφρανθήσομαι καὶ ἀγαλλιάσομαι ἐν σοὶ, ψαλῶ τῷ ὀνόματί σου ὕψιστε.

\vs{4}Ἐν τῷ ἀποστραφῆναι τὸν ἐχθρόν μου εἰς τὰ ὀπίσω, ἀσθενήσουσι καὶ ἀπολοῦνται ἀπὸ προσώπου σου.
\vs{5}Ὅτι ἐποίησας τὴν κρίσιν μου καὶ τὴν δίκην μου, ἐκάθισας ἐπὶ θρόνου ὁ κρίνων δικαιοσύνην.
\vs{6}Ἐπετίμησας ἔθνεσι, καὶ ἀπώλετο ὁ ἀσεβὴς· τὸ ὄνομα αὐτῶν ἐξήλειψας εἰς τὸν αἰῶνα, καὶ εἰς τὸν αἰῶνα τοῦ αἰῶνος.
\vs{7}Τοῦ ἐχθροῦ ἐξέλιπον αἱ ῥομφαῖαι εἰς τέλος, καὶ πόλεις καθεῖλες· ἀπώλετο τὸ μνημόσυνον αὐτῶν μετʼ ἤχου,
\vs{8}καὶ ὁ Κύριος εἰς τὸν αἰῶνα μένει· ἡτοίμασεν ἐν κρίσει τὸν θρόνον αὐτοῦ,
\vs{9}καὶ αὐτὸς κρινεῖ τὴν οἰκουμένην ἐν δικαιοσύνῃ, κρινεῖ λαοὺς ἐν εὐθύτητι.
\vs{10}Καὶ ἐγένετο Κύριος καταφυγὴ τῷ πένητι, βοηθὸς ἐν εὐκαιρίαις, ἐν θλίψει.
\vs{11}Καὶ ἐλπισάτωσαν ἐπὶ σὲ οἱ γινώσκοντες τὸ ὄνομά σου, ὅτι οὐκ ἐγκατέλιπες τοὺς ἐκζητοῦντάς σε Κύριε.

\vs{12}Ψάλατε τῷ Κυρίῳ τῷ κατοικοῦντι ἐν Σιὼν, ἀναγγείλατε ἐν τοῖς ἔθνεσι τὰ ἐπιτηδεύματα αὐτοῦ.
\vs{13}Ὅτι ἐκζητῶν τὰ αἵματα αὐτῶν ἐμνήσθη, οὐκ ἐπελάθετο τῆς δεήσεως τῶν πενήτων.

\vs{14}Ἐλέησόν με Κύριε, ἴδε τὴν ταπείνωσίν μου ἐκ τῶν ἐχθρῶν μου, ὁ ὑψῶν με ἐκ τῶν πυλῶν τοῦ θανάτου·
\vs{15}Ὅπως ἂν ἐξαγγείλω πάσας τὰς αἰνέσεις σου ἐν ταῖς πύλαις τῆς θυγατρὸς Σιών· ἀγαλλιάσομαι ἐπὶ τῷ σωτηρίῳ σου.

\vs{16}Ἐνεπάγησαν ἔθνη ἐν διαφθορᾷ ᾗ ἐποίησαν· ἐν παγίδι ταύτῃ ᾗ ἔκρυψαν συνελήφθη ὁ ποὺς αὐτῶν.
\vs{17}Γινώσκεται Κύριος κρίματα ποιῶν, ἐν τοῖς ἔργοις τῶν χειρῶν αὐτοῦ συνελήφθη ὁ ἁμαρτωλός· ᾠδὴ διαψάλματος.
\vs{18}Ἀποστραφήτωσαν οἱ ἁμαρτωλοὶ εἰς τὸν ᾅδην, πάντα τὰ ἔθνη τὰ ἐπιλανθανόμενα τοῦ Θεοῦ.
\vs{19}Ὅτι οὐκ εἰς τέλος ἐπιλησθήσεται ὁ πτωχὸς, ἡ ὑπομονὴ τῶν πενήτων οὐκ ἀπολεῖται εἰς τὸν αἰῶνα.
\vs{20}Ἀνάστηθι Κύριε, μὴ κραταιούσθω ἄνθρωπος, κριθήτωσαν ἔθνη ἐνώπιόν σου.
\vs{21}Κατάστησον, Κύριε, νομοθέτην ἐπʼ αὐτοὺς, γνώτωσαν ἔθνη ὅτι ἀνθρωποί εἰσι· διάψαλμα.

\vs{22}Ἱνατί, Κύριε, ἀφέστηκας μακρόθεν, ὑπερορᾷς ἐν εὐκαιρίαις, ἐν θλίψει;
\vs{23}Ἐν τῷ ὑπερηφανεύεσθαι τὸν ἀσεβῆ, ἐμπυρίζεται ὁ πτωχὸς, συλλαμβάνονται ἐν διαβουλίοις οἷς διαλογίζονται.
\vs{24}Ὅτι ἐπαινεῖται ὁ ἁμαρτωλὸς ἐν ταῖς ἐπιθυμίαις τῆς ψυχῆς αὐτοῦ, καὶ ὁ ἀδικῶν ἐνευλογεῖται.
\vs{25}Παρώξυνε τὸν Κύριον ὁ ἁμαρτωλὸς, κατὰ τὸ πλῆθος τῆς ὀργῆς αὐτοῦ οὐκ ἐκζητήσει· οὐκ ἔστιν ὁ Θεὸς ἐνώπιον αὐτοῦ.
\vs{26}Βεβηλοῦνται αἱ ὁδοὶ αὐτοῦ ἐν παντὶ καιρῷ· ἀνταναιρεῖται τὰ κρίματά σου ἀπὸ προσώπου αὐτοῦ, πάντων τῶν ἐχθρῶν αὐτοῦ κατακυριεύσει.
\vs{27}Εἶπε γὰρ ἐν καρδίᾳ αὐτοῦ, οὐ μὴ σαλευθῶ ἀπὸ γενεᾶς εἰς γενεὰν ἄνευ κακοῦ.
\vs{28}Οὗ ἀρᾶς τὸ στόμα αὐτοῦ γέμει καὶ πικρίας καὶ δόλου, ὑπὸ τὴν γλῶσσαν αὐτοῦ κόπος καὶ πόνος.
\vs{29}Ἐγκάθηται ἔνεδρα μετὰ πλουσίων ἐν ἀποκρύφοις, τοῦ ἀποκτεῖναι ἀθῶον· οἱ ὀφθαλμοὶ αὐτοῦ εἰς τὸν πένητα ἀποβλέπουσιν.
\vs{30}Ἐνεδρεύει ἐν ἀποκρύφῳ ὡς λέων ἐν τῇ μάνδρᾳ αὐτοῦ· ἐνεδρεύει τοῦ ἁρπάσαι πτωχὸν, ἁρπάσαι πτωχὸν ἐν τῷ ἑλκύσαι αὐτόν· ἐν τῇ παγίδι αὐτοῦ
\vs{31}ταπεινώσει αὐτὸν, κύψει καὶ πεσεῖται ἐν τῷ αὐτὸν κατακυριεῦσαι τῶν πενήτων.
\vs{32}Εἶπε γὰρ ἐν τῇ καρδίᾳ αὐτοῦ, ἐπιλέλησται ὁ Θεὸς, ἀπέστρεψε τὸ πρόσωπον αὐτοῦ τοῦ μὴ βλέπειν εἰς τέλος.

\vs{33}Ἀνάστηθι Κύριε ὁ Θεὸς, ὑψωθήτω ἡ χείρ σου, μὴ ἐπιλάθῃ τῶν πενήτων.
\vs{34}Ἕνεκεν τίνος παρώξυνεν ὁ ἀσεβὴς τὸν Θεόν; εἶπε γὰρ ἐν καρδίᾳ αὐτοῦ, οὐ ζητήσει.
\vs{35}Βλέπεις, ὅτι σὺ πόνον καὶ θυμὸν κατανοεῖς, τοῦ παραδοῦναι αὐτοὺς εἰς χεῖράς σου· σοὶ ἐγκαταλέλειπται ὁ πτωχὸς, ὀρφανῷ σὺ ἦσθα βοηθός.
\vs{36}Σύντριψον τὸν βραχίονα τοῦ ἁμαρτωλοῦ καὶ πονηροῦ, ζητηθήσεται ἡ ἁμαρτία αὐτοῦ καὶ οὐ μὴ εὑρεθῇ.

\vs{37}Βασιλεύσει Κύριος εἰς τὸν αἰῶνα, καὶ εἰς τὸν αἰῶνα τοῦ αἰῶνος, ἀπολεῖσθε ἔθνη ἐκ τῆς γῆς αὐτοῦ.
\vs{38}Τὴν ἐπιθυμίαν τῶν πενήτων εἰσήκουσε Κύριος, τὴν ἑτοιμασίαν τῆς καρδίας αὐτῶν προσέσχε τὸ οὖς σου·
\vs{39}Κρῖναι ὀρφανῷ καὶ ταπεινῷ, ἵνα μὴ προσθῇ ἔτι μεγαλαυχεῖν ἄνθρωπος ἐπὶ τῆς γῆς.

\begin{psalmheading}{\ch{10}{11} Εἰς τὸ τέλος, ψαλμὸς τῷ Δαυὶδ.}
\end{psalmheading}
Ἐπὶ τῷ Κυρίῳ πέποιθα· πῶς ἐρεῖτε τῇ ψυχῇ μου, μεταναστεύου ἐπὶ τὰ ὄρη ὡς στρουθίον;
\vs{2}Ὅτι ἰδοὺ οἱ ἁμαρτωλοὶ ἐνέτειναν τόξον, ἡτοίμασαν βέλη εἰς φαρέτραν, τοῦ κατατοξεῦσαι ἐν σκοτομήνῃ τοὺς εὐθεῖς τῇ καρδίᾳ.
\vs{3}Ὅτι ἃ κατηρτίσω καθεῖλον, ὁ δὲ δίκαιος τί ἐποίησε;

\vs{4}Κύριος ἐν ναῷ ἁγίῳ αὐτοῦ, Κύριος, ἐν οὐρανῷ ὁ θρόνος αὐτοῦ· οἱ ὀφθαλμοὶ αὐτοῦ εἰς τὸν πένητα ἀποβλέπουσι, τὰ βλέφαρα αὐτοῦ ἐξετάζει τοὺς υἱοὺς τῶν ἀνθρώπων·
\vs{5}Κύριος ἐξετάζει τὸν δίκαιον καὶ τὸν ἀσεβῆ, ὁ δὲ ἀγαπῶν ἀδικίαν μισεῖ τὴν ἑαυτοῦ ψυχήν.
\vs{6}Ἐπιβρέξει ἐπὶ ἁμαρτωλοὺς παγίδας, πῦρ καὶ θεῖον καὶ πνεῦμα καταιγίδος ἡ μερὶς τοῦ ποτηρίου αὐτῶν.
\vs{7}Ὅτι δίκαιος Κύριος καὶ δικαιοσύνας ἠγάπησεν, εὐθύτητα εἶδε τὸ πρόσωπον αὐτοῦ.

\begin{psalmheading}{\ch{11}{12} Εἰς τὸ τέλος, ὑπὲρ τῆς ὀγδόης, ψαλμὸς τῷ Δαυίδ.}
\end{psalmheading}
\vs{2}Σῶσον με Κύριε, ὅτι ἐκλέλοιπεν ὅσιος, ὅτι ὠλιγώθησαν αἱ ἀλήθειαι ἀπὸ τῶν υἱῶν τῶν ἀνθρώπων.
\vs{3}Μάταια ἐλάλησεν ἕκαστος πρὸς τὸν πλησίον αὐτοῦ, χείλη δόλια, ἐν καρδίᾳ καὶ ἐν καρδίᾳ ἐλάλησαν.
\vs{4}Ἐξολοθρεύσαι Κύριος πάντα τὰ χείλη τὰ δόλια, καὶ γλῶσσαν μεγαλοῤῥήμονα·
\vs{5}Τοὺς εἰπόντας, τὴν γλῶσσαν ἡμῶν μεγαλυνοῦμεν, τὰ χείλη ἡμῶν παρʼ ἡμῶν ἐστι· τίς ἡμῶν Κύριός ἐστιν;

\vs{6}Ἀπὸ τῆς ταλαιπωρίας τῶν πτωχῶν, καὶ ἀπὸ τοῦ στεναγμοῦ τῶν πενήτων, νῦν ἀναστήσομαι, λέγει Κύριος· θήσομαι ἐν σωτηρίῳ, παῤῥησιάσομαι ἐν αὐτῷ.
\vs{7}Τὰ λόγια Κυρίου, λόγια ἁγνά· ἀργύριον πεπυρωμένον, δοκίμιον τῇ γῇ, κεκαθαρισμένον ἑπταπλασίως.
\vs{8}Σὺ Κύριε φυλάξεις ἡμᾶς· καὶ διατηρήσεις ἡμᾶς ἀπὸ τῆς γενεᾶς ταύτης, καὶ εἰς τὸν αἰῶνα.
\vs{9}Κύκλῳ οἱ ἀσεβεῖς περιπατοῦσι, κατὰ τὸ ὕψος σου ἐπολυώρησας τοὺς υἱοὺς τῶν ἀνθρώπων.

\begin{psalmheading}{\ch{12}{13} Εἰς τὸ τέλος, ψαλμὸς τῷ Δαυίδ.}
\end{psalmheading}
Ἕως πότε Κύριε ἐπιλήσῃ μου, εἰς τέλος; ἕως πότε ἀποστρέψεις τὸ πρόσωπόν σου ἀπʼ ἐμοῦ;
\vs{2}Ἕως τίνος θήσομαι βουλὰς ἐν ψυχῇ μου, ὀδύνας ἐν καρδίᾳ μου ἡμέρας; ἕως πότε ὑψωθήσεται ὁ ἐχθρός μου ἐπʼ ἐμέ;
\vs{3}Ἐπίβλεψον, εἰσάκουσόν μου, Κύριε ὁ Θεός μου· φώτισον τοὺς ὀφθαλμούς μου, μή ποτε ὑπνώσω εἰς θάνατον·
\vs{4}μή ποτε εἴποι ὁ ἐχθρός μου, ἴσχυσα πρὸς αὐτόν· οἱ θλίβοντές με ἀγαλλιάσονται ἐὰν σαλευθῶ.

\vs{5}Ἐγὼ δὲ ἐπὶ τῷ ἐλέει σου ἤλπισα· ἀγαλλιάσεται ἡ καρδία μου ἐν τῷ σωτηρίῳ σου.
\vs{6}Ἄσω τῷ Κυρίῳ τῷ εὐεργετήσαντί με, καὶ ψαλῶ τῷ ὀνόματι Κυρίου τοῦ ὑψίστου.

\begin{psalmheading}{\ch{13}{14} Εἰς τὸ τέλος, ψαλμὸς τῷ Δαυίδ.}
\end{psalmheading}
Εἶπεν ἄφρων ἐν καρδίᾳ αὐτοῦ, οὐκ ἔστι Θεός· διέφθειραν καὶ ἐβδελύχθησαν ἐν ἐπιτηδεύμασιν, οὐκ ἔστι ποιῶν χρηστότητα, οὐκ ἔστιν ἕως ἑνός.
\vs{2}Κύριος ἐκ τοῦ οὐρανοῦ διέκυψεν ἐπὶ τοὺς υἱοὺς τῶν ἀνθρώπων, τοῦ ἰδεῖν εἰ ἔστι συνιὼν ἢ ἐκζητῶν τὸν Θεόν.
\vs{3}Πάντες ἐξέκλιναν, ἅμα ἠχρειώθησαν, οὐκ ἔστι ποιῶν χρηστότητα, οὐκ ἔστιν ἕως ἑνός· τάφος ἀνεῳγμένος ὁ λάρυγξ αὐτῶν, ταῖς γλώσσαις αὐτῶν ἐδολιοῦσαν, ἰὸς ἀσπίδων ὑπὸ τὰ χείλη αὐτῶν· ὧν τὸ στόμα ἀρᾶς καὶ πικρίας γέμει, ὀξεῖς οἱ πόδες αὐτῶν ἐκχέαι αἷμα· σύντριμμα καὶ ταλαιπωρία ἐν ταῖς ὁδοῖς αὐτῶν, καὶ ὁδὸν εἰρήνης οὐκ ἔγνωσαν· οὐκ ἔστι φόβος Θεοῦ ἀπέναντι τῶν ὀφθαλμῶν αὐτῶν.

\vs{4}Οὐχὶ γνώσονται πάντες οἱ ἐργαζόμενοι τὴν ἀνομίαν, οἱ κατέσθοντες τὸν λαόν μου βρώσει ἄρτου; τὸν Κύριον οὐκ ἐπεκαλέσαντο.
\vs{5}Ἐκεῖ ἐδειλίασαν φόβῳ, οὗ οὐκ ἦν φόβος, ὅτι ὁ Θεὸς ἐν γενεᾷ δικαίᾳ.
\vs{6}Βουλὴν πτωχοῦ κατῃσχύνατε, ὅτι Κύριος ἐλπὶς αὐτοῦ ἐστι.
\vs{7}Τίς δώσει ἐκ Σιὼν τὸ σωτήριον τοῦ Ἰσραήλ; ἐν τῷ ἐπιστρέψαι Κύριον τὴν αἰχμαλωσίαν τοῦ λαοῦ αὐτοῦ, ἀγαλλιάσθω Ἰακὼβ, καὶ εὐφρανθήτω Ἰσραήλ.

\begin{psalmheading}{\ch{14}{15} Ψαλμὸς τῷ Δαυίδ.}
\end{psalmheading}
Κύριε, τίς παροικήσει ἐν τῷ σκηνώματί σου; καὶ τίς κατασκηνώσει ἐν τῷ ὄρει τῷ ἁγίῳ σου;

\vs{2}Πορευόμενος ἄμωμος, καὶ ἐργαζόμενος δικαιοσύνην· λαλῶν ἀλήθειαν ἐν καρδίᾳ αὐτοῦ·
\vs{3}Ὃς οὐκ ἐδόλωσεν ἐν γλώσσῃ αὐτοῦ, οὐδὲ ἐποίησε τῷ πλησίον αὐτοῦ κακὸν, καὶ ὀνειδισμὸν οὐκ ἔλαβεν ἐπὶ τοὺς ἔγγιστα αὐτοῦ·
\vs{4}Ἐξουδένωται ἐνώπιον αὐτοῦ πονηρευόμενος, τοὺς δὲ φοβουμένους Κύριον δοξάζει· ὁ ὀμνύων τῷ πλησίον αὐτοῦ καὶ οὐκ ἐθετῶν·
\vs{5}Τὸ ἀργύριον αὐτοῦ οὐκ ἔδωκεν ἐπὶ τόκῳ, καὶ δῶρα ἐπʼ ἀθώοις οὐκ ἔλαβεν· ὁ ποιῶν ταῦτα, οὐ σαλευθήσεται εἰς τὸν αἰῶνα.

\begin{psalmheading}{\ch{15}{16} Στηλογραφία τῷ Δαυίδ.}
\end{psalmheading}
Φύλαξον με Κύριε, ὅτι ἐπὶ σοὶ ἤλπισα.
\vs{2}Εἶπα τῷ Κυρίῳ, Κύριός μου εἶ σὺ, ὅτι τῶν ἀγαθῶν μου οὐ χρείαν ἔχεις.
\vs{3}Τοῖς ἁγίοις τοῖς ἐν τῇ γῇ αὐτοῦ, ἐθαυμάστωσε πάντα τὰ θελήματα αὐτοῦ ἐν αὐτοῖς.
\vs{4}Ἐπληθύνθησαν αἱ ἀσθένειαι αὐτῶν, μετὰ ταῦτα ἐτάχυναν· οὐ μὴ συναγάγω τὰς συναγωγὰς αὐτῶν ἐξ αἱμάτων, οὐδὲ μὴ μνησθῶ τῶν ὀνομάτων αὐτῶν διὰ χειλέων μου.
\vs{5}Κύριος μερὶς τῆς κληρονομίας μου καὶ τοῦ ποτηρίου μου, σὺ εἶ ὁ ἀποκαθιστῶν τὴν κληρονομίαν μου ἐμοί.
\vs{6}Σχοινία ἐπέπεσάν μοι ἐν τοῖς κρατίστοις, καὶ γὰρ ἡ κληρονομία μου κρατίστη μοι ἐστίν.

\vs{7}Εὐλογήσω τὸν Κύριον τὸν συνετίσαντά με, ἔτι δὲ καὶ ἕως νυκτὸς ἐπαίδευσάν με οἱ νεφροί μου.
\vs{8}Προωρώμην τὸν Κύριον ἐνώπιόν μου διαπαντὸς, ὅτι ἐκ δεξιῶν μου ἐστὶν ἵνα μὴ σαλευθῶ.
\vs{9}Διὰ τοῦτο ηὐφράνθη ἡ καρδία μου, καὶ ἠγαλλιάσατο ἡ γλῶσσά μου, ἔτι δὲ καὶ ἡ σάρξ μου κατασκηνώσει ἐπʼ ἐλπίδι·
\vs{10}Ὅτι οὐκ ἐγκαταλείψεις τὴν ψυχήν μου εἰς ᾅδην, οὐδὲ δώσεις τὸν ὅσιόν σου ἰδεῖν διαφθοράν.
\vs{11}Ἐγνώρισάς μοι ὁδοὺς ζωῆς, πληρώσεις με εὐφροσύνης μετὰ τοῦ προσώπου σου, τερπνότητες ἐν τῇ δεξιᾷ σου εἰς τέλος.

\begin{psalmheading}{\ch{16}{17} Προσευχὴ τοῦ Δαυίδ.}
\end{psalmheading}
Εἰσάκουσον Κύριε τῆς δικαιοσύνης μου, πρόσχες τῇ δεήσει μου· ἐνώτισαι τὴν προσευχήν μου οὐκ ἐν χείλεσι δολίοις.
\vs{2}Ἐκ προσώπου σου τὸ κρίμα μου ἐξέλθοι, οἱ ὀφθαλμοί μου ἰδέτωσαν εὐθύτητας.
\vs{3}Ἐδοκίμασας τὴν καρδίαν μου, ἐπεσκέψω νυκτὸς, ἐπύρωσάς με, καὶ οὐχ εὑρέθη ἐν ἐμοὶ ἀδικία·
\vs{4}ὅπως ἂν μὴ λαλήσῃ τὸ στόμα μου. Τὰ ἔργα τῶν ἀνθρώπων, διὰ τοὺς λόγους τῶν χειλέων σου ἐγὼ ἐφύλαξα ὁδοὺς σκληράς.
\vs{5}Κατάρτισαι τὰ διαβήματά μου ἐν ταῖς τρίβοις σου, ἵνα μὴ σαλευθῇ τὰ διαβήματά μου.

\vs{6}Ἐγὼ ἐκεκραξα, ὅτι ἐπήκουσας μου ὁ Θεός· κλῖνον τὸ οὖς σου ἐμοὶ, καὶ εἰσάκουσον τῶν ῥημάτων μου.
\vs{7}Θαυμάστωσον τὰ ἐλέη σου, ὁ σώζων τοὺς ἐλπίζοντας ἐπὶ σέ· ἐκ τῶν ἀνθεστηκότων τῇ δεξιᾷ σου,
\vs{8}φύλαξόν με ὡς κόρην ὀφθαλμοῦ· ἐν σκέπῃ τῶν πτερύγων σου σκεπάσεις με,
\vs{9}ἀπὸ προσώπου ἀσεβῶν τῶν ταλαιπωρησάντων με· οἱ ἐχθροί μου τὴν ψυχήν μου περιέσχον.
\vs{10}Τὸ στέαρ αὐτῶν συνέκλεισαν, τὸ στόμα αὐτῶν ἐλάλησεν ὑπερηφανίαν.
\vs{11}Ἐκβαλόντες με νυνὶ περιεκύκλωσάν με, τοὺς ὀφθαλμοὺς αὐτὼν ἔθεντο ἐκκλῖναι ἐν τῇ γῇ.
\vs{12}Ὑπέλαβόν με ὡσεὶ λέων ἕτοιμος εἰς θήραν, καὶ ὡσεὶ σκύμνος οἰκῶν ἐν ἀποκρύφοις.
\vs{13}Ἀνάστηθι Κύριε, πρόφθασον αὐτοὺς, καὶ ὑποσκέλισον αὐτοὺς, ῥῦσαι τὴν ψυχήν μου ἀπὸ ἀσεβοῦς, ῥομφαίαν σου
\vs{14}ἀπὸ ἐχθρῶν τῆς χειρός σου· Κύριε ἀπολύων ἀπὸ γῆς, διαμέρισον αὐτοὺς ἐν τῇ ζωῇ αὐτῶν, καὶ τῶν κεκρυμμένων σου ἐπλήσθη ἡ γαστὴρ αὐτῶν· ἐχορτάσθησαν ὑείων, καὶ ἀφῆκαν τὰ κατάλοιπα τοῖς νηπίοις αὐτῶν.

\vs{15}Ἐγὼ δὲ ἐν δικαιοσύνῃ ὀφθήσομαι τῷ προσώπῳ σου, χορτασθήσομαι ἐν τῷ ὀφθῆναι τὴν δόξαν σου.

\begin{psalmheading}{\ch{17}{18} Εἰς τὸ τέλος τῷ παιδὶ Κυρίου τῷ Δαυὶδ, ἃ ἐλάλησε τῷ Κυρίῳ, τοὺς λόγους τῆς ᾠδῆς ταύτης, ἐν ἡμέρᾳ ᾗ ἐῤῥύσατο αὐτὸν Κύριος ἐκ χειρὸς πάντων τῶν ἐχθρῶν αὐτοῦ, καὶ ἐκ χειρὸς Σαοὺλ, καὶ εἶπεν,}
\end{psalmheading}
\vs{2}Αγαπήσω σε, Κύριε ἰσχύς μου.
\vs{3}Κύριος στερέωμά μου, καὶ καταφυγή μου, καὶ ῥύστης μου· ὁ Θεός μου βοηθός μου, ἐλπιῶ ἐπʼ αὐτόν· ὑπερασπιστής μου, καὶ κέρας σωτηρίας μου, καὶ ἀντιλήπτωρ μου.
\vs{4}Αἰνῶν ἐπικαλέσομαι Κύριον, καὶ ἐκ τῶν ἐχθρῶν μου σωθήσομαι.
\vs{5}Περιέσχον με ὠδῖνες θανάτου, καὶ χείμαῤῥοι ἀνομίας ἐξετάραξάν με.
\vs{6}Ὠδῖνες ᾅδου περιεκύκλωσάν με, προέφθασάν με παγίδες θανάτου.

\vs{7}Καὶ ἐν τῷ θλίβεσθαί με ἐπεκαλεσάμην τὸν Κύριον, καὶ πρὸς τὸν Θεόν μου ἐκέκραξα· ἤκουσεν ἐκ ναοῦ ἁγίου αὐτοῦ φωνῆς μου. καὶ ἡ κραυγή μου ἐνώπιον αὐτοῦ εἰσελεύσεται εἰς τὰ ὦτα αὐτοῦ.

\vs{8}Καὶ ἐσαλεύθη, καὶ ἔντρόμος ἐγενήθη ἡ γῆ, καὶ τὰ θεμέλια τῶν ὀρέων ἐταράχθησαν, καὶ ἐσαλεύθησαν, ὅτι ὠργίσθη αὐτοῖς ὁ Θεός.
\vs{9}Ἀνέβη καπνὸς ἐν ὀργῇ αὐτοῦ, καὶ πῦρ ἀπὸ προσώπου αὐτοῦ κατεφλόγισεν, ἄνθρακες ἀνήφθησαν ἀπʼ αὐτοῦ.
\vs{10}Καὶ ἔκλινεν οὐρανὸν καὶ κατέβη, καὶ γνόφος ὑπὸ τοὺς πόδας αὐτοῦ.
\vs{11}Καὶ ἐπέβη ἐπὶ χερουβὶμ καὶ ἐπετάσθη, ἐπετάσθη ἐπὶ πτερύγων ἀνέμων.
\vs{12}Καὶ ἔθετο σκότος ἀποκρυφὴν αὐτοῦ, κύκλῳ αὐτοῦ ἡ σκηνὴ αὐτοῦ, σκοτεινὸν ὕδωρ ἐν νεφέλαις ἀέρων.
\vs{13}Ἀπὸ τῆς τηλαυγήσεως ἐνώπιον αὐτοῦ αἱ νεφέλαι διῆλθον, χάλαζα καὶ ἄνθρακες πυρός.
\vs{14}Καὶ ἐβρόντησεν ἐξ οὐρανοῦ Κύριος, καὶ ὁ ὕψιστος ἔδωκε φωνὴν αὐτοῦ.
\vs{15}Καὶ ἐξαπέστειλε βέλη καὶ ἐσκόρπισεν αὐτοὺς, καὶ ἀστραπὰς ἐπλήθυνε καὶ συνετάραξεν αὐτούς.
\vs{16}Καὶ ὤφθησαν αἱ πηγαὶ τῶν ὑδάτων, καὶ ἀνεκαλύφθη τὰ θεμέλια τῆς οἰκουμένης· ἀπὸ ἐπιτιμήσεώς σου Κύριε, ἀπὸ ἐνπνεύσεως πνεύματος ὀργῆς σου.

\vs{17}Ἐξαπέστειλεν ἐξ ὕψους καὶ ἔλαβέ με, προσελάβετό με ἐξ ὑδάτων πολλῶν.
\vs{18}Ῥύσεταί με ἐξ ἐχθρῶν μου δυνατῶν, καὶ ἐκ τῶν μισούντων με, ὅτι ἐστερεώθησαν ὑπὲρ ἐμέ.
\vs{19}Προέφθασάν με ἐν ἡμέρᾳ κακώσεώς μου, καὶ ἐγένετο Κύριος ἀντιστήριγμά μου.
\vs{20}Καὶ ἐξήγαγέ με εἰς πλατυσμὸν, ῥύσεταί με, ὅτι ἠθέλησέ με.
\vs{21}Καὶ ἀνταποδώσει μοι Κύριος κατὰ τὴν δικαιοσύνην μου, καὶ κατὰ τὴν καθαριότητα τῶν χειρῶν μου ἀνταποδώσει μοι.
\vs{22}Ὅτι ἐφύλαξα τὰς ὁδοὺς Κυρίου, καὶ οὐκ ἠσέβησα ἀπὸ τοῦ Θεοῦ μου.
\vs{23}Ὅτι πάντα τὰ κρίματα αὐτοῦ ἐνώπιόν μου, καὶ τὰ δικαιώματα αὐτοῦ οὐκ ἀπέστησαν ἀπʼ ἐμοῦ.
\vs{24}Καὶ ἔσομαι ἄμωμος μετʼ αὐτοῦ, καὶ φυλάξομαι ἀπὸ τῆς ἀνομίας μου.
\vs{25}Καὶ ἀνταποδώσει μοι Κύριος κατὰ τὴν δικαιοσύνην μου, καὶ κατὰ τὴν καθαριότητα τῶν χειρῶν μου ἐνώπιον τῶν ὀφθαλμῶν αὐτοῦ.

\vs{26}Μετὰ ὁσίου ὁσιωθήσῃ, καὶ μετὰ ἀνδρὸς ἀθώου ἀθῶος ἔσῃ·
\vs{27}Καὶ μετὰ ἐκλεκτοῦ ἐκλεκτὸς ἔσῃ, καὶ μετὰ στρεβλοῦ διαστρέψεις.
\vs{28}Ὅτι σὺ λαὸν ταπεινὸν σώσεις, καὶ ὀφθαλμοὺς ὑπερηφάνων ταπεινώσεις.
\vs{29}Ὅτι σὺ φωτιεῖς λύχνον μου Κύριε, ὁ Θεός μου φωτιεῖς τὸ σκότος μου.
\vs{30}Ὅτι ἐν σοὶ ῥυσθήσομαι ἀπὸ πειρατηρίου, καὶ ἐν τῷ Θεῷ μου ὑπερβήσομαι τεῖχος.
\vs{31}Ὁ Θεός μου, ἄμωμος ἡ ὁδὸς αὐτοῦ, τὰ λόγια Κυρίου πεπυρωμένα, ὑπερασπιστής ἐστι πάντων τῶν ἐλπιζόντων ἐπʼ αὐτόν.
\vs{32}Ὅτι τίς Θεὸς πλὴν τοῦ Κυρίου; καὶ τίς Θεὸς πλὴν τοῦ Θεοῦ ἡμῶν;

\vs{33}Ὁ Θεὸς ὁ περιζωννύων με δύναμιν, καὶ ἔθετο ἄμωμον τὴν ὁδόν μου·
\vs{34}ὁ καταρτιζόμενος τοὺς πόδας μου ὡσεὶ ἐλάφου, καὶ ἐπὶ τὰ ὑψηλὰ ἱστῶν με·
\vs{35}Διδάσκων χεῖράς μου εἰς πόλεμον· καὶ ἔθου τόξον χαλκοῦν τοὺς βραχίονάς μου,
\vs{36}καὶ ἔδωκάς με ὑπερασπισμὸν σωτηρίας μου· καὶ ἡ δεξιά σου ἀντελάβετό μου, καὶ ἡ παιδεία σου ἀνώρθωσέ με εἰς τέλος, καὶ ἡ παιδεία σου αὐτή με διδάξει.
\vs{37}Ἐπλάτυνας τὰ διαβήματά μου ὑποκάτω μου, καὶ οὐκ ἠσθένησαν τὰ ἴχνη μου.
\vs{38}Καταδιώξω τοὺς ἐχθρούς μου, καὶ καταλήψομαι αὐτοὺς, καὶ οὐκ ἀποστραφήσομαι, ἕως ἂν ἐκλείπωσιν.
\vs{39}Ἐκθλίψω αὐτοὺς, καὶ οὐ μὴ δύνωνται στῆναι, πεσοῦνται ὑπὸ τοὺς πόδας μου.
\vs{40}Καὶ περιέζωσάς με δύναμιν εἰς πόλεμον, συνεπόδισας πάντας τοὺς ἐπανισταένους ἐπʼ ἐμὲ ὑποκάτω μου.
\vs{41}Καὶ τοὺς ἐχθρούς μου ἔδωκάς μοι νῶτον, καὶ τοὺς μισοῦντάς με ἐξωλόθρευσας.
\vs{42}Ἐκέκραξαν, καὶ οὐκ ἦν ὁ σώζων· πρὸς Κύριον, καὶ οὐκ εἰσήκουεν αὐτῶν.
\vs{43}Καὶ λεπτυνῶ αὐτοὺς ὡς χοῦν κατὰ πρόσωπον ἀνέμου, ὡς πηλὸν πλατειῶν λεανῶ αὐτούς.
\vs{44}Ῥῦσαί με ἐξ ἀντιλογιῶν λαοῦ, καταστήσεις με εἰς κεφαλὴν ἐθνῶν· λαὸς ὃν οὐκ ἔγνων, ἐδούλευσέ μοι,
\vs{45}εἰς ἀκοὴν ὠτίου ὑπήκουσέ μοι· υἱοὶ ἀλλότριοι ἐψεύσαντό μοι,
\vs{46}υἱοὶ ἀλλότριοι ἐπαλαιώθησαν, καὶ ἐχώλαναν ἀπὸ τῶν τρίβων αὐτῶν.

\vs{47}Ζῇ Κύριος, καὶ εὐλογητὸς ὁ Θεός μου, καὶ ὑψωθήτω ὁ Θεὸς τῆς σωτηρίας μου.
\vs{48}Ὁ Θεὸς ὁ διδοὺς ἐκδικήσεις ἐμοὶ, καὶ ὑποτάξας λαοὺς ὑπʼ ἐμὲ,
\vs{49}ὁ ῥύστης μου ἐξ ἐχθρῶν ὀργίλων· ἀπὸ τῶν ἐπανισταμένων ἐπʼ ἐμὲ ὑψώσεις με, ἀπὸ ἀνδρὸς ἀδίκου ῥύσῃ με.
\vs{50}Διὰ τοῦτο ἐξομολογήσομαί σοι ἐν ἔθνεσι, Κύριε, καὶ τῷ ὀνόματί σου ψαλῶ.
\vs{51}Μεγαλύνων τὰς σωτηρίας τοῦ βασιλέως αὐτοῦ, καὶ ποιῶν ἔλεος τῷ χριστῷ αὐτοῦ τῷ Δαυὶδ, καὶ τῷ σπέρματι αὐτοῦ ἕως αἰῶνος.

\begin{psalmheading}{\ch{18}{19} Εἰς τὸ τέλος, ψαλμὸς τῷ Δαυίδ.}
\end{psalmheading}
\vs{2}Οἱ οὐρανοὶ διηγοῦνται δόξαν Θεοῦ, ποίησιν δὲ χειρῶν αὐτοῦ ἀναγγέλλει τὸ στερέωμα.
\vs{3}Ἡμέρα τῇ ἡμέρᾳ ἐρεύγεται ῥῆμα, καὶ νὺξ νυκτὶ ἀναγγέλλει γνῶσιν.
\vs{4}Οὐκ εἰσὶ λαλιαὶ οὐδὲ λόγοι, ὧν οὐχὶ ἀκούονται αἱ φωναὶ αὐτῶν·
\vs{5}Εἰς πᾶσαν τὴν γῆν ἐξῆλθεν ὁ φθόγγος αὐτῶν, καὶ εἰς τὰ πέρατα τῆς οἰκουμένης τὰ ῥήματα αὐτῶν·
\vs{6}ἐν τῷ ἡλίῳ ἔθετο τὸ σκήνωμα αὐτοῦ, καὶ αὐτὸς ὡς νυμφίος ἐκπορευόμενος ἐκ παστοῦ αὐτοῦ· ἀγαλλιάσεται ὡς γίγας δραμεῖν ὁδὸν αὐτοῦ.
\vs{7}Ἀπʼ ἄκρου τοῦ οὐρανοῦ ἡ ἔξοδος αὐτοῦ· καὶ τὸ κατάντημα αὐτοῦ ἕως ἄκρου τοῦ οὐρανοῦ· καὶ οὐκ ἔστιν ὃς ἀποκρυβήσεται τὴν θέρμην αὐτοῦ.

\vs{8}Ὁ νόμος τοῦ Κυρίου ἄμωμος ἐπιστρέφων ψυχὰς, ἡ μαρτυρία Κυρίου πιστὴ σοφίζουσα νήπια.
\vs{9}Τὰ δικαιώματα Κυρίου εὐθέα εὐφραίνοντα καρδίαν, ἡ ἐντολὴ Κυρίου τηλαυγὴς φωτίζουσα ὀφθαλμούς.
\vs{10}Ὁ φόβος Κυρίου ἁγνὸς διαμένων εἰς αἰῶνα αἰῶνος, τὰ κρίματα Κυρίου ἀληθινὰ δεδικαιωμένα ἐπὶ τὸ αὐτό·
\vs{11}Ἐπιθυμητὰ ὑπὲρ χρυσίον καὶ λίθον τίμιον πολὺν, καὶ γλυκύτερα ὑπὲρ μέλι καὶ κηρίον.
\vs{12}Καὶ γὰρ ὁ δοῦλός σου φυλάσσει αὐτὰ, ἐν τῷ φυλάσσειν αὐτὰ ἀνταπόδοσις πολλή.

\vs{13}Παραπτώματα τίς συνήσει; ἐκ τῶν κρυφίων μου καθάρισόν με,
\vs{14}καὶ ἀπὸ ἀλλοτρίων φεῖσαι τοῦ δούλου σου· ἐὰν μή μου κατακυριεύσωσι, τότε ἄμωμος ἔσομαι, καὶ καθαρισθήσομαι ἀπὸ ἁμαρτίας μεγάλης.
\vs{15}Καὶ ἔσονται εἰς εὐδοκίαν τὸ λόγια τοῦ στόματός μου, καὶ ἡ μελέτη τῆς καρδίας μου ἐνώπιόν σου διαπαντός· Κύριε βοηθέ μου, καὶ λυτρωτά μου.

\begin{psalmheading}{\ch{19}{20} Εἰς τὸ τέλος, ψαλμὸς τῷ Δαυίδ.}
\end{psalmheading}
\vs{2}Ἐπακούσαι σου Κύριος ἐν ἡμέρᾳ θλίψεως, ὑπερασπίσαι σου τὸ ὄνομα τοῦ Θεοῦ Ἰακώβ.
\vs{3}Ἐξαποστείλαι σοι βοήθειαν ἐξ ἁγίου, καὶ ἐκ Σιὼν ἀντιλάβοιτό σου.
\vs{4}Μνησθείη πάσης θυσίας σου, καὶ τὸ ὁλοκαύτωμά σου πιανάτω· διάψαλμα.
\vs{5}Δῴη σοι κατὰ τὴν καρδίαν σου, καὶ πᾶσαν τὴν βουλήν σου πληρώσαι.
\vs{6}Ἀγαλλιασόμεθα ἐν τῷ σωτηρίῳ σου, καὶ ἐν ὀνόματι Θεοῦ ἡμῶν μεγαλυνθησόμεθα· πληρώσαι Κύριος πάντα τὰ αἰτήματά σου.

\vs{7}Νῦν ἔγνων ὅτι ἔσωσε Κύριος τὸν χριστὸν αὐτοῦ· ἐπακούσεται αὐτοῦ ἐξ οὐρανοῦ ἁγίου αὐτοῦ, ἐν δυναστείαις ἡ σωτηρία τῆς δεξιᾶς αὐτοῦ.
\vs{8}Οὗτοι ἐν ἅρμασι καὶ οὗτοι ἐν ἵπποις, ἡμεῖς δὲ ἐν ὀνόματι Κυρίου Θεοῦ ἡμῶν μεγαλυνθησόμεθα.
\vs{9}Αὐτοὶ συνεποδίσθησαν καὶ ἔπεσαν, ἡμεῖς δὲ ἀνέστημεν καὶ ἀνωρθώθημεν.
\vs{10}Κύριε σῶσον τὸν βασιλέα καὶ ἐπάκουσον ἡμῶν, ἐν ᾗ ἂν ἡμέρᾳ ἐπικαλεσώμεθά σε.

\begin{psalmheading}{\ch{20}{21} Εἰς τὸ τέλος, ψαλμὸς τῷ Δαυίδ.}
\end{psalmheading}
\vs{2}Κύριε, ἐν τῇ δυνάμει σου εὐφρανθήσεται ὁ βασιλεὺς, καὶ ἐπὶ τῷ σωτηρίῳ σου ἀγαλλιάσεται σφόδρα.
\vs{3}Τὴν ἐπιθυμίαν τῆς ψυχῆς αὐτοῦ ἔδωκας αὐτῷ, καὶ τὴν δέησιν τῶν χειλέων αὐτοῦ οὐκ ἐστέρησας αὐτόν· διάψαλμα.
\vs{4}Ὅτι προέφθασας αὐτὸν ἐν εὐλογίαις χρηστότητος, ἔθηκας ἐπὶ τὴν κεφαλὴν αὐτοῦ στέφανον ἐκ λίθου τιμίου.
\vs{5}Ζωὴν ᾐτήσατό σε, καὶ ἔδωκας αὐτῷ μακρότητα ἡμερῶν εἰς αἰῶνα αἰῶνος.
\vs{6}Μεγάλη ἡ δόξα αὐτοῦ ἐν τῷ σωτηρίῳ σου, δόξαν καὶ μεγαλοπρέπειαν ἐπιθήσεις ἐπʼ αὐτόν.
\vs{7}Ὅτι δώσεις αὐτῷ εὐλογίαν εἰς αἰῶνα αἰῶνος, εὐφρανεῖς αὐτὸν ἐν χαρᾷ μετὰ τοῦ προσώπου σου.
\vs{8}Ὅτι ὁ βασιλεὺς ἐλπίζει ἐπὶ Κύριον, καὶ ἐν τῷ ἐλέει τοῦ ὑψίστου οὐ μὴ σαλευθῇ.

\vs{9}Εὑρεθείη ἡ χείρ σου πᾶσι τοῖς ἐχθροῖς σου, ἡ δεξιά σου εὕροι πάντας τοὺς μισοῦντάς σε.
\vs{10}Θήσεις αὐτοὺς ὡς κλίβανον πυρὸς εἰς καιρὸν τοῦ προσώπου σου, Κύριος ἐν ὀργῇ αὐτοῦ συνταράξει αὐτοὺς, καὶ καταφάγεται αὐτοὺς πῦρ.
\vs{11}Τὸν καρπὸν αὐτῶν ἀπὸ γῆς ἀπολεῖς, καὶ τὸ σπέρμα αὐτῶν ἀπὸ υἱῶν ἀνθρώπων.
\vs{12}Ὅτι ἔκλιναν εἰς σὲ κακὰ, διελογίσαντο βουλὴν, ἣν οὐ μὴ δύνωνται στῆσαι.
\vs{13}Ὅτι θήσεις αὐτοὺς νῶτον ἐν τοῖς περιλοίποις σου, ἑτοιμάσεις τὸ πρόσωπον αὐτῶν.
\vs{14}Ὑψώθητι Κύριε ἐν τῇ δυνάμει σου· ᾄσομεν καὶ ψαλοῦμεν τὰς δυναστείας σου.

\begin{psalmheading}{\ch{21}{22} Εἰς τὸ τέλος, ὑπὲρ τῆς ἀντιλήψεως τῆς ἑωθινῆς, ψαλμὸς τῷ Δαυίδ.}
\end{psalmheading}
\vs{2}Ὁ Θεός ὁ Θεός μου, πρόσχες μοι, ἱνατί ἐγκατέλιπές με; μακρὰν ἀπὸ τῆς σωτηρίας μου οἱ λόγοι τῶν παραπτωμάτων μου.
\vs{3}Ο Θεός μου, κεκράξομαι ἡμέρας πρὸς σὲ καὶ οὐκ εἰσακούσῃ, καὶ νυκτὸς καὶ οὐκ εἰς ἄνοιαν ἐμοί.

\vs{4}Σὺ δὲ ἐν ἁγίῳ κατοικεῖς, ὁ ἔπαινος τοῦ Ἰσραήλ.
\vs{5}Ἐπὶ σοὶ ἤλπισαν οἱ πατέρες ἡμῶν, ἤλπισαν καὶ ἐῤῥύσω αὐτούς.
\vs{6}Πρὸς σὲ ἐκέκραξαν καὶ ἐσώθησαν, ἐπὶ σοὶ ἤλπισαν καὶ οὐ κατῃσχύνθησαν.
\vs{7}Ἐγὼ δέ εἰμι σκώληξ καὶ οὐκ ἄνθρωπος, ὄνειδος ἀνθρώπων καὶ ἐξουδένημα λαοῦ.
\vs{8}Πάντες οἱ θεωροῦντές με ἐξεμυκτήρισάν με, ἐλάλησαν ἐν χείλεσιν, ἐκίνησαν κεφαλὴν,
\vs{9}ἤλπισεν ἐπὶ Κύριον, ῥυσάσθω αὐτὸν, σωσάτω αὐτὸν, ὅτι θέλει αὐτόν.
\vs{10}Ὅτι σὺ εἶ ὁ ἐκσπάσας με ἐκ γαστρὸς, ἡ ἐλπίς μου ἀπὸ μαστῶν τῆς μητρός μου,
\vs{11}ἐπὶ σὲ ἐπεῤῥίφην ἐκ μήτρας· ἐκ κοιλίας μητρός μου Θεός μου εἶ σύ.

\vs{12}Μὴ ἀποστῇς ἀπʼ ἐμοῦ· ὅτι θλίψις ἐγγὺς, ὅτι οὐκ ἔστιν ὁ βοηθῶν.
\vs{13}Περιεκύκλωσάν με μόσχοι πολλοί, ταῦροι πίονες περιέσχον με.
\vs{14}Ἤνοιξαν ἐπʼ ἐμὲ τὸ στόμα αὐτῶν, ὡς λέων ὁ ἁρπάζων καὶ ὠρυόμενος.
\vs{15}Ὡσεὶ ὕδωρ ἐξεχύθην, καὶ διεσκορπίσθη πάντα τὰ ὀστᾶ μου, ἐγενήθη ἡ καρδία μου ὡσεὶ κηρὸς τηκόμενος ἐν μέσῳ τῆς κοιλίας μου.
\vs{16}Ἐξηράνθη ὡσεὶ ὄστρακον ἡ ἰσχύς μου, καὶ ἡ γλῶσσά μου κεκόλληται τῷ λάρυγγί μου, καὶ εἰς χοῦν θανάτου κατήγαγές με.
\vs{17}Ὅτι ἐκύκλωσάν με κύνες πολλοὶ, συναγωγὴ πονηρευομένων περιέσχον με· ὤρυξαν χεῖράς μου, καὶ πόδας,
\vs{18}ἐξηρίθμησαν πάντα τὰ ὀστᾶ μου· αὐτοὶ δὲ κατενόησαν καὶ ἐπεῖδόν με.
\vs{19}Διεμερίσαντο τὰ ἱμάτιά μου ἑαυτοῖς, καὶ ἐπὶ τὸν ἱματισμόν μου ἔβαλον κλῆρον.

\vs{20}Σὺ δὲ Κύριε μὴ μακρύνῃς τὴν βοήθειάν μου, εἰς τὴν ἀντίληψίν μου πρόσχες.
\vs{21}Ῥῦσαι ἀπὸ ῥομφαίας τὴν ψυχήν μου, καὶ ἐκ χειρὸς κυνὸς τὴν μονογενῆ μου.
\vs{22}Σῶσόν με ἐκ στόματος λέοντος, καὶ ἀπὸ κεράτων μονοκερώτων τὴν ταπείνωσίν μου.

\vs{23}Διηγήσομαι τὸ ὄνομά σου τοῖς ἀδελφοῖς μου, ἐν μέσῳ ἐκκλησίας ὑμνήσω σε.
\vs{24}Οἱ φοβούμενοι Κύριον αἰνέσατε αὐτὸν, ἅπαν τὸ σπέρμα Ἰακὼβ δοξάσατε αὐτὸν, φοβηθήτωσαν αὐτὸν ἅπαν τὸ σπέρμα Ἰσραήλ.
\vs{25}Οτι οὐκ ἐξουδένωσεν οὐδὲ προσώχθισε τῇ δεήσει τοῦ πτωχοῦ, οὐδὲ ἀπέστρεψε τὸ πρόσωπον αὐτοῦ ἀπʼ ἐμοῦ· καὶ ἐν τῷ κεκραγέναι με πρὸς αὐτὸν εἰσήκουσέ μου.
\vs{26}Παρὰ σοῦ ὁ ἔπαινός μου ἐν ἐκκλησίᾳ μεγάλῃ, τὰς εὐχάς μου ἀποδώσω ἐνώπιον τῶν φοβουμένων αὐτόν.

\vs{27}Φάγονται πένητες καὶ ἐμπλησθήσονται, καὶ αἰνέσουσι Κύριον οἱ ἐκζητοῦντες αὐτὸν, ζήσονται αἱ καρδίαι αὐτῶν εἰς αἰῶνα αἰῶνος.
\vs{28}Μνησθήσονται καὶ ἐπιστραφήσονται πρὸς Κύριον πάντα τὰ πέρατα τῆς γῆς, καὶ προσκυνήσουσιν ἐνώπιον αὐτοῦ πᾶσαι αἱ πατριαὶ τῶν ἐθνῶν.
\vs{29}Ὅτι τοῦ Κυρίου ἡ βασιλεία, καὶ αὐτὸς δεσπόζει τῶν ἐθνῶν.
\vs{30}Ἔφαγον καὶ προσεκύνησαν πάντες οἱ πίονες τῆς γῆς· ἐνώπιον αὐτοῦ προπεσοῦνται πάντες οἱ καταβαίνοντες εἰς τὴν γῆν· καὶ ἡ ψυχή μου αὐτῷ ζῇ,
\vs{31}καὶ τὸ σπέρμα μου δουλεύσει αὐτῷ· ἀναγγελήσεται τῷ Κυρίῳ γενεὰ ἡ ἐρχομένη·
\vs{32}Καὶ ἀναγγελοῦσι τὴν δικαιοσύνην αὐτοῦ λαῷ τῷ τεχθησομένῳ, ὃν ἐποίησεν ὁ Κύριος.

\begin{psalmheading}{\ch{22}{23} Ψαλμὸς τῷ Δαυίδ.}
\end{psalmheading}
Κύριος ποιμαίνει με, καὶ οὐδέν με ὑστερήσει.
\vs{2}Εἰς τόπον χλόης ἐκεῖ με κατεσκήνωσεν· ἐπὶ ὕδατος ἀναπαύσεως ἐξέθρεψέ με·
\vs{3}Τὴν ψυχήν μου ἐπέστρεψεν· ὡδήγησέν με ἐπὶ τρίβους δικαιοσύνης, ἕνεκεν τοῦ ὀνόματος αὐτοῦ.
\vs{4}Ἐὰν γὰρ καὶ πορευθῶ ἐν μέσῳ σκιᾶς θανάτου, οὐ φοβηθήσομαι κακὰ, ὅτι σὺ μετʼ ἐμοῦ εἶ· ἡ ῥάβδος σου καὶ ἡ βακτηρία σου, αὗταί με παρεκάλεσαν.
\vs{5}Ἡτοίμασας ἐνώπίον μου τράπεζαν, ἐξεναντίας τῶν θλιβόντων με· ἐλίπανας ἐν ἐλαίῳ τὴν κεφαλήν μου, καὶ τὸ ποτήριόν σου μεθύσκον ὡς κράτιστον.
\vs{6}Καὶ τὸ ἔλεός σου καταδιώξεταί με πάσας τὰς ἡμέρας τῆς ζωῆς μου, καὶ τὸ κατοικεῖν με ἐν οἴκῳ Κυρίου εἰς μακρότητα ἡμερῶν.

\begin{psalmheading}{\ch{23}{24} Ψαλμὸς τῷ Δαυὶδ τῆς μιᾶς σαββάτου.}
\end{psalmheading}
Τοῦ Κυρίου ἡ γῆ καὶ τὸ πλήρωμα αὐτῆς, ἡ οἰκουμένη καὶ πάντες οἱ κατοικοῦντες ἐν αὐτῇ.
\vs{2}Αὐτὸς ἐπὶ θαλασσῶν ἐθεμελιώσεν αὐτὴν, καὶ ἐπὶ ποταμῶν ἡτοίμασεν αὐτήν.

\vs{3}Τίς ἀναβήσεται εἰς τὸ ὄρος τοῦ Κυρίου, καὶ τίς στήσεται ἐν τόπῳ ἁγίῳ αὐτοῦ;
\vs{4}Ἀθῶος χερσὶ καὶ καθαρὸς τῇ καρδίᾳ, ὃς οὐκ ἔλαβεν ἐπὶ ματαίῳ τὴν ψυχὴν αὐτοῦ, καὶ οὐκ ὤμοσεν ἐπὶ δόλῳ τῷ πλησίον αὐτοῦ.
\vs{5}Οὗτος λήψεται εὐλογίαν παρὰ Κυρίου, καὶ ἐλεημοσύνην παρὰ Θεοῦ σωτῆρος αὐτοῦ.
\vs{6}Αὕτη ἡ γενεὰ ζητούντων αὐτὸν, ζητούντων τὸ πρόσωπον τοῦ Θεοῦ Ἰακώβ. διάψαλμα.

\vs{7}Ἄρατε πύλας οἱ ἄρχοντες ὑμῶν, καὶ ἐπάρθητε πύλαι αἰώνιοι, καὶ εἰσελεύσεται ὁ βασιλεὺς τῆς δόξης.
\vs{8}Τίς ἐστιν οὗτος ὁ βασιλεὺς τῆς δόξης; Κύριος κραταιὸς καὶ δυνατὸς, Κύριος δυνατὸς ἐν πολέμῳ.
\vs{9}Ἄρατε πύλας οἱ ἄρχοντες ὑμῶν, καὶ ἐπάρθητε πύλαι αἰώνιοι, καὶ εἰσελεύσεται ὁ βασιλεὺς τῆς δόξης.
\vs{10}Τίς ἐστιν οὗτος ὁ βασιλεὺς τῆς δόξης; Κύριος τῶν δυνάμεων, αὐτός ἐστιν οὗτος ὁ βασιλεὺς τῆς δόξης.

\begin{psalmheading}{\ch{24}{25} Ψαλμὸς τῷ Δαυίδ.}
\end{psalmheading}
Πρὸς σὲ, Κύριε, ᾖρα τὴν ψυχήν μου.
\vs{2}Ὁ Θεός μου ἐπὶ σοὶ πέποιθα, μὴ καταισχυνθείην· μηδὲ καταγελασάτωσάν μου οἱ ἐχθροί μου,
\vs{3}καὶ γὰρ πάντες οἱ ὑπομένοντές σε οὐ μὴ καταισχυνθῶσιν· αἰσχυνθήτωσαν οἱ ἀνομοῦντες διακενῆς.
\vs{4}Τὰς ὁδούς σου, Κύριε, γνώρισόν μοι, καὶ τὰς τρίβους σου δίδαξόν με.
\vs{5}Ὁδήγησόν με ἐπὶ τὴν ἀλήθειάν σου, καὶ δίδαξόν με, ὅτι σὺ εἶ ὁ Θεὸς ὁ σωτήρ μου, καὶ σὲ ὑπέμεινα ὅλην τὴν ἡμέραν.
\vs{6}Μνήσθητι τῶν οἰκτιρμῶν σου Κύριε, καὶ τὰ ἐλέη σου, ὅτι ἀπὸ τοῦ αἰῶνος εἰσίν.
\vs{7}Ἁμαρτίας νεότητός μου, καὶ ἀγνοίας μου μὴ μνησθῇς· κατὰ τὸ ἔλεός σου μνήσθητί μου, ἕνεκεν τῆς χρηστότητος σου, Κύριε.

\vs{8}Χρηστὸς καὶ εὐθὴς ὁ Κύριος, διὰ τοῦτο νομοθετήσει ἁμαρτάνοντας ἐν ὁδῷ.
\vs{9}Ὁδηγήσει πρᾳεῖς ἐν κρίσει, διδάξει πρᾳεῖς ὁδοὺς αὐτοῦ.
\vs{10}Πᾶσαι αἱ ὁδοὶ Κυρίου ἔλεος καὶ ἀλήθεια τοῖς ἐκζητοῦσι τὴν διαθήκην αὐτοῦ καὶ τὰ μαρτύρια αὐτοῦ.
\vs{11}Ἕνεκα τοῦ ὀνόματός σου, Κύριε, καὶ ἱλάσῃ τῇ ἁμαρτίᾳ μου, πολλὴ γάρ ἐστι.
\vs{12}Τίς ἐστιν ἄνθρωπος ὁ φοβούμενος τὸν Κύριον; νομοθετήσει αὐτῷ ἐν ὁδῷ, ᾗ ᾑρετίσατο.
\vs{13}Ἡ ψυχὴ αὐτοῦ ἐν ἀγαθοῖς αὐλισθήσεται, καὶ τὸ σπέρμα αὐτοῦ κληρονομήσει γῆν.
\vs{14}Κραταίωμα Κύριος τῶν φοβουμένων αὐτὸν, καὶ ἡ διαθήκη αὐτοῦ τοῦ δηλῶσαι αὐτοῖς.

\vs{15}Οἱ ὀφθαλμοί μου διαπαντὸς πρὸς τὸν Κύριον, ὅτι αὐτὸς ἐκσπάσει ἐκ παγίδος τοὺς πόδας μου.
\vs{16}Ἐπίβλεψον ἐπʼ ἐμὲ καὶ ἐλέησόν με, ὅτι μονογενὴς καὶ πτωχός εἰμι ἐγώ.
\vs{17}Αἱ θλίψεις τῆς καρδίας μου ἐπληθύνθησαν, ἐκ τῶν ἀναγκῶν μου ἐξάγαγέ με·
\vs{18}Ἴδε τὴν ταπείνωσίν μου καὶ τὸν κόπον μου, καὶ ἄφες πάσας τὰς ἁμαρτίας μου.
\vs{19}Ἴδε τοὺς ἐχθρούς μου, ὅτι ἐπληθύνθησαν, καὶ μῖσος ἄδικον ἐμίσησάν με.
\vs{20}Φύλαξον τὴν ψυχήν μου καὶ ῥῦσαί με· μὴ καταισχυνθείην, ὅτι ἤλπισα ἐπὶ σέ.
\vs{21}Ἄκακοι καὶ εὐθεῖς ἐκολλῶντό μοι, ὅτι ὑπέμεινά σε, Κύριε.
\vs{22}Λύτρωσαι ὁ Θεὸς τὸν Ἰσραὴλ ἐκ πασῶν τῶν θλίψεων αὐτοῦ.

\begin{psalmheading}{\ch{25}{26} Τοῦ Δαυίδ.}
\end{psalmheading}
Κρίνον με, Κύριε, ὅτι ἐγὼ ἐν ἀκακίᾳ μου ἐπορεύθην, καὶ ἐπὶ τῷ Κυρίῳ ἐλπίζων οὐ μὴ σαλευθῶ.
\vs{2}Δοκίμασόν με, Κύριε, καὶ πείρασόν με, πύρωσον τοὺς νεφρούς μου καὶ τὴν καρδίαν μου.

\vs{3}Ὅτι τὸ ἔλεός σου κατέναντι τῶν ὀφθαλμῶν μου ἐστὶ, καὶ εὐηρέστησα ἐν τῇ ἀληθείᾳ σου.
\vs{4}Οὐκ ἐκάθισα μετὰ συνεδρίου ματαιότητος, καὶ μετὰ παρανομούντων οὐ μὴ εἰσέλθω·
\vs{5}Ἐμίσησα ἐκκλησίαν πονηρευομένων, καὶ μετὰ ἀσεβῶν οὐ μὴ καθίσω.
\vs{6}Νίψομαι ἐν ἀθώοις τὰς χεῖράς μου, καὶ κυκλώσω τὸ θυσιαστήριόν σου, Κύριε·
\vs{7}Τοῦ ἀκοῦσαι φωνῆς αἰνέσεως, καὶ διηγήσασθαι πάντα τὰ θαυμάσιά σου.
\vs{8}Κύριε, ἠγάπησα εὐπρέπειαν οἴκου σου, καὶ τόπον σκηνώματος δόξης σου.
\vs{9}Μὴ συναπολέσῃς μετὰ ἀσεβῶν τὴν ψυχήν μου, καὶ μετὰ ἀνδρῶν αἱμάτων τὴν ζωήν μου,
\vs{10}ὧν ἐν χερσὶν ἀνομίαι, ἡ δεξιὰ αὐτῶν ἐπλήσθη δώρων.
\vs{11}Ἐγὼ δὲ ἐν ἀκακίᾳ μου ἐπορεύθην, λύτρωσαί με καὶ ἐλέησόν με.
\vs{12}Ὁ ποῦς μου ἔστη ἐν εὐθύτητι, ἐν ἐκκλησίαις εὐλογήσω σε Κύριε.

\begin{psalmheading}{\ch{26}{27} Τοῦ Δαυὶδ, πρὸ τοῦ χρισθῆναι.}
\end{psalmheading}
Κύριος φωτισμός μου καὶ σώτηρ μου, τίνα φοβηθήσομαι; Κύριος ὑπερασπιστὴς τῆς ζωῆς μου, ἀπὸ τίνος δειλιάσω;
\vs{2}Ἐν τῷ ἐγγίζειν ἐπʼ ἐμὲ κακοῦντας, τοῦ φαγεῖν τὰς σάρκας μου, οἱ θλίβοντες με καὶ οἱ ἐχθροί μου, αὐτοὶ ἠσθένησαν καὶ ἔπεσαν.
\vs{3}Ἐὰν παρατάξηται ἐπʼ ἐμὲ παρεμβολὴ, οὐ φοβηθήσεται ἡ καρδία μου· ἐὰν ἐπαναστῇ ἐπʼ ἐμὲ πόλεμος, ἐν ταύτῃ ἐγὼ ἐλπιζω.
\vs{4}Μίαν ᾐτησάμην παρὰ Κυρίου, ταύτην ἐκζητήσω, τοῦ κατοικεῖν με ἐν οἴκῳ Κυρίου πάσας τὰς ἡμέρας τῆς ζωῆς μου, τοῦ θεωρεῖν με τὴν τερπνότητα Κυρίου, καὶ ἐπισκέπτεσθαι τὸν ναὸν αὐτοῦ.
\vs{5}Ὅτι ἔκρυψέ με ἐν σκηνῇ αὐτοῦ ἐν ἡμέρᾳ κακῶν μου, ἐσκέπασέ με ἐν ἀποκρύφῳ τῆς σκῆνης αὐτοῦ· ἐν πέτρᾳ ὕψωσέ με,
\vs{6}καὶ νῦν ἰδοὺ ὕψωσε τὴν κεφαλήν μου ἐπʼ ἐχθρούς μου· ἐκύκλωσα καὶ ἔθυσα ἐν τῇ σκηνῇ αὐτοῦ θυσίαν ἀλαλαγμοῦ, ᾄσομαι καὶ ψαλῶ τῷ Κυρίῳ·

\vs{7}Εἰσάκουσον, Κύριε, τῆς φωνῆς μου ἧς ἐκέκραξα, ἐλέησόν με, καὶ εἰσάκουσόν μου.
\vs{8}Σοὶ εἶπεν ἡ καρδία μου, ἐξεζήτησα τὸ πρόσωπόν σου, τὸ πρόσωπόν σου Κύριε ζητήσω.
\vs{9}Μὴ ἀποστρέψῃς τὸ πρόσωπόν σου ἀπʼ ἐμοῦ, μὴ ἐκκλίνῃς ἐν ὀργῇ ἀπὸ τοῦ δούλου σου· βοηθός μου γενοῦ, μὴ ἐγκαταλίπῃς με, καὶ μὴ ὑπερίδῃς με ὁ Θεὸς ὁ σωτήρ μου.
\vs{10}Ὅτι ὁ πατήρ μου καὶ ἡ μήτηρ μου ἐγκατέλιπόν με, ὁ δὲ Κύριος προσελάβετό με.
\vs{11}Νομοθέτησόν με, Κύριε, ἐν τῇ ὁδῷ σου, καὶ ὁδήγησόν με ἐν τρίβῳ εὐθείᾳ ἕνεκα τῶν ἐχθρῶν μου.
\vs{12}Μὴ παραδῷς με εἰς ψυχὰς θλιβόντων με, ὅτι ἐπανέστησάν μοι μάρτυρες ἄδικοι, καὶ ἐψεύσατο ἡ ἀδικία ἑαυτῇ.

\vs{13}Πιστεύω τοῦ ἰδεῖν τὰ ἀγαθὰ Κυρίου ἐν γῇ ζώντων.
\vs{14}Ὑπόμεινον τὸν Κυριον, ἀνδρίζου, καὶ κραταιούσθω ἡ καρδία σου, καὶ ὑπόμεινον τὸν Κύριον.

\begin{psalmheading}{\ch{27}{28} Τοῦ Δαυίδ.}
\end{psalmheading}
Πρὸς σὲ Κύριε ἐκέκραξα, ὁ Θεός μου μὴ παρασιωπήσῃς ἐπʼ ἑμοὶ, μήποτε παρασιωπήσῃς ἐπʼ ἐμοὶ, καὶ ὁμοιωθήσομαι τοῖς καταβαίνουσιν εἰς λάκκον.
\vs{2}Εἰσάκουσον τῆς φωνῆς τῆς δεήσεώς μου, ἐν τῷ δέεσθαί με πρὸς σὲ, ἐν τῷ αἴρειν με χεῖράς μου εἰς ναὸν ἅγιόν σου.
\vs{3}Μὴ συνελκύσῃς μετὰ ἁμαρτωλῶν τὴν ψυχήν μου, καὶ μετὰ ἐργαζομένων ἀδικίαν μὴ συναπολέσῃς με, τῶν λαλούντων εἰρήνην μετὰ τῶν πλησίον αὐτῶν, κακὰ δὲ ἐν ταῖς καρδίαις αὐτῶν.
\vs{4}Δὸς αὐτοῖς κατὰ τὰ ἔργα αὐτῶν, καὶ κατὰ τὴν πονηρίαν τῶν ἐπιτηδευμάτων αὐτῶν· κατὰ τὰ ἔργα τῶν χειρῶν αὐτῶν δὸς αὐτοῖς, ἀπόδος τὸ ἀνταπόδομα αὐτῶν αὐτοῖς.
\vs{5}Ὅτι οὐ συνῆκαν εἰς τὰ ἔργα Κυρίου καὶ εἰς τὰ ἔργα τῶν χειρῶν αὐτοῦ, καθελεῖς αὐτοὺς καὶ οὐ μὴ οἰκοδομήσεις αὐτούς.

\vs{6}Εὐλογητὸς Κύριος, ὅτι εἰσήκουσε τῆς φωνῆς τῆς δεήσεώς μου.
\vs{7}Κύριος βοηθός μου καὶ ὑπερασπιστής μου· ἐπʼ αὐτῷ ἤλπισεν ἡ καρδία μου, καὶ ἐβοηθήθην, καὶ ἀνέθαλεν ἡ σάρξ μου· καὶ ἐκ θελήματός μου ἐξομολογήσομαι αὐτῷ.
\vs{8}Κύριος κραταίωμα τοῦ λαοῦ αὐτοῦ, καὶ ὑπερασπιστὴς τῶν σωτηρίων τοῦ χριστοῦ αὐτοῦ ἐστι.

\vs{9}Σῶσον τὸν λαόν σου, καὶ εὐλόγησον τὴν κληρονομίαν σου, καὶ ποίμανον αὐτοὺς, καὶ ἔπαρον αὐτοὺς ἕως τοῦ αἰῶνος.

\begin{psalmheading}{\ch{28}{29} Ψαλμὸς τῷ Δαυὶδ ἐξοδίου σκηνῆς.}
\end{psalmheading}
Ἐνέγκατε τῷ Κυρίῳ υἱοὶ Θεοῦ, ἐνέγκατε τῷ Κρίῳ υἱοὺς κριῶν· ἐνέγκατε τῷ Κυρίῳ δόξαν καὶ τιμὴν,
\vs{2}ἐνέγκατε τῷ Κυρίῳ δόξαν ὀνόματι αὐτοῦ· προσκυνήσατε τῷ Κυρίῳ ἐν αὐλῇ ἁγίᾳ αὐτοῦ.

\vs{3}Φωνὴ Κυρίου ἐπὶ τῶν ὑδάτων, ὁ Θεὸς τῆς δόξης ἐβρόντησε, Κύριος ἐπὶ ὑδάτων πολλῶν.
\vs{4}Φωνὴ Κυρίου ἐν ἰσχύϊ, φωνὴ Κυρίου ἐν μεγαλοπρεπείᾳ.
\vs{5}Φωνὴ Κυρίου συντρίβοντος κέδρους, συντρίψει Κύριος τὰς κέδρους τοῦ Λιβάνου,
\vs{6}Καὶ λεπτυνεῖ αὐτὰς ὡς τὸν μόσχον τὸν Λίβανον, καὶ ὁ ἠγαπημένος ὡς υἱὸς μονοκερώτων.
\vs{7}Φωνὴ Κυρίου διακόπτοντος φλόγα πυρός.
\vs{8}Φωνὴ Κυρίου συσσείοντος ἔρημον, συνσσείσει Κύριος τὴν ἔρημον Κάδης.
\vs{9}Φωνὴ Κυρίου καταρτιζομένου ἐλάφους, καὶ ἀποκαλύψει δρυμοὺς, καὶ ἐν τῷ ναῷ αὐτοῦ πᾶς τις λέγει δόξαν.
\vs{10}Κύριος τὸν κατακλυσμὸν κατοικιεῖ· καὶ καθιεῖται Κύριος βασιλεὺς εἰς τὸν αἰῶνα.
\vs{11}Κύριος ἰσχὺν τῷ λαῷ αὐτοῦ δώσει, Κύριος εὐλογήσει τὸν λαὸν αὐτοῦ ἐν εἰρήνῃ.

\begin{psalmheading}{\ch{29}{30} Εἰς τὸ τέλος, ψαλμὸς ᾠδῆς τοῦ ἐνκαινισμοῦ τοῦ οἴκου τοῦ Δαυίδ.}
\end{psalmheading}
\vs{2}Ὑψώσω σε, Κύριε, ὅτι ὑπέλαβές με, καὶ οὐκ εὔφρανας τοὺς ἐχθρούς μου ἐπʼ ἐμέ.
\vs{3}Κύριε ὁ Θεός μου, ἐκέκραξα πρὸς σὲ, καὶ ἰάσω με·
\vs{4}Κύριε, ἀνήγαγες ἐξ ᾅδου τὴν ψυχήν μου, ἔσωσάς με ἀπὸ τῶν καταβαινόντων εἰς λάκκον.

\vs{5}Ψάλατε τῷ Κυρίῳ οἱ ὅσιοι αὐτοῦ, καὶ ἐξομολογεῖσθε τῇ μνήμῃ τῆς ἁγιωσύνης αὐτοῦ.
\vs{6}Ὅτι ὀργὴ ἐν τῷ θυμῷ αὐτοῦ, καὶ ζωὴ ἐν τῷ θελήματι αὐτοῦ· τοεσπέρας αὐλισθήσεται κλαυθμὸς, καὶ εἰς τοπρωῒ ἀγαλλίασις.

\vs{7}Ἐγὼ δὲ εἶπα ἐν τῇ εὐθηνίᾳ μου, οὐ μὴ σαλευθῶ εἰς τὸν αἰῶνα.
\vs{8}Κύριε, ἐν τῷ θελήματί σου παρέσχου τῷ κάλλει μου δύναμιν· ἀπέστρεψας δὲ τὸ πρόσωπόν σου, καὶ ἐγενήθην τεταραγμένος.
\vs{9}Πρὸς σὲ Κύριε κεκράξομαι, καὶ πρὸς τὸν Θεόν μου δεηθήσομαι.
\vs{10}Τίς ὠφέλεια ἐν τῷ αἵματί μου, ἐν τῷ καταβῆναί με εἰς διαφθοράν; μὴ ἐξομολογήσεταί σοι χοῦς; ἢ ἀναγγελεῖ τὴν ἀλήθειάν σου;
\vs{11}Ἤκουσε Κύριος, καὶ ἠλέησέ με, Κύριος ἐγενήθη βοηθός μου.
\vs{12}Ἔστρεψας τὸν κοπετόν μου εἰς χαρὰν ἐμοὶ, διεῤῥήξας τὸν σάκκον μου, καὶ περιέζωσάς με εὐφροσύνην·
\vs{13}ὅπως ἂν ψάλῃ σοι ἡ δόξα μου, καὶ οὐ μὴ κατανυγῶ. Κύριε ὁ Θεός μου, εἰς τὸν αἰῶνα ἐξομολογήσομαί σοι.

\begin{psalmheading}{\ch{30}{31} Εἰς τὸ τέλος, ψαλμὸς τῷ Δαυὶδ, ἐκστάσεως.}
\end{psalmheading}
\vs{2}Ἐπὶ σοὶ Κύριε ἤλπισα, μὴ καταισχυνθείην εἰς τὸν αἰῶνα, ἐν τῇ δικαιοσύνῃ σου ῥῦσαί με καὶ ἐξελοῦ με.
\vs{3}Κλῖνον πρὸς μὲ τὸ οὖς σου, τὰχυνον τοῦ ἐξελέσθαι με γενοῦ μοι εἰς Θεὸν ὑπερασπιστὴν καὶ εἰς οἶκον καταφυγῆς τοῦ σῶσαί με.
\vs{4}Ὅτι κραταίωμά μου καὶ καταφυγή μου εἶ σὺ, καὶ ἕνεκεν τοῦ ὀνόματός σου ὁδηγήσεις με, καὶ διαθρέψεις με.
\vs{5}Ἐξάξεις με ἐκ παγίδος ταύτης ἧς ἔκρυψάν μοι, ὅτι σὺ εἶ ὁ ὑπερασπιστής μου, Κύριε.
\vs{6}Εἰς χεῖράς σου παραθήσομαι τὸ πνεῦμά μου, ἐλυτρώσω με Κύριε ὁ Θεὸς τῆς ἀληθέιας.
\vs{7}Ἐμίσησας τοὺς διαφυλάσσοντας ματαιότητας διακενῆς· ἐγὼ δὲ ἐπὶ τῷ Κυρίῳ ἤλπισα.
\vs{8}Ἀγαλλιάσομαι καὶ εὐφρανθήσομαι ἐπὶ τῷ ἐλέει σου· ὅτι ἐπεῖδες τὴν ταπείνωσίν μου, ἔσωσας ἐκ τῶν ἀναγκῶν τὴν ψυχήν μου.
\vs{9}Καὶ οὐ συνέκλεισάς με εἰς χεῖρας ἐχθροῦ, ἔστησας ἐν εὐρυχώρῳ τοὺς πόδας μου.

\vs{10}Ἐλέησόν με, Κύριε, ὅτι θλίβομαι· ἐταράχθη ἐν θυμῷ ὁ ὀφθαλμός μου, ἡ ψυχή μου, καὶ ἡ γαστήρ μου.
\vs{11}Ὅτι ἐξέλιπεν ἐν ὀδύνῃ ἡ ζωή μου, καὶ τὰ ἔτη μου ἐν στεναγμοῖς· ἠσθένησεν ἐν πτωχείᾳ ἡ ἰσχύς μου, καὶ τὰ ὀστᾶ μου ἐταράχθησαν.
\vs{12}Παρὰ πάντας τοὺς ἐχθρούς μου ἐγενήθην ὄνειδος, καὶ τοῖς γείτοσί μου σφόδρα, καὶ φόβος τοῖς γνωστοῖς μου· οἱ θεωροῦντές με ἔξω ἔφυγον ἀπʼ ἐμοῦ·
\vs{13}Ἐπελήσθην ὡσεὶ νεκρὸς ἀπὸ καρδίας· ἐγενήθην ὡσεὶ σκεῦος ἀπολωλὸς,
\vs{14}ὅτι ἤκουσα ψόγον πολλῶν παροικούντων κυκλόθεν· ἐν τῷ συναχθῆναι αὐτοὺς ἅμα ἐπʼ ἐμὲ, τοῦ λαβεῖν τὴν ψυχήν μου ἐβουλεύσαντο.

\vs{15}Ἐγὼ δὲ ἐπὶ σοὶ ἤλπισα, Κύριε· εἶπα, σὺ εἶ ὁ Θεός μου,
\vs{16}ἐν ταῖς χερσί σου οἱ κλῆροί μου· ῥῦσαί με ἐκ χειρὸς ἐχθρῶν μου, καὶ ἐκ τῶν καταδιωκόντων με.
\vs{17}Ἐπίφανον τὸ πρόσωπόν σου ἐπὶ τὸν δοῦλόν σου, σῶσόν με ἐν τῷ ἐλέει σου.
\vs{18}Κύριε, μὴ καταισχυνθείην, ὅτι ἐπεκαλεσάμην σε· αἰσχυνθείησαν οἱ ἀσεβεῖς, καὶ καταχθείησαν εἰς ᾅδου.
\vs{19}Ἄλαλα γενηθήτω τὰ χείλη τὰ δόλια, τὰ λαλοῦντα κατὰ τοῦ δικαίου ἀνομίαν ἐν ὑπερηφανίᾳ καὶ ἐξουδενώσει.

\vs{20}Ὡς πολὺ τὸ πλῆθος τῆς χρηστότητός σου, Κύριε, ἧς ἔκρυψας τοῖς φοβουμένοις σε; ἐξειργάσω τοῖς ἐλπίζουσιν ἐπὶ σὲ, ἐναντίον τῶν υἱῶν τῶν ἀνθρώπων.
\vs{21}Κατακρύψεις αὐτοὺς ἐν ἀποκρύφῳ τοῦ προσώπου σου ἀπὸ ταραχῆς ἀνθρώπων, σκεπάσεις αὐτοὺς ἐν σκηνῇ ἀπὸ ἀντιλογίας γλωσσῶν.
\vs{22}Εὐλογητὸς Κύριος, ὅτι ἐθαυμάστωσε τὸ ἔλεος αὐτοῦ ἐν πόλει περιοχῆς.
\vs{23}Ἐγὼ δὲ εἶπα ἐν τῇ ἐκστάσει μου, ἀπέῤῥιμμαι ἀπὸ προσώπου τῶν ὀφθαλμῶν σου· διὰ τοῦτο εἰσήκουσας, Κύριε, τῆς φωνῆς τῆς δεήσεώς μου ἐν τῷ κεκραγέναι με πρὸς σέ.

\vs{24}Ἀγαπήσατε τὸν Κύριον πάντες οἱ ὅσιοι αὐτοῦ, ὅτι ἀληθείας ἐκζητεῖ Κύριος, καὶ ἀνταποδίδωσι τοῖς περισσῶς ποιοῦσιν ὑπερηφανίαν.
\vs{25}Ἀνδρίζεσθε, καὶ κραταιούσθω ἡ καρδία ὑμῶν, πάντες οἱ ἐλπίζοντες ἐπὶ Κύριον.

\begin{psalmheading}{\ch{31}{32} Συνέσεως τῷ Δαυίδ.}
\end{psalmheading}
Μακάριοι ὧν ἀφέθησαν αἱ ἀνομίαι, καὶ ὧν ἐπεκαλύφθησαν αἱ ἁμαρτίαι.
\vs{2}Μακάριος ἀνὴρ ᾧ οὐ μὴ λογίσηται Κύριος ἁμαρτίαν, οὐδέ ἐστιν ἐν τῷ στόματι αὐτοῦ δόλος.

\vs{3}Ὅτι ἐσίγησα, ἐπαλαιώθη τὰ ὀστᾶ μου, ἀπὸ τοῦ κράζειν με ὅλην τὴν ἡμέραν.
\vs{4}Ὅτι ἡμέρας καὶ νυκτὸς ἐβαρύνθη ἐπʼ ἐμὲ ἡ χείρ σου, ἐστράφην εἰς ταλαιπωρίαν ἐν τῷ ἐμπαγῆναι ἄκανθαν· διάψαλμα.
\vs{5}Τὴν ἁμαρτίαν μου ἐγνώρισα, καὶ τὴν ἀνομίαν μου οὐκ ἐκάλυψα· εἶπα, ἐξαγορεύσω κατʼ ἐμοῦ τὴν ἀνομίαν μου τῷ Κυρίῳ, καὶ σὺ ἀφῆκας τὴν ἀσέβειαν τῆς καρδίας μου· διάψαλμα.
\vs{6}Ὑπὲρ ταύτης προσεύξεται πρὸς σὲ πᾶς ὅσιος ἐν καιρῷ εὐθέτῳ· πλὴν ἐν κατακλυσμῷ ὑδάτων πολλῶν πρὸς αὐτὸν οὐκ ἐγγιοῦσι.
\vs{7}Σύ μου εἶ καταφυγὴ ἀπὸ θλίψεως τῆς περιεχούσης με, τὸ ἀγαλλίαμά μου λύτρωσαί με ἀπὸ τῶν κυκλωσάντων με· διάψαλμα.

\vs{8}Συνετιῶ σε καὶ συμβιβῶ σε ἐν ὁδῷ ταύτῃ ᾗ πορεύσῃ, ἐπιστηριῶ ἐπὶ σὲ τοὺς ὀφθαλμούς μου.
\vs{9}Μὴ γίνεσθε ὡς ἵππος καὶ ἡμίονος, οἷς οὐκ ἔστι σύνεσις· ἐν χαλινῷ καὶ κημῷ τὰς σιαγόνας αὐτῶν ἄγξαι τῶν μὴ ἐγγιζόντων πρὸς σέ.
\vs{10}Πολλαὶ αἱ μάστιγες τοῦ ἁμαρτωλοῦ, τὸν δὲ ἐλπίζοντα ἐπὶ Κύριον ἔλεος κυκλώσει.
\vs{11}Εὐφράνθητε ἐπὶ Κύριον καὶ ἀγαλλιᾶσθε δίκαιοι, καὶ καυχᾶσθε πάντες οἱ εὐθεῖς τῇ καρδίᾳ.

\begin{psalmheading}{\ch{32}{33} Τῷ Δαυίδ.}
\end{psalmheading}
Ἀγαλλιᾶσθε δίκαιοι ἐν τῷ Κυρίῳ, τοῖς εὐθέσι πρέπει αἴνεσις.
\vs{2}Ἐξομολογεῖσθε τῷ Κυρίῳ ἐν κιθάρᾳ, ἐν ψαλτηρίῳ δεκαχόρδῳ ψάλατε αὐτῷ.
\vs{3}Ἄσατε αὐτῷ ᾆσμα καινὸν, καλῶς ψάλατε ἐν ἀλαλαγμῷ.

\vs{4}Ὅτι εὐθὴς ὁ λόγος τοῦ Κυρίου, καὶ πάντα τὰ ἔργα αὐτοῦ ἐν πίστει.
\vs{5}Ἀγαπᾷ ἐγεημοσύνην καὶ κρίσιν, τοῦ ἐλέους Κυρίου πλήρης ἡ γῆ.
\vs{6}Τῷ λόγῳ τοῦ Κυρίου οἱ οὐρανοὶ ἐστερεώθησαν, καὶ τῷ πνεύματι τοῦ στόματος αὐτοῦ πᾶσα ἡ δύναμις αὐτῶν.
\vs{7}Συνάγων ὡσεὶ ἀσκὸν ὕδατα θαλάσσης, τιθεὶς ἐν θησαυροῖς ἀβύσσους.
\vs{8}Φόβηθήτω τὸν Κύριον πᾶσα ἡ γῆ, ἀπʼ αὐτοῦ δὲ σαλευθήτωσαν πάντες οἱ κατοικοῦντες τὴν οἰκουμένην.
\vs{9}Ὅτι αὐτὸς εἶπε καὶ ἐγενήθησαν, αὐτὸς ἐνετείλατο καὶ ἐκτίσθησαν.
\vs{10}Κύριος διασκεδάζει βουλὰς ἐθνῶν, ἀθετεῖ δὲ λογισμοὺς λαῶν, καὶ ἀθετεῖ βουλὰς ἀρχόντων.
\vs{11}Ἡ δὲ βουλὴ τοῦ Κυρίου εἰς τὸν αἰῶνα μένει, λογισμοὶ τῆς καρδίας αὐτοῦ ἀπὸ γενεῶν εἰς γενεάς.
\vs{12}Μακάριον τὸ ἔθνος οὗ ἐστι Κύριος ὁ Θεὸς αὐτοῦ, λαὸς ὃν ἐξελέξατο εἰς κληρονομίαν ἑαυτῷ.
\vs{13}Ἐξ οὐρανοῦ ἐπέβλεψεν ὁ Κύριος, εἶδε πάντας τοὺς υἱοὺς τῶν ἀνθρώπων.
\vs{14}Ἐξ ἑτοίμου κατοικητηρίου αὐτοῦ ἐπέβλεψεν ἐπὶ πάντας τοὺς κατοικοῦντας τὴν γῆν.
\vs{15}Ὁ πλάσας κατὰ μόνας τὰς καρδίας αὐτῶν, ὁ συνιεὶς πάντα τὰ ἔργα αὐτῶν.
\vs{16}Οὐ σώζεται βασιλεὺς διὰ πολλὴν δύναμιν, καὶ γίγας οὐ σωθήσεται ἐν πλήθει ἰσχύος αὐτοῦ.
\vs{17}Ψευδὴς ἵππος εἰς σωτηρίαν, ἐν δὲ πλήθει δυνάμεως αὐτοῦ οὐ σωθήσεται.

\vs{18}Ἰδοὺ οἱ ὀφθαλμοὶ Κυρίου ἐπὶ τοὺς φοβουμένους αὐτὸν, τοὺς ἐλπίζοντας ἐπὶ τὸ ἔλεος αὐτοῦ,
\vs{19}ῥύσασθαι ἐκ θανάτου τὰς ψυχὰς αὐτῶν, καὶ διαθρέψαι αὐτοὺς ἐν λιμῷ.
\vs{20}Ἡ ψυχὴ ἡμῶν ὑπομένει τῷ Κυρίῳ, ὅτι βοηθὸς καὶ ὑπερασπιστὴς ἡμῶν ἐστιν.
\vs{21}Ὅτι ἐν αὐτῷ εὐφρανθήσεται ἡ καρδία ἡμῶν, καὶ ἐν τῷ ὀνόματι τῷ ἁγίῳ αὐτοῦ ἠλπίσαμεν.
\vs{22}Γένοιτο τὸ ἔλεός σου, Κύριε, ἐφʼ ἡμᾶς, καθάπερ ἠλπίσαμεν ἐπὶ σέ.

\begin{psalmheading}{\ch{33}{34} Τῷ Δαυὶδ, ὁπότε ἠλλοίωσε τὸ πρόσωπον αὐτοῦ ἐναντίον Ἀβιμέλεχ, καὶ ἀπέλυσεν αὐτὸν, καὶ ἀπῆλθεν.}
\end{psalmheading}
\vs{2}Εὐλογήσω τὸν Κύριον ἐν παντὶ καιρῷ, διαπαντὸς ἡ αἴνεσις αὐτοῦ ἐν τῷ στόματί μου.
\vs{3}Ἐν τῷ Κυρίῳ ἐπαινεθήσεται ἡ ψυχή μου· ἀκουσάτωσαν πρᾳεῖς καὶ εὐφρανθήτωσαν.
\vs{4}Μεγαλύνατε τὸν Κύριον σὺν ἐμοὶ, καὶ ὑψώσωμεν τὸ ὄνομα αὐτοῦ ἐπιτοαυτό.

\vs{5}Ἐξεζήτησα τὸν Κύριον καὶ ἐπήκουσέ μου, καὶ ἐκ πασῶν τῶν παροικιῶν μου ἐῤῥύσατό με·
\vs{6}Προσέλθατε πρὸς αὐτὸν καὶ φωτίσθητε, καὶ τὰ πρόσωπα ὑμῶν οὐ μὴ καταισχυνθῇ·
\vs{7}Οὗτος ὁ πτωχὸς ἐκέκραξε, καὶ ὁ Κύριος εἰσήκουσεν αὐτοῦ, καὶ ἐκ πασῶν τῶν θλίψεων αὐτοῦ ἔσωσεν αὐτόν.
\vs{8}Παρεμβαλεῖ ἄγγελος Κυρίου κύκλῳ τῶν φοβουμένων αὐτὸν, καὶ ῥύσεται αὐτούς.
\vs{9}Γεύσασθε καὶ ἴδετε ὅτι χρηστὸς ὁ Κύριος, μακάριος ἀνὴρ ὃς ἐλπίζει ἐπʼ αὐτόν.
\vs{10}φοβήθητε τὸν Κύριον πάντες οἱ ἅγιοι αὐτοῦ, ὅτι οὐκ ἔστιν ὑστέρημα τοῖς φοβουμένοις αὐτόν.
\vs{11}Πλούσιοι ἐπτώχευσαν καὶ ἐπείνασαν, οἱ δὲ ἐκζητοῦντες τὸν Κύριον οὐκ ἐλαττωθήσονται παντὸς ἀγαθοῦ· διάψαλμα.

\vs{12}Δεῦτε τέκνα, ἀκούσατέ μου, φόβον Κυρίου διδάξω ὑμᾶς.
\vs{13}Τίς ἐστιν ἄνθρωπος ὁ θέλων ζωὴν, ἀγαπῶν ἡμέρας ἰδεῖν ἀγαθάς;
\vs{14}Παῦσον τὴν γνῶσσάν σου ἀπὸ κακοῦ, καὶ χείλη σου τοῦ μὴ λαλῆσαι δόλον.
\vs{15}Ἔκκλινον ἀπὸ κακοῦ, καὶ ποίησον ἀγαθὸν· ζήτησον εἰρήνην, καὶ δίωξον αὐτήν.

\vs{16}Οφθαλμοὶ Κυρίου ἐπὶ δικαίους, καὶ ὦτα αὐτοῦ εἰς δέησιν αὐτῶν·
\vs{17}Πρόσωπον δὲ Κυρίου ἐπὶ ποιοῦντας κακὰ, τοῦ ἐξολοθρεῦσαι ἐκ γῆς τὸ μνημόσυνον αὐτῶν.
\vs{18}Ἐκέκραξαν οἱ δίκαιοι, καὶ ὁ Κύριος εἰσήκουσεν αὐτῶν, καὶ ἐκ πασῶν τῶν θλίψεων αὐτῶν ἐῤῥύσατο αὐτούς.
\vs{19}Ἐγγὺς Κύριος τοῖς συντετριμμένοις τὴν καρδίαν, καὶ τοὺς ταπεινοὺς τῷ πνεύματι σώσει.
\vs{20}Πολλαὶ αἱ θλίψεις τῶν δικαίων, καὶ ἐκ πασῶν αὐτῶν ῥύσεται αὐτοὺς
\vs{21}ὁ Κύριος. Φυλάσσει πάντα τὰ ὀστᾶ αὐτῶν, ἓν ἐξ αὐτῶν οὐ συντριβήσεται.
\vs{22}Θάνατος ἁμαρτωλῶν πονηρὸς, καὶ οἱ μισοῦντες τὸ δίκαιον πλημμελήσουσι.
\vs{23}Λυτρώσεται Κύριος ψυχὰς δούλων αὐτοῦ, καὶ οὐ μὴ πλημμελήσουσι πάντες οἱ ἐλπίζοντες ἐπʼ αὐτόν.

\begin{psalmheading}{\ch{34}{35} Τῷ Δαυίδ.}
\end{psalmheading}
Δικάσον, Κύριε, τοὺς ἀδικοῦντάς με, πολέμησον τοὺς πολεμοῦντάς με.
\vs{2}Ἐπιλαβοῦ ὅπλου καὶ θυρεοῦ, καὶ ἀνάστηθι εἰς βοήθειάν μοι.
\vs{3}Ἔκχεον ῥομφαίαν, καὶ σύγκλεισον ἐξεναντίας τῶν καταδιωκόντων με· εἰπὸν τῇ ψυχῇ μου, σωτηρία σου ἐγώ εἰμι.
\vs{4}Αἰσχυνθείησαν καὶ ἐντραπείησαν οἱ ζητοῦντες τὴν ψυχήν μου, ἀποστραφείησαν εἰς τὰ ὀπίσω, καὶ καταισχυνθείησαν οἱ λογιζόμενοί μοι κακά.
\vs{5}Γενηθήτωσαν ὡσεὶ χοῦς κατὰ πρόσωπον ἀνέμου, καὶ ἄγγελος Κυρίου ἐκθλίβων αὐτούς.
\vs{6}Γενηθήτω ἡ ὁδὸς αὐτῶν σκότος καὶ ὀλίσθημα, καὶ ἄγγελος Κυρίου καταδιώκων αὐτούς.
\vs{7}Ὅτι δωρεὰν ἔκρυψάν μοι διαφθορὰν παγίδος αὐτῶν, μάτην ὠνείδισαν τὴν ψυχήν μου.

\vs{8}Ἐλθέτω αὐτοῖς παγὶς ἣν οὐ γινώσκουσι, καὶ ἡ θήρα ἣν ἔκρυψαν, συλλαβέτω αὐτοὺς, καὶ ἐν τῇ παγίδι πεσοῦνται ἐν αὐτῇ.
\vs{9}Ἡ δὲ ψυχή μου ἀγαλλιάσεται ἐπὶ τῷ Κυρίῳ, τερφθήσεται ἐπὶ τῷ σωτηρίῳ αὐτοῦ.
\vs{10}Πάντα τὰ ὀστᾶ μου ἐροῦσι, Κύριε τίς ὅμοιός σοι; ῥυόμενος πτωχὸν ἐκ χειρὸς στερεωτέρων αὐτοῦ, καὶ πτωχὸν καὶ πένητα ἀπὸ τῶν διαρπαζόντων αὐτόν.

\vs{11}Ἀναστάντες μάρτυρες ἄδικοι, ἃ οὐκ ἐγίνωσκον, ἐπηρώτων με.
\vs{12}Ἀνταπεδίδοσάν μοι πονηρὰ ἀντὶ καλῶν, καὶ ἀτεκνίαν τῇ ψυχῇ μου.
\vs{13}Ἐγὼ δὲ ἐν τῷ αὐτοὺς παρενοχλεῖν μοι, ἐνεδυόμην σάκκον· καὶ ἐταπείνουν ἐν νηστείᾳ τὴν ψυχήν μου, καὶ ἡ προσευχή μου εἰς κόλπον μου ἀποστραφήσεται.
\vs{14}Ὡς πλησίον ὡς ἀδελφὸν ἡμέτερον οὕτως εὐηρέστουν, ὡς πενθῶν καὶ σκυθρωπάξων οὕτως ἐταπεινούμην.
\vs{15}Καὶ κατʼ ἐμοῦ εὐφράνθησαν, καὶ συνήχθησαν, συνήχθησαν ἐπʼ ἐμὲ μάστιγες καὶ οὐκ ἔγνων· διεσχίσθησαν καὶ οὐ κατενύγησαν.
\vs{16}Ἐπείρασάν με, ἐξεμυκτήρισάν με μυκτηρισμὸν, ἔβρυξαν ἐπʼ ἐμὲ τοὺς ὀδόντας αὐτῶν.

\vs{17}Κύριε πότε ἐπόψῃ; ἀποκατάστησον τὴν ψυχήν μου ἀπὸ τῆς κακουργίας αὐτῶν, ἀπὸ λεόντων τὴν μονογενῆ μου.
\vs{18}Ἐξομολογήσομαί σοι καὶ ἐν ἐκκλησίᾳ πολλῇ, ἐν λαῷ βαρεῖ αἰνέσω σε.
\vs{19}Μὴ ἐπιχαρείησάν μοι οἱ ἐχθραίνοντές μοι ματαίως, οἱ μισοῦντές με δωρεὰν, καὶ διανεύοντες ὀφθαλμοῖς.
\vs{20}Ὅτι ἐμοὶ μὲν εἰρηνικὰ ἐλάλουν, καὶ ἐπʼ ὀργῇ δόλους διελογίζοντο.
\vs{21}Καὶ ἐπλάτυναν ἐπʼ ἐμὲ τὸ στόμα αὐτῶν, εἶπαν, εὖγε, εὖγε, εἶδον οἱ ὀφθαλμοὶ ἡμῶν.

\vs{22}Εἶδες Κύριε, μὴ παρασιωπήσῃς, Κύριε μὴ ἀποστῇς ἀπʼ ἐμοῦ.
\vs{23}Ἐξεγέρθητι Κύριε, καὶ πρόσχες τῇ κρίσει μου, ὁ Θεός μου καὶ ὁ Κύριός μου εἰς τὴν δίκην μου.
\vs{24}Κρῖνόν με Κύριε κατὰ τὴν δικαιοσύνην σου Κύριε ὁ Θεός μου, καὶ μὴ ἐπιχαρείησάν μοι.
\vs{25}Μὴ εἴποισαν ἐν καρδίαις αὐτῶν, εὖγε, εὖγε τῇ ψυχῇ ἡμῶν· μηδὲ εἴποιεν, κατεπίομεν αὐτόν.
\vs{26}Αἰσχυνθείησαν καὶ ἐντραπείησαν ἅμα οἱ ἐπιχαίροντες τοῖς κακοῖς μου· ἐνδυσάσθωσαν αἰσχύνην καὶ ἐντροπὴν οἱ μεγαλοῤῥημονοῦντες ἐπʼ ἐμέ.
\vs{27}Ἀγαλλιάσαιντο καὶ εὐφρανθείησαν οἱ θέλοντες τὴν δικαιοσύνην μου, καὶ εἰπάτωσαν διαπαντὸς, μεγαλυνθείη ὁ Κύριος, οἱ θέλοντες τὴν εἰρήνην τοῦ δούλου αὐτοῦ.
\vs{28}Καὶ ἡ γλῶσσά μου μελετήσει τὴν δικαιοσύνην σου, ὅλην τὴν ἡμέραν τὸν ἔπαινόν σου.

\begin{psalmheading}{\ch{35}{36} Εἰς τὸ τέλος, τῷ δούλῳ Κυρίου τῷ Δαυίδ.}
\end{psalmheading}
\vs{2}Φησὶν ὁ παράνομος τοῦ ἁμαρτάνειν ἐν ἑαυτῷ, οὐκ ἔστι φόβος Θεοῦ ἀπέναντι τῶν ὀφθαλμῶν αὐτοῦ.
\vs{3}Ὅτι ἐδόλωσεν ἐνώπιον αὐτοῦ, τοῦ εὑρεῖν τὴν ἀνομίαν αὐτοῦ καὶ μισῆσαι.
\vs{4}Τὰ ῥήματα τοῦ στόματος αὐτοῦ ἀνομία καὶ δόλος, οὐκ ἠβουλήθη συνιέναι τοῦ ἀγαθῦναι.
\vs{5}Ἀνομίαν ἐλογίσατο ἐπὶ τῆς κοίτης αὐτοῦ, παρέστη πάσῃ ὁδῷ οὐκ ἀγαθῇ, τῇ δὲ κακίᾳ οὐ προσώχθισε.

\vs{6}Κύριε, ἐν τῷ οὐρανῷ τὸ ἔλεός σου, καὶ ἡ ἀλήθειά σου ἕως τῶν νεφελῶν.
\vs{7}Ἡ δικαιοσύνη σου ὡς ὄρη Θεοῦ, τὰ κρίματά σου ὡσεὶ ἄβυσσος πολλή· ἀνθρώπους καὶ κτήνη σώσεις Κύριε.
\vs{8}Ὡς ἐπλήθυνας τὸ ἔλεός σου ὁ Θεός· οἱ δὲ υἱοὶ τῶν ἀνθρώπων ἐν σκέπῃ τῶν πτερύγων σου ἐλπιοῦσι.
\vs{9}Μεθυσθήσονται ἀπὸ πιότητος οἴκου σου, καὶ τὸν χειμάῤῥουν τῆς τρυφῆς σου ποτιεῖς αὐτούς.
\vs{10}Ὅτι παρὰ σοὶ πηγὴ ζωῆς, ἐν τῷ φωτί σου ὀψόμεθα φῶς.

\vs{11}Παράτεινον τὸ ἔλεός σου τοῖς γινώσκουσί σε, καὶ τὴν δικαιοσύνην σου τοῖς εὐθέσι τῇ καρδίᾳ.
\vs{12}Μὴ ἐλθέτω μοι ποῦς ὑπερηφανίας, καὶ χεὶρ ἁμαρτωλῶν μὴ σαλεύσαι με·

\vs{13}Ἐκεῖ ἔπεσον πάντες οἱ ἐργαζόμενοι τὴν ἀνομίαν, ἐξώσθησαν καὶ οὐ μὴ δύνωνται στῆναι.

\begin{psalmheading}{\ch{36}{37} Τῷ Δαυίδ.}
\end{psalmheading}
Μὴ παραζήλου ἐν πονηρευομένοις, μηδὲ ζήλου τοὺς ποιοῦντας τὴν ἀνομίαν.
\vs{2}Ὅτι ὡσεὶ χόρτος ταχὺ ἀποξηρανθήσονται, καὶ ὡσεὶ λάχανα χλόης ταχὺ ἀποπεσοῦνται.
\vs{3}Ἔλπισον ἐπὶ Κύριον, καὶ ποίει χρηστότητα, καὶ κατασκήνου τὴν γῆν, καὶ ποιμανθήσῃ ἐπὶ τῷ πλούτῳ αὐτῆς.
\vs{4}Κατατρύφησον τοῦ Κυρίου, καὶ δώσει σοι τὰ αἰτήματα τῆς καρδίας σου.
\vs{5}Ἀποκάλυψον πρὸς Κύριον τὴν ὁδόν σου, καὶ ἔλπισον ἐπʼ αὐτὸν, καὶ αὐτὸς ποιήσει.
\vs{6}Καὶ ἐξοίσει ὡς φῶς τὴν δικαιοσύνην σου, καὶ τὸ κρίμα σου ὡς μεσημβρίαν.

\vs{7}Ὑποτάγηθι τῷ Κυρίῳ, καὶ ἱκέτευσον αὐτόν· μὴ παραζήλου ἐν τῷ κατευοδουμένῳ ἐν τῇ ὁδῷ αὐτοῦ, ἐν ἀνθρώπῳ ποιοῦντι παρανομίας.
\vs{8}Παῦσαι ἀπὸ ὀργῆς καὶ ἐνκατάλιπε θυμὸν, μὴ παραζήλου ὥστε πονηρεύεσθαι.
\vs{9}Ὅτι οἱ πονηρευόμενοι ἐξολοθρευθήσονται, οἱ δὲ ὑπομένοντες τὸν Κύριον, αὐτοὶ κληρονομήσουσι τὴν γῆν.
\vs{10}Καὶ ἔτι ὀλίγον καὶ οὐ μὴ ὑπάρξῃ ἁμαρτωλὸς, καὶ ζητήσεις τὸν τόπον αὐτοῦ, καὶ οὐ μὴ εὕρῃς.
\vs{11}Οἱ δὲ πρᾳεῖς κληρονομήσουσι γῆν, καὶ κατατρυφήσουσιν ἐπὶ πλήθει εἰρήνης.

\vs{12}Παρατηρήσεται ὁ ἁμαρτωλὸς τὸν δίκαιον, καὶ βρύξει ἐπʼ αὐτὸν τοὺς ὀδόντας αὐτοῦ.
\vs{13}Ὁ δὲ Κύριος ἐκγελάσεται αὐτὸν, ὅτι προβλέτει ὅτι ἥξει ἡ ἡμέρα αὐτοῦ.
\vs{14}Ῥομφαίαν ἐσπάσαντο οἱ ἁμαρτωλοὶ ἐνέτειναν τόξον αὐτῶν, τοῦ καταβαλεῖν πτωχὸν καὶ πένητα, τοῦ σφάξαι τοὺς εὐθεῖς τῇ καρδίᾳ.
\vs{15}Ἡ ῥομφαία αὐτῶν εἰσέλθοι εἰς τὴν καρδίαν αὐτῶν, καὶ τὰ τόξα αὐτῶν συντριβείη.

\vs{16}Κρεῖσσον ὀλίγον τῷ δικαίῳ ὑπὲρ πλοῦτον ἁμαρτωλῶν πολύν.
\vs{17}Ὅτι βραχίονες ἁμαρτωλῶν συντριβήσονται, ὑποστηρίζει δὲ τοὺς δικαίους ὁ Κύριος.

\vs{18}Γινώσκει Κύριος τὰς ὁδοὺς τῶν ἀμώμων, καὶ ἡ κληρονομία αὐτῶν εἰς τὸν αἰῶνα ἔσται.
\vs{19}Οὐ καταισχυνθήσονται ἐν καιρῷ πονηρῷ, καὶ ἐν ἡμέραις λιμοῦ χορτασθήσονται·
\vs{20}Ὅτι οἱ ἁμαρτωλοὶ ἀπολοῦνται, οἱ δὲ ἐχθροὶ τοῦ Κυρίου ἅμα τῷ δοξασθῆναι αὐτοὺς καὶ ὑψωθῆναι, ἐκλείποντες ὡσεὶ καπνὸς ἐξέλιπον.
\vs{21}Δανείζεται ὁ ἁμαρτωλὸς, καὶ οὐκ ἀποτίσει, ὁ δὲ δίκαιος οἰκτείρει καὶ διδοῖ.
\vs{22}Ὅτι οἱ εὐλογοῦντες αὐτὸν κληρονομήσουσι γῆν, οἱ δὲ καταρώμενοι αὐτὸν ἐξολοθρευθήσονται.

\vs{23}Παρὰ Κυρίου τὰ διαβήματα ἀνθρώπου κατευθύνεται, καὶ τὴν ὁδὸν αὐτοῦ θελήσει.
\vs{24}Ὅταν πέσῃ οὐ καταῤῥαχθήσεται, ὅτι Κύριος ἀντιστηρίζει χεῖρα αὐτοῦ.
\vs{25}Νεώτερος ἐγενόμην, καὶ γὰρ ἐγήρασα· καὶ οὐκ εἶδον δίκαιον ἐγκαταλελειμμένον, οὐδὲ τὸ σπέρμα αὐτοῦ ζητοῦν ἄρτους.
\vs{26}Ὅλην τὴν ἡμέραν ἐλεεῖ καὶ δανείζει, καὶ τὸ σπέρμα αὐτοῦ εἰς εὐλογίαν ἔσται.

\vs{27}Ἔκκλινον ἀπὸ κακοῦ, καὶ ποίησον ἀγαθὸν, καὶ κατασκήνου εἰς αἰῶνα αἰῶνος.
\vs{28}Ὅτι Κύριος ἀγαπᾷ κρίσιν, καὶ οὐκ ἐγκαταλείψει τοὺς ὁσίους αὐτοῦ, εἰς τὸν αἰῶνα φυλαχθήσονται· ἄμωμοι ἐκδικηθήσονται, καὶ σπέρμα ἀσεβῶν ἐξολοθρευθήσεται.
\vs{29}Δίκαιοι δὲ κληρονομήσουσι γῆν, καὶ κατασκηνώσουσιν εἰς αἰῶνα αἰῶνος ἐπʼ αὐτῆς.

\vs{30}Στόμα δικαίου μελετήσει σοφίαν, καὶ ἡ γλῶσσα αὐτοῦ λαλήσει κρίσιν.
\vs{31}Ὁ νόμος τοῦ Θεοῦ αὐτοῦ ἐν καρδίᾳ αὐτοῦ, καὶ οὐχ ὑποσκελισθήσεται τὰ διαβήματα αὐτοῦ.
\vs{32}Κατανοεῖ ὁ ἁμαρτωλὸς τὸν δίκαιον, καὶ ζητεῖ τοῦ θανατῶσαι αὐτόν.
\vs{33}Ὁ δὲ Κύριος οὐ μὴ ἐγκαταλίπῃ αὐτὸν εἰς τὰς χεῖρας αὐτοῦ, οὐδὲ μὴ καταδικάσαι αὐτὸν, ὅταν κρίνηται αὐτῷ.
\vs{34}Ὑπόμεινον τὸν Κύριον, καὶ φύλαξον τὴν ὁδὸν αὐτοῦ, καὶ ὑψώσει σε τοῦ κατακληρονομῆσαι τὴν γῆν· ἐν τῷ ἐξολοθρεύεσθαι ἁμαρτωλοὺς, ὄψει.

\vs{35}Εἶδον τὸν ἀσεβῆ ὑπερυψούμενον, καὶ ἐπαιρόμενον ὡς τὰς κέδρους τοῦ Λιβάνου·
\vs{36}Καὶ παρῆλθον, καὶ ἰδοὺ οὐκ ἦν, καὶ ἐζήτησα αὐτόν, καὶ οὐχ εὑρέθη ὁ τόπος αὐτοῦ.
\vs{37}Φύλασσε ἀκακίαν καὶ ἴδε εὐθύτητα, ὅτι ἐστὶν ἐγκατάλειμμα ἀνθρώπῳ εἰρηνικῷ.
\vs{38}Οἱ δὲ παράνομοι ἐξολοθρευθήσονται ἐπιτοαυτὸ, τὰ ἐγκαταλείμματα τῶν ἀσεβῶν ἐξολοθρευθήσονται·
\vs{39}Σωτηρία δὲ τῶν δικαίων παρὰ Κυρίου, καὶ ὑπερασπιστὴς αὐτῶν ἐστιν ἐν καιρῷ θλίψεως.
\vs{40}Καὶ βοηθήσει αὐτοῖς Κύριος, καὶ ῥύσεται αὐτοὺς, καὶ ἐξελεῖται αὐτοὺς ἐξ ἁμαρτωλῶν, καὶ σώσει αὐτοὺς; ὅτι ἤλπισαν ἐπʼ αὐτόν.

\begin{psalmheading}{\ch{37}{38} Φαλμὸς τῷ Δαυὶδ εἰς ἀνάμνησιν περὶ σαββάτου.}
\end{psalmheading}
\vs{2}Κύριε, μὴ τῷ θυμῷ σου ἐλέγξῃς με, μηδὲ τῇ ὀργῇ σου παιδεύσῃς με.
\vs{3}Ὅτι τὰ βέλη σου ἐνεπάγησάν μοι, καὶ ἐπεστήριξας ἐπʼ ἐμὲ τὴν χεῖρά σου.

\vs{4}Οὐκ ἔστιν ἴασις ἐν τῇ σαρκί μου ἀπὸ προσώπου τῆς ὀργῆς σου, οὐκ ἔστιν εἰρήνη τοῖς ὀστέοις μου ἀπὸ προσώπου τῶν ἁμαρτιῶν μου.
\vs{5}Ὅτι αἱ ἀνομίαι μου ὑπερῇραν τὴν κεφαλήν μου, ὡσεὶ φορτίον βαρὺ ἐβαρύνθησαν ἐπʼ ἐμέ.
\vs{6}Προσώζεσαν καὶ ἐσάπησαν οἱ μώλωπές μου, ἀπὸ προσώπου τῆς ἀφροσύνης μου.
\vs{7}Ἐταλαιπώρησα καὶ κατεκάμφθην ἕως τέλους, ὅλην τὴν ἡμέραν σκυθρωπάζων ἐπορευόμην.
\vs{8}Ὅτι ἡ ψυχή μου ἐπλησθη ἐμπαιγμῶν, καὶ οὐκ ἔστιν ἴασις ἐν τῇ σαρκί μου.
\vs{9}Ἐκακώθην καὶ ἐταπεινώθην ἕως σφόδρα, ὠρυόμην ἀπὸ στεναγμοῦ τῆς καρδίας μου.

\vs{10}Καὶ ἐναντίον σου πᾶσα ἡ ἐπιθυμία μου, καὶ ὁ στεναγμός μου οὐκ ἀπεκρύβη ἀπὸ σοῦ.
\vs{11}Ἡ καρδία μου ἐταράχθη, ἐγκατέλιπέ με ἡ ἰσχύς μου, καὶ τὸ φῶς τῶν ὀφθαλμῶν μου οὐκ ἔστι μετʼ ἐμοῦ.
\vs{12}Οἱ φίλοι μου καὶ οἱ πλησίον μου ἐξ ἐναντίας μου ἤγγισαν καὶ ἔστησαν, καὶ οἱ ἔγγιστά μου μακρόθεν ἔστησαν,
\vs{13}καὶ ἐξεβιάζοντο οἱ ζητοῦντες τὴν ψυχήν μου· καὶ οἱ ζητοῦντες τὰ κακά μοι ἐλάλησαν ματαιότητας, καὶ δολιότητας ὅλην τὴν ἡμέραν ἐμελέτησαν.
\vs{14}Ἐγὼ δὲ ὡσεὶ κωφὸς οὐκ ἤκουον, καὶ ὡσεὶ ἄλαλος οὐκ ἀνοίγων τὸ στόμα αὐτοῦ·
\vs{15}Καὶ ἐγενόμην ὡσεὶ ἄνθρωπος οὐκ ἀκούων, καὶ οὐκ ἔχων ἐν τῷ στόματι αὐτοῦ ἐλεγμούς.

\vs{16}Ὅτι ἐπὶ σοὶ Κύριε ἤλπισα, σὺ εἰσακούσῃ Κύριε ὁ Θεός μου.
\vs{17}Ὅτι εἶπα, μή ποτε ἐπιχαρῶσί μοι οἱ ἐχθροί μου, καὶ ἐν τῷ σαλευθῆναι πόδας μου, ἐπʼ ἐμὲ ἐμεγαλοῤῥημόνησαν.
\vs{18}Ὅτι ἐγὼ εἰς μάστιγας ἕτοιμος, καὶ ἡ ἀλγηδών μου ἐνώπιόν μου διαπαντός.
\vs{19}Ὅτι τὴν ἀνομίαν μου ἀναγγελῶ, καὶ μεριμνήσω ὑπὲρ τῆς ἁμαρτίας μου.
\vs{20}Οἱ δὲ ἐχθροί μου ζῶσι, καὶ κεκραταίωνται ὑπὲρ ἐμὲ, καὶ ἐπληθύνθησαν οἱ μισοῦντές με ἀδίκως.
\vs{21}Οἱ ἀνταποδιδόντες κακὰ ἀντὶ ἀγαθῶν, ἐνδιέβαλλόν με, ἐπεὶ κατεδίωκον δικαιοσύνην.
\vs{22}Μὴ ἐγκαταλίπῃς με Κύριε ὁ Θεός μου, μὴ ἀποστῇς ἀπʼ ἐμοῦ.
\vs{23}Πρόσχες εἰς τὴν βοήθειάν μου Κύριε τῆς σωτηρίας μου.

\begin{psalmheading}{\ch{38}{39} Εἰς τὸ τέλος, τῷ Ἰδιθοὺν ᾠδὴ τῷ Δαυίδ.}
\end{psalmheading}
\vs{2}Εἶπα, φυλάξω τὰς ὁδούς μου, τοῦ μὴ ἁμαρτάνειν ἐν γλώσσῃ μου· ἐθέμην τῷ στόματί μου φυλακὴν, ἐν τῷ συστῆναι τὸν ἁμαρτωλὸν ἐναντίον μου.
\vs{3}Ἐκωφώθην καὶ ἐταπεινώθην καὶ ἐσίγησα ἐξ ἀγαθῶν, καὶ τὸ ἄλγημά μου ἀνεκαινίσθη.
\vs{4}Ἐθερμάνθη ἡ καρδία μου ἐντός μου, καὶ ἐν τῇ μελέτῃ μου ἐκκαυθήσεται πῦρ· ἐλάλησα ἐν γλώσσῃ μου,

\vs{5}Γνώρισόν μοι Κύριε τὸ πέρας μου, καὶ τὸν ἀριθμὸν τῶν ἡμερῶν μου τίς ἐστιν, ἵνα γνῶ τί ὑστερῶ ἐγώ.
\vs{6}Ἰδοὺ παλαιὰς ἔθου τὰς ἡμέρας μου, καὶ ὑπόστασίς μου ὡσεὶ οὐθὲν ἐνώπιόν σου· πλὴν τὰ σύμπαντα ματαιότης, πᾶς ἄνθρωπος ζῶν· διάψαλμα.
\vs{7}Μέντοιγε ἐν εἰκόνι διαπορεύεται ἄνθρωπος, πλὴν μάτην ταράσσεται· θησαυρίζει, καὶ οὐ γινώσκει τίνι συνάξει αὐτά.

\vs{8}Καὶ νῦν τίς ἡ ὑπομονή μου; οὐχὶ ὁ Κύριος; καὶ ἡ ὑπόστασίς μου παρὰ σοί ἐστι· διάψαλμα.
\vs{9}Ἀπὸ πασῶν τῶν ἀνομιῶν μου ῥῦσαί με, ὄνειδος ἄφρονι ἔδωκάς με.
\vs{10}Ἐκωφώθην καὶ οὐκ ἤνοιξα τὸ στόμα μου, ὅτι σὺ εἶ ὁ ποιήσας με.
\vs{11}Ἀπόστησον ἀπʼ ἐμοῦ τὰς μάστιγάς σου· ἀπὸ τῆς ἰσχύος τῆς χειρός σου ἐγὼ ἐξέλιπον.
\vs{12}Ἐν ἐλεγμοῖς ὑπὲρ ἀνομίας ἐπαίδευσας ἄνθρωπον· καὶ ἐξέτηξας ὡς ἀράχνην τὴν ψυχὴν αὐτοῦ, πλὴν μάτην ταράσσεται πᾶς ἄνθρωπος· διάψαλμα.

\vs{13}Εἰσάκουσον τῆς προσευχῆς μου Κύριε καὶ τῆς δεήσεώς μου· ἐνώτισαι τῶν δακρύων μου· μὴ παρασιωπήσῃς, ὅτι πάροικος ἐγώ εἰμι ἐν τῇ γῇ καὶ παρεπίδημος, καθὼς πάντες οἱ πατέρες μου.
\vs{14}Ἄνες μοι ἵνα ἀναψύξω πρὸ τοῦ με ἀπελθεῖν, καὶ οὐκέτι μὴ ὑπάρξω.

\begin{psalmheading}{\ch{39}{40} Εἰς τὸ τέλος, τῷ Δαυὶδ ψαλμός.}
\end{psalmheading}
\vs{2}Ὑπομένων ὑπέμεινα τὸν Κύριον, καὶ προσέσχε μοι, καὶ εἰσήκουσε τῆς δεήσεώς μου.
\vs{3}Καὶ ἀνήγαγέ με ἐκ λάκκου ταλαιπωρίας, καὶ ἀπὸ πηλοῦ ἰλύος· καὶ ἔστησεν ἐπὶ πέτραν τοὺς πόδας μου, καὶ κατεύθυνε τὰ διαβήματά μου.
\vs{4}Καὶ ἐνέβαλεν εἰς τὸ στόμα μου ᾆσμα καινὸν, ὕμνον τῷ Θεῷ ἡμῶν· ὄψονται πολλοὶ καὶ φοβηθήσονται, καὶ ἐλπιοῦσιν ἐπὶ Κύριον.
\vs{5}Μακάριος ἀνὴρ, οὗ ἐστι τὸ ὄνομα Κυρίου ἐλπὶς αὐτοῦ, καὶ οὐκ ἐπέβλεψεν εἰς ματαιότητας καὶ μανίας ψευδεῖς.

\vs{6}Πολλὰ ἐποίησας σὺ Κύριε ὁ Θεός μου τὰ θαυμάσιά σου, καὶ τοῖς διαλογισμοῖς σου οὐκ ἔστι τίς ὁμοιωθήσεταί σοι· ἀπήγγειλα καὶ ἐλάλησα, ἐπληθύνθησαν ὑπὲρ ἀριθμόν.
\vs{7}Θυσίαν καὶ προσφορὰν οὐκ ἠθέλησας, σῶμα δὲ κατηρτίσω μοι· ὁλοκαύτωμα καὶ περὶ ἁμαρτίας οὐκ ᾔτησας.
\vs{8}Τότε εἶπον, ἰδοὺ ἥκω· ἐν κεφαλίδι βιβλίου γέγραπται περὶ ἐμοῦ,
\vs{9}τοῦ ποιῆσαι τὸ θέλημά σου ὁ Θεός μου ἠβουλήθην, καὶ τὸν νόμον σου ἐν μέσῳ τῆς καρδίας μου.
\vs{10}Εὐηγγελισάμην δικαιοσύνην ἐν ἐκκλησίᾳ μεγάλῃ, ἰδοὺ τὰ χείλη μου οὐ μὴ κωλύσω. Κύριε, σὺ ἔγνως
\vs{11}τὴν δικαιοσύνην μου, οὐκ ἔκρυψα ἐν τῇ καρδίᾳ μου τὴν ἀλήθειάν σου, καὶ τὸ σωτήριόν σου εἶπα· οὐκ ἔκρυψα τὸ ἔλεός σου καὶ τὴν ἀλήθειάν σου ἀπὸ συναγωγῆς πολλῆς.

\vs{12}Σὺ δέ Κύριε μὴ μακρύνῃς τοὺς οἰκτιρμούς σου ἀπʼ ἐμοῦ, τὸ ἔλεός σου καὶ ἡ ἀλήθειά σου διαπαντὸς ἀντελάβοντό μου.
\vs{13}Ὅτι περιέσχον με κακὰ, ὧν οὐκ ἔστιν ἀριθμὸς, κατέλαβόν με αἱ ἀνομίαι μου, καὶ οὐκ ἠδυνάσθην τοῦ βλέπειν· ἐπληθύνθησαν ὑπὲρ τὰς τρίχας τῆς κεφαλῆς μου, καὶ ἡ καρδία μου ἐγκατέλιπέ με.
\vs{14}Εὐδόκησον Κύριε τοῦ ῥύσασθαί με, Κύριε εἰς τὸ βοηθῆσαί μοι πρόσχες.
\vs{15}Καταισχυνθείησαν καὶ ἐντραπείησαν ἅμα οἱ ζητοῦντες τὴν ψυχήν μου, τοῦ ἐξάραι αὐτήν· ἀποστραφείησαν εἰς τὰ ὀπίσω, καὶ ἐντραπείησαν οἱ θέλοντές μοι κακά.
\vs{16}Κομισάσθωσαν παραχρῆμα αἰσχύνην αὐτῶν, οἱ λέγοντές μοι, εὖγε, εὖγε.
\vs{17}Ἀγαλλιάσαιντο καὶ εὐφρανθείησαν ἐπὶ σοὶ πάντες οἱ ζητοῦντές σε Κύριε· καὶ εἰπάτωσαν διαπαντὸς, μεγαλυνθήτω ὁ Κύριος, οἱ ἀγαπῶντες τὸ σωτήριόν σου.
\vs{18}Ἐγὼ δὲ πτωχὸς καὶ πένης εἰμὶ, Κύριος φροντιεῖ μου· βοηθός μου καὶ ὑπερασπιστής μου εἶ σὺ ὁ Θεός μου, μὴ χρονίσῃς.

\begin{psalmheading}{\ch{40}{41} Εἰς τὸ τέλος, ψαλμὸς τῷ Δαυίδ.}
\end{psalmheading}
\vs{2}Μακαρίος ὁ συνιῶν ἐπὶ πτωχὸν καὶ πένητα, ἐν ἡμέρᾳ πονηρᾷ ῥύσεται αὐτὸν ὁ Κύριος.
\vs{3}Κύριος φυλάξαι αὐτὸν καὶ ζήσαι αὐτὸν, καὶ μακαρίσαι αὐτὸν ἐν τῇ γῇ, καὶ μὴ παραδοῖ αὐτὸν εἰς χεῖρας ἐχθροῦ αὐτοῦ.
\vs{4}Κύριος βοηθήσαι αὐτῷ ἐπὶ κλίνης ὀδύνης αὐτοῦ, ὅλην τὴν κοίτην αὐτοῦ ἔστρεψας ἐν τῇ ἀῤῥωστίᾳ αὐτοῦ.

\vs{5}Ἐγὼ εἶπα, Κύριε ἐλέησόν με, ἴασαι τὴν ψυχήν μου, ὅτι ἥμαρτόν σοι.
\vs{6}Οἱ ἐχθροί μου εἶπαν κακά μοι, πότε ἀποθανεῖται καὶ ἀπολεῖται τὸ ὄνομα αὐτοῦ;
\vs{7}Καὶ εἰ εἰσεπορεύετο τοῦ ἰδεῖν, μάτην ἐλάλει ἡ καρδία αὐτοῦ, συνήγαγεν ἀνομίαν ἑαυτῷ, ἐξεπορεύετο ἔξω, καὶ ἐλάλει
\vs{8}ἐπὶ τὸ αὐτό. Κατʼ ἐμοῦ ἐψιθύριζον πάντες οἱ ἐχθροί μου, κατʼ ἐμοῦ ἐλογίζοντο κακά μοι.
\vs{9}Λόγον παράνομον κατέθεντο κατʼ ἐμοῦ, μὴ ὁ κοιμώμενος οὐχὶ προσθήσει τοῦ ἀναστῆναι;
\vs{10}Καὶ γὰρ ὁ ἄνθρωπος τῆς εἰρήνης μου ἐφʼ ὃν ἤλπισα, ὁ ἐσθίων ἄρτους μου ἐμεγάλυνεν ἐπʼ ἐμὲ πτερνισμόν.

\vs{11}Σὺ δὲ Κύριε ἐλέησόν με, καὶ ἀνάστησόν με, καὶ ἀνταποδώσω αὐτοῖς.
\vs{12}Ἐν τούτῳ ἔγνων, ὅτι τεθέληκάς με ὅτι οὐ μὴ ἐπιχαρῇ ὁ ἐχθρός μου ἐπʼ ἐμέ.
\vs{13}Ἐμοῦ δὲ διὰ τὴν ἀκακίαν ἀντελάβου, καὶ ἐβεβαίωσάς με ἐνώπιόν σου εἰς τὸν αἰῶνα.
\vs{14}Εὐλογητὸς Κύριος ὁ Θεὸς Ἰσραὴλ ἀπὸ τοῦ αἰῶνος καὶ εἰς τὸν αἰῶνα· γένοιτο, γένοιτο.

\begin{psalmheading}{\ch{41}{42} Εἰς τὸ τέλος, εἰς σύνεσιν τοῖς υἱοῖς Κορέ.}
\end{psalmheading}
\vs{2}Ὃν τρόπον ἐπιποθεῖ ἡ ἔλαφος ἐπὶ τὰς πηγὰς τῶν ὑδάτων, οὕτως ἐπιποθεῖ ἡ ψυχή μου πρὸς σὲ ὁ Θεός.
\vs{3}Ἐδίψησεν ἡ ψυχή μου πρὸς τὸν Θεὸν τὸν ζῶντα· πότε ἥξω καὶ ὀφθήσομαι τῷ προσώπῳ τοῦ Θεοῦ;
\vs{4}Ἐγενήθη τὰ δάκρυά μου ἐμοὶ ἄρτος ἡμέρας καὶ νυκτὸς, ἐν τῷ λέγεσθαί μοι καθʼ ἑκάστην ἡμέραν, ποῦ ἐστιν ὁ Θεός σου;
\vs{5}Ταῦτα ἐμνήσθην, καὶ ἐξέχεα ἐπʼ ἐμὲ τὴν ψυχήν μου, ὅτι διελεύσομαι ἐν τόπῳ σκηνῆς θαυμαστῆς ἕως τοῦ οἴκου τοῦ Θεοῦ· ἐν φωνῇ ἀγαλλιάσεως καὶ ἐξομολογήσεως ἤχου ἑορταζόντων.

\vs{6}Ἱνατί περίλυπος εἶ ἡ ψυχή μου, καὶ ἱνατί συνταράσσεις με; ἔλπισον ἐπὶ τὸν Θεὸν, ὅτι ἐξομολογήσομαι αὐτῷ, σωτήριον τοῦ προσώπου μου,

\vs{7}Ὁ Θεός μου· πρὸς ἐμαυτὸν ἡ ψυχή μου ἐταράχθη, διὰ τοῦτο μνησθήσομαί σου ἐκ γῆς Ἰορδάνου, καὶ Ἐρμωνιεὶμ ἀπὸ ὄρους μικροῦ.
\vs{8}Ἄβυσσος ἄβυσσον ἐπικαλεῖται εἰς φωνὴν τῶν καταῤῥακτῶν σου· πάντες οἱ μετεωρισμοί σου, καὶ τὰ κύματά σου ἐπʼ ἐμὲ διῆλθον.
\vs{9}Ἡμέρας ἐντελεῖται Κύριος τὸ ἔλεος αὐτοῦ, καὶ νυκτὸς δηλώσει· παρʼ ἐμοὶ προσευχὴ τῷ Θεῷ τῆς ζωῆς μου,
\vs{10}ἐρῶ τῷ Θεῷ, ἀντιλήπτωρ μου εἶ, διατί μου ἐπελάθου; ἱνατί σκυθρωπάζων πορεύομαι ἐν τῷ ἐκθλίβειν τὸν ἐχθρόν μου;
\vs{11}Ἐν τῷ καταθλᾶσθαι τὰ ὀστᾶ μου, ὠνείδισάν με οἱ θλίβοντές με· ἐν τῷ λέγειν αὐτούς μοι καθʼ ἑκάστην ἡμέραν, ποῦ ἐστιν ὁ Θεός σου;

\vs{12}Ἱνατί περίλυπος εἶ ἡ ψυχή μου, καὶ ἱνατί συνταράσσεις με; ἔλπισον ἐπὶ τὸν Θεὸν, ὅτι ἐξομολογήσομαι αὐτῷ, ἡ σωτηρία τοῦ προσώπου μου, καὶ ὁ Θεός μου.

\begin{psalmheading}{\ch{42}{43} Ψαλμὸς τῷ Δαυίδ.}
\end{psalmheading}
Κρίνον με ὁ Θεὸς, καὶ δίκασον τὴν δίκην μου, ἐξ ἔθνους οὐχ ὁσίου, ἀπὸ ἀνθρώπου ἀδίκου καὶ δολίου ῥῦσαί με.
\vs{2}Ὅτι σὺ εἶ ὁ Θεὸς κραταίωμά μου, ἱνατί ἀπώσω με; καὶ ἱνατί σκυθρωπάζων πορεύομαι ἐν τῷ ἐκθλίβειν τὸν ἐχθρόν μου;
\vs{3}Ἐξαπόστειλον τὸ φῶς σου καὶ τὴν ἀλήθειάν σου, αὐτά με ὡδήγησαν καὶ ἤγαγόν με εἰς ὄρος ἅγιόν σου, καὶ εἰς τὰ σκηνώματά σου.
\vs{4}Καὶ εἰσελεύσομαι πρὸς τὸ θυσιαστήριον τοῦ Θεοῦ, πρὸς τὸν Θεὸν τὸν εὐφραίνοντα τὴν νεότητά μου, ἐξομολογήσομαί σοι ἐν κιθάρᾳ ὁ Θεὸς ὁ Θεός μου.

\vs{5}Ἱνατί περίλυπος εἶ ἡ ψυχή μου, καὶ ἱνατί συνταράσσεις με; ἔλπισον ἐπὶ τὸν Θεὸν, ὅτι ἐξομολογήσομαι αὐτῷ, σωτήριον τοῦ προσώπου μου, ὁ Θεός μου.

\begin{psalmheading}{\ch{43}{44} Εἰς τὸ τέλος, τοῖς υἱοῖς Κορὲ εἰς σύνεσιν ψαλμός.}
\end{psalmheading}
\vs{2}Ὁ Θεὸς ἐν τοῖς ὠσὶν ἡμῶν ἠκούσαμεν, οἱ πατέρες ἡμῶν ἀνήγγειλαν ἡμῖν, ἔργον ὃ εἰργάσω ἐν ταῖς ἡμέραις αὐτῶν, ἐν ἡμεραίς ἀρχαίαις.
\vs{3}Ἡ χείρ σου ἔθνη ἐξωλόθρευσε, καὶ κατεφύτευσας αὐτοὺς, ἐκάκωσας λαοὺς καὶ ἐξέβαλες αὐτούς.
\vs{4}Οὐ γὰρ ἐν τῇ ῥομφαίᾳ αὐτῶν ἐκληρονόμησαν γῆν, καὶ ὁ βραχίων αὐτῶν οὐκ ἔσωσεν αὐτοὺς, ἀλλὰ ἡ δεξιά σου καὶ ὁ βραχίων σου, καὶ ὁ φωτισμὸς τοῦ προσώπου σου, ὅτι εὐδόκησας ἐν αὐτοῖς.

\vs{5}Σὺ εἶ αὐτὸς ὁ βασιλεύς μου καὶ ὁ Θεός μου, ὁ ἐντελλόμενος τὰς σωτηρίας Ἰακώβ.
\vs{6}Ἐν σοὶ τοὺς ἐχθροὺς ἡμῶν κερατιοῦμεν, καὶ ἐν τῷ ὀνόματί σου ἐξουθενώσομεν τοὺς ἐπανισταμένους ἡμῖν.
\vs{7}Οὐ γὰρ ἐπὶ τῷ τόξῳ μου ἐλπιῶ, καὶ ἡ ῥομφαία μου οὐ σώσει με.
\vs{8}Ἔσωσας γὰρ ἡμᾶς ἐκ τῶν θλιβόντων ἡμᾶς, καὶ τοὺς μισοῦντας ἡμᾶς κατῄσχυνας.
\vs{9}Ἐν τῷ Θεῷ ἐπαινεθησόμεθα ὅλην τὴν ἡμέραν, καὶ ἐν τῷ ὀνόματί σου ἐξομολογησόμεθα εἰς τὸν αἰῶνα. διάψαλμα.

\vs{10}Νυνὶ δὲ ἀπώσω καὶ κατῄσχυνας ἡμᾶς, καὶ οὐκ ἐξελεύσῃ ἐν ταῖς δυνάμεσιν ἡμῶν.
\vs{11}Ἀπέστρεψας ἡμᾶς εἰς τὰ ὀπίσω παρὰ τοὺς ἐχθροὺς ἡμῶν, καὶ οἱ μισοῦντες ἡμᾶς διήρπαζον ἑαυτοῖς.
\vs{12}Ἔδωκας ἡμᾶς ὡς πρόβατα βρώσεως, καὶ ἐν τοῖς ἔθνεσι διέσπειρας ἡμᾶς·
\vs{13}Ἀπέδου τὸν λαόν σου ἄνευ τιμῆς, καὶ οὐκ ἦν πλῆθος ἐν τοῖς ἀλαλάγμασιν αὐτῶν.
\vs{14}Ἔθου ἡμᾶς ὄνειδος τοῖς γείτοσιν ἡμῶν, μυκτηρισμὸν καὶ καταγέλωτα τοῖς κύκλῳ ἡμῶν.
\vs{15}Ἔθου ἡμᾶς εἰς παραβολὴν ἐν τοῖς ἔθνεσι, κίνησιν κεφαλῆς ἐν τοῖς λαοῖς.
\vs{16}Ὅλην τὴν ἡμέραν ἡ ἐντροπή μου κατεναντίον μου ἐστὶ, καὶ ἡ αἰσχύνη τοῦ προσώπου μου ἐκάλυψέ με,
\vs{17}ἀπὸ φωνῆς ὀνειδίζοντος καὶ παραλαλοῦντος, ἀπὸ προσώπου ἐχθροῦ καὶ ἐκδιώκοντος.

\vs{18}Ταῦτα πάντα ἦλθεν ἐφʼ ἡμᾶς, καὶ οὐκ ἐπελαθόμεθά σου, καὶ οὐκ ἠδικήσαμεν ἐν διαθήκῃ σου.
\vs{19}Καὶ οὐκ ἀπέστη εἰς τὰ ὀπίσω ἡ καρδία ἡμῶν, καὶ ἐξέκλινας τὰς τρίβους ἡμῶν ἀπὸ τῆς ὁδοῦ σου.
\vs{20}Ὅτι ἐταπείνωσας ἡμᾶς ἐν τόπῳ κακώσεως, καὶ ἐπεκάλυψεν ἡμᾶς σκιὰ θανάτου.
\vs{21}Εἰ ἐπελαθόμεθα τοῦ ὀνόματος τοῦ Θεοῦ ἡμῶν, καὶ εἰ διεπετάσαμεν χεῖρας ἡμῶν πρὸς θεὸν ἀλλότριον,
\vs{22}οὐχὶ ὁ Θεὸς ἐκζητήσει ταῦτα; αὐτὸς γὰρ γινώσκει τὰ κρύφια τῆς καρδίας.
\vs{23}Ὅτι ἕνεκά σου θανατούμεθα ὅλην τὴν ἡμέραν, ἐλογίσθημεν ὡς πρόβατα σφαγῆς.

\vs{24}Ἐξεγέρθητι, ἱνατί ὑπνοῖς Κύριε; ἀνάστηθι, καὶ μὴ ἀπώσῃ εἰς τέλος.
\vs{25}Ἱνατί τὸ πρόσωπόν σου ἀποστρέφεις; ἐπιλανθάνῃ τῆς πτωχείας ἡμῶν καὶ τῆς θλίψεως ἡμῶν;
\vs{26}Ὅτι ἐταπεινώθη εἰς χοῦν ἡ ψυχὴ ἡμῶν, ἐκολλήθη εἰς γῆν ἡ γαστὴρ ἡμῶν.
\vs{27}Ἀνάστα Κύριε, βοήθησον ἡμῖν, καὶ λύτρωσαι ἡμᾶς ἕνεκεν τοῦ ὀνόματός σου.

\begin{psalmheading}{\ch{44}{45} Εἰς τὸ τέλος, ὑπὲρ τῶν ἀλλοιωθησομένων τοῖς υἱοῖς Κορὲ εἰς σύνεσιν, ᾠδὴ ὑπὲρ τοῦ ἀγαπητοῦ.}
\end{psalmheading}
\vs{2}Ἐξηρεύξατο ἡ καρδία μου λόγον ἀγαθὸν, λέγω ἐγὼ τὰ ἔργα μου τῷ βασιλεῖ· ἡ γλῶσσά μου κάλαμος γραμματέως ὀξυγράφου.
\vs{3}Ὡραῖος κάλλει παρὰ τοὺς υἱοὺς τῶν ἀνθρώπων, ἐξεχύθη χάρις ἐν χείλεσί σου, διὰ τοῦτο εὐλόγησέ σε ὁ Θεὸς εἰς τὸν αἰῶνα.

\vs{4}Περίζωσαι τὴν ῥομφαίαν σου ἐπὶ τὸν μηρόν σου δυνατέ· τῇ ὡραιότητί σου, καὶ τῷ κάλλει σου,
\vs{5}καὶ ἔντεινον, καὶ κατευοδοῦ, καὶ βασίλευε· ἕνεκεν ἀληθείας καὶ πρᾳΰτητος καὶ δικαιοσύνης, καὶ ὁδηγήσει σε θαυμαστῶς ἡ δεξιά σου.
\vs{6}Τὰ βέλη σου ἠκονημένα δυνατὲ, λαοὶ ὑποκάτω σου πεσοῦνται, ἐν καρδίᾳ τῶν ἐχθρῶν τοῦ βασιλέως.

\vs{7}Ὁ θρόνος σου ὁ Θεὸς εἰς αἰῶνα αἰῶνος, ῥάβδος εὐθύτητος ἡ ῥάβδος τῆς βασιλείας σου.
\vs{8}Ἠγάπησας δικαιοσύνην, καὶ ἐμίσησας ἀνομίαν, διὰ τοῦτο ἔχρισέ σε ὁ Θεὸς ὁ Θεός σου ἔλαιον ἀγαλλιάσεως παρὰ τοὺς μετόχους σου.

\vs{9}Σμύρνα καὶ στακτὴ καὶ κασία ἀπὸ τῶν ἱματίων σου, ἀπὸ βάρεων ἐλεφαντίνων, ἐξ ὧν ηὔφρανάν σε
\vs{10}θυγατέρες βασιλέων ἐν τῇ τιμῇ σου· παρέστη ἡ βασίλισσα ἐκ δεξιῶν σου, ἐν ἱματισμῷ διαχρύσῳ περιβεβλημένη, πεποικιλμένη.
\vs{11}Ἄκουσον θύγατερ καὶ ἴδε καὶ κλῖνον τὸ οὖς σου, καὶ ἐπιλάθου τοῦ λαοῦ σου, καὶ τοῦ οἴκου τοῦ πατρός σου.
\vs{12}Ὅτι ἐπεθύμησεν ὁ βασιλεὺς τοῦ κάλλους σου, ὅτι αὐτός ἐστιν ὁ Κύριός σου.
\vs{13}Καὶ προσκυνήσουσιν αὐτῷ θυγατέρες Τύρου ἐν δώροις, τὸ πρόσωπόν σου λιτανεῦσουσιν οἱ πλούσιοι τοῦ λαοῦ τῆς γῆς.

\vs{14}Πᾶσα ἡ δόξα αὐτῆς θυγατρὸς τοῦ βασιλέως Ἐσεβὼν, ἐν κροσσωτοῖς χρυσοῖς περιβεβλημένη, πεποικιλμένη·
\vs{15}ἀπενεχθήσονται τῷ βασιλεῖ παρθένοι ὀπίσω αὐτῆς, αἱ πλησίον αὐτῆς ἀπενεχθήσονταί σοι.
\vs{16}Ἀπενεχθήσονται ἐν εὐφροσύνῃ καὶ ἀγαλλιάσει, ἀχθήσονται εἰς ναὸν βασιλέως.
\vs{17}Ἀντὶ τῶν πατέρων σου ἐγενήθησάν σοι υἱοὶ, καταστήσεις αὐτοὺς ἄρχοντας ἐπὶ πᾶσαν τὴν γῆν.
\vs{18}Μνησθήσονται τοῦ ὀνόματός σου ἐν πάσῃ γενεᾷ καὶ γενεᾷ, διὰ τοῦτο λαοὶ ἐξομολογήσονταί σοι εἰς τὸν αἰῶνα, καὶ εἰς τὸν αἰῶνα τοῦ αἰῶνος·

\begin{psalmheading}{\ch{45}{46} Εἰς τὸ τέλος, ὑπὲρ τῶν υἱῶν Κορὲ, ὑπὲρ τῶν κρυφίων ψαλμός.}
\end{psalmheading}
\vs{2}Ὁ Θεὸς ἡμῶν καταφυγὴ καὶ δύναμις, βοηθὸς ἐν θλίψεσι ταῖς εὑρούσαις ἡμᾶς σφόδρα.
\vs{3}Διὰ τοῦτο οὐ φοβηθησόμεθα ἐν τῷ ταράσσεσθαι τὴν γῆν, καὶ μετατίθεσθαι ὄρη ἐν καρδίαις θαλασσῶν.
\vs{4}Ἤχησαν καὶ ἐταράχθησαν τὰ ὕδατα αὐτῶν, ἐταράχθησαν τὰ ὄρη ἐν τῇ κραταιότητι αὐτοῦ. διάψαλμα.
\vs{5}Τοῦ ποταμοῦ τὰ ὁρμήματα εὐφραίνουσι τὴν πόλιν τοῦ Θεοῦ, ἡγίασε τὸ σκήνωμα αὐτοῦ ὁ ὕψιστος.
\vs{6}Ὁ Θεὸς ἐν μέσῳ αὐτῆς οὐ σαλευθήσεται, βοηθήσει αὐτῇ ὁ Θεὸς τῷ προσώπῳ.
\vs{7}Ἐταράχθησαν ἔθνη, ἔκλιναν βασιλεῖαι, ἔδωκε φωνὴν αὐτοῦ, ἐσαλεύθη ἡ γῆ.
\vs{8}Κύριος τῶν δυνάμεων μεθʼ ἡμῶν, ἀντιλήπτωρ ἡμῶν ὁ Θεὸς Ἰακώβ. διάψαλμα.

\vs{9}Δεῦτε καὶ ἴδετε τὰ ἔργα τοῦ Κυρίου, ἃ ἔθετο τέρατα ἐπὶ τῆς γῆς·
\vs{10}ἀνταναιρῶν πολέμους μέχρι τῶν περάτων τῆς γῆς, τόξον συντρίψει, καὶ συγκλάσει ὅπλον, καὶ θυρεοὺς κατακαύσει ἐν πυρί.
\vs{11}Σχολάσατε καὶ γνῶτε, ὅτι ἐγώ εἰμι ὁ Θεὸς, ὑψωθήσομαι ἐν τοῖς ἔθνεσιν, ὑψωθήσομαι ἐν τῇ γῇ·
\vs{12}Κύριος τῶν δυνάμεων μεθʼ ἡμῶν, ἀντιλήπτωρ ἡμῶν ὁ Θεὸς Ἰακώβ.

\begin{psalmheading}{\ch{46}{47} Εἰς τὸ τέλος, ὑπὲρ τῶν υἱῶν Κορὲ ψαλμός.}
\end{psalmheading}
\vs{2}Παντὰ τὰ ἔθνη κροτήσατε χεῖρας, ἀλαλάξατε τῷ Θεῷ ἐν φωνῇ ἀγαλλιάσεως.
\vs{3}Ὅτι Κύριος ὕψιστος, φοβερὸς, βασιλεὺς μέγας ἐπὶ πᾶσαν τὴν γῆν.
\vs{4}Ὑπέταξε λαοὺς ἡμῖν, καὶ ἔθνη ὑπὸ τοὺς πόδας ἡμῶν·
\vs{5}Ἐξελέξατο ἡμῖν τὴν κληρονομίαν αὐτοῦ, τὴν καλλονὴν Ἰακὼβ, ἣν ἠγάπησε· διάψαλμα.

\vs{6}Ἀνέβη ὁ Θεὸς ἐν ἀλαλαγμῷ, Κύριος ἐν φωνῇ σάλπιγγος.
\vs{7}Ψάλατε τῷ Θεῷ ἡμῶν, ψάλατε· ψάλατε τῷ βασιλεῖ ἡμῶν, ψάλατε.
\vs{8}Ὅτι βασιλεὺς πάσης τῆς γῆς ὁ Θεὸς, ψάλατε συνετῶς.
\vs{9}Ἐβασίλευσεν ὁ Θεὸς ἐπὶ τὰ ἔθνη, ὁ Θεὸς κάθηται ἐπὶ θρόνου ἁγίου αὐτοῦ.
\vs{10}Ἄρχοντες λαῶν συνήχθησαν μετὰ τοῦ Θεοῦ Ἀβραὰμ, ὅτι τοῦ Θεοῦ οἱ κραταιοὶ τῆς γῆς σφόδρα ἐπῄρθησαν.

\begin{psalmheading}{\ch{47}{48} Ψαλμὸς ᾠδῆς τοῖς υἱοῖς Κορὲ δευτέρᾳ σαββάτου.}
\end{psalmheading}
\vs{2}Μεγὰς Κύριος, καὶ αἰνετὸς σφόδρα ἐν πόλει τοῦ Θεοῦ ἡμῶν, ἐν ὄρει ἁγίῳ αὐτοῦ.
\vs{3}Εὐρίζων, ἀγαλλιάματι πάσης τῆς γῆς, ὄρη Σιὼν τὰ πλευρὰ τοῦ βοῤῥᾶ, ἡ πόλις τοῦ βασιλέως τοῦ μεγάλου.
\vs{4}Ὁ Θεὸς ἐν ταῖς βάρεσιν αὐτῆς γινώσκεται, ὅταν ἀντιλαμβάνηται αὐτῆς.

\vs{5}Ὅτι ἰδοὺ οἱ βασιλεῖς τῆς γῆς συνήχθησαν, ἤλθοσαν ἐπὶ τὸ αὐτό.
\vs{6}Αὐτοὶ ἰδόντες οὕτως ἐθαύμασαν, ἐταράχθησαν, ἐσαλεύθησαν,
\vs{7}τρόμος ἐπελάβετο αὐτῶν· ἐκεῖ ὠδῖνες ὡς τικτούσης.
\vs{8}Ἐν πνεύματι βιαίῳ συντρίψεις πλοῖα Θάρσις.
\vs{9}Καθάπερ ἠκούσαμεν, οὕτως καὶ εἴδομεν, ἐν πόλει Κυρίου τῶν δυνάμεων, ἐν πόλει τοῦ Θεοῦ ἡμῶν, ὁ Θεὸς ἐθεμελίωσεν αὐτὴν εἰς τὸν αἰῶνα· διάψαλμα.

\vs{10}Ὑπελάβομεν ὁ Θεὸς τὸ ἔλεός σου ἐν μέσῳ τοῦ λαοῦ σου.
\vs{11}Κατὰ τὸ ὄνομά σου ὁ Θεὸς, οὕτως καὶ ἡ αἴνεσίς σου ἐπὶ τὰ πέρατα τῆς γῆς, δικαιοσύνης πλήρης ἡ δεξιά σου.
\vs{12}Εὐφρανθήτω τὸ ὄρος Σιὼν, ἀγαλλιάσθωσαν αἱ θυγατέρες τῆς Ἰουδαίας ἕνεκα τῶν κριμάτων σου Κύριε.

\vs{13}Κυκλώσατε Σιὼν, καὶ περιλάβετε αὐτὴν, διηγήσασθε ἐν τοῖς πύργοις αὐτῆς.
\vs{14}Θέσθε τὰς καρδίας ὑμῶν εἰς τὴν δύναμιν αὐτῆς, καὶ καταδιέλεσθε τὰς βάρεις αὐτῆς, ὅπως ἂν διηγήσησθε εἰς γενεὰν ἑτέραν.
\vs{15}Ὅτι οὗτός ἐστιν ὁ Θεὸς ἡμῶν εἰς τὸν αἰῶνα καὶ εἰς τὸν αἰῶνα τοῦ αἰῶνος, αὐτὸς ποιμανεῖ ἡμᾶς εἰς τοὺς αἰῶνας.

\begin{psalmheading}{\ch{48}{49} Εἰς τὸ τέλος, τοῖς υἱοῖς Κορὲ ψαλμός.}
\end{psalmheading}
\vs{2}Ἀκούσατε ταῦτα πάντα τὰ ἔθνη, ἐνωτίσασθε πάντες οἱ κατοικοῦντες τὴν οἰκουμένην,
\vs{3}οἵ τε γηγενεῖς καὶ οἱ υἱοὶ τῶν ἀνθρώπων, ἐπιτοαυτὸ πλούσιος καὶ πένης.
\vs{4}Τὸ στόμα μου λαλήσει σοφίαν, καὶ ἡ μελέτη τῆς καρδίας μου σύνεσιν.
\vs{5}Κλινῶ εἰς παραβολὴν τὸ οὖς μου, ἀνοίξω ἐν ψαλτηρίῳ τὸ πρόβλημά μου.

\vs{6}Ἱνατί φοβοῦμαι ἐν ἡμέρᾳ πονηρᾷ; ἡ ἀνομία τῆς πτέρνης μου κυκλώσει με.
\vs{7}Οἱ πεποιθότες ἐπὶ τῇ δυνάμει αὐτῶν, καὶ ἐπὶ τῷ πλήθει τοῦ πλούτου αὐτῶν καυχώμενοι.
\vs{8}Ἀδελφὸς οὐ λυτροῦται, λυτρώσεται ἄνθρωπος; οὐ δώσει τῷ Θεῷ ἐξίλασμα ἑαυτοῦ,
\vs{9}καὶ τὴν τιμὴν τῆς λυτρώσεως τῆς ψυχῆς αὐτοῦ· καὶ ἐκοπίασεν εἰς τὸν αἰῶνα,
\vs{10}καὶ ζήσεται εἰς τέλος· ὅτι οὐκ ὄψεται καταφθορὰν,

\vs{11}Ὅταν ἴδῃ σοφοὺς ἀποθνήσκοντας, ἐπιτοαυτὸ ἄφρων καὶ ἄνους ἀπολοῦνται, καὶ καταλείψουσιν ἀλλοτρίοις τὸν πλοῦτον αὐτῶν.
\vs{12}Καὶ οἱ τάφοι αὐτῶν οἰκίαι αὐτῶν εἰς τὸν αἰῶνα, σκηνώματα αὐτῶν εἰς γενεὰν καὶ γενεὰν, ἐπεκαλέσαντο τὰ ὀνόματα αὐτῶν ἐπὶ τῶν γαιῶν αὐτῶν.
\vs{13}Καὶ ἄνθρωπος ἐν τιμῇ ὢν, οὐ συνῆκε, παρασυνεβλήθη τοῖς κτήνεσι τοῖς ἀνοήτοις, καὶ ὡμοιώθη αὐτοῖς.
\vs{14}Αὕτη ἡ ὁδὸς αὐτῶν σκάνδαλον αὐτοῖς, καὶ μετὰ ταῦτα ἐν τῷ στόματι αὐτῶν εὐλογήσουσι· διάψαλμα.
\vs{15}Ὡς πρόβατα ἐν ᾅδῃ ἔθεντο, θάνατος ποιμανεῖ αὐτούς· καὶ κατακυριεύσουσιν αὐτῶν οἱ εὐθεῖς τοπρωῒ, καὶ ἡ βοήθεια αὐτῶν παλαιωθήσεται ἐν τῷ ᾅδῃ ἐκ τῆς δόξης αὐτῶν.
\vs{16}Πλὴν ὁ Θεὸς λυτρώσεται τὴν ψυχήν μου ἐκ χειρὸς ᾅδου, ὅταν λαμβάνῃ με· διάψαλμα.

\vs{17}Μὴ φοβοῦ ὅταν πλουτήσῃ ἄνθρωπος, καὶ ὅταν πληθυνθῇ ἡ δόξα τοῦ οἴκου αὐτοῦ.
\vs{18}Ὅτι οὐκ ἐν τῷ ἀποθνήσκειν αὐτὸν λήψεται τὰ πάντα, οὐδὲ συγκαταβήσεται αὐτῷ ἡ δόξα αὐτοῦ.
\vs{19}Ὅτι ἡ ψυχὴ αὐτοῦ ἐν τῇ ζωῇ αὐτοῦ εὐλογηθήσεται, ἐξομολογήσεταί σοι ὅταν ἀγαθύνῃς αὐτῷ.
\vs{20}Εἰσελεύσεται ἕως γενεᾶς πατέρων αὐτοῦ, ἕως αἰῶνος οὐκ ὄψεται φῶς.
\vs{21}Ἄνθρωπος ἐν τιμῇ ὢν, οὐ συνῆκε, παρασυνεβλήθη τοῖς κτήνεσι τοῖς ἀνοήτοις, καὶ ὡμοιώθη αὐτοῖς.

\begin{psalmheading}{\ch{49}{50} Ψαλμὸς τῷ Ἀσάφ.}
\end{psalmheading}
Θεὸς θεῶν Κύριος ἐλάλησε, καὶ ἐκάλεσε τὴν γῆν ἀπὸ ἀνατολῶν ἡλίου μέχρι δυσμῶν.
\vs{2}Ἐκ Σιὼν ἡ εὐπρέπεια τῆς ὡραιότητος αὐτοῦ.
\vs{3}Ὁ Θεὸς ἐμφανῶς ἥξει, ὁ Θεὸς ἡμῶν, καὶ οὐ παρασιωπήσεται· πῦρ ἐναντίον αὐτοῦ καυθήσεται, καὶ κύκλῳ αὐτοῦ καταιγὶς σφοδρά.
\vs{4}Προσκαλέσεται τὸν οὐρανὸν ἄνω, καὶ τὴν γῆν διακρῖναι τὸν λαὸν αὐτοῦ.
\vs{5}Συναγάγετε αὐτῷ τοὺς ὁσίους αὐτοῦ, τοὺς διατιθεμένους τὴν διαθήκην αὐτοῦ ἐπὶ θυσίαις.
\vs{6}Καὶ ἀναγγελοῦσιν οἱ οὐρανοὶ τὴν δικαιοσύνην αὐτοῦ, ὅτι Θεὸς κριτής ἐστι· διάψαλμα.

\vs{7}Ἄκουσον λαός μου καὶ λαλήσω σοι, Ἰσραὴλ, καὶ διαμαρτύρομαί σοι· ὁ Θεὸς ὁ Θεός σου εἰμὶ ἐγώ.
\vs{8}Οὐκ ἐπὶ ταῖς θυσίαις σου ἐλέγξω σε, τὰ δὲ ὁλοκαυτώματά σου ἐνώπιόν μου ἐστὶ διαπαντός.
\vs{9}Οὐ δέξομαι ἐκ τοῦ οἴκου σου μόσχους, οὐδὲ ἐκ τῶν ποιμνίων σου χιυάρους·
\vs{10}Ὅτι ἐμά ἐστι πάντα τὰ θηρία τοῦ δρυμοῦ, κτήνη ἐν τοῖς ὄρεσι, καὶ βόες.
\vs{11}Ἔγνωκα πάντα τὰ πετεινὰ τοῦ οὐρανοῦ, καὶ ὡραιότης ἀγροῦ μετʼ ἐμοῦ ἐστιν.
\vs{12}Ἐὰν πεινάσω, οὐ μή σοι εἴπω, ἐμὴ γάρ ἐστιν ἡ οἰκουμένη καὶ τὸ πλήρωμα αὐτῆς.
\vs{13}Μὴ φάγομαι κρέα ταύρων, ἢ αἷμα τράγων πίομαι;
\vs{14}Θῦσον τῷ Θεῷ θυσίαν αἰνέσεως, καὶ ἀπόδος τῷ ὑψίστῳ τὰς εὐχάς σου.
\vs{15}Καὶ ἐπικάλεσαί με ἐν ἡμέρᾳ θλίψεως, καὶ ἐξελοῦμαί σε, καὶ δοξάσεις με. διάψαλμα.

\vs{16}Τῷ δὲ ἁμαρτωλῷ εἶπεν ὁ Θεὸς, ἱνατί σὺ διηγῇ τὰ δικαιώματά μου, καὶ ἀναλαμβάνεις τὴν διαθήκην μου διὰ στόματός σου;
\vs{17}Σὺ δὲ ἐμίσησας παιδείαν, καὶ ἐξέβαλες τοὺς λόγους μου εἰς τὰ ὀπίσω.
\vs{18}Εἰ ἐθεώρεις κλέπτην, συνέτρεχες αὐτῷ, καὶ μετὰ μοιχῶν τὴν μερίδα σου ἐτίθεις.
\vs{19}Τὸ στόμα σου ἐπλεόνασε κακίαν, καὶ ἡ γλῶσσά σου περιέπλεκε δολιότητα.
\vs{20}Καθήμενος κατὰ τοῦ ἀδελφοῦ σου κατελάλεις, καὶ κατὰ τοῦ υἱοῦ τῆς μητρός σου ἐτίθεις σκάνδαλον.

\vs{21}Ταῦτα ἐποίησας, καὶ ἐσίγησα, ὑπέλαβες ἀνομίαν ὅτι ἔσομαί σοι ὅμοιος· ἐλέγξω σε, καὶ παραστήσω κατὰ πρόσωπόν σου.
\vs{22}Σύνετε δὴ ταῦτα οἱ ἐπιλανθανόμενοι τοῦ Θεοῦ, μήποτε ἁρπάσῃ, καὶ μὴ ᾖ ὁ ῥυόμενος.

\vs{23}Θυσία αἰνέσεως δοξάσει με, καὶ ἐκεῖ ὁδὸς ᾗ δείξω αὐτῷ τὸ σωτήριον Θεοῦ.

\begin{psalmheading}{\ch{50}{51} Εἰς τὸ τέλος, ψαλμὸς τῷ Δαυὶδ,}\vs{2}ἐν τῷ ἐλθεῖν πρὸς αὐτὸν Νάθαν τὸν προφήτην, ἡνίκα εἰσῆλθε πρὸς Βηρσαβεέ.
\end{psalmheading}

\vs{3}Ἐλέησον με ὁ Θεὸς κατὰ τὸ μέγα ἔλεός σου, καὶ κατὰ τὸ πλῆθος τῶν οἰκτιρμῶν σου ἐξάλειψον τὸ ἀνόμημά μου.
\vs{4}Ἐπιπλεῖον πλῦνόν με ἀπὸ τῆς ἀνομίας μου, καὶ ἀπὸ τῆς ἁμαρτίας μου καθάρισόν με.

\vs{5}Ὅτι τὴν ἀνομίαν μου ἐγὼ γινώσκω, καὶ ἡ ἁμαρτία μου ἐνώπιόν μου ἐστὶ διαπαντός·
\vs{6}Σοὶ μόνῳ ἥμαρτον, καὶ τὸ πονηρὸν ἐνώπιόν σου ἐποίησα· ὅπως ἂν δικαιωθῇς ἐν τοῖς λόγοις σου, καὶ νικήσῃς ἐν τῷ κρίνεσθαί σε.
\vs{7}Ἰδοὺ γὰρ ἐν ἀνομίαις συνελήφθην, καὶ ἐν ἁμαρτίαις ἐκίσσησέ με ἡ μήτηρ μου.

\vs{8}Ἰδοὺ γὰρ ἀλήθειαν ἠγάπησας, τὰ ἄδηλα καὶ τὰ κρύφια τῆς σοφίας σου ἐδήλωσάς μοι.
\vs{9}Ῥαντιεῖς με ὑσσώπῳ καὶ καθαρισθήσομαι, πλυνεῖς με καὶ ὑπὲρ χιόνα λευκανθήσομαι.
\vs{10}Ἀκουτιεῖς με ἀγαλλίασιν καὶ εὐφροσύνην, ἀγαλλιάσονται ὀστᾶ τεταπεινωμένα.
\vs{11}Ἀπόστρεψον τὸ πρόσωπόν σου ἀπὸ τῶν ἁμαρτιῶν μου, καὶ πάσας τὰς ἀνομίας μου ἐξάλειψον.
\vs{12}Καρδίαν καθαρὰν κτίσον ἐν ἐμοὶ ὁ Θεὸς, καὶ πνεῦμα εὐθὲς ἐγκαίνισον ἐν τοῖς ἐγκάτοις μου.
\vs{13}Μὴ ἀποῤῥίψῃς με ἀπὸ τοῦ προσώπου σου, καὶ τὸ πνεῦμα τὸ ἅγιόν σου μὴ ἀντανέλῃς ἀπʼ ἐμοῦ.
\vs{14}Ἀπόδος μοι τὴν ἀγαλλίασιν τοῦ σωτηρίου σου, πνεύματι ἡγεμονικῷ στήριξόν με.

\vs{15}Διδάξω ἀνόμους τὰς ὁδούς σου, καὶ ἀσεβεῖς ἐπὶ σὲ ἐπιστρέψουσι.
\vs{16}Ῥῦσαί με ἐξ αἱμάτων ὁ Θεὸς, ὁ Θεὸς τῆς σωτηρίας μου, ἀγαλλιάσεται ἡ γλῶσσά μου τὴν δικαιοσύνην σου.
\vs{17}Κύριε, τὰ χείλη μου ἀνοίξεις, καὶ τὸ στόμα μου ἀναγγελεῖ τὴν αἴνεσίν σου.
\vs{18}Ὅτι εἰ ἠθέλησας θυσίαν, ἔδωκα ἄν· ὁλοκαυτώματα οὐκ εὐδοκήσεις.
\vs{19}Θυσία τῷ Θεῷ πνεῦμα συντετριμμένον, καρδίαν συντετριμμένην καὶ τεταπεινωμένην ὁ Θεὸς οὐκ ἐξουδενώσει.

\vs{20}Ἀγάθυνον, Κύριε, ἐν τῇ εὐδοκίᾳ σου τὴν Σιὼν, καὶ οἰκοδομηθήτω τὰ τείχη Ἱερουσαλήμ.
\vs{21}Τότε εὐδοκήσεις θυσίαν δικαιοσύνης, ἀναφορὰν, καὶ ὁλοκαυτώματα· τότε ἀνοίσουσιν ἐπὶ τὸ θυσιαστήριόν σου μόσχους.

\begin{psalmheading}{\ch{51}{52} Εἰς τὸ τέλος συνέσεως τῷ Δαυὶδ,}\vs{2}ἐν τῷ ἐλθεῖν Δωὴκ τὸν Ἰδουμαῖον, καὶ ἀναγγεῖλαι τῷ Σαοὺλ, καὶ εἰπεῖν αὐτῷ, ἦλθε Δαυὶδ εἰς τὸν οἶκον Ἀβιμέλεχ.
\end{psalmheading}

\vs{3}Τί ἐγκαυχᾷ ἐν κακίᾳ ὁ δυνατὸς ἀνομίαν; ὅλην τὴν ἡμέραν
\vs{4}ἀδικίαν ἐλογίσατο ἡ γλῶσσά σου· ὡσεὶ ξυρὸν ἠκονημένον ἐποίησας δόλον.
\vs{5}Ἠγάπησας κακίαν ὑπὲρ ἀγαθωσύνην, ἀδικίαν ὑπὲρ τὸ λαλῆσαι δικαιοσύνην· διάψαλμα.
\vs{6}Ἠγάπησας πάντα τὰ ῥήματα καταποντισμοῦ, γλῶσσαν δολίαν.

\vs{7}Διὰ τοῦτο ὁ Θεὸς καθέλοι σε εἰς τέλος, ἐκτίλαι σε καὶ μεταναστεύσαι σε ἀπὸ σκηνώματος, καὶ τὸ ῥίζωμά σου ἐκ γῆς ζώντων· διάψαλμα.
\vs{8}Καὶ ὄψονται δίκαιοι καὶ φοβηθήσονται, καὶ ἐπʼ αὐτὸν γελάσονται, καὶ ἐροῦσιν,
\vs{9}ἰδοὺ ἄνθρωπος ὃς οὐκ ἔθετο τὸν Θεὸν βοηθὸν αὐτοῦ, ἀλλʼ ἐπήλπισεν ἐπὶ τὸ πλῆθος τοῦ πλούτου αὐτοῦ, καὶ ἐνεδυναμώθη ἐπὶ τῇ ματαιότητι αὐτοῦ.

\vs{10}Ἐγὼ δὲ ὡσεὶ ἐλαία κατάκαρπος ἐν τῷ οἴκῳ τοῦ Θεοῦ, ἤλπισα ἐπὶ τὸ ἔλεος τοῦ Θεοῦ εἰς τὸν αἰῶνα καὶ εἰς τὸν αἰῶνα τοῦ αἰῶνος.
\vs{11}Ἐξομολογήσομαί σοι εἰς τὸν αἰῶνα, ὅτι ἐποίησας, καὶ ὑπομενῶ τὸ ὄνομά σου, ὅτι χρηστὸν ἐναντίον τῶν ὁσίων σου.

\begin{psalmheading}{\ch{52}{53} Εἰς τὸ τέλος, ὑπὲρ μαελὲθ συνέσεως τῷ Δαυίδ.}
\end{psalmheading}
\vs{2}Εἶπεν ἄφρων ἐν καρδίᾳ αὐτοῦ, οὐκ ἔστι Θεός· διέφθειραν, καὶ ἐβδελύχθησαν ἐν ἀνομίαις· οὐκ ἔστι ποιῶν ἀγαθόν.
\vs{3}Ὁ Θεὸς ἐκ τοῦ οὐρανοῦ διέκυψεν ἐπὶ τοὺς υἱοὺς τῶν ἀνθρώπων, τοῦ ἰδεῖν εἰ ἔστι συνιῶν, ἢ ἐκζητῶν τὸν Θεόν.
\vs{4}Πάντες ἐξέκλιναν, ἅμα ἠχρειώθησαν, οὐκ ἔστι ποιῶν ἀγαθὸν, οὐκ ἔστιν ἕως ἑνός.

\vs{5}Οὐχὶ γνώσονται πάντες οἱ ἐργαζόμενοι τὴν ἀνομίαν, οἱ κατεσθίοντες τὸν λαόν μου βρώσει ἄρτου; τὸν Θεὸν οὐκ ἐπεκαλέσαντο.
\vs{6}Ἐκεῖ ἐφοβήθησαν φόβον, οὗ οὐκ ἦν φόβος· ὅτι ὁ Θεὸς διεσκόρπισεν ὀστᾶ ἀνθρωπαρέσκων, κατῃσχύνθησαν, ὅτι ὁ Θεὸς ἐξουδένωσεν αὐτούς.
\vs{7}Τίς δώσει ἐκ Σιὼν τὸ σωτήριον τοῦ Ἰσραήλ; ἐν τῷ ἀποστρέψαι Κύριον τὴν αἰχμαλωσίαν τοῦ λαοῦ αὐτοῦ, ἀγαλλιάσεται Ἰακὼβ, καὶ εὐφρανθήσεται Ἰσραήλ.

\begin{psalmheading}{\ch{53}{54} Εἰς τὸ τέλος, ἐν ὕμνοις συνέσεως τῷ Δαυὶδ,}\vs{2}ἐν τῷ ἐλθεῖν τοὺς Ζειφαίους, καὶ εἰπεῖν τῷ Σαοὺλ, οὐκ ἰδοὺ Δαυὶδ κέκρυπται παρʼ ἡμῖν;
\end{psalmheading}

\vs{3}Ὁ Θεὸς ἐν τῷ ὀνόματί σου σῶσόν με, καὶ ἐν τῇ δυνάμει σου κρῖνόν με.
\vs{4}Ὁ Θεὸς εἰσάκουσον τῆς προσευχῆς μου, ἐνώτισαι τὰ ῥήματα τοῦ στόματός μου.
\vs{5}Ὅτι ἀλλότριοι ἐπανέστησαν ἐπʼ ἐμὲ, καὶ κραταιοὶ ἐζήτησαν τὴν ψυχήν μου, οὐ προέθεντο τὸν Θεὸν ἐνώπιον αὐτῶν· διάψαλμα.

\vs{6}Ἰδοὺ γὰρ ὁ Θεὸς βοηθεῖ μοι, καὶ ὁ Κύριος ἀντιλήπτωρ τῆς ψυχῆς μου.
\vs{7}Ἀποστρέψει τὰ κακὰ τοῖς ἐχθροῖς μου, ἐν τῇ ἀληθείᾳ σου ἐξολόθρευσον αὐτούς.
\vs{8}Ἑκουσίως θύσω σοι, ἐξομολογήσομαι τῷ ὀνόματί σου Κύριε, ὅτι ἀγαθόν.
\vs{9}Ὅτι ἐκ πάσης θλίψεως ἐῤῥύσω με, καὶ ἐν τοῖς ἐχθροῖς μου ἐπεῖδεν ὁ ὀφθαλμός μου.

\begin{psalmheading}{\ch{54}{55} Εἰς τὸ τέλος, ἐν ὕμνοις συνέσεως τῷ Δαυίδ.}
\end{psalmheading}
\vs{2}Ἐνωτίσαι ὁ Θεὸς τὴν προσευχήν μου, καὶ μὴ ὑπερίδῃς τὴν δέησίν μου·
\vs{3}Πρόσχες μοι, καὶ εἰσάκουσόν μου· ἐλυπήθην ἐν τῇ ἀδολεσχίᾳ μου, καὶ ἐταράχθην
\vs{4}ἀπὸ φωνῆς ἐχθροῦ, καὶ ἀπὸ θλίψεως ἁμαρτωλοῦ· ὅτι ἐξέκλιναν ἐπʼ ἐμὲ ἀνομίαν, καὶ ἐν ὀργῇ ἐνεκότουν μοι.

\vs{5}Ἡ καρδία μου ἐταράχθη ἐν ἐμοὶ, καὶ δειλία θανάτου ἐπέπεσεν ἐπʼ ἐμέ.
\vs{6}Φόβος καὶ τρόμος ἦλθεν ἐπʼ ἐμὲ καὶ ἐκάλυψέ με σκότος.
\vs{7}Καὶ εἶπα, τίς δώσει μοι πτέρυγας ὡσεὶ περιστερᾶς; καὶ πετασθήσομαι καὶ καταπαύσω.
\vs{8}Ἰδοὺ ἐμάκρυνα φυγαδεύων, καὶ ηὐλίσθην ἐν τῇ ἐρήμῳ· διάψαλμα.
\vs{9}Προσεδεχόμην τὸν σώζοντά με ἀπὸ ὀλιγοψυχίας καὶ καταιγίδος.

\vs{10}Καταπόντισον Κύριε καὶ καταδίελε τὰς γλώσσας αὐτῶν, ὅτι εἶδον ἀνομίαν καὶ ἀντιλογίαν ἐν τῇ πόλει.
\vs{11}Ἡμέρας καὶ νυκτὸς κυκλώσει αὐτὴν ἐπὶ τὰ τείχη αὐτῆς, ἀνομία καὶ πόνος ἐν μέσῳ αὐτῆς
\vs{12}καὶ ἀδικία, καὶ οὐκ ἐξέλιπεν ἐκ τῶν πλατειῶν αὐτῆς τόκος καὶ δόλος.

\vs{13}Ὅτι εἰ ἐχθρὸς ὠνείδισέ με, ὑπήνεγκα ἂν, καὶ εἰ ὁ μισῶν ἐπʼ ἐμὲ ἐμεγαλοῤῥημόνησεν, ἐκρύβην ἂν ἀπʼ αὐτοῦ.
\vs{14}Σὺ δὲ ἄνθρωπε ἰσόψυχε, ἡγεμών μου καὶ γνωστέ μου,
\vs{15}ὃς ἐπιτοαυτὸ ἐγλύκανας ἐδέσματα, ἐν τῷ οἴκῳ τοῦ Θεοῦ ἐπορεύθημεν ἐν ὁμονοίᾳ.
\vs{16}Ἐλθέτω θάνατος ἐπʼ αὐτοὺς, καὶ καταβήτωσαν εἰς ᾅδου ζῶντες· ὅτι πονηρία ἐν ταῖς παροικίαις αὐτῶν ἐν μέσῳ αὐτῶν.

\vs{17}Ἐγὼ πρὸς τὸν Θεὸν ἐκέκραξα, καὶ ὁ Κύριος εἰσήκουσέ μου.
\vs{18}Ἑσπέρας καὶ πρωῒ καὶ μεσημβρίας διηγήσομαι, καὶ ἀπαγγελῶ, καὶ εἰσακούσεται τῆς φωνῆς μου.
\vs{19}Λυτρώσεται ἐν εἰρήνῃ τὴν ψυχήν μου ἀπὸ τῶν ἐγγιζόντων μοι, ὅτι ἐν πολλοῖς ἦσαν σὺν ἐμοί.
\vs{20}Εἰσακούσεται ὁ Θεὸς καὶ ταπεινώσει αὐτοὺς, ὁ ὑπάρχων πρὸ τῶν αἰώνων· διάψαλμα·

\vs{21}Οὐ γάρ ἐστιν αὐτοῖς ἀντάλλαγμα, καὶ οὐκ ἐφοβήθησαν τὸν Θεόν. Ἐξέτεινε τὴν χεῖρα αὐτοῦ ἐν τῷ ἀποδιδόναι· ἐβεβήλωσαν τὴν διαθήκην αὐτοῦ.
\vs{22}Διεμερίσθησαν ἀπὸ ὀργῆς τοῦ προσώπου αὐτοῦ, καὶ ἤγγισεν ἡ καρδία αὐτοῦ· ἡπαλύνθησαν οἱ λόγοι αὐτοῦ ὑπὲρ ἔλαιον, καὶ αὐτοί εἰσι βολίδες.

\vs{23}Ἐπίῤῥιψον ἐπὶ Κύριον τὴν μέριμνάν σου, καὶ αὐτός σε διαθρέψει, οὐ δώσει εἰς τὸν αἰῶνα σάλον τῷ δικαίῳ.
\vs{24}Σὺ δὲ ὁ Θεὸς κατάξεις αὐτοὺς εἰς φρέαρ διαφθορᾶς· ἄνδρες αἱμάτων καὶ δολιότητος οὐ μὴ ἡμισεύσωσι τὰς ἡμέρας αὐτῶν· ἐγὼ δὲ ἐλπιῶ ἐπὶ σε, Κύριε.

\begin{psalmheading}{\ch{55}{56} Εἰς τὸ τέλος, ὑπὲρ τοῦ λαοῦ τοῦ ἀπὸ τῶν ἁγίων μεμακρυμμένου, τῷ Δαυὶδ εἰς στηλογραφίαν, ὁπότε ἐκράτησαν αὐτὸν οἱ ἀλλόφυλοι ἐν Γέθ.}
\end{psalmheading}
\vs{2}Ἐλεήσον με ὁ Θεὸς, ὅτι κατεπάτησέ με ἄνθρωπος, ὅλην τὴν ἡμέραν πολεμῶν ἔθλιψέ με.
\vs{3}Κατεπάτησάν με οἱ ἐχθροί μου ὅλην τὴν ἡμέραν ἀπὸ ὕψους ἡμέρας, ὅτι πολλοὶ οἱ πολεμοῦντές με.

\vs{4}Φοβηθήσονται, ἐγὼ δὲ ἐλπιῶ ἐπὶ σοί.
\vs{5}Ἐν τῷ Θεῷ ἐπαινέσω τοὺς λόγους μου, ὅλην τὴν ἡμέραν ἐν τῷ Θεῷ ἤλπισα, οὐ φοβηθήσομαι τί ποιήσει μοι σάρξ.

\vs{6}Ὅλην τὴν ἡμέραν τοὺς λόγους μου ἐβδελύσσοντο, κατʼ ἐμοῦ πάντες οἱ διαλογισμοὶ αὐτῶν εἰς κακόν.
\vs{7}Παροικήσουσι καὶ κατακρύψουσιν αὐτοὶ, τὴν πτέρναν μου φυλάξουσι· καθάπερ ὑπέμεινα τῇ ψυχῇ μου.
\vs{8}Ὑπὲρ τοῦ μηθενὸς σώσεις αὐτοὺς ἐν ὀργῇ λαοὺς κατάξεις· ὁ Θεὸς
\vs{9}τὴν ζωήν μου ἐξήγγειλά σοι, ἔθου τὰ δάκρυά μου ἐνώπιόν σου, ὡς καὶ ἐν τῇ ἐπαγγελίᾳ σου.

\vs{10}Ἐπιστρέψουσιν οἱ ἐχθροί μου εἰς τὰ ὀπίσω, ἐν ᾗ ἂν ἡμέρᾳ ἐπικαλέσωμαί σε· ἰδοὺ ἔγνων ὅτι Θεός μου εἶ σύ.
\vs{11}Ἐπὶ τῷ Θεῷ αἰνέσω ῥῆμα, ἐπὶ τῷ Κυρίῳ αἰνέσω λόγον.
\vs{12}Ἐπὶ τῷ Θεῷ ἤλπισα, οὐ φοβηθήσομαι τί ποιήσει μοι ἄνθρωπος.
\vs{13}Ἐν ἐμοὶ ὁ Θεὸς αἱ εὐχαὶ, ἃς ἀποδώσω αἰνέσεώς σου.
\vs{14}Ὅτι ἐῤῥύσω τὴν ψυχήν μου ἐκ θανάτου, καὶ τοὺς πόδας μου ἐξ ὀλισθήματος, τοῦ εὐαρεστῆσαι ἐνώπιον τοῦ Θεοῦ ἐν φωτὶ ζώντων.

\begin{psalmheading}{\ch{56}{57} Εἰς τὸ τέλος, μὴ διαφθείρῃς, τῷ Δαυὶδ εἰς στηλογραφίαν, ἐν τῷ αὐτὸν ἀποδιδράσκειν ἀπὸ προσώπου Σαοὺλ εἰς τὸ σπήλαιον.}
\end{psalmheading}
\vs{2}Ἐλέησον με ὁ Θεὸς, ἐλέησόν με, ὅτι ἐπὶ σοὶ πέποιθεν ἡ ψυχή μου, καὶ ἐν τῇ σκιᾷ τῶν πτερύγων σου ἐλπιῶ, ἕως οὗ παρέλθῃ ἡ ἀνομία.
\vs{3}Κεκράξομαι πρὸς τὸν Θεὸν τὸν ὕψιστον τὸν Θεὸν τὸν εὐεργετήσαντά με· διάψαλμα.
\vs{4}Ἐξαπέστειλεν ἐξ οὐρανοῦ καὶ ἔσωσέ με, ἔδωκεν εἰς ὄνειδος τοὺς καταπατοῦντάς με· ἐξαπέστειλεν ὁ Θεὸς τὸ ἔλεος αὐτοῦ καὶ τὴν ἀλήθειαν αὐτοῦ,
\vs{5}καὶ ἐῤῥύσατο τὴν ψυχήν μου ἐκ μέσου σκύμνων· ἐκοιμήθην τεταραγμένος· υἱοὶ ἀνθρώπων, οἱ ὀδόντες αὐτῶν, ὅπλον καὶ βέλη, καὶ ἡ γλῶσσα αὐτῶν, μάχαιρα ὀξεῖα.

\vs{6}Ὑψώθητι ἐπὶ τοὺς οὐρανοὺς ὁ Θεὸς, καὶ ἐπὶ πᾶσαν τὴν γῆν ἡ δόξα σου.
\vs{7}Παγίδας ἡτοίμασαν τοῖς ποσί μου, καὶ κατέκαμψαν τὴν ψυχήν μου· ὤρυξαν πρὸ προσώπου μου βόθρον, καὶ ἐνέπεσαν εἰς αὐτόν· διάψαλμα.
\vs{8}Ἑτοίμη ἡ καρδία μου ὁ Θεὸς, ἑτοίμη ἡ καρδία μου, ᾄσομαι καὶ ψαλῶ.
\vs{9}Ἐξεγέρθητι ἡ δόξα μου, ἐξεγέρθητι ψαλτήριον καὶ κιθάρα, ἐξεγερθήσομαι ὄρθρου.
\vs{10}Ἐξομολογήσομαί σοι ἐν λαοῖς Κύριε, ψαλῶ σοι ἐν ἔθνεσιν.
\vs{11}Ὅτι ἐμεγαλύνθη ἕως τῶν οὐρανῶν τὸ ἔλεός σου, καὶ ἕως τῶν νεφελῶν ἡ ἀλήθειά σου.
\vs{12}Ὑψώθητι ἐπὶ τοὺς οὐρανοὺς ὁ Θεὸς, καὶ ἐπὶ πᾶσαν τὴν γῆν ἡ δόξα σου.

\begin{psalmheading}{\ch{57}{58} Εἰς τὸ τέλος, μὴ διαφθείρῃς, τῷ Δαυὶδ εἰς στηλογραφίαν.}
\end{psalmheading}
\vs{2}Εἰ ἀληθῶς ἄρα δικαιοσύνην λαλεῖτε, εὐθεῖα κρίνετε οἱ υἱοὶ τῶν ἀνθρώπων.
\vs{3}Καὶ γὰρ ἐν καρδίᾳ ἀνομίας ἐργάζεσθε ἐν τῇ γῇ, ἀδικίαν αἱ χεῖρες ὑμῶν συμπλέκουσιν.
\vs{4}Ἀπηλλοτριώθησαν οἱ ἁμαρτωλοὶ ἀπὸ μήτρας, ἐπλανήθησαν ἀπὸ γαστρὸς, ἐλάλησαν ψευδῆ.
\vs{5}Θυμὸς αὐτοῖς κατὰ τὴν ὁμοίωσιν τοῦ ὄφεως, ὡσεὶ ἀσπίδος κωφῆς, καὶ βυούσης τὰ ὦτα αὐτῆς,
\vs{6}ἥτις οὐκ εἰσακούσεται φωνὴν ἐπᾳδόντων, φαρμάκου τε φαρμακευομένου παρὰ σοφοῦ.

\vs{7}Ὁ Θεὸς συνέτριψε τοὺς ὀδόντας αὐτῶν ἐν τῷ στόματι αὐτῶν, τὰς μύλας τῶν λεόντων συνέθλασεν ὁ Κύριος.
\vs{8}Ἐξουδενωθήσονται ὡς ὕδωρ διαπορευόμενον, ἐντενεῖ τὸ τόξον αὐτοῦ ἕως οὗ ἀσθενήσουσιν.
\vs{9}Ὡσεὶ κηρὸς ὁ τακεῖς ἀνταναιρεθήσονται, ἔπεσε πῦρ, καὶ οὐκ εἶδον τὸν ἥλιον.
\vs{10}Πρὸ τοῦ συνιέναι τὰς ἀκάνθας ὑμῶν τὴν ῥάμνον, ὡσεὶ ζῶντας ὡσεὶ ἐν ὀργῇ καταπίεται ὑμᾶς.

\vs{11}Εὐφρανθήσεται δίκαιος, ὅταν ἴδῃ ἐκδίκησιν ἀσεβῶν, τὰς χεῖρας αὐτοῦ νίψεται ἐν τῷ αἵματι τοῦ ἁμαρτωλοῦ.
\vs{12}Καὶ ἐρεῖ ἄνθρωπος, εἰ ἄρα ἐστὶ καρπὸς τῷ δικαίῳ, ἄρα ἐστὶν ὁ Θεὸς κρίνων αὐτοὺς ἐν τῇ γῇ.

\begin{psalmheading}{\ch{58}{59} Εἰς τὸ τέλος, μὴ διαφθείρῃς, τῷ Δαυὶδ εἰς στηλογραφίαν, ὁπότε ἀπέστειλε Σαοὺλ, καὶ ἐφύλαξε τὸν οἶκον αὐτοῦ τοῦ θανατῶσαι αὐτόν.}
\end{psalmheading}
\vs{2}Ἐξελοῦ με ἐκ τῶν ἐχθρῶν μου ὁ Θεὸς, καὶ ἐκ τῶν ἐπανισταμένων ἐπʼ ἐμὲ λύτρωσαί με.
\vs{3}Ῥῦσαί με ἐκ τῶν ἐργαζομένων τὴν ἀνομίαν, καὶ ἐξ ἀνδρῶν αἱμάτων σῶσόν με.

\vs{4}Ὅτι ἰδοὺ ἐθήρευσαν τὴν ψυχήν μου, ἐπέθεντο ἐπʼ ἐμὲ κραταιοί· οὔτε ἡ ἀνομία μου, οὔτε ἡ ἁμαρτία μου Κύριε·
\vs{5}Ανευ ἀνομίας ἔδραμον καὶ κατεύθυνα· ἐξεγέρθητι εἰς συνάντησίν μου, καὶ ἴδε.
\vs{6}Καὶ σὺ Κύριε ὁ Θεὸς τῶν δυνάμεων ὁ Θεὸς τοῦ Ἰσραὴλ, πρόσχες τοῦ ἐπισκέψασθαι πάντα τὰ ἔθνη, μὴ οἰκτειρήσῃς πάντας τοὺς ἐργαζομένους τὴν ἀνομίαν· διάψαλμα.
\vs{7}Ἐπιστρέψουσιν εἰς ἑσπέραν, καὶ λιμώξουσιν ὡς κύων, καὶ κυκλώσουσιν πόλιν.

\vs{8}Ἰδοὺ ἀποφθέγξονται ἐν τῷ στόματι αὐτῶν, καὶ ῥομφαία ἐν τοῖς χείλεσιν αὐτῶν, ὅτι τίς ἤκουσε;
\vs{9}Καὶ δὺ Κύριε ἐκγελάσῃ αὐτοὺς, ἐξουδενώσεις τάντα τὰ ἔθνη.
\vs{10}Τὸ κράτος μου πρὸς σὲ φυλάξω, ὅτι σὺ ὁ Θεὸς ἀντιλήμπτωρ μου εἶ.
\vs{11}Ὁ Θεός μου, τὸ ἔλεος αὐτοῦ προφθάσει με, ὁ Θεός μου δείξει μοι ἐν τοῖς ἐχθροῖς μου.

\vs{12}Μὴ ἀποκτείνῃς αὐτοὺς, μήποτε ἐπιλάθωνται τοῦ νόμου σου· διασκόρπισον αὐτοὺς ἐν τῇ δυνάμει σου, καὶ κατάγαγε αὐτοὺς ὁ ὑπερασπιστής μου Κύριε.
\vs{13}Ἁμαρτίαν στόματος αὐτῶν, λόγον χειλέων αὐτῶν, καὶ συλληφθήτωσαν ἐν τῇ ὑπερηφανίᾳ αὐτῶν· καὶ ἐξ ἀρᾶς καὶ ψεύδους διαγγελήσονται
\vs{14}συντέλειαι, ἐν ὀργῇ συντελείας, καὶ οὐ μὴ ὑπάρξουσι· καὶ γνώσονται ὅτι ὁ Θεὸς τοῦ Ἰακὼβ δεσπόζει τῶν περάτων τῆς γῆς· διάψαλμα.
\vs{15}Ἐπιστρέψουσιν εἰς ἑσπέραν, καὶ λιμώξουσιν ὡς κύων, καὶ κυκλώσουσι πόλιν·
\vs{16}Αὐτοὶ διασκορπισθήσονται τοῦ φαγεῖν, ἐὰν δὲ μὴ χορτασθῶσι, καὶ γογγύσουσιν.

\vs{17}Ἐγὼ δὲ ᾄσομαι τῇ δυνάμει σου, καὶ ἀγαλλιάσομαι τοπρωῒ τὸ ἔλεός σου, ὅτι ἐγενήθης ἀντιλήπτωρ μου καὶ καταφυγή μου ἐν ἡμέρᾳ θλίψεώς μου.
\vs{18}Βοηθός μου, σοὶ ψαλῶ ὁ Θεός μου, ἀντιλήμπτωρ μου εἶ ὁ Θεός μου, τὸ ἔλεός μου.

\begin{psalmheading}{\ch{59}{60} Εἰς τὸ τέλος, τοῖς ἀλλοιωθησομένοις ἔτι, εἰς στηλογραφίαν τῷ Δαυὶδ εἰς διδαχὴν, ὁπότε ἐνεπύρισε τὴν Μεσοποταμίαν Συρίας, καὶ τὴν Συρίαν Σοβὰλ, καὶ ἐπέστρεψεν Ἰωὰβ, καὶ ἐπάταξε τὴν φάραγγα τῶν ἁλῶν, δώδεκα χιλιάδας.}
\end{psalmheading}
\vs{3}Ὁ Θεὸς ἀπώσω ἡμᾶς καὶ καθεῖλες ἡμᾶς, ὠργίσθης καὶ ᾠκτείρησας ἡμᾶς.
\vs{4}Συνέσεισας τὴν γῆν καὶ συνετάραξας αὐτὴν, ἴασαι τὰ συντρίμματα αὐτῆς, ὅτι ἐσαλεύθη.
\vs{5}Ἔδειξας τῷ λαῷ σου σκληρὰ, ἐπότισας ἡμᾶς οἶνον κατανύξεως.
\vs{6}Ἔδωκας τοῖς φοβουμένοις σε σημείωσιν, τοῦ φυγεῖν ἀπὸ προσώπου τόξον· διάψαλμα.
\vs{7}Ὅπως ἂν ῥυσθῶσιν οἱ ἀγαπητοί σου, σῶσον τῇ δεξιᾷ σου καὶ ἐπάκουσόν μου.

\vs{8}Ὁ Θεὸς ἐλάλησεν ἐν τῷ ἁγίῳ αὐτοῦ, ἀγαλλιάσομαι καὶ διαμεριῶ Σίκιμα, καὶ τὴν κοιλάδα τῶν σκηνῶν διαμετρήσω.
\vs{9}Ἐμός ἐστι Γαλαὰδ, καὶ ἐμός ἐστι Μανασσῆ, καὶ Ἐφραὶμ κραταίωσις τῆς κεφαλῆς μου· Ἰούδας βασιλεύς μου,
\vs{10}Μωὰβ λέβης τῆς ἐλπίδος μου, ἐπὶ τὴν Ἰδουμαίαν ἐκτενῶ τὸ ὑπόδημά μου, ἐμοὶ ἀλλόφυλοι ὑπετάγησαν.

\vs{11}Τίς ἀπάξει με εἰς πόλιν περιοχῆς; τίς ὁδηγήσει με ἕως τῆς Ἰδουμαίας;
\vs{12}Οὐχὶ σὺ ὁ Θεὸς, ὁ ἀπωσάμενος ἡμᾶς; καὶ οὐκ ἐξελεύσῃ, ὁ Θεὸς ἐν ταῖς δυνάμεσιν ἡμῶν;
\vs{13}Δὸς ἡμῖν βοήθειαν ἐκ θλίψεως, καὶ ματαία σωτηρία ἀνθρώπου.

\vs{14}Ἐν τῷ Θεῷ ποιήσομεν δύναμιν, καὶ αὐτὸς ἐξουδενώσει τοὺς θλίβοντας ἡμᾶς.

\begin{psalmheading}{\ch{60}{61} Εἰς τὸ τέλος, ἐν ὕμνοις τῷ Δαυίδ.}
\end{psalmheading}
\vs{2}Εἰσάκουσον ὁ Θεὸς τῆς δεήσεώς μου, πρόσχες τῇ προσευχῇ μου.
\vs{3}Ἀπὸ τῶν περάτων τῆς γῆς πρὸς σὲ ἐκέκραξα, ἐν τῷ ἀκηδιάσαι τὴν καρδίαν μου, ἐν πέτρᾳ ὕψωσάς με, ὁδήγησάς με,
\vs{4}ὅτι ἐγενήθης ἐλπίς μου, πύργος ἰσχύος ἀπὸ προσώπου ἐχθροῦ.
\vs{5}Παροικήσω ἐν τῷ σκηνώματί σου εἰς τοὺς αἰῶνας, σκεπασθήσομαι ἐν σκέπῃ τῶν πτερύγων σου· διάψαλμα.

\vs{6}Ὅτι σὺ ὁ Θεὸς εἰσήκουσας τῶν προσευχῶν μου, ἔδωκας κληρονομίαν τοῖς φοβουμένοις τὸ ὄνομά σου.
\vs{7}Ἡμέρας ἐφʼ ἡμέρας βασιλέως προσθήσεις, τὰ ἔτη αὐτοῦ ἕως ἡμέρας γενεᾶς καὶ γενεᾶς.
\vs{8}Διαμενεῖ εἰς τὸν αἰῶνα ἐνώπιον τοῦ Θεοῦ, ἔλεος καὶ ἀλήθειαν αὐτοῦ τίς ἐκζητήσει αὐτῶν;
\vs{9}Οὕτως ψαγῶ τῷ ὀνόματί σου εἰς τὸν αἰῶνα τοῦ αἰῶνος, τοῦ ἀποδοῦναί με τὰς εὐχάς μου ἡμέραν ἐξ ἡμέρας.

\begin{psalmheading}{\ch{61}{62} Εἰς τὸ τέλος, ὑπὲρ Ἰδιθοὺν ψαλμὸς τῷ Δαυίδ.}
\end{psalmheading}
\vs{2}Οὐχὶ τῷ Θεῷ ὑποταγήσεται ἡ ψυχή μου; παρʼ αὐτοῦ γὰρ τὸ σωτήριόν μου.
\vs{3}Καὶ γὰρ αὐτὸς Θεός μου καὶ σωτήρ μου, ἀντιλήπτωρ μου, οὐ μὴ σαλευθῶ ἐπὶ πλεῖον.
\vs{4}Ἕως πότε ἐπιτίθεσθε ἐπʼ ἄνθρωπον; φονεύετε πάντες ὡς τοίχῳ κεκλιμένῳ καὶ φραγμῷ ὠσμένῳ.
\vs{5}Πλὴν τὴν τιμήν μου ἐβουλεύσαντο ἀπώσασθαι· ἔδραμον ἐν δίψει· τῷ στόματι αὐτῶν εὐλόγουν, καὶ τῇ καρδίᾳ αὐτῶν κατηρῶντο. διάψαλμα.

\vs{6}Πλὴν τῷ Θεῷ ὑποτάγηθι, ἡ ψυχή μου, ὅτι παρʼ αὐτοῦ ἡ ὑπομονή μου.
\vs{7}Ὅτι αὐτὸς Θεός μου καὶ σωτήρ μου, ἀντιλήπτωρ μου, οὐ μὴ μεταναστεύσω.
\vs{8}Ἐπὶ τῷ Θεῷ τὸ σωτήριόν μου, καὶ ἡ δόξα μου· ὁ Θεὸς τῆς βοηθείας μου, καὶ ἡ ἐλπίς μου ἐπὶ τῷ Θεῷ.
\vs{9}Ἐλπίσατε ἐπʼ αὐτὸν πᾶσα συναγωγὴ λαοῦ· ἐκχέετε ἐνώπιον αὐτοῦ τὰς καρδίας ὑμῶν, ὅτι ὁ Θεὸς βοηθὸς ἡμῶν· διάψαλμα.

\vs{10}Πλὴν μάταιοι οἱ νἱοὶ τῶν ἀνθρώπων, ψευδεῖς οἱ υἱοὶ τῶν ἀνθρώπων ἐν ζυγοῖς τοῦ ἀδικῆσαι, αὐτοὶ ἐκ ματαιότητος ἐπιτοαυτό.
\vs{11}Μὴ ἐλπίζετε ἐπʼ ἀδικίαν, καὶ ἐπὶ ἁρπάγματα μὴ ἐπιποθεῖτε· πλοῦτος ἐὰν ῥέῃ, μὴ προστίθεσθε καρδίαν.
\vs{12}Ἅπαξ ἐλάλησεν ὁ Θεὸς, δύο ταῦτα ἤκουσα, ὅτι τὸ κράτος τοῦ Θεοῦ·
\vs{13}καὶ σοῦ, Κύριε τὸ ἔλεος, ὅτι σὺ ἀποδώσεις ἑκάστῳ κατὰ τὰ ἔργα αὐτοῦ.

\begin{psalmheading}{\ch{62}{63} Ψαλμὸς τῷ Δαυὶδ, ἐν τῷ εἶναι αὐτὸν ἐν τῇ ἐρήμῳ τῆς Ἰδουμαίας.}
\end{psalmheading}
\vs{2}Ὁ Θεὸς ὁ Θεός μου πρὸς σὲ ὀρθρίζω, ἐδίψησέ σοι ἡ ψυχή μου, ποσαπλῶς σοι ἡ σάρξ μου, ἐν γῇ ἐρήμῳ καὶ ἀβάτῳ καὶ ἀνύδρῳ,
\vs{3}οὕτως ἐν τῷ ἁγίῳ ὤφθην σοι, τοῦ ἰδεῖν τὴν δύναμίν σου καὶ τὴν δόξαν σου.
\vs{4}Ὅτι κρεῖσσον τὸ ἔλεός σου ὑπὲρ ζωὰς, τὰ χείλη μου ἐπαινέσουσί σε.
\vs{5}Οὕτως εὐλογήσω σε ἐν τῇ ζωῇ μου, ἐν τῷ ὀνόματί σου ἀρῶ τὰς χεῖράς μου.
\vs{6}Ὡσεὶ στέατος καὶ πιότητος ἐμπλησθείη ἡ ψυχή μου, καὶ χείλη ἀγαλλιάσεως αἰνέσει τὸ ὄνομά σου.

\vs{7}Εἰ ἐμνημόνευόν σου ἐπὶ τῆς στρωμνῆς μου, ἐν τοῖς ὄρθροις ἐμελέτων εἰς σέ.
\vs{8}Ὅτι ἐγενήθης βοηθός μου, καὶ ἐν τῇ σκέπῃ τῶν πτερύγων σου ἀγαλλιάσομαι.
\vs{9}Ἐκολλήθη ἡ ψυχή μου ὀπίσω σου, ἐμοῦ ἀντελάβετο ἡ δεξιά σου.
\vs{10}Αὐτοὶ δὲ εἰς μάτην ἐζήτησαν τὴν ψυχήν μου, εἰσελεύσονται εἰς τὰ κατώτατα τῆς γῆς,
\vs{11}παραδοθήσονται εἰς χεῖρας ῥομφαίας, μερίδες ἀλωπέκων ἔσονται.
\vs{12}Ὁ δὲ βασιλεὺς εὐφρανθήσεται ἐπὶ τῷ Θεῷ, ἐπαινεθήσεται πᾶς ὁ ὀμνύων ἐν αὐτῷ, ὅτι ἐνεφράγη στόμα λαλούντων ἄδικα.

\begin{psalmheading}{\ch{63}{64} Εἰς τὸ τέλος, ψαλμὸς τῷ Δαυίδ.}
\end{psalmheading}
\vs{2}Εἰσάκουσον ὁ Θεὸς τῆς προσευχῆς μου ἐν τῷ δέεσθαί με πρὸς σὲ, ἀπὸ φόβου ἐχθροῦ ἐξελοῦ τὴν ψυχήν μου.
\vs{3}Ἐσκέπασάς με ἀπὸ συστροφῆς πονηρευομένων, ἀπὸ πλήθους ἐργαζομένων ἀδικίαν·
\vs{4}Οἵτινες ἠκόνησαν ὡς ῥομφαίαν τὰς γλώσσας αὐτῶν, ἐνέτειναν τόξον πρᾶγμα πικρὸν,
\vs{5}τοῦ κατατοξεῦσαι ἐν ἀποκρύφοις ἄμωμον, ἐξάπινα κατατοξεύσουσιν αὐτὸν, καὶ οὐ φοβηθήσονται.
\vs{6}Ἐκραταίωσαν ἑαυτοῖς λόγον πονηρὸν, διηγήσαντο τοῦ κρύψαι παγίδας· εἶπαν, τίς ὄψεται αὐτούς;
\vs{7}Ἐξηρεύνησαν ἀνομίαν, ἐξέλιπον ἐξερευνῶντες ἐξερευνήσει· προσελεύσεται ἄνθρωπος, καὶ καρδία βαθεῖα,
\vs{8}καὶ ὑψωθήσεται ὁ Θεός· βέλος νηπίων ἐγενήθησαν αἱ πληγαὶ αὐτῶν,
\vs{9}καὶ ἐξουθένησαν αὐτὸν αἱ γλῶσσαι αὐτῶν· ἐταράχθησαν πάντες οἱ θεωροῦντες αὐτοὺς,
\vs{10}καὶ ἐφοβήθη πᾶς ἄνθρωπος· καὶ ἀνήγγειλαν τὰ ἔργα τοῦ Θεοῦ, καὶ τὰ ποιήματα αὐτοῦ συνῆκαν.
\vs{11}Εὐφρανθήσεται δίκαιος ἐν τῷ Κυρίῳ, καὶ ἐλπιεῖ ἐπʼ αὐτόν· καὶ ἐπαινεθήσονται πάντες οἱ εὐθεῖς τῇ καρδίᾳ.

\begin{psalmheading}{\ch{64}{65} Εἰς τὸ τέλος, ψαλμὸς τῷ Δαυίδ, ᾠδή.}
\end{psalmheading}
\vs{2}Σοὶ πρέπει ὕμνος, ὁ Θεὸς ἐν Σιὼν, καὶ σοὶ ἀποδοθήσεται εὐχή.
\vs{3}Εἰσάκουσον προσευχῆς μου, πρὸς σὲ πᾶσα σὰρξ ἥξει.
\vs{4}Λόγοι ἀνόμων ὑπερδυνάμωσαν ἡμᾶς, καὶ τὰς ἀσεβείας ἡμῶν σὺ ἱλάσῃ.
\vs{5}Μακάριος, ὃν ἐξελέξω καὶ προσελάβου, κατασκηνώσει ἐν ταῖς αὐλαῖς σου· πλησθησόμεθα ἐν τοῖς ἀγαθοῖς τοῦ οἴκου σου, ἅγιος ὁ ναός σου,
\vs{6}θαυμαστὸς ἐν δικαιοσύνῃ. ἐπάκουσον ἡμῶν ὁ Θεὸς ὁ σωτὴρ ἡμῶν, ἡ ἐλπὶς πάντων τῶν περάτων τῆς γῆς, καὶ τῶν ἐν θαλάσσῃ μακράν·
\vs{7}ἑτοιμάζων ὄρη ἐν τῇ ἰσχύϊ σου, περιεζωσμένος ἐν δυναστείᾳ·
\vs{8}Ὁ συνταράσσων τὸ κῦτος τῆς θαλάσσης, ἤχους κυμάτων αὐτῆς. Ταραχθήσονται τὰ ἔθνη,
\vs{9}καὶ φοβηθήσονται οἱ κατοικοῦντες τὰ πέρατα ἀπὸ τῶν σημείων σου· ἐξόδους πρωΐας καὶ ἑσπέρας τέρψεις.

\vs{10}Ἐπεσκέψω τὴν γῆν καὶ ἐμέθυσας αὐτὴν, ἐπλήθυνας τοῦ πλουτίσαι αὐτήν· ὁ ποταμὸς τοῦ Θεοῦ ἐπληρώθη ὑδάτων· ἡτοίμασας τὴν τροφὴν αὐτῶν, ὅτι οὕτως ἡ ἑτοιμασία.
\vs{11}Τοὺς αὔλακας αὐτῆς μέθυσον, πλήθυνον τὰ γεννήματα αὐτῆς, ἐν ταῖς σταγόσιν αὐτῆς εὐφρανθήσεται ἀνατέλλουσα.
\vs{12}Εὐλογήσεις τὸν στέφανον τοῦ ἐνιαυτοῦ τῆς χρηστότητός σου, καὶ τὰ πεδία σου πλησθήσονται πιότητος.
\vs{13}Πιανθήσεται τὰ ὄρη τῆς ἐρήμου, καὶ ἀγαλλίασιν οἱ βουνοὶ περιζώσονται.
\vs{14}Ἐνεδύσαντο οἱ κριοὶ τῶν προβάτων, καὶ αἱ κοιλάδες πληθυνοῦσι σῖτον, κεκράξονται, καὶ γὰρ ὑμνήσουσιν.

\begin{psalmheading}{\ch{65}{66} Εἰς τὸ τέλος, ᾠδὴ ψαλμοῦ ἀναστάσεως.}
\end{psalmheading}
Ἀλαλάξατε τῷ Θεῷ, πᾶσα ἡ γῆ,
\vs{2}ψάλατε δὴ τῷ ὀνόματι αὐτοῦ, δότε δόξαν αἰνέσει αὐτοῦ.
\vs{3}Εἴπατε τῷ Θεῷ ὡς φοβερὰ τὰ ἔργα σου; ἐν τῷ πλήθει τῆς δυνάμεώς σου ψεύσονταί σε οἱ ἐχθροί σου.
\vs{4}Πᾶσα ἡ γῆ προσκυνησάτωσάν σοι, καὶ ψαλάτωσάν σοι, ψαλάτωσαν τῷ ὀνόματί σου· διάψαλμα.

\vs{5}Δεῦτε καὶ ἴδετε τὰ ἔργα τοῦ Θεοῦ, φοβερὸς ἐν βουλαῖς ὑπὲρ τοὺς υἱοὺς τῶν ἀνθρώπων.
\vs{6}Ὁ μεταστρέφων τὴν θάλασσαν εἰς ξηρὰν, ἐν ποταμῷ διελεύσονται ποδί· ἐκεῖ εὐφρανθησόμεθα ἐπʼ αὐτῷ,
\vs{7}τῷ δεσπόζοντι ἐν τῇ δυναστείᾳ αὐτοῦ τοῦ αἰῶνος· οἱ ὀφθαλμοὶ αὐτοῦ ἐπὶ τὰ ἔθνη ἐπιβλέπουσιν, οἱ παραπικραίνοντες μὴ ὑψούσθωσαν ἐν ἑαυτοῖς· διάψαλμα.

\vs{8}Εὐλογεῖτε ἔθνη τὸν Θεὸν ἡμῶν, καὶ ἀκουτίσατε τὴν φωνὴν τῆς αἰνέσεως αὐτοῦ,
\vs{9}τοῦ θεμένου τὴν ψυχήν μου εἰς ζωὴν, καὶ μὴ δόντος εἰς σάλον τοὺς πόδας μου.
\vs{10}Ὅτι ἐδοκίμασας ἡμᾶς ὁ Θεὸς, ἐπύρωσας ἡμᾶς ὡς πυροῦται τὸ ἀργύριον.
\vs{11}Εἰσήγαγες ἡμᾶς εἰς τὴν παγίδα, ἔθου θλίψεις ἐπὶ τὸν νῶτον ἡμῶν,
\vs{12}ἐπεβίβασας ἀνθρώπους ἐπὶ τὰς κεφαλὰς ἡμῶν· διήλθομεν διὰ πυρὸς καὶ ὕδατος, καὶ ἐξήγαγες ἡμᾶς εἰς ἀναψυχήν.

\vs{13}Εἰσελεύσομαι εἰς τὸν οἶκόν σου ἐν ὁλοκαυτώμασιν, ἀποδώσω σοι τὰς εὐχάς μου,
\vs{14}ἃς διέστειλε τὰ χείλη μου, καὶ ἐλάλησε τὸ στόμα μου ἐν τῇ θλίψει μου.
\vs{15}Ὁλοκαυτώματα μεμυελωμένα ἀνοίσω σοι μετὰ θυμιάματος καὶ κριῶν, ποιήσω σοι βόας μετὰ χιμάρων· διάψαλμα.

\vs{16}Δεῦτε ἀκούσατε, καὶ διηγήσομαι, πάντες οἱ φοβούμενοι τὸν Θεὸν, ὅσα ἐποίησε τῇ ψυχῇ μου.
\vs{17}Πρὸς αὐτὸν τῷ στόματί μου ἐκέκραξα, καὶ ὕψωσα ὑπὸ τὴν γλῶσσάν μου.
\vs{18}Ἀδικίαν εἰ ἐθεώρουν ἐν καρδίᾳ μου, μὴ εἰσακουσάτω Κύριος.
\vs{19}Διὰ τοῦτο εἰσήκουσέ μου ὁ Θεὸς, προσέσχε τῇ φωνῇ τῆς προσευχῆς μου.
\vs{20}Εὐλογητὸς ὁ Θεὸς, ὃς οὐκ ἀπέστησε τὴν προσευχήν μου, καὶ τὸ ἔλεος αὐτοῦ ἀπʼ ἐμοῦ.

\begin{psalmheading}{\ch{66}{67} Εἰς τὸ τέλος, ἐν ὕμνοις ψαλμὸς τῷ Δαυίδ.}
\end{psalmheading}
\vs{2}Ὁ Θεὸς οἰκτειρήσαι ἡμᾶς, καὶ εὐλογήσαι ἡμᾶς, ἐπιφάναι τὸ πρόσωπον αὐτοῦ ἐφʼ ἡμᾶς· διάψαλμα.
\vs{3}Τοῦ γνῶναι ἐν τῇ γῇ τὴν ὁδόν σου, ἐν πᾶσιν ἔθνεσι τὸ σωτήριόν σου.
\vs{4}Ἐξομολογησάσθωσάν σοι λαοὶ ὁ Θεὸς, ἐξομολογησάσθωσάν σοι λαοὶ πάντες.
\vs{5}Εὐφρανθήτωσαν καὶ ἀγαλλιάσθωσαν ἔθνη, ὅτι κρινεῖς λαοὺς ἐν εὐθύτητι, καὶ ἔθνη ἐν τῇ γῇ ὁδηγήσεις· διάψαλμα.
\vs{6}Ἐξομολογησάσθωσάν σοι λαοὶ, ὁ Θεὸς, ἐξομολογησάσθωσάν σοι λαοὶ πάντες.
\vs{7}Γῆ ἔδωκε τὸν καρπὸν αὐτῆς· εὐλογήσαι ἡμᾶς ὁ Θεὸς, ὁ Θεὸς ἡμῶν,
\vs{8}εὐλογήσαι ἡμᾶς ὁ Θεὸς, καὶ φοβηθήτωσαν αὐτὸν πάντα τὰ πέρατα τῆς γῆς.

\begin{psalmheading}{\ch{67}{68} Εἰς τὸ τέλος, τῷ Δαυὶδ ψαλμὸς ᾠδῆς.}
\end{psalmheading}
\vs{2}Αναστήτω ὁ Θεὸς, καὶ διασκορπισθήτωσαν οἱ ἐχθροὶ αὐτοῦ, καὶ φυγέτωσαν οἱ μισοῦντες αὐτὸν ἀπὸ προσώπου αὐτοῦ.
\vs{3}Ὡς ἐκλείπει καπνὸς, ἐκλιπέτωσαν· ὡς τήκεται κηρὸς ἀπὸ προσώπου πυρὸς, οὕτως ἀπόλοιντο οἱ ἁμαρτωλοὶ ἀπὸ προσώπου τοῦ Θεοῦ.
\vs{4}Καὶ οἱ δίκαιοι εὐφρανθήτωσαν· ἀγαλλιάσθωσαν ἐνώπιον τοῦ Θεοῦ, τερφθήτωσαν ἐν εὐφροσύνῃ.

\vs{5}Ἄσατε τῷ Θεῷ, ψάλατε τῷ ὀνόματι αὐτοῦ, ὁδοποιήσατε τῷ ἐπιβεβηκότι ἐπὶ δυσμῶν, Κύριος ὄνομα αὐτῷ, καὶ ἀγαλλιᾶσθε ἐνώπιον αὐτοῦ.
\vs{6}ταραχθήσονται ἀπὸ προσώπου αὐτοῦ, τοῦ πατρὸς τῶν ὀρφανῶν, καὶ κριτοῦ τῶν χηρῶν, ὁ Θεὸς ἐν τόπῳ ἁγίῳ αὐτοῦ.
\vs{7}Ὁ Θεὸς κατοικίζει μονοτρόπους ἐν οἴκῳ, ἐξάγων πεπεδημένους ἐν ἀνδρείᾳ· ὁμοίως τοὺς παραπικραίνοντας, τοὺς κατοικοῦντας ἐν τάφοις.

\vs{8}Ὁ Θεὸς, ἐν τῷ ἐκπορεύεσθαί σε ἐνώπιον τοῦ λαοῦ σου, ἐν τῷ διαβαίνειν σε τὴν ἔρημον· διάψαλμα·
\vs{9}Γῆ ἐσείσθη, καὶ γὰρ οἱ οὐρανοὶ ἔσταξαν ἀπὸ προσώπου τοῦ Θεοῦ τοῦ Σινὰ, ἀπὸ προσώπου τοῦ Θεοῦ Ἰσραήλ.
\vs{10}Βροχὴν ἑκούσιον ἀφοριεῖς ὁ Θεὸς τῇ κληρονομίᾳ σου· καὶ ἠσθένησε, σὺ δὲ κατηρτίσω αὐτήν.

\vs{11}Τὰ ζῶά σου κατοικοῦσιν ἐν αὐτῇ, ἡτοίμασας ἐν τῇ χρηστότητί σου τῷ πτωχῷ.
\vs{12}Ὁ Θεὸς Κύριος δώσει ῥῆμα τοῖς εὐαγγελιζομένοις δυνάμει πολλῇ,
\vs{13}ὁ βασιλεὺς τῶν δυνάμεων τοῦ ἀγαπητοῦ, τοῦ ἀγαπητοῦ, καὶ ὡραιότητι τοῦ οἴκου διελέσθαι σκῦλα.
\vs{14}Ἐὰν κοιμηθῆτε ἀναμέσον τῶν κλήρων, πτέρυγες περιστερᾶς περιηργυρωμέναι, καὶ τὰ μετάφρενα αὐτῆς ἐν χλωρότητι χρυσίου.
\vs{15}Ἐν τῷ διαστέλλειν τὴν ἐπουράνιον βασιλεῖς ἐπʼ αὐτῆς, χιονωθήσονται ἐν Σελμών.
\vs{16}Ὄρος τοῦ Θεοῦ ὄρος πῖον, ὄρος τετυρωμένον, ὄρος πῖον.
\vs{17}Ἱνατί ὑπολαμβάνετε ὄρη τετυρωμένα; τὸ ὄρος ὃ εὐδόκησεν ὁ Θεὸς κατοικεῖν ἐν αὐτῷ· καὶ γὰρ ὁ Κύριος κατασκηνώσει εἰς τέλος.

\vs{18}Τὸ ἅρμα τοῦ Θεοῦ μυριοπλάσιον, χιλιάδες εὐθηνούντων· Κύριος ἐν αὐτοῖς ἐν Σινὰ ἐν τῷ ἁγίῳ.
\vs{19}Ἀναβὰς εἰς ὕψος, ᾐχμαλώτευσας αἰχμαλωσίαν· ἔλαβες δόματα ἐν ἀνθρώπῳ, καὶ γὰρ ἀπειθοῦντες τοῦ κατασκηνῶσαι.

\vs{20}Κύριος ὁ Θεὸς εὐλογητὸς, εὐλογητὸς Κύριος ἡμέραν καθʼ ἡμέραν, καὶ κατευοδώσει ἡμῖν ὁ Θεὸς τῶν σωτηρίων ἡμῶν· διάψαλμα.
\vs{21}Ὁ Θεὸς ἡμῶν, ὁ Θεὸς τοῦ σώζειν, καὶ τοῦ Κυρίου αἱ διέξοδοι τοῦ θανάτου.
\vs{22}Πλὴν ὁ Θεὸς συνθλάσει κεφαλὰς ἐχθρῶν αὐτοῦ, κορυφὴν τριχὸς διαπορευομένων ἐν πλημμελείαις αὐτῶν.
\vs{23}Εἶπε Κύριος, ἐκ Βασὰν ἐπιστρέψω, ἐπιστρέψω ἐν βυθοῖς θαλάσσης.
\vs{24}Ὅπως ἂν βαφῇ ὁ ποῦς σου ἐν αἵματι, ἡ γλῶσσα τῶν κυνῶν σου ἐξ ἐχθρῶν παρʼ αὐτοῦ.

\vs{25}Ἐθεωρήθησαν αἱ πορεῖαί σου ὁ Θεὸς, αἱ πορεῖαι τοῦ Θεοῦ μου τοῦ βασιλέως τοῦ ἐν τῷ ἁγίῳ.
\vs{26}Προέφθασαν ἄρχοντες ἐχόμενοι ψαλλόντων, ἐν μέσῳ νεανίδων τυμπανιστριῶν.
\vs{27}Ἐν ἐκκλησίαις εὐλογεῖτε τὸν Θεὸν, τὸν Κύριον ἐκ πηγῶν Ἰσραήλ.
\vs{28}Ἐκεῖ Βενιαμὶν νεώτερος ἐν ἐκστάσει, ἄρχοντες Ἰούδα ἡγεμόνες αὐτῶν, ἄρχοντες Ζαβουλὼν, ἄρχοντες Νεφθαλί.

\vs{29}Ἔντειλαι ὁ Θεὸς τῇ δυνάμει σου, δυνάμωσον ὁ Θεὸς τοῦτο, ὃ κατηρτίσω ἐν ἡμῖν.
\vs{30}Ἀπὸ τοῦ ναοῦ σου ἐπὶ Ἱερουσαλὴμ, σοὶ οἴσουσι βασιλεῖς δῶρα.
\vs{31}Ἐπιτίμησον τοῖς θηρίοις τοῦ καλάμου· ἡ συναγωγὴ τῶν ταύρων ἐν ταῖς δαμάλεσι τῶν λαῶν, τοῦ μὴ ἀποκλεισθῆναι τοὺς δεδοκιμασμένους τῷ ἀργυρίῳ· διασκόρπισον ἔθνη τὰ τοὺς πολέμους θέλοντα.
\vs{32}Ἥξουσι πρέσβεις ἐξ Αἰγύπτου, Αἰθιοπία προφθάσει χεῖρα αὐτῆς τῷ Θεῷ.

\vs{33}Αἱ βασιλεῖαι τῆς γῆς ᾄσατε τῷ Θεῷ, ψάλατε τῷ Κυρίῳ. διάψαλμα.
\vs{34}Ψάλατε τῷ Θεῷ τῷ ἐπιβεβηκότι ἐπὶ τὸν οὐρανὸν τοῦ οὐρανοῦ κατὰ ἀνατολὰς, ἰδοὺ δώσει ἐν τῇ φωνῇ αὐτοῦ φωνὴν δυνάμεως.
\vs{35}Δότε δόξαν τῷ Θεῷ, ἐπὶ τὸν Ἰσραὴλ ἡ μεγαλοπρέπεια αὐτοῦ, καὶ ἡ δύναμις αὐτοῦ ἐν ταῖς νεφέλαις.
\vs{36}Θαυμαστὸς ὁ Θεὸς ἐν τοῖς ὁσίοις αὐτοῦ, ὁ Θεὸς Ἰσραὴλ, αὐτὸς δώσει δύναμιν καὶ κραταίωσιν τῷ λαῷ αὐτοῦ· εὐλογητὸς ὁ Θεός.

\begin{psalmheading}{\ch{68}{69} Εἰς τὸ τέλος, ὑπὲρ τῶν ἀλλοιωθησομένων, τῶ Δαυίδ.}
\end{psalmheading}
\vs{2}Σώσον με ὁ Θεὸς, ὅτι εἰσήλθοσαν ὕδατα ἕως ψυχῆς μου.
\vs{3}Ἐνεπάγην εἰς ἰλὺν βυθοῦ, καὶ οὐκ ἔστιν ὑπόστασις· ἦλθον εἰς τὰ βάθη τῆς θαλάσσης, καὶ καταιγὶς κατεπόντισέ με.
\vs{4}Ἐκοπίασα κράζων, ἐβραγχίασεν ὁ λάρυγξ μου, ἐξέλιπον οἱ ὀφθαλμοί μου ἀπὸ τοῦ ἐλπίζειν με ἐπὶ τὸν Θεόν μου.
\vs{5}Ἐπληθύνθησαν ὑπὲρ τὰς τρίχας τῆς κεφαλῆς μου οἱ μισοῦντές με δωρεάν· ἐκραταιώθησαν οἱ ἐχθροί μου οἱ ἐκδιώκοντές με ἀδίκως· ἃ οὐχ ἥρπασα, τότε ἀπετίννυον.

\vs{6}Ὁ Θεὸς σὺ ἔγνως τὴν ἀφροσύνην μου, καὶ αἱ πλημμελειαί μου ἀπὸ σοῦ οὐκ ἐκρύβησαν.
\vs{7}Μὴ αἰσχυνθείησαν ἐπʼ ἐμὲ οἱ ὑπομένοντές σε Κύριε τῶν δυνάμεων, μὴ ἐντραπείησαν ἐπʼ ἐμὲ οἱ ζητοῦντές σε ὁ Θεὸς τοῦ Ἰσραήλ.
\vs{8}Ὅτι ἕνεκά σου ὑπήνεγκα ὀνειδισμὸν, ἐκάλυψεν ἐντροπὴ τὸ πρόσωπόν μου.
\vs{9}Ἀπηλλοτριωμένος ἐγενήθην τοῖς ἀδελφοῖς μου, καὶ ξένος τοῖς υἱοῖς τῆς μητρός μου·
\vs{10}ὅτι ὁ ζῆλος τοῦ οἴκου σου κατάφαγέ με, καὶ οἱ ὀνειδισμοὶ τῶν ὀνειδιζόντων σε ἐπέπεσον ἐπʼ ἐμέ.
\vs{11}Καὶ συνέκαμψα ἐν νηστείᾳ τὴν ψυχήν μου, καὶ ἐγενήθη εἰς ὀνειδισμοὺς ἐμοί.
\vs{12}Καὶ ἐθέμην τὸ ἔνδυμά μου σάκκον, καὶ ἐγενόμην αὐτοῖς εἰς παραβολήν.
\vs{13}Κατʼ ἐμοῦ ἠδολέσχουν οἱ καθήμενοι ἐν πύλῃ, καὶ εἰς ἐμὲ ἔψαλλον οἱ πίνοντες τὸν οἶνον.

\vs{14}Ἐγὼ δὲ τῇ προσευχῇ μου πρὸς σὲ Κύριε, καιρὸς εὐδοκίας ὁ Θεός· ἐν τῷ πλήθει τοῦ ἐλέους σου ἐπάκουσόν μου, ἐν ἀληθείᾳ τῆς σωτηρίας σου.
\vs{15}Σῶσόν με ἀπὸ πηλοῦ, ἵνα μὴ ἐνπαγῶ· ῥυσθείην ἐκ τῶν μισούντων με, καὶ ἐκ τοῦ βάθους τῶν ὑδάτων.
\vs{16}Μή με καταποντισάτω καταιγὶς ὕδατος, μηδὲ καταπιέτω με βυθὸς, μηδὲ συνσχέτω ἐπʼ ἐμὲ φρέαρ τὸ στόμα αὐτοῦ.
\vs{17}Εἰσάκουσόν μου Κύριε, ὅτι χρηστὸν τὸ ἔλεός σου, κατὰ τὸ πλῆθος τῶν οἰκτιρμῶν σου ἐπίβλεψον ἐπʼ ἐμε.
\vs{18}Καὶ μὴ ἀποστρέψῃς τὸ πρόσωπόν σου ἀπὸ τοῦ παιδός σου· ὅτι θλίβομαι, ταχὺ ἐπάκουσόν μου.
\vs{19}Πρόσχες τῇ ψυχῇ μου, καὶ λύτρωσαι αὐτὴν, ἕνεκα τῶν ἐχθρῶν μου ῥῦσαί με.

\vs{20}Σὺ γὰρ γινώσκεις τὸν ὀνειδισμόν μου, καὶ τὴν αἰσχύνην μου, καὶ τὴν ἐντροπήν μου· ἐναντίον σου πάντες οἱ θλίβοντές με.
\vs{21}Ὀνειδισμὸν προσεδόκησεν ἡ ψυχή μου καὶ ταλαιπωρίαν· καὶ ὑπέμεινα συλλυπούμενον, καὶ οὐχ ὑπῆρξε, καὶ παρακαλοῦντα, καὶ οὐχ εὗρον.
\vs{22}Καὶ ἔδωκαν εἰς τὸ βρῶμά μου χολὴν, καὶ εἰς τὴν δίψαν μου ἐπότισάν με ὄξος.
\vs{23}Γενηθήτω ἡ τράπεζα αὐτῶν ἐνώπιον αὐτῶν εἰς παγίδα, καὶ εἰς ἀνταπόδοσιν, καὶ εἰς σκάνδαλον.
\vs{24}Σκοτισθήτωσαν οἱ ὀφθαλμοὶ αὐτῶν τοῦ μὴ βλέπειν, καὶ τὸν νῶτον αὐτῶν διαπαντὸς σύγκαμψον.
\vs{25}Ἔκχεον ἐπʼ αὐτοὺς τὴν ὀργήν σου, καὶ ὁ θυμὸς τῆς ὀργῆς σου καταλάβοι αὐτούς.
\vs{26}Γενηθήτω ἡ ἔπαυλις αὐτῶν ἠρημωμένη, καὶ ἐν τοῖς σκηνώμασιν αὐτῶν μὴ ἔστω ὁ κατοικῶν·
\vs{27}ὅτι ὃν σὺ ἐπάταξας, αὐτοὶ κατεδίωξαν, καὶ ἐπὶ τὸ ἄλγος τῶν τραυμάτων μου προσέθηκαν.
\vs{28}Πρόσθες ἀνομίαν ἐπὶ τὴν ἀνομίαν αὐτῶν, καὶ μὴ εἰσελθέτωσαν ἐν δικαιοσύνῃ σου.
\vs{29}Ἐξαλειφθήτωσαν ἐκ βίβλου ζώντων, καὶ μετὰ δικαίων μὴ γραφήτωσαν.

\vs{30}Πτωχὸς καὶ ἀλγῶν εἰμι ἐγὼ, καὶ ἡ σωτηρία τοῦ προσώπου σου ἀντελάβετό μου.
\vs{31}Αἰνέσω τὸ ὄνομα τοῦ Θεοῦ μου μετʼ ᾠδῆς, μεγαλυνῶ αὐτὸν ἐν αἰνέσει·
\vs{32}καὶ ἀρέσει τῷ Θεῷ ὑπὲρ μόσχον νέον κέρατα ἐκφέροντα καὶ ὁπλάς.
\vs{33}Ἰδέτωσαν πτωχοὶ καὶ εὐφρανθήτωσαν· ἐκζητήσατε τὸν Θεὸν, καὶ ζήσεσθε.
\vs{34}Ὅτι εἰσήκουσε τῶν πενήτων ὁ Κύριος, καὶ τοὺς πεπεδημένους αὐτοῦ οὐκ ἐξουδένωσεν.
\vs{35}Αἰνεσάτωσαν αὐτὸν οἱ οὐρανοὶ καὶ ἡ γῆ, θάλασσα καὶ πάντα τὰ ἕρποντα ἐν αὐτοῖς.
\vs{36}Ὅτι ὁ Θεὸς σώσει τὴν Σιὼν, καὶ οἰκοδομηθήσονται αἱ πόλεις τῆς Ἰουδαίας, καὶ κατοικήσουσιν ἐκεῖ, καὶ κληρονομήσουσιν αὐτήν.
\vs{37}Καὶ τὸ σπέρμα τῶν δούλων αὐτοῦ καθέξουσιν αὐτὴν, καὶ οἱ ἀγαπῶντες τὸ ὄνομα αὐτοῦ κατασκηνώσουσιν ἐν αὐτῇ.

\begin{psalmheading}{\ch{69}{70} Εἰς τὸ τέλος, τῷ Δαυὶδ εἰς ἀνάμνησιν, εἰς τὸ σῶσαί με Κύριον.}
\end{psalmheading}
\vs{2}Ὁ Θεὸς εἰς τὴν βοήθειάν μου πρόσχες.
\vs{3}Αἰσχυνθείησαν καὶ ἐντραπείησαν οἱ ζητοῦντες τὴν ψυχήν μου, ἀποστραφείησαν εἰς τὰ ὀπίσω, καὶ καταισχυνθείησαν οἱ βουλόμενοί μοι κατά·
\vs{4}Ἀποστραφείησαν παραυτίκα αἰσχυνόμενοι οἱ λέγοντές μοι, εὖγε, εὖγε.
\vs{5}Ἀγαλλιάσθωσαν καὶ εὐφρανθήτωσαν ἐπὶ σοὶ πάντες οἱ ζητοῦντές σε, καὶ λεγέτωσαν διαπαντὸς, μεγαλυνθήτω ὁ Θεὸς, οἱ ἀγαπῶντες τὸ σωτήριόν σου.
\vs{6}Ἐγὼ δὲ πτωχὸς καὶ πένης, ὁ Θεὸς βοήθησόν μοι· βοηθός μου, καὶ ῥύστης μου εἶ σὺ, Κύριε μὴ χρονίσῃς.

\begin{psalmheading}{\ch{70}{71} Τῷ Δαυὶδ υἱῶν Ἰωναδὰβ, καὶ τῶν πρώτων αἰχμαλωτισθέντων.}
\end{psalmheading}
Ἐπὶ σοὶ Κύριε ἤλπισα, μὴ καταισχυνθείην εἰς τὸν αἰῶνα.
\vs{2}Ἐν τῇ δικαιοσύνῃ σου ῥῦσαί με καὶ ἐξελοῦ με, κλῖνον πρὸς μὲ τὸ οὖς σου καὶ σῶσόν με.
\vs{3}Γενοῦ μοι εἰς Θεὸν ὑπερασπιστὴν, καὶ εἰς τόπον ὀχυρὸν τοῦ σῶσαί με, ὅτι στερέωμά μου καὶ καταφυγή μου εἶ σύ.
\vs{4}Ὁ Θεός μου ῥῦσαί με ἐκ χειρὸς ἁμαρτωλοῦ, ἐκ χειρὸς παρανομοῦντος καὶ ἀδικοῦντος.
\vs{5}Ὅτι σὺ εἶ ἡ ὑπομονή μου Κύριε, Κύριε ἡ ἐλπίς μου ἐκ νεότητός μου·
\vs{6}Ἐπὶ σὲ ἐπεστηρίχθην ἀπὸ γαστρὸς, ἐκ κοιλίας μητρός μου σύ μου εἶ σκεπαστής. ἐν σοὶ ἡ ὕμνησίς μου διαπαντός.

\vs{7}Ὡσεὶ τέρας ἐγενήθην τοῖς πολλοῖς, καὶ σὺ βοηθὸς κραταιός.
\vs{8}Πληρωθήτω τὸ στόμα μου αἰνέσεως, ὅπως ὑμνήσω τὴν δόξαν σου, ὅλην τὴν ἡμέραν τὴν μεγαλοπρέπειάν σου.
\vs{9}Μὴ ἀποῤῥίψῃς με εἰς καιρὸν γήρους, ἐν τῷ ἐκλείπειν τὴν ἰσχύν μου μὴ ἐγκαταλίπῃς με.
\vs{10}Ὅτι εἶπαν οἱ ἐχθροί μου ἐμοὶ, καὶ οἱ φυλάσσοντες τὴν ψυχήν μου ἐβουλεύσαντο ἐπιτοαυτὸ,
\vs{11}λέγοντες, ὁ Θεὸς ἐγκατέλιπεν αὐτὸν, καταδιώξατε καὶ καταλάβετε αὐτὸν, ὅτι οὐκ ἔστιν ὁ ῥυόμενος.
\vs{12}Ὁ Θεὸς μὴ μακρύνῃς ἀπʼ ἐμοῦ,
\vs{13}ὁ Θεός μου εἰς τὴν βοήθειάν μου πρόσχες. Αἰσχυνθήτωσαν καὶ ἐκλιπέτωσαν οἱ ἐνδιαβάλλοντες τὴν ψυχήν μου, περιβαλλέσθωσαν αἰσχύνην καὶ ἐντροπὴν οἱ ζητοῦντες τὰ κακά μοι.

\vs{14}Ἐγὼ δὲ διαπαντὸς ἐλπιῶ, καὶ προσθήσω ἐπὶ πᾶσαν τὴν αἴνεσίν σου.
\vs{15}Τὸ στόμα μου ἐξαγγελεῖ τὴν δικαιοσύνην σου, ὅλην τὴν ἡμέραν τὴν σωτηρίαν σου· ὅτι οὐκ ἔγνων πραγματείας.
\vs{16}Εἰσελεύσομαι ἐν δυναστείᾳ Κυρίου, Κύριε μνησθήσομαι τῆς δικαιοσύνης σου μόνου.
\vs{17}Ἐδίδαξας με ὁ Θεὸς ἐκ νεότητός μου, καὶ μέχρι νῦν ἀπαγγελῶ τὰ θαυμάσιά σου,
\vs{18}καὶ ἕως γήρους καὶ πρεσβείου· ὁ Θεὸς μὴ ἐγκαταλίπῃς με, ἕως ἂν ἀπαγγείλω τὸν βραχίονά σου πάσῃ τῇ γενεᾷ τῇ ἐρχομένῃ· Τὴν δυναστείαν σου,
\vs{19}καὶ τὴν δικαιοσύνην σου ὁ Θεὸς ἕως ὑψίστων, ἃ ἐποίησας μεγαλεῖα· ὁ Θεὸς τίς ὅμοιός σοι;

\vs{20}Ὅσας ἔδειξάς μοι θλίψεις πολλὰς καὶ κακάς; καὶ ἐπιστρέψας ἐζωοποίησάς με, καὶ ἐκ τῶν ἀβύσσων τῆς γῆς πάλιν ἀνήγαγές με.
\vs{21}Ἐπλεόνασας τὴν δικαιοσύνην σου, καὶ ἐπιστρέψας παρεκάλεσάς με, καὶ ἐκ τῶν ἀβύσσων τῆς γῆς πάλιν ἀνήγαγές με.
\vs{22}Καὶ γὰρ ἐγὼ ἐξομολογήσομαί σοι ἐν σκεύει ψαλμοῦ τὴν ἀλήθειάν σου ὁ Θεὸς, ψαλῶ σοι ἐν κιθάρᾳ ὁ ἅγιος τοῦ Ἰσραήλ.
\vs{23}Ἀγαλλιάσονται τὰ χείλη μου ὅταν ψάλω σοι, καὶ ἡ ψυχή μου ἣν ἐλυτρώσω.
\vs{24}Ἔτι δὲ καὶ ἡ γλῶσσά μου ὅλην τὴν ἡμέραν μελετήσει τὴν δικαιοσύνην σου, ὅταν αἰσχυνθῶσι καὶ ἐντραπῶσιν οἱ ζητοῦντες τὰ κακά μοι.

\begin{psalmheading}{\ch{71}{72} Εἰς Σαλωμών.}
\end{psalmheading}
Ὁ Θεὸς τὸ κρίμα σου τῷ βασιλεῖ δὸς, καὶ τὴν δικαιοσύνην σου τῷ υἱῷ τοῦ βασιλέως·
\vs{2}κρίνειν τὸν λαόν σου ἐν δικαιοσύνῃ, καὶ τοὺς πτωχούς σου ἐν κρίσει.

\vs{3}Ἀναλαβέτω τὰ ὄρη εἰρήνην τῷ λαῷ σου, καὶ οἱ βουνοί·
\vs{4}ἐν δικαιοσύνῃ κρινεῖ τοὺς πτωχοὺς τοῦ λαοῦ, καὶ σώσει τοὺς υἱοὺς τῶν πενήτων· καὶ ταπεινώσει συκοφάντην,
\vs{5}καὶ συμπαραμενεῖ τῷ ἡλίῳ, καὶ πρὸ τῆς σελήνης γενεὰς γενεῶν.
\vs{6}Καταβήσεται ὡς ὑετὸς ἐπὶ πόκον, καὶ ὡσεὶ σταγόνες στάζουσαι ἐπὶ τὴν γῆν.
\vs{7}Ἀνατελεῖ ἐν ταῖς ἡμέραις αὐτοῦ δικαιοσύνη, καὶ πλῆθος εἰρήνης ἕως οὗ ἀνταναιρεθῇ ἡ σελήνη.
\vs{8}Καὶ κατακυριεύσει ἀπὸ θαλάσσης ἕως θαλάσσης, καὶ ἀπὸ ποταμοῦ ἕως περάτων τῆς οἰκουμένης.
\vs{9}Ἐνώπιον αὐτοῦ προπεσοῦνται Αἰθίοπες, καὶ οἱ ἐχθροὶ αὐτοῦ χοῦν λείξουσι.
\vs{10}Βασιλεῖς Θαρσὶς καὶ αἱ νῆσοι δῶρα προσοίσουσι, βασιλεῖς Ἀράβων καὶ Σαβὰ δῶρα προσάξουσι.
\vs{11}Καὶ προσκυνήσουσιν αὐτῷ πάντες οἱ βασιλεῖς, πάντα τὰ ἔθνη δουλεύσουσιν αὐτῷ.
\vs{12}Ὅτι ἐῤῥύσατο πτωχὸν ἐκ δυνάστου, καὶ πένητα ᾧ οὐχ ὑπῆρχε βοηθός.
\vs{13}Φείσεται πτωχοῦ καὶ πένητος, καὶ ψυχὰς πένητων σώσει.
\vs{14}Ἐκ τόκου καὶ ἐξ ἀδικίας λυτρώσεται τὰς ψυχὰς αὐτῶν, καὶ ἔντιμον τὸ ὄνομα αὐτῶν ἐνώπιον αὐτοῦ.
\vs{15}Καὶ ζήσεται, καὶ δοθήσεται αὐτῷ ἐκ τοῦ χρυσίου τῆς Ἀραβίας, καὶ προσεύξονται περὶ αὐτοῦ διαπαντός· ὅλην τὴν ἡμέραν εὐλογήσουσιν αὐτόν.
\vs{16}Ἔσται στήριγμα ἐν τῇ γῇ ἐπʼ ἄκρων τῶν ὀρέων· ὑπεραρθήσεται ὑπὲρ τὸν Λίβανον ὁ καρπὸς αὐτοῦ, καὶ ἐξανθήσουσιν ἐκ πόλεως ὡσεὶ χόρτος τῆς γῆς.

\vs{17}Ἔστω τὸ ὄνομα αὐτοῦ εὐλογημένον εἰς τοὺς αἰῶνας, πρὸ τοῦ ἡλίου διαμενεῖ τὸ ὄνομα αὐτοῦ, καὶ εὐλογηθήσονται ἐν αὐτῷ πᾶσαι αἱ φυλαὶ τῆς γῆς· πάντα τὰ ἔθνη μακαριοῦσιν αὐτόν.

\vs{18}Εὐλογητὸς Κύριος ὁ Θεὸς τοῦ Ἰσραὴλ, ὁ ποιῶν θαυμάσια μόνος,
\vs{19}καὶ εὐλογητὸν τὸ ὄνομα τῆς δόξης αὐτοῦ εἰς τὸν αἰῶνα καὶ εἰς αἰῶνα τοῦ αἰῶνος· καὶ πληρωθήσεται τῆς δόξης αὐτοῦ πᾶσα ἡ γῆ. γένοιτο, γένοιτο.

\textit{Ἐξέλιπον οἱ ὕμνοι Δαυὶδ τοῦ υἱοῦ Ἰεσσαί.}

\begin{psalmheading}{\ch{72}{73} Ψαλμὸς τῷ Ἀσάφ.}
\end{psalmheading}
Ὡς ἀγαθὸς ὁ Θεὸς τῷ Ἰσραὴλ, τοῖς εὐθέσι καρδίᾳ.
\vs{2}Ἐμοῦ δὲ παρὰ μικρὸν ἐσαλεύθησαν οἱ πόδες, παρʼ ὀλίγον ἐξεχύθη τὰ διαβήματά μου.
\vs{3}Ὅτι ἐζήλωσα ἐπὶ τοῖς ἀνόμοις, εἰρήνην ἁμαρτωλῶν θεωρῶν.

\vs{4}Ὅτι οὐκ ἔστιν ἀνάνευσις ἐν τῷ θανάτῳ αὐτῶν, καὶ στερέωμα ἐν τῇ μάστιγι αὐτῶν.
\vs{5}Ἐν κόποις ἀνθρώπων οὐκ εἰσὶ, καὶ μετὰ ἀνθρώπων οὐ μαστιγωθήσονται.
\vs{6}Διὰ τοῦτο ἐκράτησεν αὐτοὺς ἡ ὑπερηφανία, περιεβάλοντο ἀδικίαν καὶ ἀσέβειαν αὐτῶν.
\vs{7}Ἐξελεύσεται ὡς ἐκ στέατος ἡ ἀδικία αὐτῶν· διῆλθον εἰς διάθεσιν καρδίας.
\vs{8}Διενοήθησαν, καὶ ἐλάλησαν ἐν πονηρίᾳ, ἀδικίαν εἰς τὸ ὕψος ἐλάλησαν.
\vs{9}Ἔθεντο εἰς οὐρανὸν τὸ στόμα αὐτῶν, καὶ ἡ γλῶσσα αὐτῶν διῆλθεν ἐπὶ τῆς γῆς.
\vs{10}Διὰ τοῦτο ἐπιστρέψει ὁ λαός μου ἐνταῦθα, καὶ ἡμέραι πλήρεις εὑρεθήσονται ἐν αὐτοῖς.
\vs{11}Καὶ εἶπαν, πῶς ἔγνω ὁ Θεὸς, καὶ εἰ ἔστι γνῶσις ἐν τῷ ὑψίστῳ;
\vs{12}Ἰδοὺ οὗτοι οἱ ἁμαρτωλοὶ καὶ εὐθηνοῦντες εἰς τὸν αἰῶνα, κατέσχον πλούτου.

\vs{13}Καὶ εἶπα, ἄρα ματαίως ἐδικαίωσα τὴν καρδίαν μου, καὶ ἐνιψάμην ἐν ἀθώοις τὰς χεῖράς μου·
\vs{14}Καὶ ἐγενόμην μεμαστιγωμένος ὅλην τὴν ἡμέραν, καὶ ὁ ἔλεγχός μου εἰς τὰς πρωΐας.
\vs{15}Εἰ ἔλεγον, διηγήσομαι οὕτως, ἰδοὺ τῇ γενεᾷ τῶν υἱῶν σου ἠσυνθέτηκα.
\vs{16}Καὶ ὑπέλαβον τοῦ γνῶναι, τοῦτο κόπος ἐστὶν ἐναντίον μου,
\vs{17}ἕως εἰσέλθω εἰς τὸ ἁγιαστήριον τοῦ Θεοῦ, συνῶ εἰς τὰ ἔσχατα.

\vs{18}Πλὴν διὰ τὰς δολιότητας ἔθου αὐτοῖς, κατέβαλες αὐτοὺς ἐν τῷ ἐπαρθῆναι.
\vs{19}Πῶς ἐγένοντο εἰς ἐρήμωσιν; ἐξάπινα ἐξέλιπον, ἀπώλοντο διὰ τὴν ἀνομίαν αὐτῶν.
\vs{20}Ὡσεὶ ἐνύπνιον ἐξεγειρομένου, Κύριε ἐν τῇ πόλει σου τὴν εἰκόνα αὐτῶν ἐξουδενώσεις.

\vs{21}Ὅτι ηὐφράνθη ἡ καρδία μου, καὶ οἱ νεφροί μου ἠλλοιώθησαν.
\vs{22}Κᾀγὼ ἐξουδενωμένος, καὶ οὐκ ἔγνων, κτηνώδης ἐγενόμην παρὰ σοὶ,
\vs{23}κᾀγὼ διαπαντὸς μετὰ σου· ἐκράτησας τῆς χειρὸς τῆς δεξιᾶς μου,
\vs{24}ἐν τῇ βουλῇ σου ὡδήγησάς με, καὶ μετὰ δόξης προσελάβου με.
\vs{25}Τί γάρ μοι ὑπάρχει ἐν τῷ οὐρανῷ, καὶ παρὰ σοῦ τί ἠθέλησα ἐπὶ τῆς γῆς;
\vs{26}Ἐξέλιπεν ἡ καρδία μου καὶ ἡ σάρξ μου, ὁ Θεὸς τῆς καρδίας μου, καὶ ἡ μερίς μου ὁ Θεὸς εἰς τὸν αἰῶνα.

\vs{27}Ὅτι ἰδοὺ οἱ μακρύνοντες ἑαυτοὺς ἀπὸ σοῦ, ἀπολοῦνται· ἐξωλόθρευσας πάντα τὸν πορνεύοντα ἀπὸ σοῦ.
\vs{28}Ἐμοὶ δὲ τὸ προσκολλᾶσθαι τῷ Θεῷ ἀγαθόν ἐστι, τίθεσθαι ἐν τῷ Κυρίῳ τὴν ἐλπίδα μου· τοῦ ἐξαγγεῖλαι πάσας τὰς αἰνέσεις σου ἐν ταῖς πύλαις τῆς θυγατρὸς Σιών.

\begin{psalmheading}{\ch{73}{74} Συνέσεως τῷ Ἀσάφ.}
\end{psalmheading}
Ἱνατί ἀπόσω ὁ Θεὸς εἰς τέλος; ὠργίσθη ὁ θυμός σου ἐπὶ πρόβατα νομῆς σου;
\vs{2}Μνήσθητι τῆς συναγωγῆς σου ἧς ἐκτήσω ἀπʼ ἀρχῆς· ἐλυτρώσω ῥάβδον κληρονομίας σου· ὄρος Σιὼν τοῦτο ὃ κατεσκήνωσας ἐν αὐτῷ.
\vs{3}Ἔπαρον τὰς χεῖράς σου ἐπὶ τὰς ὑπερηφανίας αὐτῶν εἰς τέλος· ὅσα ἐπονηρεύσατο ὁ ἐχθρὸς ἐν τοῖς ἁγίοις σου.

\vs{4}Καὶ ἐνεκαυχήσαντο οἱ μισοῦντές σε ἐν μέσῳ τῆς ἑορτῆς σου·
\vs{5}ἔθεντο τὰ σημεῖα αὐτῶν σημεῖα, καὶ οὐκ ἔγνωσαν, ὡς εἰς τὴν εἴσοδον ὑπεράνω· ὡς ἐν δρυμῷ ξύλων ἀξίναις ἐξέκοψαν
\vs{6}τὰς θύρας αὐτῆς ἐπιτοαυτὸ, ἐν πελέκει καὶ λαξευτηρίῳ κατέῤῥαξαν αὐτήν.
\vs{7}Ἐνεπύρισαν ἐν πυρὶ τὸ ἁγιαστήριόν σου εἰς τὴν γῆν, ἐβεβήλωσαν τὸ σκήνωμα τοῦ ὀνόματός σου.
\vs{8}Εἶπαν ἐν τῇ καρδίᾳ αὐτῶν, ἡ συγγένεια αὐτῶν ἐπιτοαυτὸ, δεῦτε, καταπαύσωμεν τὰς ἑορτὰς Κυρίου ἀπὸ τῆς γῆς.
\vs{9}Τὰ σημεῖα ἡμῶν οὐκ εἴδομεν, οὐκ ἔστιν ἔτι προφήτης, καὶ ἡμᾶς οὐ γνώσεται ἔτι.

\vs{10}Ἕως πότε ὁ Θεὸς, ὀνειδιεῖ ὁ ἐχθρὸς, παροξυνεῖ ὁ ὑπεναντίος τὸ ὄνομά σου εἰς τέλος;
\vs{11}Ἱνατί ἀποστρέφεις τὴν χεῖρά σου, καὶ τὴν δεξιάν σου ἐκ μέσου τοῦ κόλπου σου εἰς τέλος;
\vs{12}Ὁ δὲ Θεὸς βασιλεὺς ἡμῶν πρὸ αἰῶνος, εἰργάσατο σωτηρίαν ἐν μέσῳ τῆς γῆς.
\vs{13}Σὺ ἐκραταίωσας ἐν τῇ δυνάμει σου τὴν θάλασσαν, σὺ συνέτριψας τὰς κεφαλὰς τῶν δρακόντων ἐπὶ τοῦ ὕδατος. Σὺ συνέτριψας τὰς κεφαλὰς τοῦ δράκοντος,
\vs{14}ἔδωκας αὐτὸν βρῶμα λαοῖς τοῖς Αἰθίοψι.
\vs{15}Σὺ διέῤῥηξας πηγὰς καὶ χειμάῤῥους, σὺ ἐξήρανας ποταμοὺς ἠθάμ.
\vs{16}Σή ἐστιν ἡ ἡμέρα, καὶ σή ἐστιν ἡ νὺξ, σὺ κατηρτίσω ἥλιον καὶ σελήνην.
\vs{17}Σὺ ἐποίησας πάντα τὰ ὅρια τῆς γῆς, θέρος καὶ ἔαρ σὺ ἐποίησας.

\vs{18}Μνήσθητι ταύτης τῆς κτίσεώς σου· ἐχθρὸς ὠνείδισε τὸν Κύριον, καὶ λαὸς ἄφρων παρώξυνε τὸ ὄνομά σου.
\vs{19}Μὴ παραδῷς τοῖς θηρίοις ψυχὴν ἐξομολογουμένην σοι, τῶν ψυχῶν τῶν πενήτων σου μὴ ἐπιλάθῃ εἰς τέλος.
\vs{20}Ἐπίβλεψον εἰς τὴν διαθήκην σου, ὅτι ἐπληρωθήσαν οἱ ἐσκοτωμένοι τῆς γῆς οἴκων ἀνομιῶν.
\vs{21}Μὴ ἀποστραφήτω τεταπεινωμένος καὶ κατῃσχυμμένος, πτωχὸς καὶ πένης αἰνέσουσι τὸ ὄνομά σου.
\vs{22}Ἀνάστα ὁ Θεὸς, δίκασον τὴν δίκην σου, μνήσθητι τῶν ὀνειδισμῶν σου τῶν ὑπὸ ἄφρονος ὅλην τὴν ἡμέραν.
\vs{23}Μὴ ἐπιλάθῃ τῆς φωνῆς τῶν ἱκετῶν σου, ἡ ὑπερηφανία τῶν μισούντων σε ἀναβαίη διαπαντὸς πρὸς σέ.

\begin{psalmheading}{\ch{74}{75} Εἰς τὸ τέλος, μὴ διαφθείρῃς, ψαλμὸς ᾠδῆς τῷ Ἀσάφ.}
\end{psalmheading}
\vs{2}Ἐξομολογησόμεθα σοι ὁ Θεὸς, ἐξομολογησόμεθα, καὶ ἐπικαλεσόμεθα τὸ ὄνομά σου· διηγήσομαι πάντα τὰ θαυμάσιά σου.
\vs{3}Ὅταν λάβω καιρὸν, ἐγὼ εὐθύτητας κρινῶ.
\vs{4}Ἐτάκη ἡ γῆ, καὶ πάντες οἱ κατοικοῦντες αὐτὴν, ἐγὼ ἐστερέωσα τοὺς στύλους αὐτῆς· διάψαλμα.

\vs{5}Εἶπα τοῖς παρανομοῦσι, μὴ παρανομεῖν, καὶ τοῖς ἁμαρτάνουσι, μὴ ὑψοῦτε κέρας.
\vs{6}Μὴ ἐπαίρετε εἰς ὕψος τὸ κέρας ὑμῶν, μὴ λαλεῖτε κατὰ τοῦ Θεοῦ ἀδικίαν.
\vs{7}Ὅτι οὔτε ἀπὸ ἐξόδων, οὔτε ἀπὸ δυσμῶν, οὔτε ἀπὸ ἐρήμων ὀρέων,
\vs{8}ὅτι ὁ Θεὸς κριτής ἐστι· τοῦτον ταπεινοῖ, καὶ τοῦτον ὑψοῖ.
\vs{9}Ὅτι ποτήριον ἐν χειρὶ Κυρίου, οἴνου ἀκράτου πλῆρες κεράσματος· καὶ ἔκλινεν ἐκ τούτου εἰς τοῦτο, πλὴν ὁ τρυγίας αὐτοῦ οὐκ ἐξεκενώθη· πίονται πάντες οἱ ἁμαρτωλοὶ τῆς γῆς.

\vs{10}Ἐγὼ δὲ ἀγαλλιάσομαι εἰς τὸν αἰῶνα, ψαλῶ τῷ Θεῷ Ἰακώβ.
\vs{11}Καὶ πάντα τὰ κέρατα τῶν ἁμαρτωλῶν συγκλάσω, καὶ ὑψωθήσεται τὰ κέρατα τοῦ δικαίου.

\begin{psalmheading}{\ch{75}{76} Εἰς τὸ τέλος ἐν ὕμνοις, ψαλμὸς τῷ Ἀσάφ· ᾠδὴ πρὸς τὸν Ἀσσύριον.}
\end{psalmheading}
\vs{2}Γνωστὸς ἐν τῇ Ἰουδαίᾳ ὁ Θεὸς, ἐν τῷ Ἰσραὴλ μέγα τὸ ὄνομα αὐτοῦ.
\vs{3}Καὶ ἐγενήθη ἐν εἰρήνῃ ὁ τόπος αὐτοῦ, καὶ τὸ κατοικητήριον αὐτοῦ ἐν Σιών.
\vs{4}Ἐκεῖ συνέτριψε τὰ κράτη τῶν τόξων, ὅπλον καὶ ῥομφαίαν καὶ πόλεμον· διάψαλμα.

\vs{5}Φωτίζεις σὺ θαυμαστῶς ἀπὸ ὀρέων αἰωνίων,
\vs{6}ἐταράχθησαν πάντες οἱ ἀσύνετοι τῇ καρδίᾳ· ὕπνωσαν ὕπνον αὐτῶν, καὶ οὐχ εὗρον οὐδὲν πάντες οἱ ἄνδρες τοῦ πλούτου ταῖς χερσὶν αὐτῶν.
\vs{7}Ἀπὸ ἐπιτιμήσεώς σου, ὁ Θεὸς Ἰακὼβ, ἐνύσταξαν οἱ ἐπιβεβηκότες τοὺς ἵππους.
\vs{8}Σὺ φοβερὸς εἶ, καὶ τίς ἀντιστήσεταί σοι ἀπὸ τῆς ὀργῆς σου;
\vs{9}Ἐκ τοῦ οὐρανοῦ ἠκούτισας κρίσιν, γῆ ἐφοβήθη καὶ ἡσύχασεν,
\vs{10}ἐν τῷ ἀναστῆναι εἰς κρίσιν τὸν Θεὸν, τοῦ σῶσαι πάντας τοὺς πρᾳεῖς τῇ καρδίᾳ. διάψαλμα.

\vs{11}Ὅτι ἐνθύμιον ἀνθρώπου ἐξομολογήσεταί σοι, καὶ ἐγκατάλειμμα ἐνθυμίου ἑορτάσει σοι.
\vs{12}Εὔξασθε καὶ ἀπόδοτε Κυρίῳ τῷ Θεῷ ἡμῶν, πάντες οἱ κύκλῳ αὐτοῦ οἴσουσι δῶρα·
\vs{13}τῷ φοβερῷ καὶ ἀφαιρουμένῳ πνεύματα ἀρχόντων, φοβερῷ παρὰ τοῖς βασιλεῦσι τῆς γῆς.

\begin{psalmheading}{\ch{76}{77} Εἰς τὸ τέλος, ὑπὲρ Ἰδιθοὺν ψαλμὸς τῷ Ἀσάφ.}
\end{psalmheading}
\vs{2}Φωνῇ μου πρὸς Κύριον ἐκέκραξα, καὶ ἡ φωνή μου πρὸς τὸν Θεὸν, καὶ προσέσχε μοι.
\vs{3}Ἐν ἡμέρᾳ θλίψεώς μου τὸν Θεὸν ἐξεζήτησα, ταῖς χερσί μου νυκτὸς ἐναντίον αὐτοῦ, καὶ οὐκ ἠπατήθην· ἀπηνῄνατο παρακληθῆναι ἡ ψυχή μου·
\vs{4}Ἐμνήσθην τοῦ Θεοῦ, καὶ εὐφράνθην, ἠδολέσχησα, καὶ ὠλιγοψύχησε τὸ πνεῦμά μου· διάψαλμα.
\vs{5}Προκατελάβοντο φυλακὰς πάντες οἱ ἐχθροί μου, ἐταράχθην καὶ οὐκ ἐλάλησα.

\vs{6}Διελογισάμην ἡμέρας ἀρχαίας, καὶ ἔτη αἰώνια
\vs{7}ἐμνήσθην, καὶ ἐμελέτησα· νυκτὸς μετὰ τῆς καρδίας μου ἠδολέσχουν, καὶ ἔσκαλλον τὸ πνεῦμά μου.
\vs{8}Μὴ εἰς τοὺς αἰῶνας ἀπώσεται Κύριος, καὶ οὐ προσθήσει τοῦ εὐδοκῆσαι ἔτι;
\vs{9}Ἢ εἰς τέλος ἀποκόψει τὸ ἔλεος ἀπὸ γενεᾶς καὶ γενεᾶς;
\vs{10}Ἢ ἐπιλήσεται τοῦ οἰκτειρῆσαι ὁ Θεὸς, ἢ συνέξει ἐν τῇ ὀργῇ αὐτοῦ τοὺς οἰκτιρμοὺς αὐτοῦ; διάψαλμα.

\vs{11}Καὶ εἶπα, νῦν ἠρξάμην, αὕτη ἡ ἀλλοίωσις τῆς δεξιᾶς τοῦ ὑψίστου.
\vs{12}Ἐμνήσθην τῶν ἔργων Κυρίου, ὅτι μνησθήσομαι ἀπὸ τῆς ἀρχῆς τῶν θαυμασίων σου,
\vs{13}καὶ μελετήσω ἐν πᾶσι τοῖς ἔργοις σου, καὶ ἐν τοῖς ἐπιτηδεύμασί σου ἀδολεσχήσω.

\vs{14}Ὁ Θεὸς ἐν τῷ ἁγίῳ ἡ ὁδός σου, τίς θεὸς μέγας ὡς ὁ Θεὸς ἡμῶν;
\vs{15}Σὺ εἶ ὁ Θεὸς ὁ ποιῶν θαυμάσια, ἐγνώρισας ἐν τοῖς λαοῖς τὴν δύναμίν σου·
\vs{16}ἐλυτρώσω ἐν τῷ βραχίονί σου τὸν λαόν σου, τοὺς υἱοὺς Ἰακὼβ καὶ Ἰωσήφ· διάψαλμα.
\vs{17}Εἴδοσάν σε ὕδατα ὁ Θεὸς, εἴδοσάν σε ὕδατα καὶ ἐφοβήθησαν, καὶ ἐταράχθησαν ἄβυσσοι.
\vs{18}Πλῆθος ἤχους ὑδάτων, φωνὴν ἔδωκαν αἱ νεφέλαι· καὶ γὰρ τὰ βέλη σου διαπορεύονται.
\vs{19}Φωνὴ τῆς βροντῆς σου ἐν τῷ τροχῷ· ἔφαναν αἱ ἀστραπαί σου τῇ οἰκουμένῃ, ἐσαλεύθη καὶ ἔντρομος ἐγενήθη ἡ γῆ.
\vs{20}Ἐν τῇ θαλάσσῃ ἡ ὁδός σου, καὶ αἱ τρίβοι σου ἐν ὕδασι πολλοῖς, καὶ τὰ ἴχνη σου οὐ γνωσθήσονται.
\vs{21}Ὡδήγησας ὡς πρόβατα τὸν λαόν σου ἐν χειρὶ Μωυσῆ καὶ Ἀαρών.

\begin{psalmheading}{\ch{77}{78} Συνέσεως τῷ Ἀσάφ.}
\end{psalmheading}
Προσέχετε λαός μου τὸν νόμον μου, κλίνατε τὸ οὖς ὑμῶν εἰς τὰ ῥήματα τοῦ στόματός μου.
\vs{2}Ἀνοίξω ἐν παραβολαῖς τὸ στόμα μου, φθέγξομαι προβλήματα ἀπʼ ἀρχῆς.
\vs{3}Ὅσα ἠκούσαμεν καὶ ἔγνωμεν αὐτὰ, καὶ οἱ πατέρες ἡμῶν διηγήσαντο ἡμῖν.
\vs{4}Οὐκ ἐκρύβη ἀπὸ τῶν τέκνων αὐτῶν εἰς γενεὰν ἑτέραν, ἀπαγγέλλοντες τὰς αἰνέσεις Κυρίου καὶ τὰς δυναστείας αὐτοῦ, καὶ τὰ θαυμάσια αὐτοῦ ἃ ἐποίησε.

\vs{5}Καὶ ἀνέστησε μαρτύριον ἐν Ἰακὼβ, καὶ νόμον ἔθετο ἐν Ἰσραήλ· ὃν ἐνετείλατο τοῖς πατράσιν ἡμῶν, γνωρίσαι αὐτὸν τοῖς υἱοῖς αὐτῶν,
\vs{6}ὅπως ἂν γνῷ γενεὰ ἑτέρα, υἱοὶ οἱ τεχθησόμενοι, καὶ ἀναστήσονται καὶ ἀπαγγελοῦσιν αὐτὰ τοῖς υἱοῖς αὐτῶν·
\vs{7}ἵνα θῶνται ἐπὶ τὸν Θεὸν τὴν ἐλπίδα αὐτῶν, καὶ μὴ ἐπιλάθωνται τῶν ἔργων τοῦ Θεοῦ, καὶ τὰς ἐντολὰς αὐτοῦ ἐκζητήσωσιν.
\vs{8}Ἵνα μὴ γένωνται ὡς οἱ πατέρες αὐτῶν, γενεὰ σκολιὰ καὶ παραπικραίνουσα, γενεὰ ἥτις οὐ κατεύθυνεν ἐν τῇ καρδίᾳ αὐτῆς, καὶ οὐκ ἐπιστώθη μετὰ τοῦ Θεοῦ τὸ πνεῦμα αὐτῆς.

\vs{9}Υἱοὶ Ἐφραὶμ ἐντεινοντες καὶ βάλλοντες τόξον, ἐστράφησαν ἐν ἡμέρᾳ πολέμου.
\vs{10}Οὐκ ἐφύλαξαν τὴν διαθήκην τοῦ Θεοῦ, καὶ ἐν τῷ νόμῳ αὐτοῦ οὐκ ἤθελον πορεύεσθαι.
\vs{11}Καὶ ἐπελάθοντο τῶν εὐεργεσιῶν αὐτοῦ καὶ τῶν θαυμασίων αὐτοῦ, ὧν ἔδειξεν αὐτοῖς·
\vs{12}ἐναντίον τῶν πατέρων αὐτῶν ἃ ἐποίησε θαυμάσια, ἐν γῇ Αἰγύπτῳ, ἐν πεδίῳ Τάνεως.
\vs{13}Διέῤῥηξε θάλασσαν, καὶ διήγαγεν αὐτοὺς· ἔστησεν ὕδατα ὡσεὶ ἀσκόν.
\vs{14}Καὶ ὡδήγησεν αὐτοὺς ἐν νεφέλῃ ἡμέρας, καὶ ὅλην τὴν νύκτα ἐν φωτισμῷ πυρός.
\vs{15}Διέῤῥηξε πέτραν ἐν ἐρήμῳ, καὶ ἐπότισεν αὐτοὺς ὡς ἐν ἀβύσσῳ πολλῇ.
\vs{16}Καὶ ἐξήγαγεν ὕδωρ ἐκ πέτρας, καὶ κατήγαγεν ὡς ποταμοὺς ὕδατα.

\vs{17}Καὶ προσέθεντο ἔτι τοῦ ἁμαρτάνειν αὐτῷ· παρεπικραναν τὸν ὕψιστον ἐν ἀνύδρῳ,
\vs{18}καὶ ἐξεπείρασαν τὸν Θεὸν ἐν ταῖς καρδίαις αὐτῶν, τοῦ αἰτῆσαι βρώματα ταῖς ψυχαῖς αὐτῶν.
\vs{19}Καὶ κατελάλησαν τοῦ Θεοῦ, καὶ εἶπαν, μὴ δυνήσεται ὁ Θεὸς ἑτοιμάσαι τράπεζαν ἐν ἐρήμῳ;
\vs{20}Ἐπεὶ ἐπάταξε πέτραν, καὶ ἐῤῥύησαν ὕδατα, καὶ χείμαῤῥοι κατεκλύσθησαν· μὴ καὶ ἄρτον δυνήσεται δοῦναι; ἢ ἑτοιμάσαι τράπεζαν τῷ λαῷ αὐτοῦ;

\vs{21}Διὰ τοῦτο ἤκουσε Κύριος καὶ ἀνεβάλετο, καὶ πῦρ ἀνήφθη ἐν Ἰακὼβ, καὶ ὀργὴ ἀνέβη ἐπὶ τὸν Ἰσραήλ.
\vs{22}Ὅτι οὐκ ἐπίστευσαν ἐν τῷ Θεῷ, οὐδὲ ἤλπισαν ἐπὶ τὸ σωτήριον αὐτοῦ.
\vs{23}Καὶ ἐνετείλατο νεφέλαις ὑπεράνωθεν, καὶ θύρας οὐρανοῦ ἀνέῳξε·
\vs{24}καὶ ἔβρεξεν αὐτοῖς μάννα φαγεῖν, καὶ ἄρτον οὐρανοῦ ἔδωκεν αὐτοῖς.
\vs{25}Ἄρτον ἀγγέλων ἔφαγεν ἄνθρωπος, ἐπισιτισμὸν ἀπέστειλεν αὐτοῖς εἰς πλησμονήν.

\vs{26}Ἀπῇρε Νότον ἐξ οὐρανοῦ, καὶ ἐπήγαγεν ἐν τῇ δυναστείᾳ
\vs{27}αὐτοῦ Λίβα. Καὶ ἔβρεξεν ἐπ αὐτοὺς ὡσεὶ χοῦν σάρκας, καὶ ὡσεῖ ἄμμον θαλασσῶν πετεινὰ πτερωτά.
\vs{28}Καὶ ἐπέπεσον εἰς μέσον τῆς παρεμβολῆς αὐτῶν, κύκλῳ τῶν σκηνωμάτων αὐτῶν.
\vs{29}Καὶ ἐφάγοσαν καὶ ἐνεπλήσθησαν σφόδρα, καὶ τὴν ἐπιθυμίαν αὐτῶν ἤνεγκεν αὐτοῖς.

\vs{30}Οὐκ ἐστερήθησαν ἀπὸ τῆς ἐπιθυμίας αὐτῶν· ἔτι τῆς βρώσεως
\vs{31}αὐτῶν οὔσης ἐν τῷ στόματι αὐτῶν, καὶ ὀργὴ τοῦ Θεοῦ ἀνέβη ἐπʼ αὐτοὺς, καὶ ἀπέκτεινεν ἐν τοῖς πίοσιν αὐτῶν, καὶ τοὺς ἐκλεκτοὺς τοῦ Ἰσραὴλ συνεπόδισεν.

\vs{32}ʼΕν πᾶσι τούτοις ἥμαρτον ἔτι, καὶ οὐκ ἐπίστευσαν τοῖς
\vs{33}θαυμασίοις αὐτοῦ. Καὶ ἐξέλιπον ἐν ματαιότητι αἱ ἡμέραι αὐτῶν, καὶ τὰ ἔτη αὐτῶν μετὰ σπουδῆς.

\vs{34}Ὅταν ἀπέκτεινεν αὐτοὺς, ἐζήτουν αὐτὸν, καὶ ἐπέστρεφον καὶ
\vs{35}ὤρθριζον πρὸς τὸν Θεόν. Καὶ ἐμνήσθησαν ὅτι ὁ Θεὸς βοηθὸς αὐτῶν ἐστι, καὶ ὁ Θεὸς ὁ ὕψιστος λυτρωτὴς αὐτῶν ἐστι.
\vs{36}Καὶ ἠγάπησαν αὐτὸν ἐν τῷ στόματι αὐτῶν, καὶ τῇ γλώσσῃ αὐτῶν
\vs{37}ἐψεύσαντο αὐτῷ· ἡ δὲ καρδία αὐτῶν οὐκ εὐθεῖα μετʼ αὐτοῦ, οὐδὲ ἐπιστώθησαν ἐν τῇ διαθήκῃ αὐτοῦ.

\vs{38}Αὐτὸς δέ ἐστιν οἰκτίρμων, καὶ ἱλάσεται ταῖς ἁμαρτίαις αὐτῶν, καὶ οὐ διαφθερεῖ· καὶ πληθυνεῖ τοῦ ἀποστρέψαι τὸν θυμὸν αὐτοῦ, καὶ οὐχὶ ἐκκαύσει πᾶσαν τὴν ὀργὴν αὐτοῦ.
\vs{39}Καὶ ἐμνήσθη ὅτι σάρξ εἰσι, πνεῦμα πορευόμενον καὶ οὐκ ἐπιστρέφον.

\vs{40}Ποσάκις παρεπίκραναν αὐτὸν ἐν τῇ ἐρήμῳ, παρώργισαν
\vs{41}αὐτὸν ἐν γῇ ἀνύδρῳ; Καὶ ἐπέστρεψαν καὶ ἐπείρασαν τὸν Θεὸν, καὶ τὸν ἅγιον τοῦ Ἰσραὴλ παρώξυναν.
\vs{42}Οὐκ ἐμνήσθησαν τῆς χειρὸς αὐτοῦ, ἡμέρας ἧς ἐλυτρώσατο αὐτοὺς ἐκ χειρὸς θλίβοντος·
\vs{43}Ὡς ἔθετο ἐν Αἰγύπτῳ τὰ σημεῖα αὐτοῦ, καὶ τὰ
\vs{44}τέρατα αὐτοῦ ἐν πεδίῳ Τάνεως· Καὶ μετέστρεψεν εἰς αἷμα τοὺς ποταμοὺς αὐτῶν, καὶ τὰ ὀμβρήματα αὐτῶν ὅπως μὴ πίωσιν·
\vs{45}ἐξαπέστειλεν εἰς αὐτοὺς κυνόμυιαν καὶ κατέφαγεν αὐτοὺς, καὶ βάτραχον, καὶ διέφθειρεν αὐτούς·
\vs{46}Καὶ ἔδωκε τῇ ἐρυσίβῃ τὸν καρπὸν αὐτῶν, καὶ τοὺς πόνους αὐτῶν τῇ ἀκρίδι.
\vs{47}Ἀπέκτεινεν ἐν χαλάζῃ τὴν ἄμπελον αὐτῶν, καὶ τὰς συκαμίνους αὐτῶν ἐν
\vs{48}τῇ πάχνῃ. Καὶ παρέδωκεν ἐν χαλάζῃ τὰ κτήνη αὐτῶν, καὶ τὴν ὕπαρξιν αὐτῶν τῷ πυρί.
\vs{49}Ἐξαπέστειλεν εἰς αὐτοὺς ὀργὴν θυμοῦ αὐτοῦ, θυμὸν καὶ ὀργὴν καὶ θλίψιν, ἀποστολὴν διʼ
\vs{50}ἀγγέλων πονηρῶν. Ὡδοποίησε τρίβον τῇ ὀργῇ αὐτοῦ, οὐκ ἐφείσατο ἀπὸ θανάτου τῶν ψυχῶν αὐτῶν, καὶ τὰ κτήνη αὐτῶν
\vs{51}εἰς θάνατον συνέκλεισε. Καὶ ἐπάταξε πᾶν πρωτότοκον ἐν γῇ Αἰγυπτῳ, ἀπαρχὴν πόνων αὐτῶν ἐν τοῖς σκηνώμασι Χάμ.
\vs{52}Καὶ ἀπῇρεν ὡς πρόβατα τὸν λαὸν αὐτοῦ, ἤγαγεν αὐτοὺς ὡσεὶ
\vs{53}ποίμνιον ἐν ἐρήμῳ. Καὶ ὡδήγησεν αὐτοὺς ἐν ἐλπίδι, καὶ οὐκ ἐδειλίασον, καὶ τοὺς ἐχθροὺς αὐτῶν ἐκάλυψε θάλασσα.
\vs{54}Καὶ εἰσήγαγεν αὐτοὺς εἰς ὄρος ἁγιάσματος αὐτοῦ, ὄρος τοῦτο ὃ
\vs{55}ἐκτήσατο ἡ δεξιὰ αὐτοῦ. Καὶ ἐξέβαλεν ἀπὸ προσώπου αὐτῶν ἔθνη, καὶ ἐκληροδότησεν αὐτοὺς ἐν σχοινίῳ κληροδοσίας, καὶ κατεσκήνωσεν ἐν τοῖς σκηνώμασιν αὐτῶν τὰς φυλὰς τοῦ Ἰσραήλ.

\vs{56}Καὶ ἐπείρασαν καὶ παρεπίκραναν τὸν Θεὸν τὸν ὕψιστον, καὶ τὰ μαρτύρια αὐτοῦ οὐκ ἐφυλάξαντο.
\vs{57}Καὶ ἀπέστρεψαν, καὶ ἠσυνθέτησαν καθὼς καὶ οἱ πατέρες αὐτῶν, μετεστράφησαν εἰς τόξον στρεβλόν.
\vs{58}Καὶ παρώργισαν αὐτὸν ἐπὶ τοῖς βουνοῖς αὐτῶν, καὶ ἐν τοῖς γλυπτοῖς αὐτῶν παρεζήλωσαν αὐτόν.

\vs{59}Ἤκουσεν ὁ Θεὸς καὶ ὑπερεῖδε, καὶ ἐξουδένωσε σφόδρα τὸν
\vs{60}ʼΙσραήλ. Καὶ ἀπώσατο τὴν σκηνὴν Σηλὼμ, σκήνωμα αὐτοῦ
\vs{61}οὗ κατεσκήνωσεν ἐν ἀνθρώποις. Καὶ παρέδωκεν εἰς αἰχμαλωσίαν τὴν ἰσχὺν αὐτῶν, καὶ τὴν καλλονὴν αὐτῶν εἰς χεῖρα
\vs{62}ἐχθρου. Καὶ συνέκλεισεν εἰς ῥομφαίαν τὸν λαὸν αὐτοῦ, καὶ τὴν κληρονομίαν αὐτοῦ ὑπερεῖδε·
\vs{63}Τοὺς νεανίσκους αὐτῶν κατέφαγε πῦρ, καὶ αἱ παρθένοι αὐτῶν οὐκ ἐπένθησαν. Οἱ
\vs{64}ἱερεῖς αὐτῶν ἐν ῥομφαίᾳ ἔπεσον, καὶ αἱ χῆραι αὐτῶν οὐ κλαυσθήσονται.

\vs{65}Καὶ ἐξηγέρθη ὡς ὁ ὑπνῶν Κύριος, ὡς δυνατὸς κεκραιπαληκὼς
\vs{66}ἐξ οἴνου. Καὶ ἐπάταξε τοὺς ἐχθροὺς αὐτοῦ εἰς τὰ ὀπίσω, ὄνειδος αἰώνιον ἔδωκεν αὐτοῖς.

\vs{67}Καὶ ἀπώσατο τὸ σκήνωμα Ἰωσὴφ, καὶ τὴν φυλὴν Ἐφραὶμ
\vs{68}οὐκ ἐξελέξατο. Καὶ ἐξελέξατο τὴν φυλὴν Ἰούδα, τὸ ὄρος τὸ
\vs{69}Σιὼν, ὃ ἠγάπησε. Καὶ ᾠκοδόμησεν ὡς μονοκερώτων τὸ ἁγίασμα αὐτοῦ, ἐν τῇ γῇ ἐθεμελίωσεν αὐτὴν εἰς τὸν αἰῶνα.
\vs{70}Καὶ ἐξελέξατο Δαυὶδ τὸν δοῦλον αὐτοῦ, καὶ ἀνέλαβεν αὐτὸν ἐκ τῶν
\vs{71}ποιμνίων τῶν προβάτων. Ἐξόπισθεν τῶν λοχευομένων ἔλαβεν αὐτὸν, ποιμαίνειν Ἰακὼβ τὸν δοῦλον αὐτοῦ, καὶ Ἰσραὴλ τὴν
\vs{72}κληρονομίαν αὐτοῦ. Καὶ ἐποίμανεν αὐτοὺς ἐν τῇ ἀκακίᾳ τῆς καρδίας αὐτοῦ, καὶ ἐν τῇ συνέσει τῶν χειρῶν αὐτοῦ, ὡδήγησεν αὐτούς.

\begin{psalmheading}{\ch{78}{79} Ψαλμὸς τῷ Ἀσάφ.}
\end{psalmheading}
Ὁ Θεὸς, ἤλθοσαν ἔθνη εἰς τὴν κληρονομίαν σου, ἐμίαναν τὸν ναὸν τὸν ἅγιόν σου· ἔθεντο Ἱερουσαλὴμ εἰς ὀπωροφυλάκιον.
\vs{2}Ἔθεντο τὰ θνησιμαῖα τῶν δούλων σου βρώματα τοῖς πετεινοῖς τοῦ οὐρανοῦ, τὰς σάρκας τῶν ὁσίων σου τοῖς θηρίοις τῆς γῆς.
\vs{3}Ἐξέχεαν τὸ αἷμα αὐτῶν ὡς ὕδωρ, κύκλῳ Ἱερουσαλὴμ, καὶ οὐκ ἦν ὁ θάπτων.
\vs{4}Ἐγενήθημεν εἰς ὄνειδος τοῖς γείτοσιν ἡμῶν, μυκτηρισμὸς καὶ χλευασμὸς τοῖς κύκλῳ ἡμῶν.

\vs{5}Ἕως πότε, Κύριε, ὀργισθήσῃ εἰς τέλος; ἐκκαυθήσεται ὡς πῦρ ὁ ζῆλός σου;
\vs{6}Ἔκχεον τὴν ὀργήν σου ἐπὶ ἔθνη τὰ μὴ ἐπεγνωκότα σε, καὶ ἐπὶ βασιλείας αἳ τὸ ὄνομά σου οὐκ ἐπεκαλέσαντο.
\vs{7}Ὅτι κατέφαγον τὸν Ἰακὼβ, καὶ τὸν τόπον αὐτοῦ ἠρήμωσαν.

\vs{8}Μὴ μνησθῇς ἡμῶν ἀνομιῶν ἀρχαίων, ταχὺ προκαταλαβέτωσαν ἡμᾶς οἱ οἰκτιρμοί σου, ὅτι ἐπτωχεύσαμεν σφόδρα.
\vs{9}Βοήθησον ἡμῖν ὁ Θεὸς ὁ σωτὴρ ἡμῶν, ἕνεκα τῆς δόξης τοῦ ὀνόματός σου Κύριε ῥῦσαι ἡμᾶς, καὶ ἱλάσθητι ταῖς ἁμαρτίαις ἡμῶν ἕνεκα τοῦ ὀνόματός σου·
\vs{10}Μή ποτε εἴπωσιν ἐν τοῖς ἔθνεσι, ποῦ ἐστιν ὁ Θεὸς αὐτῶν; καὶ γνωσθήτω ἐν τοῖς ἔθνεσιν ἐνώπιον τῶν ὀφθαλμῶν ἡμῶν ἡ ἐκδίκησις τοῦ αἵματος τῶν δούλων σου τοῦ ἐκκεχυμένου.

\vs{11}Εἰσελθέτω ἐνώπιόν σου ὁ στεναγμὸς τῶν πεπεδημένων, κατὰ τὴν μεγαλωσύνην τοῦ βραχίονός σου περιποίησαι τοὺς υἱοὺς τῶν τεθανατωμένων.
\vs{12}Ἀπόδος τοῖς γείτοσιν ἡμῶν ἑπταπλάσια εἰς τὸν κόλπον αὐτῶν τὸν ὀνειδισμὸν αὐτῶν, ὃν ὠνείδισάν σε Κύριε.
\vs{13}Ἡμεῖς γὰρ λαός σου καὶ πρόβατα νομῆς σου, ἀνθομολογησόμεθά σοι εἰς τὸν αἰῶνα, εἰς γενεὰν καὶ γενεὰν ἐξαγγελοῦμεν τὴν αἴνεσίν σου.

\begin{psalmheading}{\ch{79}{80} Εἰς τὸ τέλος, ὑπὲρ τῶν ἀλλοιωθησομένων, μαρτύριον τῷ Ἀσὰφ, ψαλμὸς ὑπὲρ τοῦ Ἀσσυρίου.}
\end{psalmheading}
\vs{2}Ὁ Ποιμαίνων τὸν Ἰσραὴλ πρόσχες, ὁ ὁδηγῶν ὡσεὶ πρόβατα τὸν Ἰωσήφ· ὁ καθήμενος ἐπὶ τῶν χερουβὶμ ἐμφάνηθι,
\vs{3}ἐναντίον Ἐφραὶμ καὶ Βενιαμὶν καὶ Μανασσῆ· ἐξέγειρον τὴν δυναστείαν σου καὶ ἐλθὲ εἰς τὸ σῶσαι ἡμᾶς.
\vs{4}Ὁ Θεὸς ἐπίστρεψον ἡμᾶς, καὶ ἐπίφανον τὸ πρόσωπόν σου, καὶ σωθησόμεθα.

\vs{5}Κύριε ὁ Θεὸς τῶν δυνάμεων, ἕως πότε ὀργίζῃ ἐπὶ τὴν προσευχὴν τοῦ δούλου σου;
\vs{6}Ψωμιεῖς ἡμᾶς ἄρτον δακρύων, καὶ ποτιεῖς ἡμᾶς ἐν δάκρυσιν ἐν μέτρῳ.
\vs{7}Ἔθου ἡμᾶς εἰς ἀντιλογίαν τοῖς γείτοσιν ἡμῶν, καὶ οἱ ἐχθροὶ ἡμῶν ἐμυκτήρισαν ἡμᾶς.
\vs{8}Κύριε ὁ Θεὸς τῶν δυνάμεων ἐπίστρεψον ἡμᾶς, καὶ ἐπίφανον τὸ πρόσωπόν σου, καὶ σωθησόμεθα· διάψαλμα.

\vs{9}Ἄμπελον ἐξ Αἰγύπτου μετῇρας, ἐξέβαλες ἔθνη καὶ κατεφύτευσας αὐτήν.
\vs{10}Ὡδοποίησας ἔμπροσθεν αὐτῆς, καὶ κατεφύτευσας τὰς ῥίζας αὐτῆς, καὶ ἐπλήσθη ἡ γῆ.
\vs{11}Ἐκάλυψεν ὄρη ἡ σκιὰ αὐτῆς, καὶ αἱ ἀναδενδράδες αὐτῆς τὰς κέδρους τοῦ Θεοῦ.
\vs{12}Ἐξέτεινε τὰ κλήματα αὐτῆς ἕως θαλάσσης, καὶ ἕως ποταμοῦ τὰς παραφυάδας αὐτῆς.
\vs{13}Ἱνατί καθεῖλες τὸν φραγμὸν αὐτῆς, καὶ τρυγῶσιν αὐτὴν πάντες οἱ παραπορευόμενοι τὴν ὁδόν;
\vs{14}Ἐλυμῄνατο αὐτὴν σῦς ἐκ δρυμοῦ, καὶ μονιὸς ἄγριος κατενεμήσατο αὐτήν.

\vs{15}Ὁ Θεὸς τῶν δυνάμεων ἐπίστρεψον δὴ, ἐπίβλεψον ἐξ οὐρανοῦ καὶ ἴδε, καὶ ἐπίσκεψαι τὴν ἄμπελον ταύτην·
\vs{16}Καὶ κατάρτισαι αὐτὴν, ἣν ἐφύτευσεν ἡ δεξιά σου, καὶ ἐπὶ υἱὸν ἀνθρώπου ὃν ἐκραταίωσας σεαυτῷ.
\vs{17}Ἐμπεπυρισμένη πυρὶ καὶ ἀνεσκαμμένη· ἀπὸ ἐπιτιμήσεως τοῦ προσώπου σου ἀπολοῦνται.
\vs{18}Γενηθήτω ἡ χείρ σου ἐπʼ ἄνδρα δεξιᾶς σου, καὶ ἐπὶ υἱὸν ἀνθρώπου, ὃν ἐκραταίωσας σεαυτῷ.

\vs{19}Καὶ οὐ μὴ ἀποστῶμεν ἀπὸ σοῦ, ζωώσεις ἡμᾶς, καὶ τὸ ὄνομά σου ἐπικαλεσόμεθα.
\vs{20}Κύριε ὁ Θεὸς τῶν δυνάμεων ἐπίστρεψον ἡμᾶς, καὶ ἐπίφανον τὸ πρόσωπόν σου, καὶ σωθησόμεθα.

\begin{psalmheading}{\ch{80}{81} Εἰς τὸ τέλος, ὑπὲρ τῶν ληνῶν ψαλμὸς τῷ Ἀσάφ.}
\end{psalmheading}
\vs{2}Ἀγαλλιᾶσθε τῷ Θεῷ τῷ βοηθῷ ἡμῶν, ἀλαλάξατε τῷ Θεῷ Ἰακώβ.
\vs{3}Λάβετε ψαλμὸν καὶ δότε τύμπανον, ψαλτήριον τερπνὸν μετὰ κιθάρας.
\vs{4}Σαλπίσατε ἐν νεομηνίᾳ σάλπιγγι, ἐν εὐσήμῳ ἡμέρᾳ ἑορτῆς ὑμῶν.

\vs{5}Ὅτι πρόσταγμα τῷ Ἰσραήλ ἐστι, καὶ κρίμα τῷ Θεῷ Ἰακώβ.
\vs{6}Μαρτύριον ἐν τῷ Ἰωσὴφ ἔθετο αὐτὸν, ἐν τῷ ἐξελθεῖν αὐτὸν ἐκ γῆς Αἰγύπτου· γλῶσσαν ἣν οὐκ ἔγνω, ἤκουσεν.

\vs{7}Ἀπέστησεν ἀπὸ ἄρσεων τὸν νῶτον αὐτοῦ· αἱ χεῖρες αὐτοῦ ἐν τῷ κοφίνῳ ἐδούλευσαν.
\vs{8}Ἐν θλίψει ἐπεκαλέσω με καὶ ἐῤῥυσάμην σε· ἐπήκουσά σου ἐν ἀποκρύφῳ καταιγίδος, ἐδοκίμασά σε ἐπὶ ὕδατος ἀντιλογίας· διάψαλμα.
\vs{9}Ἄκουσον λαός μου καὶ λαλήσω σοι, Ἰσραὴλ, καὶ διαμαρτύρομαί σοι· ἐὰν ἀκούσῃς μου,
\vs{10}οὐκ ἔσται ἐν σοὶ θεὸς πρόσφατος, οὐδὲ προσκυνήσεις θεῷ ἀλλοτρίῳ.
\vs{11}Ἐγὼ γάρ εἰμι Κύριος ὁ Θεός σου, ὁ ἀναγαγών σε ἐκ γῆς Αἰγύπτου, πλάτυνον τὸ στόμα σου καὶ πληρώσω αὐτό.
\vs{12}Καὶ οὐκ ἤκουσεν ὁ λαός μου τῆς φωνῆς μου, καὶ Ἰσραὴλ οὐ προσέσχε μοι.
\vs{13}Καὶ ἐξαπέστειλα αὐτοὺς κατὰ τὰ ἐπιτηδεύματα τῶν καρδιῶν αὐτῶν, πορεύσονται ἐν τοῖς ἐπιτηδεύμασιν αὐτῶν.

\vs{14}Εἰ ὁ λαός μου ἤκουσέ μου, Ἰσραὴλ ταῖς ὁδοῖς μου εἰ ἐπορεύθη,
\vs{15}ἐν τῷ μηδενὶ ἂν τοὺς ἐχθροὺς αὐτῶν ἐταπείνωσα, καὶ ἐπὶ τοὺς θλίβοντας αὐτοὺς ἐπέβαλον ἂν τὴν χεῖρά μου.
\vs{16}Οἱ ἐχθροὶ Κυρίου ἐψεύσαντο αὐτῷ, καὶ ἔσται ὁ καιρὸς αὐτῶν εἰς τὸν αἰῶνα,
\vs{17}καὶ ἐψώμισεν αὐτοὺς ἐκ στέατος πυροῦ, καὶ ἐκ πέτρας μέλι ἐχόρτασεν αὐτούς.

\begin{psalmheading}{\ch{81}{82} Ψαλμὸς τῷ Ἀσάφ.}
\end{psalmheading}
\vs{2}Ὁ Θεὸς ἔστη ἐν συναγωγῇ θεῶν, ἐν μέσῳ δὲ θεοὺς διακρινεῖ.
\vs{2}Ἕως πότε κρίνετε ἀδικίαν, καὶ πρόσωπα ἁμαρτωλῶν λαμβάνετε; διάψαλμα.
\vs{3}Κρίνατε ὀρφανὸν καὶ πτωχὸν, ταπεινὸν καὶ πένητα δικαιώσατε.
\vs{4}Εξέλεσθε πένητα, καὶ πτωχὸν ἐκ χειρὸς ἁμαρτωλοῦ ῥύσασθε.

\vs{5}Οὐκ ἔγνωσαν οὐδὲ συνῆκαν, ἐν σκότει διαπορεύονται· σαλευθήσονται πάντα τὰ θεμελια τῆς γῆς.
\vs{6}Ἐγὼ εἶπα, θεοί ἐστε, καὶ υἱοὶ ὑψίστου πάντες.
\vs{7}Ὑμεῖς δὲ ὡς ἄνθρωποι ἀποθνήσκετε, καὶ ὡς εἷς τῶν ἀρχόντων πίπτετε.

\vs{8}Ἀνάστα ὁ Θεὸς, κρῖνον τὴν γῆν, ὅτι σὺ κατακληρονομήσεις ἐν πᾶσι τοῖς ἔθνεσιν.

\begin{psalmheading}{\ch{82}{83} Ὠδὴ ψαλμοῦ τῷ Ἀσάφ.}
\end{psalmheading}
\vs{2}Ὁ Θεὸς, τίς ὁμοιωθήσεταί σοι; μὴ σιγήσῃς, μηδὲ καταπραΰνῃς ὁ Θεός.

\vs{3}Ὅτι ἰδοὺ οἱ ἐχθροί σου ἤχησαν· καὶ οἱ μισοῦντές σε ᾖραν κεφαλήν.
\vs{4}Ἐπὶ τὸν λαόν σου κατεπανουργεύσαντο γνώμην, καὶ ἐβουλεύσαντο κατὰ τῶν ἁγίων σου.
\vs{5}Εἶπαν, δεῦτε καὶ ἐξολοθρεύσωμεν αὐτοὺς ἐξ ἔθνους, καὶ οὐ μὴ μνησθῇ τὸ ὄνομα Ἰσραὴλ ἔτι.
\vs{6}Ὅτι ἐβουλεύσαντο ἐν ὁμονοίᾳ ἐπιτοαυτὸ, κατὰ σοῦ διαθήκην διέθεντο·
\vs{7}Τὰ σκηνώματα τῶν Ἰδουμαίων καὶ οἱ Ἰσμαηλῖται, Μωὰβ καὶ οἱ Ἀγαρηνοὶ,
\vs{8}Γεβὰλ καὶ Ἀμμὼν καὶ Ἀμαλὴκ, καὶ ἀλλόφυλοι μετὰ τῶν κατοικούντων Τύρον.
\vs{9}Καὶ γὰρ καὶ Ἀσσοὺρ συμπαρεγένετο μετʼ αὐτῶν, ἐγενήθησαν εἰς ἀντίληψιν τοῖς υἱοῖς Λώτ· διάψαλμα.

\vs{10}Ποίησον αὐτοῖς ὡς τῇ Μαδιὰμ καὶ τῷ Σεισάρᾳ, ὡς τῷ Ἰαβεὶν ἐν τῷ χειμάῤῥῳ Κεισῶν.
\vs{11}Ἐξωλοθρεύθησαν ἐν Ἀενδὼρ, ἐγενήθησαν ὡσεὶ κόπρος τῇ γῇ.
\vs{12}Θοῦ τοὺς ἄρχοντας αὐτῶν ὡς τὸν Ὠρὴβ καὶ Ζὴβ καὶ Ζεβεὲ καὶ Σαλμανὰ, πάντας τοὺς ἄρχοντας αὐτῶν·
\vs{13}Οἵτινες εἶπαν, κληρονομήσωμεν ἑαυτοῖς τὸ θυσιαστήριον τοῦ Θεοῦ.
\vs{14}Ὁ Θεός μου θοῦ αὐτοὺς ὡς τροχὸν, ὡς καλάμην κατὰ πρόσωπον ἀνέμου.
\vs{15}Ὡσεὶ πῦρ ὃ διαφλέξει δρυμὸν, ὡσεὶ φλὸξ κατακαύσαι ὄρη·
\vs{16}Οὕτως καταδιώξεις αὐτοὺς ἐν τῇ καταιγίδι σου, καὶ ἐν τῇ ὀργῇ σου ταράξεις αὐτούς.
\vs{17}Πλήρωσον τὰ πρόσωπα αὐτῶν ἀτιμίας, καὶ ζητήσουσι τὸ ὄνομά σου Κύριε.
\vs{18}Αἰσχυνθήτωσαν καὶ ταραχθήτωσαν εἰς τὸν αἰῶνα τοῦ αἰῶνος, καὶ ἐντραπήτωσαν καὶ ἀπολέσθωσαν.
\vs{19}Καὶ γνώτωσαν ὅτι ὄνομά σοι Κύριος· σὺ μόνος ὕψιστος ἐπὶ πᾶσαν τὴν γῆν.

\begin{psalmheading}{\ch{83}{84} Εἰς τὸ τέλος, ὑπὲρ τῶν ληνῶν τοῖς υἱοῖς Κορὲ ψαλμός.}
\end{psalmheading}
\vs{2}Ὡς ἀγαπητὰ τὰ σκηνώματά σου Κύριε τῶν δυνάμεων.
\vs{3}Ἐπιποθεῖ καὶ ἐκλείπει ἡ ψυχή μου εἰς τὰς αὐλὰς τοῦ Κυρίου· ἡ καρδία μου καὶ ἡ σάρξ μου ἠγαλλιάσαντο ἐπὶ Θεὸν ζῶντα·
\vs{4}Καὶ γὰρ στρουθίον εὗρεν ἑαυτῷ οἰκίαν, καὶ τρυγὼν νοσσιὰν ἑαυτῇ, οὗ θήσει τὰ νοσσία ἑαυτῆς· τὰ θυσιαστήριά σου Κύριε τῶν δυνάμεων, ὁ βασιλεύς μου καὶ ὁ Θεός μου.

\vs{5}Μακάριοι οἱ κατοικοῦντες ἐν τῷ οἴκῳ σου, εἰς τοὺς αἰῶνας τῶν αἰώνων αἰνέσουσί σε· διάψαλμα.
\vs{6}Μακάριος ἀνὴρ οὗ ἐστιν ἡ ἀντίληψις αὐτοῦ παρὰ σοῦ, Κύριε· ἀναβάσεις ἐν τῇ καρδίᾳ αὐτοῦ διέθετο,
\vs{7}εἰς τὴν κοιλάδα τοῦ κλαυθμῶνος, εἰς τὸν τόπον ὃν ἔθετο· καὶ γὰρ εὐλογίας δώσει ὁ νομοθετῶν,
\vs{8}πορεύσονται ἐκ δυνάμεως εἰς δύναμιν, ὀφθήσεται ὁ Θεὸς τῶν θεῶν ἐν Σιών.

\vs{9}Κύριε ὁ Θεὸς τῶν δυνάμεων, εἰσάκουσον τῆς προσευχῆς μου, ἐνώτισαι ὁ Θεὸς Ἰακώβ· διάψαλμα.
\vs{10}Ὑπερασπιστὰ ἡμῶν ἴδε ὁ Θεὸς, καὶ ἐπίβλεψον ἐπὶ τὸ πρόσωπον τοῦ χριστοῦ σοῦ.
\vs{11}Ὅτι κρείσσων ἡμέρα μία ἐν ταῖς αὐλαῖς σου, ὑπὲρ χιλιάδας· ἐξελεξάμην παραῤῥιπτεῖσθαι ἐν τῷ οἴκῳ τοῦ Θεοῦ μᾶλλον ἢ οἰκεῖν με ἐπὶ σκηνώμασιν ἁμαρτωλῶν.
\vs{12}Ὅτι ἔλεον καὶ ἀλήθειαν ἀγαπᾷ Κύριος, ὁ Θεὸς χάριν καὶ δόξαν δώσει· Κύριος οὐχ ὑστερήσει τὰ ἀγαθὰ τοῖς πορευομένοις ἐν ἀκακίᾳ.
\vs{13}Κύριε τῶν δυνάμεων, μακάριος ἄνθρωπος ὁ ἐλπίζων ἐπὶ σέ.

\begin{psalmheading}{\ch{84}{85} Εἰς τὸ τέλος, τοῖς υἱοῖς Κορὲ ψαλμός.}
\end{psalmheading}
\vs{2}Εὐδόκησας Κύριε τὴν γῆν σου, ἀπέστρεψας τὴν αἰχμαλωσίαν Ἰακώβ.
\vs{3}Ἀφῆκας τὰς ἀνομίας τῷ λαῷ σου, ἐκάλυψας πάσας τὰς ἁμαρτίας αὐτῶν· διάψαλμα.
\vs{4}Κατέπαυσας πᾶσαν τὴν ὀργήν σου, ἀπέστρεψας ἀπὸ ὀργῆς θυμοῦ σου.

\vs{5}Ἐπίστρεψον ἡμᾶς ὁ Θεὸς τῶν σωτηρίων ἡμῶν, καὶ ἀπόστρεψον τὸν θυμόν σου ἀφʼ ἡμῶν.
\vs{6}Μὴ εἰς τὸν αἰῶνα ὀργισθῇς ἡμῖν; ἢ διατενεῖς τὴν ὀργήν σου ἀπὸ γενεᾶς εἰς γενεάν;
\vs{7}Ὁ Θεὸς, σὺ ἐπιστρέψας ζωώσεις ἡμᾶς, καὶ ὁ λαός σου εὐφρανθήσεται ἐπὶ σοί.
\vs{8}Δεῖξον ἡμῖν Κύριε τὸ ἔλεός σου, καὶ τὸ σωτήριόν σου δῴης ἡμῖν.

\vs{9}Ἀκούσομαι τί λαλήσει ἐν ἐμοὶ Κύριος ὁ Θεὸς, ὅτι λαλήσει εἰρήνην ἐπὶ τὸν λαὸν αὐτοῦ, καὶ ἐπὶ τοὺς ὁσίους αὐτοῦ, καὶ ἐπὶ τοὺς ἐπιστρέφοντας πρὸς αὐτὸν καρδίαν.
\vs{10}Πλὴν ἐγγὺς τῶν φοβουμένων αὐτὸν τὸ σωτήριον αὐτοῦ, τοῦ κατασκηνῶσαι δόξαν ἐν τῇ γῇ ἡμῶν.
\vs{11}Ἔλεος καὶ ἀλήθεια συνήντησαν, δικαιοσύνη καὶ εἰρήνη κατεφίλησαν.
\vs{12}Ἀλήθεια ἐκ τῆς γῆς ἀνέτειλε, καὶ δικαιοσύνη ἐκ τοῦ οὐρανοῦ διέκυψε.
\vs{13}Καὶ γὰρ ὁ Κύριος δώσει χρηστότητα, καὶ ἡ γῆ ἡμῶν δώσει τὸν καρπὸν αὐτῆς.
\vs{14}Δικαιοσύνη ἐναντίον αὐτοῦ προπορεύσεται, καὶ θήσει εἰς ὁδὸν τὰ διαβήματα αὐτοῦ.

\begin{psalmheading}{\ch{85}{86} Προσευχὴ τῷ Δαυίδ.}
\end{psalmheading}
\vs{2}Κλίνον Κύριε τὸ οὖς σου, καὶ εἰσάκουσόν μου, ὅτι πτωχὸς καὶ πένης εἰμὶ ἐγώ.
\vs{2}Φύλαξον τὴν ψυχήν μου, ὅτι ὅσιός εἰμι· σῶσον τὸν δοῦλόν σου ὁ Θεὸς, τὸν ἐλπίζοντα ἐπὶ σέ.
\vs{3}Ἐλέησόν με Κύριε, ὅτι πρὸς σὲ κεκράξομαι ὅλην τὴν ἡμέραν.
\vs{4}Εὔφρανον τὴν ψυχὴν τοῦ δούλου σου, ὅτι πρὸς σὲ Κύριε ᾖρα τὴν ψυχήν μου.
\vs{5}Ὅτι σὺ Κύριε χρηστὸς καὶ ἐπιεικὴς, καὶ πολυέλεος πᾶσι τοῖς ἐπικαλουμένοις σε.
\vs{6}Ἐνώτισαι Κύριε τὴν προσευχήν μου, καὶ πρόσχες τῇ φωνῇ τῆς δεήσεώς μου.
\vs{7}Ἐν ἡμέρᾳ θλίψεώς μου ἐκέκραξα πρὸς σὲ, ὅτι εἰσήκουσάς μου.

\vs{8}Οὐκ ἔστιν ὅμοιός σοι ἐν θεοῖς Κύριε, καὶ οὐκ ἔστι κατὰ τὰ ἔργα σου.
\vs{9}Πάντα τὰ ἔθνη ὅσα ἐποίησας ἥξουσι, καὶ προσκυνήσουσιν ἐνώπιόν σου Κύριε, καὶ δοξάσουσι τὸ ὄνομά σου,
\vs{10}ὅτι μέγας εἶ σὺ, καὶ ποιῶν θαυμάσια, σὺ εἶ ὁ Θεὸς μόνος ὁ μέγας.
\vs{11}Ὁδήγησόν με Κύριε ἐν τῇ ὁδῷ σου, καὶ πορεύσομαι ἐν τῇ ἀληθείᾳ σου· εὐφρανθήτω ἡ καρδία μου, τοῦ φοβεῖσθαι τὸ ὄνομά σου.
\vs{12}Ἐξομολογήσομαί σοι Κύριε ὁ Θεός μου ἐν ὅλῃ καρδίᾳ μου, καὶ δοξάσω τὸ ὄνομά σου εἰς τὸν αἰῶνα.
\vs{13}Ὅτι τὸ ἔλεός σου μέγα ἐπʼ ἐμὲ, καὶ ἐῤῥύσω τὴν ψυχήν μου ἐξ ᾅδου κατωτάτου.

\vs{14}Ὁ Θεὸς, παράνομοι ἐπανέστησαν ἐπʼ ἐμὲ, καὶ συναγωγὴ κραταιῶν ἐζήτησαν τὴν ψυχήν μου, καὶ οὐ προέθεντό σε ἐνώπιον αὐτῶν.
\vs{15}Καὶ σὺ Κύριε ὁ Θεὸς οἰκτίρμων καὶ ἐλεήμων, μακρόθυμος καὶ πολυέλεος καὶ ἀληθινός.
\vs{16}Ἐπίβλεψον ἐπʼ ἐμὲ, καὶ ἐλέησόν με, δὸς τὸ κράτος σου τῷ παιδί σου, καὶ σῶσον τὸν υἱὸν τῆς παιδίσκης σου.
\vs{17}Ποίησον μετʼ ἐμοῦ σημεῖον εἰς ἀγαθὸν, καὶ ἰδέτωσαν οἱ μισοῦντές με, καὶ αἰσχυνθήτωσαν· ὅτι σὺ Κύριε, ἐβοήθησάς μοι, καὶ παρεκάλεσάς με.

\begin{psalmheading}{\ch{86}{87} Τοῖς υἱοῖς Κορὲ ψαλμὸς ᾠδῆς.}
\end{psalmheading}
Οἱ θεμέλιοι αὐτοῦ ἐν τοῖς ὄρεσι τοῖς ἁγίοις.
\vs{2}Ἀγαπᾷ Κύριος τὰς πύλας Σιὼν, ὑπὲρ πάντα τὰ σκηνώματα Ἰακώβ.
\vs{3}Δεδοξασμένα ἐλαλήθη περὶ σοῦ ἡ πόλις τοῦ Θεοῦ· διάψαλμα.

\vs{4}Μνησθήσομαι Ῥαὰβ, καὶ Βαβυλῶνος, τοῖς γινώσκουσί με· καὶ ἰδοὺ ἀλλόφυλοι καὶ Τύρος καὶ λαὸς Αἰθιόπων, οὗτοι ἐγεννήθησαν ἐκεῖ.
\vs{5}Μήτηρ Σιὼν ἐρεῖ ἄνθρωπος, καὶ ἄνθρωπος ἐγενήθη ἐν αὐτῇ, καὶ αὐτὸς ἐθεμελίωσεν αὐτὴν ὁ ὕψιστος.
\vs{6}Κύριος διηγήσεται ἐν γραφῇ λαῶν, καὶ ἀρχόντων τούτων τῶν γεγενημένων ἐν αὐτῇ. διάψαλμα.
\vs{7}Ὡς εὐφραινομένων πάντων ἡ κατοικία ἐν σοί.

\begin{psalmheading}{\ch{87}{88} Ὠδὴ ψαλμοῦ τοῖς υἱοῖς Κορὲ, εἰς τὸ τέλος, ὑπὲρ μαελὲθ τοῦ ἀποκριθῆναι, συνέσεως Αἰμὰν τῷ Ἰσραηλίτῃ.}
\end{psalmheading}
\vs{2}Κύριε ὁ Θεὸς τῆς σωτηρίας μου, ἡμέρας ἐκέκραξα καὶ ἐν νυκτὶ ἐναντίον σου.
\vs{3}Εἰσελθέτω ἐνώπιόν σου ἡ προσευχή μου, κλῖνον τὸ οὖς σου εἰς τὴν δέησίν μου, Κύριε.

\vs{4}Ὅτι ἐπλήσθη κακῶν ἡ ψυχή μου, καὶ ἡ ζωή μου τῷ ᾅδῃ ἤγγισε.
\vs{5}Προσελογίσθην μετὰ τῶν καταβαινόντων εἰς λάκκον, ἐγενήθην ὡς ἄνθρωπος ἀβοήθητος,
\vs{6}ἐν νεκροῖς ἐλεύθερος, ὡσεὶ τραυματίαι ἐῤῥιμμένοι καθεύδοντες ἐν τάφῳ, ὧν οὐκ ἐμνήσθης ἔτι, καὶ αὐτοὶ ἐκ τῆς χειρός σου ἀπώσθησαν.
\vs{7}Ἔθεντό με ἐν λάκκῳ κατωτάτῳ, ἐν σκοτεινοῖς καὶ ἐν σκιᾷ θανάτου.
\vs{8}Ἐπʼ ἐμὲ ἐπεστηρίχθη ὁ θυμός σου, καὶ πάντας τοὺς μετεωρισμούς σου ἐπήγαγες ἐπʼ ἐμέ· διάψαλμα.
\vs{9}Ἐμάκρυνας τοὺς γνωστούς μου ἀπʼ ἐμοῦ, ἔθεντό με βδέλυγμα ἑαυτοῖς· παρεδόθην καὶ οὐκ ἐξεπορευόμην.
\vs{10}Οἱ ὀφθαλμοί μου ἠσθένησαυ ἀπὸ πτωχείας· καὶ ἐκέκραξα πρὸς σὲ Κύριε ὅλην τὴν ἡμέραν, διεπέτασα πρὸς σὲ τὰς χεῖράς μου.

\vs{11}Μὴ τοῖς νεκροῖς ποιήσεις θαυμάσια, ἢ ἰατροὶ ἀναστήσουσι καὶ ἐξομολογήσονταί σοι;
\vs{12}Μὴ διηγήσεταί τις ἐν τάφῳ τὸ ἔλεός σου, καὶ τὴν ἀλήθειάν σου ἐν τῇ ἀπωλείᾳ;
\vs{13}Μὴ γνωσθήσεται ἐν τῷ σκότει τὰ θαυμάσιά σου, καὶ ἡ δικαιοσύνη σου ἐν γῇ ἐπιλελησμένῃ;
\vs{14}Κᾀγὼ πρὸς σὲ Κύριε ἐκέκραξα, καὶ τοπρωῒ ἡ προσευχή μου προφθάσει σε.

\vs{15}Ἱνατί Κύριε ἀπωθεῖς τὴν προσευχήν μου, ἀποστρέφεις τὸ πρόσωπόν σου ἀπʼ ἐμοῦ;
\vs{16}Πτωχός εἰμι ἐγὼ καὶ ἐν κόποις ἐκ νεότητός μου, ὑψωθεὶς δὲ ἐταπεινώθην καὶ ἐξηπορήθην.
\vs{17}Ἐπʼ ἐμὲ διῆλθον αἱ ὀργαί σου, καὶ οἱ φοβερισμοί σου ἐξετάραξάν με.
\vs{18}Ἐκύκλωσάν με ὡς ὕδωρ, ὅλην τὴν ἡμέραν περιέσχον με ἅμα.
\vs{19}Ἐμάκρυνας ἀπʼ ἐμοῦ φίλον, καὶ τοὺς γνωστούς μου ἀπὸ ταλαιπωρίας.

\begin{psalmheading}{\ch{88}{89} Συνέσεως Αἰθὰμ τῷ Ἰσραηλίτῃ.}
\end{psalmheading}
\vs{2}Τὰ ἐλέη σου Κύριε εἰς τὸν αἰῶνα ᾄσομαι, εἰς γενεὰν καὶ γενεὰν ἀπαγγελῶ τὴν ἀλήθειάν σου ἐν τῷ στόματί μου.
\vs{3}Ὅτι εἶπας, εἰς τὸν αἰῶνα ἔλεος οἰκοδομηθήσεται, ἐν τοῖς οὐρανοῖς ἑτοιμασθήσεται ἡ ἀλήθειά σου·
\vs{4}Διεθέμην διαθήκην τοῖς ἐκλεκτοῖς μου, ὤμοσα Δαυὶδ τῷ δούλῳ μου.
\vs{5}Ἕως τοῦ αἰῶνος ἑτοιμάσω τὸ σπέρμα σου, καὶ οἰκοδομήσω εἰς γενεὰν καὶ γενεὰν τὸν θρόνον σου· διάψαλμα.

\vs{6}Ἐξομολογήσονται οἱ οὐρανοὶ τὰ θαυμάσιά σου Κύριε, καὶ τὴν ἀλήθειάν σου ἐν ἐκκλησίᾳ ἁγίων.
\vs{7}Ὅτι τίς ἐν νεφέλαις ἰσωθήσεται τῷ Κυρίῳ; καὶ τίς ὁμοιωθήσεται τῷ Κυρίῳ ἐν υἱοῖς Θεοῦ;
\vs{8}Ὁ Θεὸς ἐνδοξαζόμενος ἐν βουλῇ ἁγίων, μέγας καὶ φοβερὸς ἐπὶ πάντας τοὺς περικύκλῳ αὐτοῦ.
\vs{9}Κύριε ὁ Θεὸς τῶν δυνάμεων, τίς ὅμοιός σοι; δυνατὸς εἶ Κύριε, καὶ ἡ ἀλήθειά σου κύκλῳ σου.
\vs{10}Σὺ δεσπόζεις τοῦ κράτους τῆς θαλάσσης, τὸν δὲ σάλον τῶν κυμάτων αὐτῆς σὺ καταπραΰνεις.
\vs{11}Σὺ ἐταπείνωσας ὡς τραυματίαν ὑπερήφανον, καὶ ἐν τῷ βραχίονι τῆς δυνάμεώς σου διεσκόρπισας τοὺς ἐχθρούς σου.
\vs{12}Σοί εἰσιν οἱ οὐρανοὶ, καὶ σή ἐστιν ἡ γῆ, τὴν οἰκουμένην καὶ τὸ πλήρωμα αὐτῆς σὺ ἐθεμελίωσας.
\vs{13}Τὸν Βοῤῥᾶν καὶ θάλασσας σὺ ἔκτισας, Θαβὼρ καὶ Ἑρμὼν ἐν τῷ ὀνόματί σου ἀγαλλιάσονται.
\vs{14}Σὸς ὁ βραχίων μετὰ δυναστείας· κραταιωθήτω ἡ χείρ σου, ὑψωθήτω ἡ δεξιά σου.
\vs{15}Δικαιοσύνη καὶ κρίμα ἑτοιμασία τοῦ θρόνου σου· ἔλεος καὶ ἀλήθεια προπορεύσονται πρὸ προσώπου σου.

\vs{16}Μακάριος ὁ λαὸς ὁ γινώσκων ἀλαλαγμόν· Κύριε ἐν τῷ φωτὶ τοῦ προσώπου σου πορεύσονται,
\vs{17}καὶ ἐν τῷ ὀνόματί σου ἀγαλλιάσονται ὅλην τὴν ἡμέραν, καὶ ἐν τῇ δικαιοσύνῃ σου ὑψωθήσονται.
\vs{18}Ὅτι τὸ καύχημα τῆς δυνάμεως αὐτῶν σὺ εἶ, καὶ ἐν τῇ εὐδοκίᾳ σου ὑψωθήσεται τὸ κέρας ἡμῶν·
\vs{19}ὅτι τοῦ Κυρίου ἡ ἀντίληψις, καὶ τοῦ ἁγίου Ἰσραὴλ βασιλέως ἡμῶν.

\vs{20}Τότε ἐλάλησας ἐν ὁράσει τοῖς υἱοῖς σου, καὶ εἶπας, ἐθέμην βοήθειαν ἐπὶ δυνατὸν, ὕψωσα ἐκλεκτὸν ἐκ τοῦ λαοῦ μου.
\vs{21}Εὗρον Δαυὶδ τὸν δοῦλόν μου, ἐν ἐλέει ἁγίῳ ἔχρισα αὐτόν.
\vs{22}Ἡ γὰρ χείρ μου συναντιλήψεται αὐτῷ, καὶ ὁ βραχίων μου κατισχύσει αὐτόν.
\vs{23}Οὐκ ὠφελήσει ἐχθρὸς ἐν αὐτῷ, καὶ υἱὸς ἀνομίας οὐ προσθήσει τοῦ κακῶσαι αὐτόν.
\vs{24}Καὶ συγκόψω ἀπὸ προσώπου αὐτοῦ τοὺς ἐχθροὺς αὐτοῦ, καὶ τοὺς μισοῦντας αὐτὸν τροπώσομαι.
\vs{25}Καὶ ἡ ἀλήθειά μου καὶ τὸ ἔλεός μου μετʼ αὐτοῦ, καὶ ἐν τῷ ὀνόματί μου ὑψωθήσεται τὸ κέρας αὐτοῦ.
\vs{26}Καὶ θήσομαι ἐν θαλάσσῃ χεῖρα αὐτοῦ, καὶ ἐν ποταμοῖς δεξιὰν αὐτοῦ.
\vs{27}Αὐτὸς ἐπικαλέσεταί με, πατήρ μου εἶ σὺ, Θεός μου καὶ ἀντιλήπτωρ τῆς σωτηρίας μου.
\vs{28}Κᾀγὼ πρωτότοκον θήσομαι αὐτὸν, ὑψηλὸν παρὰ τοῖς βασιλεῦσι τῆς γῆς.
\vs{29}Εἰς τὸν αἰῶνα φυλάξω αὐτῷ τὸ ἔλεός μου, καὶ ἡ διαθήκη μου πιστὴ αὐτῷ.
\vs{30}Καὶ θήσομαι εἰς τὸν αἰῶνα τοῦ αἰῶνος τὸ σπέρμα αὐτοῦ, καὶ τὸν θρόνον αὐτοῦ ὡς τὰς ἡμέρας τοῦ οὐρανοῦ.

\vs{31}Ἐὰν ἐγκαταλίπωσιν οἱ υἱοὶ αὐτοῦ τὸν νόμον μου, καὶ τοῖς κρίμασί μου μὴ πορευθῶσιν·
\vs{32}ἐὰν τὰ δικαιώματά μου βεβηλώσωσι, καὶ τὰς ἐντολάς μου μὴ φυλάξωσιν·
\vs{33}Ἐπισκέψομαι ἐν ῥάβδῳ τὰς ἀνομίας αὐτῶν, καὶ ἐν μάστιξι τὰς ἁμαρτίας αὐτῶν.
\vs{34}Τὸ δὲ ἔλεός μου οὐ μὴ διασκεδάσω ἀπʼ αὐτοῦ, οὐδὲ μὴ ἀδικήσω ἐν τῇ ἀληθείᾳ μου,
\vs{35}οὐδὲ μὴ βεβηλώσω τὴν διαθήκην μου, καὶ τὰ ἐκπορευόμενα διὰ τῶν χειλέων μου οὐ μὴ ἀθετήσω.
\vs{36}Ἅπαξ ὤμοσα ἐν τῷ ἁγίῳ μου, εἰ τῷ Δαυὶδ ψεύσομαι.
\vs{37}Τὸ σπέρμα αὐτοῦ εἰς τὸν αἰῶνα μενεῖ, καὶ ὁ θρόνος αὐτοῦ ὡς ὁ ἥλιος ἐναντίον μου,
\vs{38}καὶ ὡς ἡ σελήνη κατηρτισμένη εἰς τὸν αἰῶνα, καὶ ὁ μάρτυς ἐν οὐρανῷ πιστός· διάψαλμα.

\vs{39}Σὺ δὲ ἀπώσω καὶ ἐξουδένωσας, ἀνεβάλου τὸν χριστόν σου.
\vs{40}Κατέστρεψας τὴν διαθήκην τοῦ δούλου σου, ἐβεβήλωσας εἰς τὴν γῆν τὸ ἁγίασμα αὐτοῦ.
\vs{41}Καθεῖλες πάντας τοὺς φραγμοῦς αὐτοῦ, ἔθου τὰ ὀχυρώματα αὐτοῦ δειλίαν.
\vs{42}Διήρπασαν αὐτὸν πάντες οἱ διοδεύοντες ὁδὸν, ἐγενήθη ὄνειδος τοῖς γείτοσιν αὐτοῦ.
\vs{43}Ὕψωσας τὴν δεξιὰν τῶν ἐχθρῶν αὐτοῦ, εὔφρανας πάντας τοὺς ἐχθροὺς αὐτοῦ.
\vs{44}Ἀπέστρεψας τὴν βοήθειαν τῆς ῥομφαίας αὐτοῦ, καὶ οὐκ ἀντελάβου αὐτοῦ ἐν τῷ πολέμῳ.
\vs{45}Κατέλυσας ἀπὸ καθαρισμοῦ αὐτὸν, τὸν θρόνον αὐτοῦ εἰς τὴν γῆν κατέῤῥαξας,
\vs{46}ἐσμίκρυνας τὰς ἡμέρας τοῦ θρόνου αὐτοῦ, κατέχεας αὐτοῦ αἰσχύνην· διάψαλμα.

\vs{47}Ἕως πότε Κύριε ἀποστρέφῃ εἰς τέλος; ἐκκαυθήσεται ὡς πῦρ ἡ ὀργή σου;
\vs{48}Μνήσθητι τίς ἡ ὑπόστασίς μου· μὴ γὰρ ματαίως ἔκτισας πάντας τοὺς υἱοὺς τῶν ἀνθρώπων;
\vs{49}Τίς ἐστιν ἄνθρωπος, ὃς ζήσεται καὶ οὐκ ὄψεται θάνατον; ῥύσεται τὴν ψυχὴν αὐτοῦ ἐκ χειρὸς ᾅδου; διάψαλμα.
\vs{50}Ποῦ ἐστι τὰ ἐλέη σου τὰ ἀρχαῖα, Κύριε, ἃ ὤμοσας τῷ Δαυὶδ ἐν τῇ ἀληθείᾳ σου;
\vs{51}Μνήσθητι Κύριε τοῦ ὀνειδισμου τῶν δούλων σου οὗ ὑπέσχον ἐν τῷ κόλπῳ μου πολλῶν ἐθνῶν·
\vs{52}οὗ ὠνείδισαν οἱ ἐχθροί σου Κύριε, οὗ ὠνείδισαν τὸ ἀντάλλαγμα τοῦ χριστοῦ σου.
\vs{53}Εὐλογητὸς Κύριος εἰς τὸν αἰῶνα· γένοιτο, γένοιτο.

\begin{psalmheading}{\ch{89}{90} Προσευχὴ τοῦ Μωυσῆ ἀνθρώπου τοῦ Θεοῦ.}
\end{psalmheading}
Κύριε, καταφυγὴ ἐγενήθης ἡμῖν ἐν γενεᾷ καὶ γενεᾷ.
\vs{2}Πρὸ τοῦ ὄρη γενηθῆναι καὶ πλασθῆναι τὴν γῆν καὶ τὴν οἰκουμένην, καὶ ἀπὸ τοῦ αἰῶνος ἕως τοῦ αἰῶνος σὺ εἶ.
\vs{3}Μὴ ἀποστρέψῃς ἄνθρωπον εἰς ταπείνωσιν, καὶ εἶπας, ἐπιστρέψατε υἱοὶ ἀνθρώπων;
\vs{4}Ὅτι χίλια ἔτη ἐν ὀφθαλμοῖς σου, ὡς ἡ ἡμέρα ἡ ἐχθὲς ἥτις διῆλθε, καὶ φυλακὴ ἐν νυκτί.
\vs{5}Τὰ ἐξουδενώματα αὐτῶν ἔτη ἔσονται, τοπρωῒ ὡσεὶ χλόη παρέλθοι·
\vs{6}Τοπρωῒ ἀνθήσαι καὶ παρέλθοι, τὸ ἑσπέρας ἀποπέσοι, σκληρυνθείη καὶ ξηρανθείη.
\vs{7}Ὅτι ἐξελίπομεν ἐν τῇ ὀργῇ σου, καὶ ἐν τῷ θυμῷ σου ἐταράχθημεν.
\vs{8}Ἔθου τὰς ἀνομίας ἡμῶν ἐνώπιόν σου, ὁ αἰὼν ἡμῶν εἰς φωτισμὸν τοῦ προσώπου σου.
\vs{9}Ὅτι πᾶσαι αἱ ἡμέραι ἡμῶν ἐξέλιπον, καὶ ἐν τῇ ὀργῇ σου ἐξελίπομεν· τὰ ἔτη ἡμῶν ὡς ἀράχνη ἐμελέτων.
\vs{10}Αἱ ἡμέραι τῶν ἐτῶν ἡμῶν ἐν αὐτοῖς ἑβδομήκοντα ἔτη, ἐὰν δὲ ἐν δυναστείαις, ὀγδοήκοντα ἔτη, καὶ τὸ πλεῖον αὐτῶν κόπος καὶ πόνος· ὅτι ἐπῆλθε πραΰτης ἐφʼ ἡμᾶς, καὶ παιδευθησόμεθα.
\vs{11}Τίς γινώσκει τὸ κράτος τῆς ὀργῆς σου, καὶ ἀπὸ τοῦ φόβου τοῦ θυμοῦ σου ἐξαριθμήσασθαι;
\vs{12}τὴν δεξιάν σου οὕτως γνώρισον, καὶ τοὺς πεπαιδευμένους τῇ καρδίᾳ ἐν σοφίᾳ.

\vs{13}Ἐπίστρεψον, Κύριε· ἕως πότε; καὶ παρακλήθητι ἐπὶ τοῖς δούλοις σου.
\vs{14}Ἐνεπλήσθημεν τοπρωῒ τοῦ ἐλέους σου, καὶ ἠγαλλιασάμεθα καὶ εὐφράνθημεν· ἐν πάσαις ταῖς ἡμέραις ἡμῶν
\vs{15}εὐφρανθείημεν, ἀνθʼ ᾧν ἡμερῶν ἐταπείνωσας ἡμᾶς, ἐτῶν ὧν εἴδομεν κακά.
\vs{16}Καὶ ἴδε ἐπὶ τοὺς δούλους σου καὶ ἐπὶ τὰ ἔργα σου, καὶ ὁδήγησον τοὺς υἱοὺς αὐτῶν.
\vs{17}Καὶ ἔστω ἡ λαμπρότης Κυρίου τοῦ Θεοῦ ἡμῶν ἐφʼ ἡμᾶς, καὶ τὰ ἔργα τῶν χειρῶν ἡμῶν κατεύθυνον ἐφʼ ἡμᾶς.

\begin{psalmheading}{\ch{90}{91} Αἶνος ᾠδῆς τῷ Δαυίδ.}
\end{psalmheading}
Ὁ κατοικῶν ἐν βοηθείᾳ τοῦ ὑψίστου, ἐν σκέπῃ τοῦ Θεοῦ τοῦ οὐρανοῦ αὐλισθήσεται.
\vs{2}Ἐρεῖ τῷ Κυρίῳ, ἀντιλήπτωρ μου εἶ καὶ καταφυγή μου, ὁ Θεός μου, ἐλπιῶ ἐπʼ αὐτόν.
\vs{3}Ὅτι αὐτὸς ῥύσεταί σε ἐκ παγίδος θηρευτῶν, ἀπὸ λόγου ταραχώδους.
\vs{4}Ἐν τοῖς μεταφρένοις αὐτοῦ ἐπισκιάσει σοι, καὶ ὑπὸ τὰς πτέρυγας αὐτοῦ ἐλπιεῖς· ὅπλῳ κυκλώσει σε ἡ ἀλήθεια αὐτοῦ.
\vs{5}Οὐ φοβηθήσῃ ἀπὸ φόβου νυκτερινοῦ, ἀπὸ βέλους πετομένου ἡμέρας,
\vs{6}ἀπὸ πράγματος διαπορευομένου ἐν σκότει, ἀπὸ συμπτώματος καὶ δαιμονίου μεσημβρινοῦ.
\vs{7}Πεσεῖται ἐκ τοῦ κλίτους σου χιλιὰς, καὶ μυριὰς ἐκ δεξιῶν σου, πρὸς σὲ δὲ οὐκ ἐγγιεῖ.
\vs{8}Πλὴν τοῖς ὀφθαλμοῖς σου κατανοήσεις, καὶ ἀνταπόδοσιν ἁμαρτωλῶν ὄψει.

\vs{9}Ὅτι σὺ Κύριε ἡ ἐλπίς μου, τὸν ὕψιστον ἔθου καταφυγήν σου.
\vs{10}Οὐ προσελεύσεται πρὸς σὲ κακὰ, καὶ μάστιξ οὐκ ἐγγιεῖ τῷ σκηνώματί σου.
\vs{11}Ὅτι τοῖς ἀγγέλοις αὐτοῦ ἐντελεῖται περὶ σοῦ, τοῦ διαφυλάξαι σε ἐν πάσαις ταῖς ὁδοῖς σου.
\vs{12}Ἐπὶ χειρῶν ἀροῦσί σε, μή ποτε προσκόψῃς πρὸς λίθον τὸν πόδα σου.
\vs{13}Ἐπʼ ἀσπίδα καὶ βασιλίσκον ἐπιβήσῃ, καὶ καταπατήσεις λέοντα καὶ δράκοντα.

\vs{14}Ὅτι ἐπʼ ἐμὲ ἤλπισε, καὶ ῥύσομαι αὐτόν· σκεπάσω αὐτὸν, ὅτι ἔγνω τὸ ὄνομά μου.
\vs{15}Ἐπικαλέσεται πρὸς μὲ, καὶ εἰσακούσομαι αὐτοῦ, μετʼ αὐτοῦ εἰμι ἐν θλίψει, καὶ ἐξελοῦμαι αὐτὸν, καὶ δοξάσω αὐτόν.
\vs{16}Μακρότητι ἡμερῶν ἐμπλήσω αὐτὸν, καὶ δείξω αὐτῷ τὸ σωτήριόν μου.

\begin{psalmheading}{\ch{91}{92} Ψαλμὸς ᾠδῆς εἰς τὴν ἡμέραν τοῦ σαββάτου.}
\end{psalmheading}
\vs{2}Ἀγαθὸν τὸ ἐξομολογεῖσθαι τῷ Κυρίῳ, καὶ ψάλλειν τῷ ὀνόματί σου, ὕψιστε·
\vs{3}τοῦ ἀναγγέλλειν τοπρωῒ τὸ ἔλεός σου, καὶ τὴν ἀλήθειάν σου κατὰ νύκτα,
\vs{4}ἐν δεκαχόρδῳ ψαλτηρίῳ, μετʼ ᾠδῆς ἐν κιθάρᾳ.
\vs{5}Ὅτι εὔφρανάς με, Κύριε, ἐν τῷ ποιήματί σου, καὶ ἐν τοῖς ἔργοις τῶν χειρῶν σου ἀγαλλιάσομαι.

\vs{6}Ὡς ἐμεγαλύνθη τὰ ἔργα σου, Κύριε; σφόδρα ἐβαθύνθησαν οἱ διαλογισμοί σου.
\vs{7}Ἀνὴρ ἄφρων οὐ γνώσεται, καὶ ἀσύνετος οὐ συνήσει ταῦτα.
\vs{8}Ἐν τῷ ἀνατεῖλαι τοὺς ἁμαρτωλοὺς ὡσεὶ χόρτον, καὶ διέκυψαν πάντες οἱ ἐργαζόμενοι τὴν ἀνομίαν, ὅπως ἂν ἐξολοθρευθῶσιν εἰς τὸν αἰῶνα τοῦ αἰῶνος.
\vs{9}Σὺ δὲ ὕψιστος εἰς τὸν αἰῶνα, Κύριε.

\vs{10}Ὅτι ἰδοὺ οἱ ἐχθροί σου ἀπολοῦνται, καὶ διασκορπισθήσονται πάντες οἱ ἐργαζόμενοι τὴν ἀνομίαν.
\vs{11}Καὶ ὑψωθήσεται ὡς μονοκέρωτος τὸ κέρας μου, καὶ τὸ γῆράς μου ἐν ἐλέῳ πίονι.
\vs{12}Καὶ ἐπεῖδεν ὁ ὀφθαλμός μου ἐν τοῖς ἐχθροῖς μου, καὶ ἐν τοῖς ἐπανισταμένοις ἐπʼ ἐμὲ πονηρευομένοις ἀκούσεται τὸ οὖς μου.

\vs{13}Δίκαιος ὡς φοῖνιξ ἀνθήσει, ὡς ἡ κέδρος ἡ ἐν τῷ Λιβάνῳ πληθυνθήσεται·
\vs{14}Πεφυτευμένοι ἐν τῷ οἴκῳ Κυρίου, ἐν ταῖς αὐλαῖς τοῦ Θεοῦ ἡμῶν ἐξανθήσουσι.
\vs{15}Τότε πληθυνθήσονται ἐν γήρει πίονι, καὶ εὐπαθοῦντες ἔσονται
\vs{16}τοῦ ἀναγγεῖλαι· ὅτι εὐθὴς Κύριος ὁ Θεός μου, καὶ οὐκ ἔστιν ἀδικία ἐν αὐτῷ.

\begin{psalmheading}{\ch{92}{93} Εἰς τὴν ἡμέραν τοῦ προσαββάτου, ὅτε κατῴκισται ἡ γῆ, αἶνος ᾠδῆς τῷ Δαυίδ.}
\end{psalmheading}
Ὁ Κύριος ἐβασίλευσεν, εὐπρέπειαν ἐνεδύσατο, ἐνεδύσατο Κύριος δύναμιν καὶ περιεζώσατο· καὶ γὰρ ἐστερέωσε τὴν οἰκουμένην, ἥτις οὐ σαλευθήσεται.
\vs{2}Ἕτοιμος ὁ θρόνος σου ἀπὸ τότε, ἀπὸ τοῦ αἰῶνος σὺ εἶ.
\vs{3}Ἐπῇραν οἱ ποταμοὶ Κύριε, ἐπῇραν οἱ ποταμοὶ φωνὰς αὐτῶν,
\vs{4}ἀπὸ φωνῶν ὑδάτων πολλῶν· θαυμαστοὶ οἱ μετεωρισμοὶ τῆς θαλάσσης· θαυμαστὸς ἐν ὑψηλοῖς ὁ Κύριος.
\vs{5}Τὰ μαρτύριά σου ἐπιστώθησαν σφόδρα· τῷ οἴκῳ σου πρέπει ἁγίασμα, Κύριε, εἰς μακρότητα ἡμερῶν.

\begin{psalmheading}{\ch{93}{94} Ψαλμὸς τῷ Δαυὶδ τετράδι σαββάτου.}
\end{psalmheading}
Θεὸς ἐκδικήσεων Κύριος, ὁ Θεὸς ἐκδικήσεων ἐπαῤῥησιάσατο.
\vs{2}Ὑψώθητι ὁ κρίνων τὴν γῆν, ἀπόδος ἀνταπόδοσιν τοῖς ὑπερηφάνοις.

\vs{3}Ἕως πότε ἁμαρτωλοὶ, Κύριε, ἕως πότε ἁμαρτωλοὶ καυχήσονται;
\vs{4}Φθέγξονται καὶ λαλήσουσιν ἀδικίαν, λαλήσουσι πάντες οἱ ἐργαζόμενοι τὴν ἀνομίαν.
\vs{5}Τὸν λαόν σου, Κύριε, ἐταπείνωσαν, καὶ τὴν κληρονομίαν σου ἐκάκωσαν.
\vs{6}Χήραν καὶ ὀρφανὸν ἀπέκτειναν, καὶ προσήλυτον ἐφόνευσαν.
\vs{7}Καὶ εἶπαν, οὐκ ὄψεται Κύριος, οὐδὲ συνήσει ὁ Θεὸς τοῦ Ἰακώβ.

\vs{8}Σύνετε δὴ ἄφρονες ἐν τῷ λαῷ, καὶ μωροὶ, ποτὲ φρονήσατε.
\vs{9}Ὁ φυτεύσας τὸ οὖς, οὐχὶ ἀκούει; ἢ ὁ πλάσας τὸν ὀφθαλμὸν, οὐχὶ κατανοεῖ;
\vs{10}Ὁ παιδεύων ἔθνη, οὐχὶ ἐλέγξει; ὁ διδάσκων ἄνθρωπον γνῶσιν;
\vs{11}Κύριος γινώσκει τοὺς διαλογισμοὺς τῶν ἀνθρώπων, ὅτι εἰσὶ μάταιοι.

\vs{12}Μακάριος ὁ ἄνθρωπος ὃν ἂν σὺ παιδεύσῃς Κύριε, καὶ ἐκ τοῦ νόμου σου διδάξῃς αὐτόν·
\vs{13}τοῦ πραΰναι αὐτῷ ἀφʼ ἡμερῶν πονηρῶν, ἕως οὗ ὀρυγῇ τῷ ἁμαρτωλῷ βόθρος.
\vs{14}Ὅτι οὐκ ἀπώσεται Κύριος τὸν λαὸν αὐτοῦ, καὶ τὴν κληρονομίαν αὐτοῦ οὐκ ἐγκαταλείψει,
\vs{15}ἕως οὗ δικαιοσύνη ἐπιστρέψῃ εἰς κρίσιν, καὶ ἐχόμενοι αὐτῆς πάντες οἱ εὐθεῖς τῇ καρδίᾳ· διάψαλμα.

\vs{16}Τίς ἀναστήσεταί μοι ἐπὶ πονηρευομένους, ἢ τίς συμπαραστήσεταί μοι ἐπὶ τοὺς ἐργαζομένους τὴν ἀνομίαν;
\vs{17}Εἰ μὴ ὅτι Κύριος ἐβοήθησέ μοι, παραβραχὺ παρῴκησε τῷ ᾅδῃ ἡ ψυχή μου.
\vs{18}Εἰ ἔλεγον, σεσάλευται ὁ πούς μου, τὸ ἔλεός σου Κύριε ἐβοήθει μοι.
\vs{19}Κύριε, κατὰ τὸ πλῆθος τῶν ὀδυνῶν μου ἐν τῇ καρδίᾳ μου, αἱ παρακλήσεις σου ἠγάπησαν τὴν ψυχήν μου.

\vs{20}Μὴ συμπροσέσται σοι θρόνος ἀνομίας, ὁ πλάσσων κόπον ἐπὶ προστάγματι.
\vs{21}Θηρεύσουσιν ἐπὶ ψυχὴν δικαίου, καὶ αἷμα ἀθῶον καταδικάσονται.
\vs{22}Καὶ ἐγένετό μοι Κύριος εἰς καταφυγὴν, καὶ ὁ Θεός μου εἰς βοηθὸν ἐλπίδος μου.
\vs{23}Καὶ ἀποδώσει αὐτοῖς τὴν ἀνομίαν αὐτῶν, καὶ τὴν πονηρίαν αὐτῶν· ἀφανιεῖ αὐτοὺς Κύριος ὁ Θεὸς ἡμῶν.

\begin{psalmheading}{\ch{94}{95} Αἶνος ᾠδῆς τῷ Δαυίδ.}
\end{psalmheading}
Δεῦτε ἀγαλλιασώμεθα τῷ Κυρίῳ, ἀλαλάξωμεν τῷ Θεῷ τῷ σωτῆρι ἡμῶν.
\vs{2}Προφθάσωμεν τὸ πρόσωπον αὐτοῦ ἐν ἐξομολογήσει, καὶ ἐν ψαλμοῖς ἀλαλάξωμεν αὐτῷ.
\vs{3}Ὅτι Θεὸς μέγας Κύριος, καὶ βασιλεὺς μέγας ἐπὶ πάντας τοὺς θεούς·
\vs{4}ὅτι οὐκ ἀπώσεται Κύριος τὸν λαὸν αὐτοῦ, ὅτι ἐν τῇ χειρὶ αὐτοῦ τὰ πέρατα τῆς γῆς, καὶ τὰ ὕψη τῶν ὀρέων αὐτοῦ ἐστιν.
\vs{5}Ὅτι αὐτοῦ ἐστιν ἡ θάλασσα καὶ αὐτὸς ἐποίησεν αὐτὴν, καὶ τὴν ξηρὰν χεῖρες αὐτοῦ ἔπλασαν.

\vs{6}Δεῦτε προσκυνήσωμεν καὶ προσπέσωμεν αὐτῷ, καὶ κλαύσωμεν ἐναντίον Κυρίου τοῦ ποίησαντος ἡμᾶς.
\vs{7}Ὅτι αὐτός ἐστιν ὁ Θεὸς ἡμῶν, καὶ ἡμεῖς λαὸς νομῆς αὐτοῦ, καὶ πρόβατα χειρὸς αὐτοῦ·
\vs{8}σήμερον ἐὰν τῆς φωνῆς αὐτοῦ ἀκούσητε, μὴ σκληρύνητε τὰς καρδίας ὑμῶν, ὡς ἐν τῷ παραπικρασμῷ, κατὰ τὴν ἡμέραν τοῦ πικρασμοῦ ἐν τῇ ἐρήμῳ,
\vs{9}οὗ ἐπείρασάν με οἱ πατέρες ὑμῶν· ἐδοκίμασαν, καὶ εἶδον τὰ ἔργα μου.
\vs{10}Τεσσαράκοντα ἔτη προσώχθισα τῇ γενεᾷ ἐκείνῃ, καὶ εἶπα, ἀεὶ πλανῶνται τῇ καρδίᾳ, καὶ αὐτοὶ οὐκ ἔγνωσαν τὰς ὁδούς μου.
\vs{11}Ὡς ὤμοσα ἐν τῇ ὀργῇ μου, εἰ εἰσελεύσονται εἰς τὴν κατάπαυσίν μου.

\begin{psalmheading}{\ch{95}{96} Ὅτε ὁ οἶκος ᾠκοδόμηται μετὰ τὴν αἰχμαλωσίαν, ᾠδὴ τῷ Δαυίδ.}
\end{psalmheading}
Ἄσατε τῷ Κυρίῳ ἆσμα καινὸν, ᾄσατε τῷ Κυρίῳ πᾶσα ἡ γῆ.
\vs{2}Ἄσατε τῷ Κυρίῳ, εὐλογήσατε τὸ ὄνομα αὐτοῦ, εὐαγγελίζεσθε ἡμέραν ἐξ ἡμέρας τὸ σωτήριον αὐτοῦ.
\vs{3}Ἀναγγείλατε ἐν τοῖς ἔθνεσι τὴν δόξαν αὐτοῦ, ἐν πᾶσι τοῖς λαοῖς τὰ θαυμάσια αὐτοῦ.

\vs{4}Ὅτι μέγας Κύριος καὶ αἰνετὸς σφόδρα, φοβερός ἐστιν ἐπὶ πάντας τοὺς θεούς.
\vs{5}Ὅτι πάντες οἱ θεοὶ τῶν ἐθνῶν δαιμόνια, ὁ δὲ Κύριος τοὺς οὐρανοὺς ἐποίησεν.
\vs{6}Ἐξομολόγησις καὶ ὡραιότης ἐνώπιον αὐτοῦ, ἁγιωσύνη καὶ μεγαλοπρέπεια ἐν τῷ ἁγιάσματι αὐτοῦ.

\vs{7}Ἐνέγκατε τῷ Κυρίῳ αἱ πατριαὶ τῶν ἐθνῶν, ἐνέγκατε τῷ Κυρίῳ δόξαν καὶ τιμὴν,
\vs{8}ἐνέγκατε τῷ Κυρίῳ δόξαν ὀνόματι αὐτοῦ, ἄρατε θυσίας καὶ εἰσπορεύεσθε εἰς τὰς αὐλὰς αὐτοῦ·
\vs{9}Προσκυνήσατε τῷ Κυρίῳ ἐν αὐλῇ ἁγίᾳ αὐτοῦ, σαλευθήτω ἀπὸ προσώπου αὐτοῦ πᾶσα ἡ γῆ.
\vs{10}Εἴπατε ἐν τοῖς ἔθνεσιν, ὁ Κύριος ἐβασίλευσε· καὶ γὰρ κατώρθωσε τὴν σἰκουμένην, ἥτις οὐ σαλευθήσεται, κρινεῖ λαοῦς ἐν εὐθύτητι.
\vs{11}Εὐφραινέσθωσαν οἱ οὐρανοὶ καὶ ἀγαλλιάσθω ἡ γῆ, σαλευθήτω ἡ θάλασσα καὶ τὸ πλήρωμα αὐτῆς.
\vs{12}Χαρήσεται τὰ πεδία, καὶ πάντα τὰ ἐν αὐτοῖς· τότε ἀγαλλιάσονται πάντα τὰ ξύλα τοῦ δρυμοῦ
\vs{13}πρὸ προσώπου τοῦ Κυρίου, ὅτι ἔρχεται, ὅτι ἔρχεται κρῖναι τὴν γῆν· κρινεῖ τὴν οἰκουμένην ἐν δικαιοσύνῃ, καὶ λαοὺς ἐν τῇ ἀληθείᾳ αὐτοῦ.

\begin{psalmheading}{\ch{96}{97} Τῷ Δαυὶδ, ὅτε ἡ γῆ αὐτοῦ καθίσταται.}
\end{psalmheading}
Ὁ Κύριος ἐβασίλευσεν, ἀγαλλιάσθω ἡ γῆ, εὐφρανθήτωσαν νῆσοι πολλαί.

\vs{2}Νεφέλη καὶ γνόφος κύκλῳ αὐτοῦ, δικαιοσύνη καὶ κρίμα κατόρθωσις τοῦ θρόνου αὐτοῦ.
\vs{3}Πῦρ ἐναντίον αὐτοῦ προπορεύσεται, καὶ φλογιεῖ κύκλῳ τοὺς ἐχθροὺς αὐτοῦ.
\vs{4}Ἔφαναν αἱ ἀστραπαὶ αὐτοῦ τῇ οἰκουμένῃ, εἶδε καὶ ἐσαλεύθη ἡ γῆ·
\vs{5}τὰ ὄρη ὡσεὶ κηρὸς ἐτάκησαν ἀπὸ προσώπου Κυρίου, ἀπὸ προσώπου Κυρίου πάσης τῆς γῆς.
\vs{6}Ἀνήγγειλαν αἱ οὐρανοὶ τὴν δικαιοσύνην αὐτοῦ, καὶ εἴδοσαν πάντες οἱ λαοὶ τὴν δόξαν αὐτοῦ.

\vs{7}Αἰσχυνθήτωσαν πάντες οἱ προσκυνοῦντες τοῖς γλυπτοῖς, οἱ ἐγκαυχώμενοι ἐν τοῖς εἰδώλοις αὐτῶν· προσκυνήσατε αὐτῷ πάντες ἄγγελοι αὐτοῦ.

\vs{8}Ἤκουσε καὶ εὐφράνθη Σιὼν, καὶ ἠγαλλιάσαντο αἱ θυγατέρες τῆς Ἰουδαίας, ἕνεκεν τῶν κριμάτων σου Κύριε.
\vs{9}Ὅτι σὺ εἶ Κύριος ὁ ὕψιστος ἐπὶ πᾶσαν τὴν γῆν, σφόδρα ὑπερυψώθης ὑπὲρ πάντας τοὺς θεούς.

\vs{10}Οἱ ἀγαπῶντες τὸν Κύριον, μισεῖτε πονηρόν· φυλάσσει Κύριος τὰς ψυχὰς τῶν ὁσίων αὐτοῦ, ἐκ χειρὸς ἁμαρτωλῶν ῥύσεται αὐτούς.
\vs{11}Φῶς ἀνέτειλε τῷ δικαίῳ, καὶ τοῖς εὐθέσι τῇ καρδίᾳ εὐφροσύνη.
\vs{12}Εὐφράνθητε δίκαιοι ἐν τῷ Κυρίῳ, καὶ ἐξομολογεῖσθε τῇ μνήμῃ τῆς ἁγιωσύνης αὐτοῦ.

\begin{psalmheading}{\ch{97}{98} Ψαλμὸς τῷ Δαυίδ.}
\end{psalmheading}
Ἄσατε τῷ Κυρίῳ ᾆσμα καινὸν, ὅτι θαυμαστὰ ἐποίησεν ὁ Κύριος· ἔσωσεν αὐτῷ ἡ δεξιὰ αὐτοῦ, καὶ ὁ βραχίων ὁ ἅγιος αὐτοῦ.

\vs{2}Ἐγνώρισε Κύριος τὸ σωτήριον αὐτοῦ, ἐναντίον τῶν ἐθνῶν ἀπεκάλυψε τὴν δικαιοσύνην αὐτοῦ.
\vs{3}Ἐμνήσθη τοῦ ἐλέους αὐτοῦ τῷ Ἰακὼβ, καὶ τῆς ἀληθείας αὐτοῦ τῷ οἴκῳ Ἰσραήλ· εἴδοσαν πάντα τὰ πέρατα τῆς γῆς τὸ σωτήριον τοῦ θεοῦ ἡμῶν.

\vs{4}Ἀλαλάξατε τῷ Θεῷ πᾶσα ἡ γῆ, ᾄσατε καὶ ἀγαλλιᾶσθε καὶ ψάλατε.
\vs{5}Ψάλατε τῷ Κυρίῳ ἐν κιθάρᾳ, ἐν κιθάρᾳ καὶ φωνῇ ψαλμοῦ.
\vs{6}Ἐν σάλπιγξιν ἐλαταῖς, καὶ φωνῇ σάλπιγγος κερατίνης· ἀλαλάξατε ἐνώπιον τοῦ βασιλέως Κυρίῳ.
\vs{7}Σαλευθήτω ἡ θάλασσα καὶ τὸ πλήρωμα αὐτῆς, ἡ οἰκουμένη καὶ οἱ κατοικοῦντες αὐτήν.
\vs{8}Ποταμοὶ κροτήσουσι χειρὶ ἐπιτοαυτὸ, τὰ ὄρη ἀγαλλιάσονται.
\vs{9}Ὅτι ἥκει κρῖναι τὴν γῆν· κρινεῖ τὴν οἰκουμένην ἐν δικαιοσύνῃ, καὶ λαοὺς ἐν εὐθύτητι.

\begin{psalmheading}{\ch{98}{99} Ψαλμὸς τῷ Δαυίδ.}
\end{psalmheading}
Ὁ Κύριος ἐβασίλευσεν, ὀργιζέσθωσαν λαοί· ὁ καθήμενος ἐπὶ τῶν χερουβὶμ, σαλευθήτω ἡ γῆ.
\vs{2}Κύριος ἐν Σιὼν μέγας, καὶ ὑψηλός ἐστιν ἐπὶ πάντας τοὺς λαούς.
\vs{3}Ἐξομολογησάσθωσαν τῷ ὀνόματί σου τῷ μεγάλῳ, ὅτι φοβερὸν καὶ ἅγιόν ἐστι,
\vs{4}καὶ τιμὴ βασιλέως κρίσιν ἀγαπᾷ· σὺ ἡτοίμασας εὐθύτητας, κρίσιν καὶ δικαιοσύνην ἐν Ἰακὼβ σὺ ἐποίησας.
\vs{5}Ὑψοῦτε Κύριον τὸν Θεὸν ἡμῶν, καὶ προσκυνεῖτε τῷ ὑποποδίῳ τῶν ποδῶν αὐτοῦ, ὅτι ἅγιός ἐστι.

\vs{6}Μωυσῆς καὶ Ἀαρὼν ἐν τοῖς ἱερεῦσιν αὐτοῦ, καὶ Σαμουὴλ ἐν τοῖς ἐπικαλουμένοις τὸ ὄνομα αὐτοῦ· ἐπεκαλοῦντο τὸν Κύριον, καὶ αὐτὸς εἰσήκουεν,
\vs{7}ἐν στύλῳ νεφέλης ἐλάλει πρὸς αὐτούς· ἐφύλασσον τὰ μαρτύρια αὐτοῦ, καὶ τὰ προστάγματα ἃ ἔδωκεν αὐτοῖς.
\vs{8}Κύριε ὁ Θεὸς ἡμῶν, σὺ ἐπήκουες αὐτῶν· ὁ Θεὸς, εὐίλατος ἐγίνου αὐτοῖς, καὶ ἐκδικῶν ἐπὶ πάντα τὰ ἐπιτηδεύματα αὐτῶν.
\vs{9}Ὑψοῦτε Κύριον τὸν Θεὸν ἡμῶν, καὶ προσκυνεῖτε εἰς ὄρος ἅγιον αὐτοῦ, ὅτι ἅγιος Κύριος ὁ Θεὸς ἡμῶν.

\begin{psalmheading}{\ch{99}{100} Ψαλμὸς εἰς ἐξομολόγησιν.}
\end{psalmheading}
Ἀλαλάξατε τῷ Κυρίῳ πᾶσα ἡ γῆ,
\vs{2}δουλεύσατε τῷ Κυρίῳ ἐν εὐφροσύνῃ· εἰσέλθατε ἐνώπιον αὐτοῦ ἐν ἀγαλλιάσει.
\vs{3}Γνῶτε ὅτι Κύριος αὐτός ἐστιν ὁ Θεός· αὐτὸς ἐποίησεν ἡμᾶς, καὶ οὐχ ἡμεῖς, λαὸς αὐτοῦ καὶ πρόβατα τῆς νομῆς αὐτοῦ.
\vs{4}Εἰσέλθατε εἰς τὰς πύλας αὐτοῦ ἐν ἐξομολογήσει, τὰς αὐλὰς αὐτοῦ ἐν ὕμνοις· ἐξομολογεῖσθε αὐτῷ, αἰνεῖτε τὸ ὄνομα αὐτοῦ.
\vs{5}Ὅτι χρηστὸς Κύριος, εἰς τὸν αἰῶνα τὸ ἔλεος αὐτοῦ, καὶ ἕως γενεᾶς καὶ γενεᾶς ἡ ἀλήθεια αὐτοῦ.

\begin{psalmheading}{\ch{100}{101} Ψαλμὸς τῷ Δαυίδ.}
\end{psalmheading}
Ἔλεος καὶ κρίσιν ᾄσομαί σοι, Κύριε·
\vs{2}ψαλῶ καὶ συνήσω ἐν ὁδῷ ἀμώμῳ· πότε ἥξεις πρὸς μέ; διεπορευόμην ἐν ἀκακίᾳ καρδίας μου, ἐν μέσῳ τοῦ οἴκου μου.
\vs{3}Οὐ προεθέμην πρὸ ὀφθαλμῶν μου πρᾶγμα παράνομον, ποιοῦντας παραβάσεις ἐμίσησα· οὐκ ἐκολλήθη μοικαρδία σκαμβὴ,
\vs{4}ἐκκλίνοντος ἀπʼ ἐμοῦ τοῦ πονηροῦ οὐκ ἐγίνωσκον.
\vs{5}Τὸν καταλαλοῦντα λάθρα τοῦ πλησίον αὐτοῦ, τοῦτον ἐξεδίωκον· ὑπερηφάνῳ ὀφθαλμῷ καὶ ἀπλήστῳ καρδίᾳ, τούτῳ οὐ συνήσθιον.
\vs{6}Οἱ ὀφθαλμοί μου ἐπὶ τοὺς πιστοὺς τῆς γῆς, τοῦ συγκαθῆσθαι αὐτοὺς μετʼ ἐμοῦ· πορευόμενος ἐν ὁδῷ ἀμώμῳ, οὗτός μοι ἐλειτούργει.
\vs{7}Οὐ κατῴκει ἐν μέσῳ τῆς οἰκίας μου ποιῶν ὑπερηφανίαν· λαλῶν ἄδικα οὐ κατεύθυνεν ἐναντίον τῶν ὀφθαλμῶν μου.
\vs{8}Εἰς τὰς πρωΐας ἀπέκτενον πάντας τοὺς ἁμαρτωλοὺς τῆς γῆς, τοῦ ἐξολοθρεῦσαι ἐκ πόλεως Κυρίου πάντας τοὺς ἐργαζομένους τὴν ἀδικίαν.

\begin{psalmheading}{\ch{101}{102} Προσευχὴ τῷ πτωχῷ, ὅταν ἀκηδιάσῃ, καὶ ἐναντίον Κυρίου ἐκχέῃ τὴν δέησιν αὐτοῦ.}
\end{psalmheading}
\vs{2}Κύριε εἰσάκουσον τῆς προσευχῆς μου, καὶ ἡ κραυγή μου πρὸς σὲ ἐλθέτω.
\vs{3}Μὴ ἀποστρέψῃς τὸ πρόσωπόν σου ἀπʼ ἐμοῦ· ἐν ᾗ ἂν ἡμέρᾳ θλίβομαι, κλῖνον πρὸς μὲ τὸ οὖς σου· ἐν ᾗ ἂν ἡμέρᾳ ἐπικαλέσωμαί σε, ταχὺ εἰσάκουσόν μου.

\vs{4}Ὅτι ἐξέλιπον ὡσεὶ καπνὸς αἱ ἡμέραι μου, καὶ τὰ ὀστᾶ μου ὡσεὶ φρύγιον συνεφρύγησαν.
\vs{5}Ἐπλήγην ὡσεὶ χόρτος, καὶ ἐξηράνθη ἡ καρδία μου, ὅτι ἐπελαθόμην τοῦ φαγεῖν τὸν ἄρτον μου.
\vs{6}Ἀπὸ φωνῆς τοῦ στεναγμοῦ μου, ἐκολλήθη τὸ ὀστοῦν μου τῇ σαρκί μου.
\vs{7}Ὡμοιώθην πελεκᾶνι ἐρημικῷ, ἐγενήθην ὡσεὶ νυκτικόραξ ἐν οἰκοπέδῳ.
\vs{8}Ἠγρύπνησα, καὶ ἐγενήθην ὡσεὶ στρουθίον μονάζον ἐπὶ δώματι.
\vs{9}Ὅλην τὴν ἡμέραν ὠνείδιζόν με οἱ ἐχθροί μου, καὶ οἱ ἐπαινοῦντές με κατʼ ἐμοῦ ὤμνυον.
\vs{10}Ὅτι σποδὸν ὡσεὶ ἄρτον ἔφαγον, καὶ τὸ πόμα μου μετὰ κλαυθμοῦ ἐκίρνων,
\vs{11}ἀπὸ προσώπου τῆς ὀργῆς σου καὶ τοῦ θυμοῦ σου, ὅτι ἐπάρας κατέῤῥαξάς με.

\vs{12}Αἱ ἡμέραι μου ὡσεὶ σκιὰ ἐκλίθησαν, κᾀγὼ ὡσεὶ χόρτος ἐξηράνθην.
\vs{13}Σὺ δέ Κύριε εἰς τὸν αἰῶνα μένεις, καὶ τὸ μνημόσυνόν σου εἰς γενεὰν καὶ γενεάν.
\vs{14}Σὺ ἀναστὰς οἰκτειρήσεις τὴν Σιὼν, ὅτι καιρὸς τοῦ οἰκτειρῆσαι αὐτὴν, ὅτι ἥκει καιρός.
\vs{15}Ὅτι εὐδόκησαν οἱ δοῦλοί σου τοὺς λίθους αὐτῆς, καὶ τὸν χοῦν αὐτῆς οἰκτειρήσουσι.
\vs{16}Καὶ φοβηθήσονται τὰ ἔθνη τὸ ὄνομά σου Κύριε, καὶ πάντες οἱ βασιλεῖς τὴν δόξαν σου.

\vs{17}Ὅτι οἰκοδομήσει Κύριος τὴν Σιὼν, καὶ ὀφθήσεται ἐν τῇ δόξῃ αὐτοῦ.
\vs{18}Ἐπέβλεψεν ἐπὶ τὴν προσευχὴν τῶν ταπεινῶν, καὶ οὐκ ἐξουδένωσε τὴν δέησιν αὐτῶν.
\vs{19}Γραφήτω αὕτη εἰς γενεὰν ἑτέραν, καὶ λαὸς ὁ κτιζόμενος αἰνέσει τὸν Κύριον.
\vs{20}Ὅτι ἐξέκυψεν ἐξ ὕψους ἁγίου αὐτοῦ, Κύριος ἐξ οὐρανοῦ ἐπὶ τὴν γῆν ἐπέβλεψε,
\vs{21}τοῦ ἀκοῦσαι τοῦ στεναγμοῦ τῶν πεπεδημένων, τοῦ λῦσαι τοὺς υἱοὺς τῶν τεθανατωμένων,
\vs{22}τοῦ ἀναγγεῖλαι ἐν Σιὼν τὸ ὄνομα Κυρίου, καὶ τὴν αἴνεσιν αὐτοῦ ἐν Ἱερουσαλήμ·
\vs{23}ἐν τῷ συναχθῆναι λαοὺς ἐπιτοαυτὸ, καὶ βασιλεῖς τοῦ δουλεύειν τῷ Κυρίῳ.

\vs{24}Ἀπεκρίθη αὐτῷ ἐν ὁδῷ ἰσχύος αὐτοῦ, τὴν ὀλιγότητα τῶν ἡμερῶν μου ἀνάγγειλόν μοι·
\vs{25}μὴ ἀναγάγῃς με ἐν ἡμίσει ἡμερῶν μου, ἐν γενεᾷ γενεῶν τὰ ἔτη σου.
\vs{26}Κατʼ ἀρχὰς τὴν γῆν σὺ Κύριε ἐθεμελίωσας, καὶ ἔργα τῶν χειρῶν σου εἰσὶν οἱ οὐρανοί.
\vs{27}Αὐτοὶ ἀπολοῦνται, σὺ δὲ διαμένεις· καὶ πάντες ὡς ἱμάτιον παλαιωθήσονται, καὶ ὡσεὶ περιβόλαιον ἑλίξεις αὐτοὺς, καὶ ἀλλαγήσονται.
\vs{28}Σὺ δὲ ὁ αὐτὸς εἶ, καὶ τὰ ἔτη σου οὐκ ἐκλείψουσιν·
\vs{29}Οἱ υἱοῖ τῶν δούλων σου κατασκηνώσουσι, καὶ τὸ σπέρμα αὐτῶν εἰς τὸν αἰῶνα κατευθυνθήσεται.

\begin{psalmheading}{\ch{102}{103} Τῷ Δαυίδ.}
\end{psalmheading}
Εὐλόγει ἡ ψυχή μου τὸν Κύριον, καὶ πάντα τὰ ἐντός μου τὸ ὄνομα τὸ ἅγιον αὐτοῦ.
\vs{2}Εὐλόγει ἡ ψυχή μου τὸν Κύριον, καὶ μὴ ἐπιλανθάνου πάσας τὰς αἰνέσεις αὐτοῦ·
\vs{3}Τὸν εὐιλατεύοντα πάσαις ταῖς ἀνομίαις σου, τὸν ἰώμενον πάσας τὰς νόσους σου,
\vs{4}τὸν λυτρούμενον ἐκ φθορᾶς τὴν ζωήν σου, τὸν στεφανοῦντά σε ἐν ἐλέει καὶ οἰκτιρμοῖς,
\vs{5}τὸν ἐμπιπλῶντα ἐν ἀγαθοῖς τὴν ἐπιθυμίαν σου· ἀνακαινισθήσεται ὡς ἀετοῦ ἡ νεότης σου.

\vs{6}Ποιῶν ἐλεημοσύνας ὁ Κύριος, καὶ κρίμα πᾶσι τοῖς ἀδικουμένοις.
\vs{7}Ἐγνώρισε τὰς ὁδοὺς αὐτοῦ τῷ Μωυσῇ, τοῖς υἱοῖς Ἰσραὴλ τὰ θελήματα αὐτοῦ.
\vs{8}Οἰκτίρμων καὶ ἐλεήμων ὁ Κύριος, μακρόθυμος καὶ πολυέλεος.
\vs{9}Οὐκ εἰς τέλος ὀργισθήσεται, οὐδὲ εἰς τὸν αἰῶνα μηνιεῖ.
\vs{10}Οὐ κατὰ τὰς ἁμαρτίας ἡμῶν ἐποίησεν ἡμῖν, οὐδὲ κατὰ τὰς ἀνομίας ἡμῶν ἀνταπέδωκεν ἡμῖν.
\vs{11}Ὅτι κατὰ τὸ ὕψος τοῦ οὐρανοῦ ἀπὸ τῆς γῆς, ἐκραταίωσε Κύριος τὸ ἔλεος αὐτοῦ ἐπὶ τοὺς φοβουμένους αὐτόν.
\vs{12}Καθόσον ἀπέχουσιν ἀνατολαὶ ἀπὸ δυσμῶν, ἐμάκρυνεν ἀφʼ ἡμῶν τὰς ἀνομίας ἡμῶν.
\vs{13}Καθὼς οἰκτείρει πατὴρ υἱοὺς, ᾠκτείρησε Κύριος τοὺς φοβουμένους αὐτόν.
\vs{14}Ὅτι αὐτὸς ἔγνω τὸ πλάσμα ἡμῶν· μνήσθητι ὅτι χοῦς ἐσμεν.

\vs{15}Ἄνθρωπος, ὡσεὶ χόρτος αἱ ἡμέραι αὐτοῦ, ὡσεὶ ἄνθος τοῦ ἀγροῦ οὕτως ἐξανθήσει.
\vs{16}Ὅτι πνεῦμα διῆλθεν ἐν αὐτῷ, καὶ οὐχ ὑπάρξει, καὶ οὐκ ἐπιγνώσεται ἔτι τὸν τόπον αὐτοῦ.
\vs{17}Τὸ δὲ ἔλεος τοῦ Κυρίου ἀπὸ τοῦ αἰῶνος καὶ ἕως τοῦ αἰῶνος ἐπὶ τοὺς φοβουμένους αὐτόν· καὶ ἡ δικαιοσύνη αὐτοῦ ἐπὶ υἱοὺς υἱῶν,
\vs{18}τοῖς φυλάσσουσι τὴν διαθήκην αὐτοῦ, καὶ μεμνημένοις τῶν ἐντολῶν αὐτοῦ τοῦ ποιῆσαι αὐτάς.

\vs{19}Κύριος ἐν τῷ οὐρανῷ ἡτοίμασε τὸν θρόνον αὐτοῦ, καὶ ἡ βασιλεία αὐτοῦ πάντων δεσπόζει.
\vs{20}Εὐλογεῖτε τὸν Κύριον πάντες ἄγγελοι αὐτοῦ, δυνατοὶ ἰσχύϊ ποιοῦντες τὸν λόγον αὐτοῦ, τοῦ ἀκοῦσαι τῆς φωνῆς τῶν λόγων αὐτοῦ.
\vs{21}Εὐλογεῖτε τὸν Κύριον πᾶσαι αἱ δυνάμεις αὐτοῦ, λειτουργοὶ αὐτοῦ ποιοῦντες τὰ θελήματα αὐτοῦ.
\vs{22}Εὐλογεῖτε τὸν Κύριον πάντα τὰ ἔργα αὐτοῦ, ἐν παντὶ τόπῳ τῆς δυναστείας αὐτοῦ· εὐλόγει ἡ ψυχή μου τὸν Κύριον.

\begin{psalmheading}{\ch{103}{104} Τῷ Δαυίδ.}
\end{psalmheading}
Εὐλόγει ἡ ψυχή μου τὸν Κύριον. Κύριε ὁ Θεός μου ἐμεγαλύνθης σφόδρα· ἐξομολόγησιν καὶ εὐπρέπειαν ἐνεδύσω,
\vs{2}ἀναβαλλόμενος φῶς ὡς ἱμάτιον, ἐκτείνων τὸν οὐρανὸν ὡσεὶ δέῤῥιν·
\vs{3}Ὁ στεγάζων ἐν ὕδασι τὰ ὑπερῷα αὐτοῦ, ὁ τιθεὶς νέφη τὴν ἐπίβασιν αὐτοῦ· ὁ περιπατῶν ἐπὶ πτερύγων ἀνέμων·
\vs{4}Ὁ ποιῶν τοὺς ἀγγέλους αὐτοῦ πνεύματα, καὶ τοὺς λειτουργοὺς αὐτοῦ πῦρ φλέγον·

\vs{5}Ὁ θεμελιῶν τὴν γῆν ἐπὶ τὴν ἀσφάλειαν αὐτῆς, οὐ κλιθήσεται εἰς τὸν αἰῶνα τοῦ αἰῶνος.
\vs{6}Ἄβυσσος ὡς ἱμάτιον τὸ περιβόλαιον αὐτοῦ, ἐπὶ τῶν ὀρέων στήσονται ὕδατα.
\vs{7}Ἀπὸ ἐπιτιμήσεώς σου φεύξονται, ἀπὸ φωνῆς βροντῆς σου δειλιάσουσιν.
\vs{8}Ἀναβαίνουσιν ὄρη, καὶ καταβαίνουσι πεδία εἰς τόπον ὃν ἐθεμελίωσας αὐτοῖς.
\vs{9}Ὅριον ἔθου ὃ οὐ παρελεύσονται, οὐδὲ ἐπιστρέψουσι καλύψαι τὴν γῆν.

\vs{10}Ὁ ἐξαποστέλλων πηγὰς ἐν φάραγξιν, ἀναμέσον τῶν ὀρέων διελεύσονται ὕδατα.
\vs{11}Ποτιοῦσι πάντα τὰ θηρία τοῦ ἀγροῦ, προσδέξονται ὄναγροι εἰς δίψαν αὐτῶν.
\vs{12}Ἐπʼ αὐτὰ τὰ πετεινὰ τοῦ οὐρανοῦ κατασκηνώσει, ἐκ μέσου τῶν πετρῶν δώσουσι φωνήν.
\vs{13}Ποτίζων ὄρη ἐκ τῶν ὑπερῴων αὐτοῦ, ἀπὸ καρποῦ τῶν ἔργων σου χορτασθήσεται ἡ γῆ.

\vs{14}Ὁ ἐξανατέλλων χόρτον τοῖς κτήνεσι, καὶ χλόην τῇ δουλείᾳ τῶν ἀνθρώπων· τοῦ ἐξαγαγεῖν ἄρτον ἐκ τῆς γῆς,
\vs{15}καὶ οἶνος εὐφραίνει καρδίαν ἀνθρώπου· τοῦ ἱλαρύναι πρόσωπον ἐν ἐλαίῳ, καὶ ἄρτος καρδίαν ἀνθρώπου στηρίζει.
\vs{16}Χορτασθήσεται τὰ ξύλα τοῦ πεδίου, αἱ κέδροι τοῦ Λιβάνου ἃς ἐφύτευσεν.
\vs{17}Ἐκεῖ στρουθία ἐννοσσεύσουσι, τοῦ ἐρωδιοῦ ἡ οἰκία ἡγεῖται αὐτῶν.
\vs{18}Ὄρη τὰ ὑψηλὰ ταῖς ἐλάφοις, πέτρα καταφυγὴ τοῖς χοιρογρυλλίοις.

\vs{19}Ἐποίησε σελήνην εἰς καιροὺς, ὁ ἥλιος ἔγνω τὴν δύσιν αὐτοῦ.
\vs{20}Ἔθου σκότος καὶ ἐγένετο νὺξ, ἐν αὐτῇ διελεύσονται πάντα τὰ θηρία τοῦ δρυμοῦ.
\vs{21}Σκυμνοι ὠρυόμενοι ἁρπάσαι, καὶ ζητῆσαι παρὰ τοῦ Θεοῦ βρῶσιν αὐτοῖς.
\vs{22}Ἀνέτειλεν ὁ ἥλιος καὶ συναχθήσονται, καὶ ἐν ταῖς μάνδραις αὐτῶν κοιτασθήσονται.
\vs{23}Ἐξελεύσεται ἄνθρωπος ἐπὶ τὸ ἔργον αὐτοῦ, καὶ ἐπὶ τὴν ἐργασίαν αὐτοῦ ἕως ἑσπέρας.

\vs{24}Ὡς ἐμεγαλύνθη τὰ ἔργα σου Κύριε, πάντα ἐν σοφίᾳ ἐποίησας· ἐπληρώθη ἡ γῆ τῆς κτίσεώς σου·
\vs{25}Αὕτη ἡ θάλασσα ἡ μεγάλη καὶ εὐρύχωρος· ἐκεῖ ἑρπετὰ ὧν οὐκ ἔστιν ἀριθμὸς, ζῶα μικρὰ μετὰ μεγάλων.
\vs{26}Ἐκεῖ πλοῖα διαπορεύονται, δράκων οὗτος ὃν ἔπλασας ἐμπαίζειν αὐτῷ.
\vs{27}Πάντα πρὸς σὲ προσδοκῶσι, δοῦναι τὴν τροφὴν αὐτοῖς εὔκαιρον.
\vs{28}Δόντος σου αὐτοῖς, συλλέξουσιν· ἀνοίξαντος δέ σου τὴν χεῖρα, τὰ σύμπαντα πλησθήσονται χρηστότητος.
\vs{29}Ἀποστρέψαντος δέ σου τὸ πρόσωπον, ταραχθήσονται· ἀντανελεῖς τὸ πνεῦμα αὐτῶν, καὶ ἐκλείψουσι, καὶ εἰς τὸν χοῦν αὐτῶν ἐπιστρέψουσιν.
\vs{30}Ἐξαποστελεῖς τὸ πνεῦμά σου καὶ κτισθήσονται, καὶ ἀνακαινιεῖς τὸ πρόσωπον τῆς γῆς.

\vs{31}Ἤτω ἡ δόξα Κυρίου εἰς τὸν αἰῶνα, εὐφρανθήσεται Κύριος ἐπὶ τοῖς ἔργοις αὐτοῦ·
\vs{32}Ὁ ἐπιβλέπων ἐπὶ τὴν γῆν καὶ ποιῶν αὐτὴν τρέμειν, ὁ ἁπτόμενος τῶν ὀρέων καὶ καπνίζονται.
\vs{33}Ἄσω τῷ Κυρίῳ ἐν τῇ ζωῇ μου, ψαλῶ τῷ Θεῷ μου ἕως ὑπάρχω.
\vs{34}Ἡδυνθείη αὐτῷ ἡ διαλογή μου, ἐγὼ δὲ εὐφρανθήσομαι ἐπὶ τῷ Κυρίῳ.
\vs{35}Ἐκλείποισαν ἁμαρτωλοὶ ἀπὸ τῆς γῆς, καὶ ἄνομοι, ὥστε μὴ ὑπάρχειν αὐτούς· εὐλόγει, ἡ ψυχή μου τὸν Κύριον.

\begin{psalmheading}{\ch{104}{105} Ἀλληλούϊα.}
\end{psalmheading}
Ἐξομολογεῖσθε τῷ Κυρίῳ, καὶ ἐπικαλεῖσθε τὸ ὄνομα αὐτοῦ· ἀπαγγείλατε ἐν τοῖς ἔθνεσι τὰ ἔργα αὐτοῦ.
\vs{2}Ασατε αὐτῷ καὶ ψάλατε αὐτῷ. διηγήσασθε πάντα τὰ θαυμάσια αὐτοῦ.
\vs{3}Ἐπαινεῖσθε ἐν τῷ ὀνόματι τῷ ἁγίῳ αὐτοῦ· εὐφρανθήτω καρδία ζητούντων τὸν Κύριον.
\vs{4}Ζητήσατε τὸν Κύριον καὶ κραταιώθητε· ζητήσατε τὸ πρόσωπον αὐτοῦ διαπαντός.
\vs{5}Μνήσθητε τῶν θαυμασίων αὐτον ὧν ἐποίησε, τὰ τέρατα αὐτοῦ, καὶ τὰ κρίματα τοῦ στόματος αὐτοῦ.
\vs{6}Σπέρμα Ἁβραὰμ δοῦλοι αὐτοῦ, υἱοὶ Ἰακὼβ ἐκλεκτοὶ αὐτοῦ.

\vs{7}Αὐτὸς Κύριος ὁ Θεὸς ἡμῶν, ἐν πάσῃ τῇ γῇ τὰ κρίματα αὐτοῦ.
\vs{8}Ἐμνήσθη εἰς τὸν αἰῶνα διαθήκης αὐτοῦ, λόγου οὗ ἐνετείλατο εἰς χιλίας γενεὰς,
\vs{9}ὃν διέθετο τῷ Ἁβραὰμ, καὶ τοῦ ὅρκου αὐτοῦ τῷ Ἰσαάκ·
\vs{10}Καὶ ἔστησεν αὐτὴν τῷ Ἰακὼβ εἰς πρόσταγμα, καὶ τῷ Ἰσραὴλ εἰς διαθήκην αἰώνιον,
\vs{11}λέγων, σοὶ δώσω τὴν γῆν Χαναὰν, σχοίνισμα κληρονομίας ὑμῶν.
\vs{12}Ἐν τῷ εἶναι αὐτοὺς ἀριθμῷ βραχεῖς, ὀλιγοστοὺς καὶ παροίκους ἐν αὐτῇ,
\vs{13}καὶ διῆλθον ἐξ ἔθνους εἰς ἔθνος, καὶ ἐκ βασιλείας εἰς λαὸν ἕτερον,
\vs{14}οὐκ ἀφῆκεν ἄνθρωπον ἀδικῆσαι αὐτοὺς, καὶ ἤλεγξεν ὑπὲρ αὐτῶν βασιλεῖς·
\vs{15}Μὴ ἅψησθε τῶν χριστῶν μου, καὶ ἐν τοῖς προφήταις μου μὴ πονηρεύεσθε.
\vs{16}Καὶ ἐκάλεσε λιμὸν ἐπὶ τὴν γῆν, πᾶν στήριγμα ἄρτου συνέτριψεν.

\vs{17}Ἀπέστειλεν ἔμπροσθεν αὐτῶν ἄνθρωπον, εἰς δοῦλον ἐπράθη Ἰωσήφ.
\vs{18}Ἐταπείνωσαν ἐν πέδαις τοὺς πόδας αὐτοῦ, σίδηρον διῆλθεν ἡ ψυχὴ αὐτοῦ·
\vs{19}Μέχρι τοῦ ἐλθεῖν τὸν λόγον αὐτοῦ· τὸ λόγιον τοῦ Κυρίου ἐπύρωσεν αὐτόν.
\vs{20}Ἀπέστειλε βασιλεὺς καὶ ἔλυσεν αὐτὸν, ἄρχων λαῶν καὶ ἀφῆκεν αὐτόν.
\vs{21}Κατέστησεν αὐτὸν κύριον τοῦ οἴκου αὐτοῦ, καὶ ἄρχοντα πάσης τῆς κτήσεως αὐτοῦ,
\vs{22}τοῦ παιδεῦσαι τοὺς ἄρχοντας αὐτοῦ ὡς ἑαυτὸν, καὶ τοὺς πρεσβυτέρους αὐτοῦ σοφίσαι.

\vs{23}Καὶ εἰσῆλθεν Ἰσραὴλ εἰς Αἴγυπτον, καὶ Ἰακὼβ παρῴκησεν ἐν γῇ Χάμ.
\vs{24}Καὶ ηὔξησε τὸν λαὸν αὐτοῦ σφόδρα, καὶ ἐκραταίωσεν αὐτὸν ὑπὲρ τοὺς ἐχθροὺς αὐτοῦ.
\vs{25}Καὶ μετέστρεψε τὴν καρδίαν αὐτῶν τοῦ μισῆσαι τὸν λαὸν αὐτοῦ, τοῦ δολιοῦσθαι ἐν τοῖς δούλοις αὐτοῦ.
\vs{26}Ἐξαπέστειλε Μωυσῆν τὸν δοῦλον αὐτοῦ, Ἀαρὼν ὃν ἐξελέξατο αὐτόν.

\vs{27}Ἔθετο ἐν αὐτοῖς τοὺς λόγους τῶν σημείων αὐτοῦ, καὶ τῶν τεράτων ἐν γῇ Χάμ.
\vs{28}Ἐξαπέστειλε σκότος καὶ ἐσκότασε, καὶ παρεπίκραναν τοὺς λόγους αὐτοῦ·
\vs{29}Μετέστρεψε τὰ ὕδατα αὐτῶν εἰς αἷμα, καὶ ἀπέκτεινε τοὺς ἰχθύας αὐτῶν.
\vs{30}Ἐξῆρψεν ἡ γῆ αὐτῶν βατράχους, ἐν τοῖς ταμείοις τῶν βασιλέων αὐτῶν.
\vs{31}Εἶπε καὶ ἦλθε κυνόμυια, καὶ σκνίπες ἐν πᾶσι τοῖς ὁρίοις αὐτῶν.
\vs{32}Ἔθετο τὰς βροχὰς αὐτῶν χάλαζαν, πῦρ καταφλέγον ἐν τῇ γῇ αὐτῶν.
\vs{33}Καὶ ἐπάταξε τὰς ἀμπέλους αὐτῶν καὶ τὰς συκὰς αὐτῶν, καὶ συνέτριψε πᾶν ξύλον ὁρίου αὐτῶν.
\vs{34}Εἶπε καὶ ἦλθεν ἀκρὶς, καὶ βροῦχος οὗ οὐκ ἦν ἀριθμὸς,
\vs{35}καὶ κατέφαγε πάντα τὸν χόρτον ἐν τῇ γῇ αὐτῶν, καὶ κατέφαγε τὸν καρπὸν τῆς γῆς αὐτῶν.
\vs{36}Καὶ ἐπάταξε πᾶν πρωτότοκον ἐκ τῆς γῆς αὐτῶν, ἀπαρχὴν παντὸς πόνου αὐτῶν.
\vs{37}Καὶ ἐξήγαγεν αὐτοὺς ἐν ἀργυρίῳ καὶ χρυσίῳ, καὶ οὐκ ἦν ἐν ταῖς φυλαῖς αὐτῶν ὁ ἀσθενῶν.
\vs{38}Εὐφράνθη Αἴγυπτος ἐν τῇ ἐξόδῳ αὐτῶν, ὅτι ἐπέπεσεν ὁ φόβος αὐτῶν ἐπʼ αὐτούς.
\vs{39}Διεπέτασε νεφέλην εἰς σκέπην αὐτοῖς, καὶ πῦρ τοῦ φωτίσαι αὐτοῖς τὴν νύκτα.
\vs{40}Ἤτησαν, καὶ ἦλθεν ὀρτυγομήτρα, καὶ ἄρτον οὐρανοῦ ἐνέπλησεν αὐτούς.
\vs{41}Διέῤῥηξε πέτραν, καὶ ἐῤῥύησαν ὕδατα, ἐπορεύθησαν ἐν ἀνύδροις ποταμοί.

\vs{42}Ὅτι ἐμνήσθη τοῦ λόγου τοῦ ἁγίου αὐτοῦ, τοῦ πρὸς Ἀβραὰμ τὸν δοῦλον αὐτοῦ·
\vs{43}Καὶ ἐξήγαγε τὸν λαὸν αὐτοῦ ἐν ἀγαλλιάσει, καὶ τοὺς ἐκλεκτοὺς αὐτοῦ ἐν εὐφροσύνῃ·
\vs{44}Καὶ ἔδωκεν αὐτοῖς χώρας ἐθνῶν, καὶ πόνους λαῶν ἐκληρονόμησαν.
\vs{45}Ὅπως ἂν φυλάξωσι τὰ δικαιώματα αὐτοῦ, καὶ τὸν νόμον αὐτοῦ ἐκζητήσωσιν.

\begin{psalmheading}{\ch{105}{106} Ἀλληλούϊα.}
\end{psalmheading}
Ἐξομολογεῖσθε τῷ Κυρίῳ, ὅτι χρηστὸς, ὅτι εἰς τὸν αἰῶνα τὸ ἔλεος αὐτοῦ.
\vs{2}Τίς λαλήσει τὰς δυναστείας τοῦ Κυρίου, ἀκουστὰς ποιήσει πάσας τὰς αἰνέσεις αὐτοῦ;
\vs{3}Μακάριοι οἱ φυλάσσοντες κρίσιν, καὶ ποιοῦντες δικαιοσύνην ἐν παντὶ καιρῷ.

\vs{4}Μνήσθητι ἡμῶν Κύριε ἐν τῇ εὐδοκίᾳ τοῦ λαοῦ σου, ἐπίσκεψαι ἡμᾶς ἐν τῷ σωτηρίῳ σου·
\vs{5}τοῦ ἰδεῖν ἐν τῇ χρηστότητι τῶν ἐκλεκτῶν σου, τοῦ εὐφρανθῆναι ἐν τῇ εὐφροσύνῃ τοῦ ἔθνους σου, τοῦ ἐπαινεῖσθαι μετὰ τῆς κληρονομίας σου.

\vs{6}Ἡμάρτομεν μετὰ τῶν πατέρων ἡμῶν, ἠνομήσαμεν, ἠδικήσαμεν.
\vs{7}Οἱ πατέρες ἡμῶν ἐν Αἰγύπτῳ οὐ συνῆκαν τὰ θαυμάσιά σου, καὶ οὐκ ἐμνήσθησαν τοῦ πλήθους τοῦ ἐλέους σου· καὶ παρεπίκραναν ἀναβαίνοντες ἐν τῇ ἐρυθρᾷ θαλάσσῃ.
\vs{8}Καὶ ἔσωσεν αὐτοὺς ἕνεκεν τοῦ ὀνόματος αὐτοῦ, τοῦ γνωρίσαι τὴν δυναστείαν αὐτοῦ.
\vs{9}Καὶ ἐπετίμησε τῇ ἐρυθρᾷ θαλάσσῃ, καὶ ἐξηράνθη· καὶ ὡδήγησεν αὐτοὺς ἐν ἀβύσσῳ ὡς ἐν ἐρήμῳ·
\vs{10}Καὶ ἔσωσεν αὐτοὺς ἐκ χειρὸς μισούντων, καὶ ἐλυτρώσατο αὐτοὺς ἐκ χειρὸς ἐχθροῦ.
\vs{11}Ἐκάλυψεν ὕδωρ τοὺς θλίβοντας αὐτοὺς, εἷς ἐξ αὐτῶν οὐχ ὑπελείφθη.
\vs{12}Καὶ ἐπίστευσαν τοῖς λόγοις αὐτοῦ, καὶ ᾔνεσαν τὴν αἴνεσιν αὐτοῦ.
\vs{13}Ἐτάχυναν, ἐπελάθοντο τῶν ἔργων αὐτοῦ, οὐχ ὑπέμειναν τὴν βουλὴν αὐτοῦ.
\vs{14}Καὶ ἐπεθύμησαν ἐπιθυμίαν ἐν τῇ ἐρήμῳ, καὶ ἐπείρασαν τὸν Θεὸν ἐν ἀνύδρῳ.
\vs{15}Καὶ ἔδωκεν αὐτοῖς τὸ αἴτημα αὐτῶν, καὶ ἐξαπέστειλε πλησμονὴν εἰς τὴν ψυχὴν αὐτῶν.

\vs{16}Καὶ παρώργισαν Μωυσῆν ἐν τῇ παρεμβολῇ, καὶ Ἀαρὼν τὸν ἅγιον Κυρίου.
\vs{17}Ἠνοίχθη ἡ γῆ καὶ κατέπιε Δαθὰν, καὶ ἐκάλυψεν ἐπὶ τὴν συναγωγὴν Ἀβειρών.
\vs{18}Καὶ ἐξεκαύθη πῦρ ἐν τῇ συναγωγῇ αὐτῶν, καὶ φλὸξ κατέφλεξεν ἁμαρτωλούς.

\vs{19}Καὶ ἐποίησαν μόσχον ἐν Χωρὴβ, καὶ προσεκύνησαν τῷ γλυπτῷ·
\vs{20}Καὶ ἠλλάξαντο τὴν δόξαν αὐτῶν ἐν ὁμοιώματι μόσχου ἔσθοντος χόρτον.
\vs{21}Ἐπελάθοντο τοῦ Θεοῦ τοῦ σώζοντος αὐτοὺς, τοῦ ποιήσαντος μεγάλα ἐν Αἰγύπτῳ,
\vs{22}θαυμαστὰ ἐν γῇ Χὰμ, καὶ φοβερὰ ἐπὶ θαλάσσης ἐρυθρᾶς.
\vs{23}Καὶ εἶπε τοῦ ἐξολοθρεῦσαι αὐτοὺς, εἰ μὴ Μωυσῆς ὁ ἐκλεκτὸς αὐτοῦ ἔστη ἐν τῇ θραύσει ἐνώπιον αὐτοῦ, τοῦ ἀποστρέψαι ἀπὸ θυμοῦ ὀργῆς αὐτοῦ, τοῦ μὴ ἐξολοθρεῦσαι.

\vs{24}Καὶ ἐξουδένωσαν γῆν ἐπιθυμητὴν, καὶ οὐκ ἐπίστευσαν τῷ λόγῳ αὐτοῦ.
\vs{25}Καὶ ἐγόγγυσαν ἐν τοῖς σκηνώμασιν αὐτῶν, οὐκ εἰσήκουσαν τῆς φωνῆς Κυρίου.
\vs{26}Καὶ ἐπῇρε τὴν χεῖρα αὐτοῦ ἐπʼ αὐτοὺς, τοῦ καταβαλεῖν αὐτοὺς ἐν τῇ ἐρήμῳ,
\vs{27}καὶ τοῦ καταβαλεῖν τὸ σπέρμα αὐτῶν ἐν τοῖς ἔθνεσι, καὶ διασκορπίσαι αὐτοὺς ἐν ταῖς χώραις.

\vs{28}Καὶ ἐτελέσθησαν τῷ Βεελφεγὼρ, καὶ ἔφαγον θυσίας νεκρῶν.
\vs{29}Καὶ παρώξυναν αὐτὸν ἐν τοῖς ἐπιτηδεύμασιν αὐτῶν, καὶ ἐπληθύνθη ἐν αὐτοῖς ἡ πτῶσις.
\vs{30}Καὶ ἔστη Φινεὲς καὶ ἐξιλάσατο, καὶ ἐκόπασεν ἡ θραῦσις.
\vs{31}Καὶ ἐλογίσθη αὐτῷ εἰς δικαιοσύνην, εἰς γενεὰν καὶ γενεὰν ἕως τοῦ αἰῶνος.

\vs{32}Καὶ παρώργισαν αὐτὸν ἐπὶ ὕδατος ἀντιλογίας, καὶ ἐκακώθη Μωυσῆς διʼ αὐτούς·
\vs{33}Ὅτι παρεπίκραναν τὸ πνεῦμα αὐτοῦ, καὶ διέστειλεν ἐν τοῖς χείλεσιν αὐτοῦ.

\vs{34}Οὐκ ἐξωλόθρευσαν τὰ ἔθνη ἃ εἶπε Κύριος αὐτοῖς.
\vs{35}Καὶ ἐμίγησαν ἐν τοῖς ἔθνεσι, καὶ ἔμαθον τὰ ἔργα αὐτῶν.
\vs{36}Καὶ ἐδούλευσαν τοῖς γλυπτοῖς αὐτῶν, καὶ ἐγενήθη αὐτοῖς εἰς σκάνδαλον.
\vs{37}Καὶ ἔθυσαν τοὺς υἱοὺς αὐτῶν καὶ τὰς θυγατέρας αὐτῶν τοῖς δαιμονίοις,
\vs{38}καὶ ἐξέχεαν αἷμα ἀθῶον, αἷμα υἱῶν αὐτῶν καὶ θυγατέρων, ὧν ἔθυσαν τοῖς γλυπτοῖς Χαναάν· καὶ ἐφονοκτονήθη ἡ γῆ ἐν τοῖς αἵμασι,
\vs{39}καὶ ἐμιάνθη ἐν τοῖς ἔργοις αὐτῶν· καὶ ἐπόρνευσαν ἐν τοῖς ἐπιτηδεύμασιν αὐτῶν.

\vs{40}Καὶ ὠργίσθη θυμῷ Κύριος ἐπὶ τὸν λαὸν αὐτοῦ, καὶ ἐβδελύξατο τὴν κληρονομίαν αὐτοῦ.
\vs{41}Καὶ παρέδωκεν αὐτοὺς εἰς χεῖρας ἐχθρῶν, καὶ ἐκυρίευσαν αὐτῶν οἱ μισοῦντες αὐτούς.
\vs{42}Καὶ ἔθλιψαν αὐτοὺς οἱ ἐχθροὶ αὐτῶν, καὶ ἐταπεινώθησαν ὑπὸ τὰς χεῖρας αὐτῶν.
\vs{43}Πλεονάκις ἐῤῥύσατο αὐτοὺς, αὐτοὶ δὲ παρεπίκραναν αὐτὸν ἐν τῇ βουλῇ αὐτῶν· καὶ ἐταπεινώθησαν ἐν ταῖς ἀνομίαις αὐτῶν.
\vs{44}Καὶ εἶδε Κύριος ἐν τῷ θλίβεσθαι αὐτοὺς, ἐν τῷ αὐτὸν εἰσακοῦσαι τῆς δεήσεως αὐτῶν.
\vs{45}Καὶ ἐμνήσθη τῆς διαθήκης αὐτοῦ, καὶ μετεμελήθη κατὰ τὸ πλῆθος τοῦ ἐλέους αὐτοῦ.
\vs{46}Καὶ ἔδωκεν αὐτοὺς εἰς οἰκτιρμοὺς ἐναντίον πάντων τῶν αἰχμαλωτευσάντων αὐτούς.

\vs{47}Σῶσον ἡμᾶς Κύριε ὁ Θεὸς ἡμῶν, καὶ ἐπισυνάγαγε ἡμᾶς ἐκ τῶν ἐθνῶν, τοῦ ἐξομολογήσασθαι τῷ ὀνόματί σου τῷ ἁγίῳ, τοῦ ἐγκαυχᾶσθαι ἐν τῇ αἰνέσει σου.
\vs{48}Εὐλογητὸς Κύριος ὁ Θεὸς Ἰσραὴλ, ἀπὸ τοῦ αἰῶνος καὶ ἕως τοῦ αἰῶνος· καὶ ἐρεῖ πᾶς ὁ λαὸς, γένοιτο, γένοιτο.

\begin{psalmheading}{\ch{106}{107} Ἀλληλούϊα.}
\end{psalmheading}
Ἐξομολογεῖσθε τῷ Κυρίῳ, ὅτι χρηστὸς, ὅτι εἰς τὸν αἰῶνα τὸ ἔλεος αὐτοῦ.
\vs{2}Εἰπάτωσαν οἱ λελυτρωμένοι ὑπὸ Κυρίου, οὓς ἐλυτρώσατο ἐκ χειρὸς ἐχθροῦ,
\vs{3}καὶ ἐκ τῶν χωρῶν συνήγαγεν αὐτούς· ἀπὸ ἀνατολῶν, καὶ δυσμῶν, καὶ βοῤῥᾶ, καὶ θαλάσσης.

\vs{4}Ἐπλανήθησαν ἐν τῇ ἐρήμῳ ἐν ἀνύδρῳ· ὁδὸν πόλεως κατοικητηρίου οὐχ εὗρον·
\vs{5}Πεινῶντες καὶ διψῶντες, ἡ ψυχὴ αὐτῶν ἐν αὐτοῖς ἐξέλιπε.
\vs{6}Καὶ ἐκέκραξαν πρὸς Κύριον ἐν τῷ θλίβεσθαι αὐτοὺς, καὶ ἐκ τῶν ἀναγκῶν αὐτῶν ἐῤῥύσατο αὐτούς·
\vs{7}Καὶ ὡδήγησεν αὐτοὺς εἰς ὁδὸν εὐθεῖαν, τοῦ πορευθῆναι εἰς πόλιν κατοικητηρίου.

\vs{8}Ἐξομολογησάσθωσαν τῷ Κυρίῳ τὰ ἐλέη αὐτοῦ, καὶ τὰ θαυμάσια αὐτοῦ τοῖς υἱοῖς τῶν ἀνθρώπων.
\vs{9}Ὅτι ἐχόρτασε ψυχὴν κενὴν, καὶ πεινῶσαν ἐνέπλησεν ἀγαθῶν.
\vs{10}Καθημένους ἐν σκότει καὶ σκιᾷ θανάτου, πεπεδημένους ἐν πτωχείᾳ καὶ σιδήρῳ·
\vs{11}Ὅτι παρεπίκραναν τὰ λόγια τοῦ Θεοῦ, καὶ τὴν βουλὴν τοῦ ὑψίστου παρώξυναν·
\vs{12}Καὶ ἐταπεινώθη ἐν κόποις ἡ καρδία αὐτῶν, ἠσθένησαν καὶ οὐκ ἦν ὁ βοηθῶν.
\vs{13}Καὶ ἐκέκραξαν πρὸς Κύριον ἐν τῷ θλίβεσθαι αὐτοὺς, καὶ ἐκ τῶν ἀναγκῶν αὐτῶν ἔσωσεν αὐτούς.
\vs{14}Καὶ ἐξήγαγεν αὐτοὺς ἐκ σκότους καὶ σκιᾶς θανάτου, καὶ τοὺς δεσμοὺς αὐτῶν διέῤῥηξεν.

\vs{15}Ἐξομολογησάσθωσαν τῷ Κυρίῳ τὰ ἐλέη αὐτοῦ, καὶ τὰ θαυμάσια αὐτοῦ τοῖς υἱοῖς τῶν ἀνθρώπων.
\vs{16}Ὅτι συνέτριψε πύλας χαλκᾶς, καὶ μοχλοὺς σιδηροὺς συνέθλασεν.

\vs{17}Ἀντελάβετο αὐτῶν ἐξ ὁδοῦ ἀνομίας αὐτῶν, διὰ γὰρ τὰς ἀνομίας αὐτῶν ἐταπεινώθησαν.
\vs{18}Πᾶν βρῶμα ἐβδελύξατο ἡ ψυχὴ αὐτῶν, καὶ ἤγγισαν ἕως τῶν πυλῶν τοῦ θανάτου.
\vs{19}Καὶ ἐκέκραξαν πρὸς Κύριον ἐν τῷ θλίβεσθαι αὐτοὺς, καὶ ἐκ τῶν ἀναγκῶν αὐτῶν ἔσωσεν αὐτούς.
\vs{20}Ἀπέστειλε τὸν λόγον αὐτοῦ, καὶ ἰάσατο αὐτοὺς, καὶ ἐῤῥύσατο αὐτοὺς ἐκ τῶν διαφθορῶν αὐτῶν.

\vs{21}Ἐξομολογησάσθωσαν τῷ Κυρίῳ τὰ ἐλέη αὐτοῦ, καὶ τὰ θαυμάσια αὐτοῦ τοῖς υἱοῖς τῶν ἀνθρώπων·
\vs{22}Καὶ θυσάτωσαν αὐτῷ θυσίαν αἰνέσεως, καὶ ἐξαγγειλάτωσαν τὰ ἔργα αὐτοῦ ἐν ἀγαλλιάσει.

\vs{23}Οἱ καταβαίνοντες εἰς θάλασσαν ἐν πλοίοις, ποιοῦντες ἐργασίαν ἐν ὕδασι πολλοῖς,
\vs{24}αὐτοὶ εἶδον τὰ ἔργα Κυρίου, καὶ τὰ θαυμάσια αὐτοῦ ἐν τῷ βυθῷ.
\vs{25}Εἶπε, καὶ ἔστη πνεῦμα καταιγίδος, καὶ ὑψώθη τὰ κύματα αὐτῆς.
\vs{26}Ἀναβαίνουσιν ἕως τῶν οὐρανῶν, καὶ καταβαίνουσιν ἕως τῶν ἀβύσσων· ἡ ψυχὴ αὐτῶν ἐν κακοῖς ἐτήκετο,
\vs{27}ἐταράχθησαν, ἐσαλεύθησαν ὡς ὁ μεθύων, καὶ πᾶσα ἡ σοφία αὐτῶν κατεπόθη.
\vs{28}Καὶ ἐκέκραξαν πρὸς Κύριον ἐν τῷ θλίβεσθαι αὐτοὺς, καὶ ἐκ τῶν ἀναγκῶν αὐτῶν ἐξήγαγεν αὐτούς.
\vs{29}Καὶ ἐπέταξε τῇ καταιγίδι, καὶ ἔστη εἰς αὖραν, καὶ ἐσίγησαν τὰ κύματα αὐτῆς.
\vs{30}Καὶ εὐφράνθησαν, ὅτι ἡσύχασαν, καὶ ὡδήγησεν αὐτοὺς ἐπὶ λιμένα θελήματος αὐτῶν.

\vs{31}Ἐξομολογησάσθωσαν τῷ Κυρίῳ τὰ ἐλέη αὐτοῦ, καὶ τὰ θαυμάσια αὐτοῦ τοῖς υἱοῖς τῶν ἀνθρώπων.
\vs{32}Ὑψωσάτωσαν αὐτὸν ἐν ἐκκλησίᾳ λαοῦ, καὶ ἐν καθέδρᾳ πρεσβυτέρων αἰνεσάτωσαν αὐτόν.

\vs{33}Ἔθετο ποταμοὺς εἰς ἐρήμον, καὶ διεξόδους ὑδάτων εἰς δίψαν·
\vs{34}Γῆν καρποφόρον εἰς ἅλμην, ἀπὸ κακίας τῶν κατοικούντων ἐν αὐτῇ.
\vs{35}Ἔθετο ἔρημον εἰς λίμνας ὑδάτων, καὶ γῆν ἄνυδρον εἰς διεξόδους ὑδάτων.
\vs{36}Καὶ κατῴκισεν ἐκεῖ πεινῶντας, καὶ συνεστήσαντο πόλεις κατοικεσίας·
\vs{37}Καὶ ἔσπειραν ἀγροὺς, καὶ ἐφύτευσαν ἀμπελῶνας, καὶ ἐποίησαν καρπὸν γεννήματος.
\vs{38}Καὶ εὐλόγησεν αὐτοὺς, καὶ ἐπληθύνθησαν σφόδρα, καὶ τὰ κτήνη αὐτῶν οὐκ ἐσμίκρυνε.
\vs{39}Καὶ ὠλιγώθησαν καὶ ἐκακώθησαν ἀπὸ θλίψεως κακῶν καὶ ὀδύνης.
\vs{40}Ἐξεχύθη ἐξουδένωσις ἐπʼ ἄρχοντας αὐτῶν, καὶ ἐπλάνησεν αὐτοὺς ἐν ἀβάτῳ καὶ οὐχ ὁδῷ.
\vs{41}Καὶ ἐβοήθησε πένητι ἐκ πτωχείας, καὶ ἔθετο ὡς πρόβατα πατριάς·
\vs{42}Ὄψονται εὐθεῖς καὶ εὐφρανθήσονται, καὶ πᾶσα ἀνομία ἐμφράξει τὸ στόμα αὐτῆς.
\vs{43}Τίς σοφὸς καὶ φυλάξει ταῦτα, καὶ συνήσει τὰ ἐλέη τοῦ Κυρίου;

\begin{psalmheading}{\ch{107}{108} Ὠδὴ ψαλμοῦ τῷ Δαυίδ.}
\end{psalmheading}
\vs{2}Ἑτοίμη ἡ καρδία μου ὁ Θεὸς, ἑτοίμη ἡ καρδία μου, ᾄσομαι καὶ ψαλῶ ἐν τῇ δόξῃ μου.
\vs{3}Ἐξεγέρθητι ψαλτήριον καὶ κιθάρα, ἐξεγερθήσομαι ὄρθρου.
\vs{4}Ἐξομολογήσομαί σοι ἐν λαοῖς Κύριε, ψαλῶ σοι ἐν ἔθνεσιν.
\vs{5}Ὅτι μέγα ἐπάνω τῶν οὐρανῶν τὸ ἔλεός σου, καὶ ἕως τῶν νεφελῶν ἡ ἀλήθειά σου.
\vs{6}Ὑψώθητι ἐπὶ τοὺς οὐρανοὺς, ὁ Θεὸς, καὶ ἐπὶ πᾶσαν τὴν γῆν ἡ δόξα σου.
\vs{7}Ὅπως ἂν ῥυσθῶσιν οἱ ἀγαπητοί σου, σῶσον τῇ δεξιᾷ σου, καὶ ἐπάκουσόν μου.
\vs{8}Ὁ Θεὸς ἐλάλησεν ἐν τῷ ἁγίῳ αὐτοῦ, ὑψωθήσομαι καὶ διαμεριῶ Σίκιμα, καὶ τὴν κοιλάδα τῶν σκηνῶν διαμετρήσω.
\vs{9}Ἐμός ἐστι Γαλαὰδ, καὶ ἐμός ἐστι Μανασσῆς, καὶ Ἐφραίμ ἀντίληψις τῆς κεφαλῆς μου· Ἰούδας βασιλεύς μου,
\vs{10}Μωὰβ λέβης τῆς ἐλπίδος μου· ἐπὶ τὴν Ἰδουμαίαν ἐπιβαλῶ τὸ ὑπόδημά μου, ἐμοὶ ἀλλόφυλοι ὑπετάγησαν.

\vs{11}Τίς ἀπάξει με εἰς πόλιν περιοχῆς; ἢ τίς ὁδηγήσει με ἕως τῆς Ἰδουμαίας;
\vs{12}Οὐχὶ σὺ ὁ Θεὸς ὁ ἀπωσάμενος ἡμᾶς; καὶ οὐκ ἐξελεύσῃ ὁ Θεὸς ἐν ταῖς δυνάμεσιν ἡμῶν;
\vs{13}Δὸς ἡμῖν βοήθειαν ἐκ θλίψεως, καὶ ματαία σωτηρία ἀνθρώπου.
\vs{14}Ἐν τῷ Θεῷ ποιήσομεν δύναμιν, καὶ αὐτὸς ἐξουδενώσει τοὺς ἐχθροὺς ἡμῶν.

\begin{psalmheading}{\ch{108}{109} Εἰς τὸ τέλος, ψαλμὸς τῷ Δαυίδ.}
\end{psalmheading}
Ὁ Θεὸς τὴν αἴνεσίν μου μὴ παρασιωπήσῃς,
\vs{2}ὅτι στόμα ἁμαρτωλοῦ καὶ στόμα δολίου ἐπʼ ἐμὲ ἠνοίχθη· ἐλάλησαν κατʼ ἐμοῦ γλώσσῃ δολίᾳ,
\vs{3}καὶ λόγοις μίσους ἐκύκλωσάν με, καὶ ἐπολέμησάν με δωρέαν.
\vs{4}Ἀντὶ τοῦ ἀγαπᾷν με, ἐνδιέβαλλόν με, ἐγὼ δὲ προσηυχόμην.
\vs{5}Καὶ ἔθεντο κατʼ ἐμοῦ κακὰ ἀντὶ ἀγαθῶν, καὶ μῖσος ἀντὶ τῆς ἀγαπήσεώς μου.

\vs{6}Κατάστησον ἐπʼ αὐτὸν ἁμαρτωλὸν, καὶ διάβολος στήτω ἐκ δεξιῶν αὐτοῦ.
\vs{7}Ἐν τῷ κρίνεσθαι αὐτὸν, ἐξέλθοι καταδεδικασμένος, καὶ ἡ προσευχὴ αὐτοῦ γενέσθω εἰς ἁμαρτίαν.
\vs{8}Γενηθήτωσαν αἱ ἡμέραι αὐτοῦ ὀλίγαι, καὶ τὴν ἐπισκοπὴν αὐτοῦ λάβοι ἕτερος.
\vs{9}Γενηθήτωσαν οἱ υἱοὶ αὐτοῦ ὀρφανοὶ, καὶ ἡ γυνὴ αὐτοῦ χήρα.
\vs{10}Σαλευόμενοι μεταναστήτωσαν οἱ υἱοὶ αὐτοῦ, καὶ ἐπαιτησάτωσαν, ἐκβληθήτωσαν ἐκ τῶν οἰκοπέδων αὐτῶν.
\vs{11}Ἐξερευνησάτω δανειστὴς πάντα ὅσα ὑπάρχει αὐτῷ, καὶ διαρπασάτωσαν ἀλλότριοι τοὺς πόνους αὐτοῦ.
\vs{12}Μὴ ὑπαρξάτω αὐτῷ ἀντιλήμπτωρ, μηδὲ γενηθήτω οἰκτίρμων τοῖς ὀρφανοῖς αὐτοῦ.
\vs{13}Γενηθήτω τὰ τέκνα αὐτοῦ εἰς ἐξολόθρευσιν, ἐν γενεᾷ μιᾷ ἐξαλειφθείη τὸ ὄνομα αὐτοῦ.
\vs{14}Ἀναμνησθείη ἡ ἀνομία τῶν πατέρων αὐτοῦ ἔναντι Κυρίου, καὶ ἡ ἁμαρτία τῆς μητρὸς αὐτοῦ μὴ ἐξαλειφθείη.
\vs{15}Γενηθήτωσαν ἐναντίον Κυρίου διαπαντὸς, καὶ ἐξολοθρευθείη ἐκ γῆς τὸ μνημόσυνον αὐτῶν·

\vs{16}Ἀνθʼ ὧν οὐκ ἐμνήσθη ποιῆσαι ἔλεος, καὶ κατεδίωξεν ἄνθρωπον πένητα καὶ πτωχὸν, καὶ κατανενυγμένον τῇ καρδίᾳ τοῦ θανατῶσαι.
\vs{17}Καὶ ἠγάπησε κατάραν, καὶ ἥξει αὐτῷ, καὶ οὐκ ἠθέλησεν εὐλογίαν, καὶ μακρυνθήσεται ἀπʼ αὐτοῦ.
\vs{18}Καὶ ἐνεδύσατο κατάραν ὡς ἱμάτιον, καὶ εἰσῆλθεν ὡσεὶ ὕδωρ εἰς τὰ ἔγκατα αὐτοῦ, καὶ ὡσεὶ ἔλαιον ἐν τοῖς ὀστέοις αὐτοῦ.
\vs{19}Γενηθήτω αὐτῷ ὡς ἱμάτιον ὃ περιβάλλεται, καὶ ὡσεὶ ζώνη ἣν διαπαντὸς περιζώννυται.
\vs{20}Τοῦτο τὸ ἔργον τῶν ἐνδιαβαλλόντων με παρὰ Κυρίου, καὶ τῶν λαλούντων πονηρὰ κατὰ τῆς ψυχῆς μου.

\vs{21}Καὶ σὺ Κύριε Κύριε ποίησον μετʼ ἐμοῦ ἕνεκεν τοῦ ὀνόματός σου, ὅτι χρηστὸν τὸ ἔλεός σου.
\vs{22}Ῥῦσαί με ὅτι πτωχὸς καὶ πένης εἰμὶ ἐγὼ, καὶ ἡ καρδία μου τετάρακται ἐντός μου.
\vs{23}Ὡσεὶ σκιὰ ἐν τῷ ἐκκλῖναι αὐτὴν ἀντανῃρέθην, ἐξετινάχθην ὡσεὶ ἀκρίδες.
\vs{24}Τὰ γόνατά μου ἠσθένησαν ἀπὸ νηστείας, καὶ ἡ σάρξ μου ἠλλοιώθη διʼ ἔλαιον.
\vs{25}Κᾀγὼ ἐγενήθην ὄνειδος αὐτοῖς· εἴδοσάν με, ἐσάλευσαν κεφαλὰς αὐτῶν.

\vs{26}Βοήθησόν μοι Κύριε ὁ Θεός μου, καὶ σῶσόν με κατὰ τὸ ἔλεός σου.
\vs{27}Καὶ γνώτωσαν ὅτι ἡ χείρ σου αὕτη, καὶ σὺ Κύριε ἐποίησας αὐτήν.
\vs{28}Καταράσονται αὐτοὶ, καὶ σὺ εὐλογήσεις· οἱ ἐπανιστάμενοί μοι αἰσχυνθήτωσαν, ὁ δὲ δοῦλός σου εὐφρανθήσεται.
\vs{29}Ἐνδυσάσθωσαν οἱ ἐνδιαβάλλοντές με ἐντροπήν· καὶ περιβαλέσθωσαν ὡς διπλοΐδα αἰσχύνην αὐτῶν.
\vs{30}Ἐξομολογήσομαι τῷ Κυρίῳ σφόδρα ἐν τῷ στόματί μου, καὶ ἐν μέσῳ πολλῶν αἰνέσω αὐτόν·
\vs{31}Ὅτι παρέστη ἐκ δεξιῶν πένητος, τοῦ σῶσαι ἐκ τῶν καταδιωκόντων τὴν ψυχήν μου.

\begin{psalmheading}{\ch{109}{110} Ψαλμὸς τῷ Δαυίδ.}
\end{psalmheading}
Εἶπεν ὁ Κύριος τῷ Κυρίῳ μου, κάθου ἐκ δεξιῶν μου, ἕως ἂν θῶ τοὺς ἐχθρούς σου ὑποπόδιον τῶν ποδῶν σου.
\vs{2}Ῥάβδον δυνάμεως ἐξαποστελεῖ σοι Κύριος ἐκ Σιὼν, κατακυρίευε ἐν μέσῳ τῶν ἐχθρῶν σου.
\vs{3}Μετὰ σοῦ ἡ ἀρχὴ ἐν ἡμέρᾳ τῆς δυνάμεώς σου, ἐν ταῖς λαμπρότησι τῶν ἁγίων σου· ἐκ γαστρὸς πρὸ Ἑωσφόρου ἐγέννησά σε.
\vs{4}Ὤμοσε Κύριος καὶ οὐ μεταμεληθήσεται, σὺ ἱερεὺς εἰς τὸν αἰῶνα, κατὰ τὴν τάξιν Μελχισεδέκ.
\vs{5}Κύριος ἐκ δεξιῶν σου συνέθλασεν ἐν ἡμέρᾳ ὀργῆς αὐτοῦ βασιλεῖς.
\vs{6}Κρινεῖ ἐν τοῖς ἔθνεσι, πληρώσει πτώματα, συνθλάσει κεφαλὰς ἐπὶ γῆν πολλῶν.
\vs{7}Ἐκ χειμάῤῥου ἐν ὁδῷ πίεται, διὰ τοῦτο ὑψώσει κεφαλήν.

\begin{psalmheading}{\ch{110}{111} Ἀλληλούϊα.}
\end{psalmheading}
Ἐξομολογήσομαι σοι Κύριε ἐν ὅλῃ καρδίᾳ μου, ἐν βουλῇ εὐθέων καὶ συναγωγῇ.
\vs{2}Μεγάλα τὰ ἔργα Κυρίου, ἐξεζητημένα εἰς πάντα τὰ θελήματα αὐτοῦ.
\vs{3}Ἐξομολόγησις καὶ μεγαλοπρέπεια τὸ ἔργον αὐτοῦ, καὶ ἡ δικαιοσύνη αὐτοῦ μένει εἰς τὸν αἰῶνα τοῦ αἰῶνος.
\vs{4}Μνείαν ἐποιήσατο τῶν θαυμασίων αὐτοῦ, ἐλεήμων καὶ οἰκτίρμων ὁ Κύριος.
\vs{5}Τροφὴν ἔδωκε τοῖς φοβουμένοις αὐτόν· μνησθήσεται εἰς τὸν αἰῶνα διαθήκης αὐτοῦ.
\vs{6}Ἰσχὺν ἔργων αὐτοῦ ἀνήγγειλε τῷ λαῷ αὐτοῦ, τοῦ δοῦναι αὐτοῖς κληρονομίαν ἐθνῶν.
\vs{7}Ἔργα χειρῶν αὐτοῦ, ἀλήθεια καὶ κρίσις· πισταὶ πᾶσαι αἱ ἐντολαὶ αὐτοῦ,
\vs{8}ἐστηριγμέναι εἰς τὸν αἰῶνα τοῦ αἰῶνος, πεποιημέναι ἐν ἀληθείᾳ καὶ εὐθύτητι.
\vs{9}Λύτρωσιν ἀπέστειλε τῷ λαῷ αὐτοῦ· ἐνετείλατο εἰς τὸν αἰῶνα διαθήκην αὐτοῦ· ἅγιον καὶ φοβερὸν τὸ ὄνομα αὐτοῦ.
\vs{10}Ἀρχὴ σοφίας φόβος Κυρίου, σύνεσις δὲ ἀγαθὴ πᾶσι τοῖς ποιοῦσιν αὐτήν· ἡ αἴνεσις αὐτοῦ μένει εἰς τὸν αἰῶνα τοῦ αἰῶνος.

\begin{psalmheading}{\ch{111}{112} Ἀλληλούϊα.}
\end{psalmheading}
Μακάριος ἀνὴρ ὁ φοβούμενος τὸν Κύριον, ἐν ταῖς ἐντολαῖς αὐτοῦ θελήσει σφόδρα.
\vs{2}Δυνατὸν ἐν τῇ γῇ ἔσται τὸ σπέρμα αὐτοῦ, γενεὰ εὐθέων εὐλογηθήσεται·
\vs{3}Δόξα καὶ πλοῦτος ἐν τῷ οἴκῳ αὐτοῦ, καὶ ἡ δικαιοσύνη αὐτοῦ μένει εἰς τὸν αἰῶνα τοῦ αἰῶνος.
\vs{4}Ἐξανέτειλεν ἐν σκότει φῶς τοῖς εὐθέσιν· ἐλεήμων καὶ οἰκτίρμων καὶ δίκαιος.
\vs{5}Χρηστὸς ἀνὴρ ὁ οἰκτείρων καὶ κιχρῶν, οἰκονομήσει τοὺς λόγους αὐτοῦ ἐν κρίσει,
\vs{6}ὅτι εἰς τὸν αἰῶνα οὐ σαλευθήσεται· εἰς μνημόσυνον αἰώνιον ἔσται δίκαιος.
\vs{7}Ἀπὸ ἀκοῆς πονηρᾶς οὐ φοβηθήσεται· ἑτοίμη ἡ καρδία αὐτοῦ ἐλπίζειν ἐπὶ Κύριον·
\vs{8}Ἐστήρικται ἡ καρδία αὐτοῦ, οὐ φοβηθῇ, ἕως οὗ ἐπίδῃ ἐπὶ τοὺς ἐχθροὺς αὐτοῦ.
\vs{9}Ἐσκόρπισεν, ἔδωκε τοῖς πένησιν, ἡ δικαιοσύνη αὐτοῦ μένει εἰς τὸν αἰῶνα τοῦ αἰῶνος· τὸ κέρας αὐτοῦ ὑψωθήσεται ἐν δόξῃ.
\vs{10}Ἁμαρτωλὸς ὄψεται καὶ ὀργισθήσεται, τοὺς ὀδόντας αὐτοῦ βρύξει καὶ τακήσεται· ἐπιθυμία ἁμαρτωλοῦ ἀπολεῖται.

\begin{psalmheading}{\ch{112}{113} Ἀλληλούϊα.}
\end{psalmheading}
Αἰνεῖτε παῖδες Κύριον, αἰνεῖτε τὸ ὄνομα Κυρίου.
\vs{2}Εἴη τὸ ὄνομα Κυρίου εὐλογημένον ἀπὸ τοῦ νῦν καὶ ἕως τοῦ αἰῶνος.
\vs{3}Ἀπὸ ἀνατολῶν ἡλίου μέχρι δυσμῶν, αἰνετὸν τὸ ὄνομα Κυρίου.
\vs{4}Ὑψηλὸς ἐπὶ πάντα τὰ ἔθνη ὁ Κύριος, ἐπὶ τοὺς οὐρανοὺς ἡ δόξα αὐτοῦ.

\vs{5}Τίς ὡς Κύριος ὁ Θεὸς ἡμῶν; ὁ ἐν ὑψηλοῖς κατοικῶν,
\vs{6}καὶ τὰ ταπεινὰ ἐφορῶν ἐν τῷ οὐρανῷ, καὶ ἐν τῇ γῇ·
\vs{7}ὁ ἐγείρων ἀπὸ γῆς πτωχὸν, καὶ ἀπὸ κοπρίας ἀνυψῶν πένητα,
\vs{8}τοῦ καθίσαι αὐτὸν μετὰ ἀρχόντων, μετὰ ἀρχόντων λαοῦ αὐτοῦ·
\vs{9}ὁ κατοικίζων στείραν ἐν οἴκῳ, μητέρα ἐπὶ τέκνοις εὐφραινομένην.

\begin{psalmheading}{\ch{113}{114-115} Ἀλληλούϊα.}
\end{psalmheading}
Ἐν ἐξόδῳ Ἰσραὴλ ἐξ Αἰγύπτου, οἴκου Ἰακὼβ ἐκ λαοῦ βαρβάρου,
\vs{2}ἐγενήθη Ἰουδαία ἁγίασμα αὐτοῦ, Ἰσραὴλ ἐξουσία αὐτοῦ.
\vs{3}Ἡ θάλασσα εἶδε καὶ ἔφυγεν, ὁ Ἰορδάνης ἐστράφη εἰς τὰ ὀπίσω.
\vs{4}Τὰ ὄρη ἐσκίρτησαν ὡσεὶ κριοὶ, καὶ οἱ βουνοὶ ὡς ἀρνία προβάτων.

\vs{5}Τί σοι ἐστὶ θάλασσα ὅτι ἔφυγες; καὶ σὺ Ἰορδάνη ὅτι ἐστράφης εἰς τὰ ὀπίσω;
\vs{6}Τὰ ὄρη ὅτι ἐσκίρτησατε ὡσεὶ κριοί; καὶ οἱ βουνοὶ ὡς ἀρνία προβάτων;
\vs{7}Ἀπὸ προσώπου Κυρίου ἐσαλεύθη ἡ γῆ, ἀπὸ προσώπου τοῦ Θεοῦ Ἰακὼβ,
\vs{8}τοῦ στρέψαντος τὴν πέτραν εἰς λίμνας ὑδάτων, καὶ τὴν ἀκρότομον εἰς πηγὰς ὑδάτων.
\vs{9}Μὴ ἡμῖν Κύριε, μὴ ἡμῖν, ἀλλʼ ἢ τῷ ὀνόματί σου δὸς δόξαν, ἐπὶ τῷ ἐλέει σου καὶ τῇ ἀληθείᾳ σου·
\vs{10}μή ποτε εἴπωσι τὰ ἔθνη, ποῦ ἐστιν ὁ Θεὸς αὐτῶν;
\vs{11}Ὁ δὲ Θεὸς ἡμῶν ἐν τῷ οὐρανῷ καὶ ἐν τῇ γῇ, πάντα ὅσα ἠθέλησεν ἐποίησε.

\vs{12}Τὰ εἴδωλα τῶν ἐθνῶν, ἀργύριον καὶ χρυσίον, ἔργα χειρῶν ἀνθρώπων.
\vs{13}Στόμα ἔχουσι καὶ οὐ λαλήσουσιν, ὀφθαλμοὺς ἔχουσι καὶ οὐκ ὄψονται·
\vs{14}Ὦτα ἔχουσι καὶ οὐκ ἀκούσονται, ῥῖνας ἔχουσι καὶ οὐκ ὀσφρανθήσονται·
\vs{15}Χεῖρας ἔχουσι καὶ οὐ ψηλαφήσουσι, πόδας ἔχουσι καὶ οὐ περιπατήσουσιν, οὐ φωνήσουσιν ἐν τῷ λάρυγγι αὐτῶν.
\vs{16}ὅμοιοι αὐτοῖς γένοιντο οἱ ποιοῦντες αὐτὰ, καὶ πάντες οἱ πεποιθότες ἐπʼ αὐτοῖς.

\vs{17}Οἶκος Ἰσραὴλ ἔλπισεν ἐπὶ Κύριον, βοηθὸς καὶ ὑπερασπιστὴς αὐτῶν ἐστιν.
\vs{18}Οἶκος Ἀαρὼν ἤλπισεν ἐπὶ Κύριον, βοηθὸς καὶ ὑπερασπιστὴς αὐτῶν ἐστιν.
\vs{19}Οἱ φοβούμενοι τὸν Κύριον ἤλπισαν ἐπὶ Κύριον, βοηθὸς καὶ ὑπερασπιστὴς αὐτῶν ἐστι.

\vs{20}Κύριος μνησθεὶς ἡμῶν εὐλόγησεν ἡμᾶς, εὐλόγησε τὸν οἶκον Ἰσραὴλ, εὐλόγησε τὸν οἶκον Ἀαρών·
\vs{21}Εὐλόγησε τοὺς φοβουμένους τὸν Κύριον, τοὺς μικροὺς μετὰ τῶν μεγάλων.
\vs{22}Προσθείη Κύριος ἐφʼ ὑμᾶς, ἐφʼ ὑμᾶς καὶ ἐπὶ τοὺς υἱοὺς ὑμῶν.
\vs{23}Εὐλογημένοι ὑμεῖς τῷ Κυρίῳ, τῷ ποιήσαντι τὸν οὐρανὸν καὶ τὴν γῆν.

\vs{24}Ὁ οὐρανὸς τοῦ οὐρανοῦ τῷ Κυρίῳ, τὴν δὲ γῆν ἔδωκε τοῖς υἱοῖς τῶν ἀνθρώπων.
\vs{25}Οὐχ οἱ νεκροὶ αἰνέσουσί σε Κύριε, οὐδὲ πάντες οἱ καταβαίνοντες εἰς ᾅδου·
\vs{26}Ἀλλʼ ἡμεῖς οἱ ζῶντες εὐλογήσομεν τὸν Κύριον, ἀπὸ τοῦ νῦν καὶ ἕως τοῦ αἰῶνος.

\begin{psalmheading}{\ch{114}{116:1-9} Ἀλληλούϊα.}
\end{psalmheading}
Ἠγαπήσα, ὅτι εἰσακούσεται Κύριος τῆς φωνῆς τῆς δεήσεώς μου·
\vs{2}Ὅτι ἔκλινε τὸ οὖς αὐτοῦ ἐμοὶ, καὶ ἐν ταῖς ἡμέραις μου ἐπικαλέσομαι.
\vs{3}Περιέσχον με ὠδῖνες θανάτου, κίνδυνοι ᾅδου εὕροσάν με· θλίψιν καὶ ὀδύνην εὗρον,
\vs{4}καὶ τὸ ὄνομα Κυρίου ἐπεκαλεσάμην, ὦ Κύριε ῥῦσαι τὴν ψυχήν μου.

\vs{5}Ἐλεήμων ὁ Κύριος καὶ δίκαιος, καὶ ὁ Θεὸς ἡμῶν ἐλεεῖ.
\vs{6}Φυλάσσων τὰ νήπια ὁ Κύριος, ἐταπεινώθην καὶ ἔσωσέ με.

\vs{7}Ἐπίστρεψον ψυχή μου εἰς τὴν ἀνάπαυσίν σου, ὅτι Κύριος εὐηργέτησέ σε.
\vs{8}Ὅτι ἐξείλετο τὴν ψυχήν μου ἐκ θανάτου, τοὺς ὀφθαλμούς μου ἀπὸ δακρύων, καὶ τοὺς πόδας μου ἀπὸ ὀλισθήματος.
\vs{9}Εὐαρεστήσω ἐνώπιον Κυρίου ἐν χώρᾳ ζώντων.

\begin{psalmheading}{\ch{115}{116:10-19} Ἀλληλούϊα.}
\end{psalmheading}
Ἐπιστεύσα, διὸ ἐλάλησα, ἐγὼ δὲ ἐταπεινώθην σφόδρα.
\vs{2}Ἐγὼ δὲ εἶπα ἐν τῇ ἐκστάσει μου, πᾶς ἄνθρωπος ψεύστης.

\vs{3}Τί ἀνταποδώσω τῷ Κυρίῳ περὶ πάντων ὧν ἀνταπέδωκέ μοι;
\vs{4}Ποτήριον σωτηρίου λήψομαι, καὶ τὸ ὄνομα Κυρίου ἐπικαλέσομαι·
\vs{4a}Τὰς εὐχάς μου τῷ Κυρίῳ ἀποδώσω, ἐναντίον παντὸς τοῦ λαοῦ αὐτοῦ.

\vs{6}Τίμιος ἐναντίον Κυρίου ὁ θάνατος τῶν ὁσίων αὐτοῦ.
\vs{7}Ὦ Κύριε ἐγὼ δοῦλος σὸς, ἐγὼ δοῦλος σὸς, καὶ υἱὸς τῆς παιδίσκης σου, διέῤῥηξας τοὺς δεσμούς μου.
\vs{8}Σοὶ θύσω θυσίαν αἰνέσεως, καὶ ἐν ὀνόματι Κυρίου ἐπικαλέσομαι·
\vs{9}Τὰς εὐχάς μου τῷ Κυρίῳ ἀποδώσω, ἐναντίον παντὸς τοῦ λαοῦ αὐτοῦ,
\vs{10}ἐν αὐλαῖς οἴκου Κυρίου, ἐν μέσῳ σου Ἱερουσαλήμ.

\begin{psalmheading}{\ch{116}{117} Ἀλληλούϊα.}
\end{psalmheading}
Αἰνεῖτε τὸν Κύριον πάντα τὰ ἔθνη, ἐπαινέσατε αὐτὸν πάντες οἱ λαοί.
\vs{2}Ὅτι ἐκραταιώθη τὸ ἔλεος αὐτοῦ ἐφʼ ἡμᾶς, καὶ ἡ ἀλήθεια τοῦ Κυρίου μένει εἰς τὸν αἰῶνα.

\begin{psalmheading}{\ch{117}{118} Ἀλληλούϊα.}
\end{psalmheading}
Ἐξομολογεῖσθε τῷ Κυρίῳ, ὅτι ἀγαθὸς, ὅτι εἰς τὸν αἰῶνα τὸ ἔλεος αὐτοῦ.
\vs{2}Εἰπάτω δὴ οἶκος Ἰσραὴλ, ὅτι ἀγαθὸς, ὅτι εἰς τὸν αἰῶνα τὸ ἔλεος αὐτοῦ.
\vs{3}Εἰπάτω δὴ οἶκος Ἀαρὼν, ὅτι ἀγαθὸς, ὅτι εἰς τὸν αἰῶνα τὸ ἔλεος αὐτοῦ.
\vs{4}Εἰπάτωσαν δὴ πάντες οἱ φοβούμενοι τὸν Κύριον, ὅτι ἀγαθὸς, ὅτι εἰς τὸν αἰῶνα τὸ ἔλεος αὐτοῦ.

\vs{5}Ἐκ θλίψεως ἐπεκαλεσάμην τὸν Κύριον, καὶ ἐπήκουσέ μου εἰς πλατυσμόν.
\vs{6}Κύριος ἐμοὶ βοηθὸς, καὶ οὐ φοβηθήσομαι τί ποιησει μοι ἄνθρωπος.
\vs{7}Κύριος ἐμοὶ βοηθὸς, κᾀγὼ ἐπόψομαι τοὺς ἐχθρούς μου.
\vs{8}Ἀγαθὸν πεποιθέναι ἐπὶ Κύριον, ἢ πεποιθέναι ἐπʼ ἄνθρωπον.
\vs{9}Ἀγαθὸν ἐλπίζειν ἐπὶ Κύριον, ἢ ἐλπίζειν ἐπʼ ἄρχουσι.

\vs{10}Πάντα τὰ ἔθνη ἐκύκλωσάν με, καὶ τῷ ὀνόματι Κυρίου ἠμυνάμην αὐτούς·
\vs{11}Κυκλώσαντες ἐκύκλωσάν με, καὶ τῷ ὀνόματι Κυρίου ἠμυνάμην αὐτούς·
\vs{12}Ἐκύκλωσάν με ὡσεὶ μέλισσαι κηρίον, καὶ ἐξεκαύθησαν ὡς πῦρ ἐν ἀκάνθαις, καὶ τῷ ὀνόματι Κυρίου ἠμυνάμην αὐτούς.
\vs{13}Ὠσθεὶς ἀνετράπην τοῦ πεσεῖν, καὶ ὁ Κύριος ἀντελάβετό μου.

\vs{14}Ἰσχύς μου καὶ ὕμνησίς μου ὁ Κύριος, καὶ ἐγένετό μοι εἰς σωτηρίαν.
\vs{15}Φωνὴ ἀγαλλιάσεως καὶ σωτηρίας ἐν σκηναῖς δικαίων· δεξιὰ Κυρίου ἐποίησε δύναμιν,
\vs{16}δεξιὰ Κυρίου ὕψωσέ με· δεξιὰ Κυρίου ἐποίησε δύναμιν.
\vs{17}Οὐκ ἀποθανοῦμαι, ἀλλὰ ζήσομαι, καὶ διηγήσομαι τὰ ἔργα Κυρίου.
\vs{18}Παιδεύων ἐπαίδευσέ με ὁ Κύριος, καὶ τῷ θανάτῳ οὐ παρέδωκέ με.

\vs{19}Ἀνοίξατέ μοι πύλας δικαιοσύνης, εἰσελθὼν ἐν αὐταῖς ἐξομολογήσομαι τῷ Κυρίῳ.
\vs{20}Αὕτη ἡ πύλη τοῦ Κυρίου, δίκαιοι εἰσελεύσονται ἐν αὐτῇ.
\vs{21}Ἐξομολογήσομαί σοι, ὅτι ἐπήκουσάς μου, καὶ ἐγένου μου εἰς σωτηρίαν.
\vs{22}Λίθον ὃν ἀπεδοκίμασαν οἱ οἰκοδομοῦντες, οὗτος ἐγενήθη εἰς κεφαλὴν γωνίας.
\vs{23}Παρὰ Κυρίου ἐγένετο αὕτη, καὶ ἔστι θαυμαστὴ ἐν ὀφθαλμοῖς ἡμῶν.
\vs{24}Αὕτη ἡ ἡμέρα ἣν ἐποίησεν ὁ Κύριος, ἀγαλλιασώμεθα καὶ εὐφρανθῶμεν ἐν αὐτῇ.
\vs{25}Ὦ Κύριε σῶσον δὴ, ὦ Κύριε εὐόδωσον δή.
\vs{26}Εὐλογημένος ὁ ἐρχόμενος ἐν ὀνόματι Κυρίου· εὐλογήκαμεν ὑμᾶς ἐξ οἴκου Κυρίου.

\vs{27}Θεὸς Κύριος, καὶ ἐπέφανεν ἡμῖν· συστήσασθε ἑορτὴν ἐν τοῖς πυκάζουσιν, ἕως τῶν κεράτων τοῦ θυσιαστηρίου.
\vs{28}Θεός μου εἶ σὺ, καὶ ἐξομολογήσομαί σοι· Θεός μου εἶ σὺ, καὶ ὑψώσω σε· ἐξομολογήσομαί σοι, ὅτι ἐπήκουσάς μου, καὶ ἐγένου μοι εἰς σωτηρίαν.
\vs{29}Ἐξομολογεῖσθε τῷ Κυρίῳ, ὅτι ἀγαθὸς, ὅτι εἰς τὸν αἰῶνα τὸ ἔλεος αὐτοῦ.

\begin{psalmheading}{\ch{118}{119} Ἀλληλούϊα.}
\end{psalmheading}
Μακάριοι ἄμωμοι ἐν ὁδῷ, οἱ πορευόμενοι ἐν νόμῳ Κυρίου.
\vs{2}Μακάριοι οἱ ἐξερευνῶντες τὰ μαρτύρια αὐτοῦ, ἐν ὅλῃ καρδίᾳ ἐκζητήσουσιν αὐτόν.
\vs{3}Οὐ γὰρ οἱ ἐργαζόμενοι τὴν ἀνομίαν ἐν ταῖς ὁδοῖς αὐτοῦ ἐπορεύθησαν.
\vs{4}Σὺ ἐνετείλω τὰς ἐντολάς σου, τοῦ φυλάξασθαι σφόδρα.
\vs{5}Ὄφελον κατευθυνθείησαν αἱ ὁδοί μου, τοῦ φυλάξασθαι τὰ δικαιώματά σου.
\vs{6}Τότε οὐ μὴ αἰσχυνθῶ, ἐν τῷ με ἐπιβλέπειν ἐπὶ πάσας τὰς ἐντολάς σου.
\vs{7}Ἐξομολογήσομαί σοι ἐν εὐθύτητι καρδίας, ἐν τῷ μεμαθηκέναι με τὰ κρίματα τῆς δικαιοσύνης σου.
\vs{8}Τὰ δικαιώματά σου φυλάξω, μή με ἐγκαταλίπῃς ἕως σφόδρα.

\vs{9}Ἐν τίνι κατορθώσει νεώτερος τὴν ὁδὸν αὐτοῦ; ἐν τῷ φυλάξασθαι τοὺς λόγους σου.
\vs{10}Ἐν ὅλῃ καρδίᾳ μου ἐξεζήτησά σε, μὴ ἀπώσῃ με ἀπὸ τῶν ἐντολῶν σου.
\vs{11}Ἐν τῇ καρδίᾳ μου ἔκρυψα τὰ λόγιά σου, ὅπως ἂν μὴ ἁμάρτω σοι.
\vs{12}Εὐλογητὸς εἶ Κύριε, δίδαξόν με τὰ δικαιώματά σου.
\vs{13}Ἐν τοῖς χείλεσί μου ἐξήγγειλα πάντα τὰ κρίματα τοῦ στόματός σου.
\vs{14}Ἐν τῇ ὁδῷ τῶν μαρτυρίων σου ἐτέρφθην, ὡς ἐπὶ παντὶ πλούτῳ.
\vs{15}Ἐν ταῖς ἐντολαῖς σου ἀδολεσχήσω, καὶ κατανοήσω τὰς ὁδούς σου.
\vs{16}Ἐν τοῖς δικαιώμασί σου μελετήσω, οὐκ ἐπιλήσομαι τῶν λόγων σου.

\vs{17}Ἀνταπόδος τῷ δούλῳ σου, ζήσομαι καὶ φυλάξω τοὺς λόγους σου.
\vs{18}Ἀποκάλυψον τοὺς ὀφθαλμούς μου, καὶ κατανοήσω τὰ θαυμάσια ἐκ τοῦ νόμου σου.
\vs{19}Πάροικος ἐγώ εἰμι ἐν τῇ γῇ, μὴ ἀποκρύψῃς ἀπʼ ἐμοῦ τὰς ἐντολάς σου.
\vs{20}Ἐπεπόθησεν ἡ ψυχή μου τοῦ ἐπιθυμῆσαι τὰ κρίματά σου ἐν παντὶ καιρῷ.
\vs{21}Ἐπετίμησας ὑπερηφάνοις, ἐπικατάρατοι οἱ ἐκκλίνοντες ἀπὸ τῶν ἐντολῶν σου.
\vs{22}Περίελε ἀπʼ ἐμοῦ ὄνειδος καὶ ἐξουδένωσιν, ὅτι τὰ μαρτύριά σου ἐξεζήτησα.
\vs{23}Καὶ γὰρ ἐκάθισαν ἄρχοντες, καὶ κατʼ ἐμοῦ κατελάλουν, ὁ δὲ δοῦλός σου ἠδολέσχει ἐν τοῖς δικαιώμασί σου.
\vs{24}Καὶ γὰρ τὰ μαρτύριά σου μελέτη μου ἐστὶ, καὶ αἱ συμβουλίαι μου τὰ δικαιώματά σου.

\vs{25}Ἐκολλήθη τῷ ἐδάφει ἡ ψυχή μου, ζῆσόν με κατὰ τὸ λόγον σου.
\vs{26}Τὰς ὁδούς σου ἐξήγγειλα καὶ ἐπήκουσάς μου, δίδαξόν με τὰ δικαιώματά σου.
\vs{27}Ὁδὸν δικαιωμάτων σου συνέτισόν με, καὶ ἀδολεσχήσω ἐν τοῖς θαυμασίοις σου.
\vs{28}Ἐνύσταξεν ἡ ψυχή μου ἀπὸ ἀκηδίας, βεβαίωσόν με ἐν τοῖς λόγοις σου.
\vs{29}Ὁδὸν ἀδικίας ἀπόστησον ἀπʼ ἐμοῦ, καὶ τῷ νόμῳ σου ἐλέησόν με.
\vs{30}Ὁδὸν ἀληθείας ᾑρετισάμην, καὶ τὰ κρίματά σου οὐκ ἐπελαθόμην.
\vs{31}Ἐκολλήθην τοῖς μαρτυρίοις σου Κύριε, μή με καταισχύνῃς.
\vs{32}Ὁδὸν ἐντολῶν σου ἔδραμον, ὅταν ἐπλάτυνας τὴν καρδίαν μου.

\vs{33}Νομοθέτησόν με Κύριε τὴν ὁδὸν τῶν δικαιωμάτων σου, καὶ ἐκζητήσω αὐτὴν διαπαντός.
\vs{34}Συνέτισόν με, καὶ ἐξερευνήσω τὸν νόμον σου, καὶ φυλάξω αὐτὸν ἐν ὅλῃ καρδίᾳ μου.
\vs{35}Ὁδήγησόν με ἐν τῇ τρίβῳ τῶν ἐντολῶν σου, ὅτι αὐτὴν ἠθέλησα.
\vs{36}Κλῖνον τὴν καρδίαν μου εἰς τὰ μαρτύριά σου, καὶ μὴ εἰς πλεονεξίαν.
\vs{37}Ἀπόστρεψον τοὺς ὀφθαλμούς μου τοῦ μὴ ἰδεῖν ματαιότητα, ἐν τῇ ὁδῷ σου ζῆσόν με.
\vs{38}Στῆσον τῷ δούλῳ σου τὸ λόγιόν σου εἰς τὸν φόβον σου.
\vs{39}Περίελε τὸν ὄνειδισμόν μου ὃν ὑπώπτευσα, ὅτι τὰ κρίματά σου χρηστά.
\vs{40}Ἰδοὺ ἐπεθύμησα τὰς ἐντολάς σου, ἐν τῇ δικαιοσύνῃ σου ζῆσόν με.

\vs{41}Καὶ ἔλθοι ἐπʼ ἐμὲ τὸ ἔλεός σου Κύριε, τὸ σωτήριόν σου κατὰ τὸν λόγον σου.
\vs{42}Καὶ ἀποκριθήσομαι τοῖς ὀνειδίζουσί μοι λόγον, ὅτι ἤλπισα ἐπὶ τοῖς λόγοις σου.
\vs{43}Καὶ μὴ περιέλῃς ἐκ τοῦ στόματός μου λόγον ἀληθείας ἕως σφόδρα, ὅτι ἐπὶ τοῖς κρίμασί σου ἐπήλπισα.
\vs{44}Καὶ φυλάξω τὸν νόμον σου διαπαντὸς, εἰς τὸν αἰῶνα καὶ εἰς τὸν αἰῶνα τοῦ αἰῶνος.
\vs{45}Καὶ ἐπορευόμην ἐν πλατυσμῷ, ὅτι τὰς ἐντολάς σου ἐξεζήτησα.
\vs{46}Καὶ ἐλάλουν ἐν τοῖς μαρτυρίοις σου ἐναντίον βασιλέων, καὶ οὐκ ᾐσχυνόμην.
\vs{47}Καὶ ἐμελέτων ἐν ταῖς ἐντολαῖς σου, αἷς ἠγάπησα σφόδρα.
\vs{48}Καὶ ᾖρα τὰς χεῖράς μου πρὸς τὰς ἐντολάς σου ἃς ἠγάπησα, καὶ ἠδολέσχουν ἐν τοῖς δικαιώμασί σου.

\vs{49}Μνήσθητι τῶν λόγων σου τῷ δούλῳ σου ὧν ἐπήλπισάς με.
\vs{50}Αὕτη με παρεκάλεσεν ἐν τῇ ταπεινώσει μου, ὅτι τὸ λόγιόν σου ἔζησέ με.
\vs{51}Ὑπερήφανοι παρηνόμουν ἕως σφόδρα, ἀπὸ δὲ τοῦ νόμου σου οὐκ ἐξέκλινα.
\vs{52}Ἐμνήσθην τῶν κριμάτων σου ἀπʼ αἰῶνος Κύριε, καὶ παρεκλήθην.
\vs{53}Ἀθυμία κατέσχε με ἀπὸ ἁμαρτωλῶν τῶν ἐγκαταλιμπανόντων τὸν νόμον σου.
\vs{54}Ψαλτὰ ἦσάν μοι τὰ δικαιώματά σου, ἐν τόπῳ παροικίας μου·
\vs{55}Ἐμνήσθην ἐν νυκτὶ τοῦ ὀνόματός σου Κύριε, καὶ ἐφύλαξα τὸν νόμον σου.
\vs{56}Αὕτη ἐγενήθη μοι, ὅτι τὰ δικαιώματά σου ἐξεζήτησα.

\vs{57}Μερίς μου εἶ Κύριε, εἶπα τοῦ φυλάξασθαι τὸν νόμον σου.
\vs{58}Ἐδεήθην τοῦ προσώπου σου ἐν ὅλῃ καρδίᾳ μου, ἐλέησόν με κατὰ τὸ λόγιόν σου.
\vs{59}Διελογισάμην τὰς ὁδούς σου, καὶ ἐπέστρεψα τοὺς πόδας μου εἰς τὰ μαρτύριά σου.
\vs{60}Ἡτοιμάσθην καὶ οὐκ ἐταράχθην, τοῦ φυλάξασθαι τὰς ἐντολάς σου.
\vs{61}Σχοινία ἁμαρτωλῶν περιεπλάκησάν μοι, καὶ τοῦ νόμου σου οὐκ ἐπελαθόμην.
\vs{62}Μεσονύκτιον ἐξεγειρόμην, τοῦ ἐξομολογεῖσθαί σοι ἐπὶ τὰ κρίματα τῆς δικαιοσύνης σου.
\vs{63}Μέτοχος ἐγώ εἰμι πάντων τῶν φοβουμένων σε, καὶ τῶν φυλασσόντων τὰς ἐντολάς σου.
\vs{64}Τοῦ ἐλέους σου Κύριε πλήρης ἡ γῆ, τὰ δικαιώματά σου δίδαξόν με.

\vs{65}Χρηστότητα ἐποίησας μετὰ τοῦ δούλου σου Κύριε κατὰ τὸν λόγον σου.
\vs{66}Χρηστότητα καὶ παιδείαν καὶ γνῶσιν δίδαξόν με, ὅτι ταῖς ἐντολαῖς σου ἐπίστευσα.
\vs{67}Πρὸ τοῦ με ταπεινωθῆναι, ἐγὼ ἐπλημμέλησα, διὰ τοῦτο τὸ λόγιόν σου ἐφύλαξα.
\vs{68}Χρηστὸς εἶ σὺ Κύριε· καὶ ἐν τῇ χρηστότητί σου δίδαξόν με τὰ δικαιωματά σου.
\vs{69}Ἐπληθύνθη ἐπʼ ἐμὲ ἀδικία ὑπερηφάνων, ἐγὼ δὲ ἐν ὅλῃ καρδίᾳ μου ἐξερευνήσω τὰς ἐντολάς σου.
\vs{70}Ἐτυρώθη ὡς γάλα ἡ καρδία αὐτῶν, ἐγὼ δὲ τὸν νόμον σου ἐμελέτησα.
\vs{71}Ἀγαθόν μοι ὅτι ἐταπείνωσάς με, ὅπως ἂν μάθω τὰ δικαιώματά σου.
\vs{72}Ἀγαθόν μοι ὁ νόμος τοῦ στόματός σου, ὑπὲρ χιλιάδας χρυσίου καὶ ἀργυρίου.

\vs{73}Αἱ χεῖρές σου ἔποίησάν με καὶ ἐπλασάν με, συνέτισόν με καὶ μαθήσομαι τὰς ἐντολάς σου.
\vs{74}Οἱ φοβούμενοί σε ὄψονταί με καὶ εὐφρανθήσονται, ὅτι εἰς τοὺς λόγους σου ἐπήλπισα.
\vs{75}Ἔγνων Κύριε ὅτι δικαιοσύνη τὰ κρίματά σου, καὶ ἀληθείᾳ ἐταπείνωσάς με.
\vs{76}Γενηθήτω δὴ τὸ ἔλεός σου τοῦ παρακαλέσαι με, κατὰ τὸ λόγιόν σου τῷ δούλῳ σου.
\vs{77}Ἐλθέτωσάν μοι οἱ οἰκτιρμοί σου, καὶ ζήσομαι, ὅτι ὁ νόμος σου μελέτη μου ἐστίν.
\vs{78}Αἰσχυνθήτωσαν ὑπερήφανοι, ὅτι ἀδίκως ἠνόμησαν εἰς ἐμὲ, ἐγὼ δὲ ἀδολεσχήσω ἐν ταῖς ἐντολαῖς σου.
\vs{79}Ἐπιστρεψάτωσάν με οἱ φοβούμενοί σε, καὶ οἱ γινώσκοντες τὰ μαρτύριά σου.
\vs{80}Γενηθήτω ἡ καρδία μου ἄμωμος ἐν τοῖς δικαιώμασί σου, ὅπως ἂν μὴ αἰσχυνθῶ.

\vs{81}Ἐκλείπει εἰς τὸ σωτήριόν σου ἡ ψυχή μου, εἰς τοὺς λόγους σου ἐπήλπισα.
\vs{82}Ἐξέλιπον οἱ ὀφθαλμοί μου εἰς τὸ λόγιόν σου, λέγοντες, πότε παρακαλέσεις με;
\vs{83}Ὅτι ἐγενήθην ὡς ἀσκὸς ἐν πάχνῃ· τὰ δικαιώματά σου οὐκ ἐπελαθόμην.
\vs{84}Πόσαι εἰσὶν αἱ ἡμέραι τοῦ δούλου σου; πότε ποιήσεις μοι ἐκ τῶν καταδιωκόντων με κρίσιν;
\vs{85}Διηγήσαντό μοι παράνομοι ἀδολεσχίας, ἀλλʼ οὐχ ὡς ὁ νόμος σου Κύριε.
\vs{86}Πᾶσαι αἱ ἐντολαί σου ἀλήθεια· ἀδίκως κατεδίωξάν με, βοήθησόν μοι.
\vs{87}Παρὰ βραχὺ συνετέλεσάν με ἐν τῇ γῇ, ἐγὼ δὲ οὐκ ἐγκατέλιπον τὰς ἐντολάς σου.
\vs{88}Κατὰ τὸ ἔλεός σου ζῆσόν με, καὶ φυλάξω τὰ μαρτύρια τοῦ στόματός σου.

\vs{89}Εἰς τὸν αἰῶνα, Κύριε, ὁ λόγος σου διαμένει ἐν τῷ οὐρανῷ,
\vs{90}εἰς γενεὰν καὶ γενεὰν ἡ ἀλήθειά σου· ἐθεμελίωσας τὴν γῆν καὶ διαμένει.
\vs{91}Τῇ διατάξει σου διαμένει ἡμέρα, ὅτι τὰ σύμπαντα δοῦλα σά.
\vs{92}Εἰ μὴ ὅτι ὁ νόμος σου μελέτη μου ἐστὶ, τότε ἂν ἀπωλόμην ἐν τῇ ταπεινώσει μου.
\vs{93}Εἰς τὸν αἰῶνα οὐ μὴ ἐπιλάθωμαι τῶν δικαιωμάτων σου, ὅτι ἐν αὐτοῖς ἔζησάς με.
\vs{94}Σός εἰμι ἐγὼ, σῶσόν με, ὅτι τὰ δικαιώματά σου ἐξεζήτησα.
\vs{95}Ἐμὲ ὑπέμειναν ἁμαρτωλοὶ τοῦ ἀπολέσαι με, τὰ μαρτύριά σου συνῆκα.
\vs{96}Πάσης συντελείας εἶδον πέρας, πλατεία ἡ ἐντολή σου σφόδρα.

\vs{97}Ὡς ἠγάπησα τὸν νόμον σου Κύριε; ὅλην τὴν ἡμέραν μελέτη μου ἐστίν.
\vs{98}Ὑπὲρ τοὺς ἐχθρούς μου ἐσόφισάς με τὴν ἐντολήν σου, ὅτι εἰς τὸν αἰῶνα ἐμή ἐστιν.
\vs{99}Ὑπὲρ πάντας τοὺς διδάσκοντάς με συνῆκα, ὅτι τὰ μαρτύριά σου μελέτη μου ἐστίν.
\vs{100}Ὑπὲρ πρεσβυτέρους συνῆκα, ὅτι τὰς ἐντολάς σου ἐξεζήτησα.
\vs{101}Ἐκ πάσης ὁδοῦ πονηρᾶς ἐκώλυσα τοὺς πόδας μου, ὅπως ἂν φυλάξω τοὺς λόγους σου.
\vs{102}Ἀπὸ τῶν κριμάτων σου οὐκ ἐξέκλινα, ὅτι σὺ ἐνομοθέτησάς με.

\vs{103}Ὡς γλυκέα τῷ λάρυγγί μου τὰ λόγιά σου, ὑπὲρ μέλι τῷ στόματί μου.
\vs{104}Ἀπὸ τῶν ἐντολῶν σου συνῆκα, διὰ τοῦτο ἐμίσησα πᾶσαν ὁδὸν ἀδικίας.

\vs{105}Λύχνος τοῖς ποσί μου ὁ νόμος σου, καὶ φῶς ταῖς τρίβοις μου.
\vs{106}Ὤμοσα καὶ ἔστησα τοῦ φυλάξασθαι τὰ κρίματα τῆς δικαιοσύνης σου.
\vs{107}Ἐταπεινώθην ἕως σφόδρα Κύριε, ζῆσόν με κατὰ τὸν λόγον σου.
\vs{108}Τὰ ἑκούσια τοῦ στόματός μου εὐλόκησον δὴ Κύριε, καὶ τὰ κρίματά σου δίδαξόν με.
\vs{109}Ἡ ψυχή μου ἐν ταῖς χερσὶ σου διαπαντὸς, καὶ τὸν νόμον σου οὐκ ἐπελαθόμην.
\vs{110}Ἔθεντο ἁμαρτωλοὶ παγίδα μοι, καὶ ἐκ τῶν ἐντολῶν σου οὐκ ἐπλανήθην.
\vs{111}Ἐκληρονόμησα τὰ μαρτύριά σου εἰς τὸν αἰῶνα, ὅτι ἀγαλλιάμα τῆς καρδίας μού εἰσιν.
\vs{112}Ἔκλινα τὴν καρδίαν μου τοῦ ποιῆσαι τὰ δικαιώματά σου εἰς τὸν αἰῶνα διʼ ἀντάμειψιν.

\vs{113}Παρανόμους ἐμίσησα, τὸν δὲ νόμον σου ἠγάπησα.
\vs{114}Βοηθός μου, καὶ ἀντιλήπτωρ μου εἶ σὺ, εἰς τοὺς λόγους σου ἐπήλπισα.
\vs{115}Ἐκκλίνατε ἀπʼ ἐμοῦ οἱ πονηρευόμενοι, καὶ ἐξεραυνήσω τὰς ἐντολὰς τοῦ Θεοῦ μου.
\vs{116}Ἀντιλαβοῦ μου κατὰ τὸ λόγιόν σου, καὶ ζῆσόν με, καὶ μὴ καταισχύνῃς με ἀπὸ τῆς προσδοκίας μου.
\vs{117}Βοήθησόν μοι, καὶ σωθήσομαι, καὶ μελετήσω ἐν τοῖς δικαιώμασί σου διαπαντός.
\vs{118}Ἐξουδένωσας πάντας τοὺς ἀποστατοῦντας ἀπὸ τῶν δικαιωμάτων σου, ὅτι ἄδικον τὸ ἐνθύμημα αὐτῶν.
\vs{119}Παραβαίνοντας ἐλογισάμην πάντας τοὺς ἁμαρτωλοὺς τῆς γῆς, διὰ τοῦτο ἠγάπησα τὰ μαρτύριά σου.
\vs{120}Καθήλωσον ἐκ τοῦ φόβου σου τὰς σάρκας μου, ἀπὸ γὰρ τῶν κριμάτων σου ἐφοβήθην.

\vs{121}Ἐποίησα κρίμα καὶ δικαιοσύνην, μὴ παραδῷς με τοῖς ἀδικοῦσί με.
\vs{122}Ἔνδεξαι τὸν δοῦλόν σου εἰς ἀγαθὸν, μὴ συκοφαντησάτωσάν με ὑπερήφανοι.
\vs{123}Οἱ ὀφθαλμοί μου ἐξέλιπον εἰς τὸ σωτήριόν σου, καὶ εἰς τὸ λόγιον τῆς δικαιοσύνης σου.
\vs{124}Ποίησον μετὰ τοῦ δούλου σου κατὰ τὸ ἔλεός σου, καὶ τὰ δικαιώματά σου δίδαξόν με.
\vs{125}Δοῦλός σου εἰμὶ ἐγὼ, συνέτισόν με καὶ γνώσομαι τὰ μαρτύριά σου.
\vs{126}Καιρὸς τοῦ ποιῆσαι τῷ Κυρίῳ, διεσκέδασαν τὸν νόμον σου.
\vs{127}Διὰ τοῦτο ἠγάπησα τὰς ἐντολάς σου ὑπὲρ χρυσίον καὶ τοπάζιον.
\vs{128}Διὰ τοῦτο πρὸς πάσας τὰς ἐντολάς σου κατωρθούμην, πᾶσαν ὁδὸν ἄδικον ἐμίσησα.

\vs{129}Θαυμαστὰ τὰ μαρτύριά σου, διὰ τοῦτο ἐξηρεύνησεν αὐτὰ ἡ ψυχή μου.
\vs{130}Ἡ δήλωσις τῶν λόγων σου φωτιεῖ καὶ συνετιεῖ νηπίους.
\vs{131}Τὸ στόμα μου ἤνοιξα, καὶ εἵλκυσα πνεῦμα, ὅτι τὰς ἐντολάς σου ἐπεπόθουν.
\vs{132}Ἐπίβλεψον ἐπʼ ἐμὲ καὶ ἐλέησόν με, κατὰ τὸ κρίμα τῶν ἀγαπώντων τὸ ὄνομά σου.
\vs{133}Τὰ διαβήματά μου κατεύθυνον κατὰ τὸ λόγιόν σου, καὶ μὴ κατακυριευσάτω μου πᾶσα ἀνομία.
\vs{134}Λύτρωσαί με ἀπὸ συκοφαντίας ἀνθρώπων, καὶ φυλάξω τὰς ἐντολάς σου.
\vs{135}Τὸ πρόσωπόν σου ἐπίφανον ἐπὶ τὸν δοῦλόν σου, καὶ δίδαξόν με τὰ δικαιώματά σου.
\vs{136}Διεξόδους ὑδάτων κατέβησαν οἱ ὀφθαλμοί μου, ἐπεὶ οὐκ ἐφύλαξα τὸν νόμον σου.

\vs{137}Δίκαιος εἶ Κύριε, καὶ εὐθεῖς αἱ κρίσεις σου.
\vs{138}Ἐνετείλω δικαιοσύνην τὰ μαρτύριά σου, καὶ ἀλήθειαν σφόδρα.
\vs{139}Ἐξέτηξέ με ὁ ζῆλός σου, ὅτι ἐπελάθοντο τῶν λόγων σου οἱ ἐχθροί μου.
\vs{140}Πεπυρωμένον τὸ λόγιόν σου σφόδρα, καὶ ὁ δοῦλός σου ἠγάπησεν αὐτό.
\vs{141}Νεώτερος ἐγὼ εἰμι καὶ ἐξουδενωμένος, τὰ δικαιώματά σου οὐκ ἐπελαθόμην.
\vs{142}Ἡ δικαιοσύνη σου δικαιοσύνη εἰς τὸν αἰῶνα, καὶ ὁ νόυος σου ἀλήθεια.
\vs{143}Θλίψεις καὶ ἀνάγκαι εὕροσάν με, ἐντολαί σου μελέτη μου.
\vs{144}Δικαιοσύνη τὰ μαρτύριά σου εἰς τὸν αἰῶνα· συνέτισόν με, καὶ ζήσομαι.

\vs{145}Ἐκέκραξα ἐν ὅλῃ καρδίᾳ μου, ἐπάκουσόν μου Κύριε, τὰ δικαιώματά σου ἐκζητήσω.
\vs{146}Ἐκέκραξά σοι, σῶσόν με, καὶ φυλάξω τὰ μαρτύριά σου.
\vs{147}Προέφθασα ἐν ἀωρίᾳ καὶ ἐκέκραξα, εἰς τοὺς λόγους σου ἐπήλπισα.
\vs{148}Προέφθασαν οἱ ὀφθαλμοί μου πρὸς ὄρθρον, τοῦ μελετᾷν τὰ λόγιά σου.
\vs{149}Τῆς φωνῆς μου ἄκουσον Κύριε κατὰ τὸ ἔλεός σου, κατὰ τὸ κρίμα σου ζῆσόν με.
\vs{150}Προσήγγισαν οἱ καταδιώκοντές με ἀνομίᾳ, ἀπὸ δὲ τοῦ νόμου σου ἐμακρύνθησαν.
\vs{151}Ἐγγὺς εἶ Κύριε, καὶ πᾶσαι αἱ ὁδοί σου ἀλήθεια.
\vs{152}Κατʼ ἀρχὰς ἔγνων ἐκ τῶν μαρτυρίων σου, ὅτι εἰς τὸν αἰῶνα ἐθεμελίωσας αὐτά.

\vs{153}Ἴδε τὴν ταπείνωσίν μου καὶ ἐξελοῦ με, ὅτι τοῦ νόμου σου οὐκ ἐπελαθόμην.
\vs{154}Κρῖνον τὴν κρίσιν μου καὶ λύτρωσαί με, διὰ τὸν λόγον σου ζῆσόν με.
\vs{155}Μακρὰν ἀπὸ ἁμαρτωλῶν σωτηρία, ὅτι τὰ δικαιώματά σου οὐκ ἐξεζήτησαν.
\vs{156}Οἱ οἰκτιρμοί σου πολλοὶ Κύριε, κατὰ τὸ κρίμα σου ζῆσόν με.
\vs{157}Πολλοὶ οἱ ἐκδιώκοντές με καὶ θλίβοντές με· ἐκ τῶν μαρτυρίων σου οὐκ ἐξέκλινα.
\vs{158}Εἶδον ἀσυνετοῦντας καὶ ἐξετηκόμην, ὅτι τὰ λόγιά σου οὐκ ἐφυλάξαντο.
\vs{159}Ἴδε ὅτι τὰς ἐντολάς σου ἠγάπησα Κύριε, ἐν τῷ ἐλέει σου ζῆσόν με.
\vs{160}Ἀρχὴ τῶν λόγων σου ἀλήθεια, καὶ εἰς τὸν αἰῶνα πάντα τὰ κρίματα τῆς δικαιοσύνης σου.

\vs{161}Ἄρχοντες κατεδίωξάν με δωρεὰν, καὶ ἀπὸ τῶν λόγων σου ἐδειλίασεν ἡ καρδία μου.
\vs{162}Ἀγαλλιάσομαι ἐγὼ ἐπὶ τὰ λόγιά σου, ὡς ὁ εὑρίσκων σκῦλα πολλά.
\vs{163}Ἀδικίαν ἐμίσησα καὶ ἐβδελυξάμην, τὸν δὲ νόμον σου ἠγάπησα.
\vs{164}Ἑπτάκις τῆς ἡμέρας ᾔνεσά σε ἐπὶ τὰ κρίματα τῆς δικαιοσύνης σου.
\vs{165}Εἰρήνη πολλὴ τοῖς ἀγαπῶσι τὸν νόμον σου, καὶ οὐκ ἔστιν αὐτοῖς σκάνδαλον.
\vs{166}Προσεδόκων τὸ σωτήριόν σου Κύριε, καὶ τὰς ἐντολάς σου ἠγάπησα.
\vs{167}Ἐφύλαξεν ἡ ψυχή μου τὰ μαρτύριά σου, καὶ ἠγάπησεν αὐτὰ σφόδρα.
\vs{168}Ἐφύλαξα τὰς ἐντολάς σου καὶ τὰ μαρτύριά σου, ὅτι πᾶσαι αἱ ὁδοί μου ἐναντίον σου Κύριε.

\vs{169}Ἐγγυσάτω ἡ δέησίς μου ἐνώπιόν σου Κύριε, κατὰ τὸ λόγιόν σου συνέτισόν με.
\vs{170}Εἰσέλθοι τὸ ἀξίωμά μου ἐνώπιόν σου Κύριε, κατὰ τὸ λόγιόν σου ῥῦσαί με.
\vs{171}Ἐξερεύξαιντο τὰ χείλη μου ὕμνον, ὅταν διδάξῃς με τὰ δικαιώματά σου.
\vs{172}Φθέγξαιτο ἡ γλῶσσά μου τὰ λόγιά σου, ὅτι πᾶσαι αἱ ἐντολαί σου δικαιοσύνη.
\vs{173}Γενέσθω ἡ χείρ σου τοῦ σῶσαί με, ὅτι τὰς ἐντολάς σου ᾑρετισάμην.
\vs{174}Ἐπεπόθησα τὸ σωτήριόν σου Κύριε, καὶ ὁ νόμος σου μελέτη μου ἐστί.
\vs{175}Ζήσεται ἡ ψυχή μου καὶ αἰνέσει σε, καὶ τὰ κρίματά σου βοηθήσει μοι.
\vs{176}Ἐπλανήθην ὡς πρόβατον ἀπολωλὸς, ζήτησον τὸν δοῦλόν σου, ὅτι τὰς ἐντολάς σου οὐκ ἐπελαθόμην.

\begin{psalmheading}{\ch{119}{120} Ὠδὴ τῶν ἀναβαθμῶν.}
\end{psalmheading}
Πρὸς Κύριον ἐν τῷ θλίβεσθαί με ἐκέκραξα, καὶ εἰσήκουσέ μου.
\vs{2}Κύριε, ῥῦσαι τὴν ψυχήν μου ἀπὸ χειλέων ἀδίκων καὶ ἀπὸ γλώσσης δολίας.

\vs{3}Τί δοθείη σοι, καὶ τί προστεθείη σοι πρὸς γλῶσσαν δολίαν;
\vs{4}Τὰ βέλη τοῦ δυνατοῦ ἠκονημένα σὺν τοῖς ἄνθραξι τοῖς ἐρημικοῖς.

\vs{5}Οἴμοι ὅτι ἡ παροικία μου ἐμακρύνθη, κατεσκήνωσα μετὰ τῶν σκηνωμάτων Κηδάρ.
\vs{6}Πολλὰ παρῴκησεν ἡ ψυχή μου· μετὰ τῶν μισούντων τὴν εἰρήνην.
\vs{7}Ἤμην εἰρηνικός· ὅταν ἐλάλουν αὐτοῖς, ἐπολέμουν με δωρεάν.

\begin{psalmheading}{\ch{120}{121} Ὠδὴ τῶν ἀναβαθμῶν.}
\end{psalmheading}
Ἦρα τοὺς ὀφθαλμούς μου εἰς τὰ ὄρη, ὅθεν ἥξει ἡ βοήθειά μου.
\vs{2}Ἡ βοήθειά μου παρὰ Κυρίου τοῦ ποιήσαντος τὸν οὐρανὸν καὶ τὴν γῆν.
\vs{3}Μὴ δῴης εἰς σάλον τὸν πόδα σου, μηδὲ νυστάξῃ ὁ φυλάσσων σε.
\vs{4}Ἰδοὺ οὐ νυστάξει οὐδὲ ὑπνώσει ὁ φυλάσσων τὸν Ἰσραήλ.
\vs{5}Κύριος φυλάξει σε, Κύριος σκέπη σου ἐπὶ χεῖρα δεξιάν σου.
\vs{6}Ἡμέρας ὁ ἥλιος οὐ συγκαύσει σε, οὐδὲ ἡ σελήνη τὴν νύκτα.
\vs{7}Κύριος φυλάξαι σε ἀπὸ παντὸς κακοῦ, φυλάξει τὴν ψυχήν σου ὁ Κύριος.
\vs{8}Κύριος φυλάξει τὴν εἴσοδόν σου, καὶ τὴν ἔξοδόν σου, ἀπὸ τοῦ νῦν καὶ ἕως τοῦ αἰῶνος.

\begin{psalmheading}{\ch{121}{122} Ὠδὴ τῶν ἀναβαθμῶν.}
\end{psalmheading}
Εὐφράνθην ἐπὶ τοῖς εἰρηκόσι μοι, εἰς οἶκον Κυρίου πορευσόμεθα.
\vs{2}Ἑστῶτες ἦσαν οἱ πόδες ἡμῶν ἐν ταῖς αὐλαῖς σου Ἱερουσαλήμ.
\vs{3}Ἱερουσαλὴμ οἰκοδομουμένη ὡς πόλις, ἧς ἡ μετοχὴ αὐτῆς ἐπιτοαυτό.
\vs{4}Ἐκεῖ γὰρ ἀνέβησαν αἱ φυλαὶ, φυλαὶ Κυρίου μαρτύριον τῷ Ἰσραὴλ, τοῦ ἐξομολογήσασθαι τῷ ὀνόματι Κυρίου.
\vs{5}Ὅτι ἐκεῖ ἐκάθισαν θρόνοι εἰς κρίσιν, θρόνοι ἐπὶ οἶκον Δαυίδ.

\vs{6}Ἐρωτήσατε δὴ τὰ εἰς εἰρήνην τὴν Ἱερουσαλὴμ, καὶ εὐθηνία τοῖς ἀγαπῶσί σε.
\vs{7}Γενέσθω δὴ εἰρήνη ἐν τῇ δυνάμει σου, καὶ εὐθηνία ἐν ταῖς πυργοβάρεσί σου.
\vs{8}Ἕνεκα τῶν ἀδελφῶν μου καὶ τῶν πλησίον μου, ἐλάλουν δὴ εἰρήνην περὶ σοῦ.
\vs{9}Ἕνεκα τοῦ οἴκου Κυρίου τοῦ Θεοῦ ἡμῶν ἐξεζήτησα ἀγαθά σοι.

\begin{psalmheading}{\ch{122}{123} Ὠδὴ τῶν ἀναβαθμῶν.}
\end{psalmheading}
Πρὸς σὲ ᾖρα τοὺς ὀφθαλμούς μου, τὸν κατοικοῦντα ἐν τῷ οὐρανῷ.
\vs{2}Ἰδοὺ ὡς ὀφθαλμοὶ δούλων εἰς χεῖρας τῶν κυρίων αὐτῶν, ὡς ὀφθαλμοὶ παιδίσκης εἰς χεῖρας τῆς κυρίας αὐτῆς, οὕτως οἱ ὀφθαλμοὶ ἡμῶν πρὸς Κύριον τὸν Θεὸν ἡμῶν, ἕως οὗ οἰκτειρῆσαι ἡμᾶς.
\vs{3}Ἐλέησον ἡμᾶς Κύριε, ἐλέησον ἡμᾶς, ὅτι ἐπὶ πολὺ ἐπλήσθημεν ἐξουδενώσεως.
\vs{4}Ἐπὶ πλεῖον ἐπλήσθη ἡ ψυχὴ ἡμῶν· τὸ ὄνειδος τοῖς εὐθηνοῦσι καὶ ἡ ἐξουδένωσις τοῖς ὑπερηφάνοις.

\begin{psalmheading}{\ch{123}{124} Ὠδὴ τῶν ἀναβαθμῶν.}
\end{psalmheading}
Εἰ μὴ ὅτι Κύριος ἦν ἐν ἡμῖν, εἰπάτω δὴ Ἰσραὴλ,
\vs{2}εἰ μὴ ὅτι Κύριος ἦν ἐν ἡμῖν, ἐν τῷ ἐπαναστῆναι ἀνθρώπους ἐφʼ ἡμᾶς,
\vs{3}ἄρα ζῶντας ἂν κατέπιον ἡμᾶς· ἐν τῷ ὀργισθῆναι τὸν θυμὸν αὐτῶν ἐφʼ ἡμᾶς,
\vs{4}ἄρα τὸ ὕδωρ ἂν κατεπόντισεν ἡμᾶς· χείμαῤῥον διῆλθεν ἡ ψυχὴ ἡμῶν.
\vs{5}Ἄρα διῆλθεν ἡ ψυχὴ ἡμῶν τὸ ὕδωρ τὸ ἀνυπόστατον.

\vs{6}Εὐλογητὸς Κύριος, ὃς οὐκ ἔδωκεν ἡμᾶς εἰς θήραν τοῖς ὀδοῦσιν αὐτῶν.
\vs{7}Ἡ ψυχὴ ἡμῶν ὡς στρουθίον ἐῤῥύσθη ἐκ τῆς παγίδος τῶν θηρευόντων· ἡ παγὶς συνετρίβη, καὶ ἡμεῖς ἐῤῥύσθημεν.
\vs{8}Ἡ βοήθεια ἡμῶν ἐν ὀνόματι Κυρίου, τοῦ ποιήσαντος τὸν οὐρανὸν καὶ τὴν γῆν.

\begin{psalmheading}{\ch{124}{125} Ὠδὴ τῶν ἀναβαθμῶν.}
\end{psalmheading}
Οἱ πεποιθότες ἐπὶ Κύριον ὡς ὄρος Σιὼν, οὐ σαλευθήσεται εἰς αἰῶνα ὁ κατοικῶν Ἱερουσαλήμ·
\vs{2}ὄρη κύκλῳ αὐτῆς, καὶ ὁ Κύριος κύκλῳ τοῦ λαοῦ αὐτοῦ, ἀπὸ τοῦ νῦν καὶ ἕως τοῦ αἰῶνος.
\vs{3}Ὅτι οὐκ ἀφήσει Κύριος τὴν ῥάβδον τῶν ἁμαρτωλῶν ἐπὶ τὸν κλῆρον τῶν δικαίων, ὅπως ἂν μὴ ἐκτείνωσιν οἱ δίκαιοι ἐν ἀνομίαις χεῖρας αὐτῶν.

\vs{4}Ἀγάθυνον, Κύριε, τοῖς ἀγαθοῖς καὶ τοῖς εὐθέσι τῇ καρδίᾳ.
\vs{5}Τοὺς δὲ ἐκκλίνοντας εἰς τὰς στραγγαλιὰς, ἀπάξει Κύριος μετὰ τῶν ἐργαζομένων τὴν ἀνομίαν. εἰρήνη ἐπὶ τὸν Ἰσραήλ.

\begin{psalmheading}{\ch{125}{126} Ὠδὴ τῶν ἀναβαθμῶν.}
\end{psalmheading}
Ἐν τῷ ἐπιστρέψαι Κύριον τὴν αἰχμαλωσίαν Σιὼν, ἐγενήθημεν ὡσεὶ παρακεκλημένοι.
\vs{2}Τότε ἐπλήσθη χαρᾶς τὸ στόμα ἡμῶν, καὶ ἡ γλῶσσα ἡμῶν ἀγαλλιάσεως· τότε ἐροῦσιν ἐν τοῖς ἔθνεσιν, ἐμεγάλυνε Κύριος τοῦ ποιῆσαι μετʼ αὐτῶν.
\vs{3}Ἐμεγάλυνε Κύριος τοῦ ποιῆσαι μεθʼ ἡμῶν, ἐγενήθημεν εὐφραινόμενοι.

\vs{4}Ἐπίστρεψον, Κύριε, τὴν αἰχμαλωσίαν ἡμῶν ὡς χειμάῤῥους ἐν τῷ Νότῳ.
\vs{5}Οἱ σπείροντες ἐν δάκρυσιν, ἐν ἀγαλλιάσει θεριοῦσι.
\vs{6}Πορευόμενοι ἐπορεύοντο, καὶ ἔκλαιον βάλλοντες τὰ σπέρματα αὐτῶν· ἐρχόμενοι δὲ ἥξουσιν ἐν ἀγαλλιάσει αἴροντες τὰ δράγματα αὐτῶν.

\begin{psalmheading}{\ch{126}{127} Ὠδὴ τῶν ἀναβαθμῶν.}
\end{psalmheading}
Ἐὰν μὴ Κύριος οἰκοδομήσῃ οἶκον, εἰς μάτην ἐκοπίασαν οἱ οἰκοδομοῦντες· ἐὰν μὴ Κύριος φυλάξῃ πόλιν, εἰς μάτην ἠγρύπνησεν ὁ φυλάσσων.
\vs{2}Εἰς μάτην ὑμῖν ἐστι τὸ ὀρθρίζειν· ἐγείρεσθε μετὰ τὸ καθῆσθαι, οἱ ἐσθίοντες ἄρτον ὀδύνης, ὅταν δῷ τοῖς ἀγαπητοῖς αὐτοῦ ὕπνον.

\vs{3}Ἰδοὺ ἡ κληρονομία Κυρίου, υἱοί, ὁ μισθὸς τοῦ καρποῦ τῆς γαστρός.
\vs{4}Ὡσεὶ βέλη ἐν χειρὶ δυνατοῦ, οὕτως οἱ υἱοὶ τῶν ἐκτετιναγμένων.
\vs{5}Μακάριος ὃς πληρώσει τὴν ἐπιθυμίαν αὐτοῦ ἐξ αὐτῶν· οὐ καταισχυνθήσονται, ὅταν λαλῶσι τοῖς ἐχθροῖς αὐτῶν ἐν πύλαις.

\begin{psalmheading}{\ch{127}{128} Ὠδὴ τῶν ἀναβαθμῶν.}
\end{psalmheading}
Μακάριοι πάντες οἱ φοβούμενοι τὸν Κύριον, οἱ πορευόμενοι ἐν ταῖς ὁδοῖς αὐτοῦ.
\vs{2}Τοὺς πόνους τῶν καρπῶν σου φάγεσαι· μακάριος εἶ καὶ καλῶς σοι ἔσται.
\vs{3}Ἡ γυνή σου ὡς ἄμπελος εὐθηνοῦσα ἐν ταῖς κλίτεσι τῆς οἰκίας σου· οἱ υἱοί σου ὡς νεόφυτα ἐλαιῶν κύκλῳ τῆς τραπέζης σου.

\vs{4}Ἰδοὺ οὕτως εὐλογηθήσεται ἄνθρωπος ὁ φοβούμενος τὸν Κύριον.
\vs{5}Εὐλογήσαι σε Κύριος ἐκ Σιὼν, καὶ ἴδοις τὰ ἀγαθὰ Ἱερουσαλὴμ πάσας τὰς ἡμέρας τῆς ζωῆς σου·
\vs{6}Καὶ ἴδοις υἱοὺς τῶν υἱῶν σου· εἰρήνη ἐπὶ τὸν Ἰσραήλ.

\begin{psalmheading}{\ch{128}{129} Ὠδὴ τῶν ἀναβαθμῶν.}
\end{psalmheading}
Πλεονάκις ἐπολέμησάν με ἐκ νεότητός μου, εἰπάτω δὴ Ἰσραήλ·
\vs{2}Πλεονάκις ἐπολέμησάν με ἐκ νεότητός μου, καὶ γὰρ οὐκ ἠδυνήθησάν μοι.
\vs{3}Ἐπὶ τὸν νῶτόν μου ἐτέκταινον οἱ ἁμαρτωλοὶ, ἐμάκρυναν τὴν ἀνομίαν αὐτῶν.
\vs{4}Κύριος δίκαιος συνέκοψεν αὐχένας ἁμαρτωλῶν.

\vs{5}Αἰσχυνθήτωσαν καὶ ἀποστραφήτωσαν εἰς τὰ ὀπίσω πάντες οἱ μισοῦντες Σιών.
\vs{6}Γενηθήτωσαν ὡσεὶ χόρτος δωμάτων, ὃς πρὸ τοῦ ἐκσπασθῆναι ἐξηράνθη·
\vs{7}Οὗ οὐκ ἐπλήρωσεν τὴν χεῖρα αὐτοῦ ὁ θερίζων, καὶ τὸν κόλπον αὐτοῦ ὁ τὰ δράγματα συλλέγων.
\vs{8}Καὶ οὐκ εἶπαν οἱ παράγοντες, εὐλογία Κυρίου ἐφʼ ὑμᾶς, εὐλογήκαμεν ὑμᾶς ἐν ὀνόματι Κυρίου.

\begin{psalmheading}{\ch{129}{130} Ὠδὴ τῶν ἀναβαθμῶν.}
\end{psalmheading}
Ἐκ βαθέων ἐκέκραξά σοι Κύριε.
\vs{2}Κύριε εἰσάκουσον τῆς φωνῆς μου· γενηθήτω τὰ ὦτά σου προσέχοντα εἰς τὴν φωνὴν τῆς δεήσεώς μου.
\vs{3}Ἐὰν ἀνομίας παρατηρήσῃς Κύριε, Κύριε, τίς ὑποστήσεται;
\vs{4}Ὅτι παρὰ σοὶ ὁ ἱλασμός ἐστιν· ἕνεκεν τοῦ ὀνόματός σου
\vs{5}ὑπέμεινά σε Κύριε, ὑπέμεινεν ἡ ψυχή μου εἰς τὸν λόγον σου,
\vs{6}ἤλπισεν ἡ ψυχή μου ἐπὶ τὸν Κύριον· ἀπὸ φυλακῆς πρωΐας μέχρι νυκτός,

\vs{7}Ἐλπισάτω Ἰσραὴλ ἐπὶ τὸν Κύριον· ὅτι παρὰ τῷ Κυρίῳ τὸ ἔλεος, καὶ πολλὴ παρ αὐτῷ λύτρωσις.
\vs{8}Καὶ αὐτὸς λυτρώσεται τὸν Ἰσραὴλ ἐκ πασῶν τῶν ἀνομιῶν αὐτοῦ.

\begin{psalmheading}{\ch{130}{131} Ὠδὴ τῶν ἀναβαθμῶν.}
\end{psalmheading}
Κύριε, οὐχ ὑψώθη ἡ καρδία μου, οὐδὲ ἐμετεωρίσθησαν οἱ ὀφθαλμοί μου· οὐδὲ ἐπορεύθην ἐν μεγάλοις, οὐδὲ ἐν θαυμασίοις ὑπὲρ ἐμέ.
\vs{2}Εἰ μὴ ἐταπεινοφρόνουν, ἀλλὰ ὕψωσα τὴν ψυχήν μου· ὡς τὸ ἀπογεγαλακτισμένον ἐπὶ τὴν μητέρα αὐτοῦ, ὡς ἀνταποδώσεις ἐπὶ τὴν ψυχήν μου.
\vs{3}Ἐλπισάτω Ἰσραὴλ ἐπὶ τὸν Κύριον ἀπὸ τοῦ νῦν καὶ ἕως τοῦ αἰῶνος.

\begin{psalmheading}{\ch{131}{132} Ὠδὴ τῶν ἀναβαθμῶν.}
\end{psalmheading}
Μνήσθητι Κύριε τοῦ Δαυὶδ, καὶ πάσης τῆς πρᾳότητος αὐτοῦ·
\vs{2}Ὡς ὤμοσε τῷ Κυρίῳ, ηὔξατο τῷ Θεῷ Ἰακώβ·
\vs{3}Εἰ εἰσελεύσομαι εἰς σκήνωμα οἴκου μου, εἰ ἀναβήσομαι ἐπὶ κλίνης στρωμνῆς μου·
\vs{4}Εἰ δώσω ὕπνον τοῖς ὀφθαλμοῖς μοῦ, καὶ τοῖς βλεφάροις μου νυσταγμὸν, καὶ ἀνάπαυσιν τοῖς κροτάφοις μου·
\vs{5}Ἕως οὗ εὕρω τόπον τῷ Κυρίῳ, σκήνωμα τῷ Θεῷ Ἰακώβ.
\vs{6}Ἰδοὺ ἠκούσαμεν αὐτὴν ἐν Ἐφραθὰ, εὕρομεν αὐτὴν ἐν τοῖς πεδίοις τοῦ δρυμοῦ.
\vs{7}Εἰσελευσώμεθα εἰς τὰ σκηνώματα αὐτοῦ· προσκυνήσωμεν εἰς τὸν τόπον οὗ ἔστησαν οἱ πόδες αὐτοῦ.

\vs{8}Ἀνάστηθι Κύριε εἰς τὴν ἀνάπαυσίν σου, σὺ καὶ ἡ κιβωτὸς τοῦ ἁγιάσματός σου.
\vs{9}Οἱ ἱερεῖς σου ἐνδύσονται δικαιοσύνην, καὶ οἱ ὅσιοί σου ἀγαλλιάσονται.
\vs{10}Ἕνεκεν Δαυὶδ τοῦ δούλου σου, μὴ ἀποστρέψῃς τὸ πρόσωπον τοῦ χριστοῦ σου.

\vs{11}Ὤμοσε Κύριος τῷ Δαυὶδ ἀλήθειαν, καὶ οὐ μὴ ἀθετήσει αὐτὴν, ἐκ καρποῦ τῆς κοιλίας σου, θήσομαι ἐπὶ τοῦ θρόνον σου.
\vs{12}Ἐὰν φυλάξωνται οἱ υἱοί σου τὴν διαθήκην μου, καὶ τὰ μαρτύριά μου ταῦτα ἃ διδάξω αὐτοὺς, καὶ οἱ υἱοὶ αὐτῶν ἕως τοῦ αἰῶνος καθιοῦνται ἐπὶ τοῦ θρόνου σου.
\vs{13}Ὅτι ἐξελέξατο Κύριος τὴν Σιὼν, ᾑρετίσατο αὐτὴν εἰς κατοικίαν ἑαυτῷ.
\vs{14}Αὕτη ἡ κατάπαυσίς μου εἰς αἰῶνα αἰῶνος, ὧδε κατοικήσω ὅτι ᾑρετισάμην αὐτήν.
\vs{15}Τὴν θήραν αὐτῆς εὐλογῶν εὐλογήσω, τοὺς πτωχοὺς αὐτῆς χορτάσω ἄρτων.
\vs{16}Τοὺς ἱερεῖς αὐτῆς ἐνδύσω σωτηρίαν, καὶ οἱ ὅσιοι αὐτῆς ἀγαλλιάσει ἀγαλλιάσονται.
\vs{17}Ἐκεῖ ἐξανατελῶ κέρας τῷ Δαυὶδ, ἡτοίμασα λύχνον τῷ χριστῷ σου.
\vs{18}Τοὺς ἐχθροὺς αὐτοῦ ἐνδύσω αἰσχύνην, ἐπὶ δὲ αὐτὸν ἐξανθήσει τὸ ἁγίασμά μου.

\begin{psalmheading}{\ch{132}{133} Ὠδὴ τῶν ἀναβαθμῶν.}
\end{psalmheading}
Ἰδοὺ δὴ τί καλὸν, ἢ τί τερπνὸν, ἀλλʼ ἢ τὸ κατοικεῖν ἀδελφοὺς ἐπιτοαυτό;
\vs{2}Ὡς μῦρον ἐπὶ κεφαλῆς τὸ καταβαῖνον ἐπὶ πώγωνα, τὸν πώγωνα τὸν Ἀαρὼν, τὸ καταβαῖνον ἐπὶ τὴν ὦαν τοῦ ἐνδύματος αὐτοῦ.
\vs{3}Ὡς δρόσος Ἀερμῶν ὁ καταβαίνουσα ἐπὶ τὰ ὄρη Σιών· ὅτι ἐκεῖ ἐνετείλατο Κύριος τὴν εὐλογίαν, ζωὴν ἕως τοῦ αἰῶνος.

\begin{psalmheading}{\ch{133}{134} Ὠδὴ τῶν ἀναβαθμῶν.}
\end{psalmheading}
Ἰδοὺ δὴ εὐλογεῖτε τὸν Κύριον πάντες οἱ δοῦλοι Κυρίου, οἱ ἑστῶτες ἐν οἴκῳ Κυρίου ἐν αὐλαῖς οἴκου Θεοῦ ἡμῶν·
\vs{2}ἐν ταῖς νυξὶν ἐπάρατε τὰς χεῖρας ὑμῶν εἰς τὰ ἅγια, καὶ εὐλογεῖτε τὸν Κύριον.
\vs{3}Εὐλογήσαι σε Κύριος ἐκ Σιὼν, ὁ ποιήσας τὸν οὐρανὸν καὶ τὴν γῆν.

\begin{psalmheading}{\ch{134}{135} Ἀλληλούϊα.}
\end{psalmheading}
Αἰνεῖτε τὸ ὄνομα Κυρίου, αἰνεῖτε δοῦλοι Κύριον,
\vs{2}Οἱ ἑστῶτες ἐν οἴκω Κυρίου, ἐν αὐλαῖς οἴκου Θεοῦ ἡμῶν.
\vs{3}Αἰνεῖτε τὸν Κύριον, ὅτι ἀγαθὸς Κύριος· ψάλατε τῷ ὀνόματι αὐτοῦ, ὅτι καλόν.

\vs{4}Ὅτι τὸν Ἰακὼβ ἐξελέξατο ἑαυτῷ ὁ Κύριος, Ἰσραὴλ εἰς περιουσιασμὸν ἑαυτῷ.
\vs{5}Ὅτι ἐγὼ ἔγνωκα, ὅτι μέγας ὁ Κύριος, καὶ ὁ Κύριος ἡμῶν παρὰ πάντας τοὺς θεούς.
\vs{6}Πάντα ὅσα ἠθέλησεν ὁ Κύριος, ἐποίησεν ἐν τῷ οὐρανῷ καὶ ἐν τῇ γῇ, ἐν ταῖς θαλάσσαις καὶ ἐν πάσαις ταῖς ἀβύσσοις.
\vs{7}Ἀνάγων νεφέλας ἐξ ἐσχάτου τῆς γῆς, ἀστραπὰς εἰς ὑετὸν ἐποίησεν· ὁ ἐξάγων ἀνέμους ἐκ θησαυρῶν αὐτοῦ.
\vs{8}Ὃς ἐπάταξε τὰ πρωτότοκα Αἰγύπτου ἀπὸ ἀνθρώπου ἕως κτήνους.
\vs{9}Ἐξαπέστειλεν σημεῖα καὶ τέρατα ἐν μέσῳ σου Αἴγυπτε, ἐν Φαραῷ καὶ ἐν πᾶσι τοῖς δούλοις αὐτοῦ.
\vs{10}Ὃς ἐπάταξεν ἔθνη πολλά, καὶ ἀπέκτεινε βασιλεῖς κραταιούς·
\vs{11}τὸν Σηὼν βασιλέα τῶν Ἀμοῤῥαίων, καὶ τὸν Ὢγ βασιλέα τῆς Βασὰν, καὶ πάσας τὰς βασιλείας Χαναάν·
\vs{12}Καὶ ἔδωκε τὴν γῆν αὐτῶν κληρονομίαν, κληρονομίαν Ἰσραὴλ λαῷ αὐτοῦ.

\vs{13}Κύριε τὸ ὄνομά σου εἰς τὸν αἰῶνα, καὶ τὸ μνημόσυνόν σου εἰς γενεὰν καὶ γενεάν.
\vs{14}Ὅτι κρινεῖ Κύριος τὸν λαὸν αὐτοῦ, καὶ ἐπὶ τοῖς δούλοις αὐτοῦ παρακληθήσεται.
\vs{15}Τὰ εἴδωλα τῶν ἐθνῶν ἀργύριον καὶ χρυσίον, ἔργα χειρῶν ἀνθρώπων.
\vs{16}Στόμα ἔχουσι καὶ οὐ λαλήσουσιν, ὀφθαλμοὺς ἔχουσι καὶ οὐκ ὄψονται·
\vs{17}Ὦτα ἔχουσι καὶ οὐκ ἐνωτισθήσονται, οὐδὲ γάρ ἐστι πνεῦμα ἐν τῷ στόματι αὐτῶν.
\vs{18}Ὅμοιοι αὐτοῖς γένοιντο οἱ ποιοῦντες αὐτὰ, καὶ πάντες οἱ πεποιθότες ἐπʼ αὐτοῖς.

\vs{19}Οἶκος Ἰσραὴλ εὐλογήσατε τὸν Κύριον, οἶκος Ἀαρὼν εὐλογήσατε τὸν Κύριον·
\vs{20}Οἶκος Λευὶ, εὐλογήσατε τὸν Κύριον, οἱ φοβούμενοι τὸν Κύριον εὐλογήσατε τὸν Κύριον.
\vs{21}Εὐλογητὸς Κύριος ἐκ Σιὼν, ὁ κατοικῶν Ἱερουσαλήμ.

\begin{psalmheading}{\ch{135}{136} Ἀλληλούϊα.}
\end{psalmheading}
Ἐξομολογεῖσθε τῷ Κυρίῳ, ὅτι ἀγαθὸς, ὅτι εἰς τὸν αἰῶνα τὸ ἔλεος αὐτοῦ.
\vs{2}Ἐξομολογεῖσθε τῷ Θεῷ τῶν θεῶν, ὅτι εἰς τὸν αἰῶνα τὸ ἔλεος αὐτοῦ.
\vs{3}Ἐξομολογεῖσθε τῷ Κυρίῳ τῶν κυρίων, ὅτι εἰς τὸν αἰῶνα τὸ ἔλεος αὐτοῦ.

\vs{4}Τῷ ποιήσαντι θαυμάσια μεγάλα μόνῳ, ὅτι εἰς τὸν αἰῶνα τὸ ἔλεος αὐτοῦ.
\vs{5}Τῷ ποιήσαντι τοὺς οὐρανοὺς ἐν συνέσει, ὅτι εἰς τὸν αἰῶνα τὸ ἔλεος αὐτοῦ.
\vs{6}Τῷ στερεώσαντι τὴν γῆν ἐπὶ τῶν ὑδάτων, ὅτι εἰς τὸν αἰῶνα τὸ ἔλεος αὐτοῦ.
\vs{7}Τῷ ποιήσαντι φῶτα μεγάλα μόνῳ, ὅτι εἰς τὸν αἰῶνα τὸ ἔλεος αὐτοῦ.
\vs{8}Τὸν ἥλιον εἰς ἐξουσίαν τῆς ἡμέρας, ὅτι εἰς τὸν αἰῶνα τὸ ἔλεος αὐτοῦ·
\vs{9}Τὴν σελήνην καὶ τοὺς ἀστέρας εἰς ἐξουσίαν τῆς νυκτός, ὅτι εἰς τὸν αἰῶνα τὸ ἔλεος αὐτοῦ.

\vs{10}Τῷ πατάξαντι Αἴγυπτον σὺν τοῖς πρωτοτόκοις αὐτῶν, ὅτι εἰς τὸν αἰῶνα τὸ ἔλεος αὐτοῦ.
\vs{11}Καὶ ἐξαγαγόντι τὸν Ἰσραὴλ ἐκ μέσου αὐτῶν, ὅτι εἰς τὸν αἰῶνα τὸ ἔλεος αὐτοῦ·
\vs{12}Ἐν χειρὶ κραταιᾷ καὶ ἐν βραχίονι ὑψηλῷ, ὅτι εἰς τὸν αἰῶνα τὸ ἔλεος αὐτοῦ.
\vs{13}Τῷ καταδιελόντι τὴν ἐρυθρὰν θάλασσαν εἰς διαιρέσεις, ὅτι εἰς τὸν αἰῶνα τὸ ἔλεος αὐτοῦ·
\vs{14}Καὶ διαγαγόντι τὸν Ἰσραὴλ διὰ μέσου αὐτῆς, ὅτι εἰς τὸν αἰῶνα τὸ ἔλεος αὐτοῦ·
\vs{15}Καὶ ἐκτινάξαντι Φαραὼ καὶ τὴν δύναμιν αὐτοῦ εἰς θάλασσαν ἐρυθράν, ὅτι εἰς τὸν αἰῶνα τὸ ἔλεος αὐτοῦ.
\vs{16}Τῷ διαγαγόντι τὸν λαὸν αὐτοῦ ἐν τῇ ἐρήμῳ, ὅτι εἰς τὸν αἰῶνα τὸ ἔλεος αὐτοῦ.

\vs{17}Τῷ πατάξαντι βασιλεῖς μεγάλους, ὅτι εἰς τὸν αἰῶνα τὸ ἔλεος αὐτοῦ.
\vs{18}Καὶ ἀποκτείναντι βασιλεῖς κραταιούς, ὅτι εἰς τὸν αἰῶνα τὸ ἔλεος αὐτοῦ.
\vs{19}Τὸν Σηὼν βασιλέα τῶν Ἀμοῤῥαίων, ὅτι εἰς τὸν αἰῶνα τὸ ἔλεος αὐτοῦ.
\vs{20}Καὶ τὸν Ὢγ βασιλέα τῆς Βασάν, ὅτι εἰς τὸν αἰῶνα τὸ ἔλεος αὐτοῦ.
\vs{21}Καὶ δόντι τὴν γῆν αὐτῶν κληρονομίαν, ὅτι εἰς τὸν αἰῶνα τὸ ἔλεος αὐτοῦ·
\vs{22}κληρονομίαν Ἰσραὴλ δούλῳ αὐτοῦ, ὅτι εἰς τὸν αἰῶνα τὸ ἔλεος αὐτοῦ.

\vs{23}Ὅτι ἐν τῇ ταπεινώσει ἡμῶν ἐμνήσθη ἡμῶν ὁ Κύριος, ὅτι εἰς τὸν αἰῶνα τὸ ἔλεος αὐτοῦ.
\vs{24}Καὶ ἐλυτρώσατο ἡμᾶς ἐκ τῶν ἐχθρῶν ἡμῶν, ὅτι εἰς τὸν αἰῶνα τὸ ἔλεος αὐτοῦ.
\vs{25}Ὁ διδοὺς τροφὴν πάσῃ σαρκὶ, ὅτι εἰς τὸν αἰῶνα τὸ ἔλεος αὐτοῦ.
\vs{26}Ἐξομολογεῖσθε τῷ Θεῷ τοῦ οὐρανοῦ, ὅτι εἰς τὸν αἰῶνα τὸ ἔλεος αὐτοῦ.

\begin{psalmheading}{\ch{136}{137} Τῷ Δαυὶδ, Ἱερεμίου.}
\end{psalmheading}
Ἐπὶ τῶν ποταμῶν Βαβυλῶνος ἐκεῖ ἐκαθίσαμεν, καὶ ἐκλαύσαμεν ἐν τῷ μνησθῆναι ἡμᾶς τῆς Σιών.
\vs{2}Ἐπὶ ταῖς ἰτέαις ἐν μέσῳ αὐτῆς ἐκρεμάσαμεν τὰ ὄργανα ἡμῶν.
\vs{3}Ὅτι ἐκεῖ ἐπηρώτησαν ἡμᾶς οἱ αἰχμαλωτεύσαντες ἡμᾶς, λόγους ᾠδῶν, καὶ οἱ ἀπαγαγόντες ἡμᾶς, ὕμνον· ᾄσατε ἡμῖν ἐκ τῶν ᾠδῶν Σιών.

\vs{4}Πῶς ᾄσωμεν τὴν ᾠδὴν Κυρίου ἐπὶ γῆς ἀλλοτρίας;
\vs{5}Ἐὰν ἐπιλάθωμαί σου Ἱερουσαλήμ, ἐπιλησθείη ἡ δεξιά μου.
\vs{6}Κολληθείη ἡ γλῶσσά μου τῷ λάρυγγί μου, ἐὰν μή σου μνησθῷ· ἐὰν μὴ προανατάξωμαι τὴν Ἱερουσαλὴμ ἐν ἀρχῇ τῆς εὐφροσύνης μου.

\vs{7}Μνήσθητι, Κύριε, τῶν υἱῶν Ἐδὼμ τὴν ἡμέραν Ἱερουσαλήμ· τῶν λεγόντων, ἐκκενοῦτε ἐκκενοῦτε, ἕως τῶν θεμελίων αὐτῇς.
\vs{8}Θυγάτηρ Βαβυλῶνος ἡ ταλαίπωρος, μακάριος ὃς ἀνταποδώσει σοι τὸ ἀνταπόδομά σου, ὃ ἀνταπέδωκας ἡμῖν.
\vs{9}Μακάριος ὃς κρατήσει καὶ ἐδαφιεῖ τὰ νήπιά σου πρὸς τὴν πέτραν.

\begin{psalmheading}{\ch{137}{138} Ψαλμὸς τῷ Δαυὶδ, Ἀγγαίου καὶ Ζαχαρίου.}
\end{psalmheading}
Ἐξομολογήσομαί σοι Κύριε ἐν ὅλῃ καρδίᾳ μου, καὶ ἐναντίον ἀγγέλων ψαλῶ σοι, ὅτι ἤκουσας πάντα τὰ ῥήματα τοῦ στόματός μου.
\vs{2}Προσκυνήσω πρὸς ναὸν ἅγιόν σου, καὶ ἐξομολογήσομαι τῷ ὀνόματί σου, ἐπὶ τῷ ἐλέει σου καὶ τῇ ἀληθείᾳ σου· ὅτι ἐμεγάλυνας ἐπὶ πᾶν τὸ ὄνομα τὸ ἅγιόν σου.
\vs{3}Ἐν ᾗ ἂν ἡμέρᾳ ἐπικαλέσωμαί σε, ταχὺ ἐπάκουσόν μου· πολυωρήσεις με ἐν ψυχῇ μου δυνάμει σου.
\vs{4}Ἐξομολογησάσθωσάν σοι Κύριε πάντες οἱ βασιλεῖς τῆς γῆς, ὅτι ἤκουσαν πάντα τὰ ῥήματα τοῦ στόματός σου.
\vs{5}Καὶ ᾀσάτωσαν ἐν ταῖς ὁδοῖς Κυρίου, ὅτι μεγάλη ἡ δόξα Κυρίου.

\vs{6}Ὅτι ὑψηλὸς Κύριος, καὶ τὰ ταπεινὰ ἐφορᾷ, καὶ τὰ ὑψηλὰ ἀπομακρόθεν γινώσκει.
\vs{7}Ἐὰν πορευθῶ ἐν μέσῳ θλίψεως, ζήσεις με· ἐπʼ ὀργὴν ἐχθρῶν μου ἐξέτεινας χεῖράς σου, καὶ ἔσωσέ με ἡ δεξιά σου.
\vs{8}Κύριε ἀνταποδώσεις ὑπὲρ ἐμοῦ· Κύριε τὸ ἔλεός σου εἰς τὸν αἰῶνα, τὰ ἔργα τῶν χειρῶν σου μὴ παρίδῃς.

\begin{psalmheading}{\ch{138}{139} Εἰς τὸ τέλος, ψαλμὸς τῷ Δαυίδ.}
\end{psalmheading}
Κύριε ἐδοκίμασάς με, καὶ ἔγνως με.
\vs{2}Σὺ ἔγνως τὴν καθέδραν μου, καὶ τὴν ἔγερσίν μου· σὺ συνῆκας τοὺς διαλογισμούς μου ἀπὸ μακρόθεν.
\vs{3}Τὴν τρίβον μου καὶ τὴν σχοῖνόν μου ἑξιχνίασας· καὶ πάσας τὰς ὁδούς μου προεῖδες,
\vs{4}ὅτι οὐκ ἔστι λόγος ἄδικος ἐν γλώσσῃ μου· ἰδοὺ Κύριε, σὺ ἔγνως πάντα
\vs{5}τὰ ἔσχατα καὶ τὰ ἀρχαῖα· σὺ ἔπλασάς με καὶ ἔθηκας ἐπʼ ἐμὲ τὴν χεῖρά σου.

\vs{6}Ἐθαυμαστώθη ἡ γνῶσίς σου ἐξ ἐμοῦ, ἐκραταιώθη, οὐ μὴ δύνωμαι πρὸς αὐτήν.
\vs{7}Ποῦ πορευθῶ ἀπὸ τοῦ πνεύματός σου, καὶ ἀπὸ τοῦ προσώπου σου ποῦ φύγω;
\vs{8}Ἐὰν ἀναβῶ εἰς τὸν οὐρανὸν, σὺ ἐκεῖ εἶ· ἐὰν καταβῶ εἰς τὸν ᾅδην, πάρει.
\vs{9}Ἐὰν ἀναλάβω τὰς πτέρυγάς μου κατʼ ὀρθὸν, καὶ κατασκηνώσω εἰς τὰ ἔσχατα τῆς θαλάσσης,
\vs{10}καὶ γὰρ ἐκεῖ ἡ χείρ σου ὁδηγήσει με, καὶ καθέξει με ἡ δεξιά σου.
\vs{11}Καὶ εἶπα, ἄρα σκότος καταπατήσει με, καὶ νὺξ φωτισμὸς ἐν τῇ τρυφῇ μου.
\vs{12}Ὅτι σκότος οὐ σκοτισθήσεται ἀπὸ σοῦ, καὶ νὺξ ὡς ἡμέρα φωτισθήσεται· ὡς τὸ σκότος αὐτῆς, οὕτως καὶ τὸ φῶς αὐτῆς.
\vs{13}Ὅτι σὺ ἐκτήσω τοὺς νεφρούς μου Κύριε, ἀντελάβου μου ἐκ γαστρὸς μητρός μου.
\vs{14}Ἐξομολογήσομαί σοι, ὅτι φοβερῶς ἐθαυμαστώθης· θαυμάσια τὰ ἔργα σου, καὶ ἡ ψυχή μου γινώσκει σφόδρα.
\vs{15}Οὐκ ἐκρύβη τὸ ὀστοῦν μου ἀπὸ σοῦ, ὃ ἐποίησας ἐν κρυφῇ, καὶ ἡ ὑπόστασίς μου ἐν τοῖς κατωτάτω τῆς γῆς.
\vs{16}Ἀκατέργαστόν μου εἶδου οἱ ἀφθαλμοί σου, καὶ ἐπὶ τὸ βιβλίον σου πάντες γραφήσονται· ἡμέρας πλασθήσονται καὶ οὐθεὶς ἐν αὐτοῖς.

\vs{17}Ἐμοὶ δὲ λίαν ἐτιμήθησαν οἱ φίλοι σου ὁ Θεὸς, λίαν ἐκραταιώθησαν αἱ ἀρχαὶ αὐτῶν.
\vs{18}Ἐξαριθμήσομαι αὐτοὺς καὶ ὑπὲρ ἄμμον πληθυνθήσονται· ἐξηγέρθην, καὶ ἔτι εἰμὶ μετὰ σοῦ.

\vs{19}Ἐὰν ἀποκτείνῃς ἁμαρτωλοὺς ὁ Θεός· ἄνδρες αἱμάτων ἐκκλίνατε ἀπʼ ἐμοῦ,
\vs{20}ὅτι ἐρεῖς εἰς διαλογισμόν· λήψονται εἰς ματαιότητα τὰς πόλεις σου.
\vs{21}Οὐχὶ τοὺς μισοῦντάς σε Κύριε ἐμίσησα, καὶ ἐπὶ τοὺς ἐχθρούς σου ἐξετηκόμην;
\vs{22}Τέλειον μῖσος ἐμίσουν αὐτοὺς, εἰς ἐχθροὺς ἐγένοντό μοι.
\vs{23}Δοκίμασόν με ὁ Θεὸς, καὶ γνῶθι τὴν καρδίαν μου· ἔτασόν με, καὶ γνῶθι τὰς τρίβους μου.
\vs{24}Καὶ ἴδε εἰ ὁδὸς ἀνομίας ἐν ἐμοὶ, καὶ ὁδήγησόν με ἐν ὁδῷ αἰωνίᾳ.

\begin{psalmheading}{\ch{139}{140} Εἰς τὸ τέλος, τῷ Δαυὶδ ψαλμός.}
\end{psalmheading}
\vs{2}Ἐξελοῦ με Κύριε ἐξ ἀνθρώπου πονηροῦ, ἀπὸ ἀνδρὸς ἀδίκου ῥῦσαί με·
\vs{3}οἵτινες ἐλογίσαντο ἀδικίας ἐν καρδίᾳ, ὅλην τὴν ἡμέραν παρετάσσοντο πολέμους.
\vs{4}Ἠκόνησαν γλῶσσαν αὐτῶν ὡσεὶ ὄφεως, ἰὸς ἀσπίδων ὑπὸ τὰ χείλη αὐτῶν· διάψαλμα.
\vs{5}Φύλαξόν με Κύριε ἐκ χειρὸς ἁμαρτωλοῦ, ἀπὸ ἀνθρώπων ἀδίκων ἐξελοῦ με· οἵτινες ἐλογίσαντο τοῦ ὑποσκελίσαι τὰ διαβήματά μου.
\vs{6}Ἔκρυψαν ὑπερήφανοι παγίδα μοι· καὶ σχοινία διέτειναν παγίδας τοῖς ποσί μου, ἐχόμενα τρίβου σκάνδαλον ἔθεντό μοι· διάψαλμα.

\vs{7}Εἶπα τῷ Κυρίῳ, Θεός μου εἶ σὺ· ἐνώτισαι, Κύριε, τὴν φωνὴν τῆς δεήσεώς μου.
\vs{8}Κύριε Κύριε, δύναμις τῆς σωτηρίας μου, ἐπεσκίασας ἐπὶ τὴν κεφαλήν μου ἐν ἡμέρᾳ πολέμου.
\vs{9}Μὴ παραδῷς με, Κύριε, ἀπὸ τῆς ἐπιθυμίας μου ἁμαρτωλῷ· διελογίσαντο κατʼ ἐμοῦ, μὴ ἐγκαταλίπῃς με, μή ποτε ὑψωθῶσι· διάψαλμα.

\vs{10}Ἡ κεφαλὴ τοῦ κυκλώματος αὐτῶν, κόπος τῶν χειλέων αὐτῶν καλύψει αὐτούς.
\vs{11}Πεσοῦνται ἐπʼ αὐτοὺς ἄνθρακες πυρὸς ἐπὶ τῆς γῆς, καὶ καταβαλεῖς αὐτοὺς ἐν ταλαιπωρίαις, οὐ μὴ ὑποστῶσιν.
\vs{12}Ἀνὴρ γλωσσώδης οὐ κατευθυνθήσεται ἐπὶ τῆς γῆς· ἄνδρα ἄδικον κακὰ θηρεύσει εἰς καταφθοράν.
\vs{13}Ἔγνων ὅτι ποιήσει Κύριος τὴν κρίσιν τοῦ πτωχοῦ, καὶ τὴν δίκην τῶν πενήτων.
\vs{14}Πλὴν δίκαιοι ἐξομολογήσονται τῷ ὀνόματί σου, κατοικήσουσιν εὐθεῖς σὺν τῷ προσώπῳ σου.

\begin{psalmheading}{\ch{140}{141} Ψαλμὸς τῷ Δαυίδ.}
\end{psalmheading}
Κύριε ἐκέκραξα πρὸς σὲ, εἰσάκουσόν μου· πρόσχες τῇ φωνῇ τῆς δεήσεώς μου, ἐν τῷ κεκραγέναι με πρὸς σέ.
\vs{2}Κατευθυνθήτω ἡ προσευχή μου ὡς θυμίαμα ἐνώπιόν σου· ἔπαρσις τῶν χειρῶν μου θυσία ἑσπερινή.
\vs{3}Θοῦ, Κύριε, φυλακὴν τῷ στόματί μου, καὶ θύραν περιοχῆς περὶ τὰ χείλη μου.
\vs{4}Μὴ ἐκκλίνῃς τὴν καρδίαν μου εἰς λόγους πονηρίας, τοῦ προφασίζεσθαι προφάσεις ἐν ἁμαρτίαις, σὺν ἀνθρώποις ἐργαζομένοις τὴν ἀνομίαν, καὶ οὐ μὴ συνδοιάσω μετὰ τῶν ἐκλεκτῶν αὐτῶν.
\vs{5}Παιδεύσει με δίκαιος ἐν ἐλέει καὶ ἐλέγξει με, ἔλαιον δὲ ἁμαρτωλοῦ μὴ λιπανάτω τὴν κεφαλήν μου, ὅτι ἔτι καὶ ἡ προσευχή μου ἐν ταῖς εὐδοκίαις αὐτῶν.

\vs{6}Κατεπόθησαν ἐχόμενα πέτρας οἱ κραταιοὶ αὐτῶν· ἀκούσονται τὰ ῥήματά μου, ὅτι ἠδύνθησαν.
\vs{7}Ὡσεὶ πάχος γῆς διεῤῥάγη ἐπὶ τῆς γῆς, διεσκορπίσθη τὰ ὀστᾶ ἡμῶν παρὰ τὸν ᾅδην.
\vs{8}Ὅτι πρὸς σὲ, Κύριε Κύριε, οἱ ὀφθαλμοί μου, ἐπὶ σοὶ ἤλπισα, μὴ ἀντανέλῃς τὴν ψυχήν μου.
\vs{9}Φύλαξόν με ἀπὸ παγίδος ἧς συνεστήσαντό μοι, καὶ ἀπὸ σκανδάλων τῶν ἐργαζομένων τὴν ἀνομίαν.
\vs{10}Πεσοῦνται ἐν ἀμφιβλήστρῳ αὐτοῦ ἁμαρτωλοὶ, καταμόνας εἰμὶ ἐγὼ ἕως οὗ ἂν παρέλθω.

\begin{psalmheading}{\ch{141}{142} Συνέσεως τῷ Δαυὶδ, ἐν τῷ εἶναι αὐτὸν ἐν τῷ σπηλαίῳ, προσευχή.}
\end{psalmheading}
\vs{2}Φωνῇ μου πρὸς Κύριον ἐκέκραξα, φωνῇ μου πρὸς Κύριον ἐδεήθην.
\vs{3}Ἐκχεῶ ἐναντίον αὐτοῦ τὴν δέησίν μου, τὴν θλίψιν μου ἐνώπιον αὐτοῦ ἀπαγγελῶ.
\vs{4}Ἐν τῷ ἐκλείπειν ἐξ ἐμοῦ τὸ πνεῦμά μου, καὶ σὺ ἔγνως τὰς τρίβους μου· ἐν ὁδῷ ταύτῃ ᾗ ἐπορευόμην, ἔκρυψαν παγίδα μοι.
\vs{5}Κατενόουν εἰς τὰ δεξιὰ καὶ ἐπέβλεπον, ὅτι οὐκ ἦν ὁ ἐπιγινώσκων με· ἀπώλετο φυγὴ ἀπʼ ἐμοῦ, καὶ οὐκ ἔστιν ὁ ἐκζητῶν τὴν ψυχήν μου.
\vs{6}Πρὸς σὲ, Κύριε, ἐκέραξα, καὶ εἶπα, σὺ εἶ ἡ ἐλπίς μου, μερίς μου ἐν γῇ ζώντων.
\vs{7}Πρόσχες πρὸς τὴν δέησίν μου, ὅτι ἐταπεινώθην σφόδρα· ῥῦσαί με ἐκ τῶν καταδιωκόντων με, ὅτι ἐκραταιώθησαν ὑπὲρ ἐμέ.
\vs{8}Ἐξάγαγε ἐκ φυλακῆς τὴν ψυχήν μου, τοῦ ἐξομολογήσασθαι τῷ ὀνόματί σου, Κύριε· ἐμὲ ὑπομενοῦσι δίκαιοι, ἕως οὗ ἀνταποδῷς μοι.

\begin{psalmheading}{\ch{142}{143} Ψαλμὸς τῷ Δαυὶδ, ὅτε αὐτὸν ὁ υἱὸς κατεδιώκει.}
\end{psalmheading}
Κύριε εἰσάκουσον τῆς προσευχῆς μου, ἐνώτισαι τὴν δέησίν μου ἐν τῇ ἀληθείᾳ σου, ἐπάκουσόν μου ἐν τῇ δικαιοσύνῃ σου.
\vs{2}Καὶ μὴ εἰσέλθῃς εἰς κρίσιν μετὰ τοῦ δούλου σου, ὅτι οὐ δικαιωθήσεται ἐνώπιόν σου πᾶς ζῶν.

\vs{3}Ὅτι κατεδίωξεν ὁ ἐχθρὸς τὴν ψυχήν μου· ἐταπείνωσεν εἰς τὴν γῆν τὴν ζωήν μου, ἐκάθισέ με ἐν σκοτεινοῖς ὡς νεκροὺς αἰῶνος,
\vs{4}καὶ ἠκηδίασεν ἐπʼ ἐμὲ τὸ πνεῦμά μου· ἐν ἐμοὶ ἐταράχθη ἡ καρδία μου.
\vs{5}Ἐμνήσθην ἡμερῶν ἀρχαίων· καὶ ἐμελέτησα ἐν πᾶσι τοῖς ἔργοις σου, ἐν ποιήμασι τῶν χειρῶν σου ἐμελέτων.
\vs{6}Διεπέτασα πρὸς σὲ τὰς χεῖράς μου, ἡ ψυχή μου ὡς γῆ ἄνυδρός σοι· διάψαλμα.

\vs{7}Ταχὺ εἰσάκουσόν μου, Κύριε, ἐξέλιπε τὸ πνεῦμά μου· μὴ ἀποστρέψῃς τὸ πρόσωπόν σου ἀπʼ ἐμοῦ, καὶ ὁμοιωθήσομαι τοῖς καταβαίνουσιν εἰς λάκκον.
\vs{8}Ἀκουστὸν ποίησόν μοι τοπρωῒ τὸ ἔλεός σου, ὅτι ἐπὶ σοὶ ἤλπισα· γνώρισόν μοι, Κύριε, ὁδὸν ἐν ᾗ πορεύσομαι, ὅτι πρὸς σὲ ᾖρα τὴν ψυχήν μου.
\vs{9}Ἐξελοῦ με ἐκ τῶν ἐχθρῶν μου Κύριε, ὅτι πρὸς σὲ κατέφυγον.
\vs{10}Δίδαξόν με τοῦ ποιεῖν τὸ θέλημά σου, ὅτι Θεός μου εἶ σὺ, τὸ πνεῦμά σου τὸ ἁγαθὸν ὁδηγήσει με ἐν τῇ εὐθείᾳ.
\vs{11}Ἕνεκα τοῦ ὀνόματός σου, Κύριε, ζήσεις με, ἐν τῇ δικαιοσύνῃ σου ἐξάξεις ἐκ θλίψεως τὴν ψυχήν μου.
\vs{12}Καὶ ἐν τῷ ἐλέει σου ἐξολοθρεύσεις τοὺς ἐχθρούς μου, καὶ ἀπολεῖς πάντας τοὺς θλίβοντας τὴν ψυχήν μου, ὅτι δοῦλός σου εἰμὶ ἐγώ.

\begin{psalmheading}{\ch{143}{144} Τῷ Δαυὶδ πρὸς τὸν Γολιάδ.}
\end{psalmheading}
Εὐλογήτος Κύριος ὁ Θεός μου, ὁ διδάσκων τὰς χεῖράς μου εἰς παράταξιν, τοὺς δακτύλους μου εἰς πόλεμον·
\vs{2}Ἔλεός μου καὶ καταφυγή μου, ἀντιλήπτωρ μου καὶ ῥύστης μου, ὑπερασπιστής μου, καὶ ἐπʼ αὐτῷ ἤλπισα, ὁ ὑποτάσσων τὸν λαόν μου ὑπʼ ἐμέ.

\vs{3}Κύριε, τί ἐστιν ἄνθρωπος, ὅτι ἐγνώσθης αὐτῷ; ἢ υἱὸς ἀνθρώπου, ὅτι λογίζῃ αὐτόν;
\vs{4}Ἄνθρωπος ματαιότητι ὡμοιώθη, αἱ ἡμέραι αὐτοῦ ὡσεὶ σκιὰ παράγουσι.

\vs{5}Κύριε, κλῖνον οὐρανούς σου καὶ κατάβηθι, ἅψαι τῶν ὀρέων καὶ καπνισθήσονται.
\vs{6}Ἄστραψον ἀστραπὴν καὶ σκορπιεῖς αὐτοὺς, ἐξαπόστειλον τὰ βέλη σου καὶ συνταράξεις αὐτούς.
\vs{7}Ἐξαπόστειλον τὴν χεῖρά σου ἐξ ὕψους, ἐξελοῦ με καὶ ῥῦσαί με ἐξ ὑδάτων πολλῶν, ἐκ χειρὸς υἱῶν ἀλλοτρίων·
\vs{8}ὧν τὸ στόμα ἐλάλησε ματαιότητα, καὶ ἡ δεξιὰ αὐτῶν δεξιὰ ἀδικίας.

\vs{9}Ὁ Θεὸς, ᾠδὴν καινὴν ᾄσομαί σοι, ἐν ψαλτηρίῳ δεκαχόρδῳ ψαλῶ σοι·
\vs{10}Τῷ διδόντι τὴν σωτηρίαν τοῖς βασιλεῦσι, τῷ λυτρουμένῳ Δαυὶδ τὸν δοῦλον αὐτοῦ ἐκ ῥομφαίας πονηρᾶς.
\vs{11}Ῥῦσαί με καὶ ἐξελοῦ με ἐκ χειρὸς υἱῶν ἀλλοτρίων, ὧν τὸ στόμα ἐλάλησε ματαιότητα, καὶ ἡ δεξιὰ αὐτῶν δεξιὰ ἀδικίας·
\vs{12}ὧν οἱ υἱοὶ ὡς νεόφυτα ἱδρυμένα ἐν τῇ νεότητι αὐτῶν· αἱ θυγατέρες αὐτῶν κεκαλλωπισμέναι, περικεκοσμημέναι ὡς ὁμοίωμα ναοῦ.
\vs{13}Τὰ ταμεῖα αὐτῶν πλήρη, ἐξερευγόμενα ἐκ τούτου εἰς τοῦτο· τὰ πρόβατα αὐτῶν πολυτόκα, πληθύνοντα ἐν ταῖς ἐξόδοις αὐτῶν·
\vs{14}Οἱ βόες αὐτῶν παχεῖς· οὐκ ἔστι κατάπτωμα φραγμοῦ, οὐδὲ διέξοδος, οὐδὲ κραυγὴ ἐν ταῖς ἐπαύλεσιν αὐτῶν.
\vs{15}Ἐμακάρισαν τὸν λαὸν ᾧ ταῦτά ἐστι· μακάριος ὁ λαὸς οὗ Κύριος ὁ Θεὸς αὐτοῦ.

\begin{psalmheading}{\ch{144}{145} Αἴνεσις τοῦ Δαυίδ.}
\end{psalmheading}
Ὑψώσω σε, ὁ Θεός μου ὁ βασιλεύς μου, καὶ εὐλογήσω τὸ ὄνομά σου εἰς τὸν αἰῶνα καὶ εἰς τὸν αἰῶνα τοῦ αἰῶνος.
\vs{2}Καθʼ ἑκάστην ἡμέραν εὐλογήσω σε, καὶ αἰνέσω τὸ ὄνομά σου εἰς τὸν αἰῶνα καὶ εἰς τὸν αἰῶνα τοῦ αἰῶνος.
\vs{3}Μέγας ὁ Κύριος καὶ αἰνετὸς σφόδρα, καὶ τῆς μεγαλωσύνης αὐτοῦ οὐκ ἔστι πέρας.
\vs{4}Γενεὰ καὶ γενεὰ ἐπαινέσει τὰ ἔργα σου, καὶ τὴν δύναμίν σου ἀπαγγελοῦσι.
\vs{5}Καὶ τὴν μεγαλοπρέπειαν τῆς δόξης τῆς ἁγιωσύνης σου λαλήσουσι, καὶ τὰ θαυμάσιά σου διηγήσονται.
\vs{6}Καὶ τὴν δύναμιν τῶν φοβερῶν σου ἐροῦσι, καὶ τὴν μεγαλωσύνην σου διηγήσονται.
\vs{7}Μνήμην τοῦ πλήθους τῆς χρηστότητός σου ἐξερεύξονται, καὶ τῇ δικαιοσύνῃ σου ἀγαλλιάσονται.

\vs{8}Οἰκτίρμων καὶ ἐλεήμων ὁ Κύριος, μακρόθυμος καὶ πολυέλεος.
\vs{9}Χρηστὸς Κύριος τοῖς ὑπομένουσι, καὶ οἱ οἰκτιρμοὶ αὐτοῦ ἐπὶ πάντα τὰ ἔργα αὐτοῦ.
\vs{10}Ἐξομολογησάσθωσάν σοι, Κύριε, πάντα τὰ ἔργα σου, καὶ οἱ ὅσιοί σου εὐλογησάτωσάν σε.
\vs{11}Δόξαν τῆς βασιλείας σου ἐροῦσι, καὶ τὴν δυναστείαν σου λαλήσουσιν·
\vs{12}Τοῦ γνωρίσαι τοῖς υἱοῖς τῶν ἀνθρώπων τὴν δυναστείαν σου, καὶ τὴν δόξαν τῆς μεγαλοπρεπείας τῆς βασιλείας σου.
\vs{13}Ἡ βασιλεία σου βασιλεία πάντων τῶν αἰώνων, καὶ ἡ δεσποτία σου ἐν πάσῃ γενεᾷ καὶ γενεᾷ·
\vs{13a}πιστὸς Κύριος ἐν τοῖς λόγοις αὐτοῦ, καὶ ὅσιος ἐν πᾶσι τοῖς ἔργοις αὐτοῦ.

\vs{14}Ὑποστηρίζει Κύριος πάντας τοὺς καταπίπτοντας, καὶ ἀνορθοῖ πάντας τοὺς κατεῤῥαγμένους.
\vs{15}Οἱ ὀφθαλμοὶ πάντων εἰς σὲ ἐλπίζουσι, καὶ σὺ δίδως τὴν τροφὴν αὐτῶν ἐν εὐκαιρίᾳ.
\vs{16}Ἀνοίγεις σὺ τὰς χεῖράς σου, καὶ ἐμπιπλᾷς πᾶν ζῶον εὐδοκίας.
\vs{17}Δίκαιος Κύριος ἐν πάσαις ταῖς ὁδοῖς αὐτοῦ, καὶ ὅσιος ἐν πᾶσι τοῖς ἔργοις αὐτοῦ.

\vs{18}Ἐγγὺς Κύριος πᾶσιν τοῖς ἐπικαλουμένοις αὐτόν, πᾶσι τοῖς ἐπικαλουμένοις αὐτὸν ἐν ἀληθείᾳ.
\vs{19}Θέλημα τῶν φοβουμένων αὐτὸν ποιήσει, καὶ τῆς δεήσεως αὐτῶν ἐπακούσεται, καὶ σώσει αὐτούς.
\vs{20}Φυλάσσει Κύριος πάντας τοὺς ἀγαπῶντας αὐτόν, καὶ πάντας τοὺς ἁμαρτωλοὺς ἐξολοθρεύσει.
\vs{21}Αἴνεσιν Κυρίου λαλήσει τὸ στόμα μου, καὶ εὐλογείτω πᾶσα σὰρξ τὸ ὄνομα τὸ ἅγιον αὐτοῦ, εἰς τὸν αἰῶνα καὶ εἰς τὸν αἰῶνα τοῦ αἰῶνος.

\begin{psalmheading}{\ch{145}{146} Ἀλληλούϊα· Ἀγγαίου καὶ Ζαχαρίου.}
\end{psalmheading}
Αἰνεῖ ἡ ψυχή μου τὸν Κύριον.
\vs{2}Αἰνέσω Κύριον ἐν ζωῇ μοῦ, ψαλῶ τῷ Θεῷ μου ἕως ὑπάρχω.
\vs{3}Μὴ πεποίθατε ἐπʼ ἄρχοντας, καὶ ἐφʼ υἱοὺς ἀνθρώπων, οἷς οὐκ ἔστι σωτηρία.
\vs{4}Ἐξελεύσεται τὸ πνεῦμα αὐτοῦ καὶ ἐπιστρέψει εἰς τὴν γῆν αὐτοῦ, ἐν ἐκείνῃ τῇ ἡμέρᾳ ἀπολοῦνται πάντες οἱ διαλογισμοὶ αὐτῶν.

\vs{5}Μακάριος οὗ ὁ Θεὸς Ἰακὼβ βοηθός αὐτοῦ, ἡ ἐλπὶς αὐτοῦ ἐπὶ Κύριον τὸν Θεὸν αὐτοῦ·
\vs{6}Τὸν ποιήσαντα τὸν οὐρανὸν καὶ τὴν γῆν, τὴν θάλασσαν καὶ πάντα τὰ ἐν αὐτοῖς· τὸν φυλάσσοντα ἀλήθειαν εἰς τὸν αἰῶνα,
\vs{7}ποιοῦντα κρίμα τοῖς ἀδικουμένοις, διδόντα τροφὴν τοῖς πεινῶσι. Κύριος λύει πεπεδημένους,
\vs{8}Κύριος σοφοῖ τυφλοὺς,

Κύριος ἀνορθοῖ κατεῤῥαγμένους, Κύριος ἀγαπᾷ δικαίους,
\vs{9}Κύριος φυλάσσει τοὺς προσηλύτους· ὀρφανὸν καὶ χήραν ἀναλήμψεται, καὶ ὁδὸν ἁμαρτωλῶν ἀφανιεῖ.
\vs{10}Βασιλεύσει Κύριος εἰς τὸν αἰῶνα, ὁ Θεός σου, Σιὼν, εἰς γενεὰν καὶ γενεάν.

\begin{psalmheading}{\ch{146}{147:1-11} Ἀλληλούϊα· Ἀγγαίου καὶ Ζαχαρίου.}
\end{psalmheading}
Αἰνεῖτε τὸν Κύριον ὅτι ἀγαθὸν ψαλμός, τῷ Θεῷ ἡμῶν ἡδυνθείη αἴνεσις.
\vs{2}Οἰκοδομῶν Ἱερουσαλὴμ ὁ Κύριος, καὶ τὰς διασπορὰς τοῦ Ἰσραὴλ ἐπισυνάξει·
\vs{3}Ὁ ἰώμενος τοὺς συντετριμμένους τὴν καρδίαν, καὶ δεσμεύων τὰ συντρίμματα αὐτῶν·
\vs{4}Ὁ ἀριθμῶν πλήθη ἄστρων, καὶ πᾶσιν αὐτοῖς ὀνόματα καλῶν.
\vs{5}Μέγας ὁ Κύριος ἡμῶν, καὶ μεγάλη ἡ ἰσχὺς αὐτοῦ, καὶ τῆς συνέσεως αὐτοῦ οὐκ ἔστιν ἀριθμός.
\vs{6}Ἀναλαμβάνων πρᾳεῖς ὁ Κύριος, ταπεινῶν δὲ ἁμαρτωλοὺς ἕως τῆς γῆς.

\vs{7}Ἐξάρξατε τῷ Κυρίῳ ἐν ἐξομολογήσει, ψάλατε τῷ Θεῷ ἡμῶν ἐν κιθάρᾳ·
\vs{8}Τῷ περιβάλλοντι τὸν οὐρανὸν ἐν νεφέλαις, τῷ ἑτοιμάζοντι τῇ γῇ ὑετόν· τῷ ἐξανατέλλοντι ἐν ὄρεσι χόρτον, καὶ χλόην τῇ δουλείᾳ τῶν ἀνθρώπων·
\vs{9}καὶ διδόντι τοῖς κτήνεσι τροφὴν αὐτῶν, καὶ τοῖς νεοσσοῖς τῶν κοράκων τοῖς ἐπικαλουμένοις αὐτόν.
\vs{10}Οὐκ ἐν τῇ δυναστείᾳ τοῦ ἵππου θελήσει, οὐδὲ ἐν ταῖς κνήμαις τοῦ ἀνδρὸς εὐδοκεῖ.
\vs{11}Εὐδοκεῖ Κύριος ἐν τοῖς φοβουμένοις αὐτὸν, καὶ ἐν πᾶσιν τοῖς ἐλπιζουσιν ἐπὶ τὸ ἔλεος αὐτοῦ.

\begin{psalmheading}{\ch{147}{147:12-20} Ἀλληλούϊα· Ἀγγαίου καὶ Ζαχαρίου.}
\end{psalmheading}
Ἐπαίνει, Ἱερουσαλήμ, τὸν Κύριον, αἴνει τὸν Θεόν σου Σιών.
\vs{2}Ὅτι ἐνίσχυσεν τοὺς μοχλοὺς τῶν πυλῶν σου, εὐλόγησεν τοὺς υἱούς σου ἐν σοί.
\vs{3}Ὁ τιθεὶς τὰ ὅριά σου εἰρήνην, καὶ στέαρ πυροῦ ἐμπιπλῶν σε·
\vs{4}Ὁ ἀποστέλλων τὸ λόγιον αὐτοῦ τῇ γῇ, ἕως τάχους δραμεῖται ὁ λόγος αὐτοῦ·
\vs{5}Τοῦ διδόντος χιόνα ὡσεὶ ἔριον, ὁμίχλην ὡσεὶ σποδὸν πάσσοντος·
\vs{6}Βάλλοντος κρύσταλλον αὐτοῦ ὠσεὶ ψωμούς· κατὰ πρόσωπον ψύχους αὐτοῦ τίς ὑποστήσεται;
\vs{7}Ἀποστελεῖ τὸν λόγον αὐτοῦ, καὶ τήξει αὐτά, πνεύσει τὸ πνεῦμα αὐτοῦ, καὶ ῥυήσεται ὕδατα.
\vs{8}Ἀπαγγέλλων τὸν λόγον αὐτοῦ τῷ Ἰακώβ, δικαιώματα καὶ κρίματα αὐτοῦ τῷ Ἰσραήλ.
\vs{9}Οὐκ ἐποίησεν οὕτως παντὶ ἔθνει, καὶ τὰ κρίματα αὐτοῦ οὐκ ἐδήλωσεν αὐτοῖς.

\begin{psalmheading}{\ch{148}{148} Ἀλληλούϊα· Ἀγγαίου καὶ Ζαχαρίου.}
\end{psalmheading}
Αἰνεῖτε τὸν Κύριον ἐκ τῶν οὐρανῶν, αἰνεῖτε αὐτὸν ἐν τοῖς ὑψίστοις.
\vs{2}Αἰνεῖτε αὐτόν πάντες οἱ ἄγγελοι αὐτοῦ, αἰνεῖτε αὐτόν πᾶσαι αἱ δυνάμεις αὐτοῦ.
\vs{3}Αἰνεῖτε αὐτόν ἥλιος καὶ σελήνη, αἰνεῖτε αὐτόν πάντα τὰ ἄστρα καὶ τὸ φῶς.
\vs{4}Αἰνεῖτε αὐτόν οἱ οὐρανοὶ τῶν οὐρανῶν, καὶ τὸ ὕδωρ τὸ ὑπεράνω τῶν οὐρανῶν.
\vs{5}Αἰνεσάτωσαν τὸ ὄνομα Κυρίου· ὅτι αὐτὸς εἶπεν καὶ ἐγενήθησαν, αὐτὸς ἐνετείλατο καὶ ἐκτίσθησαν.
\vs{6}Ἔστησεν αὐτὰ εἰς τὸν αἰῶνα, καὶ εἰς τὸν αἰῶνα τοῦ αἰῶνος· πρόσταγμα ἔθετο, καὶ οὐ παρελεύσεται.

\vs{7}Αἰνεῖτε τὸν Κύριον ἐκ τῆς γῆς, δράκοντες καὶ πᾶσαι ἄβυσσοι·
\vs{8}Πῦρ, χάλαζα, χιὼν, κρύσταλλος, πνεῦμα καταιγίδος, τὰ ποιοῦντα τὸν λόγον αὐτοῦ·
\vs{9}Τὰ ὄρη καὶ πάντες βουνοί, ξύλα καρποφόρα καὶ πᾶσαι κέδροι·
\vs{10}Τὰ θηρία καὶ πάντα τὰ κτήνη, ἑρπετὰ καὶ πετεινὰ πτερωτά·
\vs{11}Βασιλεῖς τῆς γῆς καὶ πάντες λαοί, ἄρχοντες καὶ πάντες κριταὶ γῆς·
\vs{12}Νεανίσκοι καὶ παρθένοι, πρεσβύται μετὰ νεωτέρων
\vs{13}αἰνεσάτωσαν τὸ ὄνομα Κυρίου, ὅτι ὑψώθη τὸ ὄνομα αὐτοῦ μόνου· ἡ ἐξομολόγησις αὐτοῦ ἐπὶ γῆς καὶ οὐρανοῦ,
\vs{14}καὶ ὑψώσει κέρας λαοῦ αὐτοῦ· ὕμονς πᾶσι τοῖς ὁσίοις αὐτοῦ, τοῖς υἱοῖς Ἰσραήλ, λαῷ ἐγγίζοντι αὐτῷ.

\begin{psalmheading}{\ch{149}{149} Ἀλληλούϊα.}
\end{psalmheading}
Ἄσατε τῷ Κυρίῳ ᾆσμα καινόν· ἡ αἴνεσις αὐτοῦ ἐν ἐκκλησίᾳ ὁσίων.
\vs{2}Εὐφρανθήτω Ἰσραὴλ ἐπὶ τῷ ποιήσαντι αὐτὸν, καὶ υἱοὶ Σιὼν ἀγαλλιάσθωσαν ἐπὶ τῷ βασιλεῖ αὐτῶν.
\vs{3}Αἰνεσάτωσαν τὸ ὄνομα αὐτοῦ ἐν χορῷ, ἐν τυμπάνῳ καὶ ψαλτηρίῳ ψαλάτωσαν αὐτῷ.
\vs{4}Ὅτι εὐδοκεῖ Κύριος ἐν λαῷ αὐτοῦ, καὶ ὑψώσει πρᾳεῖς ἐν σωτηρίᾳ.

\vs{5}Καυχήσονται ὅσιοι ἐν δόξῃ, καὶ ἀγαλλιάσονται ἐπὶ τῶν κοιτῶν αὐτῶν.
\vs{6}Αἱ ὑψώσεις τοῦ Θεοῦ ἐν λάρυγγι αὐτῶν, καὶ ῥομφαῖαι δίστομοι ἐν ταῖς χερσὶν αὐτῶν·
\vs{7}Τοῦ ποιῆσαι ἐκδίκησιν ἐν τοῖς ἔθνεσιν, ἐλεγμοὺς ἐν τοῖς λαοῖς·
\vs{8}Τοῦ δῆσαι τοὺς βασιλεῖς αὐτῶν ἐν πέδαις, καὶ τοὺς ἐνδόξους αὐτῶν ἐν χειροπέδαις σιδηραῖς·
\vs{9}Τοῦ ποιῆσαι ἐν αὐτοῖς κρίμα ἔγγραπτον· δόξα αὕτη ἐστὶ πᾶσι τοῖς ὁσίοις αὐτοῦ.

\begin{psalmheading}{\ch{150}{150} Ἁλληλούϊα.}
\end{psalmheading}
Αἰνεῖτε τὸν Θεὸν ἐν τοῖς ἁγίοις αὐτοῦ, αἰνεῖτε αὐτὸν ἐν στερεώματι δυνάμεως αὐτοῦ.
\vs{2}Αἰνεῖτε αὐτὸν ἐπὶ ταῖς δυναστείαις αὐτοῦ, αἰνεῖτε αὐτὸν κατὰ τὸ πλῆθος τῆς μεγαλωσύνης αὐτοῦ.
\vs{3}Αἰνεῖτε αὐτὸν ἐν ἤχῳ σάλπιγγος, αἰνεῖτε αὐτὸν ἐν ψαλτηρίῳ καὶ κιθάρᾳ.
\vs{4}Αἰνεῖτε αὐτὸν ἐν τυμπάνῳ καὶ χορῷ, αἰνεῖτε αὐτὸν ἐν χορδαῖς καὶ ὀργάνῳ.
\vs{5}Αἰνεῖτε αὐτὸν ἐν κυμβάλοις εὐήχοις, αἰνεῖτε αὐτὸν ἐν κυμβάλοις ἀλαλαγμοῦ.
\vs{6}Πᾶσα πνοὴ αἰνεσάτω τὸν Κύριον.

\begin{psalmheading}{\ch{151}{} Οὗτος ὁ ψαλμὸς ἰδιόγραφος εἰς Δαυὶδ, καὶ ἔξωθεν τοῦ ἀριθμοῦ, ὅτε ἐμονομάχησε τῷ Γολιάδ.}
\end{psalmheading}
Μικρὸς ἤμην ἐν τοῖς ἀδελφοίς μου, καὶ νεώτερος ἐν τῷ οἴκῳ τοῦ πατρός μου, ἐποίμαινον τὰ πρόβατα τοῦ πατρός μου.
\vs{2}Αἱ χεῖρές μου ἐποίησαν ὄργανον, καὶ οἱ δάκτυλοί μου ἥρμοσαν ψαλτήριον.
\vs{3}Καὶ τίς ἀναγγελεῖ τῷ Κυρίῳ μου; αὐτὸς Κύριος, αὐτὸς εἰσακούει.
\vs{4}Αὐτὸς ἐξαπέστειλεν τὸν ἄγγελον αὐτοῦ, καὶ ᾖρέ με ἐκ τῶν προβάτων τοῦ πατρός μου, καὶ ἔχρισέ με ἐν τῷ ἐλαίῳ τῆς χρίσεως αὐτοῦ.
\vs{5}Οἱ ἀδελφοί μου καλοὶ καὶ μεγάλοι, καὶ οὐκ εὐδόκησεν ἐν αὐτοῖς Κύριος.
\vs{6}Ἐξῆλθον εἰς συνάντησιν τῷ ἀλλοφύλῳ, καὶ ἐπικατηράσατό με ἐν τοῖς εἰδώλοις αὐτοῦ·
\vs{7}Ἐγὼ δὲ σπασάμενος τὴν παρʼ αὐτοῦ μάχαιραν, ἀπεκεφάλισα αὐτόν, καὶ ᾖρα ὄνειδος ἐξ υἱῶν Ἰσραήλ.
\end{psalms}


\def\book{ΠΑΡΟΙΜΙΑΙ ΣΑΛΩΜΩΝΤΟΣ}
\biblebook{ΠΑΡΟΙΜΙΑΙ ΣΑΛΩΜΩΝΤΟΣ}


\lettrine[lines=2, loversize=0.2, nindent=0em, findent=.25em]{\textcolor{bookheadingcolor}{Π}}{ΑΡΟΙΜΙΑΙ} Σαλωμῶντος υἱοῦ Δαυὶδ, ὃς ἐβασίλευσεν ἐν Ἰσραήλ·
\vs{2}γνῶναι σοφίαν καὶ παιδείαν, νοῆσαί τε λόγους φρονήσεως,
\vs{3}δέξασθαί τε στροφὰς λόγων, νοῆσαί τε δικαιοσύνην ἀληθῆ, καὶ κρίμα κατευθύνειν·
\vs{4}Ἵνα δῷ ἀκάκοις πανουργίαν, παιδὶ δὲ νέῳ αἴσθησίν τε καὶ ἔννοιαν.
\vs{5}Τῶν δὲ γὰρ ἀκούσας σοφὸς σοφώτερος ἔσται, ὁ δὲ νοήμων κυβέρνησιν κτήσεται·
\vs{6}Νοήσει τε παραβολὴν καὶ σκοτεινὸν λόγον, ῥήσεις τε σοφῶν καὶ αἰνίγματα.

\vs{7}Ἀρχὴ σοφίας φόβος Κυριου, σύνεσις δὲ ἀγαθὴ πᾶσι τοῖς ποιοῦσιν αὐτήν· εὐσέβεια δὲ εἰς Θεὸν ἀρχὴ αἰσθήσεως, σοφίαν δὲ καὶ παιδείαν ἀσεβεῖς ἐξουθενήσουσιν.
\vs{8}Ἄκουε υἱὲ παιδείαν πατρός σου, καὶ μὴ ἀπώσῃ θεσμοὺς μητρός σου.
\vs{9}Στέφανον γὰρ χαρίτων δέξῃ σῇ κορυφῇ, καὶ κλοιὸν χρύσεον περὶ σῷ τραχήλῳ.

\vs{10}Υἱὲ μή σε πλανήσωσιν ἄνδρες ἀσεβεῖς, μηδὲ βουληθῇς.
\vs{11}Ἐὰν παρακαλέσωσί σε, λέγοντες, ἐλθὲ μεθʼ ἡμῶν, κοινώνησον αἵματος, κρύψωμεν δὲ εἰς γῆν ἄνδρα δίκαιον ἀδίκως,
\vs{12}καταπίωμεν δὲ αὐτὸν ὥσπερ ᾅδης ζῶντα, καὶ ἄρωμεν αὐτοῦ τὴν μνήμην ἐκ γῆς,
\vs{13}τὴν κτῆσιν αὐτοῦ τὴν πολυτελῆ καταλαβώμεθα, πλήσωμεν δὲ οἴκους ἡμετέρους σκύλων·
\vs{14}Τὸν δὲ σὸν κλῆρον βάλε ἐν ἡμῖν, κοινὸν δὲ βαλάντιον κτησώμεθα πάντες, καὶ μαρσίππιον ἓν γενηθήτω ἡμῖν·
\vs{15}Μὴ πορευθῇς ἐν ὁδῷ μετʼ αὐτῶν, ἔκκλινον δὲ τὸν πόδα σου ἐκ τῶν τρίβων αὐτῶν.
\vs{17}Οὐ γὰρ ἀδίκως ἐκτείνεται δίκτυα πτερωτοῖς.
\vs{18}Αὐτοὶ γὰρ οἱ φόνου μετέχοντες, θησαυρίζουσιν ἑαυτοῖς κακά· ἡ δὲ καταστροφὴ ἀνδρῶν παρανόμων κακή.
\vs{19}Αὗται αἱ ὁδοί εἰσι πάντων τῶν συντελούντων τὰ ἄνομα· τῇ γὰρ ἀσεβείᾳ τὴν ἑαυτῶν ψυχὴν ἀφαιροῦνται.

\vs{20}Σοφία ἐν ἐξόδοις ὑμνεῖται, ἐν δὲ πλατείαις παῤῥησίαν ἄγει.
\vs{21}Ἐπʼ ἄκρων δὲ τειχέων κηρύσσεται, ἐπὶ δὲ πύλαις δυναστῶν παρεδρεύει, ἐπὶ δὲ πύλαις πόλεως θαῤῥοῦσα λέγει,
\vs{22}ὅσον ἂν χρόνον ἄκακοι ἔχονται τῆς δικαιοσύνης, οὐκ αἰσχυνθήσονται· οἱ δὲ ἄφρονες τῆς ὕβρεως ὄντες ἐπιθυμηταί, ἀσεβεῖς γενόμενοι ἐμίσησαν αἴσθησιν,
\vs{23}καὶ ὑπεύθυνοι ἐγένοντο ἐλέγχοις· ἰδοὺ προήσομαι ὑμῖν ἐμῆς πνοῆς ῥῆσιν· διδάξω δὲ ὑμᾶς τὸν ἐμὸν λόγον.

\vs{24}Ἐπειδὴ ἐκάλουνμ, καὶ οὐχ ὑπηκούσατε· καὶ ἐξέτεινον λόγους, καὶ οὐ προσείχετε·
\vs{25}ἀλλὰ ἀκύρους ἐποιεῖτε ἐμὰς βουλὰς, τοῖς δὲ ἐμοῖς ἐλέγχοις ἠπειθήσατε·
\vs{26}Τοιγαροῦν κᾀγὼ τῇ ὑμετέρᾳ ἀπωλείᾳ ἐπιγελάσομαι, καταχαροῦμαι δὲ ἡνίκα ἔρχηται ὑμῖν ὄλεθρος·
\vs{27}Καὶ ὡς ἂν ἀφίκηται ὑμῖν ἄφνω θόρυβος, ἡ δὲ καταστροφὴ ὁμοίως καταιγίδι παρῇ, καὶ ὅταν ἔρχηται ὑμῖν θλίψις καὶ πολιορκία, ἢ ὅταν ἔρχηται ὑμῖν ὄλεθρος.
\vs{28}Ἔσται γὰρ ὅταν ἐπικαλέσησθέ με, ἐγὼ δὲ οὐκ εἰσακούσομαι ὑμῶν· ζητήσουσί με κακοὶ, καὶ οὐχ εὑρήσουσιν.
\vs{29}Ἐμίσησαν γὰρ σοφίαν, τὸν δὲ λόγον τοῦ Κυρίου οὐ προείλαντο,
\vs{30}οὐδὲ ἤθελον ἐμαῖς προσέχειν βουλαῖς, ἐμυκτήριζον δὲ ἐμοὺς ἐλέγχους·
\vs{31}Τοιγαροῦν ἔδονται τῆς ἑαυτῶν ὁδοῦ τοὺς καρποὺς, καὶ τῆς ἐαυτῶν ἀσεβείας πλησθήσονται.
\vs{32}Ἀνθʼ ὧν γὰρ ἠδίκουν νηπίους, φονευθήσονται, καὶ ἐξετασμὸς ἀσεβεῖς ὀλεῖ.
\vs{33}Ὁ δὲ ἐμοῦ ἀκούων κατασκηνώσει ἐπʼ ἐλπίδι, καὶ ἡσυχάσει ἀφόβως ἀπὸ παντὸς κακοῦ.

\ch{2}
Υἱὲ, ἐὰν δεξάμενος ῥῆσιν ἐμῆς ἐντολῆς κρύψῃς παρὰ σεαυτῷ,
\vs{2}ὑπακούσεται σοφίας τὸ οὖς σου, καὶ παραβαλεῖς καρδίαν σου εἰς σύνεσιν, παραβαλεῖς δὲ αὐτὴν ἐπὶ νουθέτησιν τῷ υἱῷ σου·

\vs{3}Ἐὰν γὰρ τὴν σοφίαν ἐπικαλέσῃ, καὶ τῇ συνέσει δῷς φωνήν σου,
\vs{4}καὶ ἐὰν ζητήσῃς αὐτὴν ὡς ἀργύριον, καὶ ὡς θησευροὺς ἐξεραυνήσῃς αὐτήν·
\vs{5}Τότε συνήσεις φόβον Κυρίου, καὶ ἐπίγνωσιν Θεοῦ εὑρήσεις.

\vs{6}Ὅτι Κύριος δίδωσι σοφίαν, καὶ ἀπὸ προσώπου αὐτοῦ γνῶσις καὶ σύνεσις.
\vs{7}Καὶ θησαυρίζει τοῖς κατορθοῦσι σωτηρίαν, ὑπερασπιεῖ τὴν πορείαν αὐτῶν,
\vs{8}τοῦ φυλάξαι ὁδοὺς δικαιωμάτων, καὶ ὁδὸν εὐλαβουμένων αὐτὸν διαφυλάξει.
\vs{9}Τότε συνήσεις δικαιοσύνην καὶ κρίμα, καὶ κατορθώσεις πάντας ἄξονας ἀγαθούς.

\vs{10}Ἐὰν γὰρ ἔλθῃ ἡ σοφία εἰς σὴν διάνοιαν, ἡ δὲ αἴσθησις τῇ σῇ ψυχῇ καλὴ εἶναι δόξῃ,
\vs{11}βουλὴ καλὴ φυλάξει σε, ἔννοια δὲ ὁσία τηρήσει σε·
\vs{12}Ἵνα ῥύσηταί σε ἀπὸ ὁδοῦ κακῆς, καὶ ἀπὸ ἀνδρὸς λαλοῦντος μηδὲν πιστόν.

\vs{13}Ὦ οἱ ἐγκαταλείποντες ὁδοὺς εὐθείας τοῦ πορεύεσθαι ἐν ὁδοῖς σκότους·
\vs{14}Οἱ εὐφραινόμενοι ἐπὶ κακοῖς καὶ χαίροντες ἐπὶ διαστροφῇ κακῇ·
\vs{15}Ὧν αἱ τρίβοι σκολιαὶ, καὶ καμπύλαι αἱ τροχιαὶ αὐτῶν,
\vs{16}τοῦ μακράν σε ποιῆσαι ἀπὸ ὁδοῦ εὐθείας, καὶ ἀλλότριον τῆς δικαίας γνώμης· υἱὲ, μή σε καταλάβῃ κακὴ βουλή·
\vs{17}Ἡ ἀπολιποῦσα διδασκαλίαν νεότητος, καὶ διαθήκην θείαν ἐπιλελησμένη.
\vs{18}Ἔθετο γὰρ παρὰ τῷ θανάτῳ τὸν οἶκον αὐτῆς, καὶ παρὰ τῷ ᾅδῃ μετὰ τῶν γηγενῶν τοὺς ἄξονας αὐτῆς.
\vs{19}Πάντες οἱ πορευόμενοι ἐν αὐτῇ οὐκ ἀναστρέψουσιν, οὐδὲ μὴ καταλάβωσι τρίβους εὐθείας· οὐ γὰρ καταλαμβάνονται ὑπὸ ἐνιαυτῶν ζωῆς.
\vs{20}Εἰ γὰρ ἐπορεύοντο τρίβους ἀγαθὰς, εὕροσαν ἂν τρίβους δικαιοσύνης λείας.
\vs{21}Ὅτι εὐθεῖς κατασκηνώσουσι γῆν, καὶ ὅσιοι ὑπολειφθήσονται ἐν αὐτῇ.
\vs{22}Ὁδοὶ ἀσεβῶν ἐκ γῆς ὀλοῦνται, οἱ δὲ παράνομοι ἐξωσθήσονται ἀπʼ αὐτῆς.

\ch{3}
Υἱὲ, ἐμῶννομίμων μὴ ἐπιλανθάνου, τὰ δὲ ῥήματά μου τηρείτω σὴ καρδία·
\vs{2}Μῆκος γὰρ βίου, καὶ ἔτη ζωῆς, καὶ εἰρήνην προσθήσουσί σοι.
\vs{3}Ἐλεημοσύναι καὶ πίστεις μὴ ἐκλειπέτωσάν σε· ἄφαψαι δὲ αὐτὰς ἐπὶ σῷ τραχήλῳ, καὶ εὑρήσεις χάριν·
\vs{4}καὶ προνοοῦ καλὰ ἐνώπιον Κυρίου καὶ ἀνθρώπων.

\vs{5}Ἴσθι πεποιθὼς ἐν ὅλῃ τῇ καρδίᾳ ἐπὶ Θεῷ, ἐπὶ δὲ σῇ σοφίᾳ μὴ ἐπαίρου.
\vs{6}Πάσαις ὁδοῖς σου γνώριζε αὐτὴν, ἵνα ὀρθοτομῇ τὰς ὁδούς σου.
\vs{7}Μὴ ἴσθι φρόνιμος παρὰ σεαυτῷ, φοβοῦ δὲ τὸν Θεὸν, καὶ ἔκκλινε ἀπὸ παντὸς κακοῦ.
\vs{8}Τότε ἴασις ἔσται τῷ σώματί σου, καὶ ἐπιμέλεια τοῖς ὀστέοις σου.

\vs{9}Τίμα τὸν Κύριον ἀπὸ σῶν δικαίων πόνων, καὶ ἀπάρχου αὐτῷ ἀπὸ σῶν καρπῶν δικαιοσύνης·
\vs{10}Ἵνα πίμπληται τὰ ταμιεῖά σου πλησμονῆς σίτῳ, οἴνῳ δὲ αἱ ληνοί σου ἐκβλύζωσιν.

\vs{11}Υἱὲ, μὴ ὀλιγώρει παιδείας Κυρίου, μηδὲ ἐκλύου ὑπʼ αὐτοῦ ἐλεγχόμενος.
\vs{12}Ὃν γὰρ ἀγαπᾷ Κύριος, ἐλέγχει, μαστιγοῖ δὲ πάντα υἱὸν ὃν παραδέχεται.

\vs{13}Μακάριος ἄνθρωπος ὃς εὗρε σοφίαν, καὶ θνητὸς ὃς εἶδε φρόνησιν.
\vs{14}Κρεῖσσον γὰρ αὐτὴν ἐμπορεύεσθαι, ἢ χρυσίου καὶ ἀργυρίου θησαυρούς.
\vs{15}Τιμιωτέρα δέ ἐστι λίθων πολυτελῶν, οὐκ ἀντιτάξεται αὐτῇ οὐδὲν πονηρόν· εὔγνωστός ἐστι πᾶσι τοῖς ἐγγίζουσιν αὐτῇ, πᾶν δὲ τίμιον οὐκ ἄξιον αὐτῆς ἐστι.
\vs{16}Μῆκος γὰρ βίου καὶ ἔτη ζωῆς ἐν τῇ δεξιᾷ αὐτῆς, ἐν δὲ τῇ ἀριστερᾷ αὐτῆς πλοῦτος καὶ δόξα·
\vs{16a}ἐκ τοῦ στόματος αὐτῆς ἐκπορεύεται δικαιοσύνη, νόμον δὲ καὶ ἔλεον ἐπὶ γλώσσης φορεῖ.
\vs{17}Αἱ ὁδοὶ αὐτῆς ὁδοὶ καλαί, καὶ πάσαι αἱ τρίβοι αὐτῆς ἐν εἰρήνῃ.
\vs{18}Ξύλον ζωῆς ἐστι πᾶσι τοῖς ἀντεχομένοις αὐτῆς, καὶ τοῖς ἐπερειδομένοις ἐπʼ αὐτὴν ὡς ἐπὶ Κύριον ἀσφαλής.

\vs{19}Ὁ Θεὸς τῇ σοφίᾳ ἐθεμελίωσε τὴν γῆν, ἡτοίμασε δὲ οὐρανοὺς φρονήσει.
\vs{20}Ἐν αἰσθήσει ἄβυσσοι ἐῤῥάγησαν, νέφη δὲ ἐῤῥύησαν δρόσους.

\vs{21}Υἱὲ, μὴ παραῤῥυῇς, τήρησον δὲ ἐμὴν βουλὴν καὶ ἔννοιαν·
\vs{22}ἵνα ζήσῃ ἡ ψυχή σου, καὶ χάρις ᾖ περὶ σῷ τραχήλῳ·
\vs{22a}ἔσται δὲ ἴασις ταῖς σαρξί σου, καὶ ἐπιμέλεια τοῖς σοῖς ὀστέοις·
\vs{23}ἵνα πορεύῃ πεποιθὼς ἐν εἰρήνῃ πάσας τὰς ὁδούς σου, ὁ δὲ πούς σου οὐ μὴ προσκόψῃ.
\vs{24}Ἐὰν γὰρ κάθῃ, ἄφοβος ἔσῃ· ἐὰν δὲ καθεύδῃς, ἡδέως ὑπνώσεις.
\vs{25}Καὶ οὐ φοβηθήσῃ πτόησιν ἐπελθοῦσαν, οὐδὲ ὁρμὰς ἀσεβῶν ἐπερχομένας.
\vs{26}Ὁ γὰρ Κύριος ἔσται ἐπὶ πασῶν ὁδῶν σου, καὶ ἐρείσει σὸν πόδα ἵνα μὴ σαλευθῇς.

\vs{27}Μὴ ἀπόσχῃ εὖ ποιεῖν ἐνδεῆ, ἡνίκα ἂν ἔχῃ ἡ χείρ σου βοηθεῖν.
\vs{28}Μὴ εἴπῃς, ἐπανελθὼν ἐπάνηκε, αὔριον δώσω, δυνατοῦ σου ὄντος εὖ ποιεῖν· οὐ γὰρ οἶδας τί τέξεται ἡ ἐπιοῦσα.
\vs{29}Μὴ τεκτῄνῃ ἐπὶ σὸν φίλον κακὰ παροικοῦντα καὶ πεποιθότα ἐπὶ σοί.

\vs{30}Μὴ φιλεχθρήσῃς πρὸς ἄνθρωπον μάτην, μήτί σε ἐργάσηται κακόν.

\vs{31}Μὴ κτήσῃ κακῶν ἀνδρῶν ὀνείδη, μηδὲ ζηλώσῃς τὰς ὁδοὺς αὐτῶν.
\vs{32}Ἀκάθαρτος γὰρ ἔναντι Κυρίου πᾶς παράνομος, ἐν δὲ δικαίοις οὐ συνεδριάζει.
\vs{33}Κατάρα Θεοῦ ἐν οἴκοις ἀσεβῶν, ἐπαύλεις δὲ δικαίων εὐλογοῦνται.
\vs{34}Κύριος ὑπερηφάνοις ἀντιτάσσεται, ταπεινοῖς δὲ δίδωσι χάριν.
\vs{35}Δόξαν σοφοὶ κληρονομήσουσιν, οἱ δὲ ἀσεβεῖς ὕψωσαν ἀτιμίαν.

\ch{4}
Ἀκούσατε, παῖδες, παιδείαν πατρὸς, καὶ προσέχετε γνῶναι ἔννοιαν.
\vs{2}Δῶρον γὰρ ἀγαθὸν δωροῦμαι ὑμῖν, τὸν ἐμὸν νόμον μὴ ἐγκαταλίπητε.
\vs{3}Υἱὸς γὰρ ἐγενόμην κᾀγὼ πατρὶ ὑπήκοος, καὶ ἀγαπώμενος ἐν προσώπῳ μητρός.
\vs{4}Οἳ ἔλεγον καὶ ἐδίδασκόν με, ἐρειδέτω ὁ ἡμέτερος λόγος εἰς σὴν καρδίαν· φύλασσε ἐντολὰς,
\vs{5}μὴ ἐπιλάθῃ· Μηδὲ παρίδῃς ῥῆσιν ἐμοῦ στόματος,
\vs{6}μηδὲ ἐγκαταλίπῃς αὐτὴν, καὶ ἀνθέξεταί σου· ἐράσθητι αὐτῆς, καὶ τηρήσει σε.
\vs{8}Περιχαράκωσον αὐτὴν, καὶ ὑψώσει σε· τίμησον αὐτὴν, ἵνα σε περιλάβῃ·
\vs{9}Ἵνα δῷ τῇ σῇ κεφαλῇ στέφανον χαρίτων, στεφάνῳ δὲ τρυφῆς ὑπερασπίσῃ σου.

\vs{10}Ἄκουε υἱὲ καὶ δέξαι ἐμοὺς λόγους, καὶ πληθυνθήσεται ἔτη ζωῆς σου, ἵνα σοι γένωνται πολλαὶ ὁδοὶ βίου.
\vs{11}Ὁδοὺς γὰρ σοφίας διδάσκω σε, ἐμβιβάζω δέ σε τροχιαῖς ὀρθαῖς.
\vs{12}Ἐὰν γὰρ πορεύῃ, οὐ συγκλεισθήσεταί σου τὰ διαβήματα· ἐὰν δὲ τρέχῃς, οὐ κοπιάσεις.
\vs{13}Ἐπιλαβοῦ ἐμῆς παιδείας, μὴ ἀφῇς, ἀλλὰ φύλαξον αὐτὴν σεαυτῷ εἰς ζωήν σου.

\vs{14}Ὁδοὺς ἀσεβῶν μὴ ἐπέλθῃς, μηδὲ ζηλώσῃς ὁδοὺς παρανόμων.
\vs{15}Ἐν ᾧ ἂν τόπῳ στρατοπεδεύσωσι, μὴ ἐπέλθῃς ἐκεὶ, ἔκκλινον δὲ ἀπʼ αὐτῶν καὶ παράλλαξον.
\vs{16}Οὐ γὰρ μὴ ὑπνώσωσιν, ἐὰν μὴ κακοποιήσωσιν· ἀφῄρηται ὁ ὕπνος αὐτῶν, καὶ οὐ κοιμῶνται.
\vs{17}Οἵδε γὰρ σιτοῦνται σῖτα ἀσεβείας, οἴνῳ δὲ παρανόμῳ μεθύσκονται.
\vs{18}Αἱ δὲ ὁδοὶ τῶν δικαίων ὁμοίως φωτὶ λάμπουσι, προπορεύονται καὶ φωτίζουσιν, ἕως κατορθώσῃ ἡ ἡμέρα.
\vs{19}Αἱ δὲ ὁδοὶ τῶν ἀσεβῶν σκοτειναὶ, οὐκ οἴδασι πῶς προσκόπτουσιν.

\vs{20}Υἱὲ ἐμῇ ῥήσει πρόσεχε, τοῖς δὲ ἐμοῖς λόγοις παράβαλλε σὸν οὖς.
\vs{21}Ὅπως μὴ ἐκλίπωσί σε αἱ πηγαί σου, φύλασσε αὐτὰς ἐν καρδίᾳ.
\vs{22}Ζωὴ γάρ ἐστι τοῖς εὑρίσκουσιν αὐτὰς, καὶ πάσῃ σαρκὶ ἴασις.
\vs{23}Πάσῃ φυλακῇ τήρει σὴν καρδίαν, ἐκ γὰρ τούτων ἔξοδοι ζωῆς.
\vs{24}Περίελε σεαυτοῦ σκολιὸν στόμα, καὶ ἄδικα χείλη μακρὰν ἀπὸ σοῦ ἄπωσαι.
\vs{25}Οἱ ὀφθαλμοί σου ὀρθὰ βλεπέτωσαν, τὰ δὲ βλέφαρά σου νευέτω δίκαια.
\vs{26}Ὀρθὰς τροχιὰς ποίει σοῖς ποσί, καὶ τὰς ὁδούς σου κατεύθυνε.
\vs{27}Μὴ ἐκκλίνῃς εἰς τὰ δεξιὰ, μηδὲ εἰς τὰ ἀριστερά, ἀπόστρεψον δὲ σὸν πόδα ἀπὸ ὁδοῦ κακῆς·
\vs{27a}ὁδοὺς γὰρ τὰς ἐκ δεξιῶν οἶδεν ὁ Θεὸς, διεστραμμέναι δέ εἰσιν αἱ ἐξ ἀριστερῶν·
\vs{27b}αὐτὸς δὲ ὀρθὰς ποιήσει τὰς τροχιάς σου, τὰς δὲ πορείας σου ἐν εἰρήνῃ προάξει.

\ch{5}
Υἱὲ, ἐμῇ σοφίᾳ πρόσεχε, ἐμοῖς δὲ λόγοις παράβαλλε σὸν οὖς,
\vs{2}ἵνα φυλάξῃς ἔννοιαν ἀγαθήν· αἴσθησις δὲ ἐμῶν χειλέων ἐντέλλεταί σοι·

\vs{3}Μὴ πρόσεχε φαύλῃ γυναικί. Μέλι γὰρ ἀποστάζει ἀπὸ χειλέων γυναικὸς πόρνης, ἣ πρὸς καιρὸν λιπαίνει σὸν φάρυγγα,
\vs{4}ὕστερον μέντοι πικρότερον χολῆς εὑρήσεις, καὶ ἠκονημένον μᾶλλον μαχαίρας διστόμου.
\vs{5}Τῆς γὰρ ἀφροσύνης οἱ πόδες κατάγουσι τοὺς χρωμένους αὐτῇ μετὰ θανάτου εἰς τὸν ᾅδην, τὰ δὲ ἴχνη αὐτῆς οὐκ ἐρείδεται.
\vs{6}Ὁδοὺς γὰρ ζωῆς οὐκ ἐπέρχεται, σφαλεραὶ δὲ αἱ τροχιαὶ αὐτῆς, καὶ οὐκ εὔγνωστοι.

\vs{7}Νῦν οὖν υἱὲ ἄκουέ μου, καὶ μὴ ἀκύρους ποιήσεις ἐμοὺς λόγους.
\vs{8}Μακρὰν ποίησον ἀπʼ αὐτῆς σὴν ὁδόν· μὴ ἐγγίσῃς πρὸς θύραις οἴκων αὐτῆς,
\vs{9}ἵνα μὴ πρόῃ ἄλλοις ζωήν σου, καὶ σὸν βίον ἀνελεήμοσιν·
\vs{10}Ἵνα μὴ πλησθῶσιν ἀλλότριοι σῆς ἰσχύος, οἱ δὲ σοὶ πόνοι εἰς οἴκους ἀλλοτρίων ἔλθεσι·
\vs{11}Καὶ μεταμεληθήσῃ ἐπʼ ἐσχάτων, ἡνίκα ἂν κατατριβῶσι σάρκες σώματός σου,
\vs{12}καὶ ἐρεῖς, πῶς ἐμίσησα παιδείαν, καὶ ἐλέγχους ἐξέκλινεν ἡ καρδία μου;
\vs{13}Οὐκ ἤκουον φωνὴν παιδεύοντός με καὶ διδάσκοντός με, οὐδὲ παρέβαλλον τὸ οὖς μου.
\vs{14}Παρʼ ὀλίγον ἐγενόμην ἐν παντὶ κακῷ, ἐν μέσῳ ἐκκλησίας καὶ συναγωγῆς.

\vs{15}Πίνε ὕδατα ἀπὸ σῶν ἀγγείων, καὶ ἀπὸ σῶν φρεάτων πηγῆς.
\vs{16}Μὴ ὑπερεκχείσθω σοι ὕδατα ἐκ τῆς σῆς πηγῆς, εἰς δὲ σὰς πλατείας διαπορευέσθω τὰ σὰ ὕδατα.
\vs{17}Ἔστω σοι μόνῳ ὑπάρχοντα, καὶ μηδεὶς ἀλλότριος μετασχέτω σοι.
\vs{18}Ἡ πηγή σου τοῦ ὕδατος ἔστω σοι ἰδία, καὶ συνευφραίνου μετὰ γυναικὸς τῆς ἐκ νεότητός σου.
\vs{19}Ἔλαφος φιλίας καὶ πῶλος σῶν χαρίτων ὁμιλείτω σοι, ἡ δὲ ἰδία ἡγείσθω σου καὶ συνέστω σοι ἐν παντὶ καιρῷ· ἐν γὰρ τῇ ταύτης φιλίᾳ συμπεριφερόμενος, πολλοστὸς ἔσῃ.
\vs{20}Μὴ πολὺς ἴσθι πρὸς ἀλλοτρίαν, μηδὲ συνέχου ἀγκάλαις τῆς μὴ ἰδίας.
\vs{21}Ἐνώπιον γάρ εἰσι τῶν τοῦ Θεοῦ ὀφθαλμῶν ὁδοὶ ἀνδρὸς, εἰς δὲ πάσας τὰς τροχιὰς αὐτοῦ σκοπεύει.
\vs{22}Παρανομίαι ἄνδρα ἀγρεύουσι, σειραῖς δὲ τῶν ἑαυτοῦ ἁμαρτιῶν ἕκαστος σφίγγεται.
\vs{23}Οὗτος τελευτᾷ μετὰ ἀπαιδεύτων, ἐκ δὲ πλήθους τῆς ἑαυτοῦ βιότητος ἐξεῤῥίφη, καὶ ἀπώλετο διʼ ἀφροσύνην.

\ch{6}
Υἱὲ, ἐὰν ἐγγυήσῃ σὸν φίλον, παραδώσεις σὴν χεῖρα ἐχθρῷ.
\vs{2}Παγὶς γὰρ ἰσχυρὰ ἀνδρὶ τὰ ἴδια χείλη, καὶ ἁλίσκεται χείλεσιν ἰδίου στόματος.
\vs{3}Ποίει υἱὲ ἃ ἐγώ σοι ἐντέλλομαι, καὶ σώζου· ἥκεις γὰρ εἰς χεῖρας κακῶν διὰ σὸν φίλον· ἴσθι μὴ ἐκλυόμενος, παρόξυνε δὲ καὶ τὸν φίλον σου ὃν ἐνεγγυήσω.
\vs{4}Μὴ δῷς ὕπνον σοῖς ὄμμασι, μηδὲ ἐπινυστάξῃς σοῖς βλεφάροις,
\vs{5}ἵνα σώζῃ ὥσπερ δορκὰς ἐκ βρόχων, καὶ ὥσπερ ὄρνεον ἐκ παγίδος.

\vs{6}Ἴθι πρὸς τὸν μύρμηκα ὦ ὀκνηρὲ, καὶ ζήλωσον ἰδὼν τὰς ὁδοὺς αὐτοῦ, καὶ γενοῦ ἐκείνου σοφώτερος.

\vs{7}Ἐκείνῳ γὰρ γεωργίου μὴ ὑπάρχοντος, μηδὲ τὸν ἀναγκάζοντα ἔχων, μηδὲ ὑπὸ δεσπότην ὢν,
\vs{8}ἐτοιμάζεται θέρους τὴν τροφὴν, πολλήν τε ἐν τῷ ἀμητῷ ποιεῖται τὴν παράθεσιν·
\vs{8a}ἢ πορεύθητι πρὸς τὴν μέλισσαν, καὶ μάθε ὡς ἐργάτις ἐστὶ, τήν τε ἐργασίαν ὡς σεμνὴν ποιεῖται·
\vs{8b}ἧς τοὺς πόνους βασιλεῖς καὶ ἰδιῶται πρὸς ὑγίειαν προσφέρονται· ποθεινὴ δέ ἐστι πᾶσι καὶ ἐπίδοξος,
\vs{8c}καίπερ οὖσα τῇ ῥώμῃ ἀσθενὴς, τὴν σοφίαν τιμήσασα προήχθη.
\vs{9}Ἕως τίνος ὀκνηρὲ κατάκεισαι; πότε δὲ ἐξ ὕπνου ἐγερθήσῃ;
\vs{10}ὀλίγον μὲν ὑπνοῖς, ὀλίγον δὲ κάθησαι, μικρὸν δὲ νυστάζεις, ὀλίγον δὲ ἐναγκαλίζῃ χερσὶ στήθη.
\vs{11}Εἶτʼ ἐνπαραγίνεταί σοι ὥσπερ κακὸς ὁδοιπόρος ἡ πενία, καὶ ἡ ἔνδεια ὥσπερ ἀγαθὸς δρομεύς·
\vs{11a}ἐὰν δὲ ἄοκνος ᾖς, ἥξει ὥσπερ πηγὴ ὁ ἀμητός σου· ἡ δὲ ἔνδεια, ὥσπερ κακὸς δρομεὺς ἀπαυτομολήσει.

\vs{12}Ἀνὴρ ἄφρων καὶ παράνομος πορεύεται ὁδοὺς οὐκ ἀγαθάς.
\vs{13}Ὁ δʼ αὐτὸς ἐννεύει ὀφθαλμῷ, σημαίνει δὲ ποδὶ, διδάσκει δὲ ἐννεύμασι δακτύλων.
\vs{14}Διεστραμμένη καρδία τεκταίνεται κακὰ, ἐν παντὶ καιρῷ ὁ τοιοῦτος ταραχὰς συνίστησιν πόλει.
\vs{15}Διὰ τοῦτο ἐξαπίνης ἔρχεται ἡ ἀπώλεια αὐτοῦ, διακοπὴ καὶ συντριβὴ ἀνίατος.

\vs{16}Ὅτι χαίρει πᾶσιν οἷς μισεῖ ὁ Θεὸς, συντρίβεται δὲ διʼ ἀκαθαρσίαν ψυχῆς.
\vs{17}Ὀφθαλμὸς ὑβριστοῦ, γλῶσσα ἄδικος· χεῖρες ἐκχέουσαι αἷμα δικαίου,
\vs{18}καὶ καρδία τεκταινομένη λογισμοὺς κακοὺς, καὶ πόδες ἐπισπεύδοντες κακοποιεῖν.
\vs{19}Ἐκκαίει ψευδῆ μάρτυς ἄδικος, καὶ ἐπιπέμπει κρίσεις ἀναμέσον ἀδελφῶν.

\vs{20}Υἱὲ, φύλασσε νόμους πατρός σου, καὶ μὴ ἀπώσῃ θεσμοὺς μητρός σου·
\vs{21}Ἄφαψαι δὲ αὐτοὺς ἐπὶ σῇ ψυχῇ διαπαντὸς, καὶ ἐλκλοίωσαι περὶ σῷ τραχήλῳ·
\vs{22}Ἡνίκα ἂν περιπατῇς, ἐπάγου αὐτὴν καὶ μετὰ σοῦ ἔστω, ὡς δʼ ἂν καθεύδῃς φυλασσέτω σε, ἵνα ἐγειρομένῳ συλλαλῇ σοι.
\vs{23}Ὅτι λύχνος ἐντολὴ νόμου καὶ φῶς, ὁδὸς ζωῆς, καὶ ἔλεγχος καὶ παιδεία,
\vs{24}τοῦ διαφυλάσσειν σε ἀπὸ γυναικὸς ὑπάνδρου, καὶ ἀπὸ διαβολῆς γλώσσης ἀλλοτρίας.

\vs{25}Μή σε νικήσῃ κάλλους ἐπιθυμία, μηδὲ ἀγρευθῇς σοῖς ὀφθαλμοῖς, μηδὲ συναρπασθῇς ἀπὸ τῶν αὐτῆς βλεφάρων.
\vs{26}Τιμὴ γὰρ πόρνης ὅση καὶ ἑνὸς ἄρτου, γυνὴ δὲ ἀνδρῶν τιμίας ψυχὰς ἀγρεύει.
\vs{27}Ἀποδήσει τις πῦρ ἐν κόλπῳ, τὰ δὲ ἱμάτια οὐ κατακαύσει;
\vs{28}ἢ περιπατήσει τις ἐπʼ ἀνθράκων πυρὸς, τοὺς δὲ πόδας οὐ κατακαύσει;
\vs{29}Οὕτως ὁ εἰσελθὼν πρὸς γυναῖκα ὕπανδρον, οὐκ ἀθωωθήσεται, οὐδὲ πᾶς ὁ ἁπτόμενος αὐτῆς.
\vs{30}Οὐ θαυμαστὸν ἐὰν ἁλῷ τις κλέπτων, κλέπτει γὰρ ἵνα ἐμπλήσῃ τὴν ψυχὴν πεινῶν.
\vs{31}Ἐὰν δὲ ἁλῷ, ἀποτίσει ἑπταπλάσια, καὶ πάντα τὰ ὑπάρχοντα αὐτοῦ δοὺς ῥύσεται ἑσυτόν.
\vs{32}Ὁ δὲ μοιχὸς διʼ ἔνδειαν φρενῶν ἀπώλειαν τῇ ψυχῇ αὐτοῦ περιποιεῖται,
\vs{33}ὀδύνας τε καὶ ἀτιμίας ὑποφέρει, τὸ δὲ ὄνειδος αὐτοῦ οὐκ ἐξαλειφθήσεται εἰς τὸν αἰῶνα.
\vs{34}Μεστὸς γὰρ ζήλου θυμὸς ἀνδρὸς αὐτῆς, οὐ φείσεται ἐν ἡμέρᾳ κρίσεως.
\vs{35}Οὐκ ἀνταλλάξεται οὐδενὸς λύτρου τὴν ἔχθραν, οὐδὲ μὴ διαλυθῇ πολλῶν δώρων.

\ch{7}
Υἱὲ φύλασσε ἐμοὺς λόγους, τὰς δὲ ἐμὰς ἐντολὰς κρύψον παρὰ σεαυτῷ·
\vs{1a}Υἱὲ τίμα τὸν Κύριον καὶ ἰσχύσεις, πλὴν δὲ αὐτοῦ μὴ φοβοῦ ἄλλον·
\vs{2}φύλαξον ἐμὰς ἐντολὰς καὶ βιώσεις, τοὺς δὲ ἐμοὺς λόγους ὥσπερ κόρας ὀμμάτων.
\vs{3}Περίθου δὲ αὐτοὺς σοῖς δακτύλοις, ἐπίγραψον δὲ ἐπὶ τὸ πλάτος τῆς καρδίας σου.

\vs{4}Εἰπὸν τὴν σοφίαν σὴν ἀδελφὴν εἶναι, τὴν δὲ φρόνησιν γνώριμον περιποίησαι σεαυτῷ.
\vs{5}Ἵνα σε τηρήσῃ ἀπὸ γυναικὸς ἀλλοτρίας καὶ πονηρᾶς, ἐάν σε λόγοις τοῖς πρὸς χάριν ἐμβάληται.

\vs{6}Ἀπὸ γὰρ θυρίδος ἐκ τοῦ οἴκου αὐτῆς εἰς τὰς πλατείας παρακύπτουσα,
\vs{7}ὃν ἂν ἴδῃ τῶν ἀφρόνων τέκνων νεανίαν ἐνδεῆ φρενῶν,
\vs{8}παραπορευόμενον παρὰ γωνίαν ἐν διόδοις οἴκων αὐτῆς, καὶ λαλοῦντα
\vs{9}ἐν σκότει ἑσπερινῷ, ἡνίκα ἂν ἡσυχία νυκτερινὴ καὶ γνοφώδης,
\vs{10}ἡ δὲ γυνὴ συναντᾷ αὐτῷ, εἶδος ἔχουσα πορνικὸν, ἣ ποιεῖ νέων ἐξίπτασθαι καρδίας.
\vs{11}Ἀνεπτερωμένη δέ ἐστι καὶ ἄσωτος, ἐν οἴκῳ δὲ οὐχ ἡσυχάζουσιν οἱ πόδες αὐτῆς.
\vs{12}Χρόνον γάρ τινα ἔξω ῥέμβεται, χρόνον δὲ ἐν πλατείαις παρὰ πᾶσαν γωνίαν ἐνεδρεύει.
\vs{13}Εἶτα ἐπιλαβομένη ἐφίλησεν αὐτὸν, ἀναιδεῖ δὲ προσώπῳ προσεῖπεν αὐτῷ,
\vs{14}θυσία εἰρηνική μοι ἐστὶ, σήμερον ἀποδίδωμι τὰς εὐχάς μου.
\vs{15}Ἕνεκα τούτου ἐξῆλθον εἰς συνάντησίν σοι, ποθοῦσα τὸ σὸν πρόσωπον, εὕρηκά σε.
\vs{16}κειρίαις τέτακα τὴν κλίνην μου, ἀμφιτάποις δὲ ἔστρωκα τοῖς ἀπʼ Αἰγύπτου.
\vs{17}Διέῤῥαγκα τὴν κοίτην μου κροκίνῳ, τὸν δὲ οἶκόν μου κινναμώμῳ·
\vs{18}Ἐλθὲ καὶ ἀπολαύσωμεν φιλίας ἕως ὄρθρου, δεῦρο καὶ ἐλκυλισθῶμεν ἔρωτι.
\vs{19}Οὐ γὰρ πάρεστιν ὁ ἀνήρ μου ἐν οἴκω, πεπόρευται δὲ ὁδὸν μακράν,
\vs{20}ἔνδεσμον ἀργυρίου λαβὼν ἐν χειρὶ αὐτοῦ, διʼ ἡμερῶν πολλῶν ἐπανήξει εἰς τὸν οἶκον αὐτοῦ.

\vs{21}Ἀπεπλάνησε δὲ αὐτὸν πολλῇ ὁμιλίᾳ, βρόχοις τε τοῖς ἀπὸ χειλέων ἐξώκειλεν αὐτόν.
\vs{22}Ὁ δὲ ἐπηκολούθησεν αὐτῇ κεπφωθείς· ὥσπερ δὲ βοῦς ἐπὶ σφαγὴν ἄγεται, καὶ ὥσπερ κύων ἐπὶ δεσμοὺς,
\vs{23}ἢ ὡς ἔλαφος τοξεύματι πεπληγὼς εἰς τὸ ἧπαρ· σπεύδει δὲ ὥσπερ ὄρνεον εἰς παγίδα, οὐκ εἰδὼς ὅτι περὶ ψυχῆς τρέχει.

\vs{24}Νῦν οὖν υἱὲ ἄκουέ μου, καὶ πρόσεχε ῥήμασι στόματός μου.
\vs{25}Μὴ ἐκκλινάτω εἰς τὰς ὁδοὺς αὐτῆς ἡ καρδία σου,
\vs{26}πολλοὺς γὰρ τρώσασα καταβέβληκε, καὶ ἀναρίθμητοί εἰσιν οὓς πεφόνευκεν.
\vs{27}Ὁδοὶ ᾅδου ὁ οἶκος αὐτῆς, κατάγουσαι εἰς τὰ ταμιεῖα τοῦ θανάτου.

\ch{8}
Σὺ τὴν σοφίαν κηρύξεις, ἵνα φρόνησίς σοι ὑπακούσῃ.
\vs{2}Ἐπὶ γὰρ τῶν ὑψηλῶν ἄκρων ἐστὶν, ἀναμέσον δὲ τῶν τρίβων ἕστηκε.
\vs{3}Παρὰ γὰρ πύλαις δυναστῶν παρεδρεύει, ἐν δὲ εἰσόδοις ὑμνεῖται.
\vs{4}Ὑμᾶς ὦ ἄνθρωποι παρακαλῶ, καὶ προΐεμαι ἐμὴν φωνὴν υἱοῖς ἀνθρώπων.
\vs{5}Νοήσατε ἄκακοι πανουργίαν, οἱ δὲ ἀπαίδευτοι ἔνθεσθε καρδίαν.
\vs{6}Εἰσακούσατέ μου, σεμνὰ γὰρ ἐρῶ, καὶ ἀνοίσω ἀπὸ χειλέων ὀρθά.
\vs{7}Ὅτι ἀλήθειαν μελετήσει ὁ φάρυγξ μου, ἐβδελυγμένα δὲ ἐναντίον ἐμοῦ χείλη ψευδῆ.
\vs{8}Μετὰ δικαιοσύνης πάντα τὰ ῥήματα τοῦ στόματός μου, οὐδὲν ἐαυτοῖς σκολιὸν οὐδὲ στραγγαλιῶδες.
\vs{9}Πάντα ἐνώπια τοῖς συνιοῦσι, καὶ ὀρθὰ τοῖς εὑρίσκουσι γνῶσιν.
\vs{10}Λάβετε παιδείαν καὶ μὴ ἀργύριον, καὶ γνῶσιν ὑπὲρ χρυσίον δεδοκιμασμένον·
\vs{11}Κρείσσων γὰρ σοφία λίθων πολυτελῶν, πᾶν δὲ τίμιον οὐκ ἄξιον αὐτῆς ἐστιν.

\vs{12}Ἐγὼ ἡ σοφία κατεσκήνωσα βουλὴν καὶ γνῶσιν, καὶ ἔννοιαν ἐγὼ ἐπεκαλεσάμην.
\vs{13}Φόβος Κυρίου μισεῖ ἀδικίαν, ὕβριν τε καὶ ὑπερηφανίαν καὶ ὁδοὺς πονηρῶν· μεμίσηκα δὲ ἐγὼ διεστραμμένας ὁδοὺς κακῶν.
\vs{14}Ἐμὴ βουλὴ καὶ ἀσφάλεια, ἐμὴ φρόνησις, ἐμὴ δὲ ἰσχύς.
\vs{15}Διʼ ἐμοῦ βασιλεῖς βασιλεύουσι, καὶ οἱ δυνάσται γράφουσιν δικαιοσύνην.
\vs{16}Διʼ ἐμοῦ μεγιστᾶνες μεγαλύνονται, καὶ τύραννοι διʼ ἐμοῦ κρατοῦσι γῆς.
\vs{17}Ἐγὼ τοὺς ἐμὲ φιλοῦντας ἀγαπῶ, οἱ δὲ ἐμὲ ζητοῦντες εὑρήσουσιν.

\vs{18}Πλοῦτος καὶ δόξα ἐμοὶ ὑπάρχει, καὶ κτῆσις πολλῶν καὶ δικαιοσύνη.
\vs{19}Βέλτιον ἐμὲ καρπίζεσθαι ὑπὲρ χρυσίον καὶ λίθον τίμιον, τὰ δὲ ἐμὰ γεννήματα κρείσσω ἀργυρίου ἐκλεκτοῦ.
\vs{20}Ἐν ὁδοῖς δικαιοσύνης περιπατῶ, καὶ ἀναμέσον τρίβων δικαιώματος ἀναστρέφομαι·
\vs{21}ἵνα μερίσω τοῖς ἐμὲ ἀγαπῶσιν ὕπαρξιν, καὶ τοὺς θησαυροὺς αὐτῶν ἐμπλήσω ἀγαθῶν·
\vs{21a}ἐὰν ἀναγγείλω ὑμῖν τὰ καθʼ ἡμέραν γινόμενα, μνημονεύσω τὰ ἐξ αἰῶνος ἀριθμῆσαι.

\vs{22}Κύριος ἔκτισέ με ἀρχὴν ὁδῶν αὐτοῦ εἰς ἔργα αὐτοῦ,
\vs{23}πρὸ τοῦ αἰῶνος ἐθεμελίωσέ με, ἐν ἀρχῇ πρὸ τοῦ τὴν γῆν ποιῆσαι,
\vs{24}καὶ πρὸ τοῦ τὰς ἀβύσσους ποιῆσαι, πρὸ τοῦ προελθεῖν τὰς πηγὰς τῶν ὑδάτων·
\vs{25}Πρὸ τοῦ ὄρη ἑδρασθῆναι, πρὸ δὲ πάντων βουνῶν, γεννᾷ με.
\vs{26}Κύριος ἐποίησε χώρας καὶ ἀοικήτους, καὶ ἄκρα οἰκούμενα τῆς ὑπʼ οὐρανῶν.
\vs{27}Ἡνίκα ἡτοίμαζε τὸν οὐρανὸν, συμπαρήμην αὐτῷ, καὶ ὅτε ἀφώριζε τὸν ἑαυτοῦ θρόνον ἐπʼ ἀνέμων,
\vs{28}καὶ ὡς ἰσχυρὰ ἐποίει τὰ ἄνω νέφη, καὶ ὡς ἀσφαλεῖς ἐτίθει πηγὰς τῆς ὑπʼ οὐρανὸν,
\vs{29}καὶ ὡς ἰσχυρὰ ἐποίει τὰ θεμέλια τῆς γῆς,
\vs{30}ἤμην παρʼ αὐτῷ ἁρμόζουσα· ἐγὼ ἤμην ᾗ προσέχαιρε· καθʼ ἡμέραν δὲ εὐφραινόμην ἐν προσώπῳ αὐτοῦ ἐν παντὶ καιρῷ,
\vs{31}ὅτε ἐνευφραίνετο τὴν οἰκουμένην συντελέσας, καὶ ἐνευφραίνετο ἐν υἱοῖς ἀνθρώπων.

\vs{32}Νῦν οὖν υἱὲ ἄκουέ μου,
\vs{34}μακάριος ἀνὴρ ὃς εἰσακούσεταί μου, καὶ ἄνθρωπος ὃς τὰς ἐμὰς ὁδοὺς φυλάξει, ἀγρυπνῶν ἐπʼ ἐμαῖς θύραις καθʼ ἡμέραν, τηρῶν σταθμοὺς ἐμῶν εἰσόδων.
\vs{35}Αἱ γὰρ ἔξοδοί μου, ἔξοδοι ζωῆς, καὶ ἑτοιμάζεται θέλησις παρὰ Κυρίου.
\vs{36}Οἱ δὲ ἁμαρτάνοντες εἰς ἐμὲ, ἀσεβοῦσιν εἰς τὰς ἑαυτῶν ψυχὰς, καὶ οἱ μισοῦντές με ἀγαπῶσι θάνατον.

\ch{9}
Ἡ σοφία ᾠκοδόμησεν ἑαυτῇ οἶκον, καὶ ὑπήρεισε στύλους ἑπτά.
\vs{2}Ἔσφαξε τὰ ἑαυτῆς θύματα, ἐκέρασεν εἰς κρατῆρα τὸν ἑαυτῆς οἶνον, καὶ ἡτοιμάσατο τὴν ἑαυτῆς τράπεζαν.
\vs{3}Ἀπέστειλε τοὺς ἑαυτῆς δούλους, συγκαλοῦσα μετὰ ὑψηλοῦ κηρύγματος ἐπὶ κρατῆρα, λέγουσα,
\vs{4}Ὅς ἐστιν ἄφρων, ἐκκλινάτω πρὸς μέ· καὶ τοῖς ἐνδεέσι φρενῶν εἶπεν,
\vs{5}ἔλθατε, φάγετε τῶν ἐμῶν ἄρτων, καὶ πίετε οἶνον ὃν ἐκέρασα ὑμῖν.

\vs{6}Ἀπολείπετε ἀφροσύνην, ἵνα εἰς τὸν αἰῶνα βασιλεύσητε· καὶ ζητήσατε φρόνησιν, καὶ κατορθώσατε ἐν γνώσει σύνεσιν.
\vs{7}Ὁ παιδεύων κακοὺς λήψεται ἑαυτῷ ἀτιμίαν· ἐλέγχων δὲ τὸν ἀσεβῆ μωμήσεται ἑαυτόν.
\vs{8}Μὴ ἔλεγχε κακοὺς, ἵνα μὴ μισήσωσί σε· ἔλεγχε σοφὸν, καὶ ἀγαπήσει σε.
\vs{9}Δίδου σοφῷ ἀφορμὴν, καὶ σοφώτερος ἔσται· γνώριζε δικαίῳ, καὶ προσθήσει τοῦ δέχεσθαι.
\vs{10}Ἀρχὴ σοφίας φόβος Κυρίου, καὶ βουλὴ ἁγίων σύνεσις·
\vs{10a}τὸ γὰρ γνῶναι νόμον, διανοίας ἐστὶν ἀγαθῆς.
\vs{11}Τούτῳ γὰρ τῷ τρόπῳ πολὺν ζήσεις χρόνον, καὶ προστεθήσεταί σοι ἔτη ζωῆς σου.

\vs{12}Υἱὲ ἐὰν σοφὸς γένῃ σεαυτῷ, σοφὸς ἔσῃ καὶ τοῖς πλησίον· ἐὰν δὲ κακὸς ἀποβῇς, μόνος ἂν ἀντλήσεις κακά·
\vs{12a}ὃς ἐρείδεται ἐπὶ ψεύδεσιν, οὗτος ποιμαίνει ἀνέμους, ὁ δʼ αὐτὸς διώξεται ὄρνεα πετόμενα·
\vs{12b}ἀπέλιπε γὰρ ὁδοὺς τοῦ ἑαυτοῦ ἀμπελῶνος, τοὺς δὲ ἄξονας τοῦ ἰδίου γεωργίου πεπλάνηται·
\vs{12c}διαπορεύεται δὲ διʼ ἀνύδρου ἐρήμου, καὶ γῆν διατεταγμένην ἐν διψώδεσι, συνάγει δὲ χερσὶν ἀκαρπίαν.

\vs{13}Γυνὴ ἄφρων καὶ θρασεῖα ἐνδεὴς ψωμοῦ γίνεται, ἣ οὐκ ἐπίσταται αἰσχύνην.
\vs{14}Ἐκάθισεν ἐπὶ θύραις τοῦ ἑαυτῆς οἴκου, ἐπὶ δίφρου ἐμφανῶς ἐν πλατείαις,
\vs{15}προσκαλουμένη τοὺς παριόντας καὶ κατευθύνοντας ἐν ταῖς ὁδοῖς αὐτῶν·
\vs{16}Ὅς ἐστιν ὑμῶν ἀφρονέστατος, ἐκκλινάτω πρὸς μέ· καὶ τοῖς ἐνδεέσι φρονήσεως παρακελεύομαι, λέγουσα,
\vs{17}ἄρτων κρυφίων ἡδέως ἅψασθε, καὶ ὕδατος κλοπῆς γλυκεροῦ.

\vs{18}Ὁ δὲ οὐκ οἶδεν ὅτι γηγενεῖς παρʼ αὐτῇ ὄλλυνται, καὶ ἐπὶ πέταυρον ᾅδου συναντᾷ·
\vs{18a}ἀλλὰ ἀποπήδησον, μὴ χρονίσῃς ἐν τῷ τόπῳ, μηδὲ ἐπιστήσῃς τὸ σὸν ὄμμα πρὸς αὐτὴν,
\vs{18b}οὕτως γὰρ διαβήσῃ ὕδωρ ἀλλότριον·
\vs{18c}ἀπὸ δὲ ὕδατος ἀλλοτρίου ἀπόσχου, καὶ ἀπὸ πηγῆς ἀλλοτρίας μὴ πίῃς
\vs{18d}ἵνα πολὺν ζησῃς χρόνον, προστεθῇ δέ σοι ἔτη ζωῆς.

\ch{10}
Υἱὸς σοφὸς εὐφραίνει πατέρα, υἱὸς δὲ ἄφρων λύπη τῇ μητρί.
\vs{2}Οὐκ ὠφελήσουσι θησαυροὶ ἀνόμους, δικαιοσύνη δὲ ῥύσεται ἐκ θανάτου.
\vs{3}Οὐ λιμοκτονήσει Κύριος ψυχὴν δικαίαν, ζωὴν δὲ ἀσεβῶν ἀνατρέψει.

\vs{4}Πενία ἄνδρα ταπεινοῖ, χεῖρες δὲ ἀνδρείων πλουτίζουσιν·
\vs{4a}υἱὸς πεπαιδευμένος σοφὸς ἔσται, τῷ δὲ ἄφρονι διακόνῳ χρήσεται.
\vs{5}Διεσώθη ἀπὸ καύματος υἱὸς νοήμων, ἀνεμόφθορος δὲ γίνεται ἐν ἀμητῷ υἱὸς παράνομος.

\vs{6}Εὐλογία Κυρίου ἐπὶ κεφαλὴν δικαίου, στόμα δὲ ἀσεβῶν καλύψει πένθος ἄωρον.
\vs{7}Μνήμη δικαίων μετʼ ἐγκωμίων, ὄνομα δὲ ἀσεβοῦς σβέννυται.
\vs{8}Σοφὸς καρδίᾳ δέξεται ἐντολὰς, ὁ δὲ ἄστεγος χείλες σκολιάζων ὑποσκελισθήσεται.
\vs{9}Ὃς πορεύεται ἁπλῶς, πορεύεται πεποιθώς· ὁ δὲ διαστρέφων τὰς ὁδοὺς αὐτοῦ, γνωσθήσεται.
\vs{10}Ὁ ἐννεύων ὀφθαλμοῖς μετὰ δόλου, συνάγει ἀνδράσι λύπας· ὁ δὲ ἐλέγχων μετὰ παῤῥησίας, εἰρηνοποιεῖ.
\vs{11}Πηγὴ ζωῆς ἐν χειρὶ δικαίου, στόμα δὲ ἀσεβοῦς καλύψει ἀπώλεια.

\vs{12}Μῖσος ἐγείρει νεῖκος, πάντας δὲ τοὺς μὴ φιλονεικοῦντας καλύπτει φιλία.
\vs{13}Ὃ ἐκ χειλέων προφέρει σοφίαν, ῥάβδῳ τύπτει ἄνδρα ἀκάρδιον.
\vs{14}Σοφοὶ κρύψουσιν αἴσθησιν, στόμα δὲ προπετοῦς ἐγγίζει συντριβῇ.
\vs{15}Κτῆσις πλουσίων πόλις ὀχυρὰ, συντριβὴ δὲ ἀσεβῶν πενία.
\vs{16}Ἔργα δικαίων ζωὴν ποιεῖ, καρποὶ δὲ ἀσεβῶν ἁμαρτίας.
\vs{17}Ὁδοὺς δικαίας ζωῆς φυλάσσει παιδεία, παιδεία δὲ ἀνεξέλεγκτος πλανᾶται.

\vs{18}Καλύπτουσιν ἔχθραν χείλη δίκαια, οἱ δὲ ἐκφέροντες λοιδορίας ἀφρονέστατοί εἰσιν.
\vs{19}Ἐκ πολυλογίας οὐκ ἐκφεύξῃ ἁμαρτίαν, φειδόμενος δὲ χειλέων νοήμων ἔσῃ.
\vs{20}Ἄργυρος πεπυρωμένος γλῶσσα δικαίου, καρδία δὲ ἀσεβοῦς ἐκλείψει.
\vs{21}Χείλη δικαίων ἐπίσταται ὑψηλὰ, οἱ δὲ ἄφρονες ἐν ἐνδείᾳ τελευτῶσιν.
\vs{22}Εὐλογία Κυρίου ἐπὶ κεφαλὴν δικαίου, αὕτη πλουτίζει, καὶ οὐ μὴ προστεθῇ αὐτῇ λύπη ἐν καρδίᾳ.

\vs{23}Ἐν γέλωτι ἄφρων πράσσει κακὰ, ἡ δὲ σοφία ἀνδρὶ τίκτει φρόνησιν.

\vs{24}Ἐν ἀπωλείᾳ ἀσεβὴς περιφέρεται, ἐπιθυμία δὲ δικαίου δεκτή.
\vs{25}Παραπορευομένης καταιγίδος ἀφανίζεται ἀσεβὴς, δίκαιος δὲ ἐκκλίνας σώζεται εἰς τὸν αἰῶνα.
\vs{26}Ὥσπερ ὄμφαξ ὀδοῦσι βλαβερὸν, καὶ καπνὸς ὄμμασιν, οὕτως παρανομία τοῖς χρωμένοις αὐτῇ.
\vs{27}Φόβος Κυρίου προστίθησιν ἡμέρας, ἔτη δὲ ἀσεβῶν ὀλιγωθήσεται.
\vs{28}Ἐγχρονίζει δικαίοις εὐφροσύνη, ἐλπὶς δὲ ἀσεβῶν ἀπολεῖται.
\vs{29}Ὀχύρωμα ὁσίου φόβος Κυρίου, συντριβὴ δὲ τοῖς ἐργαζομένοις κακά.

\vs{30}Δίκαιος εἰς τὸν αἰῶνα οὐκ ἐνδώσει, ἀσεβεῖς δὲ οὐκ οἰκήσουσι γῆν.
\vs{31}Στόμα δικαίου ἀποστάζει σοφίαν, γλῶσσα δὲ ἀδίκου ἐξολεῖται.
\vs{32}Χείλη ἀνδρῶν δικαίων ἀποστάζει χάριτας, στόμα δὲ ἀσεβῶν ἀποστρέφεται.

\ch{11}
Ζυγοί δόλιοι βδέλυγμα ἐνώπιον Κυρίου, στάθμιον δὲ δίκαιον δεκτὸν αὐτῷ.
\vs{2}Οὗ ἐὰν εἰσέλθῃ ὕβρις, ἐκεῖ καὶ ἀτιμία· στόμα δὲ ταπεινῶν μελετᾷ σοφίαν.
\vs{3}Ἀποθανὼν δίκαιος ἔλιπε μετάμελον, πρόχειρος δὲ γίνεται καὶ ἐπίχαρτος ἀσεβῶν ἀπώλεια.
\vs{5}Δικαιοσύνη ἀμώμους ὀρθοτομεῖ ὁδοὺς, ἀσέβεια δὲ περιπίπτει ἀδικίᾳ.

\vs{6}Δικαιοσύνη ἀνδρῶν ὀρθῶν ῥύεται αὐτούς, τῇ δὲ ἀπωλείᾳ αὐτῶν ἁλίσκονται παράνομοι.
\vs{7}Τελευτήσαντος ἀνδρὸς δικαίου, οὐκ ὄλλυται ἐλπίς, τὸ δὲ καύχημα τῶν ἀσεβῶν ὄλλυται.
\vs{8}Δίκαιος ἐκ θήρας ἐκδύνει, ἀντʼ αὐτοῦ δὲ παραδίδοται ὁ ἀσεβής.
\vs{9}Ἐν στόματι ἀσεβῶν παγὶς πολίταις, αἴσθησις δὲ δικαίων εὔοδος.
\vs{10}Ἐν ἀγαθοῖς δικαίων κατώρθωσε πόλις,
\vs{11}στόμασι δὲ ἀσεβῶν κατεσκάφη.

\vs{12}Μυκτηρίζει πολίτας ἐνδεὴς φρενῶν, ἀνὴρ δὲ φρόνιμος ἡσυχίαν ἄγει.
\vs{13}Ἀνὴρ δίγλωσσος ἀποκαλύπτει βουλὰς ἐν συνεδρίῳ, πιστὸς δὲ πνοῇ κρύπτει πράγματα.
\vs{14}Οἷς μὴ ὑπάρχει κυβέρνησις, πίπτουσιν ὥσπερ φύλλα, σωτηρία δὲ ὑπάρχει ἐν πολλῇ βουλῇ.

\vs{15}Πονηρὸς κακοποιεῖ ὅταν συνμίξῃ δικαίῳ, μισεῖ δὲ ἦχον ἀσφαλείας.
\vs{16}Γυνὴ εὐχάριστος ἐγείρει ἀνδρὶ δόξαν, θρόνος δὲ ἀτιμίας γυνὴ μισοῦσα δίκαια· πλούτου ὀκνηροὶ ἐνδεεῖς γίνονται, οἱ δὲ ἀνδρεῖοι ἐρείδονται πλούτῳ.
\vs{17}Τῇ ψυχῇ αὐτοῦ ἀγαθὸν ποιεῖ ἀνὴρ ἐλεήμων, ἐξολλύει δὲ αὐτοῦ σῶμα ὁ ἀνελεήμων.

\vs{18}Ἀσεβὴς ποιεῖ ἔργα ἄδικα, σπέρμα δὲ δικαίων μισθὸς ἀληθείας.
\vs{19}Υἱὸς δίκαιος γεννᾶται εἰς ζωὴν, διωγμὸς δὲ ἀσεβοῦς εἰς θάνατον.
\vs{20}Βδέλυγμα Κυρίῳ διεστραμμέναι ὁδοὶ, προσδεκτοὶ δὲ αὐτῷ πάντες ἄμωμοι ἐν ταῖς ὁδοῖς αὐτῶν.
\vs{21}Χειρὶ χεῖρας ἐμβαλὼν ἀδίκως οὐκ ἀτιμώρητος ἔσται, ὁ δὲ σπείρων δικαιοσύνην λήψεται μισθὸν πιστόν.
\vs{22}Ὥσπερ ἐνώτιον ἐν ῥινὶ ὑός, οὕτως γυναικὶ κακόφρονι κάλλος.
\vs{23}Ἐπιθυμία δικαίων πᾶσα ἀγαθὴ, ἐλπὶς δὲ ἀσεβῶν ἀπολεῖται.

\vs{24}Εἰσὶν, οἳ τὰ ἴδια σπείροντες πλείονα ποιοῦσιν· εἰσὶδέ καὶ οἳ συνάγοντες ἐλαττονοῦνται.
\vs{25}Ψυχὴ εὐλογουμένη πᾶσα ἁπλῇ, ἀνὴρ δὲ θυμώδης οὐκ εὐσχήμων.
\vs{26}Ὁ συνέχων σῖτον ὑπολείποιτο αὐτὸν τοῖς ἔθνεσιν· εὐλογία δὲ εἰς κεφαλὴν τοῦ μεταδιδόντος.
\vs{27}Τεκταινόμενος ἀγαθὰ ζητεῖ χάριν ἀγαθὴν, ἐκζητοῦντα δὲ κακὰ καταλήψεται αὐτόν.
\vs{28}Ὁ πεποιθὼς ἐπὶ πλούτῳ οὗτος πεσεῖται, ὁ δὲ ἀντιλαμβανόμενος δικαίων οὗτος ἀνατελεῖ.
\vs{29}Ὁ μὴ συμπεριφερόμενος τῷ ἑαυτοῦ οἴκῳ, κληρονομήσει ἄνεμον, δουλεύσει δὲ ἄφρων φρονίμῳ.
\vs{30}Ἐκ καρποῦ δικαιοσύνης φύεται δένδρὅν ζωῆς, ἀφαιροῦνται δὲ ἄωροι ψυχαὶ παρανόμων.
\vs{31}Εἰ ὁ μὲν δίκαιος μόλις σώζεται, ὁ ἀσεβὴς καὶ ἁμαρτωλὸς ποῦ φανεῖται;

\ch{12}
Ὁ ἀγαπῶν παιδείαν, ἀγαπᾷ αἴσθησιν· ὁ δὲ μισῶν ἐλέγχους, ἄφρων.
\vs{2}Κρείσσων ὁ εὑρὼν χάριν παρὰ Κυρίῳ, ἀνὴρ δὲ παράνομος παρασιωπηθήσεται.
\vs{3}Οὐ κατορθώσει ἄνθρωπος ἐξ ἀνόμου, αἱ δὲ ῥίζαι τῶν δικαίων οὐκ ἐξαρθήσονται.
\vs{4}Γυνὴ ἀνδρεία στέφανος τῷ ἀνδρὶ αὐτῆς· ὥσπερ δὲ ἐν ξύλῳ σκώληξ, οὕτως ἄνδρα ἀπόλλυσι γυνὴ κακοποιός.

\vs{5}Λογισμοὶ δικαίων κρίματα, κυβερνῶσι δὲ ἀσεβεῖς δόλους.

\vs{6}Λόγοι ἀσεβῶν δόλιοι, στόμα δὲ ὀρθῶν ῥύσεται αὐτούς.
\vs{7}Οὗ ἐὰν στραφῇ ὁ ἀσεβὴς, ἀφανίζεται, οἶκοι δὲ δικαίων παραμένουσι·
\vs{8}Στόμα συνετοῦ ἐγκωμιάζεται ὑπὸ ἀνδρὸς, νωθροκάρδιος δὲ μυκτηρίζεται.
\vs{9}Κρείσσων ἀνὴρ ἐν ἀτιμίᾳ δουλεύων ἑαυτῷ, ἢ τιμὴν ἑαυτῷ περιτιθεὶς καὶ προσδεόμενος ἄρτου.

\vs{10}Δίκαιος οἰκτείρει ψυχὰς κτηνῶν αὐτοῦ, τὰ δὲ σπλάγχνα τῶν ἀσεβῶν ἀνελεήμονα.
\vs{11}Ὁ ἐργαζόμενος τὴν ἑαυτοῦ γῆν, ἐμπλησθήσεται ἄρτων, οἱ δὲ διώκοντες μάταια, ἐνδεεῖς φρενῶν·
\vs{11a}ὅς ἐστιν ἡδὺς ἐν οἴνων διατριβαῖς, ἐν τοῖς ἑαυτοῦ ὀχυρώμασι καταλείψει ἀτιμίαν.

\vs{12}Ἐπιθυμίαι ἀσεβῶν κακαὶ, αἱ δὲ ῥίζαι τῶν εὐσεβῶν ἐν ὀχυρώμασι.
\vs{13}Διʼ ἁμαρτίαν χειλέων ἐμπίπτει εἰς παγίδας ἁμαρτωλὸς, ἐκφεύγει δὲ ἐξ αὐτῶν δίκαιος·
\vs{13a}ὁ βλέπων λεῖα ἐλεηθήσεται, ὁ δὲ συναντῶν ἐν πύλαις ἐκθλίψει ψυχάς.
\vs{14}Ἀπὸ καρπῶν στόματος ψυχὴ ἀνδρὸς πλησθήσεται ἀγαθῶν, ἀνταπόδομα δὲ χειλέων αὐτοῦ δοθήσεται αὐτῷ.
\vs{15}Ὁδοὶ ἀφρόνων ὀρθαὶ ἐνώπιον αὐτῶν, εἰσακούει δὲ συμβουλίας σοφός.
\vs{16}Ἄφρων αὐθημερὸν ἐξαγγέλλει ὀργὴν αὐτοῦ, κρύπτει δὲ τὴν ἑαυτοῦ ἀτιμίαν ἀνὴρ πανοῦργος.
\vs{17}Ἐπιδεικνυμένην πίστιν ἀπαγγέλλει δίκαιος, ὁ δὲ μάρτυς τῶν ἀδίκων δόλιος.

\vs{18}Εἰσὶν οἳ λέγοντες τιτρώσκουσι, μάχαιραι· γλῶσσαι δὲ σοφῶν ἰῶνται.
\vs{19}Χείλη ἀληθινὰ κατορθοῖ μαρτυρίαν, μάρτυς δὲ ταχὺς γλῶσσαν ἔχει ἄδικον.
\vs{20}Δόλος ἐν καρδίᾳ τεκταινομένου κακὰ, οἱ δὲ βουλόμενοι εἰρήνην εὐφρανθήσονται.
\vs{21}Οὐκ ἀρέσει τῷ δικαίῳ οὐδὲν ἄδικον, οἱ δὲ ἀσεβεῖς πλησθήσονται κακῶν.
\vs{22}Βδέλυγμα Κυρίῳ χείλη ψευδῆ, ὁ δὲ ποιῶν πίστεις δεκτὸς παρʼ αὐτῷ.
\vs{23}Ἀνὴρ συνετὸς θρόνος αἰσθήσεως, καρδία δὲ ἀφρόνων συναντήσεται ἀραῖς.

\vs{24}Χεὶρ ἐκλεκτῶν κρατήσει εὐχερῶς, δόλιοι δὲ ἔσονται ἐν προνομῇ.
\vs{25}Φοβερὸς λόγος καρδίαν ταράσσει ἀνδρὸς δικαίου, ἀγγελία δὲ ἀγαθὴ εὐφραίνει αὐτόν.
\vs{26}Ἐπιγνώμων δίκαιος ἑαυτοῦ φίλος ἔσται, ἁμαρτάνοντας δὲ καταδιώξεται κακὰ, ἡ δὲ ὁδὸς τῶν ἀσεβῶν πλανήσει αὐτούς.
\vs{27}Οὐκ ἐπιτεύξεται δόλιος θήρας, κτῆμα δὲ τίμιον ἀνὴρ καθαρός.
\vs{28}Ἐν ὁδοῖς δικαιοσύνης ζωὴ, ὁδοὶ δὲ μνησικάκων εἰς θάνατον.

\ch{13}
Υἱὸς πανοῦργος ὑπήκοος πατρὶ, υἱὸς δὲ ἀνήκοος ἐν ἀπωλείᾳ.
\vs{2}Ἀπὸ καρπῶν δικαιοσύνης φάγεται ἀγαθὸς, ψυχαὶ δὲ παρανόμων ὀλοῦνται ἄωροι.
\vs{3}Ὃς φυλάσσει τὸ ἑαυτοῦ στόμα τηρεῖ τὴν ἑαυτοῦ ψυχὴν, ὁ δὲ προπετὴς χείλεσι πτοήσει ἑαυτόν.
\vs{4}Ἐν ἐπιθυμίαις ἐστὶ πᾶς ἀεργὸς, χεῖρες δὲ ἀνδρείων ἐν ἐπιμελείᾳ.
\vs{5}Λόγον ἄδικον μισεῖ δίκαιος, ἀσεβὴς δὲ αἰσχύνεται, καὶ οὐκ ἕξει παῤῥησίαν.
\vs{7}Εἰσὶν οἱ πλουτίζοντες ἑαυτοὺς μηδὲν ἔχοντες, καὶ εἰσὶν οἱ ταπεινοῦντες ἑαυτοὺς ἐν πολλῷ πλούτῳ.

\vs{8}Λύτρον ἀνδρὸς ψυχῆς ὁ ἴδιος πλοῦτος, πτωχὸς δὲ οὐχ ὑφίσταται ἀπειλήν.
\vs{9}Φῶς δικαίοις διαπαντὸς, φῶς δὲ ἀσεβῶν σβέννυται·
\vs{9a}ψυχαὶ δόλιαι πλανῶνται ἐν ἁμαρτίαις, δίκαιοι δὲ οἰκτείρουσι καὶ ἐλεοῦσι.
\vs{10}Κακὸς μεθʼ ὕβρεως πράσσει κακὰ, οἱ δʼ ἑαυτῶν ἐπιγνώμονες σοφοί.
\vs{11}Ὕπαρξις ἐπισπουδαζομένη μετὰ ἀνομίας, ἐλάσσων γίνεται, ὁ δὲ συνάγων ἑαυτῷ μετʼ εὐσεβείας πληθυνθήσεται· δίκαιος οἰκτείρει καὶ κιχρᾷ.
\vs{12}Κρείσσων ἐναρχόμενος βοηθῶν καρδίᾳ, τοῦ ἐπαγγελλομένου καὶ εἰς ἐλπίδα ἄγοντος· δένδρον γὰρ ζωῆς, ἐπιθυμία ἀγαθή.
\vs{13}Ὃς καταφρονεῖ πράγματος, καταφρονηθήσεται ὑπʼ αὐτοῦ· ὁ δὲ φοβούμενος ἐντολὴν, οὗτος ὑγιαίνει·
\vs{13a}υἱῷ δολίῳ οὐδὲν ἔσται ἀγαθὸν, οἰκέτῃ δὲ σοφῷ εὔοδοι ἔσονται πράξεις, καὶ κατευθυνθήσεται ἡ ὁδὸς αὐτοῦ.

\vs{14}Νόμος σοφοῦ πηγὴ ζωῆς, ὁ δὲ ἄνους ὑπὸ παγίδος θανεῖται.
\vs{15}Σύνεσις ἀγαθὴ δίδωσι χάριν, τὸ δὲ γνῶναι νόμον διανοίας ἐστὶν ἀγαθῆς, ὁδοὶ δὲ καταφρονούντων ἐν ἀπωλείᾳ.

\vs{16}Πᾶς πανοῦργος πράσσει μετὰ γνώσεως, ὁ δὲ ἄφρων ἐξεπέτασεν ἑαυτοῦ κακίαν.
\vs{17}Βασιλεὺς θρασὺς ἐμπεσεῖται εἰς κακὰ, ἄγγελος δὲ σοφὸς ῥύσεται αὐτόν.
\vs{18}Πενίαν καὶ ἀτιμίαν ἀφαιρεῖται παιδεία, ὁ δὲ φυλάσσων ἐλέγχους δοξασθήσεται.
\vs{19}Ἐπιθυμίαι εὐσεβῶν ἡδύνουσι ψυχὴν, ἔργα δὲ ἀσεβῶν μακρὰν ἀπὸ γνώσεως.
\vs{20}Συμπορευόμενος σοφοῖς σοφὸς ἔσῃ, ὁ δὲ συμπορευόμενος ἄφροσι γνωσθήσεται.
\vs{21}Ἁμαρτάνοντας καταδιώξεται κακὰ, τοὺς δὲ δικαίους καταλήψεται ἀγαθά.
\vs{22}Ἀγαθὸς ἀνὴρ κληρονομήσει υἱοὺς υἱῶν, θησαυρίζεται δὲ δικαίοις πλοῦτος ἀσεβῶν.
\vs{23}Δίκαιοι ποιήσουσιν ἐν πλούτῳ ἔτη πολλὰ, ἄδικοι δὲ ἀπολοῦνται συντόμως.

\vs{24}Ὃς φείδεται τῆς βακτηρίας, μισεῖ τὸν υἱὸν αὐτοῦ ὁ δὲ ἀγαπῶν, ἐπιμελῶς παιδεύει.
\vs{25}Δίκαιος ἔσθων ἐμπιπλᾷ τὴν ψυχὴν αὐτοῦ, ψυχαὶ δὲ ἀσεβῶν ἐνδεεῖς.

\ch{14}
Σοφαὶ γυναῖκες ᾠκοδόμησαν οἴκους, ἡ δὲ ἄφρων κατέσκαψε ταῖς χερσὶν αὐτῆς.
\vs{2}Ὁ πορευόμενος ὀρθῶς φοβεῖται τὸν Κύριον, ὁ δὲ σκολιάζων ταῖς ὁδοῖς αὐτοῦ ἀτιμασθήσεται.
\vs{3}Ἐκ στόματος ἀφρόνων βακτηρία ὕβρεως, χείλη δὲ σοφῶν φυλάσσει αὐτούς.
\vs{4}Οὗ μή εἰσι βόες, φάτναι καθαραί· οὗ δὲ πολλὰ γεννήματα, φανερὰ βοὸς ἰσχύς.
\vs{5}Μάρτυς πιστὸς οὐ ψεύδεται, ἐκκαίει δὲ ψευδῆ μάρτυς ἄδικος.
\vs{6}Ζητήσεις σοφίαν παρὰ κακοῖς καὶ οὐχ εὑρήσεις, αἴσθησις δὲ παρὰ φρονίμοις εὐχερής.

\vs{7}Πάντα ἐναντία ἀνδρὶ ἄφρονι, ὅπλα δὲ αἰσθήσεως χείλη σοφά.
\vs{8}Σοφία πανούργων ἐπιγνώσεται τὰς ὁδοὺς αὐτῶν, ἄνοια δὲ ἀφρόνων ἐν πλάνῃ.
\vs{9}Οἰκίαι παρανόμων ὀφειλήσουσι καθαρισμὸν, οἰκίαι δὲ δικαίων δεκταί.

\vs{10}Καρδία ἀνδρὸς αἰσθητικὴ, λυπηρὰ ψυχὴ αὐτοῦ, ὅταν δὲ εὐφραίνηται οὐκ ἐπιμίγνυται ὕβρει.
\vs{11}Οἰκίαι ἀσεβῶν ἀφανισθήσονται, σκηναὶ δὲ κατορθούντων στήσονται.
\vs{12}Ἔστιν ὁδὸς ἣ δοκεῖ παρὰ ἀνθρώποις ὀρθὴ εἶναι, τὰ δὲ τελευταῖα αὐτῆς ἔρχεται εἰς πυθμένα ᾅδου.
\vs{13}Ἐν εὐφροσύναις οὐ προσμίγνυται λύπη, τελευταῖα δὲ χαρὰ εἰς πένθος ἔρχεται.
\vs{14}Τῶν ἑαυτοῦ ὁδῶν πλησθήσεται θρασυκάρδιος, ἀπὸ δὲ τῶν διανοημάτων αὐτοῦ ἀνὴρ ἀγαθός.
\vs{15}Ἄκακος πιστεύει παντὶ λόγῳ, πανοῦργος δὲ ἔρχεται εἰς μετάνοιαν.
\vs{16}Σοφὸς φοβηθεὶς ἐξέκλινεν ἀπὸ κακοῦ, ὁ δὲ ἄφρων ἑαυτῷ πεποιθὼς μίγνυται ἀνόμῳ.
\vs{17}Ὀξύθυμος πράσσει μετὰ ἀβουλίας, ἀνὴρ δὲ φρόνιμος πολλὰ ὑποφέρει.

\vs{18}Μεριοῦνται ἄφρονες κακίαν, οἱ δὲ πανοῦργοι κρατήσουσιν αἰσθήσεως.
\vs{19}Ὀλισθήσουσι κακοὶ ἔναντι ἀγαθῶν, καὶ ἀσεβεῖς θεραπεύσουσι θύρας δικαίων.
\vs{20}Φίλοι μισήσουσι φίλους πτωχούς, φίλοι δὲ πλουσίων πολλοί.
\vs{21}Ὁ ἀτιμάζων πένητας ἁμαρτάνει, ἐλεῶν δὲ πτωχοὺς μακαριστός.
\vs{22}Πλανώμενοι τεκταίνουσι κακά, ἔλεον δὲ καὶ ἀλήθειαν τεκταίνουσιν ἀγαθοί· οὐκ ἐπίστανται ἔλεον καὶ πίστιν τέκτονες κακῶν, ἐλεημοσύναι δὲ καὶ πίστεις παρὰ τέκτοσιν ἀγαθοῖς.
\vs{23}Ἐν παντὶ μεριμνῶντι ἔνεστι περισσόν, ὁ δὲ ἡδὺς καὶ ἀνάλγητος ἐν ἐνδείᾳ ἔσται.
\vs{24}Στέφανος σοφῶν πανοῦργος, ἡ δὲ διατριβὴ ἀφρόνων κακή.

\vs{25}Ῥύσεται ἐκ κακῶν ψυχὴν μάρτυς πιστὸς, ἐκκαίει δὲ ψευδῆ δόλιος.
\vs{26}Ἐν φόβῳ Κυρίου ἐλπὶς ἰσχύος, τοῖς δὲ τέκνοις αὐτοῦ καταλείπει ἔρεισμα.
\vs{27}Πρόσταγμα Κυρίου πηγὴ ζωῆς, ποιεῖ δὲ ἐκκλίνειν ἐκ παγίδος θανάτου.

\vs{28}Ἐν πολλῷ ἔθνει δόξα βασιλέως, ἐν δὲ ἐκλείψει λαοῦ συντριβὴ δυνάστου.
\vs{29}Μακρόθυμος ἀνὴρ πολὺς ἐν φρονήσει, ὁ δὲ ὀλιγόψυχος ἰσχυρῶς ἄφρων.
\vs{30}Πρᾳΰθυμος ἀνὴρ καρδίας ἰατρὸς, σὴς δὲ ὀστέων καρδία αἰσθητική·
\vs{31}Ὁ συκοφαντῶν πένητα παροξύνει τὸν ποιήσαντα αὐτὸν, ὁ δὲ τιμῶν αὐτὸν ἐλεεῖ πτωχόν.
\vs{32}Ἐν κακίᾳ αὐτοῦ ἀπωσθήσεται ἀσεβής, ὁ δὲ πεποιθὼς τῇ ἑαυτοῦ ὁσιότητι δίκαιος.

\vs{33}Ἐν καρδίᾳ ἀγαθῇ ἀνδρὸς σοφία, ἐν δὲ καρδίᾳ ἀφρόνων οὐ διαγινώσκεται.
\vs{34}Δικαιοσύνη ὑψοῖ ἔθνος, ἐλασσονοῦσι δὲ φυλὰς ἁμαρτίαι.
\vs{35}Δεκτὸς βασιλεῖ ὑπηρέτης νοήμων, τῇ δὲ ἑαυτοῦ εὐστροφίᾳ ἀφαιρεῖται ἀτιμίαν.

\ch{15}
Ὀργὴ ἀπόλλυσι καὶ φρονίμους, ἀπόκρισις δὲ ὑποπίπτουσα ἀποστρέφει θυμὸν, λόγος δὲ λυπηρὸς ἐγείρει ὀργάς.
\vs{2}Γλῶσσα σοφῶν καλὰ ἐπίσταται, στόμα δὲ ἀφρόνων ἀναγγέλλει κακά.

\vs{3}Ἐν παντὶ τόπῳ ὀφθαλμοὶ Κυρίου σκοπεύουσι κακούς τε καὶ ἀγαθούς.
\vs{4}Ἴασις γλώσσης δένδρον ζωῆς, ὁ δὲ συντηρῶν αὐτὴν πλησθήσεται πνεύματος.
\vs{5}Ἄφρων μυκτηρίζει παιδείαν πατρὸς, ὁ δὲ φυλάσσων ἐντολὰς, πανουργότερος· ἐν πλεοναζούσῃ δικαιοσύνῃ ἰσχὺς πολλὴ, οἱ δὲ ἀσεβεῖς ὁλόῤῥιζοι ἐκ γῆς ἀπολοῦνται.

\vs{6}Οἴκοις δικαίων ἰσχὺς πολλή, καρποὶ δὲ ἀσεβῶν ἀπολοῦνται.
\vs{7}Χείλη σοφῶν δέδεται αἰσθήσει, καρδίαι δὲ ἀφρόνων οὐκ ἀσφαλεῖς.
\vs{8}Θυσίαι ἀσεβῶν βδέλυγμα Κυρίῳ, εὐχαὶ δὲ κατευθυνόντων δεκταὶ παρʼ αὐτῷ.
\vs{9}Βδέλυγμα Κυρίῳ ὁδοὶ ἀσεβοῦς, διώκοντας δὲ δικαιοσύνην ἀγαπᾷ.
\vs{10}Παιδεία ἀκάκου γνωρίζεται ὑπὸ τῶν παριόντων, οἱ δὲ μισοῦντες ἐλέγχους τελευτῶσιν αἰσχρῶς.

\vs{11}Ἅδης καὶ ἀπώλεια φανερὰ παρὰ τῷ Κυρίῳ· πῶς οὐχὶ καὶ αἱ καρδίαι τῶν ἀνθρώπων;
\vs{12}Οὐκ ἀγαπήσει ἀπαίδευτος τοὺς ἐλέγχοντας αὐτόν, μετὰ δὲ σοφῶν οὐχ ὁμιλήσει.
\vs{13}Καρδίας εὐφραινομένης πρόσωπον θάλλει, ἐν δὲ λύπαις οὔσης σκυθρωπάζει.
\vs{14}Καρδία ὀρθὴ ζητεῖ αἴσθησιν, στόμα δὲ ἀπαιδεύτων γνώσεται κακά.

\vs{15}Πάντα τὸν χρόνον οἱ ὀφθαλμοὶ τῶν κακῶν προσδέχονται κακὰ, οἱ δὲ ἀγαθοὶ ἡσυχάζουσι διαπαντός.
\vs{16}Κρεῖσσον μικρὰ μερὶς μετὰ φόβου Κυρίου, ἢ θησαυροὶ μεγάλοι μετὰ ἀφοβίας.
\vs{17}Κρείσσων ξενισμὸς μετὰ λαχάνων πρὸς φιλίαν καὶ χάριν, ἢ παράθεσις μόσχων μετὰ ἔχθρας.
\vs{18}Ἀνὴρ θυμώδης παρασκευάζει μάχας· μακρόθυμος δὲ καὶ τὴν μέλλουσαν καταπρᾳΰνει·
\vs{18a}μακρόθυμος ἀνὴρ κατασβέσει κρίσεις, ὁ δὲ ἀσεβὴς ἐγείρει μᾶλλον.
\vs{19}Ὁδοὶ ἀεργῶν ἐστρωμέναι ἀκάνθαις, αἱ δὲ τῶν ἀνδρείων τετριμμέναι.
\vs{20}Υἱὸς σοφὸς εὐφραίνει πατέρα, υἱὸς δὲ ἄφρων μυκτηρίζει μητέρα αὐτοῦ.
\vs{21}Ἀνοήτου τρίβοι ἐνδεεῖς φρενῶν, ἀνὴρ δὲ φρόνιμος κατευθύνων πορεύεται.
\vs{22}Ὑπερτίθενται λογισμοὺς οἱ μὴ τιμῶντες συνέδρια, ἐν δὲ καρδίαις βουλευομένων μένει βουλή.

\vs{23}Οὐ μὴ ὑπακούσει ὁ κακὸς αὐτῇ, οὐδὲ μὴ εἴπῃ καίριόν τι καὶ καλὸν τῷ κοινῷ.

\vs{24}Ὁδοὶ ζωῆς διανοήματα συνετοῦ, ἵνα ἐκκλίνας ἐκ τοῦ ᾅδου σωθῇ.
\vs{25}Οἴκους ὑβριστῶν κατασπᾷ Κύριος, ἐστήρισε δὲ ὅριον χήρας.
\vs{26}Βδέλυγμα Κυρίῳ λογισμὸς ἄδικος, ἁγνῶν δὲ ῥήσεις σεμναί.
\vs{27}Ἐξόλλυσιν ἑαυτὸν ὁ δωρολήπτης, ὁ δὲ μισῶν δώρων λήψεις σώζεται·
\vs{27a}ἐλεημοσύναις καὶ πίστεσιν ἀποκαθαίρονται ἁμαρτίαι, τῷ δὲ φόβῳ Κυρίου ἐκκλίνει πᾶς ἀπὸ κακοῦ.

\vs{28}Καρδίαι δικαίων μελετῶσι πίστεις, στόμα δὲ ἀσεβῶν ἀποκρίνεται κακά·
\vs{28a}δεκταὶ παρὰ Κυρίῳ ὁδοὶ ἀνθρώπων δικαίων, διὰ δὲ αὐτῶν καὶ οἱ ἐχθροὶ φίλοι γίνονται.
\vs{29}Μακρὰν ἀπέχει ὁ Θεὸς ἀπὸ ἀσεβῶν, εὐχαῖς δὲ δικαίων ἐπακούει·
\vs{29a}κρείσσων ὀλίγη λῆψις μετὰ δικαιοσύνης, ἢ πολλὰ γεννήματα μετὰ ἀδικίας.

\vs{29b}Καρδία ἀνδρὸς λογιζέσθω δίκαια, ἵνα ὑπὸ τοῦ Θεοῦ διορθωθῇ τὰ διαβήματα αὐτοῦ.
\vs{30}Θεωρῶν ὀφθαλμὸς καλὰ εὐφραίνει καρδίαν, φημη δὲ ἀγαθὴ πιαίνει ὀστᾶ.
\vs{32}Ὃς ἀπωθεῖται παιδείαν, μισεῖ ἑαυτὸν· ὁ δὲ τηρῶν ἐλέγχους, ἀγαπᾷ ψυχὴν αὐτοῦ.
\vs{33}Φόβος Κυρίου παιδεία καὶ σοφία, καὶ ἀρχὴ δόξης ἀποκριθήσεται αὐτῇ.

\ch{16}\vs{2}Πάντα τὰ ἔργα τοῦ ταπεινοῦ φανερὰ παρὰ τῷ Θεῷ, οἱ δὲ ἀσεβεῖς ἐν ἡμέρᾳ κακῇ ὀλοῦνται.
\vs{5}Ἀκάθαρτος παρὰ Θεῷ πᾶς ὑψηλοκάρδιος, χειρὶ δὲ χεῖρας ἐμβαλὼν ἀδίκως οὐκ ἀθωωθήσεται·
\vs{7}ἀρχὴ ὁδοῦ ἀγαθῆς τὸ ποιεῖν τὰ δίκαια, δεκτὰ δὲ παρὰ Θεῷ μᾶλλον ἢ θύειν θυσίας·
\vs{8}ὁ ζητῶν τὸν Κύριον εὑρήσει γνῶσιν μετὰ δικαιοσύνης, οἱ δὲ ὀρθῶς ζητοῦντες αὐτὸν εὑρήσουσιν εἰρήνην.
\vs{9}Πάντα τὰ ἔργα τοῦ Κυρίου μετὰ δικαιοσύνης, φυλάσσεται δὲ ὁ ἀσεβὴς εἰς ἡμέραν κακήν.

\vs{10}Μαντεῖον ἐπὶ χείλεσι βασιλέως, ἐν δὲ κρίσει οὐ μὴ πλανηθῇ τὸ στόμα αὐτοῦ.
\vs{11}Ῥοπὴ ζυγοῦ δικαιοσύνη παρὰ Κυρίῳ, τὰ δὲ ἔργα αὐτοῦ στάθμια δίκαια.
\vs{12}Βδέλυγμα βασιλεῖ ὁ ποιῶν κακὰ, μετὰ γὰρ δικαιοσύνης ἑτοιμάζεται θρόνος ἀρχῆς.
\vs{13}Δεκτὰ βασιλεῖ χείλη δίκαια, λόγους δέ ὀρθοὺς ἀγαπᾷ.
\vs{14}Θυμὸ βασιλέως ἄγγελος θανάτου, ἀνὴρ δὲ σοφὸς ἐξιλάσεται αὐτόν.
\vs{15}Ἐν φωτὶ ζωῆς υἱὸς βασιλέως, οἱ δὲ προσδεκτοὶ αὐτῷ ὥσπερ νέφος ὄψιμον.
\vs{16}Νοσσιαὶ σοφίας αἱρετώτεραι χρυσίου, νοσσιαὶ δὲ φρονήσεως αἱρετώτεραι ὑπὲρ ἀργύριον.
\vs{17}Τρίβοι ζωῆς ἐκκλίνουσιν ἀπὸ κακῶν, μῆκος δὲ βίου ὁδοὶ δικαιοσύνης. Ὁ δεχόμενος παιδείαν ἐν ἀγαθοῖς ἔσται, ὁ δὲ φυλάσσων ἐλέγχους σοφισθήσεται· ὃς φυλάσσει τὰς ἑαυτοῦ ὁδοὺς, τηρεῖ τὴν ἑαυτοῦ ψυχήν· ἀγαπῶν δὲ ζωὴν αὐτοῦ, φείσεται στόματος αὐτοῦ.

\vs{18}Πρὸ συντριβῆς ἡγεῖται ὕβρις, πρὸ δὲ πτώματος κακοφροσύνη.
\vs{19}Κρείσσων πρᾳΰθυμος μετὰ ταπεινώσεως, ἢ ὃς διαιρεῖται σκῦλα μετὰ ὑβριστῶν.
\vs{20}Συνετὸς ἐν πράγμασιν εὑρετὴς ἀγαθῶν, πεποιθὼς δὲ ἐπὶ Θεῷ μακαριστός.
\vs{21}Τοὺς σοφοὺς καὶ συνετοὺς φαύλους καλοῦσιν, οἱ δὲ γλυκεῖς ἐν λόγῳ πλείονα ἀκούσονται.
\vs{22}Πηγὴ ζωῆς ἔννοια τοῖς κεκτημένοις, παιδεία δὲ ἀφρόνων κακή.
\vs{23}Καρδία σοφοῦ νοήσει τὰ ἀπὸ τοῦ ἰδίου στόματος, ἐπὶ δὲ χείλεσι φορέσει ἐπιγνωμοσύνην·
\vs{24}Κηρία μέλιτος λόγοι καλοί, γλύκασμα δὲ αὐτοῦ ἴασις ψυχῆς.

\vs{25}Εἰσὶν ὁδοὶ δοκοῦσαι εἶναι ὀρθαὶ ἀνδρὶ, τὰ μέντοι τελευταῖα αὐτῶν βλέπει εἰς πυθμένα ᾅδου.
\vs{26}Ἀνὴρ ἐν πόνοις πονεῖ ἑαυτῷ, καὶ ἐκβιάζεται τὴν ἀπώλειαν ἑαυτοῦ. Ὁ μέντοι σκολιὸς ἐπὶ τῷ ἑαυτοῦ στόματι φορεῖ τὴν ἀπώλειαν·
\vs{27}ἀνὴρ ἄφρων ὀρύσσει ἑαυτῷ κακὰ, ἐπὶ δὲ τῶν ἑαυτοῦ χειλέων θησαυρίζει πῦρ.
\vs{28}Ἀνὴρ σκολιὸς διαπέμπεται κακὰ, καὶ λαμπτῆρα δόλου πυρσεύσει κακοῖς, καὶ διαχωρίζει φίλους.
\vs{29}Ἀνὴρ παράνομος ἀποπειρᾶται φίλων, καὶ ἀπάγει αὐτοὺς ὁδοὺς οὐκ ἀγαθάς.

\vs{30}Στηρίζων δὲ ὀφθαλμοὺς αὐτοῦ διαλογίζεται διεστραμμένα, ὁρίζει δὲ τοῖς χείλεσιν αὐτοῦ πάντα τὰ κακά· οὗτος κάμινός ἐστι κακίας.
\vs{31}Στέφανος καυχήσεως γῆρας, ἐν δὲ ὁδοῖς δικαιοσύνης εὑρίσκεται.
\vs{32}Κρείσσων ἀνὴρ μακρόθυμος ἰσχυροῦ, ὁ δὲ κρατῶν ὀργῆς κρείσσων καταλαμβανομένου πόλιν.
\vs{33}Εἰς κόλπους ἐπέρχεται πάντα τοῖς ἀδίκοις, παρὰ δὲ Κυρίου πάντα τὰ δίκαια.

\ch{17}
Κρείσσων ψωμὸς μεθʼ ἡδονῆς ἐν εἰρήνῃ, ἢ οἶκος πολλῶν ἀγαθῶν καὶ ἀδίκων θυμάτων μετὰ μάχης.
\vs{2}Οἰκέτης νοήμων κρατήσει δεσποτῶν ἀφρόνων, ἐν δὲ ἀδελφοῖς διελεῖται μέρη.
\vs{3}Ὥσπερ δοκιμάζεται ἐν καμίνῳ ἄργυρος καὶ χρυσὸς, οὕτως ἐκλεκταὶ καρδίαι παρὰ Κυρίῳ.
\vs{4}Κακὸς ὑπακούει γλώσσης παρανόμων, δίκαιος δὲ οὐ προσέχει χείλεσι ψευδέσιν.
\vs{5}Ὁ καταγελῶν πτωχοῦ παροξύνει τὸν ποιήσαντα αὐτὸν, ὁ δὲ ἐπιχαίρων ἀπολλυμένῳ οὐκ ἀθωωθήσεται, ὁ δὲ ἐπισπλαγχνιζόμενος ἐλεηθήσεται.

\vs{6}Στέφανος γερόντων τέκνα τέκνων, καύχημα δὲ τέκνων πατέρες αὐτῶν·
\vs{6a}τοῦ πιστοῦ ὅλος ὁ κόσμος τῶν χρημάτων, τοῦ δὲ ἀπίστου οὐδὲ ὀβολός.
\vs{7}Οὐχ ἁρμόσει ἄφρονι χείλη πιστὰ, οὐδὲ δικαίῳ χείλη ψευδῆ.
\vs{8}Μισθὸς χαρίτων παιδεία τοῖς χρωμένοις, οὗ δʼ ἂν ἐπιστρέψῃ εὐοδωθήσεται.
\vs{9}Ὃς κρύπτει ἀδικήματα, ζητεῖ φιλίαν· ὃς δὲ μισεῖ κρύπτειν, διΐστησι φίλους καὶ οἰκείους.
\vs{10}Συντρίβει ἀπειλὴ καρδίαν φρονίμου, ἄφρων δὲ μαστιγωθεὶς οὐκ αἰσθάνεται.
\vs{11}Ἀντιλογίας ἐγείρει πᾶς κακὸς, ὁ δὲ Κύριος ἄγγελον ἀνελεήμονα ἐκπέμψει αὐτῷ.

\vs{12}Ἐμπεσεῖται μέριμνα ἀνδρὶ νοήμονι, οἱ δὲ ἄφρονες διαλογιοῦνται κακά.
\vs{13}Ὃς ἀποδίδωσι κακὰ ἀντὶ ἀγαθῶν, οὐ κινηθήσεται κακὰ ἐκ τοῦ οἴκου αὐτοῦ.
\vs{14}Ἐξουσίαν δίδωσι λόγοις ἀρχὴ δικαιοσύνης, προηγεῖται δὲ τῆς ἐνδείας στάσις καὶ μάχη.
\vs{15}Ὃς δίκαιον κρίνει τὸν ἄδικον, ἄδικον δὲ τὸν δίκαιον, ἀκάθαρτος καὶ βδελυκτὸς παρὰ Θεῷ.
\vs{16}Ἱνατί ὑπῆρξε χρήματα ἄφρονι; κτήσασθαι γὰρ σοφίαν ἀκάρδιος οὐ δυνήσεται·
\vs{16a}ὃς ὑψηλὸν ποιεῖ τὸν ἑαυτοῦ οἶκον, ζητεῖ συντριβήν· ὁ δὲ σκολιάζων τοῦ μαθεῖν, ἐμπεσεῖται εἰς κακά.
\vs{17}Εἰς πάντα καιρὸν φίλος ὑπαρχέτω σοι, ἀδελφοὶ δὲ ἐν ἀνάγκαις χρήσιμοι ἔστωσαν, τούτου γὰρ χάριν γεννῶνται.
\vs{18}Ἀνὴρ ἄφρων ἐπικροτεῖ καὶ ἐπιχαίρει ἑαυτῷ, ὡς καὶ ὁ ἐγγυώμενος ἐγγύῃ τῶν ἑαυτοῦ φίλων.

\vs{19}Φιλαμαρτήμων χαίρει μάχαις,
\vs{20}ὁ δὲ σκληροκάρδιος οὐ συναντᾷ ἀγαθοῖς· ἀνὴρ εὐμετάβολος γλώσσῃ ἐμπεσεῖται εἰς κακὰ,
\vs{21}καρδία δὲ ἄφρονος ὀδύνη τῷ κεκτημένῳ αὐτήν· οὐκ εὐφραίνεται πατὴρ ἐφʼ υἱῷ ἀπαιδεύτῳ, υἱὸς δὲ φρόνιμος εὐφραίνει μητέρα αὐτοῦ.
\vs{22}Καρδία εὐφραινομένη εὐεκτεῖν ποιεῖ, ἀνδρὸς δὲ λυπηροῦ ξηραίνεται τὰ ὀστᾶ.
\vs{23}Λαμβάνοντος δῶρα ἀδίκως ἐν κόλποις οὐ κατευοδοῦνται ὁδοὶ, ἀσεβὴς δὲ ἐκκλίνει ὁδοὺς δικαιοσύνης.
\vs{24}Πρόσωπον συνετὸν ἀνδρὸς σοφοῦ, οἱ δὲ ὀφθαλμοὶ τοῦ ἄφρονος ἐπʼ ἄκρα γῆς.
\vs{25}Ὀργὴ πατρὶ υἱὸς ἄφρων, καὶ ὀδύνη τῇ τεκούσῃ αὐτόν.

\vs{26}Ζημιοῦν ἄνδρα δίκαιον οὐ καλὸν, οὐδὲ ὅσιον ἐπιβουλεύειν δυνάσταις δικαίοις.
\vs{27}Ὃς φείδεται ῥῆμα προέσθαι σκληρὸν, ἐπιγνώμων· μακρόθυμος δὲ ἀνὴρ φρόνιμος.
\vs{28}Ἀνοήτῳ ἐπερωτήσαντι σοφίαν σοφία λογισθήσεται, ἐνεὸν δέ τις ἑαυτὸν ποιήσας, δόξει φρόνιμος εἶναι.

\ch{18}
Προφάσεις ζητεῖ ἀνὴρ βουλόμενος χωρίζεσθαι ἀπὸ φίλων, ἐν παντὶ δὲ καιρῷ ἐπονείδιστος ἔσται.
\vs{2}Οὐ χρείαν ἔχει σοφίας ἐνδεὴς φρενῶν, μᾶλλον γᾶρ ἄγεται ἀφροσύνῃ.
\vs{3}Ὅταν ἔλθῃ ἀσεβὴς εἰς βάθος κακῶν, καταφρονεῖ, ἐπέρχεται δὲ αὐτῷ ἀτιμία καὶ ὄνειδος.
\vs{4}Ὕδωρ βαθὺ λόγος ἐν καρδίᾳ ἀνδρὸς, ποταμὸς δὲ ἀναπηδύει καὶ πηγὴ ζωῆς.
\vs{5}Θαυμάσαι πρόσωπον ἀσεβοῦς οὐ καλὸν, οὐδὲ ὅσιον ἐκκλίνειν τὸ δίκαιον ἐν κρίσει.

\vs{6}Χείλη ἄφρονος ἄγουσιν αὐτὸν εἰς κακὰ, τὸ δὲ στόμα αὐτοῦ τὸ θρασὺ θάνατον ἐπικαλεῖται.
\vs{7}Στόμα ἄφρονος συντριβὴ αὐτῷ, τὰ δὲ χείλη αὐτοῦ παγὶς τῇ ψυχῇ αὐτοῦ.
\vs{8}Ὀκνηροὺς καταβάλλει φόβος, ψυχαὶ δὲ ἀνδρογύνων πεινάσουσιν.
\vs{9}Ὁ μὴ ἰώμενος αὐτὸν ἐν τοῖς ἔργοις αὐτοῦ, ἀδελφός ἐστι τοῦ λυμαινομένου ἑαυτόν.
\vs{10}Ἐκ μεγαλωσύνης ἰσχύος ὄνομα Κυρίου, αὐτῷ δὲ προσδραμόντες δίκαιοι ὑψοῦνται.
\vs{11}Ὕπαρξις πλουσίου ἀνδρὸς πόλις ὀχυρὰ, ἡ δὲ δόξα αὐτῆς μέγα ἐπισκιάζει.
\vs{12}Πρὸ συντριβῆς ὑψοῦται καρδία ἀνδρὸς, καὶ πρὸ δόξης ταπεινοῦνται.
\vs{13}Ὃς ἀποκρίνεται λόγον πρὶν ἀκοῦσαι, ἀφροσύνη αὐτῷ ἐστι καὶ ὄνειδος.
\vs{14}Θυμὸν ἀνδρὸς πρᾳΰνει θεράπων φρόνιμος, ὀλιγόψυχον δὲ ἄνδρα τίς ὑποίσει;
\vs{15}Καρδία φρονίμου κτᾶται αἴσθησιν, ὦτα δὲ σοφῶν ζητεῖ ἔννοιαν.
\vs{16}Δόμα ἀνθρώπου ἐμπλατύνει αὐτὸν, καὶ παρὰ δυνάσταις καθιζάνει αὐτόν.
\vs{17}Δίκαιος ἑαυτοῦ κατήγορος ἐν πρωτολογίᾳ, ὡς δʼ ἂν ἐπιβάλῃ ὁ ἀντίδικος ἐλέγχεται.

\vs{18}Ἀντιλογίας παύει σιγηρὸς, ἐν δὲ δυναστείαις ὁρίζει.
\vs{19}Ἀδελφὸς ὑπὸ ἀδελφοῦ βοηθούμενος, ὡς πόλις ὀχυρὰ καὶ ὑψηλὴ, ἰσχύει δὲ ὥσπερ τεθεμελιωμένον βασίλειον.
\vs{20}Ἀπὸ καρπῶν στόματος ἀνὴρ πίμπλησι κοιλίαν αὐτοῦ, ἀπὸ δὲ καρπῶν χειλέων αὐτοῦ ἐμπλησθήσεται.
\vs{21}Θάνατος καὶ ζωὴ ἐν χειρὶ γλώσσης, οἱ δὲ κρατοῦντες αὐτῆς ἔδονται τοὺς καρποὺς αὐτῆς.
\vs{22}Ὃς εὗρε γυναῖκα ἀγαθὴν, εὗρε χάριτας, ἔλαβε δὲ παρὰ Θεοῦ ἱλαρότητα·
\vs{22a}ὃς ἐκβάλλει γυναῖκα ἀγαθὴν, ἐκβάλλει τὰ ἀγαθὰ, ὁ δὲ κατέχων μοιχαλίδα, ἄφρων καὶ ἀσεβής.

\ch{19}
\vs{3}Ἀφροσύνη ἀνδρὸς λυμαίνεται τὰς ὁδοὺς αὐτοῦ, τὸν δὲ Θεὸν αἰτιᾶται τῇ καρδίᾳ αὐτοῦ.

\vs{4}Πλοῦτος προστίθησι φίλους πολλοὺς, ὁ δὲ πτωχὸς καὶ ἀπὸ τοῦ ὑπάρχοντος φίλου λείπεται.
\vs{5}Μάρτυς ψευδὴς οὐκ ἀτιμώρητος ἔσται, ὁ δὲ ἐγκαλῶν ἀδίκως οὐ διαφεύξεται.
\vs{6}Πολλοὶ θεραπεύουσι πρόσωπα βασιλέων, πᾶς δὲ ὁ κακὸς γίνεται ὄνειδος ἀνδρί.
\vs{7}Πᾶς ὃς ἀδελφὸν πτωχὸν μισεῖ, καὶ φιλίας μακρὰν ἔσται· ἔννοια ἀγαθὴ τοῖς εἰδόσιν αὐτὴν ἐγγιεῖ, ἀνὴρ δὲ φρόνιμος εὑρήσει αὐτήν· ὁ πολλὰ κακοποιῶν τελεσιουργεῖ κακίαν, ὃς δὲ ἐρεθίζει λόγους, οὐ σωθήσεται.

\vs{8}Ὁ κτώμενος φρόνησιν ἀγαπᾷ ἑαυτὸν, ὃς δὲ φυλάσσει φρόνησιν, εὑρήσει ἀγαθά.
\vs{9}Μάρτυς ψευδὴς οὐκ ἀτιμώρητος ἔσται, ὃς δʼ ἂν ἐκκαύσῃ κακίαν, ἀπολεῖται ὑπʼ αὐτῆς.
\vs{10}Οὐ συμφέρει ἄφρονι τρυφὴ, καὶ ἐὰν οἰκέτης ἄρξηται μεθʼ ὕβρεως δυναστεύειν.
\vs{11}Ἐλεήμων ἀνὴρ μακροθυμεῖ, τὸ δὲ καύχημα αὐτοῦ ἐπέρχεται παρανόμοις.
\vs{12}Βασιλέως ἀπειλὴ ὁμοία βρυγμῷ λέοντος· ὥσπερ δὲ δρόσος ἐπὶ χόρτῳ, οὕτως τὸ ἱλαρὸν αὐτοῦ.

\vs{13}Αἰσχύνη πατρὶ υἱὸς ἄφρων, οὐχ ἁγναὶ εὐχαὶ ἀπὸ μισθώματος ἑταίρας.
\vs{14}Οἶκον καὶ ὕπαρξιν μερίζουσι πατέρες παισὶ, παρὰ δὲ Κυρίου ἁρμόζεται γυνὴ ἀνδρί.
\vs{15}Δειλία κατέχει ἀνδρόγυνον, ψυχὴ δὲ ἀεργοῦ πεινάσει.
\vs{16}Ὃς φυλάσσει ἐντολὴν, τηρεῖ τὴν ἑαυτοῦ ψυχήν· ὁ δὲ καταφρονῶν τῶν ἑαυτοῦ ὁδῶν, ἀπολεῖται.
\vs{17}Δανείζει Θεῷ ὁ ἐλεῶν πτωχὸν, κατὰ δὲ τὸ δόμα αὐτοῦ ἀνταποδώσει αὐτῷ.
\vs{18}Παίδευε υἱόν σου, οὕτως γὰρ ἔσται εὔελπις, εἰς δὲ ὕβριν μὴ ἐπαίρου τῇ ψυχῇ σου.
\vs{19}Κακόφρων ἀνὴρ πολλὰ ζημιωθήσεται· ἐὰν δὲ λοιμεύηται, καὶ τὴν ψυχὴν αὐτοῦ προσθήσει.

\vs{20}Ἄκουε, υἱὲ, παιδείαν πατρός σου, ἵνα σοφὸς γένῃ ἐπʼ ἐσχάτων σου.
\vs{21}Πολλοὶ λογισμοὶ ἐν καρδίᾳ ἀνδρὸς, ἡ δὲ βουλὴ τοῦ Κυρίου εἰς τὸν αἰῶνα μένει.
\vs{22}Καρπὸς ἀνδρὶ ἐλεημοσύνη, κρείσσων δὲ πτωχὸς δίκαιος ἢ πλούσιος ψευδής.
\vs{23}Φόβος Κυρίου εἰς ζωὴν ἀνδρὶ· ὁ δὲ ἄφοβος αὐλισθήσεται ἐν τόποις οὗ οὐκ ἐπισκοπεῖται γνῶσις.
\vs{24}Ὁ ἐγκρύπτων εἰς τὸν κόλπον αὐτοῦ χεῖρας ἀδίκως, οὐδὲ τῷ στόματι οὐ μὴ προσενείκῃ αὐτάς.
\vs{25}Λοιμοῦ μαστιγουμένου, ἄφρων πανουργότερος γίνεται· ἐὰν δὲ ἐλέγχῃς ἄνδρα φρόνιμον, νοήσει αἴσθησιν.

\vs{26}Ὁ ἀτιμάζων πατέρα καὶ ἀπωθούμενος μητέρα αὐτοῦ, καταισχυνθήσεται καὶ ἐπονείδιστος ἔσται.
\vs{27}Υἱὸς ἀπολειπόμενος φυλάξαι παιδείαν πατρὸς, μελετήσει ῥήσεις κακάς.
\vs{28}Ὁ ἐγγυώμενος παῖδα ἄφρονα, καθυβρίσει δικαίωμα· στόμα δὲ ἀσεβῶν καταπίεται κρίσεις.
\vs{29}Ἑτοιμάζονται ἀκολάστοις μάστιγες, καὶ τιμωρίαι ὁμοίως ἄφροσιν.

\ch{20}
Ἀκολάστον οἶνος, καὶ ὑβριστικὸν μέθη, πᾶς δὲ ἄφρων τοιούτοις συμπλέκεται.
\vs{2}Οὐ διαφέρει ἀπειλὴ βασιλέως θυμοῦ λέοντος, ὁ δὲ παροξύνων αὐτὸν ἁμαρτάνει εἰς τὴν ἑαυτοῦ ψυχήν.
\vs{3}Δόξα ἀνδρὶ ἀποστρέφεσθαι λοιδορίας, πᾶς δὲ ἄφρων τοιούτοις συμπλέκεται.
\vs{4}Ὀνειδιζόμενος ὀκνηρὸς οὐκ αἰσχύνεται, ὡσαύτως καὶ ὁ δανειζόμενος σῖτον ἐν ἀμητῷ.

\vs{5}Ὕδωρ βαθὺ βουλὴ ἐν καρδίᾳ ἀνδρὸς, ἀνὴρ δὲ φρόνιμος ἐξαντλήσει αὐτήν.
\vs{6}Μέγα ἄνθρωπος, καὶ τίμιον ἀνὴρ ἐλεήμων, ἄνδρα δὲ πιστὸν ἔργον εὑρεῖν.
\vs{7}Ὃς ἀναστρέφεται ἄμωμος ἐν δικαιοσύνῃ, μακαρίους τοὺς παῖδας αὐτοῦ καταλείψει.
\vs{8}Ὅταν βασιλεὺς δίκαιος καθίσῃ ἐπὶ θρόνου, οὐκ ἐναντιοῦται ἐν ὀφθαλμοῖς αὐτοῦ πᾶν πονηρόν.
\vs{9}Τίς καυχήσεται ἁγνὴν ἔχειν τὴν καρδίαν; ἢ τίς παῤῥησιάσεται καθαρὸς εἶναι ἀπὸ ἁμαρτιῶν;
\vs{9a}Κακολογοῦντος πατέρα ἢ μητέρα σβεσθήσεται λαμπτὴρ, αἱ δὲ κόραι τῶν ὀφθαλμῶν αὐτοῦ ὄψονται σκότος.

\vs{9b}Μερὶς ἐπισπουδαζομένη ἐν πρώτοις, ἐν τοῖς τελευταίοις οὐκ εὐλογηθήσεται.
\vs{9c}Μὴ εἴπῃς, τίσομαι τὸν ἐχθρὸν, ἀλλʼ ὑπόμεινον τὸν Κύριον, ἵνα σοι βοηθήσῃ.

\vs{10}Στάθμιον μέγα καὶ μικρὸν, καὶ μέτρα δισσὰ, ἀκάθαρτα ἐνώπιον Κυρίου καὶ ἀμφότερα, καὶ ὁ ποιῶν αὐτά.
\vs{11}Ἐν τοῖς ἐπιτηδεύμασιν αὐτοῦ συμποδισθήσεται νεανίσκος μετὰ ὁσίου, καὶ εὐθεῖα ἡ ὁδὸς αὐτοῦ.
\vs{12}Οὖς ἀκούει, καὶ ὀφθαλμὸς ὁρᾷ, Κυρίου ἔργα καὶ ἀμφότερα.
\vs{13}Μὴ ἀγάπα καταλαλεῖν, ἵνα μὴ ἐξαρθῇς· διάνοιξον τοὺς ὀφθαλμούς σου, καὶ ἐμπλήσθητι ἄρτων.

\vs{23}Βδέλυγμα Κυρίῳ δισσὸν στάθμιον, καὶ ζυγὸς δόλιος οὐ καλὸν ἐνώπιον αὐτοῦ.
\vs{24}Παρὰ Κυρίου εὐθύνεται τὰ διαβήματα ἀνδρὶ, θνητὸς δὲ πῶς ἂν νοήσαι τὰς ὁδοὺς αὐτοῦ;
\vs{25}Παγὶς ἀνδρὶ ταχύ τι τῶν ἰδίων ἁγιάσαι, μετὰ γὰρ τὸ εὔξασθαι μετανοεῖν γίνεται.
\vs{26}Λικμήτωρ ἀσεβῶν βασιλεὺς σοφὸς, καὶ ἐπιβαλεῖ αὐτοῖς τροχόν.

\vs{27}Φῶς Κυρίου πνοὴ ἀνθρώπων, ὃς ἐρευνᾷ ταμιεῖα κοιλίας.
\vs{28}Ἐλεημοσύνη καὶ ἀλήθεια φυλακὴ βασιλεῖ, καὶ περικυκλώσουσιν ἐν δικαιοσύνῃ τὸν θρόνον αὐτοῦ.
\vs{29}Κόσμος νεανίαις σοφία, δόξα δὲ πρεσβυτέρων πολιαί.
\vs{30}Ὑπώπια καὶ συντρίμματα συναντᾷ κακοῖς, πληγαὶ δὲ εἰς ταμιεῖα κοιλίας.

\ch{21}
Ὥσπερ ὁρμὴ ὕδατος, οὕτως καρδία βασιλέως ἐν χειρὶ Θεοῦ, οὗ ἐὰν θέλων νεῦσαι ἐκεῖ ἔκλινεν αὐτήν.
\vs{2}Πᾶς ἀνὴρ φαίνεται ἑαυτῷ δίκαιος, κατευθύνει δὲ καρδίας Κύριος.
\vs{3}Ποιεῖν δίκαια καὶ ἀληθεύειν, ἀρεστὰ παρὰ Θεῷ μᾶλλον ἢ θυσιῶν αἷμα.
\vs{4}Μεγαλόφρων ἐν ὕβρει θρασυκάρδιος, λαμπτὴρ δὲ ἀσεβῶν ἁμαρτία.
\vs{6}Ὁ ἐνεργῶν θησαυρίσματα γλώσσῃ ψευδεῖ, μάταια διώκει ἐπὶ παγίδας θανάτου.
\vs{7}Ὄλεθρος ἀσεβέσιν ἐπιξενωθήσεται, οὐ γὰρ βούλονται πράσσειν τὰ δίκαια.
\vs{8}Πρὸς τοὺς σκολιοὺς σκολιὰς ὁδοὺς ἀποστέλλει ὁ Θεὸς, ἁγνὰ γὰρ καὶ ὀρθὰ τὰ ἔργα αὐτοῦ.
\vs{9}Κρεῖσσον οἰκεῖν ἐπὶ γωνίας ὑπαίθρου, ἢ ἐν κεκονιαμένοις μετὰ ἀδικίας καὶ ἐν οἴκῳ κοινῷ.
\vs{10}Ψυχὴ ἀσεβοῦς οὐκ ἐλεηθήσεται ὑπʼ οὐδενὸς τῶν ἀνθρώπων.
\vs{11}Ζημιουμένου ἀκολάστου πανουργότερος γίνεται ὁ ἄκακος, συνιῶν δὲ σοφὸς δέξεται γνῶσιν.
\vs{12}Συνιεῖ δίκαιος καρδίας ἀσεβῶν, καὶ φαυλίζει ἀσεβεῖς ἐν κακοῖς.

\vs{13}Ὃς φράσσει τὰ ὦτα αὐτοῦ τοῦ μὴ ἐπακοῦσαι ἀσθενοῦς, καὶ αὐτὸς ἐπικαλέσεται καὶ οὐκ ἔσται ὁ εἰσακούων.
\vs{14}Δόσις λάθριος ἀνατρέπει ὀργάς, δώρων δὲ ὁ φειδόμενος θυμὸν ἐγείρει ἰσχυρόν.
\vs{15}Εὐφροσύνη δικαίων ποιεῖν κρίμα, ὅσιος δὲ ἀκάθαρτος παρὰ κακούργοις.
\vs{16}Ἀνὴρ πλανώμενος ἐξ ὁδοῦ δικαιοσύνης, ἐν συναγωγῇ γιγάντων ἀναπαύσεται.
\vs{17}Ἀνὴρ ἐνδεὴς ἀγαπᾷ εὐφροσύνην, φιλῶν οἶνον καὶ ἔλαιον εἰς πλοῦτον·
\vs{18}Περικάθαρμα δὲ δικαίου ἄνομος.
\vs{19}Κρεῖσσον οἰκεῖν ἐν τῇ ἐρήμῳ, ἢ μετὰ γυναικὸς μαχίμου καὶ γλωσσώδους καὶ καὶ ὀργίλου.
\vs{20}Θησαυρὸς ἐπιθυμητὸς ἀναπαύσεται ἐπὶ στόματος σοφοῦ, ἄφρονες δὲ ἄνδρες καταπίονται αὐτόν.
\vs{21}Ὁδὸς δικαιοσύνης καὶ ἐλεημοσύνης εὑρήσει ζωὴν καὶ δόξαν.
\vs{22}Πόλεις ὀχυρὰς ἐπέβη σοφὸς, καὶ καθεῖλε τὸ ὀχύρωμα ἐφʼ ᾧ ἐπεποίθεισαν οἱ ἀσεβεῖς.
\vs{23}Ὃς φυλάσσει τὸ στόμα αὐτοῦ καὶ τὴν γλῶσσαν, διατηρεῖ ἐκ θλίψεως τὴν ψυχὴν αὐτοῦ.

\vs{24}Θρασὺς καὶ αὐθάδης καὶ ἀλαζὼν λοιμὸς καλεῖται, ὃς δὲ μνησικακεῖ παράνομος.
\vs{25}Ἐπιθυμίαι ὀκνηρὸν ἀποκτείνουσιν, οὐ γὰρ προαιροῦνται αἱ χεῖρες αὐτοῦ ποιεῖν τι.
\vs{26}Ἀσεβὴς ἐπιθυμεῖ ὅλην τὴν ἡμέραν ἐπιθυμίας κακὰς, ὁ δὲ δίκαιος ἐλεᾷ καὶ οἰκτείρει ἀφειδῶς.
\vs{27}Θυσίαι ἀσεβῶν βδέλυγμα Κυρίῳ, καὶ γὰρ παρανόμως προσφέρουσιν αὐτάς.
\vs{28}Μάρτυς ψευδὴς ἀπολεῖται, ἀνὴρ δὲ ὑπήκοος φυλασσόμενος λαλήσει.
\vs{29}Ἀσεβὴς ἀνὴρ ἀναιδῶς ὑφίσταται προσώπῳ, ὁ δὲ εὐθὺς αὐτὸς συνιεῖ τὰς ὁδοὺς αὐτοῦ.
\vs{30}Οὐκ ἔστι σοφία, οὐκ ἔστιν ἀνδρεία, οὐκ ἔστι βουλὴ πρὸς τὸν ἀσεβῆ.
\vs{31}Ἵππος ἑτοιμάζεται εἰς ἡμέραν πολέμου, παρὰ δὲ Κυρίου ἡ βοήθεια.

\ch{22}
Αἱρετώτερον ὄνομα καλὸν ἢ πλοῦτος πολύς, ὑπὲρ δὲ ἀργύριον καὶ χρυσίον χάρις ἀγαθή.
\vs{2}Πλούσιος καὶ πτωχὸς συνήντησαν ἀλλήλοις, ἀμφοτέρους δὲ ὁ κύριος ἐποίησε.
\vs{3}Πανοῦργος ἰδὼν πονηρὸν τιμωρούμενον κραταιῶς αὐτὸς παιδεύεται, οἱ δὲ ἄφρονες παρελθόντες ἐζημιώθησαν.
\vs{4}Γενεὰ σοφίας φόβος Κυρίου, καὶ πλοῦτος, καὶ δόξα, καὶ ζωή.
\vs{5}Τρίβολοι καὶ παγίδες ἐν ὁδοῖς σκολιαῖς, ὁ δὲ φυλάσσων τὴν ἑαυτοῦ ψυχὴν ἀφέξεται αὐτῶν.
\vs{7}Πλούσιοι πτωχῶν ἄρξουσι, καὶ οἰκέται ἰδίοις δεσπόταις δανειοῦσιν.

\vs{8}Ὁ σπείρων φαῦλα θερίσει κακὰ, πληγὴν δὲ ἔργων αὐτοῦ συντελέσει·
\vs{8a}ἄνδρα ἱλαρὸν καὶ δότην εὐλογεῖ ὁ Θεὸς, ματαιότητα δὲ ἔργων αὐτοῦ συντελέσει.
\vs{9}Ὁ ἐλεῶν πτωχὸν αὐτὸς διατραφήσεται, τῶν γὰρ ἑαυτοῦ ἄρτων ἔδωκε τῷ πτωχῷ·
\vs{9a}νίκην καὶ τιμὴν περιποιεῖται ὁ δῶρα δοὺς, τὴν μέντοι ψυχὴν ἀφαιρεῖται τῶν κεκτημένων.
\vs{10}Ἔκβαλε ἐκ συνεδρίου λοιμὸν, καὶ συνεξελεύσεται αὐτῷ νεῖκος, ὅταν γὰρ καθίσῃ ἐν συνεδρίῳ πάντας ἀτιμάζει.

\vs{11}Ἀγαπᾷ Κύριος ὁσίας καρδίας, δεκτοὶ δὲ αὐτῷ πάντες ἄμωμοι· χείλεσι ποιμαίνει βασιλεύς.
\vs{12}Οἱ δὲ ὀφθαλμοὶ Κυρίου διατηροῦσιν αἴσθησιν, φαυλίζει δὲ λόγους παράνομος.
\vs{13}Προφασίζεται, καὶ λέγει ὀκνηρὸς, λέων ἐν ταῖς ὁδοῖς, ἐν δὲ ταῖς πλατείαις φονευταί.
\vs{14}Βόθρος βαθὺς στόμα παρανόμου, ὁ δὲ μισηθεὶς ὑπὸ Κυρίου ἐμπεσεῖται εἰς αὐτόν.
\vs{14a}εἰσὶν ὁδοὶ κακαὶ ἐνώπιον ἀνδρὸς, καὶ οὐκ ἀγαπᾷ τοῦ ἀποστρέψαι ἀπʼ αὐτῶν, ἀποστρέφειν δὲ δεῖ ἀπὸ ὁδοῦ σκολιᾶς καὶ κακῆς.
\vs{15}Ἄνοια ἐξῆπται καρδίας νέου, ῥάβδος δὲ καὶ παιδεία μακρὰν ἀπʼ αὐτοῦ.

\vs{16}Ὁ συκοφαντῶν πένητα, πολλὰ ποιεῖ τὰ ἑαυτοῦ, δίδωσι δὲ πλουσίῳ ἐπʼ ἐλάσσονι.

\vs{17}Λόγοις σοφῶν παράβαλλε σὸν οὖς, καὶ ἄκουε ἐμὸν λόγον, τὴν δὲ σὴν καρδίαν ἐπίστησον, ἵνα γνῷς ὅτι καλοί εἰσι·
\vs{18}καὶ ἐὰν ἐμβάλῃς αὐτοὺς εἰς τὴν καρδίαν σου, εὐφρανοῦσί σε ἅμα ἐπὶ σοῖς χείλεσιν·
\vs{19}Ἵνα σου γένηται ἐπὶ Κύριον ἡ ἐλπὶς, καὶ γνωρίσῃ σοι τὴν ὁδόν σου.
\vs{20}Καὶ σὺ δὲ ἀπόγραψαι αὐτὰ σεαυτῷ τρισσῶς, εἰς βουλὴν καὶ γνῶσιν ἐπὶ τὸ πλάτος τῆς καρδίας σου.
\vs{21}Διδάσκω οὖν σε ἀληθῆ λόγον, καὶ γνῶσιν ἀγαθὴν ὑπακούειν, τοῦ ἀποκρίνεσθαί σε λόγους ἀληθείας τοῖς προβαλλομένοις σοι.

\vs{22}Μὴ ἀποβιάζου πένητα, πτωχὸς γὰρ ἐστι, καὶ μὴ ἀτιμάσῃς ἀσθενῆ ἐν πύλαις.
\vs{23}Ὁ γὰρ Κύριος κρινεῖ αὐτοῦ τὴν κρίσιν, καὶ ῥύσῃ σὴν ἄσυλον ψυχήν.

\vs{24}Μὴ ἴσθι ἑταῖρος ἀνδρὶ θυμώδει, φίλῳ δὲ ὀργίλῳ μὴ συναυλίζου·
\vs{25}μήποτε μάθῃς τῶν ὁδῶν αὐτοῦ, καὶ λάβῃς βρόχους τῇ σῇ ψυχῇ.

\vs{26}Μὴ δίδου σεαυτὸν εἰς ἐγγύην αἰσχυνόμενος πρόσωπον·
\vs{27}Ἐὰν γὰρ μὴ ἔχῃ πόθεν ἀποτίσῃς, λήψονται τὸ στρῶμα τὸ ὑπὸ τὰς πλευράς σου.
\vs{28}Μὴ μέταιρε ὅρια αἰώνια, ἃ ἔθεντο οἱ πατέρες σου.

\vs{29}Ὁρατικὸν ἄνδρα καὶ ὀξὺν ἐν τοῖς ἔργοις αὐτοῦ βασιλεῦσι δεῖ παρεστάναι, καὶ μὴ παρεστάναι ἀνδράσι νωθροῖς.

\ch{23}
Ἐὰν καθίσῃς δειπνεῖν ἐπὶ τραπέζης δυνάστου, νοητῶς νόει τὰ παρατιθέμενά σοι.
\vs{2}Καὶ ἐπίβαλλε τὴν χεῖρά σου, εἰδὼς ὅτι τοιαῦτά σε δεῖ παρασκευάσαι· εἰ δὲ ἀπληστότερος εἶ,
\vs{3}μὴ ἐπιθύμει τῶν ἐδεσμάτων αὐτοῦ, ταῦτα γὰρ ἔχεται ζωῆς ψευδοῦς.

\vs{4}Μὴ παρεκτείνου πένης ὢν πλουσίῳ, τῇ δὲ σῇ ἐννοίᾳ ἀπόσχου.
\vs{5}Ἐὰν ἐπιστήσῃς τὸ σὸν ὄμμα πρὸς αὐτὸν, οὐδαμοῦ φανεῖται· κατεσκεύασται γὰρ αὐτῷ πτέρυγες ὥσπερ ἀετοῦ, καὶ ὑποστρέφει εἰς τὸν οἶκον τοῦ προεστηκότος αὐτοῦ.
\vs{6}Μὴ συνδείπνει ἀνδρὶ βασκάνῳ, μηδὲ ἐπιθύμει τῶν βρωμάτων αὐτοῦ,
\vs{7}ὃν τρόπον γὰρ εἴ τις καταπίοι τρίχα, οὕτως ἐσθίει καὶ πίνει· μηδὲ πρὸς σὲ εἰσαγάγῃς αὐτὸν, καὶ φάγῃς τὸν ψωμόν σου μετʼ αὐτοῦ,
\vs{8}ἐξεμέσει γὰρ αὐτὸν, καὶ λυμανεῖται τοὺς λόγους σου τοὺς καλούς.

\vs{9}Εἰς ὦτα ἄφρονος μηδὲν λέγε, μήποτε μυκτηρίσῃ τοὺς συνετοὺς λόγους σου.
\vs{10}Μὴ μεταθῇς ὅρια αἰώνια, εἰς δὲ κτῆμα ὀρφανῶν μὴ εἰσέλθῃς·
\vs{11}Ὁ γὰρ λυτρούμενος αὐτοὺς Κύριος, κραταιός ἐστι, καὶ κρινεῖ τὴν κρίσιν αὐτῶν μετὰ σοῦ.
\vs{12}Δὸς εἰς παιδείαν τὴν καρδίαν σου, τὰ δὲ ὦτά σου ἑτοίμασον λόγοις αἰσθήσεως.

\vs{13}Μὴ ἀπόσχῃ νήπιον παιδεύειν, ὅτι ἐὰν πατάξῃς αὐτὸν ῥάβδῳ, οὐ μὴ ἀποθάνῃ.
\vs{14}Συ μὲν γὰρ πατάξεις αὐτὸν ῥάβδῳ, τὴν δὲ ψυχὴν αὐτοῦ ἐκ θανάτου ῥύσῃ.

\vs{15}Υἱὲ, ἐὰν σοφὴ γένηταί σου ἡ καρδία, εὐφρανεῖς καὶ τὴν ἐμὴν καρδίαν,
\vs{16}καὶ ἐνδιατρίψει λόγοις τὰ σὰ χείλη πρὸς τὰ ἐμὰ χείλη ἐὰν ὀρθὰ ὦσι.
\vs{17}Μὴ ζηλούτω ἡ καρδία σου ἁμαρτωλοὺς, ἀλλὰ ἐν φόβῳ Κυρίου ἴσθι ὅλην τὴν ἡμέραν.
\vs{18}Ἐὰν γὰρ τηρήσῃς αὐτὰ, ἔσται σοι ἔκγονα, ἡ δὲ ἐλπίς σου οὐκ ἀποστήσεται.

\vs{19}Ἄκουε υἱὲ, καὶ σοφὸς γίνου, καὶ κατεύθυνε ἐννοίας σῆς καρδίας.
\vs{20}Μὴ ἴσθι οἰνοπότης, μηδὲ ἐκτείνου συμβολαῖς, κρεῶν τε ἀγορασμοῖς.
\vs{21}Πᾶς γὰρ μέθυσος καὶ πορνοκόπος πτωχεύσει, καὶ ἐνδύσεται διεῤῥηγμένα καὶ ῥακώδη πᾶς ὑπνώδης.

\vs{22}Ἄκουε, υἱὲ, πατρὸς τοῦ γεννήσαντός σε, καὶ μὴ καταφρόνει ὅτι γεγήρακέ σου ἡ μήτηρ.
\vs{24}Καλῶς ἐκτρέφει πατὴρ δίκαιος, ἐπὶ δὲ υἱῷ σοφῷ εὐφραίνεται ἡ ψυχὴ αὐτοῦ.
\vs{25}Εὐφραινέσθω ὁ πατὴρ καὶ ἡ μήτηρ ἐπὶ σοὶ, καὶ χαιρέτω ἡ τεκοῦσά σε.

\vs{26}Δός μοι υἱὲ σὴν καρδίαν, οἱ δὲ σοὶ ὀφθαλμοὶ ἐμὰς ὁδοὺς τηρείτωσαν.
\vs{27}Πίθος γὰρ τετρημένος ἐστὶν ἀλλότριος οἶκος, καὶ φρέαρ στενὸν ἀλλότριον.
\vs{28}Οὗτος γὰρ συντόμως ἀπολεῖται, καὶ πᾶς παράνομος ἀναλωθήσεται.

\vs{29}Τίνι οὐαί; τίνι θόρυβος; τίνι κρίσεις; τίνι δὲ ἀηδίαι καὶ λέσχαι; τίνι συντρίμματα διακενῆς; τίνος πελιδνοὶ οἱ ὀφθαλμοί;
\vs{30}Οὐ τῶν ἐγχρονιζόντων ἐν οἴνοις; οὐ τῶν ἰχνευόντων ποῦ πότοι γίνονται; μὴ μεθύσκεσθε ἐν οἴνοις, ἀλλὰ ὁμιλεῖτε ἀνθρώποις δικαίοις καὶ ὁμιλεῖτε ἐν περιπάτοις.
\vs{31}Ἐὰν γὰρ εἰς τὰς φιάλας καὶ τὰ ποτήρια δῷς τοὺς ὀφθαλμούς σου, ὕστερον περιπατήσεις γυμνότερος ὑπέρου.
\vs{32}Τὸ δὲ ἔσχατον ὥσπερ ὑπὸ ὄφεως πεπληγὼς ἐκτείνεται, καὶ ὥσπερ ὑπὸ κεράστου διαχεῖται αὐτῷ ὁ ἰός.

\vs{33}Οἱ ὀφθαλμοί σου ὅταν ἴδωσιν ἀλλοτρίαν, τὸ στόμα σου τότε λαλήσει σκολιά.
\vs{34}Καὶ κατακείσῃ ὥσπερ ἐν καρδίᾳ θαλάσσης, καὶ ὥσπερ κυβερνήτης ἐν πολλῷ κλύδωνι.
\vs{35}Ἐρεῖς δὲ, τύπτουσί με καὶ οὐκ ἐπόνεσα, καὶ ἐνέπαιξάν μοι, ἐγὼ δὲ οὐκ ᾔδειν· πότε ὄρθρος ἔσται, ἵνα ἐλθὼν ζητήσω μεθʼ ὧν συνελεύσομαι;

\ch{24}
Υἱέ, μὴ ζηλώσῃς κακοὺς ἄνδρας, μηδὲ ἐπιθυμήσῃς εἶναι μετʼ αὐτῶν.
\vs{2}Ψευδῆ γὰρ μελετᾷ ἡ καρδία αὐτῶν, καὶ πόνους τὰ χείλη αὐτῶν λαλεῖ.
\vs{3}Μετὰ σοφίας οἰκοδομεῖται οἶκος, καὶ μετὰ συνέσεως ἀνορθοῦται.
\vs{4}Μετὰ αἰσθήσεως ἐμπίμπλανται ταμιεῖα ἐκ παντὸς πλούτου τιμίου καὶ καλοῦ.
\vs{5}Κρείσσων σοφὸς ἰσχυροῦ, καὶ ἀνὴρ φρόνησιν ἔχων γεωργίου μεγάλου.
\vs{6}Μετὰ κυβερνήσεως γίνεται πόλεμος, βοήθεια δὲ μετὰ καρδίας βουλευτικῆς.

\vs{7}Σοφία καὶ ἔννοια ἀγαθὴ ἐν πύλαις σοφῶν· σοφοὶ οὐκ ἐκκλίνουσιν ἐκ στόματος Κυρίου,
\vs{8}ἀλλὰ λογίζονται ἐν συνεδρίοις· ἀπαιδεύτοις συναντᾷ θάνατος,
\vs{9}ἀποθνήσκει δὲ ἄφρων ἐν ἁμαρτίαις· ἀκαθαρσία δὲ ἀνδρὶ λοιμῷ,
\vs{10}ἐμμολυνθήσεται ἐν ἡμέρᾳ κακῇ, καὶ ἐν ἡμέρᾳ θλίψεως ἕως ἂν ἐκλίπῃ.

\vs{11}Ῥῦσαι ἀγομένους εἰς θάνατον, καὶ ἐκπρίου κτεινομένους, μὴ φείσῃ.
\vs{12}Ἐὰν δὲ εἴπῃς, οὐκ οἶδα τοῦτον, γίνωσκε, ὅτι Κύριος καρδίας πάντων γινώσκει· καὶ ὁ πλάσας πνοὴν πᾶσιν, αὐτὸς οἶδε πάντα, ὃς ἀποδίδωσιν ἑκάστῳ κατὰ τὰ ἔργα αὐτοῦ.
\vs{13}Φάγε μέλι υἱὲ, ἀγαθὸν γὰρ κηρίον, ἵνα γλυκανθῇ σου ὁ φάρυγξ.
\vs{14}Οὕτως αἰσθητήσῃ σοφίαν τῇ σῇ ψυχῇ· ἐὰν γὰρ εὕρῃς, ἔσται καλὴ ἡ τελευτή σου, καὶ ἐλπίς σε οὐκ ἐγκαταλείψει.

\vs{15}Μὴ προσαγάγῃς ἀσεβῆ νομῇ δικαίων, μηδὲ ἀπατηθῇς χορτασίᾳ κοιλίας.
\vs{16}Ἑπτάκις γὰρ πεσεῖται δίκαιος καὶ ἀναστήσεται, οἱ δὲ ἀσεβεῖς ἀσθενήσουσιν ἐν κακοῖς.
\vs{17}Ἐὰν πέσῃ ὁ ἐχθρός σου, μὴ ἐπιχαρῇς αὐτῷ, ἐν δὲ τῷ ὑποσκελίσματι αὐτοῦ μὴ ἐπαίρου.
\vs{18}Ὅτι ὄψεται Κύριος καὶ οὐκ ἀρέσει αὐτῷ, καὶ ἀποστρέψει τὸν θυμὸν αὐτοῦ ἀπʼ αὐτοῦ.
\vs{19}Μὴ χαῖρε ἐπὶ κακοποιοῖς, μηδὲ ζήλου ἁμαρτωλούς.
\vs{20}Οὐ γὰρ μὴ γένηται ἔκγονα πονηρῷ, λαμπτὴρ δὲ ἀσεβῶν σβεσθήσεται.

\vs{21}Φοβοῦ τὸν Θεὸν υἱὲ, καὶ βασιλέα, καὶ μηθʼ ἑτέρῳ αὐτῶν ἀπειθήσῃς.
\vs{22}Ἐξαίφνης γὰρ τίσονται τοὺς ἀσεβεῖς, τὰς δὲ τιμωρίας ἀμφοτέρων τίς γνώσεται;

\vs{22a}Λόγον φυλασσόμενος υἱὸς ἀπωλείας ἐκτὸς ἔσται, [δεχόμενος δὲ ἐδέξατο αὐτόν·
\vs{22b}μηδὲν ψεῦδος ἀπὸ γλώσσης βασιλεῖ λεγέσθω, καὶ οὐδὲν ψεῦδος ἀπὸ γλώσσης αὐτοῦ οὐ μή ἐξέλθῃ·
\vs{22c}μάχαιρα γλῶσσα βασιλέως καὶ οὐ σαρκίνη, ὃς δʼ ἂν παραδοθῇ συντριβήσεται·
\vs{22d}ἐὰν γὰρ ὀξυνθῇ ὁ θυμὸς αὐτοῦ, σὺν νεύροις ἀνθρώπους ἀναλίσκει,
\vs{22e}καὶ ὀστᾶ ἀνθρώπων κατατρώγει, καὶ συγκαίει ὥσπερ φλὸξ, ὥστε ἄβρωτα εἶναι νεοσσοῖς ἀετῶν·
\vs{22f}τοὺς ἐμοὺς λόγους υἱὲ φοβήθητι, καὶ δεξάμενος αὐτοὺς μετανόει.]

Τάδε λέγει ὁ ἀνὴρ τοῖς πιστεύουσι Θεῷ, καὶ παύομαι.

\vs{22g}Ἀφρονέστατος γάρ εἰμι ἁπάντων ἀνθρώπων, καὶ φρόνησις ἀνθρώπων οὐκ ἔστιν ἐν ἐμοί.
\vs{22h}Θεὸς δεδίδαχέ με σοφίαν, καὶ γνῶσιν ἁγίων ἔγνωκα.
\vs{22i}Τίς ἀνέβη εἰς τὸν οὐρανὸν καὶ κατέβη; τίς συνήγαγεν ἀνέμους ἐν κόλπῳ; τίς συνέστρεψεν ὕδωρ ἐν ἱματίῳ; τίς ἐκράτησε πάντων τῶν ἄκρων τῆς γῆς; τί ὄνομα αὐτῷ; ἢ τί ὄνομα τοῖς τέκνοις αὐτοῦ;
\vs{22k}Πάντες γὰρ λόγοι Θεοῦ πεπυρωμένοι, ὑπερασπίζει δὲ αὐτὸς τῶν εὐλαβουμένων αὐτόν.
\vs{22l}Μὴ προσθῇς τοῖς λόγοις αὐτοῦ, ἵνα μὴ ἐλέγξῃ σε, καὶ ψευδὴς γένῃ.

\vs{22m}Δύο αἰτοῦμαι παρὰ σοῦ, μὴ ἀφέλῃς μου χάριν πρὸ τοῦ ἀποθανεῖν με.
\vs{22n}Μάταιον λόγον καὶ ψευδῆ μακράν μου ποίησον, πλοῦτον δὲ καὶ πενίαν μή μοι δῷς, σύνταξον δέ μοι τὰ δέοντα καὶ τὰ αὐτάρκη·
\vs{22o}Ἵνα μὴ πλησθεὶς ψευδὴς γένωμαι, καὶ εἴπω, τίς με ὁρᾷ; ἢ πενηθεὶς κλέψω, καὶ ὀμόσω τὸ ὄνομα τοῦ Θεοῦ.

\vs{22p}Μὴ παραδῷς οἰκέτην εἰς χεῖρας δεσπότου, μήποτε καταράσηταί σε καὶ ἀφανισθῇς.
\vs{22q}Ἔκγονον κακὸν πατέρα καταρᾶται, τὴς δὲ μητέρα οὐκ εὐλογεῖ.
\vs{22r}Ἔκγονον κακὸν δίκαιον ἑαυτὸν κρίνει, τὴν δʼ ἔξοδον αὐτοῦ οὐκ ἀπένιψεν.
\vs{22s}Ἔκγονον κακὸν ὑψηλοὺς ὀφθαλμοὺς ἔχει, τοῖς δὲ βλεφάροις αὐτοῦ ἐπαίρεται.
\vs{22t}Ἔκγονον κακὸν μαχαίρας τοὺς ὀδόντας ἔχει, καὶ τὰς μύλας, τομίδας, ὥστε ἀναλίσκειν καὶ κατεσθίειν τοὺς ταπεινοὺς ἀπὸ τῆς γῆς, καὶ τοὺς πένητας αὐτῶν ἐξ ἀνθρώπων.

\vs{23}Ταῦτα δὲ λέγω ὑμῖν τοῖς σοφοῖς ἐπιγινώσκειν· αἰδεῖσθαι πρόσωπον ἐν κρίσει οὐ καλόν.
\vs{24}Ὁ εἰπὼν τὸν ἀσεβῆ, δίκαιός ἐστιν, ἐπικατάρατος λαοῖς ἔσται καὶ μισητὸς εἰς ἔθνη.
\vs{25}Οἱ δὲ ἐλέγχοντες βελτίους φανοῦνται, ἐπʼ αὐτοὺς δὲ ἥξει εὐλογία·
\vs{26}χείλη δὲ φιλήσουσιν ἀποκρινόμενα λόγους ἀγαθούς.
\vs{27}Ἑτοίμαζε εἰς τὴν ἔξοδον τὰ ἔργα σου, καὶ παρασκευάζου εἰς τὸν ἀγρὸν, καὶ πορεύου κατόπισθέν μου, καὶ ἀνοικοδομήσεις τὸν οἶκόν σου.
\vs{28}Μὴ ἴσθι ψευδὴς μάρτυς ἐπὶ σὸν πολίτην, μηδὲ πλατύνου σοῖς χείλεσι.
\vs{29}Μὴ εἴπῃς, ὃν τρόπον ἐχρήσατό μοι, χρήσομαι αὐτῷ, τίσομαι δὲ αὐτὸν ἅ με ἠδίκησεν.
\vs{30}Ὥσπερ γεώργιον ἀνὴρ ἄφρων, καὶ ὥσπερ ἀμπελὼν ἄνθρωπος ἐνδεὴς φρενῶν.
\vs{31}Ἐὰν ἀφῇς αὐτὸν, χερσωθήσεται καὶ χορτομανήσει ὅλος, καὶ γίνεται ἐκλελειμμένος, οἱ δὲ φραγμοὶ τῶν λίθων αὐτοῦ κατασκάπτονται.
\vs{32}Ὕστερον ἐγὼ μετενόησα, ἐπέβλεψα τοῦ ἐκλέξασθαι παιδείαν.
\vs{33}Ὀλίγον νυστάζω, ὀλίγον δὲ καθυπνῶ, ὀλίγον δὲ ἐναγκαλίζομαι χερσὶ στήθη.
\vs{34}Ἐὰν δὲ τοῦτο ποιῇς, ἥξει προπορευομένη ἡ πενία σου, καὶ ἡ ἔνδειά σου ὥσπερ ἀγαθὸς δρομεύς.

\vs{35}Τῇ βδέλλῃ τρεῖς θυγατέρες ἦσαν ἀγαπήσει ἀγαπώμεναι, καὶ αἱ τρεῖς αὗται οὐκ ἐνεπίμπλασαν αὐτὴν, καὶ ἡ τετάρτη οὐκ ἠρκέσθη εἰπεῖν, ἱκανόν.
\vs{36}Ἄδης καὶ ἔρως γυναικὸς, καὶ γῆ οὐκ ἐμπιπλαμένη ὕδατος, καὶ ὕδωρ καὶ πῦρ οὐ μὴ εἴπωσιν, ἀρκεῖ.

\vs{37}Ὀφθαλμὸν καταγελῶντα πατρὸς, καὶ ἀτιμάζοντα γῆρας μητρὸς, ἐκκόψαισαν αὐτὸν κόρακες ἐκ τῶν φαράγγων, καὶ καταφάγοισαν αὐτὸν νεοσσοὶ ἀετῶν.
\vs{38}Τρία δέ ἐστιν ἀδύνατά μοι νοῆσαι, καὶ τὸ τέταρτον οὐκ ἐπιγινώσκω·
\vs{39}Ἴχνη ἀετοῦ πετομένου, καὶ ὁδοὺς ὄφεως ἐπὶ πέτρας, καὶ τρίβους νηὸς ποντοπορούσης, καὶ ὁδοὺς ἀνδρὸς ἐν νεότητι.
\vs{40}Τοιαύτη ὁδὸς γυναικὸς μοιχαλίδος, ἣ ὅτʼ ἂν πράξῃ ἀπονιψαμένη, οὐδέν φησι πεπραχέναι ἄτοπον.

\vs{41}Διὰ τριῶν σείεται ἡ γῆ, τὸ δὲ τέταρτον οὐ δύναται φέρειν·
\vs{42}Ἐὰν οἰκέτης βασιλεύσῃ, καὶ ἄφρων πλησθῇ σιτίων,
\vs{43}καὶ οἰκέτις ἐὰν ἐκβάλῃ τὴν ἑαυτῆς κυρίαν, καὶ μισητὴ γυνὴ ἐὰν τύχῃ ἀνδρὸς ἀγαθοῦ.

\vs{44}Τέσσαρα δὲ ἐλάχιστα ἐπὶ τῆς γῆς, ταῦτα δέ ἐστι σοφώτερα τῶν σοφῶν·
\vs{45}Οἱ μύρμηκες οἷς μή ἐστιν ἰσχὺς, καὶ ἑτοιμάζονται θέρους τὴν τροφήν·
\vs{46}Καὶ οἱ χοιρογρύλλιοι ἔθνος οὐκ ἰσχυρὸν, οἳ ἐποιήσαντο ἐν πέτραις τοὺς ἑαυτῶν οἴκους·
\vs{47}Ἀβασίλευτόν ἐστιν ἡ ἀκρὶς, καὶ στρατεύει ἀφʼ ἑνὸς κελεύσματος εὐτάκτως·
\vs{48}Καὶ καλαβώτης χερσὶν ἐρειδόμενος, καὶ εὐάλωτος ὢν, κατοικεῖ ἐν ὀχυρώμασι βασιλέων.

\vs{49}Τρία δέ ἐστιν ἃ εὐόδως πορεύεται, καὶ τέταρτον ὃ καλῶς διαβαίνει·
\vs{50}Σκύμνος λέοντος ἰσχυρότερος κτηνῶν, ὃς οὐκ ἀποστρέφεται, οὐδὲ καταπτήσσει κτῆνος·
\vs{51}Καὶ ἀλέκτωρ ἐμπεριπατῶν θηλείαις εὔψυχος, καὶ τράγος ἡγούμενος αἰπολίου, καὶ βασιλεὺς δημηγορῶν ἐν ἔθνει.

\vs{52}Ἐὰν πρόῃ σεαυτὸν ἐν εὐφροσύνῃ, καὶ ἐκτείνῃς τὴν χεῖρά σου μετὰ μάχης, ἀτιμασθήσῃ.
\vs{53}Ἄμελγε γάλα, καὶ ἔσται βούτυρον· ἐὰν δὲ ἐκπιέζῃς μυκτῆρας ἐξελεύσεται αἷμα, ἐὰν δὲ ἐξέλκῃς λόγους, ἐξελεύσονται κρίσεις καὶ μάχαι.

\vs{54}Οἱ ἐμοὶ λόγοι εἴρηνται ὑπὸ Θεοῦ, βασιλέως χρηματισμὸς, ὃν ἐπαίδευσεν ἡ μήτηρ αὐτοῦ.

\vs{55}Τί τέκνον τηρήσεις; τί; ῥήσεις Θεοῦ· πρωτογενὲς σοὶ λέγω υἱέ· τί τέκνον ἐμῆς κοιλίας; τί τέκνον ἐμῶν εὐχῶν;
\vs{56}Μὴ δῷς γυναιξὶ σὸν πλοῦτον, καὶ τὸν σὸν νοῦν καὶ βίον εἰς ὑστεροβουλίαν·
\vs{57}μετὰ βουλῆς πάντα ποίει, μετὰ βουλῆς οἰνοπότει. Οἱ δυνάσται θυμώδεις εἰσὶν, οἶνον δὲ μὴ πινέτωσαν,
\vs{58}ἵνα μὴ πιόντες ἐπιλάθωνται τῆς σοφίας, καὶ ὀρθὰ κρῖναι οὐ μὴ δύνωνται τοὺς ἀσθενεῖς.
\vs{59}Δίδοτε μέθην τοῖς ἐν λύπαις, καὶ οἶνον πίνειν τοῖς ἐν ὀδύναις,
\vs{60}ἵνα ἐπιλάθωνται τῆς πενίας, καὶ τῶν πόνων μὴ μνησθῶσιν ἔτι.
\vs{61}Ἄνοιγε σὸν στόμα λόγῳ Θεοῦ, καὶ κρίνε πάντας ὑγιῶς.
\vs{62}Ἄνοιγε σὸν στόμα καὶ κρίνε δικαίως, διάκρινε δὲ πένητα καὶ ἀσθενῆ.

\ch{25}
Αὗται αἱ παιδεῖαι Σαλωμῶντος αἱ ἀδιάκριτοι, ἃς ἐξεγράψαντο οἱ φίλοι Ἐζεκίου τοῦ βασιλέως τῆς Ἰουδαίας.

\vs{2}Δόξα Θεοῦ κρύπτει λόγον, δόξα δὲ βασιλέως τιμᾷ πράγματα.
\vs{3}Οὐρανὸς ὑψηλὸς, γῆ δὲ βαθεῖα, καρδία δὲ βασιλέως ἀνεξέλεγκτος.
\vs{4}Τύπτε ἀδόκιμον ἀργύριον, καὶ καθαρισθήσεται καθαρὸν ἅπαν.
\vs{5}Κτεῖνε ἀσεβεῖς ἐκ προσώπου βασιλέως, καὶ κατορθώσει ἐν δικαιοσύνῃ ὁ θρόνος αὐτοῦ.

\vs{6}Μὴ ἀλαζονεύου ἐνώπιον βασιλέως, μηδὲ ἐν τόποις δυναστῶν ὑφίστασο·
\vs{7}Κρεῖσσον γάρ σοι τὸ ῥηθῆναι, ἀνάβαινε πρὸς μὲ, ἢ ταπεινῶσαί σε ἐν προσώπῳ δυνάστου· ἃ εἶδον οἱ ὀφθαλμοί σου λέγε.

\vs{8}Μὴ πρόσπιπτε εἰς μάχην ταχέως, ἵνα μὴ μεταμεληθῇς ἐπʼ ἐσχάτων· ἡνίκα ἄν σε ὀνειδίσῃ ὁ σὸς φίλος,
\vs{9}ἀναχώρει εἰς τὰ ὀπίσω· μὴ καταφρόνει,
\vs{10}μή σε ὀνειδίσῃ μὲν ὁ φίλος, ἡ δὲ μάχη σου καὶ ἡ ἔχθρα οὐκ ἀπέσται, ἀλλὰ ἔσται σοι ἴση θανάτῳ·
\vs{10a}χάρις καὶ φιλία ἐλευθεροῖ, ἃς τήρησον σεαυτῷ, ἵνα μὴ ἐπονείδιστος γένῃ, ἀλλὰ φύλαξον τὰς ὁδούς σου εὐσυναλλάκτως.

\vs{11}Μῆλον χρυσοῦν ἐν ὁρμίσκῳ σαρδίου, οὕτως εἰπεῖν λόγον.
\vs{12}Εἰς ἐνώτιον χρυσοῦν καὶ σάρδιον πολυτελὲς δέδεται, λόγος σοφὸς εἰς εὐήκοον οὖς.
\vs{13}Ὥσπερ ἔξοδος χιόνος ἐν ἀμητῷ κατὰ καῦμα ὠφελεῖ, οὕτως ἄγγελος πιστὸς τοὺς ἀποστείλαντας αὐτόν· ψυχὰς γὰρ τῶν αὐτῷ χρωμένων ὠφελεῖ.

\vs{14}Ὥσπερ ἄνεμοι καὶ νέφη καὶ ὑετοὶ, ἐπιφανέστατα, οὕτως ὁ καυχώμενος ἐπὶ δόσει ψευδεῖ.
\vs{15}Ἐν μακροθυμίᾳ εὐοδία βασιλεῦσι, γλῶσσα δὲ μαλακὴ συντρίβει ὀστᾶ.
\vs{16}Μέλι εὑρὼν φάγε τὸ ἱκανὸν, μή ποτε πλησθεὶς ἐξεμέσῃς.
\vs{17}Σπάνιον εἴσαγε σὸν πόδα πρὸς σεαυτοῦ φίλον, μή ποτε πλησθείς σου μισήσῃ σε.
\vs{18}Ῥόπαλον καὶ μάχαιρα καὶ τόξευμα ἀκιδωτὸν, οὕτως καὶ ἀνὴρ ὁ καταμαρτυρῶν τοῦ φίλου αὐτοῦ μαρτυρίαν ψευδῆ.
\vs{19}Οδὸς κακοῦ καὶ ποὺς παρανόμου ὀλεῖται ἐν ἡμέρᾳ κακῇ.

\vs{20}Ὥσπερ ὄξος ἕλκει ἀσύμφορον, οὕτως προσπεσὸν πάθος ἐν σώματι καρδίαν λυπεῖ·
\vs{20a}ὥσπερ σὴς ἐν ἱματίῳ καὶ σκώληξ ξύλῳ, οὕτως λύπη ἀνδρὸς βλάπτει καρδίαν.

\vs{21}Ἐὰν πεινᾷ ὁ ἐχθρός σου, ψώμιζε αὐτὸν, ἐὰν διψᾷ, πότιζε αὐτόν·
\vs{22}Τοῦτο γὰρ ποιῶν ἄνθρακας πυρὸς σωρεύσεις ἐπὶ τὴν κεφαλὴν αὐτοῦ, ὁ δὲ Κύριος ἀνταποδώσει σοι ἀγαθά.
\vs{23}Ἄνεμος Βορέας ἐξεγείρει νέφη, πρόσωπον δὲ ἀναιδὲς γλῶσσαν ἐρεθίζει·
\vs{24}Κρεῖσσον οἰκεῖν ἐπὶ γωνίας δώματος, ἢ μετὰ γυναικὸς λοιδόρου ἐν οἰκίᾳ κοινῇ.
\vs{25}Ὥσπερ ὕδωρ ψυχρὸν ψυχῇ διψώσῃ προσηνὲς, οὕτως ἀγγελία ἀγαθὴ ἐκ γῆς μακρόθεν.
\vs{26}Ὥσπερ εἴ τις πηγὴν φράσσοι καὶ ὕδατος ἔξοδον λυμαίνοιτο, οὕτως ἄκοσμον δίκαιον πεπτωκέναι ἐνώπιον ἀσεβοῦς.
\vs{27}Ἐσθίειν μέλι πολὺ οὐ καλὸν, τιμᾷν δὲ χρὴ λόγους ἐνδόξους.
\vs{28}Ὥσπερ πόλις τὰ τείχη καταβεβλημένη καὶ ἀτείχιστος, οὕτως ἀνὴρ ὃς οὐ μετὰ βουλῆς τι πράσσει.

\ch{26}
Ὥσπερ δρόσος ἐν ἀμητῷ, καὶ ὥσπερ ὑετὸς ἐν θέρει, οὕτως οὐκ ἔστιν ἄφρουι τιμή.
\vs{2}Ὥσπερ ὄρνεα πέταται καὶ στρουθοί, οὕτως ἀρὰ ματαία οὐκ ἐπελεύσεται οὐδενί.
\vs{3}Ὥσπερ μάστιξ ἵππῳ καὶ κέντρον ὄνῳ, οὕτως ῥάβδος ἔθνει παρανόμῳ.
\vs{4}Μὴ ἀποκρίνου ἄφρονι πρὸς τὴν ἐκείνου ἀφροσύνην, ἵνα μὴ ὅμοιος γένῃ αὐτῷ.
\vs{5}Ἀλλὰ ἀποκρίνου ἄφρονι κατὰ τὴν ἀφροσύνην αὐτοῦ, ἵνα μὴ φαίνηται σοφὸς παρʼ ἑαυτῷ.
\vs{6}Ἐκ τῶν ὁδῶν ἑαυτοῦ ὄνειδος ποιεῖται ὁ ἀποστείλας διʼ ἀγγέλου ἄφρονος λόγον.
\vs{7}Ἄφελοῦ πορείαν σκελῶν, καὶ παρανομίαν ἐκ στόματος ἀφρόνων.
\vs{8}Ὃς ἀποδεσμεύει λίθον ἐν σφενδόνῃ, ὅμοιός ἐστι τῷ διδόντι ἄφρονι δόξαν.
\vs{9}Ἄκανθαι φύονται ἐν χειρὶ μεθύσου, δουλεία δὲ ἐν χειρὶ τῶν ἀφρόνων.
\vs{10}Πολλὰ χειμάζεται πᾶσα σὰρξ ἀφρόνων, συντρίβεται γὰρ ἡ ἔκστασις αὐτῶν.
\vs{11}Ὥσπερ κύων ὅταν ἐπέλθῃ ἐπὶ τὸν ἑαυτοῦ ἔμετον καὶ μισητὸς γένηται, οὕτως ἄφρων τῇ ἑαυτοῦ κακίᾳ ἀναστρέψας ἐπὶ τὴν ἑαυτοῦ ἁμαρτίαν·
\vs{11a}ἔστιν αἰσχύνη ἐπάγουσα ἁμαρτίαν, καὶ ἐστιν αἰσχύνη δόξα καὶ χάρις.
\vs{12}Εἶδον ἄνδρα δόξαντα παρʼ αὐτῷ σοφὸν εἶναι, ἐλπίδα μέντοι ἔσχε μᾶλλον ἄφρων αὐτοῦ.
\vs{13}Λέγει ὀκνηρὸς ἀποστελλόμενος εἰς ὁδὸν, λέων ἐν ταῖς ἐν δὲ ταῖς πλατείαις φονευταί.

\vs{14}Ὥσπερ θύρα στρέφεται ἐπὶ τοῦ στρόφιγγος, οὕτως ὀκνηρὸς ἐπὶ τῆς κλίνης αὐτοῦ.
\vs{15}Κρύψας ὀκνηρὸς τὴν χεῖρα ἐν τῷ κόλπῳ αὐτοῦ, οὐ δυνήσεται ἐπενεγκεῖν ἐπὶ στόμα.
\vs{16}Σοφώτερος ἑαυτῷ ὀκνηρὸς φαίνεται, τοῦ ἐν πλησμονῇ ἀποκομίζοντος ἀγγελίαν.

\vs{17}Ὥσπερ ὁ κρατῶν κέρκου κυνὸς, οὕτως ὁ προεστὼς ἀλλοτρίας κρίσεως.
\vs{18}Ὥσπερ οἱ ἰώμενοι προβάλλουσι λόγους εἰς ἀνθρώπους, ὁ δὲ ἀπαντήσας τῷ λόγῳ πρῶτος ὑποσκελισθήσεται·
\vs{19}Οὕτως πάντες οἱ ἐνεδρεύοντες τοὺς ἑαυτῶν φίλους, ὅταν δὲ ὁραθῶσι, λέγουσιν, ὅτι παίζων ἔπραξα.
\vs{20}Ἐν πολλοῖς ξύλοις θάλλει πῦρ, ὅπου δὲ οὐκ ἔστι δίθυμος, ἡσυχάζει μάχη.
\vs{21}Ἐσχάρα ἄνθραξι καὶ ξύλα πυρὶ, ἀνὴρ δὲ λοίδορος εἰς ταραχὴν μάχης.
\vs{22}Λόγοι κερκώπων μαλακοὶ, οὗτοι δὲ τύπτουσιν εἰς ταμιεῖα σπλάγχνων.

\vs{23}Ἀργύριον διδόμενον μετὰ δόλου, ὥσπερ ὄστρακον ἡγητέον· χείλη λεῖα καρδίαν καλύπτει λυπηράν.
\vs{24}Χείλεσι πάντα ἐπινεύει ἀποκλαιόμενος ἐχθρὸς, ἐν δὲ τῇ καρδίᾳ τεκταίνεται δόλους.
\vs{25}Ἐάν σου δέηται ὁ ἐχθρὸς μεγάλῃ τῇ φωνῇ, μὴ πεισθῇς, ἑπτὰ γάρ πονηρίαι ἐν τῇ ψυχῇ αὐτοῦ.
\vs{26}Ὁ κρύπτων ἔχθραν συνίστησι δόλον, ἐκκαλύπτει δὲ τὰς ἑαυτοῦ ἁμαρτίας εὔγνωστος ἐν συνεδρίοις.
\vs{27}Ὁ ὀρύσσων βόθρον τῷ πλησίον, ἐμπεσεῖται εἰς αὐτόν· ὁ δὲ κυλίων λίθον, ἐφʼ ἑαυτὸν κυλίει.
\vs{28}Γλῶσσα ψευδὴς μισεῖ ἀλήθειαν, στόμα δὲ ἄστεγον ποιεῖ ἀκαταστασίας.

\ch{27}
Μὴ καυχῶ τὰ εἰς αὔριον, οὐ γὰρ γινώσκεις τί τέξεται ἡ ἐπιοῦσα.
\vs{2}Ἐγκωμιαζέτω σε ὁ πέλας καὶ μὴ τὸ σὸν στόμα, ἀλλότριος καὶ μὴ τὰ σὰ χείλη.
\vs{3}Βαρὺ λίθος καὶ δυσβάστακτον ἄμμος, ὀργὴ δὲ ἄφρονος βαρυτέρα ἀμφοτέρων.
\vs{4}Ἀνελεήμων θυμὸς καὶ ὀξεῖα ὀργὴ, ἀλλʼ οὐδὲν ὑφίσταται ζῆλος.
\vs{5}Κρείσσους ἔλεγχοι ἀποκεκαλυμμένοι κρυπτομένης φιλίας.
\vs{6}Ἀξιοπιστότερά ἐστι τραύματα φίλου, ἢ ἑκούσια φιλήματα ἐχθροῦ.

\vs{7}Ψυχὴ ἐν πλησμονῇ οὖσα κηρίοις ἐμπαίζει, ψυχῇ δὲ ἐνδεεῖ καὶ τὰ πικρὰ γλυκέα φαίνεται.
\vs{8}Ὥσπερ ὅταν ὄρνεον καταπετασθῇ ἐκ τῆς ἰδίας νοσσιᾶς, οὕτως ἄνθρωπος δουλοῦται ὅταν ἀποξενωθῇ ἐκ τῶν ἰδίων τόπων.
\vs{9}Μύροις καὶ οἴνοις καὶ θυμιάμασι τέρπεται καρδία, καταῤῥήγνυται δὲ ὑπὸ συμπτωμάτων ψυχή.

\vs{10}Φίλον σὸν ἢ φίλον πατρῷον μὴ ἐγκαταλίπῃς, εἰς δὲ τὸν οἶκον τοῦ ἀδελφοῦ σου μὴ εἰσέλθῃς ἀτυχῶν· κρείσσων φίλος ἐγγὺς, ἢ ἀδελφὸς μακρὰν οἰκῶν.
\vs{11}Σοφὸς γίνου υἱὲ, ἵνα σου εὐφραίνηται ἡ καρδία, καὶ ἀπόστρεψον ἀπὸ σοῦ ἐπονειδίστους λόγους.
\vs{12}Πανοῦργος κακῶν ἐπερχομένων ἀπεκρύβη, ἄφρονες δὲ ἐπελθόντες ζημίαν τίσουσιν.
\vs{13}Ἀφελοῦ τὸ ἱμάτιον αὐτοῦ, παρῆλθε γὰρ ὑβριστὴς, ὅστις τὰ ἀλλότρια λυμαίνεται.
\vs{14}Ὃς ἂν εὐλογῇ θίλον τοπρωῒ μεγάλῃ τῇ φωνῇ, καταρωμένου οὐδὲν διαφέρειν δόξει.

\vs{15}Σταγόνες ἐκβάλλουσιν ἄνθρωπον ἐν ἡμέρᾳ χειμερινῇ ἐκ τοῦ οἴκου αὐτοῦ, ὡσαύτως καὶ γυνὴ λοίδορος ἐκ τοῦ ἰδίου οἴκου.
\vs{16}Βορέας σκληρὸς ἄνεμος, ὀνόματι δὲ ἐπιδέξιος καλεῖται.
\vs{17}Σίδηρος σίδηρον ὀξύνει, ἀνὴρ δὲ παροξύνει πρόσωπον ἑταίρου.
\vs{18}Ὃς φυτεύει συκὴν φάγεται τοὺς καρποὺς αὐτῆς, ὃς δὲ φυλάσσει τὸν ἑαυτοῦ κύριον τιμηθήσεται.
\vs{19}Ὥσπερ οὐχ ὅμοια πρόσωπα προσώποις, οὕτως οὐδὲ αἱ διάνοιαι τῶν ἀνθρώπων.
\vs{20}Ἅδης καὶ ἀπώλεια οὐκ ἐμπίμπλανται, ὡσαύτως καὶ οἱ ὀφθαλμοὶ τῶν ἀνθρώπων ἄπληστοι·
\vs{20a}βδέλυγμα Κυρίῳ στηρίζων ὀφθαλμὸν, καὶ οἱ ἀπαίδευτοι ἀκρατεῖς γλώσσῃ.
\vs{21}Δοκίμιον ἀργυρίῳ καὶ χρυσῷ πύρωσις, ἀνὴρ δὲ δοκιμάζεται διὰ στόματος ἐγκωμιαζόντων αὐτόν.
\vs{21a}καρδία ἀνόμου ἐκζητεῖ κακὰ, καρδία δὲ εὐθὴς ζητεῖ γνῶσιν.
\vs{22}Ἐὰν μαστιγοῖς ἄφρονα ἐν μέσῳ συνεδρίου ἀτιμάζων, οὐ μὴ περιέλῃς τὴν ἀφροσύνην αὐτοῦ.

\vs{23}Γνωστῶς ἐπιγνώσῃ ψυχὰς ποιμνίου σου, καὶ ἐπιστήσεις καρδίαν σου σαῖς ἀγέλαις.
\vs{24}Ὅτι οὐκ εἰς τὸν αἰῶνα ἀνδρὶ κράτος καὶ ἰσχὺς, οὐδὲ παραδίδωσιν ἐκ γενεᾶς εἰς γενεάν.
\vs{25}Ἐπιμελοῦ τῶν ἐν τῷ πεδίῳ χλωρῶν, καὶ κερεῖς πόαν, καὶ σύναγε χόρτον ὀρεινὸν,
\vs{26}ἵνα ἔχῃς πρόβατα εἰς ἱματισμόν· τίμα πεδίον, ἵνα ὠσί σοι ἄρνες.
\vs{27}Υἱὲ, παρʼ ἐμοῦ ἔχεις ῥήσεις ἰσχυρὰς εἰς τὴν ζωήν σου, καὶ εἰς τὴν ζωὴν σῶν θεραπόντων.

\ch{28}
Φεύγει ἀσεβὴς μηδενὸς διώκοντος, δίκαιος δὲ ὥσπερ λέων πέποιθε.
\vs{2}Διʼ ἁμαρτίας ἀσεβῶν κρίσεις ἐγείρονται, ἀνὴρ δὲ πανοῦργος κατασβέσει αὐτάς.
\vs{3}Ἀνδρεῖος ἐν ἀσεβείαις συκοφαντεῖ πτωχούς· ὥσπερ ὑετὸς λάβρος καὶ ἀνωφελὴς,
\vs{4}οὕτως οἱ ἐγκαταλείποντες τὸν νόμον ἐγκωμιάζουσιν ἀσέβειαν· οἱ δὲ ἀγαπῶντες τὸν νόμον, περιβάλλουσιν ἑαυτοῖς τεῖχος.
\vs{5}Ἄνδρες κακοὶ οὐ συνήσουσι κρίμα, οἱ δὲ ζητοῦντες τὸν Κύριον συνήσουσιν ἐν παντί.

\vs{6}Κρείσσων πτωχὸς πορευόμενος ἐν ἀληθείᾳ, πλουσίου ψευδοῦς.
\vs{7}Φυλάσσει νόμον υἱὸς συνετὸς, ὃς δὲ ποιμαίνει ἀσωτίαν ἀτιμάζει πατέρα.
\vs{8}Ὁ πληθύνων τὸν πλοῦτον αὐτοῦ μετὰ τόκων καὶ πλεονασμῶν, τῷ ἐλεῶντι πτωχοὺς συνάγει αὐτόν.
\vs{9}Ὁ ἐκκλίνων τὸ οὖς αὐτοῦ μὴ εἰσακοῦσαι νόμου, καὶ αὐτὸς τὴν προσευχὴν αὐτοῦ ἐβδέλυκται.

\vs{10}Ὃς πλανᾷ εὐθεῖς ἐν ὁδῷ κακῇ, εἰς διαφθορὰν αὐτὸς ἐμπεσεῖται· οἱ δὲ ἄνομοι διελεύσονται ἀγαθὰ, καὶ οὐκ εἰσελεύσονται εἰς αὐτά.
\vs{11}Σοφὸς παρʼ ἑαυτῷ ἀνὴρ πλούσιος, πένης δὲ νοήμων καταγνώσεται αὐτοῦ.
\vs{12}Διὰ βοήθειαν δικαίων πολλὴ γίνεται δόξα, ἐν δὲ τόποις ἀσεβῶν ἁλίσκονται ἄνθρωποι.

\vs{13}Ὁ ἐπικαλύπτων ἀσέβειαν ἑαυτοῦ οὐκ εὐοδωθήσεται, ὁ δὲ ἐξηγούμενος ἐλέγχους ἀγαπηθήσεται.
\vs{14}Μακάριος ἀνὴρ ὃς καταπτήσσει πάντα διʼ εὐλάβειαν, ὁ δὲ σκληρὸς τὴν καρδίαν ἐμπεσεῖται κακοῖς.
\vs{15}Λέων πεινῶν καὶ λύκος διψῶν, ὃς τυραννεῖ, πτωχὸς ὢν, ἔθνους πενιχροῦ.
\vs{16}Βασιλεὺς ἐνδεὴς προσόδων μέγας συκοφάντης, ὁ δὲ μισῶν ἀδικίαν μακρὸν χρόνον ζήσεται.

\vs{17}Ἄνδρα τὸν ἐν αἰτίᾳ φόνου ὁ ἐγγυώμενος, φυγὰς ἔσται καὶ οὐκ ἐν ἀσφαλείᾳ·
\vs{17a}παίδευε υἱὸν καὶ ἀγαπήσει σε, καὶ δώσει κόσμον τῇ σῇ ψυχῇ, οὐ μὴ ὑπακούσει ἔθνει παρανόμῳ.
\vs{18}Ὁ πορευόμενος δικαίως βεβοήθηται, ὁ δὲ σκολιαῖς ὁδοῖς πορευόμενος ἐμπλακήσεται.
\vs{19}Ὁ ἐργαζόμενος τὴν ἑαυτοῦ γῆν πλησθήσεται ἄρτων, ὁ δὲ διώκων σχολὴν πλησθήσεται πενίας.

\vs{20}Ἀνὴρ ἀξιόπιστος πολλὰ εὐλογηθήσεται, ὁ δὲ κακὸς οὐκ ἀτιμώρητος ἔσται.
\vs{21}Ὃς οὐκ αἰσχύνεται πρόσωπα δικαίων, οὐκ ἀγαθὸς, ὁ τοιοῦτος ψωμοῦ ἄρτου ἀποδώσεται ἄνδρα.
\vs{22}Σπεύδει πλουτεῖν ἀνὴρ βάσκανος, καὶ οὐκ οἶδεν ὅτι ἐλεήμων κρατήσει αὐτοῦ.

\vs{23}Ὁ ἐλέγχων ἀνθρώπου ὁδοὺς, χάριτας ἕξει μᾶλλον τοῦ γλωσσοχαριτοῦντος.
\vs{24}Ὃς ἀποβάλλεται πατέρα ἢ μητέρα, καὶ δοκεῖ μὴ ἁμαρτάνειν, οὗτος κοινωνός ἐστιν ἀνδρὸς ἀσεβοῦς.
\vs{25}Ἄπιστος ἀνὴρ κρίνει εἰκῆ, ὃς δὲ πέποιθεν ἐπὶ Κύριον ἐν ἐπιμελείᾳ ἔσται.
\vs{26}Ὃς πέποιθε θρασείᾳ καρδίᾳ, ὁ τοιοῦτος ἄφρων, ὃς δὲ πορεύεται σοφίᾳ σωθήσεται.
\vs{27}Ὃς δίδωσι πτωχοῖς οὐκ ἐνδεηθήσεται, ὃς δὲ ἀποστρέφει τὸν ὀφθαλμὸν αὐτοῦ ἐν πολλῇ ἀπορίᾳ ἔσται.
\vs{28}Ἐν τόποις ἀσεβῶν στένουσι δίκαιοι, ἐν δὲ τῇ ἐκείνων ἀπωλείᾳ πληθυνθήσονται δίκαιοι.

\ch{29}
Κρείσσων ἀνὴρ ἐλέγχων ἀνδρὸς σκληροτραχήλου, ἐξαπίνης γὰρ φλεγομένου αὐτοῦ οὐκ ἔστιν ἴασις.
\vs{2}Ἐγκωμιαζομένων δικαίων εὐφρανθήσονται λαοὶ, ἀρχόντων δὲ ἀσεβῶν στένουσιν ἄνδρες.
\vs{3}Ἀνδρὸς φιλοῦντος σοφίαν εὐφραίνεται πατὴρ αὐτοῦ, ὃς δὲ ποιμαίνει πόρνας ἀπολεῖ πλοῦτον.
\vs{4}Βασιλεὺς δίκαιος ἀνίστησι χώραν, ἀνὴρ δὲ παράνομος κατασκάπτει.
\vs{5}Ὃς παρασκευάζεται ἐπὶ πρόσωπον τοῦ ἑαυτοῦ φιλου δίκτυον, περιβάλλει αὐτὸ τοῖς ἑαυτοῦ ποσίν.
\vs{6}Ἁμαρτάνοντι ἀνδρὶ μεγάλη παγὶς, δίκαιος δὲ ἐν χαρᾷ καὶ ἐν εὐφροσύνῃ ἔσται.
\vs{7}Ἐπίσταται δίκαιος κρίνειν πενιχροῖς, ὁ δὲ ἀσεβὴς οὐ νοεῖ γνῶσιν, καὶ πτωχῷ οὐχ ὑπάρχει νοῦς ἐπιγνώμων.

\vs{8}Ἄνδρες ἄνομοι ἐξέκαυσαν πόλιν, σοφοὶ δὲ ἀπέστρεψαν ὀργήν.
\vs{9}Ἀνὴρ σοφὸς κρινεῖ ἔθνη, ἀνὴρ δὲ φαῦλος ὀργιζόμενος καταγελᾶται καὶ οὐ καταπτήσσει.
\vs{10}Ἄνδρες αἱμάτων μέτοχοι μισοῦσιν ὅσιον, οἱ δὲ εὐθεῖς ἐκζητήσουσι ψυχὴν αὐτοῦ.
\vs{11}Ὅλον τὸν θυμὸν αὐτοῦ ἐκφέρει ἄφρων, σοφὸς δὲ ταμιεύεται κατὰ μέρος.
\vs{12}Βασιλέως ὑπακούοντος λόγον ἄδικον, πάντες οἱ ὑπʼ αὐτὸν παράνομοι.
\vs{13}Δανιστοῦ καὶ χρεωφειλέτου ἀλλήλοις συνελθόντων, ἐπισκοπὴν ἀμφοτέρων ποιεῖται ὁ Κύριος.
\vs{14}Βασιλέως ἐν ἀληθείᾳ κρίνοντος πτωχοὺς, ὁ θρόνος αὐτοῦ εἰς μαρτύριον κατασταθήσεται.
\vs{15}Πληγαὶ καὶ ἔλεγχοι διδόασι σοφίαν, παῖς δὲ πλανώμενος αἰσχύνει γονεῖς αὐτοῦ.
\vs{16}Πολλῶν ὄντων ἀσεβῶν πολλαὶγίνονται ἁμαρτίαι, οἱ δὲ δίκαιοι ἐκείνων πιπτόντων κατάφοβοι γίνονται.

\vs{17}Παίδευε υἱόν σου, καὶ ἀναπαύσει σε, καὶ δώσει κόσμον τῇ ψυχῇ σου.
\vs{18}Οὐ μὴ ὑπάρξῃ ἐξηγητὴς ἔθνει παρανόμῳ, ὁ δὲ φυλάσσων τὸν νόμον μακαριστός.
\vs{19}Λόγοις οὐ παιδευθήσεται οἰκέτης σκληρός· ἐὰν γὰρ καὶ νοήσῃ, ἀλλʼ οὐχ ὑπακούσεται.
\vs{20}Ἐὰν ἴδῃς ἄνδρα ταχὺν ἐν λόγοις, γίνωσκε ὅτι ἐλπίδα ἔχει μᾶλλον ὁ ἄφρων αὐτοῦ.
\vs{21}Ὃς κατασπαταλᾷ ἐκ παιδὸς, οἰκέτης ἔσται, ἔσχατον δὲ ὀδυνηθήσεται ἐφʼ ἑαυτῷ·
\vs{22}Ἀνὴρ θυμώδης ἐγείρει νεῖκος, ἀνὴρ δὲ ὀργίλος ἐξώρυξεν ἁμαρτίαν.
\vs{23}Ὕβρις ἄνδρα ταπεινοῖ, τοὺς δὲ ταπεινόφρονας ἐρείδει δόξῃ Κύριος.

\vs{24}Ὃς μερίζεται κλέπτῃ, μισεῖ τὴν ἑαυτοῦ ψυχήν· ἐὰν δὲ ὅρκου προτεθέντος ἀκούσαντες μὴ ἀναγγείλωσι,
\vs{25}φοβηθέντες καὶ αἰσχυνθέντες ἀνθρώπους ὑπεσκελίσθησαν, ὁ δὲ πεποιθὼς ἐπὶ Κυρίῳ εὐφρανθήσεται· ἀσέβεια ἀνδρὶ δίδωσι σφάλμα, ὃς δὲ πέποιθεν ἐπὶ τῷ δεσπότῃ σωθήσεται.
\vs{26}Πολλοὶ θεραπεύουσι πρόσωπα ἡγουμένων, παρὰ δὲ Κυρίου γίνεται τὸ δίκαιον ἀνδρί.
\vs{27}Βδέλυγμα δίκαιος ἀνὴρ ἀνδρὶ ἀδίκῳ, βδέλυγμα δὲ ἀνόμῳ κατευθύνουσα ὁδός.


\ch{31}
\vs{10}Γυναῖκα ἀνδρείαν τίς εὑρήσει; τιμιωτέρα δέ ἐστι λίθων πολυτελῶν ἡ τοιαύτη.
\vs{11}Θαρσεῖ ἐπʼ αὐτῇ ἡ καρδία τοῦ ἀνδρὸς αὐτῆς· ἡ τοιαύτη καλῶν σκύλων οὐκ ἀπορήσει.
\vs{12}Ἐνεργεῖ γὰρ τῷ ἀνδρὶ εἰς ἀγαθὰ πάντα τὸν βίον.
\vs{13}Μηρυομένη ἔρια καὶ λινὸν, ἐποίησεν εὔχρηστον ταῖς χερσὶν αὐτῆς.
\vs{14}Ἐγένετο ὡσεὶ ναῦς ἐμπορευομένη μακρόθεν, συνάγει δὲ αὕτη τὸν βίον.
\vs{15}Καὶ ἀνίσταται ἐκ νυκτῶν, καὶ ἔδωκε βρώματα τῷ οἴκῳ, καὶ ἔργα ταῖς θεραπαίναις.
\vs{16}Θεωρήσασα γεώργιον ἐπρίατο, ἀπὸ δὲ καρπῶν χειρῶν αὐτῆς κατεφύτευσεν κτῆμα.
\vs{17}Ἀναζωσαμένη ἰσχυρῶς τὴν ὀσφῦν αὐτῆς ἤρεισε τοὺς βραχίονας αὐτῆς εἰς ἔργον.
\vs{18}Καὶ ἐγεύσατο ὅτι καλόν ἐστι τὸ ἐργάζεσθαι, καὶ οὐκ ἀποσβέννυται ὁ λύχνος αὐτῆς ὅλην τὴν νύκτα.
\vs{19}τοὺς πήχεις αὐτῆς ἐκτείνει ἐπὶ τὰ συμφέροντα, τὰς δὲ χεῖρας αὐτῆς ἐρείδει εἰς ἄτρακτον.
\vs{20}Χεῖρας δὲ αὐτῆς διήνοιξε πένητι, καρπὸν δὲ ἐξέτεινεν πτωχῷ.

\vs{21}Οὐ φροντίζει τῶν ἐν οἴκῳ ὁ ἀνὴρ αὐτῆς ὅταν που χρονίζῃ, πάντες γὰρ οἱ παρʼ αὐτῆς ἐνδεδυμένοι εἰαί.
\vs{22}Δισσὰς χλαίνας ἐποίησε τῷ ἀνδρὶ αὐτῆς, ἐκ δὲ βύσσου καὶ πορφύρας ἑαυτῇ ἐνδύματα.
\vs{23}Περίβλεπτος δὲ γίνεται ὁ ἀνὴρ αὐτῆς ἐν πύλαις, ἡνίκα ἂν καθίσῃ ἐν συνεδρίῳ μετὰ τῶν γερόντων κατοίκων τῆς γῆς.
\vs{24}Σινδόνας ἐποίησε καὶ ἀπέδοτο περιζώματα τοῖς Χαναναίοις. στόμα αὐτῆς διήνοιξε προσεχόντως καὶ ἐννόμως, καὶ τάξιν ἐστείλατο τῇ γλώσσῃ αὐτῆς.
\vs{25}Ἰσχὺν καὶ εὐπρέπειαν ἐνεδύσατο, καὶ εὐφράνθη ἐν ἡμέραις ἐσχάταις.
\vs{26}Στεγναὶ διατριβαὶ οἴκων αὐτῆς, σῖτα δὲ ὀκνηρὰ οὐκ ἔφαγεν.
\vs{27}Τὸ στόμα δὲ ἀνοίγει σοφῶς καὶ νομοθέσμως. Ἡ δὲ ἐλεημοσύνη αὐτῆς
\vs{28}ἀνέστησε τὰ τέκνα αὐτῆς καὶ ἐπλούτησαν, καὶ ὁ ἀνὴρ αὐτῆς ᾔνεσεν αὐτήν.
\vs{29}Πολλαὶ θυγατέρες ἐκτήσαντο πλοῦτον, πολλαὶ ἐποίησαν δύναμιν· σὺ δὲ ὑπέρκεισαι, ὑπερῇρας πάσας.
\vs{30}Ψευδεῖς ἀρέσκειαι, καὶ μάταιον κάλλος γυναικος· γυνὴ γὰρ συνετὴ εὐλογεῖται, φόβον δὲ Κυρίου αὕτη αἰνείτω.
\vs{31}Δότε αὐτῇ ἀπὸ καρπῶν χειλέων αὐτῆς, καὶ αἰνείσθω ἐν πύλαις ὁ ἀνὴρ αὐτῆς.


\def\book{ΠΡΟΣΕΥΧΗ ΜΑΝΑΣΣΗ ΥΙΟΥ ΕΖΕΚΙΟΥ}
\biblebook{ΠΡΟΣΕΥΧΗ ΜΑΝΑΣΣΗ ΥΙΟΥ ΕΖΕΚΙΟΥ}

 
\lettrine[lines=2, loversize=0.2, nindent=0em, findent=.25em]{\textcolor{bookheadingcolor}{Κ}}{ΥΡΙΕ} παντοκράτωπ, ὁ Θεὸς τῶν πατέρων ἡμῶν τοῦʼ Αβραὰμ καὶ Ἰσαὰκ καὶ Ἰακὼβ καὶ τοῦ σπέρματος αὐτῶν τοῦ δικαίου.
\vs{2}Ὁ ποιήσας τὸν οὐρανὸν καὶ τὴν γῆν σὺν πάντι τῷ κόσμῳ αὐτῶν.
\vs{3}Ὁ πεδήσας τὴν θάλασσαν τῷ λόγῳ τοῦ προστάγματός σου, ὁ κλείσας τὴν ἄβυσσον καὶ σφραγισάμενος αὐτὴν τῷ φοβερῷ καὶ ἐνδόξῳ ὀνόματί σου·
\vs{4}ὃν πάντα φρίσσει καὶ τρέμει ἀπὸ προσώπου δυνάμεώς σου,
\vs{5}ὅτι ἄστεκτος ἡ μεγαλοπρέπεια τῆς δόξης σου, καὶ ἀνυπόστατος ἡ ὀργὴ τῆς ἐπὶ ἁμαρτωλοὺς ἀπειλῆς σου·
\vs{6}ἀμέτρητόν τε καὶ ἀνεξιχνίαστον τὸ ἔλεος τῆς ἐπαγγελίας σου·
\vs{7}σὺ γὰρ εἶ Κύριος ὕψιστος, εὔσπλαγχνος, μακρόθυμος καὶ πολυέλεος, μετανοῶν ἐπὶ κακίαις ἀνθρώπων. Σὺ, Κύριε, κατὰ τὸ πλῆθος τῆς χρηστότητός σου ἐπηγγείλω μετάνοιαν καὶ ἄφεσιν τοῖς ἡμαρτηκόσιν σοι, καὶ τῷ πλήθει τῶν οἰκτιρμῶν σου ὥρισας μετάνοιαν ἁμαρτωλοῖς εἰς σωτηρίαν.
\vs{8}Σὺ οὖν, Κύριε, ὁ Θεὸς τῶν δικαίων, οὐκ ἔθου μετάνοιαν δικαίοις, τῷ Ἀβραὰμ καὶ Ἰσαὰκ καὶ Ἰακὼβ, τοῖς οὐχ ἡμαρτηκόσιν σοι,
\vs{9}Ἐπλήθυναν αἱ ἀνομίαι μου, Κύριε, ἐπλήθυναν, καὶ οὐκ εἰμὶ ἄξιος ἀτενίσαι καὶ ἰδεῖν τὸ ὕψος τοῦ οὐρανοῦ ἀπὸ πλήθους τῶν ἀδικιῶν μου,
\vs{10}κατακαμπτόμενος πολλῷ δεσμῷ σιδηρῷ εἰς τὸ μὴ ἀνανεῦσαι τὴν κεφαλήν μου, καὶ οὐκ ἔστιν μοι ἄνεσις, διότι παρώργισα τὸν θυμόν σου, καὶ τὸ πονηρὸν ἐνώπιόν σου ἐποίησα, μὴ ποιήσας τὸ θέλημά σου καὶ μὴ φυλάξας τὰ προστάγματά σου, στήσας βδελύγματα καὶ πληθύνας προσοχθίσματα.
\vs{11}Καὶ νῦν κλίνω γόνυ καρδίας μου δεόμενος τῆς παθὰ σοῦ χρηστότητος· ἡμάρτηκα,
\vs{12}Κύριε, ἡμάρτηκα, καὶ τὰς ἀνομίας μου ἐγὼ γινώσκω. Ἀλλʼ αἰτοῦμαι δεόμενός σου· ἄνες μοι,
\vs{13}Κύριε, ἄνες μοι, καὶ μὴ συναπολέσῃς με ταῖς ἀνομίαις μου, μηδὲ εἰς τὸν αἰῶνα μηνίσας τηρήσῃς τὰ κακά μαι, μηδὲ καταδικάσῃς με ἐν τοῖς κατωτάτοις τῆς γῆς, διότι σὺ εἶ Θεὸς, Θεὶς τῶν μετανοούντων.
\vs{14}Καὶ ἐν ἐμοὶ δείξεις πᾶσαν τῆν ἀγαθωσύνην σου, ὁτι ἀνάξιον ὄντα σώσεις με κατὰ τὸ πολὺ ἔλεός σου.
\vs{15}Καὶ αἰνέσω σε διὰ παντὸς ἐν ταῖς ἡμέραις τῆς ζωῆς μου, ὅτι σὲ ὑμνεῖ πᾶσα ἡ δύναμις τῶν οὐρανῶν, καὶ σοῦ ἐστὶν ἡ δόξα εἰς τοὺς αἰῶναι. Ἀμήν

\cleardoublepage
\thispagestyle{empty}
\vspace*{3cm}
\phantomsection
\addcontentsline{toc}{part}{Η ΚΑΙΝΗ ΔΙΑΘΗΚΗ}
\begin{center}
  {\Huge Η ΚΑΙΝΗ ΔΙΑΘΗΚΗ}\\[2em]
\end{center}
\newpage
\thispagestyle{empty}
\null

\def\book{ΚΑΤΑ ΜΑΘΘΑΙΟΝ}
\biblebook{ΚΑΤΑ ΜΑΘΘΑΙΟΝ}


\lettrine[lines=2, loversize=0.2, nindent=0em, findent=.25em]{\textcolor{bookheadingcolor}{Β}}{ίβλος} γενέσεως Ἰησοῦ Χριστοῦ υἱοῦ Δαυὶδ υἱοῦ Ἀβραάμ.

\postdropcapindent\vs{2}Ἀβραὰμ ἐγέννησεν τὸν Ἰσαάκ, Ἰσαὰκ δὲ ἐγέννησεν τὸν Ἰακώβ, Ἰακὼβ δὲ ἐγέννησεν τὸν Ἰούδαν καὶ τοὺς ἀδελφοὺς αὐτοῦ,
\vs{3}Ἰούδας δὲ ἐγέννησεν τὸν Φαρὲς καὶ τὸν Ζαρὰ ἐκ τῆς Θάμαρ, Φαρὲς δὲ ἐγέννησεν τὸν Ἑσρώμ, Ἑσρὼμ δὲ ἐγέννησεν τὸν Ἀράμ,
\vs{4}Ἀρὰμ δὲ ἐγέννησεν τὸν Ἀμιναδάβ, Ἀμιναδὰβ δὲ ἐγέννησεν τὸν Ναασσών, Ναασσὼν δὲ ἐγέννησεν τὸν Σαλμών,
\vs{5}Σαλμὼν δὲ ἐγέννησεν τὸν Βόες ἐκ τῆς Ῥαχάβ, Βόες δὲ ἐγέννησεν τὸν Ἰωβὴδ ἐκ τῆς Ῥούθ, Ἰωβὴδ δὲ ἐγέννησεν τὸν Ἰεσσαί,
\vs{6}Ἰεσσαὶ δὲ ἐγέννησεν τὸν Δαυὶδ τὸν βασιλέα.

Δαυὶδ δὲ ἐγέννησεν τὸν Σολομῶνα ἐκ τῆς τοῦ Οὐρίου,
\vs{7}Σολομὼν δὲ ἐγέννησεν τὸν Ῥοβοάμ, Ῥοβοὰμ δὲ ἐγέννησεν τὸν Ἀβιά, Ἀβιὰ δὲ ἐγέννησεν τὸν Ἀσάφ,
\vs{8}Ἀσὰφ δὲ ἐγέννησεν τὸν Ἰωσαφάτ, Ἰωσαφὰτ δὲ ἐγέννησεν τὸν Ἰωράμ, Ἰωρὰμ δὲ ἐγέννησεν τὸν Ὀζίαν,
\vs{9}Ὀζίας δὲ ἐγέννησεν τὸν Ἰωαθάμ, Ἰωαθὰμ δὲ ἐγέννησεν τὸν Ἄχαζ, Ἄχαζ δὲ ἐγέννησεν τὸν Ἑζεκίαν,
\vs{10}Ἑζεκίας δὲ ἐγέννησεν τὸν Μανασσῆ, Μανασσῆς δὲ ἐγέννησεν τὸν Ἀμώς, Ἀμὼς δὲ ἐγέννησεν τὸν Ἰωσίαν,
\vs{11}Ἰωσίας δὲ ἐγέννησεν τὸν Ἰεχονίαν καὶ τοὺς ἀδελφοὺς αὐτοῦ ἐπὶ τῆς μετοικεσίας Βαβυλῶνος.

\vs{12}Μετὰ δὲ τὴν μετοικεσίαν Βαβυλῶνος Ἰεχονίας ἐγέννησεν τὸν Σαλαθιήλ, Σαλαθιὴλ δὲ ἐγέννησεν τὸν Ζοροβαβέλ,
\vs{13}Ζοροβαβὲλ δὲ ἐγέννησεν τὸν Ἀβιούδ, Ἀβιοὺδ δὲ ἐγέννησεν τὸν Ἐλιακίμ, Ἐλιακὶμ δὲ ἐγέννησεν τὸν Ἀζώρ,
\vs{14}Ἀζὼρ δὲ ἐγέννησεν τὸν Σαδώκ, Σαδὼκ δὲ ἐγέννησεν τὸν Ἀχίμ, Ἀχὶμ δὲ ἐγέννησεν τὸν Ἐλιούδ,
\vs{15}Ἐλιοὺδ δὲ ἐγέννησεν τὸν Ἐλεάζαρ, Ἐλεάζαρ δὲ ἐγέννησεν τὸν Ματθάν, Ματθὰν δὲ ἐγέννησεν τὸν Ἰακώβ,
\vs{16}Ἰακὼβ δὲ ἐγέννησεν τὸν Ἰωσὴφ τὸν ἄνδρα Μαρίας, ἐξ ἧς ἐγεννήθη Ἰησοῦς ὁ λεγόμενος Χριστός.

\vs{17}Πᾶσαι οὖν αἱ γενεαὶ ἀπὸ Ἀβραὰμ ἕως Δαυὶδ γενεαὶ δεκατέσσαρες, καὶ ἀπὸ Δαυὶδ ἕως τῆς μετοικεσίας Βαβυλῶνος γενεαὶ δεκατέσσαρες, καὶ ἀπὸ τῆς μετοικεσίας Βαβυλῶνος ἕως τοῦ Χριστοῦ γενεαὶ δεκατέσσαρες.

\vs{18}Τοῦ δὲ Ἰησοῦ Χριστοῦ ἡ γένεσις οὕτως ἦν. μνηστευθείσης τῆς μητρὸς αὐτοῦ Μαρίας τῷ Ἰωσήφ, πρὶν ἢ συνελθεῖν αὐτοὺς εὑρέθη ἐν γαστρὶ ἔχουσα ἐκ πνεύματος ἁγίου.
\vs{19}Ἰωσὴφ δὲ ὁ ἀνὴρ αὐτῆς, δίκαιος ὢν καὶ μὴ θέλων αὐτὴν δειγματίσαι, ἐβουλήθη λάθρᾳ ἀπολῦσαι αὐτήν.
\vs{20}Ταῦτα δὲ αὐτοῦ ἐνθυμηθέντος ἰδοὺ ἄγγελος Κυρίου κατ᾽ ὄναρ ἐφάνη αὐτῷ λέγων· Ἰωσὴφ υἱὸς Δαυίδ, μὴ φοβηθῇς παραλαβεῖν Μαρίαν τὴν γυναῖκά σου· τὸ γὰρ ἐν αὐτῇ γεννηθὲν ἐκ Πνεύματός ἐστιν Ἁγίου.
\vs{21}τέξεται δὲ υἱὸν, καὶ καλέσεις τὸ ὄνομα αὐτοῦ Ἰησοῦν· αὐτὸς γὰρ σώσει τὸν λαὸν αὐτοῦ ἀπὸ τῶν ἁμαρτιῶν αὐτῶν.
\vs{22}Τοῦτο δὲ ὅλον γέγονεν ἵνα πληρωθῇ τὸ ῥηθὲν ὑπὸ Κυρίου διὰ τοῦ προφήτου λέγοντος·
\begin{poetryblock}

\begin{quote} \vs{23}Ἰδοὺ ἡ παρθένος ἐν γαστρὶ ἕξει καὶ τέξεται υἱόν,\end{quote} 

\begin{quote}καὶ καλέσουσιν τὸ ὄνομα αὐτοῦ Ἐμμανουήλ,\end{quote}
\end{poetryblock}

ὅ ἐστιν μεθερμηνευόμενον Μεθ᾽ ἡμῶν ὁ Θεός.
\vs{24}Ἐγερθεὶς δὲ ὁ Ἰωσὴφ ἀπὸ τοῦ ὕπνου ἐποίησεν ὡς προσέταξεν αὐτῷ ὁ ἄγγελος Κυρίου καὶ παρέλαβεν τὴν γυναῖκα αὐτοῦ,
\vs{25}καὶ οὐκ ἐγίνωσκεν αὐτὴν ἕως οὗ ἔτεκεν υἱόν· καὶ ἐκάλεσεν τὸ ὄνομα αὐτοῦ Ἰησοῦν.

\ch{2}
Τοῦ δὲ Ἰησοῦ γεννηθέντος ἐν Βηθλέεμ τῆς Ἰουδαίας ἐν ἡμέραις Ἡρῴδου τοῦ βασιλέως, ἰδοὺ μάγοι ἀπὸ ἀνατολῶν παρεγένοντο εἰς Ἱεροσόλυμα
\vs{2}λέγοντες· Ποῦ ἐστιν ὁ τεχθεὶς βασιλεὺς τῶν Ἰουδαίων; εἴδομεν γὰρ αὐτοῦ τὸν ἀστέρα ἐν τῇ ἀνατολῇ καὶ ἤλθομεν προσκυνῆσαι αὐτῷ.
\vs{3}Ἀκούσας δὲ ὁ βασιλεὺς Ἡρῴδης ἐταράχθη καὶ πᾶσα Ἱεροσόλυμα μετ᾽ αὐτοῦ,
\vs{4}καὶ συναγαγὼν πάντας τοὺς ἀρχιερεῖς καὶ γραμματεῖς τοῦ λαοῦ ἐπυνθάνετο παρ᾽ αὐτῶν ποῦ ὁ Χριστὸς γεννᾶται.
\vs{5}Οἱ δὲ εἶπαν αὐτῷ· Ἐν Βηθλέεμ τῆς Ἰουδαίας· οὕτως γὰρ γέγραπται διὰ τοῦ προφήτου·
\begin{poetryblock}

\begin{quote} \vs{6}Καὶ σύ Βηθλέεμ, γῆ Ἰούδα,\end{quote} 

\begin{quote}οὐδαμῶς ἐλαχίστη εἶ ἐν τοῖς ἡγεμόσιν Ἰούδα·\end{quote} 

\begin{quote}ἐκ σοῦ γὰρ ἐξελεύσεται ἡγούμενος,\end{quote} 

\begin{quote}ὅστις ποιμανεῖ τὸν λαόν μου τὸν Ἰσραήλ.\end{quote}
\end{poetryblock}

\vs{7}Τότε Ἡρῴδης λάθρᾳ καλέσας τοὺς μάγους ἠκρίβωσεν παρ᾽ αὐτῶν τὸν χρόνον τοῦ φαινομένου ἀστέρος,
\vs{8}καὶ πέμψας αὐτοὺς εἰς Βηθλέεμ εἶπεν· Πορευθέντες ἐξετάσατε ἀκριβῶς περὶ τοῦ παιδίου· ἐπὰν δὲ εὕρητε, ἀπαγγείλατέ μοι, ὅπως κἀγὼ ἐλθὼν προσκυνήσω αὐτῷ.
\vs{9}Οἱ δὲ ἀκούσαντες τοῦ βασιλέως ἐπορεύθησαν καὶ ἰδοὺ ὁ ἀστὴρ, ὃν εἶδον ἐν τῇ ἀνατολῇ, προῆγεν αὐτούς, ἕως ἐλθὼν ἐστάθη ἐπάνω οὗ ἦν τὸ παιδίον.
\vs{10}ἰδόντες δὲ τὸν ἀστέρα ἐχάρησαν χαρὰν μεγάλην σφόδρα.
\vs{11}καὶ ἐλθόντες εἰς τὴν οἰκίαν εἶδον τὸ παιδίον μετὰ Μαρίας τῆς μητρὸς αὐτοῦ, καὶ πεσόντες προσεκύνησαν αὐτῷ καὶ ἀνοίξαντες τοὺς θησαυροὺς αὐτῶν προσήνεγκαν αὐτῷ δῶρα, χρυσὸν καὶ λίβανον καὶ σμύρναν.
\vs{12}Καὶ χρηματισθέντες κατ᾽ ὄναρ μὴ ἀνακάμψαι πρὸς Ἡρῴδην, δι᾽ ἄλλης ὁδοῦ ἀνεχώρησαν εἰς τὴν χώραν αὐτῶν.

\vs{13}Ἀναχωρησάντων δὲ αὐτῶν ἰδοὺ ἄγγελος κυρίου φαίνεται κατ᾽ ὄναρ τῷ Ἰωσὴφ λέγων· Ἐγερθεὶς παράλαβε τὸ παιδίον καὶ τὴν μητέρα αὐτοῦ καὶ φεῦγε εἰς Αἴγυπτον καὶ ἴσθι ἐκεῖ ἕως ἂν εἴπω σοι· μέλλει γὰρ Ἡρῴδης ζητεῖν τὸ παιδίον τοῦ ἀπολέσαι αὐτό.
\vs{14}Ὁ δὲ ἐγερθεὶς παρέλαβεν τὸ παιδίον καὶ τὴν μητέρα αὐτοῦ νυκτὸς καὶ ἀνεχώρησεν εἰς Αἴγυπτον,
\vs{15}καὶ ἦν ἐκεῖ ἕως τῆς τελευτῆς Ἡρῴδου· ἵνα πληρωθῇ τὸ ῥηθὲν ὑπὸ κυρίου διὰ τοῦ προφήτου λέγοντος· 
\begin{poetryblock}

\begin{quote}Ἐξ Αἰγύπτου ἐκάλεσα τὸν υἱόν μου.\end{quote}
\end{poetryblock}

\vs{16}Τότε Ἡρῴδης ἰδὼν ὅτι ἐνεπαίχθη ὑπὸ τῶν μάγων ἐθυμώθη λίαν, καὶ ἀποστείλας ἀνεῖλεν πάντας τοὺς παῖδας τοὺς ἐν Βηθλέεμ καὶ ἐν πᾶσι τοῖς ὁρίοις αὐτῆς ἀπὸ διετοῦς καὶ κατωτέρω, κατὰ τὸν χρόνον ὃν ἠκρίβωσεν παρὰ τῶν μάγων.
\vs{17}τότε ἐπληρώθη τὸ ῥηθὲν διὰ Ἰερεμίου τοῦ προφήτου λέγοντος·
\begin{poetryblock}

\begin{quote} \vs{18}Φωνὴ ἐν Ῥαμὰ ἠκούσθη,\end{quote} 

\begin{quote}κλαυθμὸς καὶ ὀδυρμὸς πολύς·\end{quote} 

\begin{quote}Ῥαχὴλ κλαίουσα τὰ τέκνα αὐτῆς,\end{quote} 

\begin{quote}καὶ οὐκ ἤθελεν παρακληθῆναι,\end{quote} 

\begin{quote}ὅτι οὐκ εἰσίν.\end{quote}
\end{poetryblock}

\vs{19}Τελευτήσαντος δὲ τοῦ Ἡρῴδου ἰδοὺ ἄγγελος Κυρίου φαίνεται κατ᾽ ὄναρ τῷ Ἰωσὴφ ἐν Αἰγύπτῳ
\vs{20}λέγων· Ἐγερθεὶς παράλαβε τὸ παιδίον καὶ τὴν μητέρα αὐτοῦ καὶ πορεύου εἰς γῆν Ἰσραήλ· τεθνήκασιν γὰρ οἱ ζητοῦντες τὴν ψυχὴν τοῦ παιδίου.
\vs{21}Ὁ δὲ ἐγερθεὶς παρέλαβεν τὸ παιδίον καὶ τὴν μητέρα αὐτοῦ καὶ εἰσῆλθεν εἰς γῆν Ἰσραήλ.

\vs{22}ἀκούσας δὲ ὅτι Ἀρχέλαος βασιλεύει τῆς Ἰουδαίας ἀντὶ τοῦ πατρὸς αὐτοῦ Ἡρῴδου ἐφοβήθη ἐκεῖ ἀπελθεῖν· χρηματισθεὶς δὲ κατ᾽ ὄναρ ἀνεχώρησεν εἰς τὰ μέρη τῆς Γαλιλαίας,
\vs{23}καὶ ἐλθὼν κατῴκησεν εἰς πόλιν λεγομένην Ναζαρέτ· ὅπως πληρωθῇ τὸ ῥηθὲν διὰ τῶν προφητῶν ὅτι Ναζωραῖος κληθήσεται.

\ch{3}
Ἐν δὲ ταῖς ἡμέραις ἐκείναις παραγίνεται Ἰωάννης ὁ βαπτιστὴς κηρύσσων ἐν τῇ ἐρήμῳ τῆς Ἰουδαίας
\vs{2}καὶ λέγων· Μετανοεῖτε· ἤγγικεν γὰρ ἡ βασιλεία τῶν οὐρανῶν.

\vs{3}οὗτος γάρ ἐστιν ὁ ῥηθεὶς διὰ Ἠσαΐου τοῦ προφήτου λέγοντος· 
\begin{poetryblock}

\begin{quote}Φωνὴ βοῶντος ἐν τῇ ἐρήμῳ·\end{quote} 

\begin{quote}Ἑτοιμάσατε τὴν ὁδὸν Κυρίου,\end{quote} 

\begin{quote}εὐθείας ποιεῖτε τὰς τρίβους αὐτοῦ.\end{quote}
\end{poetryblock}

\vs{4}Αὐτὸς δὲ ὁ Ἰωάννης εἶχεν τὸ ἔνδυμα αὐτοῦ ἀπὸ τριχῶν καμήλου καὶ ζώνην δερματίνην περὶ τὴν ὀσφὺν αὐτοῦ, ἡ δὲ τροφὴ ἦν αὐτοῦ ἀκρίδες καὶ μέλι ἄγριον.
\vs{5}Τότε ἐξεπορεύετο πρὸς αὐτὸν Ἱεροσόλυμα καὶ πᾶσα ἡ Ἰουδαία καὶ πᾶσα ἡ περίχωρος τοῦ Ἰορδάνου,
\vs{6}καὶ ἐβαπτίζοντο ἐν τῷ Ἰορδάνῃ ποταμῷ ὑπ᾽ αὐτοῦ ἐξομολογούμενοι τὰς ἁμαρτίας αὐτῶν.

\vs{7}Ἰδὼν δὲ πολλοὺς τῶν Φαρισαίων καὶ Σαδδουκαίων ἐρχομένους ἐπὶ τὸ βάπτισμα αὐτοῦ εἶπεν αὐτοῖς· Γεννήματα ἐχιδνῶν, τίς ὑπέδειξεν ὑμῖν φυγεῖν ἀπὸ τῆς μελλούσης ὀργῆς;
\vs{8}ποιήσατε οὖν καρπὸν ἄξιον τῆς μετανοίας
\vs{9}καὶ μὴ δόξητε λέγειν ἐν ἑαυτοῖς· Πατέρα ἔχομεν τὸν Ἀβραάμ. λέγω γὰρ ὑμῖν ὅτι δύναται ὁ Θεὸς ἐκ τῶν λίθων τούτων ἐγεῖραι τέκνα τῷ Ἀβραάμ.
\vs{10}ἤδη δὲ ἡ ἀξίνη πρὸς τὴν ῥίζαν τῶν δένδρων κεῖται· πᾶν οὖν δένδρον μὴ ποιοῦν καρπὸν καλὸν ἐκκόπτεται καὶ εἰς πῦρ βάλλεται.

\vs{11}Ἐγὼ μὲν ὑμᾶς βαπτίζω ἐν ὕδατι εἰς μετάνοιαν, ὁ δὲ ὀπίσω μου ἐρχόμενος ἰσχυρότερός μού ἐστιν, οὗ οὐκ εἰμὶ ἱκανὸς τὰ ὑποδήματα βαστάσαι· αὐτὸς ὑμᾶς βαπτίσει ἐν Πνεύματι Ἁγίῳ καὶ πυρί·
\vs{12}οὗ τὸ πτύον ἐν τῇ χειρὶ αὐτοῦ καὶ διακαθαριεῖ τὴν ἅλωνα αὐτοῦ καὶ συνάξει τὸν σῖτον αὐτοῦ εἰς τὴν ἀποθήκην, τὸ δὲ ἄχυρον κατακαύσει πυρὶ ἀσβέστῳ.

\vs{13}Τότε παραγίνεται ὁ Ἰησοῦς ἀπὸ τῆς Γαλιλαίας ἐπὶ τὸν Ἰορδάνην πρὸς τὸν Ἰωάννην τοῦ βαπτισθῆναι ὑπ᾽ αὐτοῦ.
\vs{14}ὁ δὲ Ἰωάννης διεκώλυεν αὐτὸν λέγων· Ἐγὼ χρείαν ἔχω ὑπὸ σοῦ βαπτισθῆναι, καὶ σὺ ἔρχῃ πρός με;
\vs{15}Ἀποκριθεὶς δὲ ὁ Ἰησοῦς εἶπεν πρὸς αὐτόν· Ἄφες ἄρτι, οὕτως γὰρ πρέπον ἐστὶν ἡμῖν πληρῶσαι πᾶσαν δικαιοσύνην. τότε ἀφίησιν αὐτόν.
\vs{16}Βαπτισθεὶς δὲ ὁ Ἰησοῦς εὐθὺς ἀνέβη ἀπὸ τοῦ ὕδατος· καὶ ἰδοὺ ἠνεῴχθησαν αὐτῷ οἱ οὐρανοί, καὶ εἶδεν τὸ Πνεῦμα τοῦ Θεοῦ καταβαῖνον ὡσεὶ περιστερὰν καὶ ἐρχόμενον ἐπ᾽ αὐτόν·
\vs{17}καὶ ἰδοὺ φωνὴ ἐκ τῶν οὐρανῶν λέγουσα· Οὗτός ἐστιν ὁ Υἱός μου ὁ ἀγαπητός, ἐν ᾧ εὐδόκησα.

\ch{4}
Τότε ὁ Ἰησοῦς ἀνήχθη εἰς τὴν ἔρημον ὑπὸ τοῦ Πνεύματος πειρασθῆναι ὑπὸ τοῦ διαβόλου.
\vs{2}καὶ νηστεύσας ἡμέρας τεσσεράκοντα καὶ νύκτας τεσσεράκοντα, ὕστερον ἐπείνασεν.
\vs{3}Καὶ προσελθὼν ὁ πειράζων εἶπεν αὐτῷ· Εἰ Υἱὸς εἶ τοῦ Θεοῦ, εἰπὲ ἵνα οἱ λίθοι οὗτοι ἄρτοι γένωνται.
\vs{4}Ὁ δὲ ἀποκριθεὶς εἶπεν· Γέγραπται· Οὐκ ἐπ᾽ ἄρτῳ μόνῳ ζήσεται ὁ ἄνθρωπος, Ἀλλ᾽ ἐπὶ παντὶ ῥήματι ἐκπορευομένῳ διὰ στόματος Θεοῦ.

\vs{5}Τότε παραλαμβάνει αὐτὸν ὁ διάβολος εἰς τὴν ἁγίαν πόλιν καὶ ἔστησεν αὐτὸν ἐπὶ τὸ πτερύγιον τοῦ ἱεροῦ
\vs{6}καὶ λέγει αὐτῷ· Εἰ Υἱὸς εἶ τοῦ Θεοῦ, βάλε σεαυτὸν κάτω· γέγραπται γὰρ ὅτι 
\begin{poetryblock}

\begin{quote}Τοῖς ἀγγέλοις αὐτοῦ ἐντελεῖται περὶ σοῦ\end{quote} 

\begin{quote}καὶ ἐπὶ χειρῶν ἀροῦσίν σε,\end{quote} 

\begin{quote}μήποτε προσκόψῃς πρὸς λίθον τὸν πόδα σου.\end{quote}
\end{poetryblock}

\vs{7}Ἔφη αὐτῷ ὁ Ἰησοῦς· Πάλιν γέγραπται· Οὐκ ἐκπειράσεις Κύριον τὸν Θεόν σου.

\vs{8}Πάλιν παραλαμβάνει αὐτὸν ὁ διάβολος εἰς ὄρος ὑψηλὸν λίαν καὶ δείκνυσιν αὐτῷ πάσας τὰς βασιλείας τοῦ κόσμου καὶ τὴν δόξαν αὐτῶν
\vs{9}καὶ εἶπεν αὐτῷ· Ταῦτά σοι πάντα δώσω, ἐὰν πεσὼν προσκυνήσῃς μοι.
\vs{10}Τότε λέγει αὐτῷ ὁ Ἰησοῦς· Ὕπαγε, Σατανᾶ· γέγραπται γάρ· Κύριον τὸν θεόν σου προσκυνήσεις καὶ αὐτῷ μόνῳ λατρεύσεις.

\vs{11}Τότε ἀφίησιν αὐτὸν ὁ διάβολος, καὶ ἰδοὺ ἄγγελοι προσῆλθον καὶ διηκόνουν αὐτῷ.

\vs{12}Ἀκούσας δὲ ὅτι Ἰωάννης παρεδόθη ἀνεχώρησεν εἰς τὴν Γαλιλαίαν.
\vs{13}καὶ καταλιπὼν τὴν Ναζαρὰ ἐλθὼν κατῴκησεν εἰς Καφαρναοὺμ τὴν παραθαλασσίαν ἐν ὁρίοις Ζαβουλὼν καὶ Νεφθαλίμ·
\vs{14}ἵνα πληρωθῇ τὸ ῥηθὲν διὰ Ἠσαΐου τοῦ προφήτου λέγοντος·
\begin{poetryblock}

\begin{quote} \vs{15}Γῆ Ζαβουλὼν καὶ γῆ Νεφθαλίμ,\end{quote} 

\begin{quote}ὁδὸν θαλάσσης, πέραν τοῦ Ἰορδάνου,\end{quote} 

\begin{quote}Γαλιλαία τῶν ἐθνῶν,\end{quote}

\begin{quote} \vs{16}ὁ λαὸς ὁ καθήμενος ἐν σκοτίᾳ\end{quote} 

\begin{quote}φῶς εἶδεν μέγα,\end{quote} 

\begin{quote}καὶ τοῖς καθημένοις ἐν χώρᾳ καὶ σκιᾷ θανάτου\end{quote} 

\begin{quote}φῶς ἀνέτειλεν αὐτοῖς.\end{quote}
\end{poetryblock}

\vs{17}Ἀπὸ τότε ἤρξατο ὁ Ἰησοῦς κηρύσσειν καὶ λέγειν· Μετανοεῖτε· ἤγγικεν γὰρ ἡ βασιλεία τῶν οὐρανῶν.

\vs{18}Περιπατῶν δὲ παρὰ τὴν θάλασσαν τῆς Γαλιλαίας εἶδεν δύο ἀδελφούς, Σίμωνα τὸν λεγόμενον Πέτρον καὶ Ἀνδρέαν τὸν ἀδελφὸν αὐτοῦ, βάλλοντας ἀμφίβληστρον εἰς τὴν θάλασσαν· ἦσαν γὰρ ἁλιεῖς.
\vs{19}καὶ λέγει αὐτοῖς· Δεῦτε ὀπίσω μου, καὶ ποιήσω ὑμᾶς ἁλιεῖς ἀνθρώπων.
\vs{20}οἱ δὲ εὐθέως ἀφέντες τὰ δίκτυα ἠκολούθησαν αὐτῷ.
\vs{21}Καὶ προβὰς ἐκεῖθεν εἶδεν ἄλλους δύο ἀδελφούς, Ἰάκωβον τὸν τοῦ Ζεβεδαίου καὶ Ἰωάννην τὸν ἀδελφὸν αὐτοῦ, ἐν τῷ πλοίῳ μετὰ Ζεβεδαίου τοῦ πατρὸς αὐτῶν καταρτίζοντας τὰ δίκτυα αὐτῶν, καὶ ἐκάλεσεν αὐτούς.
\vs{22}οἱ δὲ εὐθέως ἀφέντες τὸ πλοῖον καὶ τὸν πατέρα αὐτῶν ἠκολούθησαν αὐτῷ.

\vs{23}Καὶ περιῆγεν ἐν ὅλῃ τῇ Γαλιλαίᾳ διδάσκων ἐν ταῖς συναγωγαῖς αὐτῶν καὶ κηρύσσων τὸ εὐαγγέλιον τῆς βασιλείας καὶ θεραπεύων πᾶσαν νόσον καὶ πᾶσαν μαλακίαν ἐν τῷ λαῷ.

\vs{24}καὶ ἀπῆλθεν ἡ ἀκοὴ αὐτοῦ εἰς ὅλην τὴν Συρίαν· καὶ προσήνεγκαν αὐτῷ πάντας τοὺς κακῶς ἔχοντας ποικίλαις νόσοις καὶ βασάνοις συνεχομένους καὶ δαιμονιζομένους καὶ σεληνιαζομένους καὶ παραλυτικούς, καὶ ἐθεράπευσεν αὐτούς.
\vs{25}Καὶ ἠκολούθησαν αὐτῷ ὄχλοι πολλοὶ ἀπὸ τῆς Γαλιλαίας καὶ Δεκαπόλεως καὶ Ἱεροσολύμων καὶ Ἰουδαίας καὶ πέραν τοῦ Ἰορδάνου.

\ch{5}
Ἰδὼν δὲ τοὺς ὄχλους ἀνέβη εἰς τὸ ὄρος, καὶ καθίσαντος αὐτοῦ προσῆλθαν αὐτῷ οἱ μαθηταὶ αὐτοῦ·
\vs{2}καὶ ἀνοίξας τὸ στόμα αὐτοῦ ἐδίδασκεν αὐτοὺς λέγων·
\begin{poetryblock}

\begin{quote} \vs{3}Μακάριοι οἱ πτωχοὶ τῷ πνεύματι,\end{quote} 

\begin{quote}Ὅτι αὐτῶν ἐστιν ἡ βασιλεία τῶν οὐρανῶν.\end{quote}

\begin{quote} \vs{4}Μακάριοι οἱ πενθοῦντες, Ὅτι αὐτοὶ παρακληθήσονται.\end{quote}

\begin{quote} \vs{5}Μακάριοι οἱ πραεῖς,\end{quote} 

\begin{quote}Ὅτι αὐτοὶ κληρονομήσουσιν τὴν γῆν.\end{quote}

\begin{quote} \vs{6}Μακάριοι οἱ πεινῶντες καὶ διψῶντες τὴν δικαιοσύνην,\end{quote} 

\begin{quote}Ὅτι αὐτοὶ χορτασθήσονται.\end{quote}

\begin{quote} \vs{7}Μακάριοι οἱ ἐλεήμονες,\end{quote} 

\begin{quote}Ὅτι αὐτοὶ ἐλεηθήσονται.\end{quote}

\begin{quote} \vs{8}Μακάριοι οἱ καθαροὶ τῇ καρδίᾳ,\end{quote} 

\begin{quote}Ὅτι αὐτοὶ τὸν Θεὸν ὄψονται.\end{quote}

\begin{quote} \vs{9}Μακάριοι οἱ εἰρηνοποιοί,\end{quote} 

\begin{quote}Ὅτι αὐτοὶ υἱοὶ Θεοῦ κληθήσονται.\end{quote}

\begin{quote} \vs{10}Μακάριοι οἱ δεδιωγμένοι ἕνεκεν δικαιοσύνης,\end{quote} 

\begin{quote}Ὅτι αὐτῶν ἐστιν ἡ βασιλεία τῶν οὐρανῶν.\end{quote}

\begin{quote} \vs{11}Μακάριοί ἐστε\end{quote} 

\begin{quote}ὅταν ὀνειδίσωσιν ὑμᾶς καὶ διώξωσιν καὶ εἴπωσιν πᾶν πονηρὸν καθ᾽ ὑμῶν ψευδόμενοι ἕνεκεν ἐμοῦ.\end{quote}

\begin{quote} \vs{12}χαίρετε καὶ ἀγαλλιᾶσθε, ὅτι ὁ μισθὸς ὑμῶν πολὺς ἐν τοῖς οὐρανοῖς· οὕτως γὰρ ἐδίωξαν τοὺς προφήτας τοὺς πρὸ ὑμῶν.\end{quote}
\end{poetryblock}

\vs{13}Ὑμεῖς ἐστε τὸ ἅλας τῆς γῆς· ἐὰν δὲ τὸ ἅλας μωρανθῇ, ἐν τίνι ἁλισθήσεται; εἰς οὐδὲν ἰσχύει ἔτι εἰ μὴ βληθὲν ἔξω καταπατεῖσθαι ὑπὸ τῶν ἀνθρώπων.

\vs{14}Ὑμεῖς ἐστε τὸ φῶς τοῦ κόσμου. οὐ δύναται πόλις κρυβῆναι ἐπάνω ὄρους κειμένη·
\vs{15}οὐδὲ καίουσιν λύχνον καὶ τιθέασιν αὐτὸν ὑπὸ τὸν μόδιον ἀλλ᾽ ἐπὶ τὴν λυχνίαν, καὶ λάμπει πᾶσιν τοῖς ἐν τῇ οἰκίᾳ.
\vs{16}οὕτως λαμψάτω τὸ φῶς ὑμῶν ἔμπροσθεν τῶν ἀνθρώπων, ὅπως ἴδωσιν ὑμῶν τὰ καλὰ ἔργα καὶ δοξάσωσιν τὸν πατέρα ὑμῶν τὸν ἐν τοῖς οὐρανοῖς.

\vs{17}Μὴ νομίσητε ὅτι ἦλθον καταλῦσαι τὸν νόμον ἢ τοὺς προφήτας· οὐκ ἦλθον καταλῦσαι ἀλλὰ πληρῶσαι.
\vs{18}ἀμὴν γὰρ λέγω ὑμῖν· ἕως ἂν παρέλθῃ ὁ οὐρανὸς καὶ ἡ γῆ, ἰῶτα ἓν ἢ μία κεραία οὐ μὴ παρέλθῃ ἀπὸ τοῦ νόμου, ἕως ἂν πάντα γένηται.
\vs{19}Ὃς ἐὰν οὖν λύσῃ μίαν τῶν ἐντολῶν τούτων τῶν ἐλαχίστων καὶ διδάξῃ οὕτως τοὺς ἀνθρώπους, ἐλάχιστος κληθήσεται ἐν τῇ βασιλείᾳ τῶν οὐρανῶν· ὃς δ᾽ ἂν ποιήσῃ καὶ διδάξῃ, οὗτος μέγας κληθήσεται ἐν τῇ βασιλείᾳ τῶν οὐρανῶν.

\vs{20}λέγω γὰρ ὑμῖν ὅτι ἐὰν μὴ περισσεύσῃ ὑμῶν ἡ δικαιοσύνη πλεῖον τῶν γραμματέων καὶ Φαρισαίων, οὐ μὴ εἰσέλθητε εἰς τὴν βασιλείαν τῶν οὐρανῶν.

\vs{21}Ἠκούσατε ὅτι ἐρρέθη τοῖς ἀρχαίοις· Οὐ φονεύσεις· ὃς δ᾽ ἂν φονεύσῃ, ἔνοχος ἔσται τῇ κρίσει.
\vs{22}ἐγὼ δὲ λέγω ὑμῖν ὅτι πᾶς ὁ ὀργιζόμενος τῷ ἀδελφῷ αὐτοῦ ἔνοχος ἔσται τῇ κρίσει· ὃς δ᾽ ἂν εἴπῃ τῷ ἀδελφῷ αὐτοῦ· Ῥακά, ἔνοχος ἔσται τῷ συνεδρίῳ· ὃς δ᾽ ἂν εἴπῃ· Μωρέ, ἔνοχος ἔσται εἰς τὴν γέενναν τοῦ πυρός.
\vs{23}Ἐὰν οὖν προσφέρῃς τὸ δῶρόν σου ἐπὶ τὸ θυσιαστήριον κἀκεῖ μνησθῇς ὅτι ὁ ἀδελφός σου ἔχει τι κατὰ σοῦ,
\vs{24}ἄφες ἐκεῖ τὸ δῶρόν σου ἔμπροσθεν τοῦ θυσιαστηρίου καὶ ὕπαγε πρῶτον διαλλάγηθι τῷ ἀδελφῷ σου, καὶ τότε ἐλθὼν πρόσφερε τὸ δῶρόν σου.
\vs{25}ἴσθι εὐνοῶν τῷ ἀντιδίκῳ σου ταχὺ, ἕως ὅτου εἶ μετ᾽ αὐτοῦ ἐν τῇ ὁδῷ, μήποτέ σε παραδῷ ὁ ἀντίδικος τῷ κριτῇ καὶ ὁ κριτὴς τῷ ὑπηρέτῃ καὶ εἰς φυλακὴν βληθήσῃ·
\vs{26}ἀμὴν λέγω σοι, οὐ μὴ ἐξέλθῃς ἐκεῖθεν, ἕως ἂν ἀποδῷς τὸν ἔσχατον κοδράντην.

\vs{27}Ἠκούσατε ὅτι ἐρρέθη· Οὐ μοιχεύσεις.
\vs{28}ἐγὼ δὲ λέγω ὑμῖν ὅτι πᾶς ὁ βλέπων γυναῖκα πρὸς τὸ ἐπιθυμῆσαι αὐτὴν ἤδη ἐμοίχευσεν αὐτὴν ἐν τῇ καρδίᾳ αὐτοῦ.
\vs{29}εἰ δὲ ὁ ὀφθαλμός σου ὁ δεξιὸς σκανδαλίζει σε, ἔξελε αὐτὸν καὶ βάλε ἀπὸ σοῦ· συμφέρει γάρ σοι ἵνα ἀπόληται ἓν τῶν μελῶν σου καὶ μὴ ὅλον τὸ σῶμά σου βληθῇ εἰς γέενναν.
\vs{30}καὶ εἰ ἡ δεξιά σου χεὶρ σκανδαλίζει σε, ἔκκοψον αὐτὴν καὶ βάλε ἀπὸ σοῦ· συμφέρει γάρ σοι ἵνα ἀπόληται ἓν τῶν μελῶν σου καὶ μὴ ὅλον τὸ σῶμά σου εἰς γέενναν ἀπέλθῃ.

\vs{31}Ἐρρέθη δέ· Ὃς ἂν ἀπολύσῃ τὴν γυναῖκα αὐτοῦ, δότω αὐτῇ ἀποστάσιον.
\vs{32}ἐγὼ δὲ λέγω ὑμῖν ὅτι πᾶς ὁ ἀπολύων τὴν γυναῖκα αὐτοῦ παρεκτὸς λόγου πορνείας ποιεῖ αὐτὴν μοιχευθῆναι, καὶ ὃς ἐὰν ἀπολελυμένην γαμήσῃ, μοιχᾶται.

\vs{33}Πάλιν ἠκούσατε ὅτι ἐρρέθη τοῖς ἀρχαίοις· Οὐκ ἐπιορκήσεις, ἀποδώσεις δὲ τῷ Κυρίῳ τοὺς ὅρκους σου.
\vs{34}ἐγὼ δὲ λέγω ὑμῖν μὴ ὀμόσαι ὅλως· μήτε ἐν τῷ οὐρανῷ, ὅτι θρόνος ἐστὶν τοῦ Θεοῦ,
\vs{35}μήτε ἐν τῇ γῇ, ὅτι ὑποπόδιόν ἐστιν τῶν ποδῶν αὐτοῦ, μήτε εἰς Ἱεροσόλυμα, ὅτι πόλις ἐστὶν τοῦ μεγάλου Βασιλέως,
\vs{36}μήτε ἐν τῇ κεφαλῇ σου ὀμόσῃς, ὅτι οὐ δύνασαι μίαν τρίχα λευκὴν ποιῆσαι ἢ μέλαιναν.
\vs{37}ἔστω δὲ ὁ λόγος ὑμῶν Ναὶ ναί, οὒ Οὔ· τὸ δὲ περισσὸν τούτων ἐκ τοῦ πονηροῦ ἐστιν.

\vs{38}Ἠκούσατε ὅτι ἐρρέθη· Ὀφθαλμὸν ἀντὶ ὀφθαλμοῦ καὶ ὀδόντα ἀντὶ ὀδόντος.
\vs{39}ἐγὼ δὲ λέγω ὑμῖν μὴ ἀντιστῆναι τῷ πονηρῷ· ἀλλ᾽ ὅστις σε ῥαπίζει εἰς τὴν δεξιὰν σιαγόνα σου, στρέψον αὐτῷ καὶ τὴν ἄλλην·
\vs{40}καὶ τῷ θέλοντί σοι κριθῆναι καὶ τὸν χιτῶνά σου λαβεῖν, ἄφες αὐτῷ καὶ τὸ ἱμάτιον·
\vs{41}καὶ ὅστις σε ἀγγαρεύσει μίλιον ἕν, ὕπαγε μετ᾽ αὐτοῦ δύο.
\vs{42}τῷ αἰτοῦντί σε δός, καὶ τὸν θέλοντα ἀπὸ σοῦ δανίσασθαι μὴ ἀποστραφῇς.

\vs{43}Ἠκούσατε ὅτι ἐρρέθη· Ἀγαπήσεις τὸν πλησίον σου καὶ μισήσεις τὸν ἐχθρόν σου.
\vs{44}ἐγὼ δὲ λέγω ὑμῖν· ἀγαπᾶτε τοὺς ἐχθροὺς ὑμῶν καὶ προσεύχεσθε ὑπὲρ τῶν διωκόντων ὑμᾶς,
\vs{45}ὅπως γένησθε υἱοὶ τοῦ Πατρὸς ὑμῶν τοῦ ἐν οὐρανοῖς, ὅτι τὸν ἥλιον αὐτοῦ ἀνατέλλει ἐπὶ πονηροὺς καὶ ἀγαθοὺς καὶ βρέχει ἐπὶ δικαίους καὶ ἀδίκους.
\vs{46}ἐὰν γὰρ ἀγαπήσητε τοὺς ἀγαπῶντας ὑμᾶς, τίνα μισθὸν ἔχετε; οὐχὶ καὶ οἱ τελῶναι τὸ αὐτὸ ποιοῦσιν;
\vs{47}καὶ ἐὰν ἀσπάσησθε τοὺς ἀδελφοὺς ὑμῶν μόνον, τί περισσὸν ποιεῖτε; οὐχὶ καὶ οἱ ἐθνικοὶ τὸ αὐτὸ ποιοῦσιν;
\vs{48}Ἔσεσθε οὖν ὑμεῖς τέλειοι ὡς ὁ Πατὴρ ὑμῶν ὁ οὐράνιος τέλειός ἐστιν.

\ch{6}
Προσέχετε δὲ τὴν δικαιοσύνην ὑμῶν μὴ ποιεῖν ἔμπροσθεν τῶν ἀνθρώπων πρὸς τὸ θεαθῆναι αὐτοῖς· εἰ δὲ μή γε, μισθὸν οὐκ ἔχετε παρὰ τῷ Πατρὶ ὑμῶν τῷ ἐν τοῖς οὐρανοῖς.
\vs{2}Ὅταν οὖν ποιῇς ἐλεημοσύνην, μὴ σαλπίσῃς ἔμπροσθέν σου, ὥσπερ οἱ ὑποκριταὶ ποιοῦσιν ἐν ταῖς συναγωγαῖς καὶ ἐν ταῖς ῥύμαις, ὅπως δοξασθῶσιν ὑπὸ τῶν ἀνθρώπων· ἀμὴν λέγω ὑμῖν, ἀπέχουσιν τὸν μισθὸν αὐτῶν.
\vs{3}σοῦ δὲ ποιοῦντος ἐλεημοσύνην μὴ γνώτω ἡ ἀριστερά σου τί ποιεῖ ἡ δεξιά σου,
\vs{4}ὅπως ᾖ σου ἡ ἐλεημοσύνη ἐν τῷ κρυπτῷ· καὶ ὁ Πατήρ σου ὁ βλέπων ἐν τῷ κρυπτῷ ἀποδώσει σοι.
\vs{5}Καὶ ὅταν προσεύχησθε, οὐκ ἔσεσθε ὡς οἱ ὑποκριταί, ὅτι φιλοῦσιν ἐν ταῖς συναγωγαῖς καὶ ἐν ταῖς γωνίαις τῶν πλατειῶν ἑστῶτες προσεύχεσθαι, ὅπως φανῶσιν τοῖς ἀνθρώποις· ἀμὴν λέγω ὑμῖν, ἀπέχουσιν τὸν μισθὸν αὐτῶν.
\vs{6}σὺ δὲ ὅταν προσεύχῃ, εἴσελθε εἰς τὸ ταμεῖόν σου καὶ κλείσας τὴν θύραν σου πρόσευξαι τῷ Πατρί σου τῷ ἐν τῷ κρυπτῷ· καὶ ὁ Πατήρ σου ὁ βλέπων ἐν τῷ κρυπτῷ ἀποδώσει σοι.
\vs{7}Προσευχόμενοι δὲ μὴ βατταλογήσητε ὥσπερ οἱ ἐθνικοί, δοκοῦσιν γὰρ ὅτι ἐν τῇ πολυλογίᾳ αὐτῶν εἰσακουσθήσονται.
\vs{8}μὴ οὖν ὁμοιωθῆτε αὐτοῖς· οἶδεν γὰρ ὁ Πατὴρ ὑμῶν ὧν χρείαν ἔχετε πρὸ τοῦ ὑμᾶς αἰτῆσαι αὐτόν.

\vs{9}Οὕτως οὖν προσεύχεσθε ὑμεῖς· 
\begin{poetryblock}

\begin{quote}Πάτερ ἡμῶν ὁ ἐν τοῖς οὐρανοῖς·\end{quote} 

\begin{quote}Ἁγιασθήτω τὸ ὄνομά σου·\end{quote}

\begin{quote} \vs{10}Ἐλθέτω ἡ βασιλεία σου·\end{quote} 

\begin{quote}Γενηθήτω τὸ θέλημά σου,\end{quote} 

\begin{quote}Ὡς ἐν οὐρανῷ καὶ ἐπὶ γῆς·\end{quote}

\begin{quote} \vs{11}Τὸν ἄρτον ἡμῶν τὸν ἐπιούσιον δὸς ἡμῖν σήμερον·\end{quote}

\begin{quote} \vs{12}Καὶ ἄφες ἡμῖν τὰ ὀφειλήματα ἡμῶν,\end{quote} 

\begin{quote}Ὡς καὶ ἡμεῖς ἀφήκαμεν τοῖς ὀφειλέταις ἡμῶν·\end{quote}

\begin{quote} \vs{13}Καὶ μὴ εἰσενέγκῃς ἡμᾶς εἰς πειρασμόν,\end{quote} 

\begin{quote}Ἀλλὰ ῥῦσαι ἡμᾶς ἀπὸ τοῦ πονηροῦ.\end{quote}
\end{poetryblock}

\vs{14}Ἐὰν γὰρ ἀφῆτε τοῖς ἀνθρώποις τὰ παραπτώματα αὐτῶν, ἀφήσει καὶ ὑμῖν ὁ Πατὴρ ὑμῶν ὁ οὐράνιος·
\vs{15}ἐὰν δὲ μὴ ἀφῆτε τοῖς ἀνθρώποις, οὐδὲ ὁ Πατὴρ ὑμῶν ἀφήσει τὰ παραπτώματα ὑμῶν.

\vs{16}Ὅταν δὲ νηστεύητε, μὴ γίνεσθε ὡς οἱ ὑποκριταὶ σκυθρωποί, ἀφανίζουσιν γὰρ τὰ πρόσωπα αὐτῶν ὅπως φανῶσιν τοῖς ἀνθρώποις νηστεύοντες· ἀμὴν λέγω ὑμῖν, ἀπέχουσιν τὸν μισθὸν αὐτῶν.
\vs{17}σὺ δὲ νηστεύων ἄλειψαί σου τὴν κεφαλὴν καὶ τὸ πρόσωπόν σου νίψαι,
\vs{18}ὅπως μὴ φανῇς τοῖς ἀνθρώποις νηστεύων ἀλλὰ τῷ Πατρί σου τῷ ἐν τῷ κρυφαίῳ· καὶ ὁ Πατήρ σου ὁ βλέπων ἐν τῷ κρυφαίῳ ἀποδώσει σοι.
\vs{19}Μὴ θησαυρίζετε ὑμῖν θησαυροὺς ἐπὶ τῆς γῆς, ὅπου σὴς καὶ βρῶσις ἀφανίζει καὶ ὅπου κλέπται διορύσσουσιν καὶ κλέπτουσιν·
\vs{20}θησαυρίζετε δὲ ὑμῖν θησαυροὺς ἐν οὐρανῷ, ὅπου οὔτε σὴς οὔτε βρῶσις ἀφανίζει καὶ ὅπου κλέπται οὐ διορύσσουσιν οὐδὲ κλέπτουσιν·
\vs{21}ὅπου γάρ ἐστιν ὁ θησαυρός σου, ἐκεῖ ἔσται καὶ ἡ καρδία σου.

\vs{22}Ὁ λύχνος τοῦ σώματός ἐστιν ὁ ὀφθαλμός. ἐὰν οὖν ᾖ ὁ ὀφθαλμός σου ἁπλοῦς, ὅλον τὸ σῶμά σου φωτεινὸν ἔσται·
\vs{23}ἐὰν δὲ ὁ ὀφθαλμός σου πονηρὸς ᾖ, ὅλον τὸ σῶμά σου σκοτεινὸν ἔσται. εἰ οὖν τὸ φῶς τὸ ἐν σοὶ σκότος ἐστίν, τὸ σκότος πόσον.

\vs{24}Οὐδεὶς δύναται δυσὶ κυρίοις δουλεύειν· ἢ γὰρ τὸν ἕνα μισήσει καὶ τὸν ἕτερον ἀγαπήσει, ἢ ἑνὸς ἀνθέξεται καὶ τοῦ ἑτέρου καταφρονήσει. οὐ δύνασθε Θεῷ δουλεύειν καὶ μαμωνᾷ.

\vs{25}Διὰ τοῦτο λέγω ὑμῖν· μὴ μεριμνᾶτε τῇ ψυχῇ ὑμῶν τί φάγητε ἢ τί πίητε, μηδὲ τῷ σώματι ὑμῶν τί ἐνδύσησθε. οὐχὶ ἡ ψυχὴ πλεῖόν ἐστιν τῆς τροφῆς καὶ τὸ σῶμα τοῦ ἐνδύματος;
\vs{26}ἐμβλέψατε εἰς τὰ πετεινὰ τοῦ οὐρανοῦ ὅτι οὐ σπείρουσιν οὐδὲ θερίζουσιν οὐδὲ συνάγουσιν εἰς ἀποθήκας, καὶ ὁ Πατὴρ ὑμῶν ὁ οὐράνιος τρέφει αὐτά· οὐχ ὑμεῖς μᾶλλον διαφέρετε αὐτῶν;
\vs{27}τίς δὲ ἐξ ὑμῶν μεριμνῶν δύναται προσθεῖναι ἐπὶ τὴν ἡλικίαν αὐτοῦ πῆχυν ἕνα;
\vs{28}Καὶ περὶ ἐνδύματος τί μεριμνᾶτε; καταμάθετε τὰ κρίνα τοῦ ἀγροῦ πῶς αὐξάνουσιν· οὐ κοπιῶσιν οὐδὲ νήθουσιν·
\vs{29}λέγω δὲ ὑμῖν ὅτι οὐδὲ Σολομὼν ἐν πάσῃ τῇ δόξῃ αὐτοῦ περιεβάλετο ὡς ἓν τούτων.
\vs{30}εἰ δὲ τὸν χόρτον τοῦ ἀγροῦ σήμερον ὄντα καὶ αὔριον εἰς κλίβανον βαλλόμενον ὁ Θεὸς οὕτως ἀμφιέννυσιν, οὐ πολλῷ μᾶλλον ὑμᾶς, ὀλιγόπιστοι;
\vs{31}Μὴ οὖν μεριμνήσητε λέγοντες· Τί φάγωμεν; ἤ· Τί πίωμεν; ἤ· Τί περιβαλώμεθα;
\vs{32}πάντα γὰρ ταῦτα τὰ ἔθνη ἐπιζητοῦσιν· οἶδεν γὰρ ὁ Πατὴρ ὑμῶν ὁ οὐράνιος ὅτι χρῄζετε τούτων ἁπάντων.
\vs{33}ζητεῖτε δὲ πρῶτον τὴν βασιλείαν τοῦ θεοῦ καὶ τὴν δικαιοσύνην αὐτοῦ, καὶ ταῦτα πάντα προστεθήσεται ὑμῖν.
\vs{34}Μὴ οὖν μεριμνήσητε εἰς τὴν αὔριον, ἡ γὰρ αὔριον μεριμνήσει ἑαυτῆς· ἀρκετὸν τῇ ἡμέρᾳ ἡ κακία αὐτῆς.

\ch{7}
Μὴ κρίνετε, ἵνα μὴ κριθῆτε·
\vs{2}ἐν ᾧ γὰρ κρίματι κρίνετε κριθήσεσθε, καὶ ἐν ᾧ μέτρῳ μετρεῖτε μετρηθήσεται ὑμῖν.
\vs{3}Τί δὲ βλέπεις τὸ κάρφος τὸ ἐν τῷ ὀφθαλμῷ τοῦ ἀδελφοῦ σου, τὴν δὲ ἐν τῷ σῷ ὀφθαλμῷ δοκὸν οὐ κατανοεῖς;
\vs{4}ἢ πῶς ἐρεῖς τῷ ἀδελφῷ σου· Ἄφες ἐκβάλω τὸ κάρφος ἐκ τοῦ ὀφθαλμοῦ σου, καὶ ἰδοὺ ἡ δοκὸς ἐν τῷ ὀφθαλμῷ σοῦ;
\vs{5}ὑποκριτά, ἔκβαλε πρῶτον ἐκ τοῦ ὀφθαλμοῦ σοῦ τὴν δοκόν, καὶ τότε διαβλέψεις ἐκβαλεῖν τὸ κάρφος ἐκ τοῦ ὀφθαλμοῦ τοῦ ἀδελφοῦ σου.

\vs{6}Μὴ δῶτε τὸ ἅγιον τοῖς κυσίν μηδὲ βάλητε τοὺς μαργαρίτας ὑμῶν ἔμπροσθεν τῶν χοίρων, μήποτε καταπατήσουσιν αὐτοὺς ἐν τοῖς ποσὶν αὐτῶν καὶ στραφέντες ῥήξωσιν ὑμᾶς.

\vs{7}Αἰτεῖτε καὶ δοθήσεται ὑμῖν, ζητεῖτε καὶ εὑρήσετε, κρούετε καὶ ἀνοιγήσεται ὑμῖν·
\vs{8}πᾶς γὰρ ὁ αἰτῶν λαμβάνει καὶ ὁ ζητῶν εὑρίσκει καὶ τῷ κρούοντι ἀνοιγήσεται.
\vs{9}Ἢ τίς ἐστιν ἐξ ὑμῶν ἄνθρωπος, ὃν αἰτήσει ὁ υἱὸς αὐτοῦ ἄρτον, μὴ λίθον ἐπιδώσει αὐτῷ;
\vs{10}ἢ καὶ ἰχθὺν αἰτήσει, μὴ ὄφιν ἐπιδώσει αὐτῷ;
\vs{11}εἰ οὖν ὑμεῖς πονηροὶ ὄντες οἴδατε δόματα ἀγαθὰ διδόναι τοῖς τέκνοις ὑμῶν, πόσῳ μᾶλλον ὁ Πατὴρ ὑμῶν ὁ ἐν τοῖς οὐρανοῖς δώσει ἀγαθὰ τοῖς αἰτοῦσιν αὐτόν.

\vs{12}Πάντα οὖν ὅσα ἐὰν θέλητε ἵνα ποιῶσιν ὑμῖν οἱ ἄνθρωποι, οὕτως καὶ ὑμεῖς ποιεῖτε αὐτοῖς· οὗτος γάρ ἐστιν ὁ νόμος καὶ οἱ προφῆται.

\vs{13}Εἰσέλθατε διὰ τῆς στενῆς πύλης· ὅτι πλατεῖα ἡ πύλη καὶ εὐρύχωρος ἡ ὁδὸς ἡ ἀπάγουσα εἰς τὴν ἀπώλειαν καὶ πολλοί εἰσιν οἱ εἰσερχόμενοι δι᾽ αὐτῆς·
\vs{14}ὅτι στενὴ ἡ πύλη καὶ τεθλιμμένη ἡ ὁδὸς ἡ ἀπάγουσα εἰς τὴν ζωήν καὶ ὀλίγοι εἰσὶν οἱ εὑρίσκοντες αὐτήν.

\vs{15}Προσέχετε ἀπὸ τῶν ψευδοπροφητῶν, οἵτινες ἔρχονται πρὸς ὑμᾶς ἐν ἐνδύμασιν προβάτων, ἔσωθεν δέ εἰσιν λύκοι ἅρπαγες.
\vs{16}ἀπὸ τῶν καρπῶν αὐτῶν ἐπιγνώσεσθε αὐτούς. μήτι συλλέγουσιν ἀπὸ ἀκανθῶν σταφυλὰς ἢ ἀπὸ τριβόλων σῦκα;
\vs{17}οὕτως πᾶν δένδρον ἀγαθὸν καρποὺς καλοὺς ποιεῖ, τὸ δὲ σαπρὸν δένδρον καρποὺς πονηροὺς ποιεῖ.
\vs{18}οὐ δύναται δένδρον ἀγαθὸν καρποὺς πονηροὺς ποιεῖν οὐδὲ δένδρον σαπρὸν καρποὺς καλοὺς ποιεῖν.
\vs{19}πᾶν δένδρον μὴ ποιοῦν καρπὸν καλὸν ἐκκόπτεται καὶ εἰς πῦρ βάλλεται.
\vs{20}ἄρα γε ἀπὸ τῶν καρπῶν αὐτῶν ἐπιγνώσεσθε αὐτούς.

\vs{21}Οὐ πᾶς ὁ λέγων μοι· Κύριε Κύριε, εἰσελεύσεται εἰς τὴν βασιλείαν τῶν οὐρανῶν, ἀλλ᾽ ὁ ποιῶν τὸ θέλημα τοῦ Πατρός μου τοῦ ἐν τοῖς οὐρανοῖς.
\vs{22}πολλοὶ ἐροῦσίν μοι ἐν ἐκείνῃ τῇ ἡμέρᾳ· Κύριε Κύριε, οὐ τῷ σῷ ὀνόματι ἐπροφητεύσαμεν, καὶ τῷ σῷ ὀνόματι δαιμόνια ἐξεβάλομεν, καὶ τῷ σῷ ὀνόματι δυνάμεις πολλὰς ἐποιήσαμεν;
\vs{23}καὶ τότε ὁμολογήσω αὐτοῖς ὅτι Οὐδέποτε ἔγνων ὑμᾶς· ἀποχωρεῖτε ἀπ᾽ ἐμοῦ οἱ ἐργαζόμενοι τὴν ἀνομίαν.

\vs{24}Πᾶς οὖν ὅστις ἀκούει μου τοὺς λόγους τούτους καὶ ποιεῖ αὐτούς, ὁμοιωθήσεται ἀνδρὶ φρονίμῳ, ὅστις ᾠκοδόμησεν αὐτοῦ τὴν οἰκίαν ἐπὶ τὴν πέτραν·
\vs{25}καὶ κατέβη ἡ βροχὴ καὶ ἦλθον οἱ ποταμοὶ καὶ ἔπνευσαν οἱ ἄνεμοι καὶ προσέπεσαν τῇ οἰκίᾳ ἐκείνῃ, καὶ οὐκ ἔπεσεν, τεθεμελίωτο γὰρ ἐπὶ τὴν πέτραν.
\vs{26}καὶ πᾶς ὁ ἀκούων μου τοὺς λόγους τούτους καὶ μὴ ποιῶν αὐτοὺς ὁμοιωθήσεται ἀνδρὶ μωρῷ, ὅστις ᾠκοδόμησεν αὐτοῦ τὴν οἰκίαν ἐπὶ τὴν ἄμμον·
\vs{27}καὶ κατέβη ἡ βροχὴ καὶ ἦλθον οἱ ποταμοὶ καὶ ἔπνευσαν οἱ ἄνεμοι καὶ προσέκοψαν τῇ οἰκίᾳ ἐκείνῃ, καὶ ἔπεσεν καὶ ἦν ἡ πτῶσις αὐτῆς μεγάλη.
\vs{28}Καὶ ἐγένετο ὅτε ἐτέλεσεν ὁ Ἰησοῦς τοὺς λόγους τούτους, ἐξεπλήσσοντο οἱ ὄχλοι ἐπὶ τῇ διδαχῇ αὐτοῦ·
\vs{29}ἦν γὰρ διδάσκων αὐτοὺς ὡς ἐξουσίαν ἔχων καὶ οὐχ ὡς οἱ γραμματεῖς αὐτῶν.

\ch{8}
Καταβάντος δὲ αὐτοῦ ἀπὸ τοῦ ὄρους ἠκολούθησαν αὐτῷ ὄχλοι πολλοί.
\vs{2}καὶ ἰδοὺ λεπρὸς προσελθὼν προσεκύνει αὐτῷ λέγων· Κύριε, ἐὰν θέλῃς δύνασαί με καθαρίσαι.
\vs{3}Καὶ ἐκτείνας τὴν χεῖρα ἥψατο αὐτοῦ λέγων· Θέλω, καθαρίσθητι· καὶ εὐθέως ἐκαθαρίσθη αὐτοῦ ἡ λέπρα.
\vs{4}καὶ λέγει αὐτῷ ὁ Ἰησοῦς· Ὅρα μηδενὶ εἴπῃς, ἀλλὰ ὕπαγε σεαυτὸν δεῖξον τῷ ἱερεῖ καὶ προσένεγκον τὸ δῶρον ὃ προσέταξεν Μωϋσῆς, εἰς μαρτύριον αὐτοῖς.

\vs{5}Εἰσελθόντος δὲ αὐτοῦ εἰς Καφαρναοὺμ προσῆλθεν αὐτῷ ἑκατόνταρχος παρακαλῶν αὐτὸν
\vs{6}καὶ λέγων· Κύριε, ὁ παῖς μου βέβληται ἐν τῇ οἰκίᾳ παραλυτικός, δεινῶς βασανιζόμενος.
\vs{7}Καὶ λέγει αὐτῷ· Ἐγὼ ἐλθὼν θεραπεύσω αὐτόν.
\vs{8}Καὶ ἀποκριθεὶς ὁ ἑκατόνταρχος ἔφη· Κύριε, οὐκ εἰμὶ ἱκανὸς ἵνα μου ὑπὸ τὴν στέγην εἰσέλθῃς, ἀλλὰ μόνον εἰπὲ λόγῳ, καὶ ἰαθήσεται ὁ παῖς μου.
\vs{9}καὶ γὰρ ἐγὼ ἄνθρωπός εἰμι ὑπὸ ἐξουσίαν, ἔχων ὑπ᾽ ἐμαυτὸν στρατιώτας, καὶ λέγω τούτῳ· Πορεύθητι, καὶ πορεύεται, καὶ ἄλλῳ· Ἔρχου, καὶ ἔρχεται, καὶ τῷ δούλῳ μου· Ποίησον τοῦτο, καὶ ποιεῖ.
\vs{10}Ἀκούσας δὲ ὁ Ἰησοῦς ἐθαύμασεν καὶ εἶπεν τοῖς ἀκολουθοῦσιν· Ἀμὴν λέγω ὑμῖν, παρ᾽ οὐδενὶ τοσαύτην πίστιν ἐν τῷ Ἰσραὴλ εὗρον.
\vs{11}λέγω δὲ ὑμῖν ὅτι πολλοὶ ἀπὸ ἀνατολῶν καὶ δυσμῶν ἥξουσιν καὶ ἀνακλιθήσονται μετὰ Ἀβραὰμ καὶ Ἰσαὰκ καὶ Ἰακὼβ ἐν τῇ βασιλείᾳ τῶν οὐρανῶν,
\vs{12}οἱ δὲ υἱοὶ τῆς βασιλείας ἐκβληθήσονται εἰς τὸ σκότος τὸ ἐξώτερον· ἐκεῖ ἔσται ὁ κλαυθμὸς καὶ ὁ βρυγμὸς τῶν ὀδόντων.
\vs{13}Καὶ εἶπεν ὁ Ἰησοῦς τῷ ἑκατοντάρχῃ· Ὕπαγε, ὡς ἐπίστευσας γενηθήτω σοι. καὶ ἰάθη ὁ παῖς αὐτοῦ ἐν τῇ ὥρᾳ ἐκείνῃ.

\vs{14}Καὶ ἐλθὼν ὁ Ἰησοῦς εἰς τὴν οἰκίαν Πέτρου εἶδεν τὴν πενθερὰν αὐτοῦ βεβλημένην καὶ πυρέσσουσαν·
\vs{15}καὶ ἥψατο τῆς χειρὸς αὐτῆς, καὶ ἀφῆκεν αὐτὴν ὁ πυρετός, καὶ ἠγέρθη καὶ διηκόνει αὐτῷ.

\vs{16}Ὀψίας δὲ γενομένης προσήνεγκαν αὐτῷ δαιμονιζομένους πολλούς· καὶ ἐξέβαλεν τὰ πνεύματα λόγῳ καὶ πάντας τοὺς κακῶς ἔχοντας ἐθεράπευσεν,
\vs{17}ὅπως πληρωθῇ τὸ ῥηθὲν διὰ Ἠσαΐου τοῦ προφήτου λέγοντος· 
\begin{poetryblock}

\begin{quote}Αὐτὸς τὰς ἀσθενείας ἡμῶν ἔλαβεν\end{quote} 

\begin{quote}καὶ τὰς νόσους ἐβάστασεν.\end{quote}
\end{poetryblock}

\vs{18}Ἰδὼν δὲ ὁ Ἰησοῦς ὄχλον περὶ αὐτὸν ἐκέλευσεν ἀπελθεῖν εἰς τὸ πέραν.
\vs{19}Καὶ προσελθὼν εἷς γραμματεὺς εἶπεν αὐτῷ· Διδάσκαλε, ἀκολουθήσω σοι ὅπου ἐὰν ἀπέρχῃ.
\vs{20}Καὶ λέγει αὐτῷ ὁ Ἰησοῦς· Αἱ ἀλώπεκες φωλεοὺς ἔχουσιν καὶ τὰ πετεινὰ τοῦ οὐρανοῦ κατασκηνώσεις, ὁ δὲ Υἱὸς τοῦ ἀνθρώπου οὐκ ἔχει ποῦ τὴν κεφαλὴν κλίνῃ.
\vs{21}Ἕτερος δὲ τῶν μαθητῶν αὐτοῦ εἶπεν αὐτῷ· Κύριε, ἐπίτρεψόν μοι πρῶτον ἀπελθεῖν καὶ θάψαι τὸν πατέρα μου.
\vs{22}Ὁ δὲ Ἰησοῦς λέγει αὐτῷ· Ἀκολούθει μοι καὶ ἄφες τοὺς νεκροὺς θάψαι τοὺς ἑαυτῶν νεκρούς.

\vs{23}Καὶ ἐμβάντι αὐτῷ εἰς τὸ πλοῖον ἠκολούθησαν αὐτῷ οἱ μαθηταὶ αὐτοῦ.
\vs{24}καὶ ἰδοὺ σεισμὸς μέγας ἐγένετο ἐν τῇ θαλάσσῃ, ὥστε τὸ πλοῖον καλύπτεσθαι ὑπὸ τῶν κυμάτων, αὐτὸς δὲ ἐκάθευδεν.
\vs{25}καὶ προσελθόντες ἤγειραν αὐτὸν λέγοντες· Κύριε, σῶσον, ἀπολλύμεθα.
\vs{26}Καὶ λέγει αὐτοῖς· Τί δειλοί ἐστε, ὀλιγόπιστοι; τότε ἐγερθεὶς ἐπετίμησεν τοῖς ἀνέμοις καὶ τῇ θαλάσσῃ, καὶ ἐγένετο γαλήνη μεγάλη.
\vs{27}Οἱ δὲ ἄνθρωποι ἐθαύμασαν λέγοντες· Ποταπός ἐστιν οὗτος ὅτι καὶ οἱ ἄνεμοι καὶ ἡ θάλασσα αὐτῷ ὑπακούουσιν;

\vs{28}Καὶ ἐλθόντος αὐτοῦ εἰς τὸ πέραν εἰς τὴν χώραν τῶν Γαδαρηνῶν ὑπήντησαν αὐτῷ δύο δαιμονιζόμενοι ἐκ τῶν μνημείων ἐξερχόμενοι, χαλεποὶ λίαν, ὥστε μὴ ἰσχύειν τινὰ παρελθεῖν διὰ τῆς ὁδοῦ ἐκείνης.
\vs{29}Καὶ ἰδοὺ ἔκραξαν λέγοντες· Τί ἡμῖν καὶ σοί, Υἱὲ τοῦ Θεοῦ; ἦλθες ὧδε πρὸ καιροῦ βασανίσαι ἡμᾶς;
\vs{30}Ἦν δὲ μακρὰν ἀπ᾽ αὐτῶν ἀγέλη χοίρων πολλῶν βοσκομένη.
\vs{31}οἱ δὲ δαίμονες παρεκάλουν αὐτὸν λέγοντες· Εἰ ἐκβάλλεις ἡμᾶς, ἀπόστειλον ἡμᾶς εἰς τὴν ἀγέλην τῶν χοίρων.
\vs{32}Καὶ εἶπεν αὐτοῖς· Ὑπάγετε. οἱ δὲ ἐξελθόντες ἀπῆλθον εἰς τοὺς χοίρους· καὶ ἰδοὺ ὥρμησεν πᾶσα ἡ ἀγέλη κατὰ τοῦ κρημνοῦ εἰς τὴν θάλασσαν καὶ ἀπέθανον ἐν τοῖς ὕδασιν.
\vs{33}Οἱ δὲ βόσκοντες ἔφυγον, καὶ ἀπελθόντες εἰς τὴν πόλιν ἀπήγγειλαν πάντα καὶ τὰ τῶν δαιμονιζομένων.
\vs{34}καὶ ἰδοὺ πᾶσα ἡ πόλις ἐξῆλθεν εἰς ὑπάντησιν τῷ Ἰησοῦ καὶ ἰδόντες αὐτὸν παρεκάλεσαν ὅπως μεταβῇ ἀπὸ τῶν ὁρίων αὐτῶν.

\ch{9}
Καὶ ἐμβὰς εἰς πλοῖον διεπέρασεν καὶ ἦλθεν εἰς τὴν ἰδίαν πόλιν.

\vs{2}Καὶ ἰδοὺ προσέφερον αὐτῷ παραλυτικὸν ἐπὶ κλίνης βεβλημένον. καὶ ἰδὼν ὁ Ἰησοῦς τὴν πίστιν αὐτῶν εἶπεν τῷ παραλυτικῷ· Θάρσει, τέκνον, ἀφίενταί σου αἱ ἁμαρτίαι.
\vs{3}Καὶ ἰδού τινες τῶν γραμματέων εἶπαν ἐν ἑαυτοῖς· Οὗτος βλασφημεῖ.
\vs{4}Καὶ ἰδὼν ὁ Ἰησοῦς τὰς ἐνθυμήσεις αὐτῶν εἶπεν· Ἵνατί ἐνθυμεῖσθε πονηρὰ ἐν ταῖς καρδίαις ὑμῶν;
\vs{5}τί γάρ ἐστιν εὐκοπώτερον, εἰπεῖν· Ἀφίενταί σου αἱ ἁμαρτίαι, ἢ εἰπεῖν· Ἔγειρε καὶ περιπάτει;
\vs{6}ἵνα δὲ εἰδῆτε ὅτι ἐξουσίαν ἔχει ὁ Υἱὸς τοῦ ἀνθρώπου ἐπὶ τῆς γῆς ἀφιέναι ἁμαρτίας— τότε λέγει τῷ παραλυτικῷ· Ἐγερθεὶς ἆρόν σου τὴν κλίνην καὶ ὕπαγε εἰς τὸν οἶκόν σου.
\vs{7}καὶ ἐγερθεὶς ἀπῆλθεν εἰς τὸν οἶκον αὐτοῦ.
\vs{8}Ἰδόντες δὲ οἱ ὄχλοι ἐφοβήθησαν καὶ ἐδόξασαν τὸν Θεὸν τὸν δόντα ἐξουσίαν τοιαύτην τοῖς ἀνθρώποις.

\vs{9}Καὶ παράγων ὁ Ἰησοῦς ἐκεῖθεν εἶδεν ἄνθρωπον καθήμενον ἐπὶ τὸ τελώνιον, Μαθθαῖον λεγόμενον, καὶ λέγει αὐτῷ· Ἀκολούθει μοι. καὶ ἀναστὰς ἠκολούθησεν αὐτῷ.

\vs{10}Καὶ ἐγένετο αὐτοῦ ἀνακειμένου ἐν τῇ οἰκίᾳ, καὶ ἰδοὺ πολλοὶ τελῶναι καὶ ἁμαρτωλοὶ ἐλθόντες συνανέκειντο τῷ Ἰησοῦ καὶ τοῖς μαθηταῖς αὐτοῦ.
\vs{11}καὶ ἰδόντες οἱ Φαρισαῖοι ἔλεγον τοῖς μαθηταῖς αὐτοῦ· Διὰ τί μετὰ τῶν τελωνῶν καὶ ἁμαρτωλῶν ἐσθίει ὁ διδάσκαλος ὑμῶν;
\vs{12}Ὁ δὲ ἀκούσας εἶπεν· Οὐ χρείαν ἔχουσιν οἱ ἰσχύοντες ἰατροῦ ἀλλ᾽ οἱ κακῶς ἔχοντες.
\vs{13}πορευθέντες δὲ μάθετε τί ἐστιν· Ἔλεος θέλω καὶ οὐ θυσίαν· οὐ γὰρ ἦλθον καλέσαι δικαίους ἀλλὰ ἁμαρτωλούς.
\vs{14}Τότε προσέρχονται αὐτῷ οἱ μαθηταὶ Ἰωάννου λέγοντες· Διὰ τί ἡμεῖς καὶ οἱ Φαρισαῖοι νηστεύομεν πολλά, οἱ δὲ μαθηταί σου οὐ νηστεύουσιν;
\vs{15}Καὶ εἶπεν αὐτοῖς ὁ Ἰησοῦς· Μὴ δύνανται οἱ υἱοὶ τοῦ νυμφῶνος πενθεῖν ἐφ᾽ ὅσον μετ᾽ αὐτῶν ἐστιν ὁ νυμφίος; ἐλεύσονται δὲ ἡμέραι ὅταν ἀπαρθῇ ἀπ᾽ αὐτῶν ὁ νυμφίος, καὶ τότε νηστεύσουσιν.
\vs{16}Οὐδεὶς δὲ ἐπιβάλλει ἐπίβλημα ῥάκους ἀγνάφου ἐπὶ ἱματίῳ παλαιῷ· αἴρει γὰρ τὸ πλήρωμα αὐτοῦ ἀπὸ τοῦ ἱματίου καὶ χεῖρον σχίσμα γίνεται.
\vs{17}Οὐδὲ βάλλουσιν οἶνον νέον εἰς ἀσκοὺς παλαιούς· εἰ δὲ μή γε, ῥήγνυνται οἱ ἀσκοί καὶ ὁ οἶνος ἐκχεῖται καὶ οἱ ἀσκοὶ ἀπόλλυνται· ἀλλὰ βάλλουσιν οἶνον νέον εἰς ἀσκοὺς καινούς, καὶ ἀμφότεροι συντηροῦνται.

\vs{18}Ταῦτα αὐτοῦ λαλοῦντος αὐτοῖς, ἰδοὺ ἄρχων εἷς ἐλθὼν προσεκύνει αὐτῷ λέγων ὅτι Ἡ θυγάτηρ μου ἄρτι ἐτελεύτησεν· ἀλλὰ ἐλθὼν ἐπίθες τὴν χεῖρά σου ἐπ᾽ αὐτήν, καὶ ζήσεται.
\vs{19}Καὶ ἐγερθεὶς ὁ Ἰησοῦς ἠκολούθει αὐτῷ καὶ οἱ μαθηταὶ αὐτοῦ.

\vs{20}Καὶ ἰδοὺ γυνὴ αἱμορροοῦσα δώδεκα ἔτη προσελθοῦσα ὄπισθεν ἥψατο τοῦ κρασπέδου τοῦ ἱματίου αὐτοῦ·
\vs{21}ἔλεγεν γὰρ ἐν ἑαυτῇ· Ἐὰν μόνον ἅψωμαι τοῦ ἱματίου αὐτοῦ σωθήσομαι.
\vs{22}Ὁ δὲ Ἰησοῦς στραφεὶς καὶ ἰδὼν αὐτὴν εἶπεν· Θάρσει, θύγατερ· ἡ πίστις σου σέσωκέν σε. καὶ ἐσώθη ἡ γυνὴ ἀπὸ τῆς ὥρας ἐκείνης.

\vs{23}Καὶ ἐλθὼν ὁ Ἰησοῦς εἰς τὴν οἰκίαν τοῦ ἄρχοντος καὶ ἰδὼν τοὺς αὐλητὰς καὶ τὸν ὄχλον θορυβούμενον
\vs{24}ἔλεγεν· Ἀναχωρεῖτε, οὐ γὰρ ἀπέθανεν τὸ κοράσιον ἀλλὰ καθεύδει. καὶ κατεγέλων αὐτοῦ.
\vs{25}Ὅτε δὲ ἐξεβλήθη ὁ ὄχλος εἰσελθὼν ἐκράτησεν τῆς χειρὸς αὐτῆς, καὶ ἠγέρθη τὸ κοράσιον.
\vs{26}καὶ ἐξῆλθεν ἡ φήμη αὕτη εἰς ὅλην τὴν γῆν ἐκείνην.

\vs{27}Καὶ παράγοντι ἐκεῖθεν τῷ Ἰησοῦ ἠκολούθησαν αὐτῷ δύο τυφλοὶ κράζοντες καὶ λέγοντες· Ἐλέησον ἡμᾶς, υἱὸς Δαυίδ.
\vs{28}Ἐλθόντι δὲ εἰς τὴν οἰκίαν προσῆλθον αὐτῷ οἱ τυφλοί, καὶ λέγει αὐτοῖς ὁ Ἰησοῦς· Πιστεύετε ὅτι δύναμαι τοῦτο ποιῆσαι; Λέγουσιν αὐτῷ· Ναί Κύριε.
\vs{29}Τότε ἥψατο τῶν ὀφθαλμῶν αὐτῶν λέγων· Κατὰ τὴν πίστιν ὑμῶν γενηθήτω ὑμῖν.
\vs{30}καὶ ἠνεῴχθησαν αὐτῶν οἱ ὀφθαλμοί. καὶ ἐνεβριμήθη αὐτοῖς ὁ Ἰησοῦς λέγων· Ὁρᾶτε μηδεὶς γινωσκέτω.
\vs{31}οἱ δὲ ἐξελθόντες διεφήμισαν αὐτὸν ἐν ὅλῃ τῇ γῇ ἐκείνῃ.

\vs{32}Αὐτῶν δὲ ἐξερχομένων ἰδοὺ προσήνεγκαν αὐτῷ ἄνθρωπον κωφὸν δαιμονιζόμενον.
\vs{33}καὶ ἐκβληθέντος τοῦ δαιμονίου ἐλάλησεν ὁ κωφός. καὶ ἐθαύμασαν οἱ ὄχλοι λέγοντες· Οὐδέποτε ἐφάνη οὕτως ἐν τῷ Ἰσραήλ.
\vs{34}Οἱ δὲ Φαρισαῖοι ἔλεγον· Ἐν τῷ ἄρχοντι τῶν δαιμονίων ἐκβάλλει τὰ δαιμόνια.

\vs{35}Καὶ περιῆγεν ὁ Ἰησοῦς τὰς πόλεις πάσας καὶ τὰς κώμας διδάσκων ἐν ταῖς συναγωγαῖς αὐτῶν καὶ κηρύσσων τὸ εὐαγγέλιον τῆς βασιλείας καὶ θεραπεύων πᾶσαν νόσον καὶ πᾶσαν μαλακίαν.
\vs{36}Ἰδὼν δὲ τοὺς ὄχλους ἐσπλαγχνίσθη περὶ αὐτῶν, ὅτι ἦσαν ἐσκυλμένοι καὶ ἐρριμμένοι ὡσεὶ πρόβατα μὴ ἔχοντα ποιμένα.
\vs{37}Τότε λέγει τοῖς μαθηταῖς αὐτοῦ· Ὁ μὲν θερισμὸς πολύς, οἱ δὲ ἐργάται ὀλίγοι·
\vs{38}δεήθητε οὖν τοῦ Κυρίου τοῦ θερισμοῦ ὅπως ἐκβάλῃ ἐργάτας εἰς τὸν θερισμὸν αὐτοῦ.

\ch{10}
Καὶ προσκαλεσάμενος τοὺς δώδεκα μαθητὰς αὐτοῦ ἔδωκεν αὐτοῖς ἐξουσίαν πνευμάτων ἀκαθάρτων ὥστε ἐκβάλλειν αὐτὰ καὶ θεραπεύειν πᾶσαν νόσον καὶ πᾶσαν μαλακίαν.
\vs{2}Τῶν δὲ δώδεκα ἀποστόλων τὰ ὀνόματά ἐστιν ταῦτα· πρῶτος Σίμων ὁ λεγόμενος Πέτρος καὶ Ἀνδρέας ὁ ἀδελφὸς αὐτοῦ, καὶ Ἰάκωβος ὁ τοῦ Ζεβεδαίου καὶ Ἰωάννης ὁ ἀδελφὸς αὐτοῦ,
\vs{3}Φίλιππος καὶ Βαρθολομαῖος, Θωμᾶς καὶ Μαθθαῖος ὁ τελώνης, Ἰάκωβος ὁ τοῦ Ἁλφαίου καὶ Θαδδαῖος,
\vs{4}Σίμων ὁ Καναναῖος καὶ Ἰούδας ὁ Ἰσκαριώτης ὁ καὶ παραδοὺς αὐτόν.

\vs{5}Τούτους τοὺς δώδεκα ἀπέστειλεν ὁ Ἰησοῦς παραγγείλας αὐτοῖς λέγων· Εἰς ὁδὸν ἐθνῶν μὴ ἀπέλθητε καὶ εἰς πόλιν Σαμαριτῶν μὴ εἰσέλθητε·
\vs{6}πορεύεσθε δὲ μᾶλλον πρὸς τὰ πρόβατα τὰ ἀπολωλότα οἴκου Ἰσραήλ.
\vs{7}πορευόμενοι δὲ κηρύσσετε λέγοντες ὅτι Ἤγγικεν ἡ βασιλεία τῶν οὐρανῶν.
\vs{8}ἀσθενοῦντας θεραπεύετε, νεκροὺς ἐγείρετε, λεπροὺς καθαρίζετε, δαιμόνια ἐκβάλλετε· δωρεὰν ἐλάβετε, δωρεὰν δότε.
\vs{9}Μὴ κτήσησθε χρυσὸν μηδὲ ἄργυρον μηδὲ χαλκὸν εἰς τὰς ζώνας ὑμῶν,
\vs{10}μὴ πήραν εἰς ὁδὸν μηδὲ δύο χιτῶνας μηδὲ ὑποδήματα μηδὲ ῥάβδον· ἄξιος γὰρ ὁ ἐργάτης τῆς τροφῆς αὐτοῦ.
\vs{11}Εἰς ἣν δ᾽ ἂν πόλιν ἢ κώμην εἰσέλθητε, ἐξετάσατε τίς ἐν αὐτῇ ἄξιός ἐστιν· κἀκεῖ μείνατε ἕως ἂν ἐξέλθητε.
\vs{12}εἰσερχόμενοι δὲ εἰς τὴν οἰκίαν ἀσπάσασθε αὐτήν·
\vs{13}καὶ ἐὰν μὲν ᾖ ἡ οἰκία ἀξία, ἐλθάτω ἡ εἰρήνη ὑμῶν ἐπ᾽ αὐτήν, ἐὰν δὲ μὴ ᾖ ἀξία, ἡ εἰρήνη ὑμῶν πρὸς ὑμᾶς ἐπιστραφήτω.
\vs{14}καὶ ὃς ἂν μὴ δέξηται ὑμᾶς μηδὲ ἀκούσῃ τοὺς λόγους ὑμῶν, ἐξερχόμενοι ἔξω τῆς οἰκίας ἢ τῆς πόλεως ἐκείνης ἐκτινάξατε τὸν κονιορτὸν τῶν ποδῶν ὑμῶν.
\vs{15}ἀμὴν λέγω ὑμῖν, ἀνεκτότερον ἔσται γῇ Σοδόμων καὶ Γομόρρων ἐν ἡμέρᾳ κρίσεως ἢ τῇ πόλει ἐκείνῃ.

\vs{16}Ἰδοὺ ἐγὼ ἀποστέλλω ὑμᾶς ὡς πρόβατα ἐν μέσῳ λύκων· γίνεσθε οὖν φρόνιμοι ὡς οἱ ὄφεις καὶ ἀκέραιοι ὡς αἱ περιστεραί.

\vs{17}Προσέχετε δὲ ἀπὸ τῶν ἀνθρώπων· παραδώσουσιν γὰρ ὑμᾶς εἰς συνέδρια καὶ ἐν ταῖς συναγωγαῖς αὐτῶν μαστιγώσουσιν ὑμᾶς·
\vs{18}καὶ ἐπὶ ἡγεμόνας δὲ καὶ βασιλεῖς ἀχθήσεσθε ἕνεκεν ἐμοῦ εἰς μαρτύριον αὐτοῖς καὶ τοῖς ἔθνεσιν.
\vs{19}ὅταν δὲ παραδῶσιν ὑμᾶς, μὴ μεριμνήσητε πῶς ἢ τί λαλήσητε· δοθήσεται γὰρ ὑμῖν ἐν ἐκείνῃ τῇ ὥρᾳ τί λαλήσητε·
\vs{20}οὐ γὰρ ὑμεῖς ἐστε οἱ λαλοῦντες ἀλλὰ τὸ Πνεῦμα τοῦ Πατρὸς ὑμῶν τὸ λαλοῦν ἐν ὑμῖν.

\vs{21}Παραδώσει δὲ ἀδελφὸς ἀδελφὸν εἰς θάνατον καὶ πατὴρ τέκνον, καὶ ἐπαναστήσονται τέκνα ἐπὶ γονεῖς καὶ θανατώσουσιν αὐτούς.
\vs{22}καὶ ἔσεσθε μισούμενοι ὑπὸ πάντων διὰ τὸ ὄνομά μου· ὁ δὲ ὑπομείνας εἰς τέλος οὗτος σωθήσεται.

\vs{23}Ὅταν δὲ διώκωσιν ὑμᾶς ἐν τῇ πόλει ταύτῃ, φεύγετε εἰς τὴν ἑτέραν· ἀμὴν γὰρ λέγω ὑμῖν, οὐ μὴ τελέσητε τὰς πόλεις τοῦ Ἰσραὴλ ἕως ἂν ἔλθῃ ὁ Υἱὸς τοῦ ἀνθρώπου.

\vs{24}Οὐκ ἔστιν μαθητὴς ὑπὲρ τὸν διδάσκαλον οὐδὲ δοῦλος ὑπὲρ τὸν κύριον αὐτοῦ.
\vs{25}ἀρκετὸν τῷ μαθητῇ ἵνα γένηται ὡς ὁ διδάσκαλος αὐτοῦ καὶ ὁ δοῦλος ὡς ὁ κύριος αὐτοῦ. εἰ τὸν οἰκοδεσπότην Βεελζεβοὺλ ἐπεκάλεσαν, πόσῳ μᾶλλον τοὺς οἰκιακοὺς αὐτοῦ.

\vs{26}Μὴ οὖν φοβηθῆτε αὐτούς· οὐδὲν γάρ ἐστιν κεκαλυμμένον ὃ οὐκ ἀποκαλυφθήσεται καὶ κρυπτὸν ὃ οὐ γνωσθήσεται.
\vs{27}ὃ λέγω ὑμῖν ἐν τῇ σκοτίᾳ εἴπατε ἐν τῷ φωτί, καὶ ὃ εἰς τὸ οὖς ἀκούετε κηρύξατε ἐπὶ τῶν δωμάτων.
\vs{28}Καὶ μὴ φοβεῖσθε ἀπὸ τῶν ἀποκτεννόντων τὸ σῶμα, τὴν δὲ ψυχὴν μὴ δυναμένων ἀποκτεῖναι· φοβεῖσθε δὲ μᾶλλον τὸν δυνάμενον καὶ ψυχὴν καὶ σῶμα ἀπολέσαι ἐν γεέννῃ.
\vs{29}Οὐχὶ δύο στρουθία ἀσσαρίου πωλεῖται; καὶ ἓν ἐξ αὐτῶν οὐ πεσεῖται ἐπὶ τὴν γῆν ἄνευ τοῦ Πατρὸς ὑμῶν.
\vs{30}ὑμῶν δὲ καὶ αἱ τρίχες τῆς κεφαλῆς πᾶσαι ἠριθμημέναι εἰσίν.
\vs{31}μὴ οὖν φοβεῖσθε· πολλῶν στρουθίων διαφέρετε ὑμεῖς.

\vs{32}Πᾶς οὖν ὅστις ὁμολογήσει ἐν ἐμοὶ ἔμπροσθεν τῶν ἀνθρώπων, ὁμολογήσω κἀγὼ ἐν αὐτῷ ἔμπροσθεν τοῦ Πατρός μου τοῦ ἐν τοῖς οὐρανοῖς·
\vs{33}ὅστις δ᾽ ἂν ἀρνήσηταί με ἔμπροσθεν τῶν ἀνθρώπων, ἀρνήσομαι κἀγὼ αὐτὸν ἔμπροσθεν τοῦ Πατρός μου τοῦ ἐν τοῖς οὐρανοῖς.

\vs{34}Μὴ νομίσητε ὅτι ἦλθον βαλεῖν εἰρήνην ἐπὶ τὴν γῆν· οὐκ ἦλθον βαλεῖν εἰρήνην ἀλλὰ μάχαιραν.
\vs{35}ἦλθον γὰρ διχάσαι Ἄνθρωπον κατὰ τοῦ πατρὸς αὐτοῦ Καὶ θυγατέρα κατὰ τῆς μητρὸς αὐτῆς Καὶ νύμφην κατὰ τῆς πενθερᾶς αὐτῆς,
\vs{36}Καὶ ἐχθροὶ τοῦ ἀνθρώπου οἱ οἰκιακοὶ αὐτοῦ.

\vs{37}Ὁ φιλῶν πατέρα ἢ μητέρα ὑπὲρ ἐμὲ οὐκ ἔστιν μου ἄξιος, καὶ ὁ φιλῶν υἱὸν ἢ θυγατέρα ὑπὲρ ἐμὲ οὐκ ἔστιν μου ἄξιος·
\vs{38}καὶ ὃς οὐ λαμβάνει τὸν σταυρὸν αὐτοῦ καὶ ἀκολουθεῖ ὀπίσω μου, οὐκ ἔστιν μου ἄξιος.
\vs{39}ὁ εὑρὼν τὴν ψυχὴν αὐτοῦ ἀπολέσει αὐτήν, καὶ ὁ ἀπολέσας τὴν ψυχὴν αὐτοῦ ἕνεκεν ἐμοῦ εὑρήσει αὐτήν.

\vs{40}Ὁ δεχόμενος ὑμᾶς ἐμὲ δέχεται, καὶ ὁ ἐμὲ δεχόμενος δέχεται τὸν ἀποστείλαντά με.
\vs{41}ὁ δεχόμενος προφήτην εἰς ὄνομα προφήτου μισθὸν προφήτου λήμψεται, καὶ ὁ δεχόμενος δίκαιον εἰς ὄνομα δικαίου μισθὸν δικαίου λήμψεται.
\vs{42}καὶ ὃς ἂν ποτίσῃ ἕνα τῶν μικρῶν τούτων ποτήριον ψυχροῦ μόνον εἰς ὄνομα μαθητοῦ, ἀμὴν λέγω ὑμῖν, οὐ μὴ ἀπολέσῃ τὸν μισθὸν αὐτοῦ.

\ch{11}
Καὶ ἐγένετο ὅτε ἐτέλεσεν ὁ Ἰησοῦς διατάσσων τοῖς δώδεκα μαθηταῖς αὐτοῦ, μετέβη ἐκεῖθεν τοῦ διδάσκειν καὶ κηρύσσειν ἐν ταῖς πόλεσιν αὐτῶν.

\vs{2}Ὁ δὲ Ἰωάννης ἀκούσας ἐν τῷ δεσμωτηρίῳ τὰ ἔργα τοῦ Χριστοῦ πέμψας διὰ τῶν μαθητῶν αὐτοῦ
\vs{3}εἶπεν αὐτῷ· Σὺ εἶ ὁ ἐρχόμενος ἢ ἕτερον προσδοκῶμεν;
\vs{4}Καὶ ἀποκριθεὶς ὁ Ἰησοῦς εἶπεν αὐτοῖς· Πορευθέντες ἀπαγγείλατε Ἰωάννῃ ἃ ἀκούετε καὶ βλέπετε·
\vs{5}τυφλοὶ ἀναβλέπουσιν καὶ χωλοὶ περιπατοῦσιν, λεπροὶ καθαρίζονται καὶ κωφοὶ ἀκούουσιν, καὶ νεκροὶ ἐγείρονται καὶ πτωχοὶ εὐαγγελίζονται·
\vs{6}καὶ μακάριός ἐστιν ὃς ἐὰν μὴ σκανδαλισθῇ ἐν ἐμοί.

\vs{7}Τούτων δὲ πορευομένων ἤρξατο ὁ Ἰησοῦς λέγειν τοῖς ὄχλοις περὶ Ἰωάννου· Τί ἐξήλθατε εἰς τὴν ἔρημον θεάσασθαι; κάλαμον ὑπὸ ἀνέμου σαλευόμενον;
\vs{8}ἀλλὰ τί ἐξήλθατε ἰδεῖν; ἄνθρωπον ἐν μαλακοῖς ἠμφιεσμένον; ἰδοὺ οἱ τὰ μαλακὰ φοροῦντες ἐν τοῖς οἴκοις τῶν βασιλέων εἰσίν.
\vs{9}ἀλλὰ τί ἐξήλθατε ἰδεῖν; προφήτην; ναί λέγω ὑμῖν, καὶ περισσότερον προφήτου.
\vs{10}οὗτός ἐστιν περὶ οὗ γέγραπται· 
\begin{poetryblock}

\begin{quote}Ἰδοὺ ἐγὼ ἀποστέλλω τὸν ἄγγελόν μου πρὸ προσώπου σου,\end{quote} 

\begin{quote}Ὃς κατασκευάσει τὴν ὁδόν σου ἔμπροσθέν σου.\end{quote}
\end{poetryblock}

\vs{11}Ἀμὴν λέγω ὑμῖν· οὐκ ἐγήγερται ἐν γεννητοῖς γυναικῶν μείζων Ἰωάννου τοῦ Βαπτιστοῦ· ὁ δὲ μικρότερος ἐν τῇ βασιλείᾳ τῶν οὐρανῶν μείζων αὐτοῦ ἐστιν.
\vs{12}ἀπὸ δὲ τῶν ἡμερῶν Ἰωάννου τοῦ Βαπτιστοῦ ἕως ἄρτι ἡ βασιλεία τῶν οὐρανῶν βιάζεται καὶ βιασταὶ ἁρπάζουσιν αὐτήν.
\vs{13}πάντες γὰρ οἱ προφῆται καὶ ὁ νόμος ἕως Ἰωάννου ἐπροφήτευσαν·
\vs{14}καὶ εἰ θέλετε δέξασθαι, αὐτός ἐστιν Ἠλίας ὁ μέλλων ἔρχεσθαι.
\vs{15}Ὁ ἔχων ὦτα ἀκουέτω.

\vs{16}Τίνι δὲ ὁμοιώσω τὴν γενεὰν ταύτην; ὁμοία ἐστὶν παιδίοις καθημένοις ἐν ταῖς ἀγοραῖς ἃ προσφωνοῦντα τοῖς ἑτέροις
\vs{17}λέγουσιν· 
\begin{poetryblock}

\begin{quote}Ηὐλήσαμεν ὑμῖν Καὶ οὐκ ὠρχήσασθε,\end{quote} 

\begin{quote}Ἐθρηνήσαμεν Καὶ οὐκ ἐκόψασθε.\end{quote}
\end{poetryblock}

\vs{18}Ἦλθεν γὰρ Ἰωάννης μήτε ἐσθίων μήτε πίνων, καὶ λέγουσιν· Δαιμόνιον ἔχει.
\vs{19}ἦλθεν ὁ Υἱὸς τοῦ ἀνθρώπου ἐσθίων καὶ πίνων, καὶ λέγουσιν· Ἰδοὺ ἄνθρωπος φάγος καὶ οἰνοπότης, τελωνῶν φίλος καὶ ἁμαρτωλῶν. καὶ ἐδικαιώθη ἡ σοφία ἀπὸ τῶν ἔργων αὐτῆς.

\vs{20}Τότε ἤρξατο ὀνειδίζειν τὰς πόλεις ἐν αἷς ἐγένοντο αἱ πλεῖσται δυνάμεις αὐτοῦ, ὅτι οὐ μετενόησαν·
\vs{21}Οὐαί σοι, Χοραζίν, οὐαί σοι, Βηθσαϊδά· ὅτι εἰ ἐν Τύρῳ καὶ Σιδῶνι ἐγένοντο αἱ δυνάμεις αἱ γενόμεναι ἐν ὑμῖν, πάλαι ἂν ἐν σάκκῳ καὶ σποδῷ μετενόησαν.
\vs{22}πλὴν λέγω ὑμῖν, Τύρῳ καὶ Σιδῶνι ἀνεκτότερον ἔσται ἐν ἡμέρᾳ κρίσεως ἢ ὑμῖν.
\vs{23}Καὶ σύ, Καφαρναούμ, μὴ ἕως οὐρανοῦ ὑψωθήσῃ; ἕως ᾅδου καταβήσῃ· ὅτι εἰ ἐν Σοδόμοις ἐγενήθησαν αἱ δυνάμεις αἱ γενόμεναι ἐν σοί, ἔμεινεν ἂν μέχρι τῆς σήμερον.
\vs{24}πλὴν λέγω ὑμῖν ὅτι γῇ Σοδόμων ἀνεκτότερον ἔσται ἐν ἡμέρᾳ κρίσεως ἢ σοί.

\vs{25}Ἐν ἐκείνῳ τῷ καιρῷ ἀποκριθεὶς ὁ Ἰησοῦς εἶπεν· Ἐξομολογοῦμαί σοι, Πάτερ, Κύριε τοῦ οὐρανοῦ καὶ τῆς γῆς, ὅτι ἔκρυψας ταῦτα ἀπὸ σοφῶν καὶ συνετῶν καὶ ἀπεκάλυψας αὐτὰ νηπίοις·
\vs{26}ναί ὁ Πατήρ, ὅτι οὕτως εὐδοκία ἐγένετο ἔμπροσθέν σου.
\vs{27}Πάντα μοι παρεδόθη ὑπὸ τοῦ Πατρός μου, καὶ οὐδεὶς ἐπιγινώσκει τὸν Υἱὸν εἰ μὴ ὁ Πατήρ, οὐδὲ τὸν Πατέρα τις ἐπιγινώσκει εἰ μὴ ὁ Υἱὸς καὶ ᾧ ἐὰν βούληται ὁ Υἱὸς ἀποκαλύψαι.

\vs{28}Δεῦτε πρός με πάντες οἱ κοπιῶντες καὶ πεφορτισμένοι, κἀγὼ ἀναπαύσω ὑμᾶς.
\vs{29}ἄρατε τὸν ζυγόν μου ἐφ᾽ ὑμᾶς καὶ μάθετε ἀπ᾽ ἐμοῦ, ὅτι πραΰς εἰμι καὶ ταπεινὸς τῇ καρδίᾳ, καὶ εὑρήσετε ἀνάπαυσιν ταῖς ψυχαῖς ὑμῶν·
\vs{30}ὁ γὰρ ζυγός μου χρηστὸς καὶ τὸ φορτίον μου ἐλαφρόν ἐστιν.

\ch{12}
Ἐν ἐκείνῳ τῷ καιρῷ ἐπορεύθη ὁ Ἰησοῦς τοῖς σάββασιν διὰ τῶν σπορίμων· οἱ δὲ μαθηταὶ αὐτοῦ ἐπείνασαν καὶ ἤρξαντο τίλλειν στάχυας καὶ ἐσθίειν.
\vs{2}οἱ δὲ Φαρισαῖοι ἰδόντες εἶπαν αὐτῷ· Ἰδοὺ οἱ μαθηταί σου ποιοῦσιν ὃ οὐκ ἔξεστιν ποιεῖν ἐν σαββάτῳ.
\vs{3}Ὁ δὲ εἶπεν αὐτοῖς· Οὐκ ἀνέγνωτε τί ἐποίησεν Δαυὶδ ὅτε ἐπείνασεν καὶ οἱ μετ᾽ αὐτοῦ,
\vs{4}πῶς εἰσῆλθεν εἰς τὸν οἶκον τοῦ Θεοῦ καὶ τοὺς ἄρτους τῆς προθέσεως ἔφαγον, ὃ οὐκ ἐξὸν ἦν αὐτῷ φαγεῖν οὐδὲ τοῖς μετ᾽ αὐτοῦ εἰ μὴ τοῖς ἱερεῦσιν μόνοις;
\vs{5}Ἢ οὐκ ἀνέγνωτε ἐν τῷ νόμῳ ὅτι τοῖς σάββασιν οἱ ἱερεῖς ἐν τῷ ἱερῷ τὸ σάββατον βεβηλοῦσιν καὶ ἀναίτιοί εἰσιν;
\vs{6}λέγω δὲ ὑμῖν ὅτι τοῦ ἱεροῦ μεῖζόν ἐστιν ὧδε.
\vs{7}Εἰ δὲ ἐγνώκειτε τί ἐστιν· Ἔλεος θέλω καὶ οὐ θυσίαν, οὐκ ἂν κατεδικάσατε τοὺς ἀναιτίους.
\vs{8}κύριος γάρ ἐστιν τοῦ σαββάτου ὁ Υἱὸς τοῦ ἀνθρώπου.

\vs{9}Καὶ μεταβὰς ἐκεῖθεν ἦλθεν εἰς τὴν συναγωγὴν αὐτῶν·
\vs{10}καὶ ἰδοὺ ἄνθρωπος χεῖρα ἔχων ξηράν. καὶ ἐπηρώτησαν αὐτὸν λέγοντες· Εἰ ἔξεστιν τοῖς σάββασιν θεραπεῦσαι; ἵνα κατηγορήσωσιν αὐτοῦ.
\vs{11}Ὁ δὲ εἶπεν αὐτοῖς· Τίς ἔσται ἐξ ὑμῶν ἄνθρωπος ὃς ἕξει πρόβατον ἕν καὶ ἐὰν ἐμπέσῃ τοῦτο τοῖς σάββασιν εἰς βόθυνον, οὐχὶ κρατήσει αὐτὸ καὶ ἐγερεῖ;
\vs{12}πόσῳ οὖν διαφέρει ἄνθρωπος προβάτου. ὥστε ἔξεστιν τοῖς σάββασιν καλῶς ποιεῖν.
\vs{13}Τότε λέγει τῷ ἀνθρώπῳ· Ἔκτεινόν σου τὴν χεῖρα. καὶ ἐξέτεινεν καὶ ἀπεκατεστάθη ὑγιὴς ὡς ἡ ἄλλη.
\vs{14}ἐξελθόντες δὲ οἱ Φαρισαῖοι συμβούλιον ἔλαβον κατ᾽ αὐτοῦ ὅπως αὐτὸν ἀπολέσωσιν.
\vs{15}Ὁ δὲ Ἰησοῦς γνοὺς ἀνεχώρησεν ἐκεῖθεν. καὶ ἠκολούθησαν αὐτῷ ὄχλοι πολλοί, καὶ ἐθεράπευσεν αὐτοὺς πάντας
\vs{16}καὶ ἐπετίμησεν αὐτοῖς ἵνα μὴ φανερὸν αὐτὸν ποιήσωσιν,
\vs{17}ἵνα πληρωθῇ τὸ ῥηθὲν διὰ Ἠσαΐου τοῦ προφήτου λέγοντος·
\begin{poetryblock}

\begin{quote} \vs{18}Ἰδοὺ ὁ παῖς μου ὃν ᾑρέτισα,\end{quote} 

\begin{quote}ὁ ἀγαπητός μου εἰς ὃν εὐδόκησεν ἡ ψυχή μου·\end{quote} 

\begin{quote}θήσω τὸ Πνεῦμά μου ἐπ᾽ αὐτόν,\end{quote} 

\begin{quote}καὶ κρίσιν τοῖς ἔθνεσιν ἀπαγγελεῖ.\end{quote}

\begin{quote} \vs{19}οὐκ ἐρίσει οὐδὲ κραυγάσει,\end{quote} 

\begin{quote}οὐδὲ ἀκούσει τις ἐν ταῖς πλατείαις τὴν φωνὴν αὐτοῦ.\end{quote}

\begin{quote} \vs{20}κάλαμον συντετριμμένον οὐ κατεάξει\end{quote} 

\begin{quote}καὶ λίνον τυφόμενον οὐ σβέσει,\end{quote} 

\begin{quote}ἕως ἂν ἐκβάλῃ εἰς νῖκος τὴν κρίσιν.\end{quote}

\begin{quote} \vs{21}καὶ τῷ ὀνόματι αὐτοῦ ἔθνη ἐλπιοῦσιν.\end{quote}
\end{poetryblock}

\vs{22}Τότε προσηνέχθη αὐτῷ δαιμονιζόμενος τυφλὸς καὶ κωφός, καὶ ἐθεράπευσεν αὐτόν, ὥστε τὸν κωφὸν λαλεῖν καὶ βλέπειν.
\vs{23}καὶ ἐξίσταντο πάντες οἱ ὄχλοι καὶ ἔλεγον· Μήτι οὗτός ἐστιν ὁ υἱὸς Δαυίδ;
\vs{24}Οἱ δὲ Φαρισαῖοι ἀκούσαντες εἶπον· Οὗτος οὐκ ἐκβάλλει τὰ δαιμόνια εἰ μὴ ἐν τῷ Βεελζεβοὺλ ἄρχοντι τῶν δαιμονίων.
\vs{25}Εἰδὼς δὲ τὰς ἐνθυμήσεις αὐτῶν εἶπεν αὐτοῖς· Πᾶσα βασιλεία μερισθεῖσα καθ᾽ ἑαυτῆς ἐρημοῦται καὶ πᾶσα πόλις ἢ οἰκία μερισθεῖσα καθ᾽ ἑαυτῆς οὐ σταθήσεται.
\vs{26}καὶ εἰ ὁ Σατανᾶς τὸν Σατανᾶν ἐκβάλλει, ἐφ᾽ ἑαυτὸν ἐμερίσθη· πῶς οὖν σταθήσεται ἡ βασιλεία αὐτοῦ;
\vs{27}Καὶ εἰ ἐγὼ ἐν Βεελζεβοὺλ ἐκβάλλω τὰ δαιμόνια, οἱ υἱοὶ ὑμῶν ἐν τίνι ἐκβάλλουσιν; διὰ τοῦτο αὐτοὶ κριταὶ ἔσονται ὑμῶν.
\vs{28}εἰ δὲ ἐν Πνεύματι Θεοῦ ἐγὼ ἐκβάλλω τὰ δαιμόνια, ἄρα ἔφθασεν ἐφ᾽ ὑμᾶς ἡ βασιλεία τοῦ Θεοῦ.
\vs{29}Ἢ πῶς δύναταί τις εἰσελθεῖν εἰς τὴν οἰκίαν τοῦ ἰσχυροῦ καὶ τὰ σκεύη αὐτοῦ ἁρπάσαι, ἐὰν μὴ πρῶτον δήσῃ τὸν ἰσχυρόν; καὶ τότε τὴν οἰκίαν αὐτοῦ διαρπάσει.
\vs{30}Ὁ μὴ ὢν μετ᾽ ἐμοῦ κατ᾽ ἐμοῦ ἐστιν, καὶ ὁ μὴ συνάγων μετ᾽ ἐμοῦ σκορπίζει.
\vs{31}Διὰ τοῦτο λέγω ὑμῖν, πᾶσα ἁμαρτία καὶ βλασφημία ἀφεθήσεται τοῖς ἀνθρώποις, ἡ δὲ τοῦ Πνεύματος βλασφημία οὐκ ἀφεθήσεται.
\vs{32}καὶ ὃς ἐὰν εἴπῃ λόγον κατὰ τοῦ Υἱοῦ τοῦ ἀνθρώπου, ἀφεθήσεται αὐτῷ· ὃς δ᾽ ἂν εἴπῃ κατὰ τοῦ Πνεύματος τοῦ Ἁγίου, οὐκ ἀφεθήσεται αὐτῷ οὔτε ἐν τούτῳ τῷ αἰῶνι οὔτε ἐν τῷ μέλλοντι.

\vs{33}Ἢ ποιήσατε τὸ δένδρον καλὸν καὶ τὸν καρπὸν αὐτοῦ καλόν, ἢ ποιήσατε τὸ δένδρον σαπρὸν καὶ τὸν καρπὸν αὐτοῦ σαπρόν· ἐκ γὰρ τοῦ καρποῦ τὸ δένδρον γινώσκεται.
\vs{34}γεννήματα ἐχιδνῶν, πῶς δύνασθε ἀγαθὰ λαλεῖν πονηροὶ ὄντες; ἐκ γὰρ τοῦ περισσεύματος τῆς καρδίας τὸ στόμα λαλεῖ.
\vs{35}ὁ ἀγαθὸς ἄνθρωπος ἐκ τοῦ ἀγαθοῦ θησαυροῦ ἐκβάλλει ἀγαθά, καὶ ὁ πονηρὸς ἄνθρωπος ἐκ τοῦ πονηροῦ θησαυροῦ ἐκβάλλει πονηρά.
\vs{36}λέγω δὲ ὑμῖν ὅτι πᾶν ῥῆμα ἀργὸν ὃ λαλήσουσιν οἱ ἄνθρωποι ἀποδώσουσιν περὶ αὐτοῦ λόγον ἐν ἡμέρᾳ κρίσεως·
\vs{37}ἐκ γὰρ τῶν λόγων σου δικαιωθήσῃ, καὶ ἐκ τῶν λόγων σου καταδικασθήσῃ.

\vs{38}Τότε ἀπεκρίθησαν αὐτῷ τινες τῶν γραμματέων καὶ Φαρισαίων λέγοντες· Διδάσκαλε, θέλομεν ἀπὸ σοῦ σημεῖον ἰδεῖν.
\vs{39}Ὁ δὲ ἀποκριθεὶς εἶπεν αὐτοῖς· Γενεὰ πονηρὰ καὶ μοιχαλὶς σημεῖον ἐπιζητεῖ, καὶ σημεῖον οὐ δοθήσεται αὐτῇ εἰ μὴ τὸ σημεῖον Ἰωνᾶ τοῦ προφήτου.
\vs{40}ὥσπερ γὰρ ἦν Ἰωνᾶς ἐν τῇ κοιλίᾳ τοῦ κήτους τρεῖς ἡμέρας καὶ τρεῖς νύκτας, οὕτως ἔσται ὁ Υἱὸς τοῦ ἀνθρώπου ἐν τῇ καρδίᾳ τῆς γῆς τρεῖς ἡμέρας καὶ τρεῖς νύκτας.
\vs{41}Ἄνδρες Νινευῖται ἀναστήσονται ἐν τῇ κρίσει μετὰ τῆς γενεᾶς ταύτης καὶ κατακρινοῦσιν αὐτήν, ὅτι μετενόησαν εἰς τὸ κήρυγμα Ἰωνᾶ, καὶ ἰδοὺ πλεῖον Ἰωνᾶ ὧδε.
\vs{42}βασίλισσα νότου ἐγερθήσεται ἐν τῇ κρίσει μετὰ τῆς γενεᾶς ταύτης καὶ κατακρινεῖ αὐτήν, ὅτι ἦλθεν ἐκ τῶν περάτων τῆς γῆς ἀκοῦσαι τὴν σοφίαν Σολομῶνος, καὶ ἰδοὺ πλεῖον Σολομῶνος ὧδε.

\vs{43}Ὅταν δὲ τὸ ἀκάθαρτον πνεῦμα ἐξέλθῃ ἀπὸ τοῦ ἀνθρώπου, διέρχεται δι᾽ ἀνύδρων τόπων ζητοῦν ἀνάπαυσιν καὶ οὐχ εὑρίσκει.
\vs{44}τότε λέγει· Εἰς τὸν οἶκόν μου ἐπιστρέψω ὅθεν ἐξῆλθον· καὶ ἐλθὸν εὑρίσκει σχολάζοντα σεσαρωμένον καὶ κεκοσμημένον.
\vs{45}τότε πορεύεται καὶ παραλαμβάνει μεθ᾽ ἑαυτοῦ ἑπτὰ ἕτερα πνεύματα πονηρότερα ἑαυτοῦ καὶ εἰσελθόντα κατοικεῖ ἐκεῖ· καὶ γίνεται τὰ ἔσχατα τοῦ ἀνθρώπου ἐκείνου χείρονα τῶν πρώτων. οὕτως ἔσται καὶ τῇ γενεᾷ ταύτῃ τῇ πονηρᾷ.

\vs{46}Ἔτι αὐτοῦ λαλοῦντος τοῖς ὄχλοις ἰδοὺ ἡ μήτηρ καὶ οἱ ἀδελφοὶ αὐτοῦ εἱστήκεισαν ἔξω ζητοῦντες αὐτῷ λαλῆσαι.
\vs{47}εἶπεν δέ τις αὐτῷ· Ἰδοὺ ἡ μήτηρ σου καὶ οἱ ἀδελφοί σου ἔξω ἑστήκασιν ζητοῦντές σοι λαλῆσαι.
\vs{48}Ὁ δὲ ἀποκριθεὶς εἶπεν τῷ λέγοντι αὐτῷ· Τίς ἐστιν ἡ μήτηρ μου καὶ τίνες εἰσὶν οἱ ἀδελφοί μου;
\vs{49}καὶ ἐκτείνας τὴν χεῖρα αὐτοῦ ἐπὶ τοὺς μαθητὰς αὐτοῦ εἶπεν· Ἰδοὺ ἡ μήτηρ μου καὶ οἱ ἀδελφοί μου.
\vs{50}ὅστις γὰρ ἂν ποιήσῃ τὸ θέλημα τοῦ Πατρός μου τοῦ ἐν οὐρανοῖς αὐτός μου ἀδελφὸς καὶ ἀδελφὴ καὶ μήτηρ ἐστίν.

\ch{13}
Ἐν τῇ ἡμέρᾳ ἐκείνῃ ἐξελθὼν ὁ Ἰησοῦς τῆς οἰκίας ἐκάθητο παρὰ τὴν θάλασσαν·
\vs{2}καὶ συνήχθησαν πρὸς αὐτὸν ὄχλοι πολλοί, ὥστε αὐτὸν εἰς πλοῖον ἐμβάντα καθῆσθαι, καὶ πᾶς ὁ ὄχλος ἐπὶ τὸν αἰγιαλὸν εἱστήκει.

\vs{3}Καὶ ἐλάλησεν αὐτοῖς πολλὰ ἐν παραβολαῖς λέγων· Ἰδοὺ ἐξῆλθεν ὁ σπείρων τοῦ σπείρειν.
\vs{4}καὶ ἐν τῷ σπείρειν αὐτὸν ἃ μὲν ἔπεσεν παρὰ τὴν ὁδόν, καὶ ἐλθόντα τὰ πετεινὰ κατέφαγεν αὐτά.
\vs{5}Ἄλλα δὲ ἔπεσεν ἐπὶ τὰ πετρώδη ὅπου οὐκ εἶχεν γῆν πολλήν, καὶ εὐθέως ἐξανέτειλεν διὰ τὸ μὴ ἔχειν βάθος γῆς·
\vs{6}ἡλίου δὲ ἀνατείλαντος ἐκαυματίσθη καὶ διὰ τὸ μὴ ἔχειν ῥίζαν ἐξηράνθη.
\vs{7}Ἄλλα δὲ ἔπεσεν ἐπὶ τὰς ἀκάνθας, καὶ ἀνέβησαν αἱ ἄκανθαι καὶ ἔπνιξαν αὐτά.
\vs{8}Ἄλλα δὲ ἔπεσεν ἐπὶ τὴν γῆν τὴν καλὴν καὶ ἐδίδου καρπόν, ὃ μὲν ἑκατὸν, ὃ δὲ ἑξήκοντα, ὃ δὲ τριάκοντα.
\vs{9}Ὁ ἔχων ὦτα ἀκουέτω.

\vs{10}Καὶ προσελθόντες οἱ μαθηταὶ εἶπαν αὐτῷ· Διὰ τί ἐν παραβολαῖς λαλεῖς αὐτοῖς;
\vs{11}Ὁ δὲ ἀποκριθεὶς εἶπεν αὐτοῖς· Ὅτι Ὑμῖν δέδοται γνῶναι τὰ μυστήρια τῆς βασιλείας τῶν οὐρανῶν, ἐκείνοις δὲ οὐ δέδοται.
\vs{12}ὅστις γὰρ ἔχει, δοθήσεται αὐτῷ καὶ περισσευθήσεται· ὅστις δὲ οὐκ ἔχει, καὶ ὃ ἔχει ἀρθήσεται ἀπ᾽ αὐτοῦ.
\vs{13}διὰ τοῦτο ἐν παραβολαῖς αὐτοῖς λαλῶ, Ὅτι βλέποντες οὐ βλέπουσιν Καὶ ἀκούοντες οὐκ ἀκούουσιν οὐδὲ συνίουσιν,
\vs{14}Καὶ ἀναπληροῦται αὐτοῖς ἡ προφητεία Ἠσαΐου ἡ λέγουσα· 
\begin{poetryblock}

\begin{quote}Ἀκοῇ ἀκούσετε καὶ οὐ μὴ συνῆτε,\end{quote} 

\begin{quote}Καὶ βλέποντες βλέψετε καὶ οὐ μὴ ἴδητε.\end{quote}

\begin{quote} \vs{15}Ἐπαχύνθη γὰρ ἡ καρδία τοῦ λαοῦ τούτου,\end{quote} 

\begin{quote}Καὶ τοῖς ὠσὶν βαρέως ἤκουσαν\end{quote} 

\begin{quote}Καὶ τοὺς ὀφθαλμοὺς αὐτῶν ἐκάμμυσαν,\end{quote} 

\begin{quote}Μήποτε ἴδωσιν τοῖς ὀφθαλμοῖς\end{quote} 

\begin{quote}Καὶ τοῖς ὠσὶν ἀκούσωσιν\end{quote} 

\begin{quote}Καὶ τῇ καρδίᾳ συνῶσιν\end{quote} 

\begin{quote}Καὶ ἐπιστρέψωσιν Καὶ ἰάσομαι αὐτούς.\end{quote}
\end{poetryblock}

\vs{16}Ὑμῶν δὲ μακάριοι οἱ ὀφθαλμοὶ ὅτι βλέπουσιν καὶ τὰ ὦτα ὑμῶν ὅτι ἀκούουσιν.
\vs{17}ἀμὴν γὰρ λέγω ὑμῖν ὅτι πολλοὶ προφῆται καὶ δίκαιοι ἐπεθύμησαν ἰδεῖν ἃ βλέπετε καὶ οὐκ εἶδαν, καὶ ἀκοῦσαι ἃ ἀκούετε καὶ οὐκ ἤκουσαν.

\vs{18}Ὑμεῖς οὖν ἀκούσατε τὴν παραβολὴν τοῦ σπείραντος.
\vs{19}Παντὸς ἀκούοντος τὸν λόγον τῆς βασιλείας καὶ μὴ συνιέντος ἔρχεται ὁ πονηρὸς καὶ ἁρπάζει τὸ ἐσπαρμένον ἐν τῇ καρδίᾳ αὐτοῦ, οὗτός ἐστιν ὁ παρὰ τὴν ὁδὸν σπαρείς.
\vs{20}Ὁ δὲ ἐπὶ τὰ πετρώδη σπαρείς, οὗτός ἐστιν ὁ τὸν λόγον ἀκούων καὶ εὐθὺς μετὰ χαρᾶς λαμβάνων αὐτόν,
\vs{21}οὐκ ἔχει δὲ ῥίζαν ἐν ἑαυτῷ ἀλλὰ πρόσκαιρός ἐστιν, γενομένης δὲ θλίψεως ἢ διωγμοῦ διὰ τὸν λόγον εὐθὺς σκανδαλίζεται.
\vs{22}Ὁ δὲ εἰς τὰς ἀκάνθας σπαρείς, οὗτός ἐστιν ὁ τὸν λόγον ἀκούων, καὶ ἡ μέριμνα τοῦ αἰῶνος καὶ ἡ ἀπάτη τοῦ πλούτου συμπνίγει τὸν λόγον καὶ ἄκαρπος γίνεται.
\vs{23}Ὁ δὲ ἐπὶ τὴν καλὴν γῆν σπαρείς, οὗτός ἐστιν ὁ τὸν λόγον ἀκούων καὶ συνιείς, ὃς δὴ καρποφορεῖ καὶ ποιεῖ ὃ μὲν ἑκατὸν, ὃ δὲ ἑξήκοντα, ὃ δὲ τριάκοντα.

\vs{24}Ἄλλην παραβολὴν παρέθηκεν αὐτοῖς λέγων· Ὡμοιώθη ἡ βασιλεία τῶν οὐρανῶν ἀνθρώπῳ σπείραντι καλὸν σπέρμα ἐν τῷ ἀγρῷ αὐτοῦ.
\vs{25}ἐν δὲ τῷ καθεύδειν τοὺς ἀνθρώπους ἦλθεν αὐτοῦ ὁ ἐχθρὸς καὶ ἐπέσπειρεν ζιζάνια ἀνὰ μέσον τοῦ σίτου καὶ ἀπῆλθεν.
\vs{26}ὅτε δὲ ἐβλάστησεν ὁ χόρτος καὶ καρπὸν ἐποίησεν, τότε ἐφάνη καὶ τὰ ζιζάνια.
\vs{27}Προσελθόντες δὲ οἱ δοῦλοι τοῦ οἰκοδεσπότου εἶπον αὐτῷ· Κύριε, οὐχὶ καλὸν σπέρμα ἔσπειρας ἐν τῷ σῷ ἀγρῷ; πόθεν οὖν ἔχει ζιζάνια;
\vs{28}Ὁ δὲ ἔφη αὐτοῖς· Ἐχθρὸς ἄνθρωπος τοῦτο ἐποίησεν. Οἱ δὲ δοῦλοι λέγουσιν αὐτῷ· Θέλεις οὖν ἀπελθόντες συλλέξωμεν αὐτά;
\vs{29}Ὁ δέ φησιν· Οὔ, μήποτε συλλέγοντες τὰ ζιζάνια ἐκριζώσητε ἅμα αὐτοῖς τὸν σῖτον.
\vs{30}ἄφετε συναυξάνεσθαι ἀμφότερα ἕως τοῦ θερισμοῦ, καὶ ἐν καιρῷ τοῦ θερισμοῦ ἐρῶ τοῖς θερισταῖς· Συλλέξατε πρῶτον τὰ ζιζάνια καὶ δήσατε αὐτὰ εἰς δέσμας πρὸς τὸ κατακαῦσαι αὐτά, τὸν δὲ σῖτον συναγάγετε εἰς τὴν ἀποθήκην μου.

\vs{31}Ἄλλην παραβολὴν παρέθηκεν αὐτοῖς λέγων· Ὁμοία ἐστὶν ἡ βασιλεία τῶν οὐρανῶν κόκκῳ σινάπεως, ὃν λαβὼν ἄνθρωπος ἔσπειρεν ἐν τῷ ἀγρῷ αὐτοῦ·
\vs{32}ὃ μικρότερον μέν ἐστιν πάντων τῶν σπερμάτων, ὅταν δὲ αὐξηθῇ μεῖζον τῶν λαχάνων ἐστὶν καὶ γίνεται δένδρον, ὥστε ἐλθεῖν τὰ πετεινὰ τοῦ οὐρανοῦ καὶ κατασκηνοῦν ἐν τοῖς κλάδοις αὐτοῦ.

\vs{33}Ἄλλην παραβολὴν ἐλάλησεν αὐτοῖς· Ὁμοία ἐστὶν ἡ βασιλεία τῶν οὐρανῶν ζύμῃ, ἣν λαβοῦσα γυνὴ ἐνέκρυψεν εἰς ἀλεύρου σάτα τρία ἕως οὗ ἐζυμώθη ὅλον.
\vs{34}Ταῦτα πάντα ἐλάλησεν ὁ Ἰησοῦς ἐν παραβολαῖς τοῖς ὄχλοις καὶ χωρὶς παραβολῆς οὐδὲν ἐλάλει αὐτοῖς,
\vs{35}ὅπως πληρωθῇ τὸ ῥηθὲν διὰ τοῦ προφήτου λέγοντος· 
\begin{poetryblock}

\begin{quote}Ἀνοίξω ἐν παραβολαῖς τὸ στόμα μου,\end{quote} 

\begin{quote}ἐρεύξομαι κεκρυμμένα ἀπὸ καταβολῆς κόσμου.\end{quote}
\end{poetryblock}

\vs{36}Τότε ἀφεὶς τοὺς ὄχλους ἦλθεν εἰς τὴν οἰκίαν. Καὶ προσῆλθον αὐτῷ οἱ μαθηταὶ αὐτοῦ λέγοντες· Διασάφησον ἡμῖν τὴν παραβολὴν τῶν ζιζανίων τοῦ ἀγροῦ.
\vs{37}Ὁ δὲ ἀποκριθεὶς εἶπεν· Ὁ σπείρων τὸ καλὸν σπέρμα ἐστὶν ὁ Υἱὸς τοῦ ἀνθρώπου,
\vs{38}ὁ δὲ ἀγρός ἐστιν ὁ κόσμος, τὸ δὲ καλὸν σπέρμα οὗτοί εἰσιν οἱ υἱοὶ τῆς βασιλείας· τὰ δὲ ζιζάνιά εἰσιν οἱ υἱοὶ τοῦ πονηροῦ,
\vs{39}ὁ δὲ ἐχθρὸς ὁ σπείρας αὐτά ἐστιν ὁ διάβολος, ὁ δὲ θερισμὸς συντέλεια αἰῶνός ἐστιν, οἱ δὲ θερισταὶ ἄγγελοί εἰσιν.
\vs{40}Ὥσπερ οὖν συλλέγεται τὰ ζιζάνια καὶ πυρὶ κατακαίεται, οὕτως ἔσται ἐν τῇ συντελείᾳ τοῦ αἰῶνος·
\vs{41}ἀποστελεῖ ὁ Υἱὸς τοῦ ἀνθρώπου τοὺς ἀγγέλους αὐτοῦ, καὶ συλλέξουσιν ἐκ τῆς βασιλείας αὐτοῦ πάντα τὰ σκάνδαλα καὶ τοὺς ποιοῦντας τὴν ἀνομίαν
\vs{42}καὶ βαλοῦσιν αὐτοὺς εἰς τὴν κάμινον τοῦ πυρός· ἐκεῖ ἔσται ὁ κλαυθμὸς καὶ ὁ βρυγμὸς τῶν ὀδόντων.
\vs{43}τότε οἱ δίκαιοι ἐκλάμψουσιν ὡς ὁ ἥλιος ἐν τῇ βασιλείᾳ τοῦ Πατρὸς αὐτῶν. Ὁ ἔχων ὦτα ἀκουέτω.

\vs{44}Ὁμοία ἐστὶν ἡ βασιλεία τῶν οὐρανῶν θησαυρῷ κεκρυμμένῳ ἐν τῷ ἀγρῷ, ὃν εὑρὼν ἄνθρωπος ἔκρυψεν, καὶ ἀπὸ τῆς χαρᾶς αὐτοῦ ὑπάγει καὶ πωλεῖ πάντα ὅσα ἔχει καὶ ἀγοράζει τὸν ἀγρὸν ἐκεῖνον.

\vs{45}Πάλιν ὁμοία ἐστὶν ἡ βασιλεία τῶν οὐρανῶν ἀνθρώπῳ ἐμπόρῳ ζητοῦντι καλοὺς μαργαρίτας·
\vs{46}εὑρὼν δὲ ἕνα πολύτιμον μαργαρίτην ἀπελθὼν πέπρακεν πάντα ὅσα εἶχεν καὶ ἠγόρασεν αὐτόν.

\vs{47}Πάλιν ὁμοία ἐστὶν ἡ βασιλεία τῶν οὐρανῶν σαγήνῃ βληθείσῃ εἰς τὴν θάλασσαν καὶ ἐκ παντὸς γένους συναγαγούσῃ·
\vs{48}ἣν ὅτε ἐπληρώθη ἀναβιβάσαντες ἐπὶ τὸν αἰγιαλὸν καὶ καθίσαντες συνέλεξαν τὰ καλὰ εἰς ἄγγη, τὰ δὲ σαπρὰ ἔξω ἔβαλον.
\vs{49}Οὕτως ἔσται ἐν τῇ συντελείᾳ τοῦ αἰῶνος· ἐξελεύσονται οἱ ἄγγελοι καὶ ἀφοριοῦσιν τοὺς πονηροὺς ἐκ μέσου τῶν δικαίων
\vs{50}καὶ βαλοῦσιν αὐτοὺς εἰς τὴν κάμινον τοῦ πυρός· ἐκεῖ ἔσται ὁ κλαυθμὸς καὶ ὁ βρυγμὸς τῶν ὀδόντων.

\vs{51}Συνήκατε ταῦτα πάντα; Λέγουσιν αὐτῷ· Ναί.
\vs{52}Ὁ δὲ εἶπεν αὐτοῖς· Διὰ τοῦτο πᾶς γραμματεὺς μαθητευθεὶς τῇ βασιλείᾳ τῶν οὐρανῶν ὅμοιός ἐστιν ἀνθρώπῳ οἰκοδεσπότῃ, ὅστις ἐκβάλλει ἐκ τοῦ θησαυροῦ αὐτοῦ καινὰ καὶ παλαιά.

\vs{53}Καὶ ἐγένετο ὅτε ἐτέλεσεν ὁ Ἰησοῦς τὰς παραβολὰς ταύτας, μετῆρεν ἐκεῖθεν.
\vs{54}καὶ ἐλθὼν εἰς τὴν πατρίδα αὐτοῦ ἐδίδασκεν αὐτοὺς ἐν τῇ συναγωγῇ αὐτῶν, ὥστε ἐκπλήσσεσθαι αὐτοὺς καὶ λέγειν· Πόθεν τούτῳ ἡ σοφία αὕτη καὶ αἱ δυνάμεις;
\vs{55}οὐχ οὗτός ἐστιν ὁ τοῦ τέκτονος υἱός; οὐχ ἡ μήτηρ αὐτοῦ λέγεται Μαριὰμ καὶ οἱ ἀδελφοὶ αὐτοῦ Ἰάκωβος καὶ Ἰωσὴφ καὶ Σίμων καὶ Ἰούδας;
\vs{56}καὶ αἱ ἀδελφαὶ αὐτοῦ οὐχὶ πᾶσαι πρὸς ἡμᾶς εἰσιν; πόθεν οὖν τούτῳ ταῦτα πάντα;
\vs{57}καὶ ἐσκανδαλίζοντο ἐν αὐτῷ. Ὁ δὲ Ἰησοῦς εἶπεν αὐτοῖς· Οὐκ ἔστιν προφήτης ἄτιμος εἰ μὴ ἐν τῇ πατρίδι καὶ ἐν τῇ οἰκίᾳ αὐτοῦ.
\vs{58}καὶ οὐκ ἐποίησεν ἐκεῖ δυνάμεις πολλὰς διὰ τὴν ἀπιστίαν αὐτῶν.

\ch{14}
Ἐν ἐκείνῳ τῷ καιρῷ ἤκουσεν Ἡρῴδης ὁ τετραάρχης τὴν ἀκοὴν Ἰησοῦ,
\vs{2}καὶ εἶπεν τοῖς παισὶν αὐτοῦ· Οὗτός ἐστιν Ἰωάννης ὁ Βαπτιστής· αὐτὸς ἠγέρθη ἀπὸ τῶν νεκρῶν καὶ διὰ τοῦτο αἱ δυνάμεις ἐνεργοῦσιν ἐν αὐτῷ.

\vs{3}Ὁ γὰρ Ἡρῴδης κρατήσας τὸν Ἰωάννην ἔδησεν αὐτὸν καὶ ἐν φυλακῇ ἀπέθετο διὰ Ἡρῳδιάδα τὴν γυναῖκα Φιλίππου τοῦ ἀδελφοῦ αὐτοῦ·
\vs{4}ἔλεγεν γὰρ ὁ Ἰωάννης αὐτῷ· Οὐκ ἔξεστίν σοι ἔχειν αὐτήν.
\vs{5}καὶ θέλων αὐτὸν ἀποκτεῖναι ἐφοβήθη τὸν ὄχλον, ὅτι ὡς προφήτην αὐτὸν εἶχον.

\vs{6}Γενεσίοις δὲ γενομένοις τοῦ Ἡρῴδου ὠρχήσατο ἡ θυγάτηρ τῆς Ἡρῳδιάδος ἐν τῷ μέσῳ καὶ ἤρεσεν τῷ Ἡρῴδῃ,
\vs{7}ὅθεν μεθ᾽ ὅρκου ὡμολόγησεν αὐτῇ δοῦναι ὃ ἐὰν αἰτήσηται.
\vs{8}Ἡ δὲ προβιβασθεῖσα ὑπὸ τῆς μητρὸς αὐτῆς· Δός μοι, φησίν, Ὧδε ἐπὶ πίνακι τὴν κεφαλὴν Ἰωάννου τοῦ Βαπτιστοῦ.
\vs{9}Καὶ λυπηθεὶς ὁ βασιλεὺς διὰ τοὺς ὅρκους καὶ τοὺς συνανακειμένους ἐκέλευσεν δοθῆναι,
\vs{10}καὶ πέμψας ἀπεκεφάλισεν τὸν Ἰωάννην ἐν τῇ φυλακῇ.
\vs{11}Καὶ ἠνέχθη ἡ κεφαλὴ αὐτοῦ ἐπὶ πίνακι καὶ ἐδόθη τῷ κορασίῳ, καὶ ἤνεγκεν τῇ μητρὶ αὐτῆς.
\vs{12}καὶ προσελθόντες οἱ μαθηταὶ αὐτοῦ ἦραν τὸ πτῶμα καὶ ἔθαψαν αὐτόν καὶ ἐλθόντες ἀπήγγειλαν τῷ Ἰησοῦ.

\vs{13}Ἀκούσας δὲ ὁ Ἰησοῦς ἀνεχώρησεν ἐκεῖθεν ἐν πλοίῳ εἰς ἔρημον τόπον κατ᾽ ἰδίαν· καὶ ἀκούσαντες οἱ ὄχλοι ἠκολούθησαν αὐτῷ πεζῇ ἀπὸ τῶν πόλεων.
\vs{14}Καὶ ἐξελθὼν εἶδεν πολὺν ὄχλον καὶ ἐσπλαγχνίσθη ἐπ᾽ αὐτοῖς καὶ ἐθεράπευσεν τοὺς ἀρρώστους αὐτῶν.

\vs{15}Ὀψίας δὲ γενομένης προσῆλθον αὐτῷ οἱ μαθηταὶ λέγοντες· Ἔρημός ἐστιν ὁ τόπος καὶ ἡ ὥρα ἤδη παρῆλθεν· ἀπόλυσον τοὺς ὄχλους, ἵνα ἀπελθόντες εἰς τὰς κώμας ἀγοράσωσιν ἑαυτοῖς βρώματα.
\vs{16}Ὁ δὲ Ἰησοῦς εἶπεν αὐτοῖς· Οὐ χρείαν ἔχουσιν ἀπελθεῖν, δότε αὐτοῖς ὑμεῖς φαγεῖν.
\vs{17}Οἱ δὲ λέγουσιν αὐτῷ· Οὐκ ἔχομεν ὧδε εἰ μὴ πέντε ἄρτους καὶ δύο ἰχθύας.
\vs{18}Ὁ δὲ εἶπεν· Φέρετέ μοι ὧδε αὐτούς.
\vs{19}καὶ κελεύσας τοὺς ὄχλους ἀνακλιθῆναι ἐπὶ τοῦ χόρτου, λαβὼν τοὺς πέντε ἄρτους καὶ τοὺς δύο ἰχθύας, ἀναβλέψας εἰς τὸν οὐρανὸν εὐλόγησεν καὶ κλάσας ἔδωκεν τοῖς μαθηταῖς τοὺς ἄρτους, οἱ δὲ μαθηταὶ τοῖς ὄχλοις.
\vs{20}Καὶ ἔφαγον πάντες καὶ ἐχορτάσθησαν, καὶ ἦραν τὸ περισσεῦον τῶν κλασμάτων δώδεκα κοφίνους πλήρεις.
\vs{21}οἱ δὲ ἐσθίοντες ἦσαν ἄνδρες ὡσεὶ πεντακισχίλιοι χωρὶς γυναικῶν καὶ παιδίων.

\vs{22}Καὶ εὐθέως ἠνάγκασεν τοὺς μαθητὰς ἐμβῆναι εἰς τὸ πλοῖον καὶ προάγειν αὐτὸν εἰς τὸ πέραν, ἕως οὗ ἀπολύσῃ τοὺς ὄχλους.
\vs{23}καὶ ἀπολύσας τοὺς ὄχλους ἀνέβη εἰς τὸ ὄρος κατ᾽ ἰδίαν προσεύξασθαι. ὀψίας δὲ γενομένης μόνος ἦν ἐκεῖ.
\vs{24}τὸ δὲ πλοῖον ἤδη σταδίους πολλοὺς ἀπὸ τῆς γῆς ἀπεῖχεν βασανιζόμενον ὑπὸ τῶν κυμάτων, ἦν γὰρ ἐναντίος ὁ ἄνεμος.
\vs{25}Τετάρτῃ δὲ φυλακῇ τῆς νυκτὸς ἦλθεν πρὸς αὐτοὺς περιπατῶν ἐπὶ τὴν θάλασσαν.
\vs{26}οἱ δὲ μαθηταὶ ἰδόντες αὐτὸν ἐπὶ τῆς θαλάσσης περιπατοῦντα ἐταράχθησαν λέγοντες ὅτι Φάντασμά ἐστιν, καὶ ἀπὸ τοῦ φόβου ἔκραξαν.
\vs{27}Εὐθὺς δὲ ἐλάλησεν ὁ Ἰησοῦς αὐτοῖς λέγων· Θαρσεῖτε, ἐγώ εἰμι· μὴ φοβεῖσθε.
\vs{28}Ἀποκριθεὶς δὲ αὐτῷ ὁ Πέτρος εἶπεν· Κύριε, εἰ σὺ εἶ, κέλευσόν με ἐλθεῖν πρὸς σὲ ἐπὶ τὰ ὕδατα.
\vs{29}Ὁ δὲ εἶπεν· Ἐλθέ. Καὶ καταβὰς ἀπὸ τοῦ πλοίου ὁ Πέτρος περιεπάτησεν ἐπὶ τὰ ὕδατα καὶ ἦλθεν πρὸς τὸν Ἰησοῦν.
\vs{30}βλέπων δὲ τὸν ἄνεμον ἰσχυρὸν ἐφοβήθη, καὶ ἀρξάμενος καταποντίζεσθαι ἔκραξεν λέγων· Κύριε, σῶσόν με.
\vs{31}Εὐθέως δὲ ὁ Ἰησοῦς ἐκτείνας τὴν χεῖρα ἐπελάβετο αὐτοῦ καὶ λέγει αὐτῷ· Ὀλιγόπιστε, εἰς τί ἐδίστασας;
\vs{32}Καὶ ἀναβάντων αὐτῶν εἰς τὸ πλοῖον ἐκόπασεν ὁ ἄνεμος.
\vs{33}οἱ δὲ ἐν τῷ πλοίῳ προσεκύνησαν αὐτῷ λέγοντες· Ἀληθῶς Θεοῦ Υἱὸς εἶ.

\vs{34}Καὶ διαπεράσαντες ἦλθον ἐπὶ τὴν γῆν εἰς Γεννησαρέτ.
\vs{35}καὶ ἐπιγνόντες αὐτὸν οἱ ἄνδρες τοῦ τόπου ἐκείνου ἀπέστειλαν εἰς ὅλην τὴν περίχωρον ἐκείνην καὶ προσήνεγκαν αὐτῷ πάντας τοὺς κακῶς ἔχοντας
\vs{36}καὶ παρεκάλουν αὐτὸν ἵνα μόνον ἅψωνται τοῦ κρασπέδου τοῦ ἱματίου αὐτοῦ· καὶ ὅσοι ἥψαντο διεσώθησαν.

\ch{15}
Τότε προσέρχονται τῷ Ἰησοῦ ἀπὸ Ἱεροσολύμων Φαρισαῖοι καὶ γραμματεῖς λέγοντες·
\vs{2}Διὰ τί οἱ μαθηταί σου παραβαίνουσιν τὴν παράδοσιν τῶν πρεσβυτέρων; οὐ γὰρ νίπτονται τὰς χεῖρας αὐτῶν ὅταν ἄρτον ἐσθίωσιν.
\vs{3}Ὁ δὲ ἀποκριθεὶς εἶπεν αὐτοῖς· Διὰ τί καὶ ὑμεῖς παραβαίνετε τὴν ἐντολὴν τοῦ Θεοῦ διὰ τὴν παράδοσιν ὑμῶν;
\vs{4}ὁ γὰρ Θεὸς εἶπεν· Τίμα τὸν πατέρα καὶ τὴν μητέρα, καί· Ὁ κακολογῶν πατέρα ἢ μητέρα θανάτῳ τελευτάτω.
\vs{5}ὑμεῖς δὲ λέγετε· Ὃς ἂν εἴπῃ τῷ πατρὶ ἢ τῇ μητρί· Δῶρον ὃ ἐὰν ἐξ ἐμοῦ ὠφεληθῇς,
\vs{6}οὐ μὴ τιμήσει τὸν πατέρα αὐτοῦ· καὶ ἠκυρώσατε τὸν λόγον τοῦ Θεοῦ διὰ τὴν παράδοσιν ὑμῶν.
\vs{7}ὑποκριταί, καλῶς ἐπροφήτευσεν περὶ ὑμῶν Ἠσαΐας λέγων·
\begin{poetryblock}

\begin{quote} \vs{8}Ὁ λαὸς οὗτος τοῖς χείλεσίν με τιμᾷ,\end{quote} 

\begin{quote}Ἡ δὲ καρδία αὐτῶν πόρρω ἀπέχει ἀπ᾽ ἐμοῦ·\end{quote}

\begin{quote} \vs{9}Μάτην δὲ σέβονταί με\end{quote} 

\begin{quote}Διδάσκοντες διδασκαλίας ἐντάλματα ἀνθρώπων.\end{quote}
\end{poetryblock}

\vs{10}Καὶ προσκαλεσάμενος τὸν ὄχλον εἶπεν αὐτοῖς· Ἀκούετε καὶ συνίετε·
\vs{11}οὐ τὸ εἰσερχόμενον εἰς τὸ στόμα κοινοῖ τὸν ἄνθρωπον, ἀλλὰ τὸ ἐκπορευόμενον ἐκ τοῦ στόματος τοῦτο κοινοῖ τὸν ἄνθρωπον.

\vs{12}Τότε προσελθόντες οἱ μαθηταὶ λέγουσιν αὐτῷ· Οἶδας ὅτι οἱ Φαρισαῖοι ἀκούσαντες τὸν λόγον ἐσκανδαλίσθησαν;
\vs{13}Ὁ δὲ ἀποκριθεὶς εἶπεν· Πᾶσα φυτεία ἣν οὐκ ἐφύτευσεν ὁ Πατήρ μου ὁ οὐράνιος ἐκριζωθήσεται.
\vs{14}ἄφετε αὐτούς· τυφλοί εἰσιν ὁδηγοί τυφλῶν· τυφλὸς δὲ τυφλὸν ἐὰν ὁδηγῇ, ἀμφότεροι εἰς βόθυνον πεσοῦνται.

\vs{15}Ἀποκριθεὶς δὲ ὁ Πέτρος εἶπεν αὐτῷ· Φράσον ἡμῖν τὴν παραβολήν ταύτην.
\vs{16}Ὁ δὲ εἶπεν· Ἀκμὴν καὶ ὑμεῖς ἀσύνετοί ἐστε;
\vs{17}οὐ νοεῖτε ὅτι πᾶν τὸ εἰσπορευόμενον εἰς τὸ στόμα εἰς τὴν κοιλίαν χωρεῖ καὶ εἰς ἀφεδρῶνα ἐκβάλλεται;
\vs{18}τὰ δὲ ἐκπορευόμενα ἐκ τοῦ στόματος ἐκ τῆς καρδίας ἐξέρχεται, κἀκεῖνα κοινοῖ τὸν ἄνθρωπον.
\vs{19}ἐκ γὰρ τῆς καρδίας ἐξέρχονται διαλογισμοὶ πονηροί, φόνοι, μοιχεῖαι, πορνεῖαι, κλοπαί, ψευδομαρτυρίαι, βλασφημίαι.
\vs{20}ταῦτά ἐστιν τὰ κοινοῦντα τὸν ἄνθρωπον, τὸ δὲ ἀνίπτοις χερσὶν φαγεῖν οὐ κοινοῖ τὸν ἄνθρωπον.

\vs{21}Καὶ ἐξελθὼν ἐκεῖθεν ὁ Ἰησοῦς ἀνεχώρησεν εἰς τὰ μέρη Τύρου καὶ Σιδῶνος.
\vs{22}καὶ ἰδοὺ γυνὴ Χαναναία ἀπὸ τῶν ὁρίων ἐκείνων ἐξελθοῦσα ἔκραζεν λέγουσα· Ἐλέησόν με, Κύριε υἱὸς Δαυίδ· ἡ θυγάτηρ μου κακῶς δαιμονίζεται.
\vs{23}Ὁ δὲ οὐκ ἀπεκρίθη αὐτῇ λόγον. καὶ προσελθόντες οἱ μαθηταὶ αὐτοῦ ἠρώτουν αὐτὸν λέγοντες· Ἀπόλυσον αὐτήν, ὅτι κράζει ὄπισθεν ἡμῶν.
\vs{24}Ὁ δὲ ἀποκριθεὶς εἶπεν· Οὐκ ἀπεστάλην εἰ μὴ εἰς τὰ πρόβατα τὰ ἀπολωλότα οἴκου Ἰσραήλ.
\vs{25}Ἡ δὲ ἐλθοῦσα προσεκύνει αὐτῷ λέγουσα· Κύριε, βοήθει μοι.
\vs{26}Ὁ δὲ ἀποκριθεὶς εἶπεν· Οὐκ ἔστιν καλὸν λαβεῖν τὸν ἄρτον τῶν τέκνων καὶ βαλεῖν τοῖς κυναρίοις.
\vs{27}Ἡ δὲ εἶπεν· Ναί κύριε, καὶ γὰρ τὰ κυνάρια ἐσθίει ἀπὸ τῶν ψιχίων τῶν πιπτόντων ἀπὸ τῆς τραπέζης τῶν κυρίων αὐτῶν.
\vs{28}Τότε ἀποκριθεὶς ὁ Ἰησοῦς εἶπεν αὐτῇ· Ὦ γύναι, μεγάλη σου ἡ πίστις· γενηθήτω σοι ὡς θέλεις. καὶ ἰάθη ἡ θυγάτηρ αὐτῆς ἀπὸ τῆς ὥρας ἐκείνης.

\vs{29}Καὶ μεταβὰς ἐκεῖθεν ὁ Ἰησοῦς ἦλθεν παρὰ τὴν θάλασσαν τῆς Γαλιλαίας, καὶ ἀναβὰς εἰς τὸ ὄρος ἐκάθητο ἐκεῖ.
\vs{30}καὶ προσῆλθον αὐτῷ ὄχλοι πολλοὶ ἔχοντες μεθ᾽ ἑαυτῶν χωλούς, τυφλούς, κυλλούς, κωφούς, καὶ ἑτέρους πολλούς καὶ ἔρριψαν αὐτοὺς παρὰ τοὺς πόδας αὐτοῦ, καὶ ἐθεράπευσεν αὐτούς·
\vs{31}ὥστε τὸν ὄχλον θαυμάσαι βλέποντας κωφοὺς λαλοῦντας, κυλλοὺς ὑγιεῖς καὶ χωλοὺς περιπατοῦντας καὶ τυφλοὺς βλέποντας· καὶ ἐδόξασαν τὸν Θεὸν Ἰσραήλ.

\vs{32}Ὁ δὲ Ἰησοῦς προσκαλεσάμενος τοὺς μαθητὰς αὐτοῦ εἶπεν· Σπλαγχνίζομαι ἐπὶ τὸν ὄχλον, ὅτι ἤδη ἡμέραι τρεῖς προσμένουσίν μοι καὶ οὐκ ἔχουσιν τί φάγωσιν· καὶ ἀπολῦσαι αὐτοὺς νήστεις οὐ θέλω, μήποτε ἐκλυθῶσιν ἐν τῇ ὁδῷ.
\vs{33}Καὶ λέγουσιν αὐτῷ οἱ μαθηταί· Πόθεν ἡμῖν ἐν ἐρημίᾳ ἄρτοι τοσοῦτοι ὥστε χορτάσαι ὄχλον τοσοῦτον;
\vs{34}Καὶ λέγει αὐτοῖς ὁ Ἰησοῦς· Πόσους ἄρτους ἔχετε; Οἱ δὲ εἶπαν· Ἑπτά καὶ ὀλίγα ἰχθύδια.
\vs{35}Καὶ παραγγείλας τῷ ὄχλῳ ἀναπεσεῖν ἐπὶ τὴν γῆν
\vs{36}ἔλαβεν τοὺς ἑπτὰ ἄρτους καὶ τοὺς ἰχθύας καὶ εὐχαριστήσας ἔκλασεν καὶ ἐδίδου τοῖς μαθηταῖς, οἱ δὲ μαθηταὶ τοῖς ὄχλοις.
\vs{37}Καὶ ἔφαγον πάντες καὶ ἐχορτάσθησαν. καὶ τὸ περισσεῦον τῶν κλασμάτων ἦραν ἑπτὰ σπυρίδας πλήρεις.
\vs{38}οἱ δὲ ἐσθίοντες ἦσαν τετρακισχίλιοι ἄνδρες χωρὶς γυναικῶν καὶ παιδίων.

\vs{39}Καὶ ἀπολύσας τοὺς ὄχλους ἐνέβη εἰς τὸ πλοῖον καὶ ἦλθεν εἰς τὰ ὅρια Μαγαδάν.

\ch{16}
Καὶ προσελθόντες οἱ Φαρισαῖοι καὶ Σαδδουκαῖοι πειράζοντες ἐπηρώτησαν αὐτὸν σημεῖον ἐκ τοῦ οὐρανοῦ ἐπιδεῖξαι αὐτοῖς.
\vs{2}Ὁ δὲ ἀποκριθεὶς εἶπεν αὐτοῖς· Ὀψίας γενομένης λέγετε· Εὐδία, πυρράζει γὰρ ὁ οὐρανός·
\vs{3}καὶ πρωΐ· Σήμερον χειμών, πυρράζει γὰρ στυγνάζων ὁ οὐρανός. τὸ μὲν πρόσωπον τοῦ οὐρανοῦ γινώσκετε διακρίνειν, τὰ δὲ σημεῖα τῶν καιρῶν οὐ δύνασθε;
\vs{4}γενεὰ πονηρὰ καὶ μοιχαλὶς σημεῖον ἐπιζητεῖ, καὶ σημεῖον οὐ δοθήσεται αὐτῇ εἰ μὴ τὸ σημεῖον Ἰωνᾶ. καὶ καταλιπὼν αὐτοὺς ἀπῆλθεν.

\vs{5}Καὶ ἐλθόντες οἱ μαθηταὶ εἰς τὸ πέραν ἐπελάθοντο ἄρτους λαβεῖν.
\vs{6}ὁ δὲ Ἰησοῦς εἶπεν αὐτοῖς· Ὁρᾶτε καὶ προσέχετε ἀπὸ τῆς ζύμης τῶν Φαρισαίων καὶ Σαδδουκαίων.
\vs{7}Οἱ δὲ διελογίζοντο ἐν ἑαυτοῖς λέγοντες Ὅτι Ἄρτους οὐκ ἐλάβομεν.
\vs{8}Γνοὺς δὲ ὁ Ἰησοῦς εἶπεν· Τί διαλογίζεσθε ἐν ἑαυτοῖς, ὀλιγόπιστοι, ὅτι ἄρτους οὐκ ἔχετε;
\vs{9}οὔπω νοεῖτε, οὐδὲ μνημονεύετε τοὺς πέντε ἄρτους τῶν πεντακισχιλίων καὶ πόσους κοφίνους ἐλάβετε;
\vs{10}οὐδὲ τοὺς ἑπτὰ ἄρτους τῶν τετρακισχιλίων καὶ πόσας σπυρίδας ἐλάβετε;
\vs{11}πῶς οὐ νοεῖτε ὅτι οὐ περὶ ἄρτων εἶπον ὑμῖν; προσέχετε δὲ ἀπὸ τῆς ζύμης τῶν Φαρισαίων καὶ Σαδδουκαίων.
\vs{12}Τότε συνῆκαν ὅτι οὐκ εἶπεν προσέχειν ἀπὸ τῆς ζύμης τῶν ἄρτων ἀλλὰ ἀπὸ τῆς διδαχῆς τῶν Φαρισαίων καὶ Σαδδουκαίων.

\vs{13}Ἐλθὼν δὲ ὁ Ἰησοῦς εἰς τὰ μέρη Καισαρείας τῆς Φιλίππου ἠρώτα τοὺς μαθητὰς αὐτοῦ λέγων· Τίνα λέγουσιν οἱ ἄνθρωποι εἶναι τὸν Υἱὸν τοῦ ἀνθρώπου;
\vs{14}Οἱ δὲ εἶπαν· Οἱ μὲν Ἰωάννην τὸν Βαπτιστήν, ἄλλοι δὲ Ἠλίαν, ἕτεροι δὲ Ἰερεμίαν ἢ ἕνα τῶν προφητῶν.
\vs{15}Λέγει αὐτοῖς· Ὑμεῖς δὲ τίνα με λέγετε εἶναι;
\vs{16}Ἀποκριθεὶς δὲ Σίμων Πέτρος εἶπεν· Σὺ εἶ ὁ Χριστὸς ὁ Υἱὸς τοῦ Θεοῦ τοῦ ζῶντος.
\vs{17}Ἀποκριθεὶς δὲ ὁ Ἰησοῦς εἶπεν αὐτῷ· Μακάριος εἶ, Σίμων Βαριωνᾶ, ὅτι σὰρξ καὶ αἷμα οὐκ ἀπεκάλυψέν σοι ἀλλ᾽ ὁ Πατήρ μου ὁ ἐν τοῖς οὐρανοῖς.
\vs{18}κἀγὼ δέ σοι λέγω ὅτι σὺ εἶ Πέτρος, καὶ ἐπὶ ταύτῃ τῇ πέτρᾳ οἰκοδομήσω μου τὴν ἐκκλησίαν καὶ πύλαι ᾅδου οὐ κατισχύσουσιν αὐτῆς.
\vs{19}δώσω σοι τὰς κλεῖδας τῆς βασιλείας τῶν οὐρανῶν, καὶ ὃ ἐὰν δήσῃς ἐπὶ τῆς γῆς ἔσται δεδεμένον ἐν τοῖς οὐρανοῖς, καὶ ὃ ἐὰν λύσῃς ἐπὶ τῆς γῆς ἔσται λελυμένον ἐν τοῖς οὐρανοῖς.
\vs{20}Τότε διεστείλατο τοῖς μαθηταῖς ἵνα μηδενὶ εἴπωσιν ὅτι αὐτός ἐστιν ὁ Χριστός.

\vs{21}Ἀπὸ τότε ἤρξατο ὁ Ἰησοῦς δεικνύειν τοῖς μαθηταῖς αὐτοῦ ὅτι δεῖ αὐτὸν εἰς Ἱεροσόλυμα ἀπελθεῖν καὶ πολλὰ παθεῖν ἀπὸ τῶν πρεσβυτέρων καὶ ἀρχιερέων καὶ γραμματέων καὶ ἀποκτανθῆναι καὶ τῇ τρίτῃ ἡμέρᾳ ἐγερθῆναι.
\vs{22}Καὶ προσλαβόμενος αὐτὸν ὁ Πέτρος ἤρξατο ἐπιτιμᾶν αὐτῷ λέγων· Ἵλεώς σοι, Κύριε· οὐ μὴ ἔσται σοι τοῦτο.
\vs{23}Ὁ δὲ στραφεὶς εἶπεν τῷ Πέτρῳ· Ὕπαγε ὀπίσω μου, Σατανᾶ· σκάνδαλον εἶ ἐμοῦ, ὅτι οὐ φρονεῖς τὰ τοῦ Θεοῦ ἀλλὰ τὰ τῶν ἀνθρώπων.

\vs{24}Τότε ὁ Ἰησοῦς εἶπεν τοῖς μαθηταῖς αὐτοῦ· Εἴ τις θέλει ὀπίσω μου ἐλθεῖν, ἀπαρνησάσθω ἑαυτὸν καὶ ἀράτω τὸν σταυρὸν αὐτοῦ καὶ ἀκολουθείτω μοι.
\vs{25}ὃς γὰρ ἐὰν θέλῃ τὴν ψυχὴν αὐτοῦ σῶσαι ἀπολέσει αὐτήν· ὃς δ᾽ ἂν ἀπολέσῃ τὴν ψυχὴν αὐτοῦ ἕνεκεν ἐμοῦ εὑρήσει αὐτήν.
\vs{26}τί γὰρ ὠφεληθήσεται ἄνθρωπος ἐὰν τὸν κόσμον ὅλον κερδήσῃ τὴν δὲ ψυχὴν αὐτοῦ ζημιωθῇ; ἢ τί δώσει ἄνθρωπος ἀντάλλαγμα τῆς ψυχῆς αὐτοῦ;
\vs{27}μέλλει γὰρ ὁ Υἱὸς τοῦ ἀνθρώπου ἔρχεσθαι ἐν τῇ δόξῃ τοῦ Πατρὸς αὐτοῦ μετὰ τῶν ἀγγέλων αὐτοῦ, καὶ τότε ἀποδώσει ἑκάστῳ κατὰ τὴν πρᾶξιν αὐτοῦ.
\vs{28}Ἀμὴν λέγω ὑμῖν ὅτι εἰσίν τινες τῶν ὧδε ἑστώτων οἵτινες οὐ μὴ γεύσωνται θανάτου ἕως ἂν ἴδωσιν τὸν Υἱὸν τοῦ ἀνθρώπου ἐρχόμενον ἐν τῇ βασιλείᾳ αὐτοῦ.

\ch{17}
Καὶ μεθ᾽ ἡμέρας ἓξ παραλαμβάνει ὁ Ἰησοῦς τὸν Πέτρον καὶ Ἰάκωβον καὶ Ἰωάννην τὸν ἀδελφὸν αὐτοῦ καὶ ἀναφέρει αὐτοὺς εἰς ὄρος ὑψηλὸν κατ᾽ ἰδίαν.
\vs{2}καὶ μετεμορφώθη ἔμπροσθεν αὐτῶν, καὶ ἔλαμψεν τὸ πρόσωπον αὐτοῦ ὡς ὁ ἥλιος, τὰ δὲ ἱμάτια αὐτοῦ ἐγένετο λευκὰ ὡς τὸ φῶς.
\vs{3}Καὶ ἰδοὺ ὤφθη αὐτοῖς Μωϋσῆς καὶ Ἠλίας συλλαλοῦντες μετ᾽ αὐτοῦ.
\vs{4}ἀποκριθεὶς δὲ ὁ Πέτρος εἶπεν τῷ Ἰησοῦ· Κύριε, καλόν ἐστιν ἡμᾶς ὧδε εἶναι· εἰ θέλεις, ποιήσω ὧδε τρεῖς σκηνάς, σοὶ μίαν καὶ Μωϋσεῖ μίαν καὶ Ἠλίᾳ μίαν.
\vs{5}Ἔτι αὐτοῦ λαλοῦντος ἰδοὺ νεφέλη φωτεινὴ ἐπεσκίασεν αὐτούς, καὶ ἰδοὺ φωνὴ ἐκ τῆς νεφέλης λέγουσα· Οὗτός ἐστιν ὁ Υἱός μου ὁ ἀγαπητός, ἐν ᾧ εὐδόκησα· ἀκούετε αὐτοῦ.
\vs{6}καὶ ἀκούσαντες οἱ μαθηταὶ ἔπεσαν ἐπὶ πρόσωπον αὐτῶν καὶ ἐφοβήθησαν σφόδρα.
\vs{7}Καὶ προσῆλθεν ὁ Ἰησοῦς καὶ ἁψάμενος αὐτῶν εἶπεν· Ἐγέρθητε καὶ μὴ φοβεῖσθε.
\vs{8}ἐπάραντες δὲ τοὺς ὀφθαλμοὺς αὐτῶν οὐδένα εἶδον εἰ μὴ αὐτὸν Ἰησοῦν μόνον.

\vs{9}Καὶ καταβαινόντων αὐτῶν ἐκ τοῦ ὄρους ἐνετείλατο αὐτοῖς ὁ Ἰησοῦς λέγων· Μηδενὶ εἴπητε τὸ ὅραμα ἕως οὗ ὁ Υἱὸς τοῦ ἀνθρώπου ἐκ νεκρῶν ἐγερθῇ.
\vs{10}Καὶ ἐπηρώτησαν αὐτὸν οἱ μαθηταὶ λέγοντες· Τί οὖν οἱ γραμματεῖς λέγουσιν ὅτι Ἠλίαν δεῖ ἐλθεῖν πρῶτον;
\vs{11}Ὁ δὲ ἀποκριθεὶς εἶπεν· Ἠλίας μὲν ἔρχεται καὶ ἀποκαταστήσει πάντα·
\vs{12}λέγω δὲ ὑμῖν ὅτι Ἠλίας ἤδη ἦλθεν, καὶ οὐκ ἐπέγνωσαν αὐτὸν ἀλλὰ ἐποίησαν ἐν αὐτῷ ὅσα ἠθέλησαν· οὕτως καὶ ὁ Υἱὸς τοῦ ἀνθρώπου μέλλει πάσχειν ὑπ᾽ αὐτῶν.
\vs{13}τότε συνῆκαν οἱ μαθηταὶ ὅτι περὶ Ἰωάννου τοῦ Βαπτιστοῦ εἶπεν αὐτοῖς.

\vs{14}Καὶ ἐλθόντων πρὸς τὸν ὄχλον προσῆλθεν αὐτῷ ἄνθρωπος γονυπετῶν αὐτὸν
\vs{15}καὶ λέγων· Κύριε, ἐλέησόν μου τὸν υἱόν, ὅτι σεληνιάζεται καὶ κακῶς πάσχει· πολλάκις γὰρ πίπτει εἰς τὸ πῦρ καὶ πολλάκις εἰς τὸ ὕδωρ.
\vs{16}καὶ προσήνεγκα αὐτὸν τοῖς μαθηταῖς σου, καὶ οὐκ ἠδυνήθησαν αὐτὸν θεραπεῦσαι.
\vs{17}Ἀποκριθεὶς δὲ ὁ Ἰησοῦς εἶπεν· Ὦ γενεὰ ἄπιστος καὶ διεστραμμένη, ἕως πότε μεθ᾽ ὑμῶν ἔσομαι; ἕως πότε ἀνέξομαι ὑμῶν; φέρετέ μοι αὐτὸν ὧδε.
\vs{18}καὶ ἐπετίμησεν αὐτῷ ὁ Ἰησοῦς καὶ ἐξῆλθεν ἀπ᾽ αὐτοῦ τὸ δαιμόνιον καὶ ἐθεραπεύθη ὁ παῖς ἀπὸ τῆς ὥρας ἐκείνης.

\vs{19}Τότε προσελθόντες οἱ μαθηταὶ τῷ Ἰησοῦ κατ᾽ ἰδίαν εἶπον· Διὰ τί ἡμεῖς οὐκ ἠδυνήθημεν ἐκβαλεῖν αὐτό;
\vs{20}Ὁ δὲ λέγει αὐτοῖς· Διὰ τὴν ὀλιγοπιστίαν ὑμῶν· ἀμὴν γὰρ λέγω ὑμῖν, ἐὰν ἔχητε πίστιν ὡς κόκκον σινάπεως, ἐρεῖτε τῷ ὄρει τούτῳ· Μετάβα ἔνθεν ἐκεῖ, καὶ μεταβήσεται· καὶ οὐδὲν ἀδυνατήσει ὑμῖν.

\vs{22}Συστρεφομένων δὲ αὐτῶν ἐν τῇ Γαλιλαίᾳ εἶπεν αὐτοῖς ὁ Ἰησοῦς· Μέλλει ὁ Υἱὸς τοῦ ἀνθρώπου παραδίδοσθαι εἰς χεῖρας ἀνθρώπων,
\vs{23}καὶ ἀποκτενοῦσιν αὐτόν, καὶ τῇ τρίτῃ ἡμέρᾳ ἐγερθήσεται. καὶ ἐλυπήθησαν σφόδρα.

\vs{24}Ἐλθόντων δὲ αὐτῶν εἰς Καφαρναοὺμ προσῆλθον οἱ τὰ δίδραχμα λαμβάνοντες τῷ Πέτρῳ καὶ εἶπαν· Ὁ διδάσκαλος ὑμῶν οὐ τελεῖ τὰ δίδραχμα;
\vs{25}Λέγει· Ναί. Καὶ ἐλθόντα εἰς τὴν οἰκίαν προέφθασεν αὐτὸν ὁ Ἰησοῦς λέγων· Τί σοι δοκεῖ, Σίμων; οἱ βασιλεῖς τῆς γῆς ἀπὸ τίνων λαμβάνουσιν τέλη ἢ κῆνσον; ἀπὸ τῶν υἱῶν αὐτῶν ἢ ἀπὸ τῶν ἀλλοτρίων;
\vs{26}Εἰπόντος δέ· Ἀπὸ τῶν ἀλλοτρίων, ἔφη αὐτῷ ὁ Ἰησοῦς· Ἄρα Γε ἐλεύθεροί εἰσιν οἱ υἱοί.
\vs{27}ἵνα δὲ μὴ σκανδαλίσωμεν αὐτούς, πορευθεὶς εἰς θάλασσαν βάλε ἄγκιστρον καὶ τὸν ἀναβάντα πρῶτον ἰχθὺν ἆρον, καὶ ἀνοίξας τὸ στόμα αὐτοῦ εὑρήσεις στατῆρα· ἐκεῖνον λαβὼν δὸς αὐτοῖς ἀντὶ ἐμοῦ καὶ σοῦ.

\ch{18}
Ἐν ἐκείνῃ τῇ ὥρᾳ προσῆλθον οἱ μαθηταὶ τῷ Ἰησοῦ λέγοντες· Τίς ἄρα μείζων ἐστὶν ἐν τῇ βασιλείᾳ τῶν οὐρανῶν;
\vs{2}Καὶ προσκαλεσάμενος παιδίον ἔστησεν αὐτὸ ἐν μέσῳ αὐτῶν
\vs{3}καὶ εἶπεν· Ἀμὴν λέγω ὑμῖν, ἐὰν μὴ στραφῆτε καὶ γένησθε ὡς τὰ παιδία, οὐ μὴ εἰσέλθητε εἰς τὴν βασιλείαν τῶν οὐρανῶν.
\vs{4}ὅστις οὖν ταπεινώσει ἑαυτὸν ὡς τὸ παιδίον τοῦτο, οὗτός ἐστιν ὁ μείζων ἐν τῇ βασιλείᾳ τῶν οὐρανῶν.
\vs{5}καὶ ὃς ἐὰν δέξηται ἓν παιδίον τοιοῦτο ἐπὶ τῷ ὀνόματί μου, ἐμὲ δέχεται.

\vs{6}Ὃς δ᾽ ἂν σκανδαλίσῃ ἕνα τῶν μικρῶν τούτων τῶν πιστευόντων εἰς ἐμέ, συμφέρει αὐτῷ ἵνα κρεμασθῇ μύλος ὀνικὸς περὶ τὸν τράχηλον αὐτοῦ καὶ καταποντισθῇ ἐν τῷ πελάγει τῆς θαλάσσης.
\vs{7}Οὐαὶ τῷ κόσμῳ ἀπὸ τῶν σκανδάλων· ἀνάγκη γὰρ ἐλθεῖν τὰ σκάνδαλα, πλὴν οὐαὶ τῷ ἀνθρώπῳ δι᾽ οὗ τὸ σκάνδαλον ἔρχεται.
\vs{8}Εἰ δὲ ἡ χείρ σου ἢ ὁ πούς σου σκανδαλίζει σε, ἔκκοψον αὐτὸν καὶ βάλε ἀπὸ σοῦ· καλόν σοί ἐστιν εἰσελθεῖν εἰς τὴν ζωὴν κυλλὸν ἢ χωλόν ἢ δύο χεῖρας ἢ δύο πόδας ἔχοντα βληθῆναι εἰς τὸ πῦρ τὸ αἰώνιον.
\vs{9}καὶ εἰ ὁ ὀφθαλμός σου σκανδαλίζει σε, ἔξελε αὐτὸν καὶ βάλε ἀπὸ σοῦ· καλόν σοί ἐστιν μονόφθαλμον εἰς τὴν ζωὴν εἰσελθεῖν ἢ δύο ὀφθαλμοὺς ἔχοντα βληθῆναι εἰς τὴν γέενναν τοῦ πυρός.

\vs{10}Ὁρᾶτε μὴ καταφρονήσητε ἑνὸς τῶν μικρῶν τούτων· λέγω γὰρ ὑμῖν ὅτι οἱ ἄγγελοι αὐτῶν ἐν οὐρανοῖς διὰ παντὸς βλέπουσι τὸ πρόσωπον τοῦ Πατρός μου τοῦ ἐν οὐρανοῖς.

\vs{12}Τί ὑμῖν δοκεῖ; ἐὰν γένηταί τινι ἀνθρώπῳ ἑκατὸν πρόβατα καὶ πλανηθῇ ἓν ἐξ αὐτῶν, οὐχὶ ἀφήσει τὰ ἐνενήκοντα ἐννέα ἐπὶ τὰ ὄρη καὶ πορευθεὶς ζητεῖ τὸ πλανώμενον;
\vs{13}καὶ ἐὰν γένηται εὑρεῖν αὐτό, ἀμὴν λέγω ὑμῖν ὅτι χαίρει ἐπ᾽ αὐτῷ μᾶλλον ἢ ἐπὶ τοῖς ἐνενήκοντα ἐννέα τοῖς μὴ πεπλανημένοις.
\vs{14}οὕτως οὐκ ἔστιν θέλημα ἔμπροσθεν τοῦ Πατρὸς ὑμῶν τοῦ ἐν οὐρανοῖς ἵνα ἀπόληται ἓν τῶν μικρῶν τούτων.

\vs{15}Ἐὰν δὲ ἁμαρτήσῃ εἰς σὲ ὁ ἀδελφός σου, ὕπαγε ἔλεγξον αὐτὸν μεταξὺ σοῦ καὶ αὐτοῦ μόνου. ἐάν σου ἀκούσῃ, ἐκέρδησας τὸν ἀδελφόν σου·
\vs{16}ἐὰν δὲ μὴ ἀκούσῃ, παράλαβε μετὰ σοῦ ἔτι ἕνα ἢ δύο, ἵνα Ἐπὶ στόματος δύο μαρτύρων ἢ τριῶν σταθῇ πᾶν ῥῆμα·
\vs{17}ἐὰν δὲ παρακούσῃ αὐτῶν, εἰπὲ τῇ ἐκκλησίᾳ· ἐὰν δὲ καὶ τῆς ἐκκλησίας παρακούσῃ, ἔστω σοι ὥσπερ ὁ ἐθνικὸς καὶ ὁ τελώνης.
\vs{18}Ἀμὴν λέγω ὑμῖν· ὅσα ἐὰν δήσητε ἐπὶ τῆς γῆς ἔσται δεδεμένα ἐν οὐρανῷ, καὶ ὅσα ἐὰν λύσητε ἐπὶ τῆς γῆς ἔσται λελυμένα ἐν οὐρανῷ.

\vs{19}Πάλιν ἀμὴν λέγω ὑμῖν ὅτι ἐὰν δύο συμφωνήσωσιν ἐξ ὑμῶν ἐπὶ τῆς γῆς περὶ παντὸς πράγματος οὗ ἐὰν αἰτήσωνται, γενήσεται αὐτοῖς παρὰ τοῦ Πατρός μου τοῦ ἐν οὐρανοῖς.
\vs{20}οὗ γάρ εἰσιν δύο ἢ τρεῖς συνηγμένοι εἰς τὸ ἐμὸν ὄνομα, ἐκεῖ εἰμι ἐν μέσῳ αὐτῶν.

\vs{21}Τότε προσελθὼν ὁ Πέτρος εἶπεν αὐτῷ· Κύριε, ποσάκις ἁμαρτήσει εἰς ἐμὲ ὁ ἀδελφός μου καὶ ἀφήσω αὐτῷ; ἕως ἑπτάκις;
\vs{22}Λέγει αὐτῷ ὁ Ἰησοῦς· Οὐ λέγω σοι ἕως ἑπτάκις ἀλλὰ ἕως ἑβδομηκοντάκις ἑπτά.

\vs{23}Διὰ τοῦτο ὡμοιώθη ἡ βασιλεία τῶν οὐρανῶν ἀνθρώπῳ βασιλεῖ, ὃς ἠθέλησεν συνᾶραι λόγον μετὰ τῶν δούλων αὐτοῦ.
\vs{24}ἀρξαμένου δὲ αὐτοῦ συναίρειν προσηνέχθη αὐτῷ εἷς ὀφειλέτης μυρίων ταλάντων.
\vs{25}μὴ ἔχοντος δὲ αὐτοῦ ἀποδοῦναι ἐκέλευσεν αὐτὸν ὁ κύριος πραθῆναι καὶ τὴν γυναῖκα καὶ τὰ τέκνα καὶ πάντα ὅσα ἔχει, καὶ ἀποδοθῆναι.
\vs{26}Πεσὼν οὖν ὁ δοῦλος προσεκύνει αὐτῷ λέγων· Μακροθύμησον ἐπ᾽ ἐμοί, καὶ πάντα ἀποδώσω σοι.
\vs{27}Σπλαγχνισθεὶς δὲ ὁ κύριος τοῦ δούλου ἐκείνου ἀπέλυσεν αὐτόν καὶ τὸ δάνειον ἀφῆκεν αὐτῷ.
\vs{28}Ἐξελθὼν δὲ ὁ δοῦλος ἐκεῖνος εὗρεν ἕνα τῶν συνδούλων αὐτοῦ, ὃς ὤφειλεν αὐτῷ ἑκατὸν δηνάρια, καὶ κρατήσας αὐτὸν ἔπνιγεν λέγων· Ἀπόδος εἴ τι ὀφείλεις.
\vs{29}Πεσὼν οὖν ὁ σύνδουλος αὐτοῦ παρεκάλει αὐτὸν λέγων· Μακροθύμησον ἐπ᾽ ἐμοί, καὶ ἀποδώσω σοι.
\vs{30}Ὁ δὲ οὐκ ἤθελεν ἀλλὰ ἀπελθὼν ἔβαλεν αὐτὸν εἰς φυλακὴν ἕως ἀποδῷ τὸ ὀφειλόμενον.
\vs{31}Ἰδόντες οὖν οἱ σύνδουλοι αὐτοῦ τὰ γενόμενα ἐλυπήθησαν σφόδρα καὶ ἐλθόντες διεσάφησαν τῷ κυρίῳ ἑαυτῶν πάντα τὰ γενόμενα.
\vs{32}Τότε προσκαλεσάμενος αὐτὸν ὁ κύριος αὐτοῦ λέγει αὐτῷ· Δοῦλε πονηρέ, πᾶσαν τὴν ὀφειλὴν ἐκείνην ἀφῆκά σοι, ἐπεὶ παρεκάλεσάς με·
\vs{33}οὐκ ἔδει καὶ σὲ ἐλεῆσαι τὸν σύνδουλόν σου, ὡς κἀγὼ σὲ ἠλέησα;
\vs{34}καὶ ὀργισθεὶς ὁ κύριος αὐτοῦ παρέδωκεν αὐτὸν τοῖς βασανισταῖς ἕως οὗ ἀποδῷ πᾶν τὸ ὀφειλόμενον.
\vs{35}Οὕτως καὶ ὁ Πατήρ μου ὁ οὐράνιος ποιήσει ὑμῖν, ἐὰν μὴ ἀφῆτε ἕκαστος τῷ ἀδελφῷ αὐτοῦ ἀπὸ τῶν καρδιῶν ὑμῶν.

\ch{19}
Καὶ ἐγένετο ὅτε ἐτέλεσεν ὁ Ἰησοῦς τοὺς λόγους τούτους, μετῆρεν ἀπὸ τῆς Γαλιλαίας καὶ ἦλθεν εἰς τὰ ὅρια τῆς Ἰουδαίας πέραν τοῦ Ἰορδάνου.
\vs{2}καὶ ἠκολούθησαν αὐτῷ ὄχλοι πολλοί, καὶ ἐθεράπευσεν αὐτοὺς ἐκεῖ.

\vs{3}Καὶ προσῆλθον αὐτῷ Φαρισαῖοι πειράζοντες αὐτὸν καὶ λέγοντες· Εἰ ἔξεστιν ἀνθρώπῳ ἀπολῦσαι τὴν γυναῖκα αὐτοῦ κατὰ πᾶσαν αἰτίαν;
\vs{4}Ὁ δὲ ἀποκριθεὶς εἶπεν· Οὐκ ἀνέγνωτε ὅτι ὁ κτίσας ἀπ᾽ ἀρχῆς Ἄρσεν καὶ θῆλυ ἐποίησεν αὐτοὺς;
\vs{5}καὶ εἶπεν· Ἕνεκα τούτου καταλείψει ἄνθρωπος τὸν πατέρα καὶ τὴν μητέρα καὶ κολληθήσεται τῇ γυναικὶ αὐτοῦ, καὶ ἔσονται οἱ δύο εἰς σάρκα μίαν.
\vs{6}ὥστε οὐκέτι εἰσὶν δύο ἀλλὰ σὰρξ μία. ὃ οὖν ὁ Θεὸς συνέζευξεν ἄνθρωπος μὴ χωριζέτω.
\vs{7}Λέγουσιν αὐτῷ· Τί οὖν Μωϋσῆς ἐνετείλατο δοῦναι βιβλίον ἀποστασίου καὶ ἀπολῦσαι αὐτήν;
\vs{8}Λέγει αὐτοῖς Ὅτι Μωϋσῆς πρὸς τὴν σκληροκαρδίαν ὑμῶν ἐπέτρεψεν ὑμῖν ἀπολῦσαι τὰς γυναῖκας ὑμῶν, ἀπ᾽ ἀρχῆς δὲ οὐ γέγονεν οὕτως.
\vs{9}λέγω δὲ ὑμῖν ὅτι ὃς ἂν ἀπολύσῃ τὴν γυναῖκα αὐτοῦ μὴ ἐπὶ πορνείᾳ καὶ γαμήσῃ ἄλλην μοιχᾶται.
\vs{10}Λέγουσιν αὐτῷ οἱ μαθηταί αὐτοῦ· Εἰ οὕτως ἐστὶν ἡ αἰτία τοῦ ἀνθρώπου μετὰ τῆς γυναικός, οὐ συμφέρει γαμῆσαι.
\vs{11}Ὁ δὲ εἶπεν αὐτοῖς· Οὐ πάντες χωροῦσιν τὸν λόγον τοῦτον ἀλλ᾽ οἷς δέδοται.
\vs{12}εἰσὶν γὰρ εὐνοῦχοι οἵτινες ἐκ κοιλίας μητρὸς ἐγεννήθησαν οὕτως, καὶ εἰσὶν εὐνοῦχοι οἵτινες εὐνουχίσθησαν ὑπὸ τῶν ἀνθρώπων, καὶ εἰσὶν εὐνοῦχοι οἵτινες εὐνούχισαν ἑαυτοὺς διὰ τὴν βασιλείαν τῶν οὐρανῶν. ὁ δυνάμενος χωρεῖν χωρείτω.

\vs{13}Τότε προσηνέχθησαν αὐτῷ παιδία ἵνα τὰς χεῖρας ἐπιθῇ αὐτοῖς καὶ προσεύξηται· οἱ δὲ μαθηταὶ ἐπετίμησαν αὐτοῖς.
\vs{14}ὁ δὲ Ἰησοῦς εἶπεν· Ἄφετε τὰ παιδία καὶ μὴ κωλύετε αὐτὰ ἐλθεῖν πρός με, τῶν γὰρ τοιούτων ἐστὶν ἡ βασιλεία τῶν οὐρανῶν.
\vs{15}καὶ ἐπιθεὶς τὰς χεῖρας αὐτοῖς ἐπορεύθη ἐκεῖθεν.

\vs{16}Καὶ ἰδοὺ εἷς προσελθὼν αὐτῷ εἶπεν· Διδάσκαλε, τί ἀγαθὸν ποιήσω ἵνα σχῶ ζωὴν αἰώνιον;
\vs{17}Ὁ δὲ εἶπεν αὐτῷ· Τί με ἐρωτᾷς περὶ τοῦ ἀγαθοῦ; εἷς ἐστιν ὁ ἀγαθός· εἰ δὲ θέλεις εἰς τὴν ζωὴν εἰσελθεῖν, τήρησον τὰς ἐντολάς.
\vs{18}Λέγει αὐτῷ· Ποίας; Ὁ δὲ Ἰησοῦς εἶπεν· Τὸ Οὐ φονεύσεις, Οὐ μοιχεύσεις, Οὐ κλέψεις, Οὐ ψευδομαρτυρήσεις,
\vs{19}Τίμα τὸν πατέρα καὶ τὴν μητέρα, καὶ Ἀγαπήσεις τὸν πλησίον σου ὡς σεαυτόν.
\vs{20}Λέγει αὐτῷ ὁ νεανίσκος· Πάντα ταῦτα ἐφύλαξα· τί ἔτι ὑστερῶ;
\vs{21}Ἔφη αὐτῷ ὁ Ἰησοῦς· Εἰ θέλεις τέλειος εἶναι, ὕπαγε πώλησόν σου τὰ ὑπάρχοντα καὶ δὸς τοῖς πτωχοῖς, καὶ ἕξεις θησαυρὸν ἐν οὐρανοῖς, καὶ δεῦρο ἀκολούθει μοι.
\vs{22}Ἀκούσας δὲ ὁ νεανίσκος τὸν λόγον ἀπῆλθεν λυπούμενος· ἦν γὰρ ἔχων κτήματα πολλά.

\vs{23}Ὁ δὲ Ἰησοῦς εἶπεν τοῖς μαθηταῖς αὐτοῦ· Ἀμὴν λέγω ὑμῖν ὅτι πλούσιος δυσκόλως εἰσελεύσεται εἰς τὴν βασιλείαν τῶν οὐρανῶν.
\vs{24}πάλιν δὲ λέγω ὑμῖν, εὐκοπώτερόν ἐστιν κάμηλον διὰ τρυπήματος ῥαφίδος διελθεῖν ἢ πλούσιον εἰσελθεῖν εἰς τὴν βασιλείαν τοῦ Θεοῦ.
\vs{25}Ἀκούσαντες δὲ οἱ μαθηταὶ ἐξεπλήσσοντο σφόδρα λέγοντες· Τίς ἄρα δύναται σωθῆναι;
\vs{26}Ἐμβλέψας δὲ ὁ Ἰησοῦς εἶπεν αὐτοῖς· Παρὰ ἀνθρώποις τοῦτο ἀδύνατόν ἐστιν, παρὰ δὲ Θεῷ πάντα δυνατά.

\vs{27}Τότε ἀποκριθεὶς ὁ Πέτρος εἶπεν αὐτῷ· Ἰδοὺ ἡμεῖς ἀφήκαμεν πάντα καὶ ἠκολουθήσαμέν σοι· τί ἄρα ἔσται ἡμῖν;
\vs{28}Ὁ δὲ Ἰησοῦς εἶπεν αὐτοῖς· Ἀμὴν λέγω ὑμῖν ὅτι ὑμεῖς οἱ ἀκολουθήσαντές μοι ἐν τῇ παλινγενεσίᾳ, ὅταν καθίσῃ ὁ Υἱὸς τοῦ ἀνθρώπου ἐπὶ θρόνου δόξης αὐτοῦ, καθήσεσθε καὶ ὑμεῖς ἐπὶ δώδεκα θρόνους κρίνοντες τὰς δώδεκα φυλὰς τοῦ Ἰσραήλ.
\vs{29}καὶ πᾶς ὅστις ἀφῆκεν οἰκίας ἢ ἀδελφοὺς ἢ ἀδελφὰς ἢ πατέρα ἢ μητέρα ἢ τέκνα ἢ ἀγροὺς ἕνεκεν τοῦ ὀνόματός μου, ἑκατονταπλασίονα λήμψεται καὶ ζωὴν αἰώνιον κληρονομήσει.
\vs{30}Πολλοὶ δὲ ἔσονται πρῶτοι ἔσχατοι καὶ ἔσχατοι πρῶτοι.

\ch{20}
Ὁμοία γάρ ἐστιν ἡ βασιλεία τῶν οὐρανῶν ἀνθρώπῳ οἰκοδεσπότῃ, ὅστις ἐξῆλθεν ἅμα πρωῒ μισθώσασθαι ἐργάτας εἰς τὸν ἀμπελῶνα αὐτοῦ.
\vs{2}συμφωνήσας δὲ μετὰ τῶν ἐργατῶν ἐκ δηναρίου τὴν ἡμέραν ἀπέστειλεν αὐτοὺς εἰς τὸν ἀμπελῶνα αὐτοῦ.
\vs{3}Καὶ ἐξελθὼν περὶ τρίτην ὥραν εἶδεν ἄλλους ἑστῶτας ἐν τῇ ἀγορᾷ ἀργούς
\vs{4}καὶ ἐκείνοις εἶπεν· Ὑπάγετε καὶ ὑμεῖς εἰς τὸν ἀμπελῶνα, καὶ ὃ ἐὰν ᾖ δίκαιον δώσω ὑμῖν.
\vs{5}οἱ δὲ ἀπῆλθον. Πάλιν δὲ ἐξελθὼν περὶ ἕκτην καὶ ἐνάτην ὥραν ἐποίησεν ὡσαύτως.
\vs{6}Περὶ δὲ τὴν ἑνδεκάτην ἐξελθὼν εὗρεν ἄλλους ἑστῶτας καὶ λέγει αὐτοῖς· Τί ὧδε ἑστήκατε ὅλην τὴν ἡμέραν ἀργοί;
\vs{7}Λέγουσιν αὐτῷ· Ὅτι οὐδεὶς ἡμᾶς ἐμισθώσατο. Λέγει αὐτοῖς· Ὑπάγετε καὶ ὑμεῖς εἰς τὸν ἀμπελῶνα.
\vs{8}Ὀψίας δὲ γενομένης λέγει ὁ κύριος τοῦ ἀμπελῶνος τῷ ἐπιτρόπῳ αὐτοῦ· Κάλεσον τοὺς ἐργάτας καὶ ἀπόδος αὐτοῖς τὸν μισθόν ἀρξάμενος ἀπὸ τῶν ἐσχάτων ἕως τῶν πρώτων.
\vs{9}Καὶ ἐλθόντες οἱ περὶ τὴν ἑνδεκάτην ὥραν ἔλαβον ἀνὰ δηνάριον.
\vs{10}καὶ ἐλθόντες οἱ πρῶτοι ἐνόμισαν ὅτι πλεῖον λήμψονται· καὶ ἔλαβον τὸ ἀνὰ δηνάριον καὶ αὐτοί.
\vs{11}Λαβόντες δὲ ἐγόγγυζον κατὰ τοῦ οἰκοδεσπότου
\vs{12}λέγοντες· Οὗτοι οἱ ἔσχατοι μίαν ὥραν ἐποίησαν, καὶ ἴσους ἡμῖν αὐτοὺς ἐποίησας τοῖς βαστάσασι τὸ βάρος τῆς ἡμέρας καὶ τὸν καύσωνα.
\vs{13}Ὁ δὲ ἀποκριθεὶς ἑνὶ αὐτῶν εἶπεν· Ἑταῖρε, οὐκ ἀδικῶ σε· οὐχὶ δηναρίου συνεφώνησάς μοι;
\vs{14}ἆρον τὸ σὸν καὶ ὕπαγε. θέλω δὲ τούτῳ τῷ ἐσχάτῳ δοῦναι ὡς καὶ σοί·
\vs{15}ἢ οὐκ ἔξεστίν μοι ὃ θέλω ποιῆσαι ἐν τοῖς ἐμοῖς; ἢ ὁ ὀφθαλμός σου πονηρός ἐστιν ὅτι ἐγὼ ἀγαθός εἰμι;
\vs{16}Οὕτως ἔσονται οἱ ἔσχατοι πρῶτοι καὶ οἱ πρῶτοι ἔσχατοι.

\vs{17}Καὶ ἀναβαίνων ὁ Ἰησοῦς εἰς Ἱεροσόλυμα παρέλαβεν τοὺς δώδεκα μαθητὰς κατ᾽ ἰδίαν καὶ ἐν τῇ ὁδῷ εἶπεν αὐτοῖς·
\vs{18}Ἰδοὺ ἀναβαίνομεν εἰς Ἱεροσόλυμα, καὶ ὁ Υἱὸς τοῦ ἀνθρώπου παραδοθήσεται τοῖς ἀρχιερεῦσιν καὶ γραμματεῦσιν, καὶ κατακρινοῦσιν αὐτὸν θανάτῳ
\vs{19}καὶ παραδώσουσιν αὐτὸν τοῖς ἔθνεσιν εἰς τὸ ἐμπαῖξαι καὶ μαστιγῶσαι καὶ σταυρῶσαι, καὶ τῇ τρίτῃ ἡμέρᾳ ἐγερθήσεται.

\vs{20}Τότε προσῆλθεν αὐτῷ ἡ μήτηρ τῶν υἱῶν Ζεβεδαίου μετὰ τῶν υἱῶν αὐτῆς προσκυνοῦσα καὶ αἰτοῦσά τι ἀπ᾽ αὐτοῦ.
\vs{21}Ὁ δὲ εἶπεν αὐτῇ· Τί θέλεις; Λέγει αὐτῷ· Εἰπὲ ἵνα καθίσωσιν οὗτοι οἱ δύο υἱοί μου εἷς ἐκ δεξιῶν σου καὶ εἷς ἐξ εὐωνύμων σου ἐν τῇ βασιλείᾳ σου.
\vs{22}Ἀποκριθεὶς δὲ ὁ Ἰησοῦς εἶπεν· Οὐκ οἴδατε τί αἰτεῖσθε. δύνασθε πιεῖν τὸ ποτήριον ὃ ἐγὼ μέλλω πίνειν; Λέγουσιν αὐτῷ· Δυνάμεθα.
\vs{23}Λέγει αὐτοῖς· Τὸ μὲν ποτήριόν μου πίεσθε, τὸ δὲ καθίσαι ἐκ δεξιῶν μου καὶ ἐξ εὐωνύμων οὐκ ἔστιν ἐμὸν τοῦτο δοῦναι, ἀλλ᾽ οἷς ἡτοίμασται ὑπὸ τοῦ Πατρός μου.

\vs{24}Καὶ ἀκούσαντες οἱ δέκα ἠγανάκτησαν περὶ τῶν δύο ἀδελφῶν.
\vs{25}ὁ δὲ Ἰησοῦς προσκαλεσάμενος αὐτοὺς εἶπεν· Οἴδατε ὅτι οἱ ἄρχοντες τῶν ἐθνῶν κατακυριεύουσιν αὐτῶν καὶ οἱ μεγάλοι κατεξουσιάζουσιν αὐτῶν.
\vs{26}οὐχ οὕτως ἔσται ἐν ὑμῖν, ἀλλ᾽ ὃς ἐὰν θέλῃ ἐν ὑμῖν μέγας γενέσθαι ἔσται ὑμῶν διάκονος,
\vs{27}καὶ ὃς ἂν θέλῃ ἐν ὑμῖν εἶναι πρῶτος ἔσται ὑμῶν δοῦλος·
\vs{28}ὥσπερ ὁ Υἱὸς τοῦ ἀνθρώπου οὐκ ἦλθεν διακονηθῆναι ἀλλὰ διακονῆσαι καὶ δοῦναι τὴν ψυχὴν αὐτοῦ λύτρον ἀντὶ πολλῶν.

\vs{29}Καὶ ἐκπορευομένων αὐτῶν ἀπὸ Ἰεριχὼ ἠκολούθησεν αὐτῷ ὄχλος πολύς.
\vs{30}καὶ ἰδοὺ δύο τυφλοὶ καθήμενοι παρὰ τὴν ὁδόν ἀκούσαντες ὅτι Ἰησοῦς παράγει, ἔκραξαν λέγοντες· Ἐλέησον ἡμᾶς, Κύριε, υἱὸς Δαυίδ.
\vs{31}Ὁ δὲ ὄχλος ἐπετίμησεν αὐτοῖς ἵνα σιωπήσωσιν· οἱ δὲ μεῖζον ἔκραξαν λέγοντες· Ἐλέησον ἡμᾶς, Κύριε, υἱὸς Δαυίδ.
\vs{32}Καὶ στὰς ὁ Ἰησοῦς ἐφώνησεν αὐτοὺς καὶ εἶπεν· Τί θέλετε ποιήσω ὑμῖν;
\vs{33}Λέγουσιν αὐτῷ· Κύριε, ἵνα ἀνοιγῶσιν οἱ ὀφθαλμοὶ ἡμῶν.
\vs{34}Σπλαγχνισθεὶς δὲ ὁ Ἰησοῦς ἥψατο τῶν ὀμμάτων αὐτῶν, καὶ εὐθέως ἀνέβλεψαν καὶ ἠκολούθησαν αὐτῷ.

\ch{21}
Καὶ ὅτε ἤγγισαν εἰς Ἱεροσόλυμα καὶ ἦλθον εἰς Βηθφαγὴ εἰς τὸ ὄρος τῶν Ἐλαιῶν, τότε Ἰησοῦς ἀπέστειλεν δύο μαθητὰς
\vs{2}λέγων αὐτοῖς· Πορεύεσθε εἰς τὴν κώμην τὴν κατέναντι ὑμῶν, καὶ εὐθέως εὑρήσετε ὄνον δεδεμένην καὶ πῶλον μετ᾽ αὐτῆς· λύσαντες ἀγάγετέ μοι.
\vs{3}καὶ ἐάν τις ὑμῖν εἴπῃ τι, ἐρεῖτε ὅτι Ὁ Κύριος αὐτῶν χρείαν ἔχει· εὐθὺς δὲ ἀποστελεῖ αὐτούς.
\vs{4}Τοῦτο δὲ γέγονεν ἵνα πληρωθῇ τὸ ῥηθὲν διὰ τοῦ προφήτου λέγοντος·
\begin{poetryblock}

\begin{quote} \vs{5}Εἴπατε τῇ θυγατρὶ Σιών·\end{quote} 

\begin{quote}Ἰδοὺ ὁ Βασιλεύς σου ἔρχεταί σοι\end{quote} 

\begin{quote}πραῢς καὶ ἐπιβεβηκὼς ἐπὶ ὄνον\end{quote} 

\begin{quote}καὶ ἐπὶ πῶλον υἱὸν ὑποζυγίου.\end{quote}
\end{poetryblock}

\vs{6}Πορευθέντες δὲ οἱ μαθηταὶ καὶ ποιήσαντες καθὼς συνέταξεν αὐτοῖς ὁ Ἰησοῦς
\vs{7}ἤγαγον τὴν ὄνον καὶ τὸν πῶλον καὶ ἐπέθηκαν ἐπ᾽ αὐτῶν τὰ ἱμάτια, καὶ ἐπεκάθισεν ἐπάνω αὐτῶν.
\vs{8}Ὁ δὲ πλεῖστος ὄχλος ἔστρωσαν ἑαυτῶν τὰ ἱμάτια ἐν τῇ ὁδῷ, ἄλλοι δὲ ἔκοπτον κλάδους ἀπὸ τῶν δένδρων καὶ ἐστρώννυον ἐν τῇ ὁδῷ.
\vs{9}Οἱ δὲ ὄχλοι οἱ προάγοντες αὐτὸν καὶ οἱ ἀκολουθοῦντες ἔκραζον λέγοντες· 
\begin{poetryblock}

\begin{quote}Ὡσαννὰ τῷ υἱῷ Δαυίδ·\end{quote} 

\begin{quote}Εὐλογημένος ὁ ἐρχόμενος ἐν ὀνόματι Κυρίου·\end{quote} 

\begin{quote}Ὡσαννὰ ἐν τοῖς ὑψίστοις.\end{quote}
\end{poetryblock}

\vs{10}Καὶ εἰσελθόντος αὐτοῦ εἰς Ἱεροσόλυμα ἐσείσθη πᾶσα ἡ πόλις λέγουσα· Τίς ἐστιν οὗτος;
\vs{11}Οἱ δὲ ὄχλοι ἔλεγον· Οὗτός ἐστιν ὁ προφήτης Ἰησοῦς ὁ ἀπὸ Ναζαρὲθ τῆς Γαλιλαίας.

\vs{12}Καὶ εἰσῆλθεν Ἰησοῦς εἰς τὸ ἱερόν καὶ ἐξέβαλεν πάντας τοὺς πωλοῦντας καὶ ἀγοράζοντας ἐν τῷ ἱερῷ, καὶ τὰς τραπέζας τῶν κολλυβιστῶν κατέστρεψεν καὶ τὰς καθέδρας τῶν πωλούντων τὰς περιστεράς,
\vs{13}καὶ λέγει αὐτοῖς· Γέγραπται· 
\begin{poetryblock}

\begin{quote}Ὁ οἶκός μου οἶκος προσευχῆς κληθήσεται,\end{quote} 

\begin{quote}ὑμεῖς δὲ αὐτὸν ποιεῖτε Σπήλαιον λῃστῶν.\end{quote}
\end{poetryblock}

\vs{14}Καὶ προσῆλθον αὐτῷ τυφλοὶ καὶ χωλοὶ ἐν τῷ ἱερῷ, καὶ ἐθεράπευσεν αὐτούς.
\vs{15}ἰδόντες δὲ οἱ ἀρχιερεῖς καὶ οἱ γραμματεῖς τὰ θαυμάσια ἃ ἐποίησεν καὶ τοὺς παῖδας τοὺς κράζοντας ἐν τῷ ἱερῷ καὶ λέγοντας· Ὡσαννὰ τῷ υἱῷ Δαυίδ, ἠγανάκτησαν
\vs{16}καὶ εἶπαν αὐτῷ· Ἀκούεις τί οὗτοι λέγουσιν; Ὁ δὲ Ἰησοῦς λέγει αὐτοῖς· Ναί. οὐδέποτε ἀνέγνωτε ὅτι 
\begin{poetryblock}

\begin{quote}Ἐκ στόματος νηπίων καὶ θηλαζόντων Κατηρτίσω αἶνον;\end{quote}
\end{poetryblock}

\vs{17}Καὶ καταλιπὼν αὐτοὺς ἐξῆλθεν ἔξω τῆς πόλεως εἰς Βηθανίαν καὶ ηὐλίσθη ἐκεῖ.

\vs{18}Πρωῒ δὲ ἐπανάγων εἰς τὴν πόλιν ἐπείνασεν.
\vs{19}καὶ ἰδὼν συκῆν μίαν ἐπὶ τῆς ὁδοῦ ἦλθεν ἐπ᾽ αὐτήν καὶ οὐδὲν εὗρεν ἐν αὐτῇ εἰ μὴ φύλλα μόνον, καὶ λέγει αὐτῇ· μηκέτι ἐκ σοῦ καρπὸς γένηται εἰς τὸν αἰῶνα. καὶ ἐξηράνθη παραχρῆμα ἡ συκῆ.

\vs{20}Καὶ ἰδόντες οἱ μαθηταὶ ἐθαύμασαν λέγοντες· Πῶς παραχρῆμα ἐξηράνθη ἡ συκῆ;
\vs{21}Ἀποκριθεὶς δὲ ὁ Ἰησοῦς εἶπεν αὐτοῖς· Ἀμὴν λέγω ὑμῖν, ἐὰν ἔχητε πίστιν καὶ μὴ διακριθῆτε, οὐ μόνον τὸ τῆς συκῆς ποιήσετε, ἀλλὰ κἂν τῷ ὄρει τούτῳ εἴπητε· Ἄρθητι καὶ βλήθητι εἰς τὴν θάλασσαν, γενήσεται·
\vs{22}καὶ πάντα ὅσα ἂν αἰτήσητε ἐν τῇ προσευχῇ πιστεύοντες λήμψεσθε.

\vs{23}Καὶ ἐλθόντος αὐτοῦ εἰς τὸ ἱερὸν προσῆλθον αὐτῷ διδάσκοντι οἱ ἀρχιερεῖς καὶ οἱ πρεσβύτεροι τοῦ λαοῦ λέγοντες· Ἐν ποίᾳ ἐξουσίᾳ ταῦτα ποιεῖς; καὶ τίς σοι ἔδωκεν τὴν ἐξουσίαν ταύτην;
\vs{24}Ἀποκριθεὶς δὲ ὁ Ἰησοῦς εἶπεν αὐτοῖς· Ἐρωτήσω ὑμᾶς κἀγὼ λόγον ἕνα, ὃν ἐὰν εἴπητέ μοι κἀγὼ ὑμῖν ἐρῶ ἐν ποίᾳ ἐξουσίᾳ ταῦτα ποιῶ·
\vs{25}τὸ βάπτισμα τὸ Ἰωάννου πόθεν ἦν; ἐξ οὐρανοῦ ἢ ἐξ ἀνθρώπων; Οἱ δὲ διελογίζοντο ἐν ἑαυτοῖς λέγοντες· Ἐὰν εἴπωμεν· Ἐξ οὐρανοῦ, ἐρεῖ ἡμῖν· Διὰ τί οὖν οὐκ ἐπιστεύσατε αὐτῷ;
\vs{26}ἐὰν δὲ εἴπωμεν· Ἐξ ἀνθρώπων, φοβούμεθα τὸν ὄχλον, πάντες γὰρ ὡς προφήτην ἔχουσιν τὸν Ἰωάννην.
\vs{27}καὶ ἀποκριθέντες τῷ Ἰησοῦ εἶπαν· Οὐκ οἴδαμεν. Ἔφη αὐτοῖς καὶ αὐτός· Οὐδὲ ἐγὼ λέγω ὑμῖν ἐν ποίᾳ ἐξουσίᾳ ταῦτα ποιῶ.

\vs{28}Τί δὲ ὑμῖν δοκεῖ; ἄνθρωπος εἶχεν τέκνα δύο. καὶ προσελθὼν τῷ πρώτῳ εἶπεν· Τέκνον, ὕπαγε σήμερον ἐργάζου ἐν τῷ ἀμπελῶνι.
\vs{29}Ὁ δὲ ἀποκριθεὶς εἶπεν· Οὐ θέλω, ὕστερον δὲ μεταμεληθεὶς ἀπῆλθεν.
\vs{30}Προσελθὼν δὲ τῷ ἑτέρῳ εἶπεν ὡσαύτως. Ὁ δὲ ἀποκριθεὶς εἶπεν· Ἐγώ, κύριε, καὶ οὐκ ἀπῆλθεν.
\vs{31}Τίς ἐκ τῶν δύο ἐποίησεν τὸ θέλημα τοῦ πατρός; Λέγουσιν· Ὁ πρῶτος. Λέγει αὐτοῖς ὁ Ἰησοῦς· Ἀμὴν λέγω ὑμῖν ὅτι οἱ τελῶναι καὶ αἱ πόρναι προάγουσιν ὑμᾶς εἰς τὴν βασιλείαν τοῦ Θεοῦ.
\vs{32}ἦλθεν γὰρ Ἰωάννης πρὸς ὑμᾶς ἐν ὁδῷ δικαιοσύνης, καὶ οὐκ ἐπιστεύσατε αὐτῷ, οἱ δὲ τελῶναι καὶ αἱ πόρναι ἐπίστευσαν αὐτῷ· ὑμεῖς δὲ ἰδόντες οὐδὲ μετεμελήθητε ὕστερον τοῦ πιστεῦσαι αὐτῷ.

\vs{33}Ἄλλην παραβολὴν ἀκούσατε. Ἄνθρωπος ἦν οἰκοδεσπότης ὅστις ἐφύτευσεν ἀμπελῶνα καὶ φραγμὸν αὐτῷ περιέθηκεν καὶ ὤρυξεν ἐν αὐτῷ ληνὸν καὶ ᾠκοδόμησεν πύργον καὶ ἐξέδετο αὐτὸν γεωργοῖς καὶ ἀπεδήμησεν.
\vs{34}Ὅτε δὲ ἤγγισεν ὁ καιρὸς τῶν καρπῶν, ἀπέστειλεν τοὺς δούλους αὐτοῦ πρὸς τοὺς γεωργοὺς λαβεῖν τοὺς καρποὺς αὐτοῦ.
\vs{35}καὶ λαβόντες οἱ γεωργοὶ τοὺς δούλους αὐτοῦ ὃν μὲν ἔδειραν, ὃν δὲ ἀπέκτειναν, ὃν δὲ ἐλιθοβόλησαν.
\vs{36}Πάλιν ἀπέστειλεν ἄλλους δούλους πλείονας τῶν πρώτων, καὶ ἐποίησαν αὐτοῖς ὡσαύτως.
\vs{37}Ὕστερον δὲ ἀπέστειλεν πρὸς αὐτοὺς τὸν υἱὸν αὐτοῦ λέγων· Ἐντραπήσονται τὸν υἱόν μου.
\vs{38}Οἱ δὲ γεωργοὶ ἰδόντες τὸν υἱὸν εἶπον ἐν ἑαυτοῖς· Οὗτός ἐστιν ὁ κληρονόμος· δεῦτε ἀποκτείνωμεν αὐτὸν καὶ σχῶμεν τὴν κληρονομίαν αὐτοῦ,
\vs{39}καὶ λαβόντες αὐτὸν ἐξέβαλον ἔξω τοῦ ἀμπελῶνος καὶ ἀπέκτειναν.
\vs{40}Ὅταν οὖν ἔλθῃ ὁ κύριος τοῦ ἀμπελῶνος, τί ποιήσει τοῖς γεωργοῖς ἐκείνοις;
\vs{41}Λέγουσιν αὐτῷ· Κακοὺς κακῶς ἀπολέσει αὐτούς καὶ τὸν ἀμπελῶνα ἐκδώσεται ἄλλοις γεωργοῖς, οἵτινες ἀποδώσουσιν αὐτῷ τοὺς καρποὺς ἐν τοῖς καιροῖς αὐτῶν.

\vs{42}Λέγει αὐτοῖς ὁ Ἰησοῦς· Οὐδέποτε ἀνέγνωτε ἐν ταῖς γραφαῖς· 
\begin{poetryblock}

\begin{quote}Λίθον ὃν ἀπεδοκίμασαν οἱ οἰκοδομοῦντες,\end{quote} 

\begin{quote}Οὗτος ἐγενήθη εἰς κεφαλὴν γωνίας·\end{quote} 

\begin{quote}Παρὰ Κυρίου ἐγένετο αὕτη\end{quote} 

\begin{quote}Καὶ ἔστιν θαυμαστὴ ἐν ὀφθαλμοῖς ἡμῶν;\end{quote}
\end{poetryblock}

\vs{43}Διὰ τοῦτο λέγω ὑμῖν ὅτι ἀρθήσεται ἀφ᾽ ὑμῶν ἡ βασιλεία τοῦ Θεοῦ καὶ δοθήσεται ἔθνει ποιοῦντι τοὺς καρποὺς αὐτῆς.
\vs{44}καὶ ὁ πεσὼν ἐπὶ τὸν λίθον τοῦτον συνθλασθήσεται· ἐφ᾽ ὃν δ᾽ ἂν πέσῃ λικμήσει αὐτόν.

\vs{45}Καὶ ἀκούσαντες οἱ ἀρχιερεῖς καὶ οἱ Φαρισαῖοι τὰς παραβολὰς αὐτοῦ ἔγνωσαν ὅτι περὶ αὐτῶν λέγει·
\vs{46}καὶ ζητοῦντες αὐτὸν κρατῆσαι ἐφοβήθησαν τοὺς ὄχλους, ἐπεὶ εἰς προφήτην αὐτὸν εἶχον.

\ch{22}
Καὶ ἀποκριθεὶς ὁ Ἰησοῦς πάλιν εἶπεν ἐν παραβολαῖς αὐτοῖς λέγων·
\vs{2}Ὡμοιώθη ἡ βασιλεία τῶν οὐρανῶν ἀνθρώπῳ βασιλεῖ, ὅστις ἐποίησεν γάμους τῷ υἱῷ αὐτοῦ.
\vs{3}καὶ ἀπέστειλεν τοὺς δούλους αὐτοῦ καλέσαι τοὺς κεκλημένους εἰς τοὺς γάμους, καὶ οὐκ ἤθελον ἐλθεῖν.
\vs{4}Πάλιν ἀπέστειλεν ἄλλους δούλους λέγων· Εἴπατε τοῖς κεκλημένοις· Ἰδοὺ τὸ ἄριστόν μου ἡτοίμακα, οἱ ταῦροί μου καὶ τὰ σιτιστὰ τεθυμένα καὶ πάντα ἕτοιμα· δεῦτε εἰς τοὺς γάμους.
\vs{5}Οἱ δὲ ἀμελήσαντες ἀπῆλθον, ὃς μὲν εἰς τὸν ἴδιον ἀγρόν, ὃς δὲ ἐπὶ τὴν ἐμπορίαν αὐτοῦ·
\vs{6}οἱ δὲ λοιποὶ κρατήσαντες τοὺς δούλους αὐτοῦ ὕβρισαν καὶ ἀπέκτειναν.
\vs{7}Ὁ δὲ βασιλεὺς ὠργίσθη καὶ πέμψας τὰ στρατεύματα αὐτοῦ ἀπώλεσεν τοὺς φονεῖς ἐκείνους καὶ τὴν πόλιν αὐτῶν ἐνέπρησεν.
\vs{8}τότε λέγει τοῖς δούλοις αὐτοῦ· Ὁ μὲν γάμος ἕτοιμός ἐστιν, οἱ δὲ κεκλημένοι οὐκ ἦσαν ἄξιοι·
\vs{9}πορεύεσθε οὖν ἐπὶ τὰς διεξόδους τῶν ὁδῶν καὶ ὅσους ἐὰν εὕρητε καλέσατε εἰς τοὺς γάμους.
\vs{10}Καὶ ἐξελθόντες οἱ δοῦλοι ἐκεῖνοι εἰς τὰς ὁδοὺς συνήγαγον πάντας οὓς εὗρον, πονηρούς τε καὶ ἀγαθούς· καὶ ἐπλήσθη ὁ γάμος ἀνακειμένων.
\vs{11}Εἰσελθὼν δὲ ὁ βασιλεὺς θεάσασθαι τοὺς ἀνακειμένους εἶδεν ἐκεῖ ἄνθρωπον οὐκ ἐνδεδυμένον ἔνδυμα γάμου,
\vs{12}καὶ λέγει αὐτῷ· Ἑταῖρε, πῶς εἰσῆλθες ὧδε μὴ ἔχων ἔνδυμα γάμου; Ὁ δὲ ἐφιμώθη.
\vs{13}Τότε ὁ βασιλεὺς εἶπεν τοῖς διακόνοις· Δήσαντες αὐτοῦ πόδας καὶ χεῖρας ἐκβάλετε αὐτὸν εἰς τὸ σκότος τὸ ἐξώτερον· ἐκεῖ ἔσται ὁ κλαυθμὸς καὶ ὁ βρυγμὸς τῶν ὀδόντων.
\vs{14}Πολλοὶ γάρ εἰσιν κλητοὶ, ὀλίγοι δὲ ἐκλεκτοί.

\vs{15}Τότε πορευθέντες οἱ Φαρισαῖοι συμβούλιον ἔλαβον ὅπως αὐτὸν παγιδεύσωσιν ἐν λόγῳ.
\vs{16}καὶ ἀποστέλλουσιν αὐτῷ τοὺς μαθητὰς αὐτῶν μετὰ τῶν Ἡρῳδιανῶν λέγοντες· Διδάσκαλε, οἴδαμεν ὅτι ἀληθὴς εἶ καὶ τὴν ὁδὸν τοῦ Θεοῦ ἐν ἀληθείᾳ διδάσκεις καὶ οὐ μέλει σοι περὶ οὐδενός· οὐ γὰρ βλέπεις εἰς πρόσωπον ἀνθρώπων,
\vs{17}εἰπὲ οὖν ἡμῖν τί σοι δοκεῖ· ἔξεστιν δοῦναι κῆνσον Καίσαρι ἢ οὔ;
\vs{18}Γνοὺς δὲ ὁ Ἰησοῦς τὴν πονηρίαν αὐτῶν εἶπεν· Τί με πειράζετε, ὑποκριταί;
\vs{19}ἐπιδείξατέ μοι τὸ νόμισμα τοῦ κήνσου. Οἱ δὲ προσήνεγκαν αὐτῷ δηνάριον.
\vs{20}Καὶ λέγει αὐτοῖς· Τίνος ἡ εἰκὼν αὕτη καὶ ἡ ἐπιγραφή;
\vs{21}Λέγουσιν αὐτῷ· Καίσαρος. Τότε λέγει αὐτοῖς· Ἀπόδοτε οὖν τὰ Καίσαρος Καίσαρι καὶ τὰ τοῦ Θεοῦ τῷ Θεῷ.
\vs{22}Καὶ ἀκούσαντες ἐθαύμασαν, καὶ ἀφέντες αὐτὸν ἀπῆλθαν.

\vs{23}Ἐν ἐκείνῃ τῇ ἡμέρᾳ προσῆλθον αὐτῷ Σαδδουκαῖοι, λέγοντες μὴ εἶναι ἀνάστασιν, καὶ ἐπηρώτησαν αὐτὸν
\vs{24}λέγοντες· Διδάσκαλε, Μωϋσῆς εἶπεν· Ἐάν τις ἀποθάνῃ μὴ ἔχων τέκνα, ἐπιγαμβρεύσει ὁ ἀδελφὸς αὐτοῦ τὴν γυναῖκα αὐτοῦ καὶ ἀναστήσει σπέρμα τῷ ἀδελφῷ αὐτοῦ.
\vs{25}ἦσαν δὲ παρ᾽ ἡμῖν ἑπτὰ ἀδελφοί· καὶ ὁ πρῶτος γήμας ἐτελεύτησεν, καὶ μὴ ἔχων σπέρμα ἀφῆκεν τὴν γυναῖκα αὐτοῦ τῷ ἀδελφῷ αὐτοῦ·
\vs{26}ὁμοίως καὶ ὁ δεύτερος καὶ ὁ τρίτος ἕως τῶν ἑπτά.
\vs{27}ὕστερον δὲ πάντων ἀπέθανεν ἡ γυνή.
\vs{28}ἐν τῇ ἀναστάσει οὖν τίνος τῶν ἑπτὰ ἔσται γυνή; πάντες γὰρ ἔσχον αὐτήν·
\vs{29}Ἀποκριθεὶς δὲ ὁ Ἰησοῦς εἶπεν αὐτοῖς· Πλανᾶσθε μὴ εἰδότες τὰς γραφὰς μηδὲ τὴν δύναμιν τοῦ Θεοῦ·
\vs{30}ἐν γὰρ τῇ ἀναστάσει οὔτε γαμοῦσιν οὔτε γαμίζονται, ἀλλ᾽ ὡς ἄγγελοι ἐν τῷ οὐρανῷ εἰσιν.
\vs{31}περὶ δὲ τῆς ἀναστάσεως τῶν νεκρῶν οὐκ ἀνέγνωτε τὸ ῥηθὲν ὑμῖν ὑπὸ τοῦ Θεοῦ λέγοντος·
\vs{32}Ἐγώ εἰμι ὁ Θεὸς Ἀβραὰμ καὶ ὁ Θεὸς Ἰσαὰκ καὶ ὁ Θεὸς Ἰακώβ; οὐκ ἔστιν ὁ Θεὸς νεκρῶν ἀλλὰ ζώντων.
\vs{33}Καὶ ἀκούσαντες οἱ ὄχλοι ἐξεπλήσσοντο ἐπὶ τῇ διδαχῇ αὐτοῦ.

\vs{34}Οἱ δὲ Φαρισαῖοι ἀκούσαντες ὅτι ἐφίμωσεν τοὺς Σαδδουκαίους συνήχθησαν ἐπὶ τὸ αὐτό,
\vs{35}καὶ ἐπηρώτησεν εἷς ἐξ αὐτῶν νομικὸς πειράζων αὐτόν·
\vs{36}Διδάσκαλε, ποία ἐντολὴ μεγάλη ἐν τῷ νόμῳ;
\vs{37}Ὁ δὲ ἔφη αὐτῷ· Ἀγαπήσεις κύριον τὸν Θεόν σου ἐν ὅλῃ τῇ καρδίᾳ σου καὶ ἐν ὅλῃ τῇ ψυχῇ σου καὶ ἐν ὅλῃ τῇ διανοίᾳ σου·
\vs{38}αὕτη ἐστὶν ἡ μεγάλη καὶ πρώτη ἐντολή.
\vs{39}δευτέρα δὲ ὁμοία αὐτῇ· Ἀγαπήσεις τὸν πλησίον σου ὡς σεαυτόν.
\vs{40}ἐν ταύταις ταῖς δυσὶν ἐντολαῖς ὅλος ὁ νόμος κρέμαται καὶ οἱ προφῆται.

\vs{41}Συνηγμένων δὲ τῶν Φαρισαίων ἐπηρώτησεν αὐτοὺς ὁ Ἰησοῦς
\vs{42}λέγων· Τί ὑμῖν δοκεῖ περὶ τοῦ Χριστοῦ; τίνος υἱός ἐστιν; Λέγουσιν αὐτῷ· Τοῦ Δαυίδ.
\vs{43}Λέγει αὐτοῖς· Πῶς οὖν Δαυὶδ ἐν Πνεύματι καλεῖ αὐτὸν Κύριον λέγων·
\begin{poetryblock}

\begin{quote} \vs{44}Εἶπεν Κύριος τῷ Κυρίῳ μου·\end{quote} 

\begin{quote}Κάθου ἐκ δεξιῶν μου,\end{quote} 

\begin{quote}Ἕως ἂν θῶ τοὺς ἐχθρούς σου\end{quote} 

\begin{quote}Ὑποκάτω τῶν ποδῶν σου;\end{quote}
\end{poetryblock}

\vs{45}Εἰ οὖν Δαυὶδ καλεῖ αὐτὸν Κύριον, πῶς υἱὸς αὐτοῦ ἐστιν;
\vs{46}Καὶ οὐδεὶς ἐδύνατο ἀποκριθῆναι αὐτῷ λόγον οὐδὲ ἐτόλμησέν τις ἀπ᾽ ἐκείνης τῆς ἡμέρας ἐπερωτῆσαι αὐτὸν οὐκέτι.

\ch{23}
Τότε ὁ Ἰησοῦς ἐλάλησεν τοῖς ὄχλοις καὶ τοῖς μαθηταῖς αὐτοῦ
\vs{2}λέγων· Ἐπὶ τῆς Μωϋσέως καθέδρας ἐκάθισαν οἱ γραμματεῖς καὶ οἱ Φαρισαῖοι.
\vs{3}πάντα οὖν ὅσα ἐὰν εἴπωσιν ὑμῖν ποιήσατε καὶ τηρεῖτε, κατὰ δὲ τὰ ἔργα αὐτῶν μὴ ποιεῖτε· λέγουσιν γὰρ καὶ οὐ ποιοῦσιν.
\vs{4}δεσμεύουσιν δὲ φορτία βαρέα καὶ δυσβάστακτα καὶ ἐπιτιθέασιν ἐπὶ τοὺς ὤμους τῶν ἀνθρώπων, αὐτοὶ δὲ τῷ δακτύλῳ αὐτῶν οὐ θέλουσιν κινῆσαι αὐτά.
\vs{5}Πάντα δὲ τὰ ἔργα αὐτῶν ποιοῦσιν πρὸς τὸ θεαθῆναι τοῖς ἀνθρώποις· πλατύνουσιν γὰρ τὰ φυλακτήρια αὐτῶν καὶ μεγαλύνουσιν τὰ κράσπεδα,
\vs{6}φιλοῦσιν δὲ τὴν πρωτοκλισίαν ἐν τοῖς δείπνοις καὶ τὰς πρωτοκαθεδρίας ἐν ταῖς συναγωγαῖς
\vs{7}καὶ τοὺς ἀσπασμοὺς ἐν ταῖς ἀγοραῖς καὶ καλεῖσθαι ὑπὸ τῶν ἀνθρώπων Ῥαββί.
\vs{8}Ὑμεῖς δὲ μὴ κληθῆτε Ῥαββί· εἷς γάρ ἐστιν ὑμῶν ὁ διδάσκαλος, πάντες δὲ ὑμεῖς ἀδελφοί ἐστε.
\vs{9}καὶ πατέρα μὴ καλέσητε ὑμῶν ἐπὶ τῆς γῆς, εἷς γάρ ἐστιν ὑμῶν ὁ Πατὴρ ὁ οὐράνιος.
\vs{10}μηδὲ κληθῆτε καθηγηταί, ὅτι καθηγητὴς ὑμῶν ἐστιν εἷς ὁ Χριστός.
\vs{11}ὁ δὲ μείζων ὑμῶν ἔσται ὑμῶν διάκονος.
\vs{12}Ὅστις δὲ ὑψώσει ἑαυτὸν ταπεινωθήσεται καὶ ὅστις ταπεινώσει ἑαυτὸν ὑψωθήσεται.

\vs{13}Οὐαὶ δὲ ὑμῖν, γραμματεῖς καὶ Φαρισαῖοι ὑποκριταί, ὅτι κλείετε τὴν βασιλείαν τῶν οὐρανῶν ἔμπροσθεν τῶν ἀνθρώπων· ὑμεῖς γὰρ οὐκ εἰσέρχεσθε οὐδὲ τοὺς εἰσερχομένους ἀφίετε εἰσελθεῖν.

\vs{15}Οὐαὶ ὑμῖν, γραμματεῖς καὶ Φαρισαῖοι ὑποκριταί, ὅτι περιάγετε τὴν θάλασσαν καὶ τὴν ξηρὰν ποιῆσαι ἕνα προσήλυτον, καὶ ὅταν γένηται ποιεῖτε αὐτὸν υἱὸν γεέννης διπλότερον ὑμῶν.

\vs{16}Οὐαὶ ὑμῖν, ὁδηγοὶ τυφλοὶ οἱ λέγοντες· Ὃς ἂν ὀμόσῃ ἐν τῷ ναῷ, οὐδέν ἐστιν· ὃς δ᾽ ἂν ὀμόσῃ ἐν τῷ χρυσῷ τοῦ ναοῦ, ὀφείλει.
\vs{17}μωροὶ καὶ τυφλοί, τίς γὰρ μείζων ἐστίν, ὁ χρυσὸς ἢ ὁ ναὸς ὁ ἁγιάσας τὸν χρυσόν;
\vs{18}καί· Ὃς ἂν ὀμόσῃ ἐν τῷ θυσιαστηρίῳ, οὐδέν ἐστιν· ὃς δ᾽ ἂν ὀμόσῃ ἐν τῷ δώρῳ τῷ ἐπάνω αὐτοῦ, ὀφείλει.
\vs{19}τυφλοί, τί γὰρ μεῖζον, τὸ δῶρον ἢ τὸ θυσιαστήριον τὸ ἁγιάζον τὸ δῶρον;
\vs{20}ὁ οὖν ὀμόσας ἐν τῷ θυσιαστηρίῳ ὀμνύει ἐν αὐτῷ καὶ ἐν πᾶσι τοῖς ἐπάνω αὐτοῦ·
\vs{21}καὶ ὁ ὀμόσας ἐν τῷ ναῷ ὀμνύει ἐν αὐτῷ καὶ ἐν τῷ κατοικοῦντι αὐτόν,
\vs{22}καὶ ὁ ὀμόσας ἐν τῷ οὐρανῷ ὀμνύει ἐν τῷ θρόνῳ τοῦ Θεοῦ καὶ ἐν τῷ καθημένῳ ἐπάνω αὐτοῦ.

\vs{23}Οὐαὶ ὑμῖν, γραμματεῖς καὶ Φαρισαῖοι ὑποκριταί, ὅτι ἀποδεκατοῦτε τὸ ἡδύοσμον καὶ τὸ ἄνηθον καὶ τὸ κύμινον καὶ ἀφήκατε τὰ βαρύτερα τοῦ νόμου, τὴν κρίσιν καὶ τὸ ἔλεος καὶ τὴν πίστιν· ταῦτα δὲ ἔδει ποιῆσαι κἀκεῖνα μὴ ἀφιέναι.
\vs{24}ὁδηγοὶ τυφλοί, οἱ διϋλίζοντες τὸν κώνωπα, τὴν δὲ κάμηλον καταπίνοντες.

\vs{25}Οὐαὶ ὑμῖν, γραμματεῖς καὶ Φαρισαῖοι ὑποκριταί, ὅτι καθαρίζετε τὸ ἔξωθεν τοῦ ποτηρίου καὶ τῆς παροψίδος, ἔσωθεν δὲ γέμουσιν ἐξ ἁρπαγῆς καὶ ἀκρασίας.
\vs{26}Φαρισαῖε τυφλέ, καθάρισον πρῶτον τὸ ἐντὸς τοῦ ποτηρίου, ἵνα γένηται καὶ τὸ ἐκτὸς αὐτοῦ καθαρόν.

\vs{27}Οὐαὶ ὑμῖν, γραμματεῖς καὶ Φαρισαῖοι ὑποκριταί, ὅτι παρομοιάζετε τάφοις κεκονιαμένοις, οἵτινες ἔξωθεν μὲν φαίνονται ὡραῖοι, ἔσωθεν δὲ γέμουσιν ὀστέων νεκρῶν καὶ πάσης ἀκαθαρσίας.
\vs{28}οὕτως καὶ ὑμεῖς ἔξωθεν μὲν φαίνεσθε τοῖς ἀνθρώποις δίκαιοι, ἔσωθεν δέ ἐστε μεστοὶ ὑποκρίσεως καὶ ἀνομίας.

\vs{29}Οὐαὶ ὑμῖν, γραμματεῖς καὶ Φαρισαῖοι ὑποκριταί, ὅτι οἰκοδομεῖτε τοὺς τάφους τῶν προφητῶν καὶ κοσμεῖτε τὰ μνημεῖα τῶν δικαίων,
\vs{30}καὶ λέγετε· Εἰ ἤμεθα ἐν ταῖς ἡμέραις τῶν πατέρων ἡμῶν, οὐκ ἂν ἤμεθα αὐτῶν κοινωνοὶ ἐν τῷ αἵματι τῶν προφητῶν.
\vs{31}ὥστε μαρτυρεῖτε ἑαυτοῖς ὅτι υἱοί ἐστε τῶν φονευσάντων τοὺς προφήτας.
\vs{32}καὶ ὑμεῖς πληρώσατε τὸ μέτρον τῶν πατέρων ὑμῶν.
\vs{33}ὄφεις, γεννήματα ἐχιδνῶν, πῶς φύγητε ἀπὸ τῆς κρίσεως τῆς γεέννης;

\vs{34}Διὰ τοῦτο ἰδοὺ ἐγὼ ἀποστέλλω πρὸς ὑμᾶς προφήτας καὶ σοφοὺς καὶ γραμματεῖς· ἐξ αὐτῶν ἀποκτενεῖτε καὶ σταυρώσετε καὶ ἐξ αὐτῶν μαστιγώσετε ἐν ταῖς συναγωγαῖς ὑμῶν καὶ διώξετε ἀπὸ πόλεως εἰς πόλιν·
\vs{35}ὅπως ἔλθῃ ἐφ᾽ ὑμᾶς πᾶν αἷμα δίκαιον ἐκχυννόμενον ἐπὶ τῆς γῆς ἀπὸ τοῦ αἵματος Ἅβελ τοῦ δικαίου ἕως τοῦ αἵματος Ζαχαρίου υἱοῦ Βαραχίου, ὃν ἐφονεύσατε μεταξὺ τοῦ ναοῦ καὶ τοῦ θυσιαστηρίου.
\vs{36}ἀμὴν λέγω ὑμῖν, ἥξει ταῦτα πάντα ἐπὶ τὴν γενεὰν ταύτην.

\vs{37}Ἰερουσαλὴμ Ἰερουσαλήμ, ἡ ἀποκτείνουσα τοὺς προφήτας καὶ λιθοβολοῦσα τοὺς ἀπεσταλμένους πρὸς αὐτήν, ποσάκις ἠθέλησα ἐπισυναγαγεῖν τὰ τέκνα σου, ὃν τρόπον ὄρνις ἐπισυνάγει τὰ νοσσία αὐτῆς ὑπὸ τὰς πτέρυγας, καὶ οὐκ ἠθελήσατε.
\vs{38}ἰδοὺ ἀφίεται ὑμῖν ὁ οἶκος ὑμῶν ἔρημος.
\vs{39}λέγω γὰρ ὑμῖν, οὐ μή με ἴδητε ἀπ᾽ ἄρτι ἕως ἂν εἴπητε· 
\begin{poetryblock}

\begin{quote}Εὐλογημένος ὁ ἐρχόμενος ἐν ὀνόματι Κυρίου.\end{quote}
\end{poetryblock}

\ch{24}
Καὶ ἐξελθὼν ὁ Ἰησοῦς ἀπὸ τοῦ ἱεροῦ ἐπορεύετο, καὶ προσῆλθον οἱ μαθηταὶ αὐτοῦ ἐπιδεῖξαι αὐτῷ τὰς οἰκοδομὰς τοῦ ἱεροῦ.
\vs{2}ὁ δὲ ἀποκριθεὶς εἶπεν αὐτοῖς· Οὐ βλέπετε ταῦτα πάντα; ἀμὴν λέγω ὑμῖν, οὐ μὴ ἀφεθῇ ὧδε λίθος ἐπὶ λίθον ὃς οὐ καταλυθήσεται.

\vs{3}Καθημένου δὲ αὐτοῦ ἐπὶ τοῦ ὄρους τῶν Ἐλαιῶν προσῆλθον αὐτῷ οἱ μαθηταὶ κατ᾽ ἰδίαν λέγοντες· Εἰπὲ ἡμῖν, πότε ταῦτα ἔσται καὶ τί τὸ σημεῖον τῆς σῆς παρουσίας καὶ συντελείας τοῦ αἰῶνος;

\vs{4}Καὶ ἀποκριθεὶς ὁ Ἰησοῦς εἶπεν αὐτοῖς· Βλέπετε μή τις ὑμᾶς πλανήσῃ·
\vs{5}πολλοὶ γὰρ ἐλεύσονται ἐπὶ τῷ ὀνόματί μου λέγοντες· Ἐγώ εἰμι ὁ Χριστός, καὶ πολλοὺς πλανήσουσιν.
\vs{6}μελλήσετε δὲ ἀκούειν πολέμους καὶ ἀκοὰς πολέμων· ὁρᾶτε μὴ θροεῖσθε· δεῖ γὰρ γενέσθαι, ἀλλ᾽ οὔπω ἐστὶν τὸ τέλος.
\vs{7}ἐγερθήσεται γὰρ ἔθνος ἐπὶ ἔθνος καὶ βασιλεία ἐπὶ βασιλείαν καὶ ἔσονται λιμοὶ καὶ σεισμοὶ κατὰ τόπους·
\vs{8}πάντα δὲ ταῦτα ἀρχὴ ὠδίνων.

\vs{9}Τότε παραδώσουσιν ὑμᾶς εἰς θλῖψιν καὶ ἀποκτενοῦσιν ὑμᾶς, καὶ ἔσεσθε μισούμενοι ὑπὸ πάντων τῶν ἐθνῶν διὰ τὸ ὄνομά μου.
\vs{10}καὶ τότε σκανδαλισθήσονται πολλοὶ καὶ ἀλλήλους παραδώσουσιν καὶ μισήσουσιν ἀλλήλους·
\vs{11}καὶ πολλοὶ ψευδοπροφῆται ἐγερθήσονται καὶ πλανήσουσιν πολλούς·
\vs{12}Καὶ διὰ τὸ πληθυνθῆναι τὴν ἀνομίαν ψυγήσεται ἡ ἀγάπη τῶν πολλῶν.
\vs{13}ὁ δὲ ὑπομείνας εἰς τέλος οὗτος σωθήσεται.
\vs{14}Καὶ κηρυχθήσεται τοῦτο τὸ εὐαγγέλιον τῆς βασιλείας ἐν ὅλῃ τῇ οἰκουμένῃ εἰς μαρτύριον πᾶσιν τοῖς ἔθνεσιν, καὶ τότε ἥξει τὸ τέλος.

\vs{15}Ὅταν οὖν ἴδητε Τὸ βδέλυγμα τῆς ἐρημώσεως τὸ ῥηθὲν διὰ Δανιὴλ τοῦ προφήτου ἑστὸς ἐν τόπῳ ἁγίῳ, ὁ ἀναγινώσκων νοείτω,
\vs{16}τότε οἱ ἐν τῇ Ἰουδαίᾳ φευγέτωσαν εἰς τὰ ὄρη,
\vs{17}ὁ ἐπὶ τοῦ δώματος μὴ καταβάτω ἆραι τὰ ἐκ τῆς οἰκίας αὐτοῦ,
\vs{18}καὶ ὁ ἐν τῷ ἀγρῷ μὴ ἐπιστρεψάτω ὀπίσω ἆραι τὸ ἱμάτιον αὐτοῦ.
\vs{19}Οὐαὶ δὲ ταῖς ἐν γαστρὶ ἐχούσαις καὶ ταῖς θηλαζούσαις ἐν ἐκείναις ταῖς ἡμέραις.

\vs{20}προσεύχεσθε δὲ ἵνα μὴ γένηται ἡ φυγὴ ὑμῶν χειμῶνος μηδὲ σαββάτῳ.
\vs{21}ἔσται γὰρ τότε θλῖψις μεγάλη οἵα οὐ γέγονεν ἀπ᾽ ἀρχῆς κόσμου ἕως τοῦ νῦν οὐδ᾽ οὐ μὴ γένηται.
\vs{22}καὶ εἰ μὴ ἐκολοβώθησαν αἱ ἡμέραι ἐκεῖναι, οὐκ ἂν ἐσώθη πᾶσα σάρξ· διὰ δὲ τοὺς ἐκλεκτοὺς κολοβωθήσονται αἱ ἡμέραι ἐκεῖναι.

\vs{23}Τότε ἐάν τις ὑμῖν εἴπῃ· Ἰδοὺ ὧδε ὁ Χριστός, ἤ· Ὧδε, μὴ πιστεύσητε·
\vs{24}ἐγερθήσονται γὰρ ψευδόχριστοι καὶ ψευδοπροφῆται καὶ δώσουσιν σημεῖα μεγάλα καὶ τέρατα ὥστε πλανῆσαι, εἰ δυνατὸν, καὶ τοὺς ἐκλεκτούς.
\vs{25}ἰδοὺ προείρηκα ὑμῖν.
\vs{26}Ἐὰν οὖν εἴπωσιν ὑμῖν· Ἰδοὺ ἐν τῇ ἐρήμῳ ἐστίν, μὴ ἐξέλθητε· Ἰδοὺ ἐν τοῖς ταμείοις, μὴ πιστεύσητε·
\vs{27}ὥσπερ γὰρ ἡ ἀστραπὴ ἐξέρχεται ἀπὸ ἀνατολῶν καὶ φαίνεται ἕως δυσμῶν, οὕτως ἔσται ἡ παρουσία τοῦ Υἱοῦ τοῦ ἀνθρώπου·
\vs{28}ὅπου ἐὰν ᾖ τὸ πτῶμα, ἐκεῖ συναχθήσονται οἱ ἀετοί.

\vs{29}Εὐθέως δὲ μετὰ τὴν θλῖψιν τῶν ἡμερῶν ἐκείνων 
\begin{poetryblock}

\begin{quote}Ὁ ἥλιος σκοτισθήσεται,\end{quote} 

\begin{quote}Καὶ ἡ σελήνη οὐ δώσει τὸ φέγγος αὐτῆς,\end{quote} 

\begin{quote}Καὶ οἱ ἀστέρες πεσοῦνται ἀπὸ τοῦ οὐρανοῦ,\end{quote} 

\begin{quote}Καὶ αἱ δυνάμεις τῶν οὐρανῶν σαλευθήσονται.\end{quote}
\end{poetryblock}

\vs{30}Καὶ τότε φανήσεται τὸ σημεῖον τοῦ Υἱοῦ τοῦ ἀνθρώπου ἐν οὐρανῷ, καὶ τότε κόψονται πᾶσαι αἱ φυλαὶ τῆς γῆς καὶ ὄψονται τὸν Υἱὸν τοῦ ἀνθρώπου ἐρχόμενον ἐπὶ τῶν νεφελῶν τοῦ οὐρανοῦ μετὰ δυνάμεως καὶ δόξης πολλῆς·
\vs{31}καὶ ἀποστελεῖ τοὺς ἀγγέλους αὐτοῦ μετὰ σάλπιγγος μεγάλης, καὶ ἐπισυνάξουσιν τοὺς ἐκλεκτοὺς αὐτοῦ ἐκ τῶν τεσσάρων ἀνέμων ἀπ᾽ ἄκρων οὐρανῶν ἕως τῶν ἄκρων αὐτῶν.
\vs{32}Ἀπὸ δὲ τῆς συκῆς μάθετε τὴν παραβολήν· ὅταν ἤδη ὁ κλάδος αὐτῆς γένηται ἁπαλὸς καὶ τὰ φύλλα ἐκφύῃ, γινώσκετε ὅτι ἐγγὺς τὸ θέρος·
\vs{33}οὕτως καὶ ὑμεῖς, ὅταν ἴδητε πάντα ταῦτα, γινώσκετε ὅτι ἐγγύς ἐστιν ἐπὶ θύραις.
\vs{34}ἀμὴν λέγω ὑμῖν ὅτι οὐ μὴ παρέλθῃ ἡ γενεὰ αὕτη ἕως ἂν πάντα ταῦτα γένηται.
\vs{35}ὁ οὐρανὸς καὶ ἡ γῆ παρελεύσεται, οἱ δὲ λόγοι μου οὐ μὴ παρέλθωσιν.

\vs{36}Περὶ δὲ τῆς ἡμέρας ἐκείνης καὶ ὥρας οὐδεὶς οἶδεν, οὐδὲ οἱ ἄγγελοι τῶν οὐρανῶν οὐδὲ ὁ Υἱός, εἰ μὴ ὁ Πατὴρ μόνος.

\vs{37}ὥσπερ γὰρ αἱ ἡμέραι τοῦ Νῶε, οὕτως ἔσται ἡ παρουσία τοῦ Υἱοῦ τοῦ ἀνθρώπου.
\vs{38}ὡς γὰρ ἦσαν ἐν ταῖς ἡμέραις ἐκείναις ταῖς πρὸ τοῦ κατακλυσμοῦ τρώγοντες καὶ πίνοντες, γαμοῦντες καὶ γαμίζοντες, ἄχρι ἧς ἡμέρας εἰσῆλθεν Νῶε εἰς τὴν κιβωτόν,
\vs{39}καὶ οὐκ ἔγνωσαν ἕως ἦλθεν ὁ κατακλυσμὸς καὶ ἦρεν ἅπαντας, οὕτως ἔσται καὶ ἡ παρουσία τοῦ Υἱοῦ τοῦ ἀνθρώπου.
\vs{40}τότε δύο ἔσονται ἐν τῷ ἀγρῷ, εἷς παραλαμβάνεται καὶ εἷς ἀφίεται·
\vs{41}δύο ἀλήθουσαι ἐν τῷ μύλῳ, μία παραλαμβάνεται καὶ μία ἀφίεται.
\vs{42}Γρηγορεῖτε οὖν, ὅτι οὐκ οἴδατε ποίᾳ ἡμέρᾳ ὁ κύριος ὑμῶν ἔρχεται.
\vs{43}ἐκεῖνο δὲ γινώσκετε ὅτι εἰ ᾔδει ὁ οἰκοδεσπότης ποίᾳ φυλακῇ ὁ κλέπτης ἔρχεται, ἐγρηγόρησεν ἂν καὶ οὐκ ἂν εἴασεν διορυχθῆναι τὴν οἰκίαν αὐτοῦ.
\vs{44}διὰ τοῦτο καὶ ὑμεῖς γίνεσθε ἕτοιμοι, ὅτι ᾗ οὐ δοκεῖτε ὥρᾳ ὁ Υἱὸς τοῦ ἀνθρώπου ἔρχεται.

\vs{45}Τίς ἄρα ἐστὶν ὁ πιστὸς δοῦλος καὶ φρόνιμος ὃν κατέστησεν ὁ κύριος ἐπὶ τῆς οἰκετείας αὐτοῦ τοῦ δοῦναι αὐτοῖς τὴν τροφὴν ἐν καιρῷ;
\vs{46}μακάριος ὁ δοῦλος ἐκεῖνος ὃν ἐλθὼν ὁ κύριος αὐτοῦ εὑρήσει οὕτως ποιοῦντα·
\vs{47}ἀμὴν λέγω ὑμῖν ὅτι ἐπὶ πᾶσιν τοῖς ὑπάρχουσιν αὐτοῦ καταστήσει αὐτόν.
\vs{48}Ἐὰν δὲ εἴπῃ ὁ κακὸς δοῦλος ἐκεῖνος ἐν τῇ καρδίᾳ αὐτοῦ· Χρονίζει μου ὁ κύριος,
\vs{49}καὶ ἄρξηται τύπτειν τοὺς συνδούλους αὐτοῦ, ἐσθίῃ δὲ καὶ πίνῃ μετὰ τῶν μεθυόντων,
\vs{50}ἥξει ὁ κύριος τοῦ δούλου ἐκείνου ἐν ἡμέρᾳ ᾗ οὐ προσδοκᾷ καὶ ἐν ὥρᾳ ᾗ οὐ γινώσκει,
\vs{51}καὶ διχοτομήσει αὐτὸν καὶ τὸ μέρος αὐτοῦ μετὰ τῶν ὑποκριτῶν θήσει· ἐκεῖ ἔσται ὁ κλαυθμὸς καὶ ὁ βρυγμὸς τῶν ὀδόντων.

\ch{25}
Τότε ὁμοιωθήσεται ἡ βασιλεία τῶν οὐρανῶν δέκα παρθένοις, αἵτινες λαβοῦσαι τὰς λαμπάδας ἑαυτῶν ἐξῆλθον εἰς ὑπάντησιν τοῦ νυμφίου.
\vs{2}πέντε δὲ ἐξ αὐτῶν ἦσαν μωραὶ καὶ πέντε φρόνιμοι.
\vs{3}αἱ γὰρ μωραὶ λαβοῦσαι τὰς λαμπάδας αὐτῶν οὐκ ἔλαβον μεθ᾽ ἑαυτῶν ἔλαιον.
\vs{4}αἱ δὲ φρόνιμοι ἔλαβον ἔλαιον ἐν τοῖς ἀγγείοις μετὰ τῶν λαμπάδων ἑαυτῶν.
\vs{5}χρονίζοντος δὲ τοῦ νυμφίου ἐνύσταξαν πᾶσαι καὶ ἐκάθευδον.
\vs{6}Μέσης δὲ νυκτὸς κραυγὴ γέγονεν· Ἰδοὺ ὁ νυμφίος, ἐξέρχεσθε εἰς ἀπάντησιν αὐτοῦ.
\vs{7}Τότε ἠγέρθησαν πᾶσαι αἱ παρθένοι ἐκεῖναι καὶ ἐκόσμησαν τὰς λαμπάδας ἑαυτῶν.
\vs{8}αἱ δὲ μωραὶ ταῖς φρονίμοις εἶπαν· Δότε ἡμῖν ἐκ τοῦ ἐλαίου ὑμῶν, ὅτι αἱ λαμπάδες ἡμῶν σβέννυνται.
\vs{9}Ἀπεκρίθησαν δὲ αἱ φρόνιμοι λέγουσαι· Μήποτε οὐ μὴ ἀρκέσῃ ἡμῖν καὶ ὑμῖν· πορεύεσθε μᾶλλον πρὸς τοὺς πωλοῦντας καὶ ἀγοράσατε ἑαυταῖς.
\vs{10}Ἀπερχομένων δὲ αὐτῶν ἀγοράσαι ἦλθεν ὁ νυμφίος, καὶ αἱ ἕτοιμοι εἰσῆλθον μετ᾽ αὐτοῦ εἰς τοὺς γάμους καὶ ἐκλείσθη ἡ θύρα.
\vs{11}Ὕστερον δὲ ἔρχονται καὶ αἱ λοιπαὶ παρθένοι λέγουσαι· Κύριε κύριε, ἄνοιξον ἡμῖν.
\vs{12}Ὁ δὲ ἀποκριθεὶς εἶπεν· Ἀμὴν λέγω ὑμῖν, οὐκ οἶδα ὑμᾶς.
\vs{13}Γρηγορεῖτε οὖν, ὅτι οὐκ οἴδατε τὴν ἡμέραν οὐδὲ τὴν ὥραν.
\vs{14}Ὥσπερ γὰρ ἄνθρωπος ἀποδημῶν ἐκάλεσεν τοὺς ἰδίους δούλους καὶ παρέδωκεν αὐτοῖς τὰ ὑπάρχοντα αὐτοῦ,
\vs{15}καὶ ᾧ μὲν ἔδωκεν πέντε τάλαντα, ᾧ δὲ δύο, ᾧ δὲ ἕν, ἑκάστῳ κατὰ τὴν ἰδίαν δύναμιν, καὶ ἀπεδήμησεν. Εὐθέως
\vs{16}Πορευθεὶς ὁ τὰ πέντε τάλαντα λαβὼν ἠργάσατο ἐν αὐτοῖς καὶ ἐκέρδησεν ἄλλα πέντε·
\vs{17}ὡσαύτως ὁ τὰ δύο ἐκέρδησεν ἄλλα δύο.
\vs{18}ὁ δὲ τὸ ἓν λαβὼν ἀπελθὼν ὤρυξεν γῆν καὶ ἔκρυψεν τὸ ἀργύριον τοῦ κυρίου αὐτοῦ.
\vs{19}Μετὰ δὲ πολὺν χρόνον ἔρχεται ὁ κύριος τῶν δούλων ἐκείνων καὶ συναίρει λόγον μετ᾽ αὐτῶν.
\vs{20}καὶ προσελθὼν ὁ τὰ πέντε τάλαντα λαβὼν προσήνεγκεν ἄλλα πέντε τάλαντα λέγων· Κύριε, πέντε τάλαντά μοι παρέδωκας· ἴδε ἄλλα πέντε τάλαντα ἐκέρδησα.
\vs{21}Ἔφη αὐτῷ ὁ κύριος αὐτοῦ· Εὖ, δοῦλε ἀγαθὲ καὶ πιστέ, ἐπὶ ὀλίγα ἦς πιστός, ἐπὶ πολλῶν σε καταστήσω· εἴσελθε εἰς τὴν χαρὰν τοῦ κυρίου σου.
\vs{22}Προσελθὼν δὲ καὶ ὁ τὰ δύο τάλαντα εἶπεν· Κύριε, δύο τάλαντά μοι παρέδωκας· ἴδε ἄλλα δύο τάλαντα ἐκέρδησα.
\vs{23}Ἔφη αὐτῷ ὁ κύριος αὐτοῦ· Εὖ, δοῦλε ἀγαθὲ καὶ πιστέ, ἐπὶ ὀλίγα ἦς πιστός, ἐπὶ πολλῶν σε καταστήσω· εἴσελθε εἰς τὴν χαρὰν τοῦ κυρίου σου.
\vs{24}Προσελθὼν δὲ καὶ ὁ τὸ ἓν τάλαντον εἰληφὼς εἶπεν· Κύριε, ἔγνων σε ὅτι σκληρὸς εἶ ἄνθρωπος, θερίζων ὅπου οὐκ ἔσπειρας καὶ συνάγων ὅθεν οὐ διεσκόρπισας,
\vs{25}καὶ φοβηθεὶς ἀπελθὼν ἔκρυψα τὸ τάλαντόν σου ἐν τῇ γῇ· ἴδε ἔχεις τὸ σόν.
\vs{26}Ἀποκριθεὶς δὲ ὁ κύριος αὐτοῦ εἶπεν αὐτῷ· Πονηρὲ δοῦλε καὶ ὀκνηρέ, ᾔδεις ὅτι θερίζω ὅπου οὐκ ἔσπειρα καὶ συνάγω ὅθεν οὐ διεσκόρπισα;
\vs{27}ἔδει σε οὖν βαλεῖν τὰ ἀργύριά μου τοῖς τραπεζίταις, καὶ ἐλθὼν ἐγὼ ἐκομισάμην ἂν τὸ ἐμὸν σὺν τόκῳ.
\vs{28}Ἄρατε οὖν ἀπ᾽ αὐτοῦ τὸ τάλαντον καὶ δότε τῷ ἔχοντι τὰ δέκα τάλαντα·
\vs{29}τῷ γὰρ ἔχοντι παντὶ δοθήσεται καὶ περισσευθήσεται, τοῦ δὲ μὴ ἔχοντος καὶ ὃ ἔχει ἀρθήσεται ἀπ᾽ αὐτοῦ.
\vs{30}καὶ τὸν ἀχρεῖον δοῦλον ἐκβάλετε εἰς τὸ σκότος τὸ ἐξώτερον· ἐκεῖ ἔσται ὁ κλαυθμὸς καὶ ὁ βρυγμὸς τῶν ὀδόντων.

\vs{31}Ὅταν δὲ ἔλθῃ ὁ υἱὸς τοῦ ἀνθρώπου ἐν τῇ δόξῃ αὐτοῦ καὶ πάντες οἱ ἄγγελοι μετ᾽ αὐτοῦ, τότε καθίσει ἐπὶ θρόνου δόξης αὐτοῦ·
\vs{32}καὶ συναχθήσονται ἔμπροσθεν αὐτοῦ πάντα τὰ ἔθνη, καὶ ἀφορίσει αὐτοὺς ἀπ᾽ ἀλλήλων, ὥσπερ ὁ ποιμὴν ἀφορίζει τὰ πρόβατα ἀπὸ τῶν ἐρίφων,
\vs{33}καὶ στήσει τὰ μὲν πρόβατα ἐκ δεξιῶν αὐτοῦ, τὰ δὲ ἐρίφια ἐξ εὐωνύμων.
\vs{34}Τότε ἐρεῖ ὁ Βασιλεὺς τοῖς ἐκ δεξιῶν αὐτοῦ· Δεῦτε οἱ εὐλογημένοι τοῦ Πατρός μου, κληρονομήσατε τὴν ἡτοιμασμένην ὑμῖν βασιλείαν ἀπὸ καταβολῆς κόσμου.
\vs{35}ἐπείνασα γὰρ καὶ ἐδώκατέ μοι φαγεῖν, ἐδίψησα καὶ ἐποτίσατέ με, ξένος ἤμην καὶ συνηγάγετέ με,
\vs{36}γυμνὸς καὶ περιεβάλετέ με, ἠσθένησα καὶ ἐπεσκέψασθέ με, ἐν φυλακῇ ἤμην καὶ ἤλθατε πρός με.
\vs{37}Τότε ἀποκριθήσονται αὐτῷ οἱ δίκαιοι λέγοντες· Κύριε, πότε σε εἴδομεν πεινῶντα καὶ ἐθρέψαμεν, ἢ διψῶντα καὶ ἐποτίσαμεν;
\vs{38}πότε δέ σε εἴδομεν ξένον καὶ συνηγάγομεν, ἢ γυμνὸν καὶ περιεβάλομεν;
\vs{39}πότε δέ σε εἴδομεν ἀσθενοῦντα ἢ ἐν φυλακῇ καὶ ἤλθομεν πρός σε;
\vs{40}Καὶ ἀποκριθεὶς ὁ Βασιλεὺς ἐρεῖ αὐτοῖς· Ἀμὴν λέγω ὑμῖν, ἐφ᾽ ὅσον ἐποιήσατε ἑνὶ τούτων τῶν ἀδελφῶν μου τῶν ἐλαχίστων, ἐμοὶ ἐποιήσατε.
\vs{41}Τότε ἐρεῖ καὶ τοῖς ἐξ εὐωνύμων· Πορεύεσθε ἀπ᾽ ἐμοῦ οἱ κατηραμένοι εἰς τὸ πῦρ τὸ αἰώνιον τὸ ἡτοιμασμένον τῷ διαβόλῳ καὶ τοῖς ἀγγέλοις αὐτοῦ.
\vs{42}ἐπείνασα γὰρ καὶ οὐκ ἐδώκατέ μοι φαγεῖν, ἐδίψησα καὶ οὐκ ἐποτίσατέ με,
\vs{43}ξένος ἤμην καὶ οὐ συνηγάγετέ με, γυμνὸς καὶ οὐ περιεβάλετέ με, ἀσθενὴς καὶ ἐν φυλακῇ καὶ οὐκ ἐπεσκέψασθέ με.
\vs{44}Τότε ἀποκριθήσονται καὶ αὐτοὶ λέγοντες· Κύριε, πότε σε εἴδομεν πεινῶντα ἢ διψῶντα ἢ ξένον ἢ γυμνὸν ἢ ἀσθενῆ ἢ ἐν φυλακῇ καὶ οὐ διηκονήσαμέν σοι;
\vs{45}Τότε ἀποκριθήσεται αὐτοῖς λέγων· Ἀμὴν λέγω ὑμῖν, ἐφ᾽ ὅσον οὐκ ἐποιήσατε ἑνὶ τούτων τῶν ἐλαχίστων, οὐδὲ ἐμοὶ ἐποιήσατε.
\vs{46}Καὶ ἀπελεύσονται οὗτοι εἰς κόλασιν αἰώνιον, οἱ δὲ δίκαιοι εἰς ζωὴν αἰώνιον.

\ch{26}
Καὶ ἐγένετο ὅτε ἐτέλεσεν ὁ Ἰησοῦς πάντας τοὺς λόγους τούτους, εἶπεν τοῖς μαθηταῖς αὐτοῦ·
\vs{2}Οἴδατε ὅτι μετὰ δύο ἡμέρας τὸ πάσχα γίνεται, καὶ ὁ Υἱὸς τοῦ ἀνθρώπου παραδίδοται εἰς τὸ σταυρωθῆναι.

\vs{3}Τότε συνήχθησαν οἱ ἀρχιερεῖς καὶ οἱ πρεσβύτεροι τοῦ λαοῦ εἰς τὴν αὐλὴν τοῦ ἀρχιερέως τοῦ λεγομένου Καϊάφα
\vs{4}καὶ συνεβουλεύσαντο ἵνα τὸν Ἰησοῦν δόλῳ κρατήσωσιν καὶ ἀποκτείνωσιν·
\vs{5}ἔλεγον δέ· Μὴ ἐν τῇ ἑορτῇ, ἵνα μὴ θόρυβος γένηται ἐν τῷ λαῷ.

\vs{6}Τοῦ δὲ Ἰησοῦ γενομένου ἐν Βηθανίᾳ ἐν οἰκίᾳ Σίμωνος τοῦ λεπροῦ,
\vs{7}προσῆλθεν αὐτῷ γυνὴ ἔχουσα ἀλάβαστρον μύρου βαρυτίμου καὶ κατέχεεν ἐπὶ τῆς κεφαλῆς αὐτοῦ ἀνακειμένου.
\vs{8}Ἰδόντες δὲ οἱ μαθηταὶ ἠγανάκτησαν λέγοντες· Εἰς τί ἡ ἀπώλεια αὕτη;
\vs{9}ἐδύνατο γὰρ τοῦτο πραθῆναι πολλοῦ καὶ δοθῆναι πτωχοῖς.
\vs{10}Γνοὺς δὲ ὁ Ἰησοῦς εἶπεν αὐτοῖς· Τί κόπους παρέχετε τῇ γυναικί; ἔργον γὰρ καλὸν ἠργάσατο εἰς ἐμέ·
\vs{11}πάντοτε γὰρ τοὺς πτωχοὺς ἔχετε μεθ᾽ ἑαυτῶν, ἐμὲ δὲ οὐ πάντοτε ἔχετε·
\vs{12}βαλοῦσα γὰρ αὕτη τὸ μύρον τοῦτο ἐπὶ τοῦ σώματός μου πρὸς τὸ ἐνταφιάσαι με ἐποίησεν.
\vs{13}ἀμὴν λέγω ὑμῖν, ὅπου ἐὰν κηρυχθῇ τὸ εὐαγγέλιον τοῦτο ἐν ὅλῳ τῷ κόσμῳ, λαληθήσεται καὶ ὃ ἐποίησεν αὕτη εἰς μνημόσυνον αὐτῆς.

\vs{14}Τότε πορευθεὶς εἷς τῶν δώδεκα, ὁ λεγόμενος Ἰούδας Ἰσκαριώτης, πρὸς τοὺς ἀρχιερεῖς
\vs{15}εἶπεν· Τί θέλετέ μοι δοῦναι, κἀγὼ ὑμῖν παραδώσω αὐτόν; οἱ δὲ ἔστησαν αὐτῷ τριάκοντα ἀργύρια.
\vs{16}καὶ ἀπὸ τότε ἐζήτει εὐκαιρίαν ἵνα αὐτὸν παραδῷ.
\vs{17}Τῇ δὲ πρώτῃ τῶν ἀζύμων προσῆλθον οἱ μαθηταὶ τῷ Ἰησοῦ λέγοντες· Ποῦ θέλεις ἑτοιμάσωμέν σοι φαγεῖν τὸ πάσχα;
\vs{18}Ὁ δὲ εἶπεν· Ὑπάγετε εἰς τὴν πόλιν πρὸς τὸν δεῖνα καὶ εἴπατε αὐτῷ· Ὁ Διδάσκαλος λέγει· Ὁ καιρός μου ἐγγύς ἐστιν, πρὸς σὲ ποιῶ τὸ πάσχα μετὰ τῶν μαθητῶν μου.
\vs{19}καὶ ἐποίησαν οἱ μαθηταὶ ὡς συνέταξεν αὐτοῖς ὁ Ἰησοῦς καὶ ἡτοίμασαν τὸ πάσχα.

\vs{20}Ὀψίας δὲ γενομένης ἀνέκειτο μετὰ τῶν δώδεκα.
\vs{21}καὶ ἐσθιόντων αὐτῶν εἶπεν· Ἀμὴν λέγω ὑμῖν ὅτι εἷς ἐξ ὑμῶν παραδώσει με.
\vs{22}Καὶ λυπούμενοι σφόδρα ἤρξαντο λέγειν αὐτῷ εἷς ἕκαστος· Μήτι ἐγώ εἰμι, Κύριε;
\vs{23}Ὁ δὲ ἀποκριθεὶς εἶπεν· Ὁ ἐμβάψας μετ᾽ ἐμοῦ τὴν χεῖρα ἐν τῷ τρυβλίῳ οὗτός με παραδώσει.
\vs{24}ὁ μὲν Υἱὸς τοῦ ἀνθρώπου ὑπάγει καθὼς γέγραπται περὶ αὐτοῦ, οὐαὶ δὲ τῷ ἀνθρώπῳ ἐκείνῳ δι᾽ οὗ ὁ Υἱὸς τοῦ ἀνθρώπου παραδίδοται· καλὸν ἦν αὐτῷ εἰ οὐκ ἐγεννήθη ὁ ἄνθρωπος ἐκεῖνος.
\vs{25}Ἀποκριθεὶς δὲ Ἰούδας ὁ παραδιδοὺς αὐτὸν εἶπεν· Μήτι ἐγώ εἰμι, ῥαββί; Λέγει αὐτῷ· Σὺ εἶπας.

\vs{26}Ἐσθιόντων δὲ αὐτῶν λαβὼν ὁ Ἰησοῦς ἄρτον καὶ εὐλογήσας ἔκλασεν καὶ δοὺς τοῖς μαθηταῖς εἶπεν· Λάβετε φάγετε, τοῦτό ἐστιν τὸ σῶμά μου.
\vs{27}Καὶ λαβὼν ποτήριον καὶ εὐχαριστήσας ἔδωκεν αὐτοῖς λέγων· Πίετε ἐξ αὐτοῦ πάντες,
\vs{28}τοῦτο γάρ ἐστιν τὸ αἷμά μου τῆς διαθήκης τὸ περὶ πολλῶν ἐκχυννόμενον εἰς ἄφεσιν ἁμαρτιῶν.
\vs{29}λέγω δὲ ὑμῖν, οὐ μὴ πίω ἀπ᾽ ἄρτι ἐκ τούτου τοῦ γενήματος τῆς ἀμπέλου ἕως τῆς ἡμέρας ἐκείνης ὅταν αὐτὸ πίνω μεθ᾽ ὑμῶν καινὸν ἐν τῇ βασιλείᾳ τοῦ Πατρός μου.
\vs{30}Καὶ ὑμνήσαντες ἐξῆλθον εἰς τὸ ὄρος τῶν Ἐλαιῶν.

\vs{31}Τότε λέγει αὐτοῖς ὁ Ἰησοῦς· Πάντες ὑμεῖς σκανδαλισθήσεσθε ἐν ἐμοὶ ἐν τῇ νυκτὶ ταύτῃ, γέγραπται γάρ· 
\begin{poetryblock}

\begin{quote}Πατάξω τὸν ποιμένα,\end{quote} 

\begin{quote}Καὶ διασκορπισθήσονται τὰ πρόβατα τῆς ποίμνης.\end{quote}
\end{poetryblock}

\vs{32}Μετὰ δὲ τὸ ἐγερθῆναί με προάξω ὑμᾶς εἰς τὴν Γαλιλαίαν.
\vs{33}Ἀποκριθεὶς δὲ ὁ Πέτρος εἶπεν αὐτῷ· Εἰ πάντες σκανδαλισθήσονται ἐν σοί, ἐγὼ οὐδέποτε σκανδαλισθήσομαι.
\vs{34}Ἔφη αὐτῷ ὁ Ἰησοῦς· Ἀμὴν λέγω σοι ὅτι ἐν ταύτῃ τῇ νυκτὶ πρὶν ἀλέκτορα φωνῆσαι τρὶς ἀπαρνήσῃ με.
\vs{35}Λέγει αὐτῷ ὁ Πέτρος· Κἂν δέῃ με σὺν σοὶ ἀποθανεῖν, οὐ μή σε ἀπαρνήσομαι. ὁμοίως καὶ πάντες οἱ μαθηταὶ εἶπαν.

\vs{36}Τότε ἔρχεται μετ᾽ αὐτῶν ὁ Ἰησοῦς εἰς χωρίον λεγόμενον Γεθσημανὶ καὶ λέγει τοῖς μαθηταῖς· Καθίσατε αὐτοῦ ἕως οὗ ἀπελθὼν ἐκεῖ προσεύξωμαι.
\vs{37}Καὶ παραλαβὼν τὸν Πέτρον καὶ τοὺς δύο υἱοὺς Ζεβεδαίου ἤρξατο λυπεῖσθαι καὶ ἀδημονεῖν.
\vs{38}τότε λέγει αὐτοῖς· Περίλυπός ἐστιν ἡ ψυχή μου ἕως θανάτου· μείνατε ὧδε καὶ γρηγορεῖτε μετ᾽ ἐμοῦ.
\vs{39}Καὶ προελθὼν μικρὸν ἔπεσεν ἐπὶ πρόσωπον αὐτοῦ προσευχόμενος καὶ λέγων· Πάτερ μου, εἰ δυνατόν ἐστιν, παρελθάτω ἀπ᾽ ἐμοῦ τὸ ποτήριον τοῦτο· πλὴν οὐχ ὡς ἐγὼ θέλω ἀλλ᾽ ὡς σύ.
\vs{40}Καὶ ἔρχεται πρὸς τοὺς μαθητὰς καὶ εὑρίσκει αὐτοὺς καθεύδοντας, καὶ λέγει τῷ Πέτρῳ· Οὕτως οὐκ ἰσχύσατε μίαν ὥραν γρηγορῆσαι μετ᾽ ἐμοῦ;
\vs{41}γρηγορεῖτε καὶ προσεύχεσθε, ἵνα μὴ εἰσέλθητε εἰς πειρασμόν· τὸ μὲν πνεῦμα πρόθυμον ἡ δὲ σὰρξ ἀσθενής.
\vs{42}Πάλιν ἐκ δευτέρου ἀπελθὼν προσηύξατο λέγων· Πάτερ μου, εἰ οὐ δύναται τοῦτο παρελθεῖν ἐὰν μὴ αὐτὸ πίω, γενηθήτω τὸ θέλημά σου.
\vs{43}καὶ ἐλθὼν πάλιν εὗρεν αὐτοὺς καθεύδοντας, ἦσαν γὰρ αὐτῶν οἱ ὀφθαλμοὶ βεβαρημένοι.
\vs{44}Καὶ ἀφεὶς αὐτοὺς πάλιν ἀπελθὼν προσηύξατο ἐκ τρίτου τὸν αὐτὸν λόγον εἰπὼν πάλιν.
\vs{45}τότε ἔρχεται πρὸς τοὺς μαθητὰς καὶ λέγει αὐτοῖς· Καθεύδετε τὸ λοιπὸν καὶ ἀναπαύεσθε· ἰδοὺ ἤγγικεν ἡ ὥρα καὶ ὁ Υἱὸς τοῦ ἀνθρώπου παραδίδοται εἰς χεῖρας ἁμαρτωλῶν.
\vs{46}ἐγείρεσθε ἄγωμεν· ἰδοὺ ἤγγικεν ὁ παραδιδούς με.

\vs{47}Καὶ ἔτι αὐτοῦ λαλοῦντος ἰδοὺ Ἰούδας εἷς τῶν δώδεκα ἦλθεν καὶ μετ᾽ αὐτοῦ ὄχλος πολὺς μετὰ μαχαιρῶν καὶ ξύλων ἀπὸ τῶν ἀρχιερέων καὶ πρεσβυτέρων τοῦ λαοῦ.
\vs{48}Ὁ δὲ παραδιδοὺς αὐτὸν ἔδωκεν αὐτοῖς σημεῖον λέγων· Ὃν ἂν φιλήσω αὐτός ἐστιν, κρατήσατε αὐτόν.
\vs{49}καὶ εὐθέως προσελθὼν τῷ Ἰησοῦ εἶπεν· Χαῖρε, ῥαββί, καὶ κατεφίλησεν αὐτόν.
\vs{50}Ὁ δὲ Ἰησοῦς εἶπεν αὐτῷ· Ἑταῖρε, ἐφ᾽ ὃ πάρει. Τότε προσελθόντες ἐπέβαλον τὰς χεῖρας ἐπὶ τὸν Ἰησοῦν καὶ ἐκράτησαν αὐτόν.
\vs{51}καὶ ἰδοὺ εἷς τῶν μετὰ Ἰησοῦ ἐκτείνας τὴν χεῖρα ἀπέσπασεν τὴν μάχαιραν αὐτοῦ καὶ πατάξας τὸν δοῦλον τοῦ ἀρχιερέως ἀφεῖλεν αὐτοῦ τὸ ὠτίον.
\vs{52}Τότε λέγει αὐτῷ ὁ Ἰησοῦς· Ἀπόστρεψον τὴν μάχαιράν σου εἰς τὸν τόπον αὐτῆς· πάντες γὰρ οἱ λαβόντες μάχαιραν ἐν μαχαίρῃ ἀπολοῦνται.
\vs{53}ἢ δοκεῖς ὅτι οὐ δύναμαι παρακαλέσαι τὸν Πατέρα μου, καὶ παραστήσει μοι ἄρτι πλείω δώδεκα λεγιῶνας ἀγγέλων;
\vs{54}πῶς οὖν πληρωθῶσιν αἱ γραφαὶ ὅτι οὕτως δεῖ γενέσθαι;
\vs{55}Ἐν ἐκείνῃ τῇ ὥρᾳ εἶπεν ὁ Ἰησοῦς τοῖς ὄχλοις· Ὡς ἐπὶ λῃστὴν ἐξήλθατε μετὰ μαχαιρῶν καὶ ξύλων συλλαβεῖν με; καθ᾽ ἡμέραν ἐν τῷ ἱερῷ ἐκαθεζόμην διδάσκων καὶ οὐκ ἐκρατήσατέ με.
\vs{56}Τοῦτο δὲ ὅλον γέγονεν ἵνα πληρωθῶσιν αἱ γραφαὶ τῶν προφητῶν. Τότε οἱ μαθηταὶ πάντες ἀφέντες αὐτὸν ἔφυγον.

\vs{57}Οἱ δὲ κρατήσαντες τὸν Ἰησοῦν ἀπήγαγον πρὸς Καϊάφαν τὸν ἀρχιερέα, ὅπου οἱ γραμματεῖς καὶ οἱ πρεσβύτεροι συνήχθησαν.
\vs{58}ὁ δὲ Πέτρος ἠκολούθει αὐτῷ ἀπὸ μακρόθεν ἕως τῆς αὐλῆς τοῦ ἀρχιερέως καὶ εἰσελθὼν ἔσω ἐκάθητο μετὰ τῶν ὑπηρετῶν ἰδεῖν τὸ τέλος.

\vs{59}Οἱ δὲ ἀρχιερεῖς καὶ τὸ συνέδριον ὅλον ἐζήτουν ψευδομαρτυρίαν κατὰ τοῦ Ἰησοῦ ὅπως αὐτὸν θανατώσωσιν,
\vs{60}καὶ οὐχ εὗρον πολλῶν προσελθόντων ψευδομαρτύρων. Ὕστερον δὲ προσελθόντες δύο
\vs{61}εἶπαν· Οὗτος ἔφη· Δύναμαι καταλῦσαι τὸν ναὸν τοῦ Θεοῦ καὶ διὰ τριῶν ἡμερῶν οἰκοδομῆσαι.
\vs{62}Καὶ ἀναστὰς ὁ ἀρχιερεὺς εἶπεν αὐτῷ· Οὐδὲν ἀποκρίνῃ τί οὗτοί σου καταμαρτυροῦσιν;
\vs{63}Ὁ δὲ Ἰησοῦς ἐσιώπα. Καὶ ὁ ἀρχιερεὺς εἶπεν αὐτῷ· Ἐξορκίζω σε κατὰ τοῦ Θεοῦ τοῦ ζῶντος ἵνα ἡμῖν εἴπῃς εἰ σὺ εἶ ὁ Χριστὸς ὁ Υἱὸς τοῦ Θεοῦ.
\vs{64}Λέγει αὐτῷ ὁ Ἰησοῦς· Σὺ εἶπας. πλὴν λέγω ὑμῖν· ἀπ᾽ ἄρτι ὄψεσθε τὸν Υἱὸν τοῦ ἀνθρώπου καθήμενον ἐκ δεξιῶν τῆς δυνάμεως καὶ ἐρχόμενον ἐπὶ τῶν νεφελῶν τοῦ οὐρανοῦ.
\vs{65}Τότε ὁ ἀρχιερεὺς διέρρηξεν τὰ ἱμάτια αὐτοῦ λέγων· Ἐβλασφήμησεν· τί ἔτι χρείαν ἔχομεν μαρτύρων; ἴδε νῦν ἠκούσατε τὴν βλασφημίαν·
\vs{66}τί ὑμῖν δοκεῖ; Οἱ δὲ ἀποκριθέντες εἶπαν· Ἔνοχος θανάτου ἐστίν.

\vs{67}Τότε ἐνέπτυσαν εἰς τὸ πρόσωπον αὐτοῦ καὶ ἐκολάφισαν αὐτόν, οἱ δὲ ἐράπισαν
\vs{68}λέγοντες· Προφήτευσον ἡμῖν, Χριστέ, τίς ἐστιν ὁ παίσας σε;

\vs{69}Ὁ δὲ Πέτρος ἐκάθητο ἔξω ἐν τῇ αὐλῇ· καὶ προσῆλθεν αὐτῷ μία παιδίσκη λέγουσα· Καὶ σὺ ἦσθα μετὰ Ἰησοῦ τοῦ Γαλιλαίου.
\vs{70}Ὁ δὲ ἠρνήσατο ἔμπροσθεν πάντων λέγων· Οὐκ οἶδα τί λέγεις.
\vs{71}Ἐξελθόντα δὲ εἰς τὸν πυλῶνα εἶδεν αὐτὸν ἄλλη καὶ λέγει τοῖς ἐκεῖ· Οὗτος ἦν μετὰ Ἰησοῦ τοῦ Ναζωραίου.
\vs{72}Καὶ πάλιν ἠρνήσατο μετὰ ὅρκου ὅτι Οὐκ οἶδα τὸν ἄνθρωπον.
\vs{73}Μετὰ μικρὸν δὲ προσελθόντες οἱ ἑστῶτες εἶπον τῷ Πέτρῳ· Ἀληθῶς καὶ σὺ ἐξ αὐτῶν εἶ, καὶ γὰρ ἡ λαλιά σου δῆλόν σε ποιεῖ.
\vs{74}Τότε ἤρξατο καταθεματίζειν καὶ ὀμνύειν ὅτι Οὐκ οἶδα τὸν ἄνθρωπον. Καὶ εὐθέως ἀλέκτωρ ἐφώνησεν.

\vs{75}Καὶ ἐμνήσθη ὁ Πέτρος τοῦ ῥήματος Ἰησοῦ εἰρηκότος ὅτι Πρὶν ἀλέκτορα φωνῆσαι τρὶς ἀπαρνήσῃ με· καὶ ἐξελθὼν ἔξω ἔκλαυσεν πικρῶς.

\ch{27}
Πρωΐας δὲ γενομένης συμβούλιον ἔλαβον πάντες οἱ ἀρχιερεῖς καὶ οἱ πρεσβύτεροι τοῦ λαοῦ κατὰ τοῦ Ἰησοῦ ὥστε θανατῶσαι αὐτόν·
\vs{2}καὶ δήσαντες αὐτὸν ἀπήγαγον καὶ παρέδωκαν Πιλάτῳ τῷ ἡγεμόνι.

\vs{3}Τότε ἰδὼν Ἰούδας ὁ παραδιδοὺς αὐτὸν ὅτι κατεκρίθη, μεταμεληθεὶς ἔστρεψεν τὰ τριάκοντα ἀργύρια τοῖς ἀρχιερεῦσιν καὶ πρεσβυτέροις
\vs{4}λέγων· Ἥμαρτον παραδοὺς αἷμα ἀθῷον. Οἱ δὲ εἶπαν· Τί πρὸς ἡμᾶς; σὺ ὄψῃ.
\vs{5}Καὶ ῥίψας τὰ ἀργύρια εἰς τὸν ναὸν ἀνεχώρησεν, καὶ ἀπελθὼν ἀπήγξατο.
\vs{6}Οἱ δὲ ἀρχιερεῖς λαβόντες τὰ ἀργύρια εἶπαν· Οὐκ ἔξεστιν βαλεῖν αὐτὰ εἰς τὸν κορβανᾶν, ἐπεὶ τιμὴ αἵματός ἐστιν.
\vs{7}συμβούλιον δὲ λαβόντες ἠγόρασαν ἐξ αὐτῶν τὸν ἀγρὸν τοῦ κεραμέως εἰς ταφὴν τοῖς ξένοις.
\vs{8}διὸ ἐκλήθη ὁ ἀγρὸς ἐκεῖνος Ἀγρὸς αἵματος ἕως τῆς σήμερον.
\vs{9}τότε ἐπληρώθη τὸ ῥηθὲν διὰ Ἰερεμίου τοῦ προφήτου λέγοντος· Καὶ ἔλαβον τὰ τριάκοντα ἀργύρια, τὴν τιμὴν τοῦ τετιμημένου ὃν ἐτιμήσαντο ἀπὸ υἱῶν Ἰσραήλ,
\vs{10}καὶ ἔδωκαν αὐτὰ εἰς τὸν ἀγρὸν τοῦ κεραμέως, καθὰ συνέταξέν μοι Κύριος.

\vs{11}Ὁ δὲ Ἰησοῦς ἐστάθη ἔμπροσθεν τοῦ ἡγεμόνος· καὶ ἐπηρώτησεν αὐτὸν ὁ ἡγεμὼν λέγων· Σὺ εἶ ὁ Βασιλεὺς τῶν Ἰουδαίων; Ὁ δὲ Ἰησοῦς ἔφη· Σὺ λέγεις.
\vs{12}Καὶ ἐν τῷ κατηγορεῖσθαι αὐτὸν ὑπὸ τῶν ἀρχιερέων καὶ πρεσβυτέρων οὐδὲν ἀπεκρίνατο.
\vs{13}Τότε λέγει αὐτῷ ὁ Πιλᾶτος· Οὐκ ἀκούεις πόσα σου καταμαρτυροῦσιν;
\vs{14}Καὶ οὐκ ἀπεκρίθη αὐτῷ πρὸς οὐδὲ ἓν ῥῆμα, ὥστε θαυμάζειν τὸν ἡγεμόνα λίαν.

\vs{15}Κατὰ δὲ ἑορτὴν εἰώθει ὁ ἡγεμὼν ἀπολύειν ἕνα τῷ ὄχλῳ δέσμιον ὃν ἤθελον.
\vs{16}εἶχον δὲ τότε δέσμιον ἐπίσημον λεγόμενον Ἰησοῦν Βαραββᾶν.
\vs{17}συνηγμένων οὖν αὐτῶν εἶπεν αὐτοῖς ὁ Πιλᾶτος· Τίνα θέλετε ἀπολύσω ὑμῖν, Ἰησοῦν τὸν Βαραββᾶν ἢ Ἰησοῦν τὸν λεγόμενον Χριστόν;
\vs{18}ᾔδει γὰρ ὅτι διὰ φθόνον παρέδωκαν αὐτόν.

\vs{19}Καθημένου δὲ αὐτοῦ ἐπὶ τοῦ βήματος ἀπέστειλεν πρὸς αὐτὸν ἡ γυνὴ αὐτοῦ λέγουσα· Μηδὲν σοὶ καὶ τῷ δικαίῳ ἐκείνῳ· πολλὰ γὰρ ἔπαθον σήμερον κατ᾽ ὄναρ δι᾽ αὐτόν.

\vs{20}Οἱ δὲ ἀρχιερεῖς καὶ οἱ πρεσβύτεροι ἔπεισαν τοὺς ὄχλους ἵνα αἰτήσωνται τὸν Βαραββᾶν, τὸν δὲ Ἰησοῦν ἀπολέσωσιν.
\vs{21}Ἀποκριθεὶς δὲ ὁ ἡγεμὼν εἶπεν αὐτοῖς· Τίνα θέλετε ἀπὸ τῶν δύο ἀπολύσω ὑμῖν; Οἱ δὲ εἶπαν· Τὸν Βαραββᾶν.
\vs{22}Λέγει αὐτοῖς ὁ Πιλᾶτος· Τί οὖν ποιήσω Ἰησοῦν τὸν λεγόμενον Χριστόν; Λέγουσιν πάντες· Σταυρωθήτω.
\vs{23}Ὁ δὲ ἔφη· Τί γὰρ κακὸν ἐποίησεν; Οἱ δὲ περισσῶς ἔκραζον λέγοντες· Σταυρωθήτω.

\vs{24}Ἰδὼν δὲ ὁ Πιλᾶτος ὅτι οὐδὲν ὠφελεῖ ἀλλὰ μᾶλλον θόρυβος γίνεται, λαβὼν ὕδωρ ἀπενίψατο τὰς χεῖρας ἀπέναντι τοῦ ὄχλου λέγων· Ἀθῷός εἰμι ἀπὸ τοῦ αἵματος τούτου· ὑμεῖς ὄψεσθε.
\vs{25}Καὶ ἀποκριθεὶς πᾶς ὁ λαὸς εἶπεν· Τὸ αἷμα αὐτοῦ ἐφ᾽ ἡμᾶς καὶ ἐπὶ τὰ τέκνα ἡμῶν.
\vs{26}Τότε ἀπέλυσεν αὐτοῖς τὸν Βαραββᾶν, τὸν δὲ Ἰησοῦν φραγελλώσας παρέδωκεν ἵνα σταυρωθῇ.

\vs{27}Τότε οἱ στρατιῶται τοῦ ἡγεμόνος παραλαβόντες τὸν Ἰησοῦν εἰς τὸ πραιτώριον συνήγαγον ἐπ᾽ αὐτὸν ὅλην τὴν σπεῖραν.
\vs{28}καὶ ἐκδύσαντες αὐτὸν χλαμύδα κοκκίνην περιέθηκαν αὐτῷ,
\vs{29}καὶ πλέξαντες στέφανον ἐξ ἀκανθῶν ἐπέθηκαν ἐπὶ τῆς κεφαλῆς αὐτοῦ καὶ κάλαμον ἐν τῇ δεξιᾷ αὐτοῦ, καὶ γονυπετήσαντες ἔμπροσθεν αὐτοῦ ἐνέπαιξαν αὐτῷ λέγοντες· Χαῖρε, Βασιλεῦ τῶν Ἰουδαίων,
\vs{30}καὶ ἐμπτύσαντες εἰς αὐτὸν ἔλαβον τὸν κάλαμον καὶ ἔτυπτον εἰς τὴν κεφαλὴν αὐτοῦ.
\vs{31}Καὶ ὅτε ἐνέπαιξαν αὐτῷ, ἐξέδυσαν αὐτὸν τὴν χλαμύδα καὶ ἐνέδυσαν αὐτὸν τὰ ἱμάτια αὐτοῦ καὶ ἀπήγαγον αὐτὸν εἰς τὸ σταυρῶσαι.
\vs{32}Ἐξερχόμενοι δὲ εὗρον ἄνθρωπον Κυρηναῖον ὀνόματι Σίμωνα, τοῦτον ἠγγάρευσαν ἵνα ἄρῃ τὸν σταυρὸν αὐτοῦ.

\vs{33}Καὶ ἐλθόντες εἰς τόπον λεγόμενον Γολγοθᾶ, ὅ ἐστιν κρανίου τόπος λεγόμενος,
\vs{34}ἔδωκαν αὐτῷ πιεῖν οἶνον μετὰ χολῆς μεμιγμένον· καὶ γευσάμενος οὐκ ἠθέλησεν πιεῖν.
\vs{35}Σταυρώσαντες δὲ αὐτὸν διεμερίσαντο τὰ ἱμάτια αὐτοῦ βάλλοντες κλῆρον,
\vs{36}καὶ καθήμενοι ἐτήρουν αὐτὸν ἐκεῖ.
\vs{37}καὶ ἐπέθηκαν ἐπάνω τῆς κεφαλῆς αὐτοῦ τὴν αἰτίαν αὐτοῦ γεγραμμένην· 
\begin{poetryblock}

\begin{quote}ΟΥΤΟΣ ΕΣΤΙΝ ΙΗΣΟΥΣ Ο ΒΑΣΙΛΕΥΣ ΤΩΝ ΙΟΥΔΑΙΩΝ.\end{quote}
\end{poetryblock}

\vs{38}Τότε σταυροῦνται σὺν αὐτῷ δύο λῃσταί, εἷς ἐκ δεξιῶν καὶ εἷς ἐξ εὐωνύμων.
\vs{39}Οἱ δὲ παραπορευόμενοι ἐβλασφήμουν αὐτὸν κινοῦντες τὰς κεφαλὰς αὐτῶν
\vs{40}καὶ λέγοντες· Ὁ καταλύων τὸν ναὸν καὶ ἐν τρισὶν ἡμέραις οἰκοδομῶν, σῶσον σεαυτόν, εἰ Υἱὸς εἶ τοῦ Θεοῦ, καὶ κατάβηθι ἀπὸ τοῦ σταυροῦ.
\vs{41}Ὁμοίως καὶ οἱ ἀρχιερεῖς ἐμπαίζοντες μετὰ τῶν γραμματέων καὶ πρεσβυτέρων ἔλεγον·
\vs{42}Ἄλλους ἔσωσεν, ἑαυτὸν οὐ δύναται σῶσαι· Βασιλεὺς Ἰσραήλ ἐστιν, καταβάτω νῦν ἀπὸ τοῦ σταυροῦ καὶ πιστεύσομεν ἐπ᾽ αὐτόν.
\vs{43}πέποιθεν ἐπὶ τὸν Θεόν, ῥυσάσθω νῦν εἰ θέλει αὐτόν· εἶπεν γὰρ ὅτι Θεοῦ εἰμι Υἱός.
\vs{44}Τὸ δ᾽ αὐτὸ καὶ οἱ λῃσταὶ οἱ συσταυρωθέντες σὺν αὐτῷ ὠνείδιζον αὐτόν.

\vs{45}Ἀπὸ δὲ ἕκτης ὥρας σκότος ἐγένετο ἐπὶ πᾶσαν τὴν γῆν ἕως ὥρας ἐνάτης.
\vs{46}περὶ δὲ τὴν ἐνάτην ὥραν ἀνεβόησεν ὁ Ἰησοῦς φωνῇ μεγάλῃ λέγων· 
\begin{poetryblock}

\begin{quote}Ἠλὶ ἠλὶ λεμὰ σαβαχθάνι;\end{quote}
\end{poetryblock}

τοῦτ᾽ ἔστιν· Θεέ μου θεέ μου, ἵνατί με ἐγκατέλιπες;
\vs{47}Τινὲς δὲ τῶν ἐκεῖ ἑστηκότων ἀκούσαντες ἔλεγον ὅτι Ἠλίαν φωνεῖ οὗτος.
\vs{48}καὶ εὐθέως δραμὼν εἷς ἐξ αὐτῶν καὶ λαβὼν σπόγγον πλήσας τε ὄξους καὶ περιθεὶς καλάμῳ ἐπότιζεν αὐτόν.
\vs{49}Οἱ δὲ λοιποὶ ἔλεγον· Ἄφες ἴδωμεν εἰ ἔρχεται Ἠλίας σώσων αὐτόν.
\vs{50}Ὁ δὲ Ἰησοῦς πάλιν κράξας φωνῇ μεγάλῃ ἀφῆκεν τὸ πνεῦμα.

\vs{51}Καὶ ἰδοὺ τὸ καταπέτασμα τοῦ ναοῦ ἐσχίσθη ἀπ᾽ ἄνωθεν ἕως κάτω εἰς δύο καὶ ἡ γῆ ἐσείσθη καὶ αἱ πέτραι ἐσχίσθησαν,
\vs{52}καὶ τὰ μνημεῖα ἀνεῴχθησαν καὶ πολλὰ σώματα τῶν κεκοιμημένων ἁγίων ἠγέρθησαν,
\vs{53}καὶ ἐξελθόντες ἐκ τῶν μνημείων μετὰ τὴν ἔγερσιν αὐτοῦ εἰσῆλθον εἰς τὴν ἁγίαν πόλιν καὶ ἐνεφανίσθησαν πολλοῖς.

\vs{54}Ὁ δὲ ἑκατόνταρχος καὶ οἱ μετ᾽ αὐτοῦ τηροῦντες τὸν Ἰησοῦν ἰδόντες τὸν σεισμὸν καὶ τὰ γενόμενα ἐφοβήθησαν σφόδρα, λέγοντες· Ἀληθῶς Θεοῦ Υἱὸς ἦν οὗτος.

\vs{55}Ἦσαν δὲ ἐκεῖ γυναῖκες πολλαὶ ἀπὸ μακρόθεν θεωροῦσαι, αἵτινες ἠκολούθησαν τῷ Ἰησοῦ ἀπὸ τῆς Γαλιλαίας διακονοῦσαι αὐτῷ·
\vs{56}ἐν αἷς ἦν Μαρία ἡ Μαγδαληνή καὶ Μαρία ἡ τοῦ Ἰακώβου καὶ Ἰωσὴφ μήτηρ καὶ ἡ μήτηρ τῶν υἱῶν Ζεβεδαίου.

\vs{57}Ὀψίας δὲ γενομένης ἦλθεν ἄνθρωπος πλούσιος ἀπὸ Ἁριμαθαίας, τοὔνομα Ἰωσήφ, ὃς καὶ αὐτὸς ἐμαθητεύθη τῷ Ἰησοῦ·
\vs{58}οὗτος προσελθὼν τῷ Πιλάτῳ ᾐτήσατο τὸ σῶμα τοῦ Ἰησοῦ. τότε ὁ Πιλᾶτος ἐκέλευσεν ἀποδοθῆναι.
\vs{59}καὶ λαβὼν τὸ σῶμα ὁ Ἰωσὴφ ἐνετύλιξεν αὐτὸ ἐν σινδόνι καθαρᾷ
\vs{60}καὶ ἔθηκεν αὐτὸ ἐν τῷ καινῷ αὐτοῦ μνημείῳ ὃ ἐλατόμησεν ἐν τῇ πέτρᾳ καὶ προσκυλίσας λίθον μέγαν τῇ θύρᾳ τοῦ μνημείου ἀπῆλθεν.
\vs{61}Ἦν δὲ ἐκεῖ Μαριὰμ ἡ Μαγδαληνὴ καὶ ἡ ἄλλη Μαρία καθήμεναι ἀπέναντι τοῦ τάφου.

\vs{62}Τῇ δὲ ἐπαύριον, ἥτις ἐστὶν μετὰ τὴν Παρασκευήν, συνήχθησαν οἱ ἀρχιερεῖς καὶ οἱ Φαρισαῖοι πρὸς Πιλᾶτον
\vs{63}λέγοντες· Κύριε, ἐμνήσθημεν ὅτι ἐκεῖνος ὁ πλάνος εἶπεν ἔτι ζῶν· Μετὰ τρεῖς ἡμέρας ἐγείρομαι.
\vs{64}κέλευσον οὖν ἀσφαλισθῆναι τὸν τάφον ἕως τῆς τρίτης ἡμέρας, μήποτε ἐλθόντες οἱ μαθηταὶ αὐτοῦ κλέψωσιν αὐτὸν καὶ εἴπωσιν τῷ λαῷ· Ἠγέρθη ἀπὸ τῶν νεκρῶν, καὶ ἔσται ἡ ἐσχάτη πλάνη χείρων τῆς πρώτης.
\vs{65}Ἔφη αὐτοῖς ὁ Πιλᾶτος· Ἔχετε κουστωδίαν· ὑπάγετε ἀσφαλίσασθε ὡς οἴδατε.
\vs{66}οἱ δὲ πορευθέντες ἠσφαλίσαντο τὸν τάφον σφραγίσαντες τὸν λίθον μετὰ τῆς κουστωδίας.

\ch{28}
Ὀψὲ δὲ σαββάτων, τῇ ἐπιφωσκούσῃ εἰς μίαν σαββάτων ἦλθεν Μαριὰμ ἡ Μαγδαληνὴ καὶ ἡ ἄλλη Μαρία θεωρῆσαι τὸν τάφον.
\vs{2}Καὶ ἰδοὺ σεισμὸς ἐγένετο μέγας· ἄγγελος γὰρ Κυρίου καταβὰς ἐξ οὐρανοῦ καὶ προσελθὼν ἀπεκύλισεν τὸν λίθον καὶ ἐκάθητο ἐπάνω αὐτοῦ.
\vs{3}ἦν δὲ ἡ εἰδέα αὐτοῦ ὡς ἀστραπὴ καὶ τὸ ἔνδυμα αὐτοῦ λευκὸν ὡς χιών.
\vs{4}ἀπὸ δὲ τοῦ φόβου αὐτοῦ ἐσείσθησαν οἱ τηροῦντες καὶ ἐγενήθησαν ὡς νεκροί.
\vs{5}Ἀποκριθεὶς δὲ ὁ ἄγγελος εἶπεν ταῖς γυναιξίν· Μὴ φοβεῖσθε ὑμεῖς, οἶδα γὰρ ὅτι Ἰησοῦν τὸν ἐσταυρωμένον ζητεῖτε·
\vs{6}οὐκ ἔστιν ὧδε, ἠγέρθη γὰρ καθὼς εἶπεν· δεῦτε ἴδετε τὸν τόπον ὅπου ἔκειτο.
\vs{7}καὶ ταχὺ πορευθεῖσαι εἴπατε τοῖς μαθηταῖς αὐτοῦ ὅτι Ἠγέρθη ἀπὸ τῶν νεκρῶν, καὶ ἰδοὺ προάγει ὑμᾶς εἰς τὴν Γαλιλαίαν, ἐκεῖ αὐτὸν ὄψεσθε· ἰδοὺ εἶπον ὑμῖν.

\vs{8}Καὶ ἀπελθοῦσαι ταχὺ ἀπὸ τοῦ μνημείου μετὰ φόβου καὶ χαρᾶς μεγάλης ἔδραμον ἀπαγγεῖλαι τοῖς μαθηταῖς αὐτοῦ.
\vs{9}καὶ ἰδοὺ Ἰησοῦς ὑπήντησεν αὐταῖς λέγων· Χαίρετε. αἱ δὲ προσελθοῦσαι ἐκράτησαν αὐτοῦ τοὺς πόδας καὶ προσεκύνησαν αὐτῷ.
\vs{10}τότε λέγει αὐταῖς ὁ Ἰησοῦς· Μὴ φοβεῖσθε· ὑπάγετε ἀπαγγείλατε τοῖς ἀδελφοῖς μου ἵνα ἀπέλθωσιν εἰς τὴν Γαλιλαίαν, κἀκεῖ με ὄψονται.

\vs{11}Πορευομένων δὲ αὐτῶν ἰδού τινες τῆς κουστωδίας ἐλθόντες εἰς τὴν πόλιν ἀπήγγειλαν τοῖς ἀρχιερεῦσιν ἅπαντα τὰ γενόμενα.
\vs{12}καὶ συναχθέντες μετὰ τῶν πρεσβυτέρων συμβούλιόν τε λαβόντες ἀργύρια ἱκανὰ ἔδωκαν τοῖς στρατιώταις
\vs{13}λέγοντες· Εἴπατε ὅτι Οἱ μαθηταὶ αὐτοῦ νυκτὸς ἐλθόντες ἔκλεψαν αὐτὸν ἡμῶν κοιμωμένων.
\vs{14}καὶ ἐὰν ἀκουσθῇ τοῦτο ἐπὶ τοῦ ἡγεμόνος, ἡμεῖς πείσομεν αὐτὸν καὶ ὑμᾶς ἀμερίμνους ποιήσομεν.
\vs{15}Οἱ δὲ λαβόντες τὰ ἀργύρια ἐποίησαν ὡς ἐδιδάχθησαν. Καὶ διεφημίσθη ὁ λόγος οὗτος παρὰ Ἰουδαίοις μέχρι τῆς σήμερον ἡμέρας.

\vs{16}Οἱ δὲ ἕνδεκα μαθηταὶ ἐπορεύθησαν εἰς τὴν Γαλιλαίαν εἰς τὸ ὄρος οὗ ἐτάξατο αὐτοῖς ὁ Ἰησοῦς,
\vs{17}καὶ ἰδόντες αὐτὸν προσεκύνησαν, οἱ δὲ ἐδίστασαν.
\vs{18}Καὶ προσελθὼν ὁ Ἰησοῦς ἐλάλησεν αὐτοῖς λέγων· Ἐδόθη μοι πᾶσα ἐξουσία ἐν οὐρανῷ καὶ ἐπὶ τῆς γῆς.
\vs{19}πορευθέντες οὖν μαθητεύσατε πάντα τὰ ἔθνη, βαπτίζοντες αὐτοὺς εἰς τὸ ὄνομα τοῦ Πατρὸς καὶ τοῦ Υἱοῦ καὶ τοῦ Ἁγίου Πνεύματος,
\vs{20}διδάσκοντες αὐτοὺς τηρεῖν πάντα ὅσα ἐνετειλάμην ὑμῖν· καὶ ἰδοὺ ἐγὼ μεθ᾽ ὑμῶν εἰμι πάσας τὰς ἡμέρας ἕως τῆς συντελείας τοῦ αἰῶνος.


\def\book{ΚΑΤΑ ΜΑΡΚΟΝ}
\biblebook{ΚΑΤΑ ΜΑΡΚΟΝ}


\lettrine[lines=2, loversize=0.2, nindent=0em, findent=.25em]{\textcolor{bookheadingcolor}{Ἀ}}{ρχὴ} τοῦ εὐαγγελίου Ἰησοῦ Χριστοῦ Υἱοῦ Θεοῦ.
\begin{poetryblock}

\begin{quote} \vs{2}Καθὼς γέγραπται ἐν τῷ Ἠσαΐᾳ τῷ προφήτῃ·\end{quote} 

\begin{quote}Ἰδοὺ ἀποστέλλω τὸν ἄγγελόν μου πρὸ προσώπου σου,\end{quote} 

\begin{quote}ὃς κατασκευάσει τὴν ὁδόν σου·\end{quote}

\begin{quote} \vs{3}Φωνὴ βοῶντος ἐν τῇ ἐρήμῳ·\end{quote} 

\begin{quote}Ἑτοιμάσατε τὴν ὁδὸν Κυρίου,\end{quote} 

\begin{quote}εὐθείας ποιεῖτε τὰς τρίβους αὐτοῦ,\end{quote}
\end{poetryblock}

\vs{4}Ἐγένετο Ἰωάννης ὁ βαπτίζων ἐν τῇ ἐρήμῳ καὶ κηρύσσων βάπτισμα μετανοίας εἰς ἄφεσιν ἁμαρτιῶν.
\vs{5}καὶ ἐξεπορεύετο πρὸς αὐτὸν πᾶσα ἡ Ἰουδαία χώρα καὶ οἱ Ἱεροσολυμῖται πάντες, καὶ ἐβαπτίζοντο ὑπ᾽ αὐτοῦ ἐν τῷ Ἰορδάνῃ ποταμῷ ἐξομολογούμενοι τὰς ἁμαρτίας αὐτῶν.
\vs{6}Καὶ ἦν ὁ Ἰωάννης ἐνδεδυμένος τρίχας καμήλου καὶ ζώνην δερματίνην περὶ τὴν ὀσφὺν αὐτοῦ καὶ ἔσθων ἀκρίδας καὶ μέλι ἄγριον.

\vs{7}καὶ ἐκήρυσσεν λέγων· Ἔρχεται ὁ ἰσχυρότερός μου ὀπίσω μου, οὗ οὐκ εἰμὶ ἱκανὸς κύψας λῦσαι τὸν ἱμάντα τῶν ὑποδημάτων αὐτοῦ.
\vs{8}ἐγὼ ἐβάπτισα ὑμᾶς ὕδατι, αὐτὸς δὲ βαπτίσει ὑμᾶς ἐν Πνεύματι Ἁγίῳ.

\vs{9}Καὶ ἐγένετο ἐν ἐκείναις ταῖς ἡμέραις ἦλθεν Ἰησοῦς ἀπὸ Ναζαρὲτ τῆς Γαλιλαίας καὶ ἐβαπτίσθη εἰς τὸν Ἰορδάνην ὑπὸ Ἰωάννου.
\vs{10}καὶ εὐθὺς ἀναβαίνων ἐκ τοῦ ὕδατος εἶδεν σχιζομένους τοὺς οὐρανοὺς καὶ τὸ Πνεῦμα ὡς περιστερὰν καταβαῖνον εἰς αὐτόν·
\vs{11}καὶ φωνὴ ἐγένετο ἐκ τῶν οὐρανῶν· Σὺ εἶ ὁ Υἱός μου ὁ ἀγαπητός, ἐν σοὶ εὐδόκησα.

\vs{12}Καὶ εὐθὺς τὸ Πνεῦμα αὐτὸν ἐκβάλλει εἰς τὴν ἔρημον.
\vs{13}καὶ ἦν ἐν τῇ ἐρήμῳ τεσσεράκοντα ἡμέρας πειραζόμενος ὑπὸ τοῦ Σατανᾶ, καὶ ἦν μετὰ τῶν θηρίων, καὶ οἱ ἄγγελοι διηκόνουν αὐτῷ.

\vs{14}Μετὰ δὲ τὸ παραδοθῆναι τὸν Ἰωάννην ἦλθεν ὁ Ἰησοῦς εἰς τὴν Γαλιλαίαν κηρύσσων τὸ εὐαγγέλιον τοῦ Θεοῦ
\vs{15}καὶ λέγων ὅτι Πεπλήρωται ὁ καιρὸς καὶ ἤγγικεν ἡ βασιλεία τοῦ Θεοῦ· μετανοεῖτε καὶ πιστεύετε ἐν τῷ εὐαγγελίῳ.

\vs{16}Καὶ παράγων παρὰ τὴν θάλασσαν τῆς Γαλιλαίας εἶδεν Σίμωνα καὶ Ἀνδρέαν τὸν ἀδελφὸν Σίμωνος ἀμφιβάλλοντας ἐν τῇ θαλάσσῃ· ἦσαν γὰρ ἁλιεῖς.
\vs{17}καὶ εἶπεν αὐτοῖς ὁ Ἰησοῦς· Δεῦτε ὀπίσω μου, καὶ ποιήσω ὑμᾶς γενέσθαι ἁλιεῖς ἀνθρώπων.
\vs{18}καὶ εὐθὺς ἀφέντες τὰ δίκτυα ἠκολούθησαν αὐτῷ.
\vs{19}Καὶ προβὰς ὀλίγον εἶδεν Ἰάκωβον τὸν τοῦ Ζεβεδαίου καὶ Ἰωάννην τὸν ἀδελφὸν αὐτοῦ καὶ αὐτοὺς ἐν τῷ πλοίῳ καταρτίζοντας τὰ δίκτυα,
\vs{20}καὶ εὐθὺς ἐκάλεσεν αὐτούς. καὶ ἀφέντες τὸν πατέρα αὐτῶν Ζεβεδαῖον ἐν τῷ πλοίῳ μετὰ τῶν μισθωτῶν ἀπῆλθον ὀπίσω αὐτοῦ.

\vs{21}Καὶ εἰσπορεύονται εἰς Καφαρναούμ· καὶ εὐθὺς τοῖς σάββασιν εἰσελθὼν εἰς τὴν συναγωγὴν ἐδίδασκεν.
\vs{22}καὶ ἐξεπλήσσοντο ἐπὶ τῇ διδαχῇ αὐτοῦ· ἦν γὰρ διδάσκων αὐτοὺς ὡς ἐξουσίαν ἔχων καὶ οὐχ ὡς οἱ γραμματεῖς.

\vs{23}Καὶ εὐθὺς ἦν ἐν τῇ συναγωγῇ αὐτῶν ἄνθρωπος ἐν πνεύματι ἀκαθάρτῳ καὶ ἀνέκραξεν
\vs{24}λέγων· Τί ἡμῖν καὶ σοί, Ἰησοῦ Ναζαρηνέ; ἦλθες ἀπολέσαι ἡμᾶς; οἶδά σε τίς εἶ, ὁ Ἅγιος τοῦ Θεοῦ.
\vs{25}Καὶ ἐπετίμησεν αὐτῷ ὁ Ἰησοῦς λέγων· Φιμώθητι καὶ ἔξελθε ἐξ αὐτοῦ.
\vs{26}καὶ σπαράξαν αὐτὸν τὸ πνεῦμα τὸ ἀκάθαρτον καὶ φωνῆσαν φωνῇ μεγάλῃ ἐξῆλθεν ἐξ αὐτοῦ.
\vs{27}Καὶ ἐθαμβήθησαν ἅπαντες ὥστε συζητεῖν πρὸς ἑαυτοὺς λέγοντας· Τί ἐστιν τοῦτο; διδαχὴ καινή κατ᾽ ἐξουσίαν· καὶ τοῖς πνεύμασι τοῖς ἀκαθάρτοις ἐπιτάσσει, καὶ ὑπακούουσιν αὐτῷ.
\vs{28}καὶ ἐξῆλθεν ἡ ἀκοὴ αὐτοῦ εὐθὺς πανταχοῦ εἰς ὅλην τὴν περίχωρον τῆς Γαλιλαίας.

\vs{29}Καὶ εὐθὺς ἐκ τῆς συναγωγῆς ἐξελθόντες ἦλθον εἰς τὴν οἰκίαν Σίμωνος καὶ Ἀνδρέου μετὰ Ἰακώβου καὶ Ἰωάννου.
\vs{30}ἡ δὲ πενθερὰ Σίμωνος κατέκειτο πυρέσσουσα, καὶ εὐθὺς λέγουσιν αὐτῷ περὶ αὐτῆς.
\vs{31}καὶ προσελθὼν ἤγειρεν αὐτὴν κρατήσας τῆς χειρός· καὶ ἀφῆκεν αὐτὴν ὁ πυρετός, καὶ διηκόνει αὐτοῖς.

\vs{32}Ὀψίας δὲ γενομένης, ὅτε ἔδυ ὁ ἥλιος, ἔφερον πρὸς αὐτὸν πάντας τοὺς κακῶς ἔχοντας καὶ τοὺς δαιμονιζομένους·
\vs{33}καὶ ἦν ὅλη ἡ πόλις ἐπισυνηγμένη πρὸς τὴν θύραν.
\vs{34}καὶ ἐθεράπευσεν πολλοὺς κακῶς ἔχοντας ποικίλαις νόσοις καὶ δαιμόνια πολλὰ ἐξέβαλεν καὶ οὐκ ἤφιεν λαλεῖν τὰ δαιμόνια, ὅτι ᾔδεισαν αὐτόν.

\vs{35}Καὶ πρωῒ ἔννυχα λίαν ἀναστὰς ἐξῆλθεν καὶ ἀπῆλθεν εἰς ἔρημον τόπον κἀκεῖ προσηύχετο.
\vs{36}καὶ κατεδίωξεν αὐτὸν Σίμων καὶ οἱ μετ᾽ αὐτοῦ,
\vs{37}καὶ εὗρον αὐτὸν καὶ λέγουσιν αὐτῷ ὅτι Πάντες ζητοῦσίν σε.
\vs{38}Καὶ λέγει αὐτοῖς· Ἄγωμεν ἀλλαχοῦ εἰς τὰς ἐχομένας κωμοπόλεις, ἵνα καὶ ἐκεῖ κηρύξω· εἰς τοῦτο γὰρ ἐξῆλθον.

\vs{39}καὶ ἦλθεν κηρύσσων εἰς τὰς συναγωγὰς αὐτῶν εἰς ὅλην τὴν Γαλιλαίαν καὶ τὰ δαιμόνια ἐκβάλλων.
\vs{40}Καὶ ἔρχεται πρὸς αὐτὸν λεπρὸς παρακαλῶν αὐτὸν καὶ γονυπετῶν καὶ λέγων αὐτῷ ὅτι Ἐὰν θέλῃς δύνασαί με καθαρίσαι.
\vs{41}Καὶ σπλαγχνισθεὶς ἐκτείνας τὴν χεῖρα αὐτοῦ ἥψατο καὶ λέγει αὐτῷ· Θέλω, καθαρίσθητι·
\vs{42}Καὶ εὐθὺς ἀπῆλθεν ἀπ᾽ αὐτοῦ ἡ λέπρα, καὶ ἐκαθαρίσθη.
\vs{43}Καὶ ἐμβριμησάμενος αὐτῷ εὐθὺς ἐξέβαλεν αὐτόν
\vs{44}καὶ λέγει αὐτῷ· Ὅρα μηδενὶ μηδὲν εἴπῃς, ἀλλὰ ὕπαγε σεαυτὸν δεῖξον τῷ ἱερεῖ καὶ προσένεγκε περὶ τοῦ καθαρισμοῦ σου ἃ προσέταξεν Μωϋσῆς, εἰς μαρτύριον αὐτοῖς.
\vs{45}Ὁ δὲ ἐξελθὼν ἤρξατο κηρύσσειν πολλὰ καὶ διαφημίζειν τὸν λόγον, ὥστε μηκέτι αὐτὸν δύνασθαι φανερῶς εἰς πόλιν εἰσελθεῖν, ἀλλ᾽ ἔξω ἐπ᾽ ἐρήμοις τόποις ἦν· καὶ ἤρχοντο πρὸς αὐτὸν πάντοθεν.

\ch{2}
Καὶ εἰσελθὼν πάλιν εἰς Καφαρναοὺμ δι᾽ ἡμερῶν ἠκούσθη ὅτι ἐν οἴκῳ ἐστίν.
\vs{2}καὶ συνήχθησαν πολλοὶ ὥστε μηκέτι χωρεῖν μηδὲ τὰ πρὸς τὴν θύραν, καὶ ἐλάλει αὐτοῖς τὸν λόγον.
\vs{3}Καὶ ἔρχονται φέροντες πρὸς αὐτὸν παραλυτικὸν αἰρόμενον ὑπὸ τεσσάρων.
\vs{4}καὶ μὴ δυνάμενοι προσενέγκαι αὐτῷ διὰ τὸν ὄχλον ἀπεστέγασαν τὴν στέγην ὅπου ἦν, καὶ ἐξορύξαντες χαλῶσι τὸν κράβαττον ὅπου ὁ παραλυτικὸς κατέκειτο.
\vs{5}Καὶ ἰδὼν ὁ Ἰησοῦς τὴν πίστιν αὐτῶν λέγει τῷ παραλυτικῷ· Τέκνον, ἀφίενταί σου αἱ ἁμαρτίαι.
\vs{6}Ἦσαν δέ τινες τῶν γραμματέων ἐκεῖ καθήμενοι καὶ διαλογιζόμενοι ἐν ταῖς καρδίαις αὐτῶν·
\vs{7}Τί οὗτος οὕτως λαλεῖ; βλασφημεῖ· τίς δύναται ἀφιέναι ἁμαρτίας εἰ μὴ εἷς ὁ Θεός;
\vs{8}Καὶ εὐθὺς ἐπιγνοὺς ὁ Ἰησοῦς τῷ πνεύματι αὐτοῦ ὅτι οὕτως διαλογίζονται ἐν ἑαυτοῖς λέγει αὐτοῖς· Τί ταῦτα διαλογίζεσθε ἐν ταῖς καρδίαις ὑμῶν;
\vs{9}τί ἐστιν εὐκοπώτερον, εἰπεῖν τῷ παραλυτικῷ· Ἀφίενταί σου αἱ ἁμαρτίαι, ἢ εἰπεῖν· Ἔγειρε καὶ ἆρον τὸν κράβαττόν σου καὶ περιπάτει;
\vs{10}ἵνα δὲ εἰδῆτε ὅτι ἐξουσίαν ἔχει ὁ Υἱὸς τοῦ ἀνθρώπου ἀφιέναι ἁμαρτίας ἐπὶ τῆς γῆς— λέγει τῷ παραλυτικῷ·
\vs{11}Σοὶ λέγω, ἔγειρε ἆρον τὸν κράβαττόν σου καὶ ὕπαγε εἰς τὸν οἶκόν σου.
\vs{12}Καὶ ἠγέρθη καὶ εὐθὺς ἄρας τὸν κράβαττον ἐξῆλθεν ἔμπροσθεν πάντων, ὥστε ἐξίστασθαι πάντας καὶ δοξάζειν τὸν Θεὸν λέγοντας ὅτι Οὕτως οὐδέποτε εἴδομεν.

\vs{13}Καὶ ἐξῆλθεν πάλιν παρὰ τὴν θάλασσαν· καὶ πᾶς ὁ ὄχλος ἤρχετο πρὸς αὐτόν, καὶ ἐδίδασκεν αὐτούς.
\vs{14}Καὶ παράγων εἶδεν Λευὶν τὸν τοῦ Ἁλφαίου καθήμενον ἐπὶ τὸ τελώνιον, καὶ λέγει αὐτῷ· Ἀκολούθει μοι. καὶ ἀναστὰς ἠκολούθησεν αὐτῷ.

\vs{15}Καὶ γίνεται κατακεῖσθαι αὐτὸν ἐν τῇ οἰκίᾳ αὐτοῦ, καὶ πολλοὶ τελῶναι καὶ ἁμαρτωλοὶ συνανέκειντο τῷ Ἰησοῦ καὶ τοῖς μαθηταῖς αὐτοῦ· ἦσαν γὰρ πολλοὶ καὶ ἠκολούθουν αὐτῷ.
\vs{16}καὶ οἱ γραμματεῖς τῶν Φαρισαίων ἰδόντες ὅτι ἐσθίει μετὰ τῶν ἁμαρτωλῶν καὶ τελωνῶν ἔλεγον τοῖς μαθηταῖς αὐτοῦ· Ὅτι μετὰ τῶν τελωνῶν καὶ ἁμαρτωλῶν ἐσθίει;
\vs{17}Καὶ ἀκούσας ὁ Ἰησοῦς λέγει αὐτοῖς ὅτι Οὐ χρείαν ἔχουσιν οἱ ἰσχύοντες ἰατροῦ ἀλλ᾽ οἱ κακῶς ἔχοντες· οὐκ ἦλθον καλέσαι δικαίους ἀλλὰ ἁμαρτωλούς.

\vs{18}Καὶ ἦσαν οἱ μαθηταὶ Ἰωάννου καὶ οἱ Φαρισαῖοι νηστεύοντες. καὶ ἔρχονται καὶ λέγουσιν αὐτῷ· Διὰ τί οἱ μαθηταὶ Ἰωάννου καὶ οἱ μαθηταὶ τῶν Φαρισαίων νηστεύουσιν, οἱ δὲ σοὶ μαθηταὶ οὐ νηστεύουσιν;
\vs{19}Καὶ εἶπεν αὐτοῖς ὁ Ἰησοῦς· Μὴ δύνανται οἱ υἱοὶ τοῦ νυμφῶνος ἐν ᾧ ὁ νυμφίος μετ᾽ αὐτῶν ἐστιν νηστεύειν; ὅσον χρόνον ἔχουσιν τὸν νυμφίον μετ᾽ αὐτῶν οὐ δύνανται νηστεύειν.
\vs{20}ἐλεύσονται δὲ ἡμέραι ὅταν ἀπαρθῇ ἀπ᾽ αὐτῶν ὁ νυμφίος, καὶ τότε νηστεύσουσιν ἐν ἐκείνῃ τῇ ἡμέρᾳ.
\vs{21}Οὐδεὶς ἐπίβλημα ῥάκους ἀγνάφου ἐπιράπτει ἐπὶ ἱμάτιον παλαιόν· εἰ δὲ μή, αἴρει τὸ πλήρωμα ἀπ᾽ αὐτοῦ τὸ καινὸν τοῦ παλαιοῦ καὶ χεῖρον σχίσμα γίνεται.
\vs{22}Καὶ οὐδεὶς βάλλει οἶνον νέον εἰς ἀσκοὺς παλαιούς· εἰ δὲ μή, ῥήξει ὁ οἶνος τοὺς ἀσκούς καὶ ὁ οἶνος ἀπόλλυται καὶ οἱ ἀσκοί· ἀλλὰ οἶνον νέον εἰς ἀσκοὺς καινούς.
\vs{23}Καὶ ἐγένετο αὐτὸν ἐν τοῖς σάββασιν παραπορεύεσθαι διὰ τῶν σπορίμων, καὶ οἱ μαθηταὶ αὐτοῦ ἤρξαντο ὁδὸν ποιεῖν τίλλοντες τοὺς στάχυας.
\vs{24}καὶ οἱ Φαρισαῖοι ἔλεγον αὐτῷ· Ἴδε τί ποιοῦσιν τοῖς σάββασιν ὃ οὐκ ἔξεστιν;
\vs{25}Καὶ λέγει αὐτοῖς· Οὐδέποτε ἀνέγνωτε τί ἐποίησεν Δαυὶδ ὅτε χρείαν ἔσχεν καὶ ἐπείνασεν αὐτὸς καὶ οἱ μετ᾽ αὐτοῦ,
\vs{26}πῶς εἰσῆλθεν εἰς τὸν οἶκον τοῦ Θεοῦ ἐπὶ Ἀβιαθὰρ ἀρχιερέως καὶ τοὺς ἄρτους τῆς προθέσεως ἔφαγεν, οὓς οὐκ ἔξεστιν φαγεῖν εἰ μὴ τοὺς ἱερεῖς, καὶ ἔδωκεν καὶ τοῖς σὺν αὐτῷ οὖσιν;
\vs{27}Καὶ ἔλεγεν αὐτοῖς· Τὸ σάββατον διὰ τὸν ἄνθρωπον ἐγένετο καὶ οὐχ ὁ ἄνθρωπος διὰ τὸ σάββατον·
\vs{28}ὥστε κύριός ἐστιν ὁ Υἱὸς τοῦ ἀνθρώπου καὶ τοῦ σαββάτου.

\ch{3}
Καὶ εἰσῆλθεν πάλιν εἰς τὴν συναγωγήν. καὶ ἦν ἐκεῖ ἄνθρωπος ἐξηραμμένην ἔχων τὴν χεῖρα.
\vs{2}καὶ παρετήρουν αὐτὸν εἰ τοῖς σάββασιν θεραπεύσει αὐτόν, ἵνα κατηγορήσωσιν αὐτοῦ.
\vs{3}Καὶ λέγει τῷ ἀνθρώπῳ τῷ τὴν χεῖρα2 ἔχοντι3 ξηράν·1 Ἔγειρε εἰς τὸ μέσον.
\vs{4}καὶ λέγει αὐτοῖς· Ἔξεστιν τοῖς σάββασιν ἀγαθὸν ποιῆσαι ἢ κακοποιῆσαι, ψυχὴν σῶσαι ἢ ἀποκτεῖναι; Οἱ δὲ ἐσιώπων.
\vs{5}Καὶ περιβλεψάμενος αὐτοὺς μετ᾽ ὀργῆς, συλλυπούμενος ἐπὶ τῇ πωρώσει τῆς καρδίας αὐτῶν λέγει τῷ ἀνθρώπῳ· Ἔκτεινον τὴν χεῖρα. καὶ ἐξέτεινεν καὶ ἀπεκατεστάθη ἡ χεὶρ αὐτοῦ.
\vs{6}Καὶ ἐξελθόντες οἱ Φαρισαῖοι εὐθὺς μετὰ τῶν Ἡρῳδιανῶν συμβούλιον ἐδίδουν κατ᾽ αὐτοῦ ὅπως αὐτὸν ἀπολέσωσιν.

\vs{7}Καὶ ὁ Ἰησοῦς μετὰ τῶν μαθητῶν αὐτοῦ ἀνεχώρησεν πρὸς τὴν θάλασσαν, καὶ πολὺ πλῆθος ἀπὸ τῆς Γαλιλαίας ἠκολούθησεν, καὶ ἀπὸ τῆς Ἰουδαίας
\vs{8}καὶ ἀπὸ Ἱεροσολύμων καὶ ἀπὸ τῆς Ἰδουμαίας καὶ πέραν τοῦ Ἰορδάνου καὶ περὶ Τύρον καὶ Σιδῶνα πλῆθος πολύ ἀκούοντες ὅσα ἐποίει ἦλθον πρὸς αὐτόν.
\vs{9}Καὶ εἶπεν τοῖς μαθηταῖς αὐτοῦ ἵνα πλοιάριον προσκαρτερῇ αὐτῷ διὰ τὸν ὄχλον ἵνα μὴ θλίβωσιν αὐτόν·
\vs{10}πολλοὺς γὰρ ἐθεράπευσεν, ὥστε ἐπιπίπτειν αὐτῷ ἵνα αὐτοῦ ἅψωνται ὅσοι εἶχον μάστιγας.
\vs{11}καὶ τὰ πνεύματα τὰ ἀκάθαρτα, ὅταν αὐτὸν ἐθεώρουν, προσέπιπτον αὐτῷ καὶ ἔκραζον λέγοντα ὅτι Σὺ εἶ ὁ Υἱὸς τοῦ Θεοῦ.
\vs{12}καὶ πολλὰ ἐπετίμα αὐτοῖς ἵνα μὴ αὐτὸν φανερὸν ποιήσωσιν.

\vs{13}Καὶ ἀναβαίνει εἰς τὸ ὄρος καὶ προσκαλεῖται οὓς ἤθελεν αὐτός, καὶ ἀπῆλθον πρὸς αὐτόν.
\vs{14}καὶ ἐποίησεν δώδεκα οὓς καὶ ἀποστόλους ὠνόμασεν ἵνα ὦσιν μετ᾽ αὐτοῦ καὶ ἵνα ἀποστέλλῃ αὐτοὺς κηρύσσειν
\vs{15}καὶ ἔχειν ἐξουσίαν ἐκβάλλειν τὰ δαιμόνια·
\vs{16}Καὶ ἐποίησεν τοὺς δώδεκα, καὶ ἐπέθηκεν ὄνομα τῷ Σίμωνι Πέτρον,
\vs{17}καὶ Ἰάκωβον τὸν τοῦ Ζεβεδαίου καὶ Ἰωάννην τὸν ἀδελφὸν τοῦ Ἰακώβου καὶ ἐπέθηκεν αὐτοῖς ὀνόματα Βοανηργές, ὅ ἐστιν Υἱοὶ Βροντῆς·
\vs{18}καὶ Ἀνδρέαν καὶ Φίλιππον καὶ Βαρθολομαῖον καὶ Μαθθαῖον καὶ Θωμᾶν καὶ Ἰάκωβον τὸν τοῦ Ἁλφαίου καὶ Θαδδαῖον καὶ Σίμωνα τὸν Καναναῖον
\vs{19}καὶ Ἰούδαν Ἰσκαριώθ, ὃς καὶ παρέδωκεν αὐτόν.

\vs{20}Καὶ ἔρχεται εἰς οἶκον· καὶ συνέρχεται πάλιν ὁ ὄχλος, ὥστε μὴ δύνασθαι αὐτοὺς μηδὲ ἄρτον φαγεῖν.
\vs{21}καὶ ἀκούσαντες οἱ παρ᾽ αὐτοῦ ἐξῆλθον κρατῆσαι αὐτόν· ἔλεγον γὰρ ὅτι Ἐξέστη.

\vs{22}Καὶ οἱ γραμματεῖς οἱ ἀπὸ Ἱεροσολύμων καταβάντες ἔλεγον ὅτι Βεελζεβοὺλ ἔχει καὶ ὅτι Ἐν τῷ ἄρχοντι τῶν δαιμονίων ἐκβάλλει τὰ δαιμόνια.

\vs{23}Καὶ προσκαλεσάμενος αὐτοὺς ἐν παραβολαῖς ἔλεγεν αὐτοῖς· Πῶς δύναται Σατανᾶς Σατανᾶν ἐκβάλλειν;
\vs{24}καὶ ἐὰν βασιλεία ἐφ᾽ ἑαυτὴν μερισθῇ, οὐ δύναται σταθῆναι ἡ βασιλεία ἐκείνη·
\vs{25}καὶ ἐὰν οἰκία ἐφ᾽ ἑαυτὴν μερισθῇ, οὐ δυνήσεται ἡ οἰκία ἐκείνη σταθῆναι.
\vs{26}καὶ εἰ ὁ Σατανᾶς ἀνέστη ἐφ᾽ ἑαυτὸν καὶ ἐμερίσθη, οὐ δύναται στῆναι ἀλλὰ τέλος ἔχει.
\vs{27}ἀλλ᾽ οὐ δύναται οὐδεὶς εἰς τὴν οἰκίαν τοῦ ἰσχυροῦ εἰσελθὼν τὰ σκεύη αὐτοῦ διαρπάσαι, ἐὰν μὴ πρῶτον τὸν ἰσχυρὸν δήσῃ, καὶ τότε τὴν οἰκίαν αὐτοῦ διαρπάσει.

\vs{28}Ἀμὴν λέγω ὑμῖν ὅτι πάντα ἀφεθήσεται τοῖς υἱοῖς τῶν ἀνθρώπων τὰ ἁμαρτήματα καὶ αἱ βλασφημίαι ὅσα ἐὰν βλασφημήσωσιν·
\vs{29}ὃς δ᾽ ἂν βλασφημήσῃ εἰς τὸ Πνεῦμα τὸ Ἅγιον, οὐκ ἔχει ἄφεσιν εἰς τὸν αἰῶνα, ἀλλὰ ἔνοχός ἐστιν αἰωνίου ἁμαρτήματος.
\vs{30}Ὅτι ἔλεγον· Πνεῦμα ἀκάθαρτον ἔχει.

\vs{31}Καὶ ἔρχονται ἡ μήτηρ αὐτοῦ καὶ οἱ ἀδελφοὶ αὐτοῦ καὶ ἔξω στήκοντες ἀπέστειλαν πρὸς αὐτὸν καλοῦντες αὐτόν.
\vs{32}καὶ ἐκάθητο περὶ αὐτὸν ὄχλος, καὶ λέγουσιν αὐτῷ· Ἰδοὺ ἡ μήτηρ σου καὶ οἱ ἀδελφοί σου καὶ αἱ ἀδελφαί σου ἔξω ζητοῦσίν σε.
\vs{33}Καὶ ἀποκριθεὶς αὐτοῖς λέγει· Τίς ἐστιν ἡ μήτηρ μου καὶ οἱ ἀδελφοί μου;
\vs{34}καὶ περιβλεψάμενος τοὺς περὶ αὐτὸν κύκλῳ καθημένους λέγει· Ἴδε ἡ μήτηρ μου καὶ οἱ ἀδελφοί μου.
\vs{35}ὃς γὰρ ἂν ποιήσῃ τὸ θέλημα τοῦ Θεοῦ, οὗτος ἀδελφός μου καὶ ἀδελφὴ καὶ μήτηρ ἐστίν.

\ch{4}
Καὶ πάλιν ἤρξατο διδάσκειν παρὰ τὴν θάλασσαν· καὶ συνάγεται πρὸς αὐτὸν ὄχλος πλεῖστος, ὥστε αὐτὸν εἰς πλοῖον ἐμβάντα καθῆσθαι ἐν τῇ θαλάσσῃ, καὶ πᾶς ὁ ὄχλος πρὸς τὴν θάλασσαν ἐπὶ τῆς γῆς ἦσαν.
\vs{2}Καὶ ἐδίδασκεν αὐτοὺς ἐν παραβολαῖς πολλά καὶ ἔλεγεν αὐτοῖς ἐν τῇ διδαχῇ αὐτοῦ·

\vs{3}Ἀκούετε. ἰδοὺ ἐξῆλθεν ὁ σπείρων σπεῖραι.
\vs{4}καὶ ἐγένετο ἐν τῷ σπείρειν ὃ μὲν ἔπεσεν παρὰ τὴν ὁδόν, καὶ ἦλθεν τὰ πετεινὰ καὶ κατέφαγεν αὐτό.
\vs{5}Καὶ ἄλλο ἔπεσεν ἐπὶ τὸ πετρῶδες ὅπου οὐκ εἶχεν γῆν πολλήν, καὶ εὐθὺς ἐξανέτειλεν διὰ τὸ μὴ ἔχειν βάθος γῆς·
\vs{6}καὶ ὅτε ἀνέτειλεν ὁ ἥλιος ἐκαυματίσθη καὶ διὰ τὸ μὴ ἔχειν ῥίζαν ἐξηράνθη.
\vs{7}Καὶ ἄλλο ἔπεσεν εἰς τὰς ἀκάνθας, καὶ ἀνέβησαν αἱ ἄκανθαι καὶ συνέπνιξαν αὐτό, καὶ καρπὸν οὐκ ἔδωκεν.
\vs{8}Καὶ ἄλλα ἔπεσεν εἰς τὴν γῆν τὴν καλήν καὶ ἐδίδου καρπὸν ἀναβαίνοντα καὶ αὐξανόμενα καὶ ἔφερεν ἓν τριάκοντα καὶ ἓν ἑξήκοντα καὶ ἓν ἑκατόν.
\vs{9}Καὶ ἔλεγεν· Ὃς ἔχει ὦτα ἀκούειν ἀκουέτω.
\vs{10}Καὶ ὅτε ἐγένετο κατὰ μόνας, ἠρώτων αὐτὸν οἱ περὶ αὐτὸν σὺν τοῖς δώδεκα τὰς παραβολάς.
\vs{11}Καὶ ἔλεγεν αὐτοῖς· Ὑμῖν τὸ μυστήριον δέδοται τῆς βασιλείας τοῦ Θεοῦ· ἐκείνοις δὲ τοῖς ἔξω ἐν παραβολαῖς τὰ πάντα γίνεται,
\begin{poetryblock}

\begin{quote} \vs{12}ἵνα Βλέποντες βλέπωσιν καὶ μὴ ἴδωσιν,\end{quote} 

\begin{quote}Καὶ ἀκούοντες ἀκούωσιν καὶ μὴ συνιῶσιν,\end{quote} 

\begin{quote}Μήποτε ἐπιστρέψωσιν Καὶ ἀφεθῇ αὐτοῖς.\end{quote}
\end{poetryblock}
\vs{13}Καὶ λέγει αὐτοῖς· Οὐκ οἴδατε τὴν παραβολὴν ταύτην, καὶ πῶς πάσας τὰς παραβολὰς γνώσεσθε;
\vs{14}Ὁ σπείρων τὸν λόγον σπείρει.
\vs{15}οὗτοι δέ εἰσιν οἱ παρὰ τὴν ὁδὸν· ὅπου σπείρεται ὁ λόγος καὶ ὅταν ἀκούσωσιν, εὐθὺς ἔρχεται ὁ Σατανᾶς καὶ αἴρει τὸν λόγον τὸν ἐσπαρμένον εἰς αὐτούς.
\vs{16}Καὶ οὗτοί εἰσιν οἱ ἐπὶ τὰ πετρώδη σπειρόμενοι, οἳ ὅταν ἀκούσωσιν τὸν λόγον εὐθὺς μετὰ χαρᾶς λαμβάνουσιν αὐτόν,
\vs{17}καὶ οὐκ ἔχουσιν ῥίζαν ἐν ἑαυτοῖς ἀλλὰ πρόσκαιροί εἰσιν, εἶτα γενομένης θλίψεως ἢ διωγμοῦ διὰ τὸν λόγον εὐθὺς σκανδαλίζονται.
\vs{18}Καὶ ἄλλοι εἰσὶν οἱ εἰς τὰς ἀκάνθας σπειρόμενοι· οὗτοί εἰσιν οἱ τὸν λόγον ἀκούσαντες,
\vs{19}καὶ αἱ μέριμναι τοῦ αἰῶνος καὶ ἡ ἀπάτη τοῦ πλούτου καὶ αἱ περὶ τὰ λοιπὰ ἐπιθυμίαι εἰσπορευόμεναι συμπνίγουσιν τὸν λόγον καὶ ἄκαρπος γίνεται.
\vs{20}Καὶ ἐκεῖνοί εἰσιν οἱ ἐπὶ τὴν γῆν τὴν καλὴν σπαρέντες, οἵτινες ἀκούουσιν τὸν λόγον καὶ παραδέχονται καὶ καρποφοροῦσιν ἓν τριάκοντα καὶ ἓν ἑξήκοντα καὶ ἓν ἑκατόν.

\vs{21}Καὶ ἔλεγεν αὐτοῖς· Μήτι ἔρχεται ὁ λύχνος ἵνα ὑπὸ τὸν μόδιον τεθῇ ἢ ὑπὸ τὴν κλίνην; οὐχ ἵνα ἐπὶ τὴν λυχνίαν τεθῇ;
\vs{22}οὐ γάρ ἐστιν κρυπτὸν ἐὰν μὴ ἵνα φανερωθῇ, οὐδὲ ἐγένετο ἀπόκρυφον ἀλλ᾽ ἵνα ἔλθῃ εἰς φανερόν.
\vs{23}Εἴ τις ἔχει ὦτα ἀκούειν ἀκουέτω.
\vs{24}Καὶ ἔλεγεν αὐτοῖς· Βλέπετε τί ἀκούετε. ἐν ᾧ μέτρῳ μετρεῖτε μετρηθήσεται ὑμῖν καὶ προστεθήσεται ὑμῖν.
\vs{25}ὃς γὰρ ἔχει, δοθήσεται αὐτῷ· καὶ ὃς οὐκ ἔχει, καὶ ὃ ἔχει ἀρθήσεται ἀπ᾽ αὐτοῦ.

\vs{26}Καὶ ἔλεγεν· Οὕτως ἐστὶν ἡ βασιλεία τοῦ Θεοῦ ὡς ἄνθρωπος βάλῃ τὸν σπόρον ἐπὶ τῆς γῆς
\vs{27}καὶ καθεύδῃ καὶ ἐγείρηται νύκτα καὶ ἡμέραν, καὶ ὁ σπόρος βλαστᾷ καὶ μηκύνηται ὡς οὐκ οἶδεν αὐτός.
\vs{28}αὐτομάτη ἡ γῆ καρποφορεῖ, πρῶτον χόρτον εἶτα στάχυν εἶτα πλήρης σῖτον ἐν τῷ στάχυϊ.
\vs{29}ὅταν δὲ παραδοῖ ὁ καρπός, εὐθὺς ἀποστέλλει τὸ δρέπανον, ὅτι παρέστηκεν ὁ θερισμός.

\vs{30}Καὶ ἔλεγεν· Πῶς ὁμοιώσωμεν τὴν βασιλείαν τοῦ Θεοῦ ἢ ἐν τίνι αὐτὴν παραβολῇ θῶμεν;
\vs{31}ὡς κόκκῳ σινάπεως, ὃς ὅταν σπαρῇ ἐπὶ τῆς γῆς, μικρότερον ὂν πάντων τῶν σπερμάτων τῶν ἐπὶ τῆς γῆς,
\vs{32}καὶ ὅταν σπαρῇ, ἀναβαίνει καὶ γίνεται μεῖζον πάντων τῶν λαχάνων καὶ ποιεῖ κλάδους μεγάλους, ὥστε δύνασθαι ὑπὸ τὴν σκιὰν αὐτοῦ τὰ πετεινὰ τοῦ οὐρανοῦ κατασκηνοῦν.

\vs{33}Καὶ τοιαύταις παραβολαῖς πολλαῖς ἐλάλει αὐτοῖς τὸν λόγον καθὼς ἠδύναντο ἀκούειν·
\vs{34}χωρὶς δὲ παραβολῆς οὐκ ἐλάλει αὐτοῖς, κατ᾽ ἰδίαν δὲ τοῖς ἰδίοις μαθηταῖς ἐπέλυεν πάντα.

\vs{35}Καὶ λέγει αὐτοῖς ἐν ἐκείνῃ τῇ ἡμέρᾳ ὀψίας γενομένης· Διέλθωμεν εἰς τὸ πέραν.
\vs{36}καὶ ἀφέντες τὸν ὄχλον παραλαμβάνουσιν αὐτὸν ὡς ἦν ἐν τῷ πλοίῳ, καὶ ἄλλα πλοῖα ἦν μετ᾽ αὐτοῦ.
\vs{37}Καὶ γίνεται λαῖλαψ μεγάλη ἀνέμου καὶ τὰ κύματα ἐπέβαλλεν εἰς τὸ πλοῖον, ὥστε ἤδη γεμίζεσθαι τὸ πλοῖον.
\vs{38}καὶ αὐτὸς ἦν ἐν τῇ πρύμνῃ ἐπὶ τὸ προσκεφάλαιον καθεύδων. καὶ ἐγείρουσιν αὐτὸν καὶ λέγουσιν αὐτῷ· Διδάσκαλε, οὐ μέλει σοι ὅτι ἀπολλύμεθα;
\vs{39}Καὶ διεγερθεὶς ἐπετίμησεν τῷ ἀνέμῳ καὶ εἶπεν τῇ θαλάσσῃ· Σιώπα, πεφίμωσο. καὶ ἐκόπασεν ὁ ἄνεμος καὶ ἐγένετο γαλήνη μεγάλη.
\vs{40}Καὶ εἶπεν αὐτοῖς· Τί δειλοί ἐστε; οὔπω ἔχετε πίστιν;
\vs{41}Καὶ ἐφοβήθησαν φόβον μέγαν καὶ ἔλεγον πρὸς ἀλλήλους· Τίς ἄρα οὗτός ἐστιν ὅτι καὶ ὁ ἄνεμος καὶ ἡ θάλασσα ὑπακούει αὐτῷ;

\ch{5}
Καὶ ἦλθον εἰς τὸ πέραν τῆς θαλάσσης εἰς τὴν χώραν τῶν Γερασηνῶν.
\vs{2}καὶ ἐξελθόντος αὐτοῦ ἐκ τοῦ πλοίου εὐθὺς ὑπήντησεν αὐτῷ ἐκ τῶν μνημείων ἄνθρωπος ἐν πνεύματι ἀκαθάρτῳ,
\vs{3}ὃς τὴν κατοίκησιν εἶχεν ἐν τοῖς μνήμασιν, καὶ οὐδὲ ἁλύσει οὐκέτι οὐδεὶς ἐδύνατο αὐτὸν δῆσαι
\vs{4}διὰ τὸ αὐτὸν πολλάκις πέδαις καὶ ἁλύσεσιν δεδέσθαι καὶ διεσπάσθαι ὑπ᾽ αὐτοῦ τὰς ἁλύσεις καὶ τὰς πέδας συντετρῖφθαι, καὶ οὐδεὶς ἴσχυεν αὐτὸν δαμάσαι·
\vs{5}καὶ διὰ παντὸς νυκτὸς καὶ ἡμέρας ἐν τοῖς μνήμασιν καὶ ἐν τοῖς ὄρεσιν ἦν κράζων καὶ κατακόπτων ἑαυτὸν λίθοις.
\vs{6}Καὶ ἰδὼν τὸν Ἰησοῦν ἀπὸ μακρόθεν ἔδραμεν καὶ προσεκύνησεν αὐτῷ
\vs{7}καὶ κράξας φωνῇ μεγάλῃ λέγει· Τί ἐμοὶ καὶ σοί, Ἰησοῦ Υἱὲ τοῦ Θεοῦ τοῦ Ὑψίστου; ὁρκίζω σε τὸν Θεόν, μή με βασανίσῃς.
\vs{8}ἔλεγεν γὰρ αὐτῷ· Ἔξελθε τὸ πνεῦμα τὸ ἀκάθαρτον ἐκ τοῦ ἀνθρώπου.
\vs{9}Καὶ ἐπηρώτα αὐτόν· Τί ὄνομά σοι; Καὶ λέγει αὐτῷ· Λεγιὼν ὄνομά μοι, ὅτι πολλοί ἐσμεν.
\vs{10}καὶ παρεκάλει αὐτὸν πολλὰ ἵνα μὴ αὐτὰ ἀποστείλῃ ἔξω τῆς χώρας.
\vs{11}Ἦν δὲ ἐκεῖ πρὸς τῷ ὄρει ἀγέλη χοίρων μεγάλη βοσκομένη·
\vs{12}καὶ παρεκάλεσαν αὐτὸν λέγοντες· Πέμψον ἡμᾶς εἰς τοὺς χοίρους, ἵνα εἰς αὐτοὺς εἰσέλθωμεν.
\vs{13}Καὶ ἐπέτρεψεν αὐτοῖς. καὶ ἐξελθόντα τὰ πνεύματα τὰ ἀκάθαρτα εἰσῆλθον εἰς τοὺς χοίρους, καὶ ὥρμησεν ἡ ἀγέλη κατὰ τοῦ κρημνοῦ εἰς τὴν θάλασσαν, ὡς δισχίλιοι, καὶ ἐπνίγοντο ἐν τῇ θαλάσσῃ.

\vs{14}Καὶ οἱ βόσκοντες αὐτοὺς ἔφυγον καὶ ἀπήγγειλαν εἰς τὴν πόλιν καὶ εἰς τοὺς ἀγρούς· καὶ ἦλθον ἰδεῖν τί ἐστιν τὸ γεγονός
\vs{15}καὶ ἔρχονται πρὸς τὸν Ἰησοῦν καὶ θεωροῦσιν τὸν δαιμονιζόμενον καθήμενον ἱματισμένον καὶ σωφρονοῦντα, τὸν ἐσχηκότα τὸν λεγιῶνα, καὶ ἐφοβήθησαν.
\vs{16}καὶ διηγήσαντο αὐτοῖς οἱ ἰδόντες πῶς ἐγένετο τῷ δαιμονιζομένῳ καὶ περὶ τῶν χοίρων.
\vs{17}καὶ ἤρξαντο παρακαλεῖν αὐτὸν ἀπελθεῖν ἀπὸ τῶν ὁρίων αὐτῶν.

\vs{18}Καὶ ἐμβαίνοντος αὐτοῦ εἰς τὸ πλοῖον παρεκάλει αὐτὸν ὁ δαιμονισθεὶς ἵνα μετ᾽ αὐτοῦ ᾖ.
\vs{19}καὶ οὐκ ἀφῆκεν αὐτόν, ἀλλὰ λέγει αὐτῷ· Ὕπαγε εἰς τὸν οἶκόν σου πρὸς τοὺς σούς καὶ ἀπάγγειλον αὐτοῖς ὅσα ὁ Κύριός σοι πεποίηκεν καὶ ἠλέησέν σε.
\vs{20}Καὶ ἀπῆλθεν καὶ ἤρξατο κηρύσσειν ἐν τῇ Δεκαπόλει ὅσα ἐποίησεν αὐτῷ ὁ Ἰησοῦς, καὶ πάντες ἐθαύμαζον.

\vs{21}Καὶ διαπεράσαντος τοῦ Ἰησοῦ ἐν τῷ πλοίῳ πάλιν εἰς τὸ πέραν συνήχθη ὄχλος πολὺς ἐπ᾽ αὐτόν, καὶ ἦν παρὰ τὴν θάλασσαν.
\vs{22}καὶ ἔρχεται εἷς τῶν ἀρχισυναγώγων, ὀνόματι Ἰάϊρος, καὶ ἰδὼν αὐτὸν πίπτει πρὸς τοὺς πόδας αὐτοῦ
\vs{23}καὶ παρακαλεῖ αὐτὸν πολλὰ λέγων ὅτι Τὸ θυγάτριόν μου ἐσχάτως ἔχει, ἵνα ἐλθὼν ἐπιθῇς τὰς χεῖρας αὐτῇ ἵνα σωθῇ καὶ ζήσῃ.
\vs{24}Καὶ ἀπῆλθεν μετ᾽ αὐτοῦ. καὶ ἠκολούθει αὐτῷ ὄχλος πολύς καὶ συνέθλιβον αὐτόν.

\vs{25}Καὶ γυνὴ οὖσα ἐν ῥύσει αἵματος δώδεκα ἔτη
\vs{26}καὶ πολλὰ παθοῦσα ὑπὸ πολλῶν ἰατρῶν καὶ δαπανήσασα τὰ παρ᾽ αὐτῆς πάντα καὶ μηδὲν ὠφεληθεῖσα ἀλλὰ μᾶλλον εἰς τὸ χεῖρον ἐλθοῦσα,
\vs{27}ἀκούσασα περὶ τοῦ Ἰησοῦ, ἐλθοῦσα ἐν τῷ ὄχλῳ ὄπισθεν ἥψατο τοῦ ἱματίου αὐτοῦ·
\vs{28}ἔλεγεν γὰρ ὅτι Ἐὰν ἅψωμαι κἂν τῶν ἱματίων αὐτοῦ σωθήσομαι.
\vs{29}καὶ εὐθὺς ἐξηράνθη ἡ πηγὴ τοῦ αἵματος αὐτῆς καὶ ἔγνω τῷ σώματι ὅτι ἴαται ἀπὸ τῆς μάστιγος.
\vs{30}Καὶ εὐθὺς ὁ Ἰησοῦς ἐπιγνοὺς ἐν ἑαυτῷ τὴν ἐξ αὐτοῦ δύναμιν ἐξελθοῦσαν ἐπιστραφεὶς ἐν τῷ ὄχλῳ ἔλεγεν· Τίς μου ἥψατο τῶν ἱματίων;
\vs{31}Καὶ ἔλεγον αὐτῷ οἱ μαθηταὶ αὐτοῦ· Βλέπεις τὸν ὄχλον συνθλίβοντά σε καὶ λέγεις· Τίς μου ἥψατο;
\vs{32}Καὶ περιεβλέπετο ἰδεῖν τὴν τοῦτο ποιήσασαν.
\vs{33}ἡ δὲ γυνὴ φοβηθεῖσα καὶ τρέμουσα, εἰδυῖα ὃ γέγονεν αὐτῇ, ἦλθεν καὶ προσέπεσεν αὐτῷ καὶ εἶπεν αὐτῷ πᾶσαν τὴν ἀλήθειαν.
\vs{34}Ὁ δὲ εἶπεν αὐτῇ· Θυγάτηρ, ἡ πίστις σου σέσωκέν σε· ὕπαγε εἰς εἰρήνην καὶ ἴσθι ὑγιὴς ἀπὸ τῆς μάστιγός σου.

\vs{35}Ἔτι αὐτοῦ λαλοῦντος ἔρχονται ἀπὸ τοῦ ἀρχισυναγώγου λέγοντες ὅτι Ἡ θυγάτηρ σου ἀπέθανεν· τί ἔτι σκύλλεις τὸν διδάσκαλον;
\vs{36}Ὁ δὲ Ἰησοῦς παρακούσας τὸν λόγον λαλούμενον λέγει τῷ ἀρχισυναγώγῳ· Μὴ φοβοῦ, μόνον πίστευε.
\vs{37}καὶ οὐκ ἀφῆκεν οὐδένα μετ᾽ αὐτοῦ συνακολουθῆσαι εἰ μὴ τὸν Πέτρον καὶ Ἰάκωβον καὶ Ἰωάννην τὸν ἀδελφὸν Ἰακώβου.
\vs{38}Καὶ ἔρχονται εἰς τὸν οἶκον τοῦ ἀρχισυναγώγου, καὶ θεωρεῖ θόρυβον καὶ κλαίοντας καὶ ἀλαλάζοντας πολλά,
\vs{39}καὶ εἰσελθὼν λέγει αὐτοῖς· Τί θορυβεῖσθε καὶ κλαίετε; τὸ παιδίον οὐκ ἀπέθανεν ἀλλὰ καθεύδει.
\vs{40}καὶ κατεγέλων αὐτοῦ. Αὐτὸς δὲ ἐκβαλὼν πάντας παραλαμβάνει τὸν πατέρα τοῦ παιδίου καὶ τὴν μητέρα καὶ τοὺς μετ᾽ αὐτοῦ καὶ εἰσπορεύεται ὅπου ἦν τὸ παιδίον.
\vs{41}καὶ κρατήσας τῆς χειρὸς τοῦ παιδίου λέγει αὐτῇ· Ταλιθὰ κούμ, ὅ ἐστιν μεθερμηνευόμενον· Τὸ Κοράσιον, σοὶ λέγω, ἔγειρε.
\vs{42}καὶ εὐθὺς ἀνέστη τὸ κοράσιον καὶ περιεπάτει· ἦν γὰρ ἐτῶν δώδεκα. καὶ ἐξέστησαν εὐθὺς ἐκστάσει μεγάλῃ.
\vs{43}καὶ διεστείλατο αὐτοῖς πολλὰ ἵνα μηδεὶς γνοῖ τοῦτο, καὶ εἶπεν δοθῆναι αὐτῇ φαγεῖν.

\ch{6}
Καὶ ἐξῆλθεν ἐκεῖθεν καὶ ἔρχεται εἰς τὴν πατρίδα αὐτοῦ, καὶ ἀκολουθοῦσιν αὐτῷ οἱ μαθηταὶ αὐτοῦ.
\vs{2}καὶ γενομένου σαββάτου ἤρξατο διδάσκειν ἐν τῇ συναγωγῇ, καὶ πολλοὶ ἀκούοντες ἐξεπλήσσοντο λέγοντες· Πόθεν τούτῳ ταῦτα, καὶ τίς ἡ σοφία ἡ δοθεῖσα τούτῳ, καὶ αἱ δυνάμεις τοιαῦται διὰ τῶν χειρῶν αὐτοῦ γινόμεναι;
\vs{3}οὐχ οὗτός ἐστιν ὁ τέκτων, ὁ υἱὸς τῆς Μαρίας καὶ ἀδελφὸς Ἰακώβου καὶ Ἰωσῆτος καὶ Ἰούδα καὶ Σίμωνος; καὶ οὐκ εἰσὶν αἱ ἀδελφαὶ αὐτοῦ ὧδε πρὸς ἡμᾶς; καὶ ἐσκανδαλίζοντο ἐν αὐτῷ.
\vs{4}Καὶ ἔλεγεν αὐτοῖς ὁ Ἰησοῦς ὅτι Οὐκ ἔστιν προφήτης ἄτιμος εἰ μὴ ἐν τῇ πατρίδι αὐτοῦ καὶ ἐν τοῖς συγγενεῦσιν αὐτοῦ καὶ ἐν τῇ οἰκίᾳ αὐτοῦ.
\vs{5}καὶ οὐκ ἐδύνατο ἐκεῖ ποιῆσαι οὐδεμίαν δύναμιν, εἰ μὴ ὀλίγοις ἀρρώστοις ἐπιθεὶς τὰς χεῖρας ἐθεράπευσεν.
\vs{6}καὶ ἐθαύμαζεν διὰ τὴν ἀπιστίαν αὐτῶν. Καὶ περιῆγεν τὰς κώμας κύκλῳ διδάσκων.
\vs{7}Καὶ προσκαλεῖται τοὺς δώδεκα καὶ ἤρξατο αὐτοὺς ἀποστέλλειν δύο δύο καὶ ἐδίδου αὐτοῖς ἐξουσίαν τῶν πνευμάτων τῶν ἀκαθάρτων,
\vs{8}καὶ παρήγγειλεν αὐτοῖς ἵνα μηδὲν αἴρωσιν εἰς ὁδὸν εἰ μὴ ῥάβδον μόνον, μὴ ἄρτον, μὴ πήραν, μὴ εἰς τὴν ζώνην χαλκόν,
\vs{9}ἀλλὰ ὑποδεδεμένους σανδάλια, καὶ μὴ ἐνδύσησθε δύο χιτῶνας.
\vs{10}Καὶ ἔλεγεν αὐτοῖς· Ὅπου ἐὰν εἰσέλθητε εἰς οἰκίαν, ἐκεῖ μένετε ἕως ἂν ἐξέλθητε ἐκεῖθεν.
\vs{11}καὶ ὃς ἂν τόπος μὴ δέξηται ὑμᾶς μηδὲ ἀκούσωσιν ὑμῶν, ἐκπορευόμενοι ἐκεῖθεν ἐκτινάξατε τὸν χοῦν τὸν ὑποκάτω τῶν ποδῶν ὑμῶν εἰς μαρτύριον αὐτοῖς.
\vs{12}Καὶ ἐξελθόντες ἐκήρυξαν ἵνα μετανοῶσιν,
\vs{13}καὶ δαιμόνια πολλὰ ἐξέβαλλον, καὶ ἤλειφον ἐλαίῳ πολλοὺς ἀρρώστους καὶ ἐθεράπευον.

\vs{14}Καὶ ἤκουσεν ὁ βασιλεὺς Ἡρῴδης, φανερὸν γὰρ ἐγένετο τὸ ὄνομα αὐτοῦ, καὶ ἔλεγον ὅτι Ἰωάννης ὁ Βαπτίζων ἐγήγερται ἐκ νεκρῶν καὶ διὰ τοῦτο ἐνεργοῦσιν αἱ δυνάμεις ἐν αὐτῷ.
\vs{15}ἄλλοι δὲ ἔλεγον ὅτι Ἠλίας ἐστίν· ἄλλοι δὲ ἔλεγον ὅτι Προφήτης ὡς εἷς τῶν προφητῶν.
\vs{16}Ἀκούσας δὲ ὁ Ἡρῴδης ἔλεγεν· Ὃν ἐγὼ ἀπεκεφάλισα Ἰωάννην, οὗτος ἠγέρθη.

\vs{17}Αὐτὸς γὰρ ὁ Ἡρῴδης ἀποστείλας ἐκράτησεν τὸν Ἰωάννην καὶ ἔδησεν αὐτὸν ἐν φυλακῇ διὰ Ἡρῳδιάδα τὴν γυναῖκα Φιλίππου τοῦ ἀδελφοῦ αὐτοῦ, ὅτι αὐτὴν ἐγάμησεν·
\vs{18}ἔλεγεν γὰρ ὁ Ἰωάννης τῷ Ἡρῴδῃ ὅτι Οὐκ ἔξεστίν σοι ἔχειν τὴν γυναῖκα τοῦ ἀδελφοῦ σου.
\vs{19}Ἡ δὲ Ἡρῳδιὰς ἐνεῖχεν αὐτῷ καὶ ἤθελεν αὐτὸν ἀποκτεῖναι, καὶ οὐκ ἠδύνατο·
\vs{20}ὁ γὰρ Ἡρῴδης ἐφοβεῖτο τὸν Ἰωάννην, εἰδὼς αὐτὸν ἄνδρα δίκαιον καὶ ἅγιον, καὶ συνετήρει αὐτόν, καὶ ἀκούσας αὐτοῦ πολλὰ ἠπόρει, καὶ ἡδέως αὐτοῦ ἤκουεν.

\vs{21}Καὶ γενομένης ἡμέρας εὐκαίρου ὅτε Ἡρῴδης τοῖς γενεσίοις αὐτοῦ δεῖπνον ἐποίησεν τοῖς μεγιστᾶσιν αὐτοῦ καὶ τοῖς χιλιάρχοις καὶ τοῖς πρώτοις τῆς Γαλιλαίας,
\vs{22}καὶ εἰσελθούσης τῆς θυγατρὸς αὐτοῦ Ἡρῳδιάδος καὶ ὀρχησαμένης ἤρεσεν τῷ Ἡρῴδῃ καὶ τοῖς συνανακειμένοις. εἶπεν ὁ βασιλεὺς τῷ κορασίῳ· Αἴτησόν με ὃ ἐὰν θέλῃς, καὶ δώσω σοι·
\vs{23}καὶ ὤμοσεν αὐτῇ Πολλά Ὅ τι ἐάν με αἰτήσῃς δώσω σοι ἕως ἡμίσους τῆς βασιλείας μου.
\vs{24}Καὶ ἐξελθοῦσα εἶπεν τῇ μητρὶ αὐτῆς· Τί αἰτήσωμαι; Ἡ δὲ εἶπεν· Τὴν κεφαλὴν Ἰωάννου τοῦ Βαπτίζοντος.
\vs{25}Καὶ εἰσελθοῦσα εὐθὺς μετὰ σπουδῆς πρὸς τὸν βασιλέα ᾐτήσατο λέγουσα· Θέλω ἵνα ἐξαυτῆς δῷς μοι ἐπὶ πίνακι τὴν κεφαλὴν Ἰωάννου τοῦ Βαπτιστοῦ.
\vs{26}Καὶ περίλυπος γενόμενος ὁ βασιλεὺς διὰ τοὺς ὅρκους καὶ τοὺς ἀνακειμένους οὐκ ἠθέλησεν ἀθετῆσαι αὐτήν·
\vs{27}καὶ εὐθὺς ἀποστείλας ὁ βασιλεὺς σπεκουλάτορα ἐπέταξεν ἐνέγκαι τὴν κεφαλὴν αὐτοῦ. καὶ ἀπελθὼν ἀπεκεφάλισεν αὐτὸν ἐν τῇ φυλακῇ
\vs{28}καὶ ἤνεγκεν τὴν κεφαλὴν αὐτοῦ ἐπὶ πίνακι καὶ ἔδωκεν αὐτὴν τῷ κορασίῳ, καὶ τὸ κοράσιον ἔδωκεν αὐτὴν τῇ μητρὶ αὐτῆς.
\vs{29}καὶ ἀκούσαντες οἱ μαθηταὶ αὐτοῦ ἦλθον καὶ ἦραν τὸ πτῶμα αὐτοῦ καὶ ἔθηκαν αὐτὸ ἐν μνημείῳ.

\vs{30}Καὶ συνάγονται οἱ ἀπόστολοι πρὸς τὸν Ἰησοῦν καὶ ἀπήγγειλαν αὐτῷ πάντα ὅσα ἐποίησαν καὶ ὅσα ἐδίδαξαν.
\vs{31}καὶ λέγει αὐτοῖς· Δεῦτε ὑμεῖς αὐτοὶ κατ᾽ ἰδίαν εἰς ἔρημον τόπον καὶ ἀναπαύσασθε ὀλίγον. ἦσαν γὰρ οἱ ἐρχόμενοι καὶ οἱ ὑπάγοντες πολλοί, καὶ οὐδὲ φαγεῖν εὐκαίρουν.

\vs{32}Καὶ ἀπῆλθον ἐν τῷ πλοίῳ εἰς ἔρημον τόπον κατ᾽ ἰδίαν.
\vs{33}καὶ εἶδον αὐτοὺς ὑπάγοντας καὶ ἐπέγνωσαν πολλοί καὶ πεζῇ ἀπὸ πασῶν τῶν πόλεων συνέδραμον ἐκεῖ καὶ προῆλθον αὐτούς.

\vs{34}Καὶ ἐξελθὼν εἶδεν πολὺν ὄχλον καὶ ἐσπλαγχνίσθη ἐπ᾽ αὐτοὺς, ὅτι ἦσαν ὡς πρόβατα μὴ ἔχοντα ποιμένα, καὶ ἤρξατο διδάσκειν αὐτοὺς πολλά.

\vs{35}Καὶ ἤδη ὥρας πολλῆς γενομένης προσελθόντες αὐτῷ οἱ μαθηταὶ αὐτοῦ ἔλεγον ὅτι Ἔρημός ἐστιν ὁ τόπος καὶ ἤδη ὥρα πολλή·
\vs{36}ἀπόλυσον αὐτούς, ἵνα ἀπελθόντες εἰς τοὺς κύκλῳ ἀγροὺς καὶ κώμας ἀγοράσωσιν ἑαυτοῖς τί φάγωσιν.
\vs{37}Ὁ δὲ ἀποκριθεὶς εἶπεν αὐτοῖς· Δότε αὐτοῖς ὑμεῖς φαγεῖν. Καὶ λέγουσιν αὐτῷ· Ἀπελθόντες ἀγοράσωμεν δηναρίων διακοσίων ἄρτους καὶ δώσομεν αὐτοῖς φαγεῖν;
\vs{38}Ὁ δὲ λέγει αὐτοῖς· Πόσους ἄρτους ἔχετε; ὑπάγετε ἴδετε. Καὶ γνόντες λέγουσιν· Πέντε, καὶ δύο ἰχθύας.
\vs{39}Καὶ ἐπέταξεν αὐτοῖς ἀνακλῖναι πάντας συμπόσια συμπόσια ἐπὶ τῷ χλωρῷ χόρτῳ.
\vs{40}καὶ ἀνέπεσαν πρασιαὶ πρασιαὶ κατὰ ἑκατὸν καὶ κατὰ πεντήκοντα.
\vs{41}Καὶ λαβὼν τοὺς πέντε ἄρτους καὶ τοὺς δύο ἰχθύας ἀναβλέψας εἰς τὸν οὐρανὸν εὐλόγησεν καὶ κατέκλασεν τοὺς ἄρτους καὶ ἐδίδου τοῖς μαθηταῖς αὐτοῦ ἵνα παρατιθῶσιν αὐτοῖς, καὶ τοὺς δύο ἰχθύας ἐμέρισεν πᾶσιν.
\vs{42}Καὶ ἔφαγον πάντες καὶ ἐχορτάσθησαν,
\vs{43}καὶ ἦραν κλάσματα δώδεκα κοφίνων πληρώματα καὶ ἀπὸ τῶν ἰχθύων.
\vs{44}καὶ ἦσαν οἱ φαγόντες τοὺς ἄρτους πεντακισχίλιοι ἄνδρες.

\vs{45}Καὶ εὐθὺς ἠνάγκασεν τοὺς μαθητὰς αὐτοῦ ἐμβῆναι εἰς τὸ πλοῖον καὶ προάγειν εἰς τὸ πέραν πρὸς Βηθσαϊδάν, ἕως αὐτὸς ἀπολύει τὸν ὄχλον.
\vs{46}καὶ ἀποταξάμενος αὐτοῖς ἀπῆλθεν εἰς τὸ ὄρος προσεύξασθαι.
\vs{47}Καὶ ὀψίας γενομένης ἦν τὸ πλοῖον ἐν μέσῳ τῆς θαλάσσης, καὶ αὐτὸς μόνος ἐπὶ τῆς γῆς.
\vs{48}καὶ ἰδὼν αὐτοὺς βασανιζομένους ἐν τῷ ἐλαύνειν, ἦν γὰρ ὁ ἄνεμος ἐναντίος αὐτοῖς, περὶ τετάρτην φυλακὴν τῆς νυκτὸς ἔρχεται πρὸς αὐτοὺς περιπατῶν ἐπὶ τῆς θαλάσσης καὶ ἤθελεν παρελθεῖν αὐτούς.
\vs{49}οἱ δὲ ἰδόντες αὐτὸν ἐπὶ τῆς θαλάσσης περιπατοῦντα ἔδοξαν ὅτι φάντασμά ἐστιν, καὶ ἀνέκραξαν·
\vs{50}πάντες γὰρ αὐτὸν εἶδον καὶ ἐταράχθησαν. ὁ Δὲ εὐθὺς ἐλάλησεν μετ᾽ αὐτῶν, καὶ λέγει αὐτοῖς· Θαρσεῖτε, ἐγώ εἰμι· μὴ φοβεῖσθε.
\vs{51}καὶ ἀνέβη πρὸς αὐτοὺς εἰς τὸ πλοῖον καὶ ἐκόπασεν ὁ ἄνεμος, καὶ λίαν ἐκ περισσοῦ ἐν ἑαυτοῖς ἐξίσταντο·
\vs{52}οὐ γὰρ συνῆκαν ἐπὶ τοῖς ἄρτοις, ἀλλ᾽ ἦν αὐτῶν ἡ καρδία πεπωρωμένη.

\vs{53}Καὶ διαπεράσαντες ἐπὶ τὴν γῆν ἦλθον εἰς Γεννησαρὲτ καὶ προσωρμίσθησαν.
\vs{54}καὶ ἐξελθόντων αὐτῶν ἐκ τοῦ πλοίου εὐθὺς ἐπιγνόντες αὐτὸν
\vs{55}περιέδραμον ὅλην τὴν χώραν ἐκείνην καὶ ἤρξαντο ἐπὶ τοῖς κραβάττοις τοὺς κακῶς ἔχοντας περιφέρειν ὅπου ἤκουον ὅτι ἐστίν.
\vs{56}καὶ ὅπου ἂν εἰσεπορεύετο εἰς κώμας ἢ εἰς πόλεις ἢ εἰς ἀγροὺς, ἐν ταῖς ἀγοραῖς ἐτίθεσαν τοὺς ἀσθενοῦντας καὶ παρεκάλουν αὐτὸν ἵνα κἂν τοῦ κρασπέδου τοῦ ἱματίου αὐτοῦ ἅψωνται· καὶ ὅσοι ἂν ἥψαντο αὐτοῦ ἐσῴζοντο.

\ch{7}
Καὶ συνάγονται πρὸς αὐτὸν οἱ Φαρισαῖοι καί τινες τῶν γραμματέων ἐλθόντες ἀπὸ Ἱεροσολύμων.
\vs{2}καὶ ἰδόντες τινὰς τῶν μαθητῶν αὐτοῦ ὅτι κοιναῖς χερσίν, τοῦτ᾽ ἔστιν ἀνίπτοις, ἐσθίουσιν τοὺς ἄρτους—
\vs{3}Οἱ γὰρ Φαρισαῖοι καὶ πάντες οἱ Ἰουδαῖοι ἐὰν μὴ πυγμῇ νίψωνται τὰς χεῖρας οὐκ ἐσθίουσιν, κρατοῦντες τὴν παράδοσιν τῶν πρεσβυτέρων,
\vs{4}καὶ ἀπ᾽ ἀγορᾶς ἐὰν μὴ βαπτίσωνται οὐκ ἐσθίουσιν, καὶ ἄλλα πολλά ἐστιν ἃ παρέλαβον κρατεῖν, βαπτισμοὺς ποτηρίων καὶ ξεστῶν καὶ χαλκίων καὶ κλινῶν—
\vs{5}Καὶ ἐπερωτῶσιν αὐτὸν οἱ Φαρισαῖοι καὶ οἱ γραμματεῖς· Διὰ τί οὐ περιπατοῦσιν οἱ μαθηταί σου κατὰ τὴν παράδοσιν τῶν πρεσβυτέρων, ἀλλὰ κοιναῖς χερσὶν ἐσθίουσιν τὸν ἄρτον;

\vs{6}Ὁ δὲ εἶπεν αὐτοῖς· Καλῶς ἐπροφήτευσεν Ἠσαΐας περὶ ὑμῶν τῶν ὑποκριτῶν, ὡς γέγραπται ὅτι 
\begin{poetryblock}

\begin{quote}Οὗτος ὁ λαὸς τοῖς χείλεσίν με τιμᾷ,\end{quote} 

\begin{quote}Ἡ δὲ καρδία αὐτῶν πόρρω ἀπέχει ἀπ᾽ ἐμοῦ·\end{quote}

\begin{quote} \vs{7}Μάτην δὲ σέβονταί με\end{quote} 

\begin{quote}Διδάσκοντες διδασκαλίας ἐντάλματα ἀνθρώπων.\end{quote}
\end{poetryblock}

\vs{8}Ἀφέντες τὴν ἐντολὴν τοῦ Θεοῦ κρατεῖτε τὴν παράδοσιν τῶν ἀνθρώπων.
\vs{9}Καὶ ἔλεγεν αὐτοῖς· Καλῶς ἀθετεῖτε τὴν ἐντολὴν τοῦ Θεοῦ, ἵνα τὴν παράδοσιν ὑμῶν τηρήσητε.
\vs{10}Μωϋσῆς γὰρ εἶπεν· Τίμα τὸν πατέρα σου καὶ τὴν μητέρα σου, καί· Ὁ κακολογῶν πατέρα ἢ μητέρα θανάτῳ τελευτάτω.
\vs{11}ὑμεῖς δὲ λέγετε· Ἐὰν εἴπῃ ἄνθρωπος τῷ πατρὶ ἢ τῇ μητρί· Κορβᾶν, ὅ ἐστιν Δῶρον, Ὃ ἐὰν ἐξ ἐμοῦ ὠφεληθῇς,
\vs{12}οὐκέτι ἀφίετε αὐτὸν οὐδὲν ποιῆσαι τῷ πατρὶ ἢ τῇ μητρί,
\vs{13}ἀκυροῦντες τὸν λόγον τοῦ Θεοῦ τῇ παραδόσει ὑμῶν ᾗ παρεδώκατε· καὶ παρόμοια τοιαῦτα πολλὰ ποιεῖτε.

\vs{14}Καὶ προσκαλεσάμενος πάλιν τὸν ὄχλον ἔλεγεν αὐτοῖς· Ἀκούσατέ μου πάντες καὶ σύνετε.
\vs{15}οὐδέν ἐστιν ἔξωθεν τοῦ ἀνθρώπου εἰσπορευόμενον εἰς αὐτὸν ὃ δύναται κοινῶσαι αὐτόν, ἀλλὰ τὰ ἐκ τοῦ ἀνθρώπου ἐκπορευόμενά ἐστιν τὰ κοινοῦντα τὸν ἄνθρωπον.
\vs{17}Καὶ ὅτε εἰσῆλθεν εἰς οἶκον ἀπὸ τοῦ ὄχλου, ἐπηρώτων αὐτὸν οἱ μαθηταὶ αὐτοῦ τὴν παραβολήν.
\vs{18}Καὶ λέγει αὐτοῖς· Οὕτως καὶ ὑμεῖς ἀσύνετοί ἐστε; οὐ νοεῖτε ὅτι πᾶν τὸ ἔξωθεν εἰσπορευόμενον εἰς τὸν ἄνθρωπον οὐ δύναται αὐτὸν κοινῶσαι
\vs{19}ὅτι οὐκ εἰσπορεύεται αὐτοῦ εἰς τὴν καρδίαν ἀλλ᾽ εἰς τὴν κοιλίαν, καὶ εἰς τὸν ἀφεδρῶνα ἐκπορεύεται, καθαρίζων πάντα τὰ βρώματα;
\vs{20}Ἔλεγεν δὲ ὅτι Τὸ ἐκ τοῦ ἀνθρώπου ἐκπορευόμενον, ἐκεῖνο κοινοῖ τὸν ἄνθρωπον.
\vs{21}ἔσωθεν γὰρ ἐκ τῆς καρδίας τῶν ἀνθρώπων οἱ διαλογισμοὶ οἱ κακοὶ ἐκπορεύονται, πορνεῖαι, κλοπαί, φόνοι,
\vs{22}μοιχεῖαι, πλεονεξίαι, πονηρίαι, δόλος, ἀσέλγεια, ὀφθαλμὸς πονηρός, βλασφημία, ὑπερηφανία, ἀφροσύνη·
\vs{23}πάντα ταῦτα τὰ πονηρὰ ἔσωθεν ἐκπορεύεται καὶ κοινοῖ τὸν ἄνθρωπον.

\vs{24}Ἐκεῖθεν δὲ ἀναστὰς ἀπῆλθεν εἰς τὰ ὅρια Τύρου. Καὶ εἰσελθὼν εἰς οἰκίαν οὐδένα ἤθελεν γνῶναι, καὶ οὐκ ἠδυνήθη λαθεῖν·
\vs{25}ἀλλ᾽ εὐθὺς ἀκούσασα γυνὴ περὶ αὐτοῦ, ἧς εἶχεν τὸ θυγάτριον αὐτῆς πνεῦμα ἀκάθαρτον, ἐλθοῦσα προσέπεσεν πρὸς τοὺς πόδας αὐτοῦ·
\vs{26}ἡ δὲ γυνὴ ἦν Ἑλληνίς, Συροφοινίκισσα τῷ γένει· καὶ ἠρώτα αὐτὸν ἵνα τὸ δαιμόνιον ἐκβάλῃ ἐκ τῆς θυγατρὸς αὐτῆς.
\vs{27}Καὶ ἔλεγεν αὐτῇ· Ἄφες πρῶτον χορτασθῆναι τὰ τέκνα, οὐ γάρ ἐστιν καλόν λαβεῖν τὸν ἄρτον τῶν τέκνων καὶ τοῖς κυναρίοις βαλεῖν.
\vs{28}Ἡ δὲ ἀπεκρίθη καὶ λέγει αὐτῷ· Κύριε· καὶ τὰ κυνάρια ὑποκάτω τῆς τραπέζης ἐσθίουσιν ἀπὸ τῶν ψιχίων τῶν παιδίων.
\vs{29}Καὶ εἶπεν αὐτῇ· Διὰ τοῦτον τὸν λόγον ὕπαγε, ἐξελήλυθεν ἐκ τῆς θυγατρός σου τὸ δαιμόνιον.
\vs{30}καὶ ἀπελθοῦσα εἰς τὸν οἶκον αὐτῆς εὗρεν τὸ παιδίον βεβλημένον ἐπὶ τὴν κλίνην καὶ τὸ δαιμόνιον ἐξεληλυθός.

\vs{31}Καὶ πάλιν ἐξελθὼν ἐκ τῶν ὁρίων Τύρου ἦλθεν διὰ Σιδῶνος εἰς τὴν θάλασσαν τῆς Γαλιλαίας ἀνὰ μέσον τῶν ὁρίων Δεκαπόλεως.
\vs{32}Καὶ φέρουσιν αὐτῷ κωφὸν καὶ μογιλάλον καὶ παρακαλοῦσιν αὐτὸν ἵνα ἐπιθῇ αὐτῷ τὴν χεῖρα.
\vs{33}Καὶ ἀπολαβόμενος αὐτὸν ἀπὸ τοῦ ὄχλου κατ᾽ ἰδίαν ἔβαλεν τοὺς δακτύλους αὐτοῦ εἰς τὰ ὦτα αὐτοῦ καὶ πτύσας ἥψατο τῆς γλώσσης αὐτοῦ,
\vs{34}καὶ ἀναβλέψας εἰς τὸν οὐρανὸν ἐστέναξεν καὶ λέγει αὐτῷ· Ἐφφαθά, ὅ ἐστιν Διανοίχθητι.
\vs{35}καὶ εὐθὺς ἠνοίγησαν αὐτοῦ αἱ ἀκοαί, καὶ ἐλύθη ὁ δεσμὸς τῆς γλώσσης αὐτοῦ καὶ ἐλάλει ὀρθῶς.
\vs{36}Καὶ διεστείλατο αὐτοῖς ἵνα μηδενὶ λέγωσιν· ὅσον δὲ αὐτοῖς διεστέλλετο, αὐτοὶ μᾶλλον περισσότερον ἐκήρυσσον.
\vs{37}καὶ ὑπερπερισσῶς ἐξεπλήσσοντο λέγοντες· Καλῶς πάντα πεποίηκεν, καὶ τοὺς κωφοὺς ποιεῖ ἀκούειν καὶ τοὺς ἀλάλους λαλεῖν.

\ch{8}
Ἐν ἐκείναις ταῖς ἡμέραις πάλιν πολλοῦ ὄχλου ὄντος καὶ μὴ ἐχόντων τί φάγωσιν, προσκαλεσάμενος τοὺς μαθητὰς λέγει αὐτοῖς·
\vs{2}Σπλαγχνίζομαι ἐπὶ τὸν ὄχλον, ὅτι ἤδη ἡμέραι τρεῖς προσμένουσίν μοι καὶ οὐκ ἔχουσιν τί φάγωσιν·
\vs{3}καὶ ἐὰν ἀπολύσω αὐτοὺς νήστεις εἰς οἶκον αὐτῶν, ἐκλυθήσονται ἐν τῇ ὁδῷ· καί τινες αὐτῶν ἀπὸ μακρόθεν ἥκασιν.
\vs{4}Καὶ ἀπεκρίθησαν αὐτῷ οἱ μαθηταὶ αὐτοῦ ὅτι Πόθεν τούτους δυνήσεταί τις ὧδε χορτάσαι ἄρτων ἐπ᾽ ἐρημίας;
\vs{5}Καὶ ἠρώτα αὐτούς· Πόσους ἔχετε ἄρτους; Οἱ δὲ εἶπαν· Ἑπτά.
\vs{6}Καὶ παραγγέλλει τῷ ὄχλῳ ἀναπεσεῖν ἐπὶ τῆς γῆς· καὶ λαβὼν τοὺς ἑπτὰ ἄρτους εὐχαριστήσας ἔκλασεν καὶ ἐδίδου τοῖς μαθηταῖς αὐτοῦ ἵνα παρατιθῶσιν, καὶ παρέθηκαν τῷ ὄχλῳ.
\vs{7}καὶ εἶχον ἰχθύδια ὀλίγα· καὶ εὐλογήσας αὐτὰ εἶπεν καὶ ταῦτα παρατιθέναι.
\vs{8}Καὶ ἔφαγον καὶ ἐχορτάσθησαν, καὶ ἦραν περισσεύματα κλασμάτων ἑπτὰ σπυρίδας.
\vs{9}ἦσαν δὲ ὡς τετρακισχίλιοι. καὶ ἀπέλυσεν αὐτούς.

\vs{10}Καὶ εὐθὺς ἐμβὰς εἰς τὸ πλοῖον μετὰ τῶν μαθητῶν αὐτοῦ ἦλθεν εἰς τὰ μέρη Δαλμανουθά.

\vs{11}Καὶ ἐξῆλθον οἱ Φαρισαῖοι καὶ ἤρξαντο συζητεῖν αὐτῷ, ζητοῦντες παρ᾽ αὐτοῦ σημεῖον ἀπὸ τοῦ οὐρανοῦ, πειράζοντες αὐτόν.
\vs{12}Καὶ ἀναστενάξας τῷ πνεύματι αὐτοῦ λέγει· Τί ἡ γενεὰ αὕτη ζητεῖ σημεῖον; ἀμὴν λέγω ὑμῖν, εἰ δοθήσεται τῇ γενεᾷ ταύτῃ σημεῖον.
\vs{13}καὶ ἀφεὶς αὐτοὺς πάλιν ἐμβὰς ἀπῆλθεν εἰς τὸ πέραν.

\vs{14}Καὶ ἐπελάθοντο λαβεῖν ἄρτους καὶ εἰ μὴ ἕνα ἄρτον οὐκ εἶχον μεθ᾽ ἑαυτῶν ἐν τῷ πλοίῳ.
\vs{15}καὶ διεστέλλετο αὐτοῖς λέγων· Ὁρᾶτε, βλέπετε ἀπὸ τῆς ζύμης τῶν Φαρισαίων καὶ τῆς ζύμης Ἡρῴδου.
\vs{16}καὶ διελογίζοντο πρὸς ἀλλήλους ὅτι ἄρτους οὐκ ἔχουσιν.
\vs{17}Καὶ γνοὺς λέγει αὐτοῖς· Τί διαλογίζεσθε ὅτι ἄρτους οὐκ ἔχετε; οὔπω νοεῖτε οὐδὲ συνίετε; πεπωρωμένην ἔχετε τὴν καρδίαν ὑμῶν;
\begin{poetryblock}

\begin{quote} \vs{18}Ὀφθαλμοὺς ἔχοντες οὐ βλέπετε\end{quote} 

\begin{quote}καὶ ὦτα ἔχοντες οὐκ ἀκούετε;\end{quote}
\end{poetryblock}

καὶ οὐ μνημονεύετε,

\vs{19}ὅτε τοὺς πέντε ἄρτους ἔκλασα εἰς τοὺς πεντακισχιλίους, πόσους κοφίνους κλασμάτων πλήρεις ἤρατε; Λέγουσιν αὐτῷ· Δώδεκα.
\vs{20}Ὅτε τοὺς ἑπτὰ εἰς τοὺς τετρακισχιλίους, πόσων σπυρίδων πληρώματα κλασμάτων ἤρατε; Καὶ λέγουσιν αὐτῷ· Ἑπτά.
\vs{21}Καὶ ἔλεγεν αὐτοῖς· Οὔπω συνίετε;

\vs{22}Καὶ ἔρχονται εἰς Βηθσαϊδάν. Καὶ φέρουσιν αὐτῷ τυφλὸν καὶ παρακαλοῦσιν αὐτὸν ἵνα αὐτοῦ ἅψηται.
\vs{23}καὶ ἐπιλαβόμενος τῆς χειρὸς τοῦ τυφλοῦ ἐξήνεγκεν αὐτὸν ἔξω τῆς κώμης καὶ πτύσας εἰς τὰ ὄμματα αὐτοῦ, ἐπιθεὶς τὰς χεῖρας αὐτῷ ἐπηρώτα αὐτόν· Εἴ τι βλέπεις;
\vs{24}Καὶ ἀναβλέψας ἔλεγεν· Βλέπω τοὺς ἀνθρώπους ὅτι ὡς δένδρα ὁρῶ περιπατοῦντας.
\vs{25}Εἶτα πάλιν ἐπέθηκεν τὰς χεῖρας ἐπὶ τοὺς ὀφθαλμοὺς αὐτοῦ, καὶ διέβλεψεν καὶ ἀπεκατέστη καὶ ἐνέβλεπεν τηλαυγῶς ἅπαντα.
\vs{26}καὶ ἀπέστειλεν αὐτὸν εἰς οἶκον αὐτοῦ λέγων· Μηδὲ εἰς τὴν κώμην εἰσέλθῃς.

\vs{27}Καὶ ἐξῆλθεν ὁ Ἰησοῦς καὶ οἱ μαθηταὶ αὐτοῦ εἰς τὰς κώμας Καισαρείας τῆς Φιλίππου· καὶ ἐν τῇ ὁδῷ ἐπηρώτα τοὺς μαθητὰς αὐτοῦ λέγων αὐτοῖς· Τίνα με λέγουσιν οἱ ἄνθρωποι εἶναι;
\vs{28}Οἱ δὲ εἶπαν αὐτῷ λέγοντες ὅτι Ἰωάννην τὸν Βαπτιστήν, καὶ ἄλλοι Ἠλίαν, ἄλλοι δὲ ὅτι εἷς τῶν προφητῶν.
\vs{29}Καὶ αὐτὸς ἐπηρώτα αὐτούς· Ὑμεῖς δὲ τίνα με λέγετε εἶναι; Ἀποκριθεὶς ὁ Πέτρος λέγει αὐτῷ· Σὺ εἶ ὁ Χριστός.
\vs{30}Καὶ ἐπετίμησεν αὐτοῖς ἵνα μηδενὶ λέγωσιν περὶ αὐτοῦ.

\vs{31}Καὶ ἤρξατο διδάσκειν αὐτοὺς ὅτι δεῖ τὸν Υἱὸν τοῦ ἀνθρώπου πολλὰ παθεῖν καὶ ἀποδοκιμασθῆναι ὑπὸ τῶν πρεσβυτέρων καὶ τῶν ἀρχιερέων καὶ τῶν γραμματέων καὶ ἀποκτανθῆναι καὶ μετὰ τρεῖς ἡμέρας ἀναστῆναι·
\vs{32}καὶ παρρησίᾳ τὸν λόγον ἐλάλει. καὶ προσλαβόμενος ὁ Πέτρος αὐτὸν ἤρξατο ἐπιτιμᾶν αὐτῷ.
\vs{33}Ὁ δὲ ἐπιστραφεὶς καὶ ἰδὼν τοὺς μαθητὰς αὐτοῦ ἐπετίμησεν Πέτρῳ καὶ λέγει· Ὕπαγε ὀπίσω μου, Σατανᾶ, ὅτι οὐ φρονεῖς τὰ τοῦ Θεοῦ ἀλλὰ τὰ τῶν ἀνθρώπων.

\vs{34}Καὶ προσκαλεσάμενος τὸν ὄχλον σὺν τοῖς μαθηταῖς αὐτοῦ εἶπεν αὐτοῖς· Εἴ τις θέλει ὀπίσω μου ἀκολουθεῖν, ἀπαρνησάσθω ἑαυτὸν καὶ ἀράτω τὸν σταυρὸν αὐτοῦ καὶ ἀκολουθείτω μοι.
\vs{35}ὃς γὰρ ἐὰν θέλῃ τὴν ψυχὴν αὐτοῦ σῶσαι ἀπολέσει αὐτήν· ὃς δ᾽ ἂν ἀπολέσει τὴν ψυχὴν αὐτοῦ ἕνεκεν ἐμοῦ καὶ τοῦ εὐαγγελίου σώσει αὐτήν.
\vs{36}Τί γὰρ ὠφελεῖ ἄνθρωπον κερδῆσαι τὸν κόσμον ὅλον καὶ ζημιωθῆναι τὴν ψυχὴν αὐτοῦ;
\vs{37}τί γὰρ δοῖ ἄνθρωπος ἀντάλλαγμα τῆς ψυχῆς αὐτοῦ;
\vs{38}ὃς γὰρ ἐὰν ἐπαισχυνθῇ με καὶ τοὺς ἐμοὺς λόγους ἐν τῇ γενεᾷ ταύτῃ τῇ μοιχαλίδι καὶ ἁμαρτωλῷ, καὶ ὁ Υἱὸς τοῦ ἀνθρώπου ἐπαισχυνθήσεται αὐτὸν, ὅταν ἔλθῃ ἐν τῇ δόξῃ τοῦ Πατρὸς αὐτοῦ μετὰ τῶν ἀγγέλων τῶν ἁγίων.

\ch{9}
Καὶ ἔλεγεν αὐτοῖς· Ἀμὴν λέγω ὑμῖν ὅτι εἰσίν τινες ὧδε τῶν ἑστηκότων οἵτινες οὐ μὴ γεύσωνται θανάτου ἕως ἂν ἴδωσιν τὴν βασιλείαν τοῦ Θεοῦ ἐληλυθυῖαν ἐν δυνάμει.

\vs{2}Καὶ μετὰ ἡμέρας ἓξ παραλαμβάνει ὁ Ἰησοῦς τὸν Πέτρον καὶ τὸν Ἰάκωβον καὶ τὸν Ἰωάννην καὶ ἀναφέρει αὐτοὺς εἰς ὄρος ὑψηλὸν κατ᾽ ἰδίαν μόνους. καὶ μετεμορφώθη ἔμπροσθεν αὐτῶν,
\vs{3}καὶ τὰ ἱμάτια αὐτοῦ ἐγένετο στίλβοντα λευκὰ λίαν, οἷα γναφεὺς ἐπὶ τῆς γῆς οὐ δύναται οὕτως λευκᾶναι.
\vs{4}καὶ ὤφθη αὐτοῖς Ἠλίας σὺν Μωϋσεῖ καὶ ἦσαν συλλαλοῦντες τῷ Ἰησοῦ.
\vs{5}Καὶ ἀποκριθεὶς ὁ Πέτρος λέγει τῷ Ἰησοῦ· Ῥαββί, καλόν ἐστιν ἡμᾶς ὧδε εἶναι, καὶ ποιήσωμεν τρεῖς σκηνάς, σοὶ μίαν καὶ Μωϋσεῖ μίαν καὶ Ἠλίᾳ μίαν.
\vs{6}οὐ γὰρ ᾔδει τί ἀποκριθῇ, ἔκφοβοι γὰρ ἐγένοντο.
\vs{7}Καὶ ἐγένετο νεφέλη ἐπισκιάζουσα αὐτοῖς, καὶ ἐγένετο φωνὴ ἐκ τῆς νεφέλης· Οὗτός ἐστιν ὁ Υἱός μου ὁ ἀγαπητός, ἀκούετε αὐτοῦ.
\vs{8}καὶ ἐξάπινα περιβλεψάμενοι οὐκέτι οὐδένα εἶδον ἀλλὰ τὸν Ἰησοῦν μόνον μεθ᾽ ἑαυτῶν.

\vs{9}Καὶ καταβαινόντων αὐτῶν ἐκ τοῦ ὄρους διεστείλατο αὐτοῖς ἵνα μηδενὶ ἃ εἶδον διηγήσωνται, εἰ μὴ ὅταν ὁ Υἱὸς τοῦ ἀνθρώπου ἐκ νεκρῶν ἀναστῇ.
\vs{10}καὶ τὸν λόγον ἐκράτησαν πρὸς ἑαυτοὺς συζητοῦντες τί ἐστιν τὸ ἐκ νεκρῶν ἀναστῆναι.

\vs{11}καὶ ἐπηρώτων αὐτὸν λέγοντες· Ὅτι Λέγουσιν οἱ γραμματεῖς ὅτι Ἠλίαν δεῖ ἐλθεῖν πρῶτον;
\vs{12}Ὁ δὲ ἔφη αὐτοῖς· Ἠλίας μὲν ἐλθὼν πρῶτον ἀποκαθιστάνει πάντα· καὶ πῶς γέγραπται ἐπὶ τὸν Υἱὸν τοῦ ἀνθρώπου ἵνα πολλὰ πάθῃ καὶ ἐξουδενηθῇ;
\vs{13}ἀλλὰ λέγω ὑμῖν ὅτι καὶ Ἠλίας ἐλήλυθεν, καὶ ἐποίησαν αὐτῷ ὅσα ἤθελον, καθὼς γέγραπται ἐπ᾽ αὐτόν.

\vs{14}Καὶ ἐλθόντες πρὸς τοὺς μαθητὰς εἶδον ὄχλον πολὺν περὶ αὐτοὺς καὶ γραμματεῖς συζητοῦντας πρὸς αὐτούς.
\vs{15}καὶ εὐθὺς πᾶς ὁ ὄχλος ἰδόντες αὐτὸν ἐξεθαμβήθησαν καὶ προστρέχοντες ἠσπάζοντο αὐτόν.
\vs{16}Καὶ ἐπηρώτησεν αὐτούς· Τί συζητεῖτε πρὸς αὑτούς;
\vs{17}Καὶ ἀπεκρίθη αὐτῷ εἷς ἐκ τοῦ ὄχλου· Διδάσκαλε, ἤνεγκα τὸν υἱόν μου πρὸς σέ, ἔχοντα πνεῦμα ἄλαλον·
\vs{18}καὶ ὅπου ἐὰν αὐτὸν καταλάβῃ ῥήσσει αὐτόν, καὶ ἀφρίζει καὶ τρίζει τοὺς ὀδόντας καὶ ξηραίνεται· καὶ εἶπα τοῖς μαθηταῖς σου ἵνα αὐτὸ ἐκβάλωσιν, καὶ οὐκ ἴσχυσαν.
\vs{19}Ὁ δὲ ἀποκριθεὶς αὐτοῖς λέγει· Ὦ γενεὰ ἄπιστος, ἕως πότε πρὸς ὑμᾶς ἔσομαι; ἕως πότε ἀνέξομαι ὑμῶν; φέρετε αὐτὸν πρός με.
\vs{20}Καὶ ἤνεγκαν αὐτὸν πρὸς αὐτόν. καὶ ἰδὼν αὐτὸν τὸ πνεῦμα εὐθὺς συνεσπάραξεν αὐτόν, καὶ πεσὼν ἐπὶ τῆς γῆς ἐκυλίετο ἀφρίζων.
\vs{21}Καὶ ἐπηρώτησεν τὸν πατέρα αὐτοῦ· Πόσος χρόνος ἐστὶν ὡς τοῦτο γέγονεν αὐτῷ; Ὁ δὲ εἶπεν· Ἐκ παιδιόθεν·
\vs{22}καὶ πολλάκις καὶ εἰς πῦρ αὐτὸν ἔβαλεν καὶ εἰς ὕδατα ἵνα ἀπολέσῃ αὐτόν· ἀλλ᾽ εἴ τι δύνῃ, βοήθησον ἡμῖν σπλαγχνισθεὶς ἐφ᾽ ἡμᾶς.
\vs{23}Ὁ δὲ Ἰησοῦς εἶπεν αὐτῷ· Τὸ Εἰ δύνῃ, πάντα δυνατὰ τῷ πιστεύοντι.
\vs{24}Εὐθὺς κράξας ὁ πατὴρ τοῦ παιδίου ἔλεγεν· Πιστεύω· βοήθει μου τῇ ἀπιστίᾳ.
\vs{25}Ἰδὼν δὲ ὁ Ἰησοῦς ὅτι ἐπισυντρέχει ὄχλος, ἐπετίμησεν τῷ πνεύματι τῷ ἀκαθάρτῳ λέγων αὐτῷ· Τὸ ἄλαλον καὶ κωφὸν πνεῦμα, ἐγὼ ἐπιτάσσω σοι, ἔξελθε ἐξ αὐτοῦ καὶ μηκέτι εἰσέλθῃς εἰς αὐτόν.
\vs{26}Καὶ κράξας καὶ πολλὰ σπαράξας ἐξῆλθεν· καὶ ἐγένετο ὡσεὶ νεκρὸς, ὥστε τοὺς πολλοὺς λέγειν ὅτι ἀπέθανεν.
\vs{27}ὁ δὲ Ἰησοῦς κρατήσας τῆς χειρὸς αὐτοῦ ἤγειρεν αὐτόν, καὶ ἀνέστη.

\vs{28}Καὶ εἰσελθόντος αὐτοῦ εἰς οἶκον οἱ μαθηταὶ αὐτοῦ κατ᾽ ἰδίαν ἐπηρώτων αὐτόν· Ὅτι ἡμεῖς οὐκ ἠδυνήθημεν ἐκβαλεῖν αὐτό;
\vs{29}Καὶ εἶπεν αὐτοῖς· Τοῦτο τὸ γένος ἐν οὐδενὶ δύναται ἐξελθεῖν εἰ μὴ ἐν προσευχῇ.
\vs{30}Κἀκεῖθεν ἐξελθόντες παρεπορεύοντο διὰ τῆς Γαλιλαίας, καὶ οὐκ ἤθελεν ἵνα τις γνοῖ·
\vs{31}ἐδίδασκεν γὰρ τοὺς μαθητὰς αὐτοῦ καὶ ἔλεγεν αὐτοῖς ὅτι Ὁ Υἱὸς τοῦ ἀνθρώπου παραδίδοται εἰς χεῖρας ἀνθρώπων, καὶ ἀποκτενοῦσιν αὐτόν, καὶ ἀποκτανθεὶς μετὰ τρεῖς ἡμέρας ἀναστήσεται.
\vs{32}οἱ δὲ ἠγνόουν τὸ ῥῆμα, καὶ ἐφοβοῦντο αὐτὸν ἐπερωτῆσαι.

\vs{33}Καὶ ἦλθον εἰς Καφαρναούμ. Καὶ ἐν τῇ οἰκίᾳ γενόμενος ἐπηρώτα αὐτούς· Τί ἐν τῇ ὁδῷ διελογίζεσθε;
\vs{34}οἱ δὲ ἐσιώπων· πρὸς ἀλλήλους γὰρ διελέχθησαν ἐν τῇ ὁδῷ τίς μείζων.
\vs{35}Καὶ καθίσας ἐφώνησεν τοὺς δώδεκα καὶ λέγει αὐτοῖς· Εἴ τις θέλει πρῶτος εἶναι, ἔσται πάντων ἔσχατος καὶ πάντων διάκονος.
\vs{36}Καὶ λαβὼν παιδίον ἔστησεν αὐτὸ ἐν μέσῳ αὐτῶν καὶ ἐναγκαλισάμενος αὐτὸ εἶπεν αὐτοῖς·
\vs{37}Ὃς ἂν ἓν τῶν τοιούτων παιδίων δέξηται ἐπὶ τῷ ὀνόματί μου, ἐμὲ δέχεται· καὶ ὃς ἂν ἐμὲ δέχηται, οὐκ ἐμὲ δέχεται ἀλλὰ τὸν ἀποστείλαντά με.

\vs{38}Ἔφη αὐτῷ ὁ Ἰωάννης· Διδάσκαλε, εἴδομέν τινα ἐν τῷ ὀνόματί σου ἐκβάλλοντα δαιμόνια καὶ ἐκωλύομεν αὐτόν, ὅτι οὐκ ἠκολούθει ἡμῖν.
\vs{39}Ὁ δὲ Ἰησοῦς εἶπεν· Μὴ κωλύετε αὐτόν. οὐδεὶς γάρ ἐστιν ὃς ποιήσει δύναμιν ἐπὶ τῷ ὀνόματί μου καὶ δυνήσεται ταχὺ κακολογῆσαί με·
\vs{40}ὃς γὰρ οὐκ ἔστιν καθ᾽ ἡμῶν, ὑπὲρ ἡμῶν ἐστιν.

\vs{41}Ὃς γὰρ ἂν ποτίσῃ ὑμᾶς ποτήριον ὕδατος ἐν ὀνόματι ὅτι Χριστοῦ ἐστε, ἀμὴν λέγω ὑμῖν ὅτι οὐ μὴ ἀπολέσῃ τὸν μισθὸν αὐτοῦ.

\vs{42}Καὶ ὃς ἂν σκανδαλίσῃ ἕνα τῶν μικρῶν τούτων τῶν πιστευόντων εἰς ἐμέ, καλόν ἐστιν αὐτῷ μᾶλλον εἰ περίκειται μύλος ὀνικὸς περὶ τὸν τράχηλον αὐτοῦ καὶ βέβληται εἰς τὴν θάλασσαν.
\vs{43}Καὶ ἐὰν σκανδαλίζῃ σε ἡ χείρ σου, ἀπόκοψον αὐτήν· καλόν ἐστίν σε κυλλὸν εἰσελθεῖν εἰς τὴν ζωὴν ἢ τὰς δύο χεῖρας ἔχοντα ἀπελθεῖν εἰς τὴν γέενναν, εἰς τὸ πῦρ τὸ ἄσβεστον.
\vs{45}καὶ ἐὰν ὁ πούς σου σκανδαλίζῃ σε, ἀπόκοψον αὐτόν· καλόν ἐστίν σε εἰσελθεῖν εἰς τὴν ζωὴν χωλὸν ἢ τοὺς δύο πόδας ἔχοντα βληθῆναι εἰς τὴν γέενναν.
\vs{47}καὶ ἐὰν ὁ ὀφθαλμός σου σκανδαλίζῃ σε, ἔκβαλε αὐτόν· καλόν σέ ἐστιν μονόφθαλμον εἰσελθεῖν εἰς τὴν βασιλείαν τοῦ Θεοῦ ἢ δύο ὀφθαλμοὺς ἔχοντα βληθῆναι εἰς τὴν γέενναν,
\vs{48}ὅπου Ὁ σκώληξ αὐτῶν οὐ τελευτᾷ καὶ τὸ πῦρ οὐ σβέννυται.

\vs{49}Πᾶς γὰρ πυρὶ ἁλισθήσεται.
\vs{50}καλὸν τὸ ἅλας· ἐὰν δὲ τὸ ἅλας ἄναλον γένηται, ἐν τίνι αὐτὸ ἀρτύσετε; ἔχετε ἐν ἑαυτοῖς ἅλα καὶ εἰρηνεύετε ἐν ἀλλήλοις.

\ch{10}
Καὶ ἐκεῖθεν ἀναστὰς ἔρχεται εἰς τὰ ὅρια τῆς Ἰουδαίας καὶ πέραν τοῦ Ἰορδάνου, καὶ συμπορεύονται πάλιν ὄχλοι πρὸς αὐτόν, καὶ ὡς εἰώθει πάλιν ἐδίδασκεν αὐτούς.

\vs{2}Καὶ προσελθόντες Φαρισαῖοι ἐπηρώτων αὐτὸν εἰ ἔξεστιν ἀνδρὶ γυναῖκα ἀπολῦσαι, πειράζοντες αὐτόν.
\vs{3}Ὁ δὲ ἀποκριθεὶς εἶπεν αὐτοῖς· Τί ὑμῖν ἐνετείλατο Μωϋσῆς;
\vs{4}Οἱ δὲ εἶπαν· Ἐπέτρεψεν Μωϋσῆς βιβλίον ἀποστασίου γράψαι καὶ ἀπολῦσαι.
\vs{5}Ὁ δὲ Ἰησοῦς εἶπεν αὐτοῖς· Πρὸς τὴν σκληροκαρδίαν ὑμῶν ἔγραψεν ὑμῖν τὴν ἐντολὴν ταύτην.
\vs{6}ἀπὸ δὲ ἀρχῆς κτίσεως Ἄρσεν καὶ θῆλυ ἐποίησεν αὐτούς·
\vs{7}Ἕνεκεν τούτου καταλείψει ἄνθρωπος τὸν πατέρα αὐτοῦ καὶ τὴν μητέρα καὶ προσκολληθήσεται πρὸς τὴν γυναῖκα αὐτοῦ,
\vs{8}καὶ ἔσονται οἱ δύο εἰς σάρκα μίαν· ὥστε οὐκέτι εἰσὶν δύο ἀλλὰ μία σάρξ.
\vs{9}ὃ οὖν ὁ Θεὸς συνέζευξεν ἄνθρωπος μὴ χωριζέτω.

\vs{10}Καὶ εἰς τὴν οἰκίαν πάλιν οἱ μαθηταὶ περὶ τούτου ἐπηρώτων αὐτόν.
\vs{11}καὶ λέγει αὐτοῖς· Ὃς ἂν ἀπολύσῃ τὴν γυναῖκα αὐτοῦ καὶ γαμήσῃ ἄλλην μοιχᾶται ἐπ᾽ αὐτήν·
\vs{12}καὶ ἐὰν αὐτὴ ἀπολύσασα τὸν ἄνδρα αὐτῆς γαμήσῃ ἄλλον μοιχᾶται.

\vs{13}Καὶ προσέφερον αὐτῷ παιδία ἵνα αὐτῶν ἅψηται· οἱ δὲ μαθηταὶ ἐπετίμησαν αὐτοῖς.
\vs{14}Ἰδὼν δὲ ὁ Ἰησοῦς ἠγανάκτησεν καὶ εἶπεν αὐτοῖς· Ἄφετε τὰ παιδία ἔρχεσθαι πρός με, μὴ κωλύετε αὐτά, τῶν γὰρ τοιούτων ἐστὶν ἡ βασιλεία τοῦ Θεοῦ.
\vs{15}ἀμὴν λέγω ὑμῖν, ὃς ἂν μὴ δέξηται τὴν βασιλείαν τοῦ Θεοῦ ὡς παιδίον, οὐ μὴ εἰσέλθῃ εἰς αὐτήν.
\vs{16}καὶ ἐναγκαλισάμενος αὐτὰ κατευλόγει τιθεὶς τὰς χεῖρας ἐπ᾽ αὐτά.

\vs{17}Καὶ ἐκπορευομένου αὐτοῦ εἰς ὁδὸν προσδραμὼν εἷς καὶ γονυπετήσας αὐτὸν ἐπηρώτα αὐτόν· Διδάσκαλε ἀγαθέ, τί ποιήσω ἵνα ζωὴν αἰώνιον κληρονομήσω;
\vs{18}Ὁ δὲ Ἰησοῦς εἶπεν αὐτῷ· Τί με λέγεις ἀγαθόν; οὐδεὶς ἀγαθὸς εἰ μὴ εἷς ὁ Θεός.
\vs{19}τὰς ἐντολὰς οἶδας· Μὴ φονεύσῃς, Μὴ μοιχεύσῃς, Μὴ κλέψῃς, Μὴ ψευδομαρτυρήσῃς, Μὴ ἀποστερήσῃς, Τίμα τὸν πατέρα σου καὶ τὴν μητέρα.
\vs{20}Ὁ δὲ ἔφη αὐτῷ· Διδάσκαλε, ταῦτα πάντα ἐφυλαξάμην ἐκ νεότητός μου.
\vs{21}Ὁ δὲ Ἰησοῦς ἐμβλέψας αὐτῷ ἠγάπησεν αὐτὸν καὶ εἶπεν αὐτῷ· Ἕν σε ὑστερεῖ· ὕπαγε, ὅσα ἔχεις πώλησον καὶ δὸς τοῖς πτωχοῖς, καὶ ἕξεις θησαυρὸν ἐν οὐρανῷ, καὶ δεῦρο ἀκολούθει μοι.
\vs{22}Ὁ δὲ στυγνάσας ἐπὶ τῷ λόγῳ ἀπῆλθεν λυπούμενος· ἦν γὰρ ἔχων κτήματα πολλά.

\vs{23}Καὶ περιβλεψάμενος ὁ Ἰησοῦς λέγει τοῖς μαθηταῖς αὐτοῦ· Πῶς δυσκόλως οἱ τὰ χρήματα ἔχοντες εἰς τὴν βασιλείαν τοῦ Θεοῦ εἰσελεύσονται.
\vs{24}οἱ δὲ μαθηταὶ ἐθαμβοῦντο ἐπὶ τοῖς λόγοις αὐτοῦ. ὁ δὲ Ἰησοῦς πάλιν ἀποκριθεὶς λέγει αὐτοῖς· Τέκνα, πῶς δύσκολόν ἐστιν εἰς τὴν βασιλείαν τοῦ Θεοῦ εἰσελθεῖν·
\vs{25}εὐκοπώτερόν ἐστιν κάμηλον διὰ τῆς τρυμαλιᾶς τῆς ῥαφίδος διελθεῖν ἢ πλούσιον εἰς τὴν βασιλείαν τοῦ Θεοῦ εἰσελθεῖν.
\vs{26}Οἱ δὲ περισσῶς ἐξεπλήσσοντο λέγοντες πρὸς ἑαυτούς· Καὶ τίς δύναται σωθῆναι;
\vs{27}Ἐμβλέψας αὐτοῖς ὁ Ἰησοῦς λέγει· Παρὰ ἀνθρώποις ἀδύνατον, ἀλλ᾽ οὐ παρὰ θεῷ· πάντα γὰρ δυνατὰ παρὰ τῷ θεῷ.

\vs{28}Ἤρξατο λέγειν ὁ Πέτρος αὐτῷ· Ἰδοὺ ἡμεῖς ἀφήκαμεν πάντα καὶ ἠκολουθήκαμέν σοι.
\vs{29}Ἔφη ὁ Ἰησοῦς· Ἀμὴν λέγω ὑμῖν, οὐδείς ἐστιν ὃς ἀφῆκεν οἰκίαν ἢ ἀδελφοὺς ἢ ἀδελφὰς ἢ μητέρα ἢ πατέρα ἢ τέκνα ἢ ἀγροὺς ἕνεκεν ἐμοῦ καὶ ἕνεκεν τοῦ εὐαγγελίου,
\vs{30}ἐὰν μὴ λάβῃ ἑκατονταπλασίονα νῦν ἐν τῷ καιρῷ τούτῳ οἰκίας καὶ ἀδελφοὺς καὶ ἀδελφὰς καὶ μητέρας καὶ τέκνα καὶ ἀγροὺς μετὰ διωγμῶν, καὶ ἐν τῷ αἰῶνι τῷ ἐρχομένῳ ζωὴν αἰώνιον.
\vs{31}πολλοὶ δὲ ἔσονται πρῶτοι ἔσχατοι καὶ οἱ ἔσχατοι πρῶτοι.

\vs{32}Ἦσαν δὲ ἐν τῇ ὁδῷ ἀναβαίνοντες εἰς Ἱεροσόλυμα, καὶ ἦν προάγων αὐτοὺς ὁ Ἰησοῦς, καὶ ἐθαμβοῦντο, οἱ δὲ ἀκολουθοῦντες ἐφοβοῦντο. καὶ παραλαβὼν πάλιν τοὺς δώδεκα ἤρξατο αὐτοῖς λέγειν τὰ μέλλοντα αὐτῷ συμβαίνειν
\vs{33}ὅτι Ἰδοὺ ἀναβαίνομεν εἰς Ἱεροσόλυμα, καὶ ὁ Υἱὸς τοῦ ἀνθρώπου παραδοθήσεται τοῖς ἀρχιερεῦσιν καὶ τοῖς γραμματεῦσιν, καὶ κατακρινοῦσιν αὐτὸν θανάτῳ καὶ παραδώσουσιν αὐτὸν τοῖς ἔθνεσιν
\vs{34}καὶ ἐμπαίξουσιν αὐτῷ καὶ ἐμπτύσουσιν αὐτῷ καὶ μαστιγώσουσιν αὐτὸν καὶ ἀποκτενοῦσιν, καὶ μετὰ τρεῖς ἡμέρας ἀναστήσεται.

\vs{35}Καὶ προσπορεύονται αὐτῷ Ἰάκωβος καὶ Ἰωάννης οἱ υἱοὶ Ζεβεδαίου λέγοντες αὐτῷ· Διδάσκαλε, θέλομεν ἵνα ὃ ἐὰν αἰτήσωμέν σε ποιήσῃς ἡμῖν.
\vs{36}Ὁ δὲ εἶπεν αὐτοῖς· Τί θέλετε με ποιήσω ὑμῖν;
\vs{37}Οἱ δὲ εἶπαν αὐτῷ· Δὸς ἡμῖν ἵνα εἷς σου ἐκ δεξιῶν καὶ εἷς ἐξ ἀριστερῶν καθίσωμεν ἐν τῇ δόξῃ σου.
\vs{38}Ὁ δὲ Ἰησοῦς εἶπεν αὐτοῖς· Οὐκ οἴδατε τί αἰτεῖσθε. δύνασθε πιεῖν τὸ ποτήριον ὃ ἐγὼ πίνω ἢ τὸ βάπτισμα ὃ ἐγὼ βαπτίζομαι βαπτισθῆναι;
\vs{39}Οἱ δὲ εἶπαν αὐτῷ· Δυνάμεθα. Ὁ δὲ Ἰησοῦς εἶπεν αὐτοῖς· Τὸ ποτήριον ὃ ἐγὼ πίνω πίεσθε καὶ τὸ βάπτισμα ὃ ἐγὼ βαπτίζομαι βαπτισθήσεσθε,
\vs{40}τὸ δὲ καθίσαι ἐκ δεξιῶν μου ἢ ἐξ εὐωνύμων οὐκ ἔστιν ἐμὸν δοῦναι, ἀλλ᾽ οἷς ἡτοίμασται.
\vs{41}Καὶ ἀκούσαντες οἱ δέκα ἤρξαντο ἀγανακτεῖν περὶ Ἰακώβου καὶ Ἰωάννου.
\vs{42}καὶ προσκαλεσάμενος αὐτοὺς ὁ Ἰησοῦς λέγει αὐτοῖς· Οἴδατε ὅτι οἱ δοκοῦντες ἄρχειν τῶν ἐθνῶν κατακυριεύουσιν αὐτῶν καὶ οἱ μεγάλοι αὐτῶν κατεξουσιάζουσιν αὐτῶν.
\vs{43}οὐχ οὕτως δέ ἐστιν ἐν ὑμῖν, ἀλλ᾽ ὃς ἂν θέλῃ μέγας γενέσθαι ἐν ὑμῖν ἔσται ὑμῶν διάκονος,
\vs{44}καὶ ὃς ἂν θέλῃ ἐν ὑμῖν εἶναι πρῶτος ἔσται πάντων δοῦλος·
\vs{45}καὶ γὰρ ὁ Υἱὸς τοῦ ἀνθρώπου οὐκ ἦλθεν διακονηθῆναι ἀλλὰ διακονῆσαι καὶ δοῦναι τὴν ψυχὴν αὐτοῦ λύτρον ἀντὶ πολλῶν.

\vs{46}Καὶ ἔρχονται εἰς Ἰεριχώ. Καὶ ἐκπορευομένου αὐτοῦ ἀπὸ Ἰεριχὼ καὶ τῶν μαθητῶν αὐτοῦ καὶ ὄχλου ἱκανοῦ ὁ υἱὸς Τιμαίου Βαρτιμαῖος, τυφλὸς προσαίτης, ἐκάθητο παρὰ τὴν ὁδόν.
\vs{47}καὶ ἀκούσας ὅτι Ἰησοῦς ὁ Ναζαρηνός ἐστιν ἤρξατο κράζειν καὶ λέγειν· Υἱὲ Δαυὶδ Ἰησοῦ, ἐλέησόν με.
\vs{48}Καὶ ἐπετίμων αὐτῷ πολλοὶ ἵνα σιωπήσῃ· ὁ δὲ πολλῷ μᾶλλον ἔκραζεν· Υἱὲ Δαυίδ, ἐλέησόν με.
\vs{49}Καὶ στὰς ὁ Ἰησοῦς εἶπεν· Φωνήσατε αὐτόν. Καὶ φωνοῦσιν τὸν τυφλὸν λέγοντες αὐτῷ· Θάρσει, ἔγειρε, φωνεῖ σε.
\vs{50}Ὁ δὲ ἀποβαλὼν τὸ ἱμάτιον αὐτοῦ ἀναπηδήσας ἦλθεν πρὸς τὸν Ἰησοῦν.
\vs{51}Καὶ ἀποκριθεὶς αὐτῷ ὁ Ἰησοῦς εἶπεν· Τί σοι θέλεις ποιήσω; Ὁ δὲ τυφλὸς εἶπεν αὐτῷ· Ραββουνι, ἵνα ἀναβλέψω.
\vs{52}Καὶ ὁ Ἰησοῦς εἶπεν αὐτῷ· Ὕπαγε, ἡ πίστις σου σέσωκέν σε. καὶ εὐθὺς ἀνέβλεψεν καὶ ἠκολούθει αὐτῷ ἐν τῇ ὁδῷ.

\ch{11}
Καὶ ὅτε ἐγγίζουσιν εἰς Ἱεροσόλυμα εἰς Βηθφαγὴ καὶ Βηθανίαν πρὸς τὸ ὄρος τῶν Ἐλαιῶν, ἀποστέλλει δύο τῶν μαθητῶν αὐτοῦ
\vs{2}καὶ λέγει αὐτοῖς· Ὑπάγετε εἰς τὴν κώμην τὴν κατέναντι ὑμῶν, καὶ εὐθὺς εἰσπορευόμενοι εἰς αὐτὴν εὑρήσετε πῶλον δεδεμένον ἐφ᾽ ὃν οὐδεὶς οὔπω ἀνθρώπων ἐκάθισεν· λύσατε αὐτὸν καὶ φέρετε.
\vs{3}καὶ ἐάν τις ὑμῖν εἴπῃ· Τί ποιεῖτε τοῦτο; εἴπατε· Ὁ Κύριος αὐτοῦ χρείαν ἔχει, καὶ εὐθὺς αὐτὸν ἀποστέλλει πάλιν ὧδε.
\vs{4}Καὶ ἀπῆλθον καὶ εὗρον πῶλον δεδεμένον πρὸς θύραν ἔξω ἐπὶ τοῦ ἀμφόδου καὶ λύουσιν αὐτόν.
\vs{5}καί τινες τῶν ἐκεῖ ἑστηκότων ἔλεγον αὐτοῖς· Τί ποιεῖτε λύοντες τὸν πῶλον;
\vs{6}Οἱ δὲ εἶπαν αὐτοῖς καθὼς εἶπεν ὁ Ἰησοῦς, καὶ ἀφῆκαν αὐτούς.
\vs{7}καὶ φέρουσιν τὸν πῶλον πρὸς τὸν Ἰησοῦν καὶ ἐπιβάλλουσιν αὐτῷ τὰ ἱμάτια αὐτῶν, καὶ ἐκάθισεν ἐπ᾽ αὐτόν.
\vs{8}Καὶ πολλοὶ τὰ ἱμάτια αὐτῶν ἔστρωσαν εἰς τὴν ὁδόν, ἄλλοι δὲ στιβάδας κόψαντες ἐκ τῶν ἀγρῶν.
\vs{9}καὶ οἱ προάγοντες καὶ οἱ ἀκολουθοῦντες ἔκραζον· 
\begin{poetryblock}

\begin{quote}Ὡσαννά·\end{quote} 

\begin{quote}Εὐλογημένος ὁ ἐρχόμενος ἐν ὀνόματι Κυρίου·\end{quote}

\begin{quote} \vs{10}Εὐλογημένη ἡ ἐρχομένη βασιλεία τοῦ πατρὸς\end{quote} 

\begin{quote}ἡμῶν Δαυίδ·\end{quote} 

\begin{quote}Ὡσαννὰ ἐν τοῖς ὑψίστοις.\end{quote}
\end{poetryblock}

\vs{11}Καὶ εἰσῆλθεν εἰς Ἱεροσόλυμα εἰς τὸ ἱερόν καὶ περιβλεψάμενος πάντα, ὀψίας ἤδη οὔσης τῆς ὥρας, ἐξῆλθεν εἰς Βηθανίαν μετὰ τῶν δώδεκα.

\vs{12}Καὶ τῇ ἐπαύριον ἐξελθόντων αὐτῶν ἀπὸ Βηθανίας ἐπείνασεν.
\vs{13}καὶ ἰδὼν συκῆν ἀπὸ μακρόθεν ἔχουσαν φύλλα ἦλθεν, εἰ ἄρα τι εὑρήσει ἐν αὐτῇ, καὶ ἐλθὼν ἐπ᾽ αὐτὴν οὐδὲν εὗρεν εἰ μὴ φύλλα· ὁ γὰρ καιρὸς οὐκ ἦν σύκων.
\vs{14}καὶ ἀποκριθεὶς εἶπεν αὐτῇ· Μηκέτι εἰς τὸν αἰῶνα ἐκ σοῦ μηδεὶς καρπὸν φάγοι. καὶ ἤκουον οἱ μαθηταὶ αὐτοῦ.
\vs{15}Καὶ ἔρχονται εἰς Ἱεροσόλυμα. Καὶ εἰσελθὼν εἰς τὸ ἱερὸν ἤρξατο ἐκβάλλειν τοὺς πωλοῦντας καὶ τοὺς ἀγοράζοντας ἐν τῷ ἱερῷ, καὶ τὰς τραπέζας τῶν κολλυβιστῶν καὶ τὰς καθέδρας τῶν πωλούντων τὰς περιστερὰς κατέστρεψεν,
\vs{16}καὶ οὐκ ἤφιεν ἵνα τις διενέγκῃ σκεῦος διὰ τοῦ ἱεροῦ.
\vs{17}καὶ ἐδίδασκεν καὶ ἔλεγεν αὐτοῖς· Οὐ γέγραπται ὅτι 
\begin{poetryblock}

\begin{quote}Ὁ οἶκός μου οἶκος προσευχῆς κληθήσεται\end{quote} 

\begin{quote}πᾶσιν τοῖς ἔθνεσιν; ὑμεῖς δὲ πεποιήκατε αὐτὸν Σπήλαιον λῃστῶν.\end{quote}
\end{poetryblock}

\vs{18}Καὶ ἤκουσαν οἱ ἀρχιερεῖς καὶ οἱ γραμματεῖς καὶ ἐζήτουν πῶς αὐτὸν ἀπολέσωσιν· ἐφοβοῦντο γὰρ αὐτόν, πᾶς γὰρ ὁ ὄχλος ἐξεπλήσσετο ἐπὶ τῇ διδαχῇ αὐτοῦ.

\vs{19}Καὶ ὅταν ὀψὲ ἐγένετο, ἐξεπορεύοντο ἔξω τῆς πόλεως.

\vs{20}Καὶ παραπορευόμενοι πρωῒ εἶδον τὴν συκῆν ἐξηραμμένην ἐκ ῥιζῶν.
\vs{21}καὶ ἀναμνησθεὶς ὁ Πέτρος λέγει αὐτῷ· Ῥαββί, ἴδε ἡ συκῆ ἣν κατηράσω ἐξήρανται.
\vs{22}Καὶ ἀποκριθεὶς ὁ Ἰησοῦς λέγει αὐτοῖς· Ἔχετε πίστιν θεοῦ.
\vs{23}ἀμὴν λέγω ὑμῖν ὅτι ὃς ἂν εἴπῃ τῷ ὄρει τούτῳ· Ἄρθητι καὶ βλήθητι εἰς τὴν θάλασσαν, καὶ μὴ διακριθῇ ἐν τῇ καρδίᾳ αὐτοῦ ἀλλὰ πιστεύῃ ὅτι ὃ λαλεῖ γίνεται, ἔσται αὐτῷ.
\vs{24}διὰ τοῦτο λέγω ὑμῖν, πάντα ὅσα προσεύχεσθε καὶ αἰτεῖσθε, πιστεύετε ὅτι ἐλάβετε, καὶ ἔσται ὑμῖν.
\vs{25}Καὶ ὅταν στήκετε προσευχόμενοι, ἀφίετε εἴ τι ἔχετε κατά τινος, ἵνα καὶ ὁ Πατὴρ ὑμῶν ὁ ἐν τοῖς οὐρανοῖς ἀφῇ ὑμῖν τὰ παραπτώματα ὑμῶν.

\vs{27}Καὶ ἔρχονται πάλιν εἰς Ἱεροσόλυμα. καὶ ἐν τῷ ἱερῷ περιπατοῦντος αὐτοῦ ἔρχονται πρὸς αὐτὸν οἱ ἀρχιερεῖς καὶ οἱ γραμματεῖς καὶ οἱ πρεσβύτεροι
\vs{28}καὶ ἔλεγον αὐτῷ· Ἐν ποίᾳ ἐξουσίᾳ ταῦτα ποιεῖς; ἢ τίς σοι ἔδωκεν τὴν ἐξουσίαν ταύτην ἵνα ταῦτα ποιῇς;
\vs{29}Ὁ δὲ Ἰησοῦς εἶπεν αὐτοῖς· Ἐπερωτήσω ὑμᾶς ἕνα λόγον, καὶ ἀποκρίθητέ μοι καὶ ἐρῶ ὑμῖν ἐν ποίᾳ ἐξουσίᾳ ταῦτα ποιῶ·
\vs{30}τὸ βάπτισμα τὸ Ἰωάννου ἐξ οὐρανοῦ ἦν ἢ ἐξ ἀνθρώπων; ἀποκρίθητέ μοι.
\vs{31}Καὶ διελογίζοντο πρὸς ἑαυτοὺς λέγοντες· Ἐὰν εἴπωμεν· Ἐξ οὐρανοῦ, ἐρεῖ· Διὰ τί οὖν οὐκ ἐπιστεύσατε αὐτῷ;
\vs{32}ἀλλὰ εἴπωμεν· Ἐξ ἀνθρώπων;— ἐφοβοῦντο τὸν ὄχλον· ἅπαντες γὰρ εἶχον τὸν Ἰωάννην ὄντως ὅτι προφήτης ἦν.
\vs{33}καὶ ἀποκριθέντες τῷ Ἰησοῦ λέγουσιν· Οὐκ οἴδαμεν. Καὶ ὁ Ἰησοῦς λέγει αὐτοῖς· Οὐδὲ ἐγὼ λέγω ὑμῖν ἐν ποίᾳ ἐξουσίᾳ ταῦτα ποιῶ.

\ch{12}
Καὶ ἤρξατο αὐτοῖς ἐν παραβολαῖς λαλεῖν· Ἀμπελῶνα ἄνθρωπος ἐφύτευσεν καὶ περιέθηκεν φραγμὸν καὶ ὤρυξεν ὑπολήνιον καὶ ᾠκοδόμησεν πύργον καὶ ἐξέδετο αὐτὸν γεωργοῖς καὶ ἀπεδήμησεν.
\vs{2}Καὶ ἀπέστειλεν πρὸς τοὺς γεωργοὺς τῷ καιρῷ δοῦλον ἵνα παρὰ τῶν γεωργῶν λάβῃ ἀπὸ τῶν καρπῶν τοῦ ἀμπελῶνος·
\vs{3}καὶ λαβόντες αὐτὸν ἔδειραν καὶ ἀπέστειλαν κενόν.
\vs{4}Καὶ πάλιν ἀπέστειλεν πρὸς αὐτοὺς ἄλλον δοῦλον· κἀκεῖνον ἐκεφαλίωσαν καὶ ἠτίμασαν.
\vs{5}καὶ ἄλλον ἀπέστειλεν· κἀκεῖνον ἀπέκτειναν, καὶ πολλοὺς ἄλλους, οὓς μὲν δέροντες, οὓς δὲ ἀποκτέννοντες.
\vs{6}Ἔτι ἕνα εἶχεν υἱὸν ἀγαπητόν· ἀπέστειλεν αὐτὸν ἔσχατον πρὸς αὐτοὺς λέγων ὅτι Ἐντραπήσονται τὸν υἱόν μου.
\vs{7}Ἐκεῖνοι δὲ οἱ γεωργοὶ πρὸς ἑαυτοὺς εἶπαν ὅτι Οὗτός ἐστιν ὁ κληρονόμος· δεῦτε ἀποκτείνωμεν αὐτόν, καὶ ἡμῶν ἔσται ἡ κληρονομία.
\vs{8}καὶ λαβόντες ἀπέκτειναν αὐτόν καὶ ἐξέβαλον αὐτὸν ἔξω τοῦ ἀμπελῶνος.
\vs{9}Τί οὖν ποιήσει ὁ κύριος τοῦ ἀμπελῶνος; ἐλεύσεται καὶ ἀπολέσει τοὺς γεωργούς καὶ δώσει τὸν ἀμπελῶνα ἄλλοις.
\vs{10}οὐδὲ τὴν γραφὴν ταύτην ἀνέγνωτε· 
\begin{poetryblock}

\begin{quote}Λίθον ὃν ἀπεδοκίμασαν οἱ οἰκοδομοῦντες,\end{quote} 

\begin{quote}Οὗτος ἐγενήθη εἰς κεφαλὴν γωνίας·\end{quote}

\begin{quote} \vs{11}Παρὰ Κυρίου ἐγένετο αὕτη\end{quote} 

\begin{quote}Καὶ ἔστιν θαυμαστὴ ἐν ὀφθαλμοῖς ἡμῶν;\end{quote}
\end{poetryblock}

\vs{12}Καὶ ἐζήτουν αὐτὸν κρατῆσαι, καὶ ἐφοβήθησαν τὸν ὄχλον, ἔγνωσαν γὰρ ὅτι πρὸς αὐτοὺς τὴν παραβολὴν εἶπεν. καὶ ἀφέντες αὐτὸν ἀπῆλθον.

\vs{13}Καὶ ἀποστέλλουσιν πρὸς αὐτόν τινας τῶν Φαρισαίων καὶ τῶν Ἡρῳδιανῶν ἵνα αὐτὸν ἀγρεύσωσιν λόγῳ.
\vs{14}καὶ ἐλθόντες λέγουσιν αὐτῷ· Διδάσκαλε, οἴδαμεν ὅτι ἀληθὴς εἶ καὶ οὐ μέλει σοι περὶ οὐδενός· οὐ γὰρ βλέπεις εἰς πρόσωπον ἀνθρώπων, ἀλλ᾽ ἐπ᾽ ἀληθείας τὴν ὁδὸν τοῦ Θεοῦ διδάσκεις· ἔξεστιν δοῦναι κῆνσον Καίσαρι ἢ οὔ; δῶμεν ἢ μὴ δῶμεν;
\vs{15}Ὁ δὲ εἰδὼς αὐτῶν τὴν ὑπόκρισιν εἶπεν αὐτοῖς· Τί με πειράζετε; φέρετέ μοι δηνάριον ἵνα ἴδω.
\vs{16}οἱ δὲ ἤνεγκαν. καὶ λέγει αὐτοῖς· Τίνος ἡ εἰκὼν αὕτη καὶ ἡ ἐπιγραφή; Οἱ δὲ εἶπαν αὐτῷ· Καίσαρος.
\vs{17}Ὁ δὲ Ἰησοῦς εἶπεν αὐτοῖς· Τὰ Καίσαρος ἀπόδοτε Καίσαρι καὶ τὰ τοῦ Θεοῦ τῷ Θεῷ. Καὶ ἐξεθαύμαζον ἐπ᾽ αὐτῷ.

\vs{18}Καὶ ἔρχονται Σαδδουκαῖοι πρὸς αὐτόν, οἵτινες λέγουσιν ἀνάστασιν μὴ εἶναι, καὶ ἐπηρώτων αὐτὸν λέγοντες·
\vs{19}Διδάσκαλε, Μωϋσῆς ἔγραψεν ἡμῖν ὅτι ἐάν τινος ἀδελφὸς ἀποθάνῃ καὶ καταλίπῃ γυναῖκα καὶ μὴ ἀφῇ τέκνον, ἵνα λάβῃ ὁ ἀδελφὸς αὐτοῦ τὴν γυναῖκα καὶ ἐξαναστήσῃ σπέρμα τῷ ἀδελφῷ αὐτοῦ.
\vs{20}ἑπτὰ ἀδελφοὶ ἦσαν· καὶ ὁ πρῶτος ἔλαβεν γυναῖκα καὶ ἀποθνῄσκων οὐκ ἀφῆκεν σπέρμα·
\vs{21}καὶ ὁ δεύτερος ἔλαβεν αὐτήν καὶ ἀπέθανεν μὴ καταλιπὼν σπέρμα· καὶ ὁ τρίτος ὡσαύτως·
\vs{22}καὶ οἱ ἑπτὰ οὐκ ἀφῆκαν σπέρμα. ἔσχατον πάντων καὶ ἡ γυνὴ ἀπέθανεν.
\vs{23}ἐν τῇ ἀναστάσει ὅταν ἀναστῶσιν τίνος αὐτῶν ἔσται γυνή; οἱ γὰρ ἑπτὰ ἔσχον αὐτὴν γυναῖκα.
\vs{24}Ἔφη αὐτοῖς ὁ Ἰησοῦς· Οὐ διὰ τοῦτο πλανᾶσθε μὴ εἰδότες τὰς γραφὰς μηδὲ τὴν δύναμιν τοῦ Θεοῦ;
\vs{25}ὅταν γὰρ ἐκ νεκρῶν ἀναστῶσιν οὔτε γαμοῦσιν οὔτε γαμίζονται, ἀλλ᾽ εἰσὶν ὡς ἄγγελοι ἐν τοῖς οὐρανοῖς.
\vs{26}Περὶ δὲ τῶν νεκρῶν ὅτι ἐγείρονται οὐκ ἀνέγνωτε ἐν τῇ βίβλῳ Μωϋσέως ἐπὶ τοῦ Βάτου πῶς εἶπεν αὐτῷ ὁ Θεὸς λέγων· Ἐγὼ ὁ Θεὸς Ἀβραὰμ καὶ ὁ Θεὸς Ἰσαὰκ καὶ ὁ Θεὸς Ἰακώβ;
\vs{27}οὐκ ἔστιν Θεὸς νεκρῶν ἀλλὰ ζώντων· πολὺ πλανᾶσθε.

\vs{28}Καὶ προσελθὼν εἷς τῶν γραμματέων ἀκούσας αὐτῶν συζητούντων, ἰδὼν ὅτι καλῶς ἀπεκρίθη αὐτοῖς ἐπηρώτησεν αὐτόν· Ποία ἐστὶν ἐντολὴ πρώτη πάντων;
\vs{29}Ἀπεκρίθη ὁ Ἰησοῦς Ὅτι Πρώτη ἐστίν· Ἄκουε, Ἰσραήλ, Κύριος ὁ Θεὸς ἡμῶν Κύριος εἷς ἐστιν,
\vs{30}καὶ ἀγαπήσεις Κύριον τὸν Θεόν σου ἐξ ὅλης τῆς καρδίας σου καὶ ἐξ ὅλης τῆς ψυχῆς σου καὶ ἐξ ὅλης τῆς διανοίας σου καὶ ἐξ ὅλης τῆς ἰσχύος σου.
\vs{31}δευτέρα αὕτη· Ἀγαπήσεις τὸν πλησίον σου ὡς σεαυτόν. μείζων τούτων ἄλλη ἐντολὴ οὐκ ἔστιν.
\vs{32}Καὶ εἶπεν αὐτῷ ὁ γραμματεύς· Καλῶς, Διδάσκαλε, ἐπ᾽ ἀληθείας εἶπες ὅτι εἷς ἐστιν καὶ οὐκ ἔστιν ἄλλος πλὴν αὐτοῦ·
\vs{33}καὶ τὸ ἀγαπᾶν αὐτὸν ἐξ ὅλης τῆς καρδίας καὶ ἐξ ὅλης τῆς συνέσεως καὶ ἐξ ὅλης τῆς ἰσχύος καὶ τὸ ἀγαπᾶν τὸν πλησίον ὡς ἑαυτὸν περισσότερόν ἐστιν πάντων τῶν ὁλοκαυτωμάτων καὶ θυσιῶν.
\vs{34}Καὶ ὁ Ἰησοῦς ἰδὼν αὐτὸν ὅτι νουνεχῶς ἀπεκρίθη εἶπεν αὐτῷ· Οὐ μακρὰν εἶ ἀπὸ τῆς βασιλείας τοῦ Θεοῦ. καὶ οὐδεὶς οὐκέτι ἐτόλμα αὐτὸν ἐπερωτῆσαι.

\vs{35}Καὶ ἀποκριθεὶς ὁ Ἰησοῦς ἔλεγεν διδάσκων ἐν τῷ ἱερῷ· Πῶς λέγουσιν οἱ γραμματεῖς ὅτι ὁ Χριστὸς υἱὸς Δαυίδ ἐστιν;
\vs{36}αὐτὸς Δαυὶδ εἶπεν ἐν τῷ Πνεύματι τῷ Ἁγίῳ· 
\begin{poetryblock}

\begin{quote}Εἶπεν Κύριος τῷ Κυρίῳ μου·\end{quote} 

\begin{quote}Κάθου ἐκ δεξιῶν μου,\end{quote} 

\begin{quote}Ἕως ἂν θῶ τοὺς ἐχθρούς σου\end{quote} 

\begin{quote}Ὑποκάτω τῶν ποδῶν σου.\end{quote}
\end{poetryblock}

\vs{37}Αὐτὸς Δαυὶδ λέγει αὐτὸν Κύριον, καὶ πόθεν αὐτοῦ ἐστιν υἱός;

Καὶ ὁ πολὺς ὄχλος ἤκουεν αὐτοῦ ἡδέως.

\vs{38}Καὶ ἐν τῇ διδαχῇ αὐτοῦ ἔλεγεν· Βλέπετε ἀπὸ τῶν γραμματέων τῶν θελόντων ἐν στολαῖς περιπατεῖν καὶ ἀσπασμοὺς ἐν ταῖς ἀγοραῖς
\vs{39}καὶ πρωτοκαθεδρίας ἐν ταῖς συναγωγαῖς καὶ πρωτοκλισίας ἐν τοῖς δείπνοις,
\vs{40}οἱ κατεσθίοντες τὰς οἰκίας τῶν χηρῶν καὶ προφάσει μακρὰ προσευχόμενοι· οὗτοι λήμψονται περισσότερον κρίμα.

\vs{41}Καὶ καθίσας κατέναντι τοῦ γαζοφυλακίου ἐθεώρει πῶς ὁ ὄχλος βάλλει χαλκὸν εἰς τὸ γαζοφυλάκιον. καὶ πολλοὶ πλούσιοι ἔβαλλον πολλά·
\vs{42}καὶ ἐλθοῦσα μία χήρα πτωχὴ ἔβαλεν λεπτὰ δύο, ὅ ἐστιν κοδράντης.
\vs{43}Καὶ προσκαλεσάμενος τοὺς μαθητὰς αὐτοῦ εἶπεν αὐτοῖς· Ἀμὴν λέγω ὑμῖν ὅτι ἡ χήρα αὕτη ἡ πτωχὴ πλεῖον πάντων ἔβαλεν τῶν βαλλόντων εἰς τὸ γαζοφυλάκιον·
\vs{44}πάντες γὰρ ἐκ τοῦ περισσεύοντος αὐτοῖς ἔβαλον, αὕτη δὲ ἐκ τῆς ὑστερήσεως αὐτῆς πάντα ὅσα εἶχεν ἔβαλεν ὅλον τὸν βίον αὐτῆς.

\ch{13}
Καὶ ἐκπορευομένου αὐτοῦ ἐκ τοῦ ἱεροῦ λέγει αὐτῷ εἷς τῶν μαθητῶν αὐτοῦ· Διδάσκαλε, ἴδε ποταποὶ λίθοι καὶ ποταπαὶ οἰκοδομαί.
\vs{2}Καὶ ὁ Ἰησοῦς εἶπεν αὐτῷ· Βλέπεις ταύτας τὰς μεγάλας οἰκοδομάς; οὐ μὴ ἀφεθῇ ὧδε λίθος ἐπὶ λίθον ὃς οὐ μὴ καταλυθῇ.
\vs{3}Καὶ καθημένου αὐτοῦ εἰς τὸ ὄρος τῶν Ἐλαιῶν κατέναντι τοῦ ἱεροῦ ἐπηρώτα αὐτὸν κατ᾽ ἰδίαν Πέτρος καὶ Ἰάκωβος καὶ Ἰωάννης καὶ Ἀνδρέας·
\vs{4}Εἰπὸν ἡμῖν, πότε ταῦτα ἔσται καὶ τί τὸ σημεῖον ὅταν μέλλῃ ταῦτα συντελεῖσθαι πάντα;
\vs{5}Ὁ δὲ Ἰησοῦς ἤρξατο λέγειν αὐτοῖς· Βλέπετε μή τις ὑμᾶς πλανήσῃ·
\vs{6}πολλοὶ ἐλεύσονται ἐπὶ τῷ ὀνόματί μου λέγοντες ὅτι Ἐγώ εἰμι, καὶ πολλοὺς πλανήσουσιν.
\vs{7}ὅταν δὲ ἀκούσητε πολέμους καὶ ἀκοὰς πολέμων, μὴ θροεῖσθε· δεῖ γενέσθαι, ἀλλ᾽ οὔπω τὸ τέλος.
\vs{8}ἐγερθήσεται γὰρ ἔθνος ἐπ᾽ ἔθνος καὶ βασιλεία ἐπὶ βασιλείαν, ἔσονται σεισμοὶ κατὰ τόπους, ἔσονται λιμοί· ἀρχὴ ὠδίνων ταῦτα.

\vs{9}Βλέπετε δὲ ὑμεῖς ἑαυτούς· παραδώσουσιν ὑμᾶς εἰς συνέδρια καὶ εἰς συναγωγὰς δαρήσεσθε καὶ ἐπὶ ἡγεμόνων καὶ βασιλέων σταθήσεσθε ἕνεκεν ἐμοῦ εἰς μαρτύριον αὐτοῖς.
\vs{10}καὶ εἰς πάντα τὰ ἔθνη πρῶτον δεῖ κηρυχθῆναι τὸ εὐαγγέλιον.
\vs{11}καὶ ὅταν ἄγωσιν ὑμᾶς παραδιδόντες, μὴ προμεριμνᾶτε τί λαλήσητε, ἀλλ᾽ ὃ ἐὰν δοθῇ ὑμῖν ἐν ἐκείνῃ τῇ ὥρᾳ τοῦτο λαλεῖτε· οὐ γάρ ἐστε ὑμεῖς οἱ λαλοῦντες ἀλλὰ τὸ Πνεῦμα τὸ Ἅγιον.
\vs{12}Καὶ παραδώσει ἀδελφὸς ἀδελφὸν εἰς θάνατον καὶ πατὴρ τέκνον, καὶ ἐπαναστήσονται τέκνα ἐπὶ γονεῖς καὶ θανατώσουσιν αὐτούς·
\vs{13}καὶ ἔσεσθε μισούμενοι ὑπὸ πάντων διὰ τὸ ὄνομά μου. ὁ δὲ ὑπομείνας εἰς τέλος οὗτος σωθήσεται.

\vs{14}Ὅταν δὲ ἴδητε τὸ βδέλυγμα τῆς ἐρημώσεως ἑστηκότα ὅπου οὐ δεῖ, ὁ ἀναγινώσκων νοείτω, τότε οἱ ἐν τῇ Ἰουδαίᾳ φευγέτωσαν εἰς τὰ ὄρη,
\vs{15}ὁ δὲ ἐπὶ τοῦ δώματος μὴ καταβάτω μηδὲ εἰσελθάτω ἆραι τι ἐκ τῆς οἰκίας αὐτοῦ,
\vs{16}καὶ ὁ εἰς τὸν ἀγρὸν μὴ ἐπιστρεψάτω εἰς τὰ ὀπίσω ἆραι τὸ ἱμάτιον αὐτοῦ.
\vs{17}Οὐαὶ δὲ ταῖς ἐν γαστρὶ ἐχούσαις καὶ ταῖς θηλαζούσαις ἐν ἐκείναις ταῖς ἡμέραις.

\vs{18}προσεύχεσθε δὲ ἵνα μὴ γένηται χειμῶνος·
\vs{19}ἔσονται γὰρ αἱ ἡμέραι ἐκεῖναι θλῖψις οἵα οὐ γέγονεν τοιαύτη ἀπ᾽ ἀρχῆς κτίσεως ἣν ἔκτισεν ὁ Θεὸς ἕως τοῦ νῦν καὶ οὐ μὴ γένηται.
\vs{20}καὶ εἰ μὴ ἐκολόβωσεν Κύριος τὰς ἡμέρας, οὐκ ἂν ἐσώθη πᾶσα σάρξ· ἀλλὰ διὰ τοὺς ἐκλεκτοὺς οὓς ἐξελέξατο ἐκολόβωσεν τὰς ἡμέρας.

\vs{21}Καὶ τότε ἐάν τις ὑμῖν εἴπῃ· Ἴδε ὧδε ὁ Χριστός, Ἴδε ἐκεῖ, μὴ πιστεύετε·
\vs{22}ἐγερθήσονται γὰρ ψευδόχριστοι καὶ ψευδοπροφῆται καὶ δώσουσιν σημεῖα καὶ τέρατα πρὸς τὸ ἀποπλανᾶν, εἰ δυνατὸν, τοὺς ἐκλεκτούς.
\vs{23}ὑμεῖς δὲ βλέπετε· προείρηκα ὑμῖν πάντα.

\vs{24}Ἀλλὰ ἐν ἐκείναις ταῖς ἡμέραις μετὰ τὴν θλῖψιν ἐκείνην 
\begin{poetryblock}

\begin{quote}Ὁ ἥλιος σκοτισθήσεται,\end{quote} 

\begin{quote}Καὶ ἡ σελήνη οὐ δώσει τὸ φέγγος αὐτῆς,\end{quote}

\begin{quote} \vs{25}Καὶ οἱ ἀστέρες ἔσονται ἐκ τοῦ οὐρανοῦ πίπτοντες,\end{quote} 

\begin{quote}Καὶ αἱ δυνάμεις αἱ ἐν τοῖς οὐρανοῖς σαλευθήσονται.\end{quote}
\end{poetryblock}

\vs{26}Καὶ τότε ὄψονται τὸν Υἱὸν τοῦ ἀνθρώπου ἐρχόμενον ἐν νεφέλαις μετὰ δυνάμεως πολλῆς καὶ δόξης.
\vs{27}καὶ τότε ἀποστελεῖ τοὺς ἀγγέλους καὶ ἐπισυνάξει τοὺς ἐκλεκτοὺς αὐτοῦ ἐκ τῶν τεσσάρων ἀνέμων ἀπ᾽ ἄκρου γῆς ἕως ἄκρου οὐρανοῦ.

\vs{28}Ἀπὸ δὲ τῆς συκῆς μάθετε τὴν παραβολήν· ὅταν ἤδη ὁ κλάδος αὐτῆς ἁπαλὸς γένηται καὶ ἐκφύῃ τὰ φύλλα, γινώσκετε ὅτι ἐγγὺς τὸ θέρος ἐστίν·
\vs{29}οὕτως καὶ ὑμεῖς, ὅταν ἴδητε ταῦτα γινόμενα, γινώσκετε ὅτι ἐγγύς ἐστιν ἐπὶ θύραις.

\vs{30}ἀμὴν λέγω ὑμῖν ὅτι οὐ μὴ παρέλθῃ ἡ γενεὰ αὕτη μέχρις οὗ ταῦτα πάντα γένηται.
\vs{31}ὁ οὐρανὸς καὶ ἡ γῆ παρελεύσονται, οἱ δὲ λόγοι μου οὐ μὴ παρελεύσονται.

\vs{32}Περὶ δὲ τῆς ἡμέρας ἐκείνης ἢ τῆς ὥρας οὐδεὶς οἶδεν, οὐδὲ οἱ ἄγγελοι ἐν οὐρανῷ οὐδὲ ὁ Υἱός, εἰ μὴ ὁ Πατήρ.

\vs{33}Βλέπετε, ἀγρυπνεῖτε· οὐκ οἴδατε γὰρ πότε ὁ καιρός ἐστιν.
\vs{34}Ὡς ἄνθρωπος ἀπόδημος ἀφεὶς τὴν οἰκίαν αὐτοῦ καὶ δοὺς τοῖς δούλοις αὐτοῦ τὴν ἐξουσίαν ἑκάστῳ τὸ ἔργον αὐτοῦ καὶ τῷ θυρωρῷ ἐνετείλατο ἵνα γρηγορῇ.
\vs{35}γρηγορεῖτε οὖν· οὐκ οἴδατε γὰρ πότε ὁ κύριος τῆς οἰκίας ἔρχεται, ἢ ὀψὲ ἢ μεσονύκτιον ἢ ἀλεκτοροφωνίας ἢ πρωΐ,
\vs{36}μὴ ἐλθὼν ἐξαίφνης εὕρῃ ὑμᾶς καθεύδοντας.
\vs{37}ὃ δὲ ὑμῖν λέγω πᾶσιν λέγω, γρηγορεῖτε.

\ch{14}
Ἦν δὲ τὸ πάσχα καὶ τὰ ἄζυμα μετὰ δύο ἡμέρας. καὶ ἐζήτουν οἱ ἀρχιερεῖς καὶ οἱ γραμματεῖς πῶς αὐτὸν ἐν δόλῳ κρατήσαντες ἀποκτείνωσιν·
\vs{2}ἔλεγον γάρ· Μὴ ἐν τῇ ἑορτῇ, μήποτε ἔσται θόρυβος τοῦ λαοῦ.

\vs{3}Καὶ ὄντος αὐτοῦ ἐν Βηθανίᾳ ἐν τῇ οἰκίᾳ Σίμωνος τοῦ λεπροῦ, κατακειμένου αὐτοῦ ἦλθεν γυνὴ ἔχουσα ἀλάβαστρον μύρου νάρδου πιστικῆς πολυτελοῦς, συντρίψασα τὴν ἀλάβαστρον κατέχεεν αὐτοῦ τῆς κεφαλῆς.
\vs{4}Ἦσαν δέ τινες ἀγανακτοῦντες πρὸς ἑαυτούς· Εἰς τί ἡ ἀπώλεια αὕτη τοῦ μύρου γέγονεν;
\vs{5}ἠδύνατο γὰρ τοῦτο τὸ μύρον πραθῆναι ἐπάνω δηναρίων τριακοσίων καὶ δοθῆναι τοῖς πτωχοῖς· καὶ ἐνεβριμῶντο αὐτῇ.
\vs{6}Ὁ δὲ Ἰησοῦς εἶπεν· Ἄφετε αὐτήν· τί αὐτῇ κόπους παρέχετε; καλὸν ἔργον ἠργάσατο ἐν ἐμοί.
\vs{7}πάντοτε γὰρ τοὺς πτωχοὺς ἔχετε μεθ᾽ ἑαυτῶν καὶ ὅταν θέλητε δύνασθε αὐτοῖς εὖ ποιῆσαι, ἐμὲ δὲ οὐ πάντοτε ἔχετε.
\vs{8}ὃ ἔσχεν ἐποίησεν· προέλαβεν μυρίσαι τὸ σῶμά μου εἰς τὸν ἐνταφιασμόν.
\vs{9}ἀμὴν δὲ λέγω ὑμῖν, ὅπου ἐὰν κηρυχθῇ τὸ εὐαγγέλιον εἰς ὅλον τὸν κόσμον, καὶ ὃ ἐποίησεν αὕτη λαληθήσεται εἰς μνημόσυνον αὐτῆς.

\vs{10}Καὶ Ἰούδας Ἰσκαριὼθ ὁ εἷς τῶν δώδεκα ἀπῆλθεν πρὸς τοὺς ἀρχιερεῖς ἵνα αὐτὸν παραδοῖ αὐτοῖς.
\vs{11}οἱ δὲ ἀκούσαντες ἐχάρησαν καὶ ἐπηγγείλαντο αὐτῷ ἀργύριον δοῦναι. καὶ ἐζήτει πῶς αὐτὸν εὐκαίρως παραδοῖ.

\vs{12}Καὶ τῇ πρώτῃ ἡμέρᾳ τῶν ἀζύμων, ὅτε τὸ πάσχα ἔθυον, λέγουσιν αὐτῷ οἱ μαθηταὶ αὐτοῦ· Ποῦ θέλεις ἀπελθόντες ἑτοιμάσωμεν ἵνα φάγῃς τὸ πάσχα;
\vs{13}Καὶ ἀποστέλλει δύο τῶν μαθητῶν αὐτοῦ καὶ λέγει αὐτοῖς· Ὑπάγετε εἰς τὴν πόλιν, καὶ ἀπαντήσει ὑμῖν ἄνθρωπος κεράμιον ὕδατος βαστάζων· ἀκολουθήσατε αὐτῷ
\vs{14}καὶ ὅπου ἐὰν εἰσέλθῃ εἴπατε τῷ οἰκοδεσπότῃ ὅτι Ὁ Διδάσκαλος λέγει· Ποῦ ἐστιν τὸ κατάλυμά μου ὅπου τὸ πάσχα μετὰ τῶν μαθητῶν μου φάγω;
\vs{15}καὶ αὐτὸς ὑμῖν δείξει ἀνάγαιον μέγα ἐστρωμένον ἕτοιμον· καὶ ἐκεῖ ἑτοιμάσατε ἡμῖν.
\vs{16}Καὶ ἐξῆλθον οἱ μαθηταὶ καὶ ἦλθον εἰς τὴν πόλιν καὶ εὗρον καθὼς εἶπεν αὐτοῖς καὶ ἡτοίμασαν τὸ πάσχα.

\vs{17}Καὶ ὀψίας γενομένης ἔρχεται μετὰ τῶν δώδεκα.
\vs{18}Καὶ ἀνακειμένων αὐτῶν καὶ ἐσθιόντων ὁ Ἰησοῦς εἶπεν· Ἀμὴν λέγω ὑμῖν ὅτι εἷς ἐξ ὑμῶν παραδώσει με ὁ ἐσθίων μετ᾽ ἐμοῦ.
\vs{19}ἤρξαντο λυπεῖσθαι καὶ λέγειν αὐτῷ εἷς κατὰ εἷς· Μήτι ἐγώ;
\vs{20}Ὁ δὲ εἶπεν αὐτοῖς· Εἷς τῶν δώδεκα, ὁ ἐμβαπτόμενος μετ᾽ ἐμοῦ εἰς τὸ τρύβλιον.
\vs{21}ὅτι ὁ μὲν Υἱὸς τοῦ ἀνθρώπου ὑπάγει καθὼς γέγραπται περὶ αὐτοῦ, οὐαὶ δὲ τῷ ἀνθρώπῳ ἐκείνῳ δι᾽ οὗ ὁ Υἱὸς τοῦ ἀνθρώπου παραδίδοται· καλὸν αὐτῷ εἰ οὐκ ἐγεννήθη ὁ ἄνθρωπος ἐκεῖνος.

\vs{22}Καὶ ἐσθιόντων αὐτῶν λαβὼν ἄρτον εὐλογήσας ἔκλασεν καὶ ἔδωκεν αὐτοῖς καὶ εἶπεν· Λάβετε, τοῦτό ἐστιν τὸ σῶμά μου.
\vs{23}Καὶ λαβὼν ποτήριον εὐχαριστήσας ἔδωκεν αὐτοῖς, καὶ ἔπιον ἐξ αὐτοῦ πάντες.
\vs{24}καὶ εἶπεν αὐτοῖς· Τοῦτό ἐστιν τὸ αἷμά μου τῆς διαθήκης τὸ ἐκχυννόμενον ὑπὲρ πολλῶν.
\vs{25}ἀμὴν λέγω ὑμῖν ὅτι οὐκέτι οὐ μὴ πίω ἐκ τοῦ γενήματος τῆς ἀμπέλου ἕως τῆς ἡμέρας ἐκείνης ὅταν αὐτὸ πίνω καινὸν ἐν τῇ βασιλείᾳ τοῦ Θεοῦ.

\vs{26}Καὶ ὑμνήσαντες ἐξῆλθον εἰς τὸ ὄρος τῶν Ἐλαιῶν.
\vs{27}Καὶ λέγει αὐτοῖς ὁ Ἰησοῦς ὅτι Πάντες σκανδαλισθήσεσθε, ὅτι γέγραπται· 
\begin{poetryblock}

\begin{quote}Πατάξω τὸν ποιμένα,\end{quote} 

\begin{quote}Καὶ τὰ πρόβατα διασκορπισθήσονται.\end{quote}
\end{poetryblock}

\vs{28}Ἀλλὰ μετὰ τὸ ἐγερθῆναί με προάξω ὑμᾶς εἰς τὴν Γαλιλαίαν.
\vs{29}Ὁ δὲ Πέτρος ἔφη αὐτῷ· Εἰ καὶ πάντες σκανδαλισθήσονται, ἀλλ᾽ οὐκ ἐγώ.
\vs{30}Καὶ λέγει αὐτῷ ὁ Ἰησοῦς· Ἀμὴν λέγω σοι ὅτι σὺ σήμερον ταύτῃ τῇ νυκτὶ πρὶν ἢ δὶς ἀλέκτορα φωνῆσαι τρίς με ἀπαρνήσῃ.
\vs{31}Ὁ δὲ ἐκπερισσῶς ἐλάλει· Ἐὰν δέῃ με συναποθανεῖν σοι, οὐ μή σε ἀπαρνήσομαι. ὡσαύτως δὲ καὶ πάντες ἔλεγον.

\vs{32}Καὶ ἔρχονται εἰς χωρίον οὗ τὸ ὄνομα Γεθσημανί καὶ λέγει τοῖς μαθηταῖς αὐτοῦ· Καθίσατε ὧδε ἕως προσεύξωμαι.
\vs{33}καὶ παραλαμβάνει τὸν Πέτρον καὶ τὸν Ἰάκωβον καὶ τὸν Ἰωάννην μετ᾽ αὐτοῦ καὶ ἤρξατο ἐκθαμβεῖσθαι καὶ ἀδημονεῖν
\vs{34}καὶ λέγει αὐτοῖς· Περίλυπός ἐστιν ἡ ψυχή μου ἕως θανάτου· μείνατε ὧδε καὶ γρηγορεῖτε.
\vs{35}Καὶ προελθὼν μικρὸν ἔπιπτεν ἐπὶ τῆς γῆς καὶ προσηύχετο ἵνα εἰ δυνατόν ἐστιν παρέλθῃ ἀπ᾽ αὐτοῦ ἡ ὥρα,
\vs{36}καὶ ἔλεγεν· Ἀββᾶ ὁ Πατήρ, πάντα δυνατά σοι· παρένεγκε τὸ ποτήριον τοῦτο ἀπ᾽ ἐμοῦ· ἀλλ᾽ οὐ τί ἐγὼ θέλω ἀλλὰ τί σύ.
\vs{37}Καὶ ἔρχεται καὶ εὑρίσκει αὐτοὺς καθεύδοντας, καὶ λέγει τῷ Πέτρῳ· Σίμων, καθεύδεις; οὐκ ἴσχυσας μίαν ὥραν γρηγορῆσαι;
\vs{38}γρηγορεῖτε καὶ προσεύχεσθε, ἵνα μὴ ἔλθητε εἰς πειρασμόν· τὸ μὲν πνεῦμα πρόθυμον ἡ δὲ σὰρξ ἀσθενής.
\vs{39}Καὶ πάλιν ἀπελθὼν προσηύξατο τὸν αὐτὸν λόγον εἰπών.
\vs{40}καὶ πάλιν ἐλθὼν εὗρεν αὐτοὺς καθεύδοντας, ἦσαν γὰρ αὐτῶν οἱ ὀφθαλμοὶ καταβαρυνόμενοι, καὶ οὐκ ᾔδεισαν τί ἀποκριθῶσιν αὐτῷ.
\vs{41}Καὶ ἔρχεται τὸ τρίτον καὶ λέγει αὐτοῖς· Καθεύδετε τὸ λοιπὸν καὶ ἀναπαύεσθε· ἀπέχει· ἦλθεν ἡ ὥρα, ἰδοὺ παραδίδοται ὁ Υἱὸς τοῦ ἀνθρώπου εἰς τὰς χεῖρας τῶν ἁμαρτωλῶν.
\vs{42}ἐγείρεσθε ἄγωμεν· ἰδοὺ ὁ παραδιδούς με ἤγγικεν.

\vs{43}Καὶ εὐθὺς ἔτι αὐτοῦ λαλοῦντος παραγίνεται Ἰούδας εἷς τῶν δώδεκα καὶ μετ᾽ αὐτοῦ ὄχλος μετὰ μαχαιρῶν καὶ ξύλων παρὰ τῶν ἀρχιερέων καὶ τῶν γραμματέων καὶ τῶν πρεσβυτέρων.
\vs{44}Δεδώκει δὲ ὁ παραδιδοὺς αὐτὸν σύσσημον αὐτοῖς λέγων· Ὃν ἂν φιλήσω αὐτός ἐστιν, κρατήσατε αὐτὸν καὶ ἀπάγετε ἀσφαλῶς.
\vs{45}καὶ ἐλθὼν εὐθὺς προσελθὼν αὐτῷ λέγει· Ῥαββί, καὶ κατεφίλησεν αὐτόν·
\vs{46}Οἱ δὲ ἐπέβαλαν τὰς χεῖρας αὐτῷ καὶ ἐκράτησαν αὐτόν.
\vs{47}εἷς δέ τις τῶν παρεστηκότων σπασάμενος τὴν μάχαιραν ἔπαισεν τὸν δοῦλον τοῦ ἀρχιερέως καὶ ἀφεῖλεν αὐτοῦ τὸ ὠτάριον.

\vs{48}Καὶ ἀποκριθεὶς ὁ Ἰησοῦς εἶπεν αὐτοῖς· Ὡς ἐπὶ λῃστὴν ἐξήλθατε μετὰ μαχαιρῶν καὶ ξύλων συλλαβεῖν με;
\vs{49}καθ᾽ ἡμέραν ἤμην πρὸς ὑμᾶς ἐν τῷ ἱερῷ διδάσκων καὶ οὐκ ἐκρατήσατέ με· ἀλλ᾽ ἵνα πληρωθῶσιν αἱ γραφαί.
\vs{50}Καὶ ἀφέντες αὐτὸν ἔφυγον πάντες.
\vs{51}Καὶ νεανίσκος τις συνηκολούθει αὐτῷ περιβεβλημένος σινδόνα ἐπὶ γυμνοῦ, καὶ κρατοῦσιν αὐτόν·
\vs{52}ὁ δὲ καταλιπὼν τὴν σινδόνα γυμνὸς ἔφυγεν.
\vs{53}Καὶ ἀπήγαγον τὸν Ἰησοῦν πρὸς τὸν ἀρχιερέα, καὶ συνέρχονται πάντες οἱ ἀρχιερεῖς καὶ οἱ πρεσβύτεροι καὶ οἱ γραμματεῖς.
\vs{54}καὶ ὁ Πέτρος ἀπὸ μακρόθεν ἠκολούθησεν αὐτῷ ἕως ἔσω εἰς τὴν αὐλὴν τοῦ ἀρχιερέως καὶ ἦν συνκαθήμενος μετὰ τῶν ὑπηρετῶν καὶ θερμαινόμενος πρὸς τὸ φῶς.

\vs{55}Οἱ δὲ ἀρχιερεῖς καὶ ὅλον τὸ συνέδριον ἐζήτουν κατὰ τοῦ Ἰησοῦ μαρτυρίαν εἰς τὸ θανατῶσαι αὐτόν, καὶ οὐχ ηὕρισκον·
\vs{56}πολλοὶ γὰρ ἐψευδομαρτύρουν κατ᾽ αὐτοῦ, καὶ ἴσαι αἱ μαρτυρίαι οὐκ ἦσαν.
\vs{57}Καί τινες ἀναστάντες ἐψευδομαρτύρουν κατ᾽ αὐτοῦ λέγοντες
\vs{58}Ὅτι Ἡμεῖς ἠκούσαμεν αὐτοῦ λέγοντος ὅτι Ἐγὼ καταλύσω τὸν ναὸν τοῦτον τὸν χειροποίητον καὶ διὰ τριῶν ἡμερῶν ἄλλον ἀχειροποίητον οἰκοδομήσω.
\vs{59}καὶ οὐδὲ οὕτως ἴση ἦν ἡ μαρτυρία αὐτῶν.
\vs{60}Καὶ ἀναστὰς ὁ ἀρχιερεὺς εἰς μέσον ἐπηρώτησεν τὸν Ἰησοῦν λέγων· Οὐκ ἀποκρίνῃ οὐδέν τί οὗτοί σου καταμαρτυροῦσιν;
\vs{61}Ὁ δὲ ἐσιώπα καὶ οὐκ ἀπεκρίνατο οὐδέν. Πάλιν ὁ ἀρχιερεὺς ἐπηρώτα αὐτὸν καὶ λέγει αὐτῷ· Σὺ εἶ ὁ Χριστὸς ὁ Υἱὸς τοῦ Εὐλογητοῦ;
\vs{62}Ὁ δὲ Ἰησοῦς εἶπεν· Ἐγώ εἰμι, καὶ ὄψεσθε τὸν Υἱὸν τοῦ ἀνθρώπου ἐκ δεξιῶν καθήμενον τῆς δυνάμεως καὶ ἐρχόμενον μετὰ τῶν νεφελῶν τοῦ οὐρανοῦ.
\vs{63}Ὁ δὲ ἀρχιερεὺς διαρρήξας τοὺς χιτῶνας αὐτοῦ λέγει· Τί ἔτι χρείαν ἔχομεν μαρτύρων;
\vs{64}ἠκούσατε τῆς βλασφημίας· τί ὑμῖν φαίνεται; Οἱ δὲ πάντες κατέκριναν αὐτὸν ἔνοχον εἶναι θανάτου.

\vs{65}Καὶ ἤρξαντό τινες ἐμπτύειν αὐτῷ καὶ περικαλύπτειν αὐτοῦ τὸ πρόσωπον καὶ κολαφίζειν αὐτὸν καὶ λέγειν αὐτῷ· Προφήτευσον, καὶ οἱ ὑπηρέται ῥαπίσμασιν αὐτὸν ἔλαβον.

\vs{66}Καὶ ὄντος τοῦ Πέτρου κάτω ἐν τῇ αὐλῇ ἔρχεται μία τῶν παιδισκῶν τοῦ ἀρχιερέως
\vs{67}καὶ ἰδοῦσα τὸν Πέτρον θερμαινόμενον ἐμβλέψασα αὐτῷ λέγει· Καὶ σὺ μετὰ τοῦ Ναζαρηνοῦ ἦσθα τοῦ Ἰησοῦ.
\vs{68}Ὁ δὲ ἠρνήσατο λέγων· Οὔτε οἶδα οὔτε ἐπίσταμαι σὺ τί λέγεις. καὶ ἐξῆλθεν ἔξω εἰς τὸ προαύλιον καὶ ἀλέκτωρ ἐφώνησεν.

\vs{69}Καὶ ἡ παιδίσκη ἰδοῦσα αὐτὸν ἤρξατο πάλιν λέγειν τοῖς παρεστῶσιν ὅτι Οὗτος ἐξ αὐτῶν ἐστιν.
\vs{70}Ὁ δὲ πάλιν ἠρνεῖτο. Καὶ μετὰ μικρὸν πάλιν οἱ παρεστῶτες ἔλεγον τῷ Πέτρῳ· Ἀληθῶς ἐξ αὐτῶν εἶ, καὶ γὰρ Γαλιλαῖος εἶ.
\vs{71}Ὁ δὲ ἤρξατο ἀναθεματίζειν καὶ ὀμνύναι ὅτι Οὐκ οἶδα τὸν ἄνθρωπον τοῦτον ὃν λέγετε.
\vs{72}καὶ εὐθὺς ἐκ δευτέρου ἀλέκτωρ ἐφώνησεν. Καὶ ἀνεμνήσθη ὁ Πέτρος τὸ ῥῆμα ὡς εἶπεν αὐτῷ ὁ Ἰησοῦς ὅτι Πρὶν ἀλέκτορα φωνῆσαι δὶς τρίς με ἀπαρνήσῃ· καὶ ἐπιβαλὼν ἔκλαιεν.

\ch{15}
Καὶ εὐθὺς πρωῒ συμβούλιον ποιήσαντες οἱ ἀρχιερεῖς μετὰ τῶν πρεσβυτέρων καὶ γραμματέων καὶ ὅλον τὸ συνέδριον, δήσαντες τὸν Ἰησοῦν ἀπήνεγκαν καὶ παρέδωκαν Πιλάτῳ.

\vs{2}Καὶ ἐπηρώτησεν αὐτὸν ὁ Πιλᾶτος· Σὺ εἶ ὁ Βασιλεὺς τῶν Ἰουδαίων; Ὁ δὲ ἀποκριθεὶς αὐτῷ λέγει· Σὺ λέγεις.
\vs{3}Καὶ κατηγόρουν αὐτοῦ οἱ ἀρχιερεῖς πολλά.
\vs{4}Ὁ δὲ Πιλᾶτος πάλιν ἐπηρώτα αὐτὸν λέγων· Οὐκ ἀποκρίνῃ οὐδέν; ἴδε πόσα σου κατηγοροῦσιν.
\vs{5}Ὁ δὲ Ἰησοῦς οὐκέτι οὐδὲν ἀπεκρίθη, ὥστε θαυμάζειν τὸν Πιλᾶτον.

\vs{6}Κατὰ δὲ ἑορτὴν ἀπέλυεν αὐτοῖς ἕνα δέσμιον ὃν παρῃτοῦντο.
\vs{7}ἦν δὲ ὁ λεγόμενος Βαραββᾶς μετὰ τῶν στασιαστῶν δεδεμένος οἵτινες ἐν τῇ στάσει φόνον πεποιήκεισαν.
\vs{8}καὶ ἀναβὰς ὁ ὄχλος ἤρξατο αἰτεῖσθαι καθὼς ἐποίει αὐτοῖς.
\vs{9}Ὁ δὲ Πιλᾶτος ἀπεκρίθη αὐτοῖς λέγων· Θέλετε ἀπολύσω ὑμῖν τὸν Βασιλέα τῶν Ἰουδαίων;
\vs{10}ἐγίνωσκεν γὰρ ὅτι διὰ φθόνον παραδεδώκεισαν αὐτὸν οἱ ἀρχιερεῖς.
\vs{11}Οἱ δὲ ἀρχιερεῖς ἀνέσεισαν τὸν ὄχλον ἵνα μᾶλλον τὸν Βαραββᾶν ἀπολύσῃ αὐτοῖς.
\vs{12}Ὁ δὲ Πιλᾶτος πάλιν ἀποκριθεὶς ἔλεγεν αὐτοῖς· Τί οὖν θέλετε ποιήσω ὃν λέγετε τὸν Βασιλέα τῶν Ἰουδαίων;
\vs{13}Οἱ δὲ πάλιν ἔκραξαν· Σταύρωσον αὐτόν.
\vs{14}Ὁ δὲ Πιλᾶτος ἔλεγεν αὐτοῖς· Τί γὰρ ἐποίησεν κακόν; Οἱ δὲ περισσῶς ἔκραξαν· Σταύρωσον αὐτόν.

\vs{15}Ὁ δὲ Πιλᾶτος βουλόμενος τῷ ὄχλῳ τὸ ἱκανὸν ποιῆσαι ἀπέλυσεν αὐτοῖς τὸν Βαραββᾶν, καὶ παρέδωκεν τὸν Ἰησοῦν φραγελλώσας ἵνα σταυρωθῇ.

\vs{16}Οἱ δὲ στρατιῶται ἀπήγαγον αὐτὸν ἔσω τῆς αὐλῆς, ὅ ἐστιν Πραιτώριον, καὶ συνκαλοῦσιν ὅλην τὴν σπεῖραν.
\vs{17}καὶ ἐνδιδύσκουσιν αὐτὸν πορφύραν καὶ περιτιθέασιν αὐτῷ πλέξαντες ἀκάνθινον στέφανον·
\vs{18}καὶ ἤρξαντο ἀσπάζεσθαι αὐτόν· Χαῖρε, Βασιλεῦ τῶν Ἰουδαίων·
\vs{19}Καὶ ἔτυπτον αὐτοῦ τὴν κεφαλὴν καλάμῳ καὶ ἐνέπτυον αὐτῷ καὶ τιθέντες τὰ γόνατα προσεκύνουν αὐτῷ.
\vs{20}καὶ ὅτε ἐνέπαιξαν αὐτῷ, ἐξέδυσαν αὐτὸν τὴν πορφύραν καὶ ἐνέδυσαν αὐτὸν τὰ ἱμάτια αὐτοῦ.

Καὶ ἐξάγουσιν αὐτὸν ἵνα σταυρώσωσιν αὐτόν.
\vs{21}Καὶ ἀγγαρεύουσιν παράγοντά τινα Σίμωνα Κυρηναῖον ἐρχόμενον ἀπ᾽ ἀγροῦ, τὸν πατέρα Ἀλεξάνδρου καὶ Ῥούφου, ἵνα ἄρῃ τὸν σταυρὸν αὐτοῦ.

\vs{22}Καὶ φέρουσιν αὐτὸν ἐπὶ τὸν Γολγοθᾶν τόπον, ὅ ἐστιν μεθερμηνευόμενον Κρανίου τόπος.
\vs{23}καὶ ἐδίδουν αὐτῷ ἐσμυρνισμένον οἶνον· ὃς δὲ οὐκ ἔλαβεν.
\vs{24}Καὶ σταυροῦσιν αὐτὸν καὶ διαμερίζονται τὰ ἱμάτια αὐτοῦ βάλλοντες κλῆρον ἐπ᾽ αὐτὰ τίς τί ἄρῃ.
\vs{25}Ἦν δὲ ὥρα τρίτη καὶ ἐσταύρωσαν αὐτόν.
\vs{26}καὶ ἦν ἡ ἐπιγραφὴ τῆς αἰτίας αὐτοῦ ἐπιγεγραμμένη· 
\begin{poetryblock}

\begin{quote}Ο ΒΑΣΙΛΕΥΣ ΤΩΝ ΙΟΥΔΑΙΩΝ.\end{quote}
\end{poetryblock}

\vs{27}Καὶ σὺν αὐτῷ σταυροῦσιν δύο λῃστάς, ἕνα ἐκ δεξιῶν καὶ ἕνα ἐξ εὐωνύμων αὐτοῦ.
\vs{29}Καὶ οἱ παραπορευόμενοι ἐβλασφήμουν αὐτὸν κινοῦντες τὰς κεφαλὰς αὐτῶν καὶ λέγοντες· Οὐὰ ὁ καταλύων τὸν ναὸν καὶ οἰκοδομῶν ἐν τρισὶν ἡμέραις,
\vs{30}σῶσον σεαυτὸν καταβὰς ἀπὸ τοῦ σταυροῦ.
\vs{31}Ὁμοίως καὶ οἱ ἀρχιερεῖς ἐμπαίζοντες πρὸς ἀλλήλους μετὰ τῶν γραμματέων ἔλεγον· Ἄλλους ἔσωσεν, ἑαυτὸν οὐ δύναται σῶσαι·
\vs{32}ὁ Χριστὸς ὁ Βασιλεὺς Ἰσραὴλ καταβάτω νῦν ἀπὸ τοῦ σταυροῦ, ἵνα ἴδωμεν καὶ πιστεύσωμεν. καὶ οἱ συνεσταυρωμένοι σὺν αὐτῷ ὠνείδιζον αὐτόν.

\vs{33}Καὶ γενομένης ὥρας ἕκτης σκότος ἐγένετο ἐφ᾽ ὅλην τὴν γῆν ἕως ὥρας ἐνάτης.
\vs{34}καὶ τῇ ἐνάτῃ ὥρᾳ ἐβόησεν ὁ Ἰησοῦς φωνῇ μεγάλῃ· 
\begin{poetryblock}

\begin{quote}Ἐλωῒ Ἐλωῒ λεμὰ σαβαχθάνι;\end{quote}
\end{poetryblock}

ὅ ἐστιν μεθερμηνευόμενον· Ὁ Θεός μου ὁ Θεός μου, εἰς τί ἐγκατέλιπές με;

\vs{35}Καί τινες τῶν παρεστηκότων ἀκούσαντες ἔλεγον· Ἴδε Ἠλίαν φωνεῖ.
\vs{36}Δραμὼν δέ τις καὶ γεμίσας σπόγγον ὄξους περιθεὶς καλάμῳ ἐπότιζεν αὐτόν λέγων· Ἄφετε ἴδωμεν εἰ ἔρχεται Ἠλίας καθελεῖν αὐτόν.
\vs{37}Ὁ δὲ Ἰησοῦς ἀφεὶς φωνὴν μεγάλην ἐξέπνευσεν.

\vs{38}Καὶ τὸ καταπέτασμα τοῦ ναοῦ ἐσχίσθη εἰς δύο ἀπ᾽ ἄνωθεν ἕως κάτω.
\vs{39}Ἰδὼν δὲ ὁ κεντυρίων ὁ παρεστηκὼς ἐξ ἐναντίας αὐτοῦ ὅτι οὕτως ἐξέπνευσεν εἶπεν· Ἀληθῶς οὗτος ὁ ἄνθρωπος Υἱὸς Θεοῦ ἦν.

\vs{40}Ἦσαν δὲ καὶ γυναῖκες ἀπὸ μακρόθεν θεωροῦσαι, ἐν αἷς καὶ Μαρία ἡ Μαγδαληνὴ καὶ Μαρία ἡ Ἰακώβου τοῦ μικροῦ καὶ Ἰωσῆτος μήτηρ καὶ Σαλώμη,
\vs{41}αἳ ὅτε ἦν ἐν τῇ Γαλιλαίᾳ ἠκολούθουν αὐτῷ καὶ διηκόνουν αὐτῷ, καὶ ἄλλαι πολλαὶ αἱ συναναβᾶσαι αὐτῷ εἰς Ἱεροσόλυμα.

\vs{42}Καὶ ἤδη ὀψίας γενομένης, ἐπεὶ ἦν Παρασκευή ὅ ἐστιν προσάββατον,
\vs{43}ἐλθὼν Ἰωσὴφ ὁ ἀπὸ Ἁριμαθαίας εὐσχήμων βουλευτής, ὃς καὶ αὐτὸς ἦν προσδεχόμενος τὴν βασιλείαν τοῦ Θεοῦ, τολμήσας εἰσῆλθεν πρὸς τὸν Πιλᾶτον καὶ ᾐτήσατο τὸ σῶμα τοῦ Ἰησοῦ.
\vs{44}Ὁ δὲ Πιλᾶτος ἐθαύμασεν εἰ ἤδη τέθνηκεν καὶ προσκαλεσάμενος τὸν κεντυρίωνα ἐπηρώτησεν αὐτὸν εἰ πάλαι ἀπέθανεν·
\vs{45}καὶ γνοὺς ἀπὸ τοῦ κεντυρίωνος ἐδωρήσατο τὸ πτῶμα τῷ Ἰωσήφ.
\vs{46}Καὶ ἀγοράσας σινδόνα καθελὼν αὐτὸν ἐνείλησεν τῇ σινδόνι καὶ ἔθηκεν αὐτὸν ἐν μνημείῳ ὃ ἦν λελατομημένον ἐκ πέτρας καὶ προσεκύλισεν λίθον ἐπὶ τὴν θύραν τοῦ μνημείου.
\vs{47}ἡ δὲ Μαρία ἡ Μαγδαληνὴ καὶ Μαρία ἡ Ἰωσῆτος ἐθεώρουν ποῦ τέθειται.

\ch{16}
Καὶ διαγενομένου τοῦ σαββάτου Μαρία ἡ Μαγδαληνὴ καὶ Μαρία ἡ τοῦ Ἰακώβου καὶ Σαλώμη ἠγόρασαν ἀρώματα ἵνα ἐλθοῦσαι ἀλείψωσιν αὐτόν.
\vs{2}καὶ λίαν πρωῒ τῇ μιᾷ τῶν σαββάτων ἔρχονται ἐπὶ τὸ μνημεῖον ἀνατείλαντος τοῦ ἡλίου.
\vs{3}καὶ ἔλεγον πρὸς ἑαυτάς· Τίς ἀποκυλίσει ἡμῖν τὸν λίθον ἐκ τῆς θύρας τοῦ μνημείου;
\vs{4}καὶ ἀναβλέψασαι θεωροῦσιν ὅτι ἀποκεκύλισται ὁ λίθος· ἦν γὰρ μέγας σφόδρα.

\vs{5}Καὶ εἰσελθοῦσαι εἰς τὸ μνημεῖον εἶδον νεανίσκον καθήμενον ἐν τοῖς δεξιοῖς περιβεβλημένον στολὴν λευκήν, καὶ ἐξεθαμβήθησαν.
\vs{6}ὁ δὲ λέγει αὐταῖς· Μὴ ἐκθαμβεῖσθε· Ἰησοῦν ζητεῖτε τὸν Ναζαρηνὸν τὸν ἐσταυρωμένον· ἠγέρθη, οὐκ ἔστιν ὧδε· ἴδε ὁ τόπος ὅπου ἔθηκαν αὐτόν.
\vs{7}ἀλλὰ ὑπάγετε εἴπατε τοῖς μαθηταῖς αὐτοῦ καὶ τῷ Πέτρῳ ὅτι Προάγει ὑμᾶς εἰς τὴν Γαλιλαίαν· ἐκεῖ αὐτὸν ὄψεσθε, καθὼς εἶπεν ὑμῖν.

\vs{8}Καὶ ἐξελθοῦσαι ἔφυγον ἀπὸ τοῦ μνημείου, εἶχεν γὰρ αὐτὰς τρόμος καὶ ἔκστασις· καὶ οὐδενὶ οὐδὲν εἶπαν· ἐφοβοῦντο γάρ.

\vs{9}[[Ἀναστὰς δὲ πρωῒ πρώτῃ σαββάτου ἐφάνη πρῶτον Μαρίᾳ τῇ Μαγδαληνῇ, παρ᾽ ἧς ἐκβεβλήκει ἑπτὰ δαιμόνια.
\vs{10}ἐκείνη πορευθεῖσα ἀπήγγειλεν τοῖς μετ᾽ αὐτοῦ γενομένοις πενθοῦσι καὶ κλαίουσιν·
\vs{11}κἀκεῖνοι ἀκούσαντες ὅτι ζῇ καὶ ἐθεάθη ὑπ᾽ αὐτῆς ἠπίστησαν.

\vs{12}Μετὰ δὲ ταῦτα δυσὶν ἐξ αὐτῶν περιπατοῦσιν ἐφανερώθη ἐν ἑτέρᾳ μορφῇ πορευομένοις εἰς ἀγρόν·
\vs{13}κἀκεῖνοι ἀπελθόντες ἀπήγγειλαν τοῖς λοιποῖς· οὐδὲ ἐκείνοις ἐπίστευσαν.

\vs{14}Ὕστερον δὲ ἀνακειμένοις αὐτοῖς τοῖς ἕνδεκα ἐφανερώθη καὶ ὠνείδισεν τὴν ἀπιστίαν αὐτῶν καὶ σκληροκαρδίαν ὅτι τοῖς θεασαμένοις αὐτὸν ἐγηγερμένον οὐκ ἐπίστευσαν.
\vs{15}Καὶ εἶπεν αὐτοῖς· Πορευθέντες εἰς τὸν κόσμον ἅπαντα κηρύξατε τὸ εὐαγγέλιον πάσῃ τῇ κτίσει.
\vs{16}ὁ πιστεύσας καὶ βαπτισθεὶς σωθήσεται, ὁ δὲ ἀπιστήσας κατακριθήσεται.
\vs{17}σημεῖα δὲ τοῖς πιστεύσασιν ταῦτα παρακολουθήσει· ἐν τῷ ὀνόματί μου δαιμόνια ἐκβαλοῦσιν, γλώσσαις λαλήσουσιν καιναῖς,
\vs{18}και ἐν ταῖς χερσὶν ὄφεις ἀροῦσιν κἂν θανάσιμόν τι πίωσιν οὐ μὴ αὐτοὺς βλάψῃ, ἐπὶ ἀρρώστους χεῖρας ἐπιθήσουσιν καὶ καλῶς ἕξουσιν.

\vs{19}Ὁ μὲν οὖν Κύριος Ἰησοῦς μετὰ τὸ λαλῆσαι αὐτοῖς ἀνελήμφθη εἰς τὸν οὐρανὸν καὶ ἐκάθισεν ἐκ δεξιῶν τοῦ Θεοῦ.
\vs{20}ἐκεῖνοι δὲ ἐξελθόντες ἐκήρυξαν πανταχοῦ, τοῦ Κυρίου συνεργοῦντος καὶ τὸν λόγον βεβαιοῦντος διὰ τῶν ἐπακολουθούντων σημείων.]]

[[πάντα δὲ τὰ παρηγγελμένα τοῖς περὶ τὸν Πέτρον συντόμως ἐξήγγειλαν. μετὰ δὲ ταῦτα καὶ αὐτὸς ὁ Ἰησοῦς ἀπὸ ἀνατολῆς καὶ ἄχρι δύσεως ἐξαπέστειλεν δι᾽ αὐτῶν τὸ ἱερὸν καὶ ἄφθαρτον κήρυγμα τῆς αἰωνίου σωτηρίας. ἀμήν.]]


\def\book{ΠΡΟΣ ΡΩΜΑΙΟΥΣ}
\biblebook{ΠΡΟΣ ΡΩΜΑΙΟΥΣ}


\lettrine[lines=2, loversize=0.2, nindent=0em, findent=.25em]{\textcolor{bookheadingcolor}{Π}}{αῦλος} δοῦλος Χριστοῦ Ἰησοῦ, κλητὸς ἀπόστολος ἀφωρισμένος εἰς εὐαγγέλιον Θεοῦ,
\vs{2}ὃ προεπηγγείλατο διὰ τῶν προφητῶν αὐτοῦ ἐν γραφαῖς ἁγίαις
\vs{3}περὶ τοῦ Υἱοῦ αὐτοῦ τοῦ γενομένου ἐκ σπέρματος Δαυὶδ κατὰ σάρκα,
\vs{4}τοῦ ὁρισθέντος Υἱοῦ Θεοῦ ἐν δυνάμει κατὰ πνεῦμα ἁγιωσύνης ἐξ ἀναστάσεως νεκρῶν, Ἰησοῦ Χριστοῦ τοῦ Κυρίου ἡμῶν,
\vs{5}δι᾽ οὗ ἐλάβομεν χάριν καὶ ἀποστολὴν εἰς ὑπακοὴν πίστεως ἐν πᾶσιν τοῖς ἔθνεσιν ὑπὲρ τοῦ ὀνόματος αὐτοῦ,
\vs{6}ἐν οἷς ἐστε καὶ ὑμεῖς κλητοὶ Ἰησοῦ Χριστοῦ,
\vs{7}Πᾶσιν τοῖς οὖσιν ἐν Ῥώμῃ ἀγαπητοῖς Θεοῦ, κλητοῖς ἁγίοις, Χάρις ὑμῖν καὶ εἰρήνη ἀπὸ Θεοῦ Πατρὸς ἡμῶν καὶ Κυρίου Ἰησοῦ Χριστοῦ.

\vs{8}Πρῶτον μὲν εὐχαριστῶ τῷ Θεῷ μου διὰ Ἰησοῦ Χριστοῦ περὶ πάντων ὑμῶν ὅτι ἡ πίστις ὑμῶν καταγγέλλεται ἐν ὅλῳ τῷ κόσμῳ.
\vs{9}μάρτυς γάρ μού ἐστιν ὁ Θεός, ᾧ λατρεύω ἐν τῷ πνεύματί μου ἐν τῷ εὐαγγελίῳ τοῦ Υἱοῦ αὐτοῦ, ὡς ἀδιαλείπτως μνείαν ὑμῶν ποιοῦμαι
\vs{10}πάντοτε ἐπὶ τῶν προσευχῶν μου δεόμενος εἴ πως ἤδη ποτὲ εὐοδωθήσομαι ἐν τῷ θελήματι τοῦ Θεοῦ ἐλθεῖν πρὸς ὑμᾶς.
\vs{11}ἐπιποθῶ γὰρ ἰδεῖν ὑμᾶς, ἵνα τι μεταδῶ χάρισμα ὑμῖν πνευματικὸν εἰς τὸ στηριχθῆναι ὑμᾶς,
\vs{12}τοῦτο δέ ἐστιν συμπαρακληθῆναι ἐν ὑμῖν διὰ τῆς ἐν ἀλλήλοις πίστεως ὑμῶν τε καὶ ἐμοῦ.
\vs{13}Οὐ θέλω δὲ ὑμᾶς ἀγνοεῖν, ἀδελφοί, ὅτι πολλάκις προεθέμην ἐλθεῖν πρὸς ὑμᾶς, καὶ ἐκωλύθην ἄχρι τοῦ δεῦρο, ἵνα τινὰ καρπὸν σχῶ καὶ ἐν ὑμῖν καθὼς καὶ ἐν τοῖς λοιποῖς ἔθνεσιν.
\vs{14}Ἕλλησίν τε καὶ Βαρβάροις, σοφοῖς τε καὶ ἀνοήτοις ὀφειλέτης εἰμί,
\vs{15}οὕτως τὸ κατ᾽ ἐμὲ πρόθυμον καὶ ὑμῖν τοῖς ἐν Ῥώμῃ εὐαγγελίσασθαι.

\vs{16}Οὐ γὰρ ἐπαισχύνομαι τὸ εὐαγγέλιον, δύναμις γὰρ Θεοῦ ἐστιν εἰς σωτηρίαν παντὶ τῷ πιστεύοντι, Ἰουδαίῳ τε πρῶτον καὶ Ἕλληνι.
\vs{17}δικαιοσύνη γὰρ Θεοῦ ἐν αὐτῷ ἀποκαλύπτεται ἐκ πίστεως εἰς πίστιν, καθὼς γέγραπται· Ὁ δὲ δίκαιος ἐκ πίστεως ζήσεται.

\vs{18}Ἀποκαλύπτεται γὰρ ὀργὴ Θεοῦ ἀπ᾽ οὐρανοῦ ἐπὶ πᾶσαν ἀσέβειαν καὶ ἀδικίαν ἀνθρώπων τῶν τὴν ἀλήθειαν ἐν ἀδικίᾳ κατεχόντων,
\vs{19}διότι τὸ γνωστὸν τοῦ Θεοῦ φανερόν ἐστιν ἐν αὐτοῖς· ὁ θεὸς γὰρ αὐτοῖς ἐφανέρωσεν.
\vs{20}τὰ γὰρ ἀόρατα αὐτοῦ ἀπὸ κτίσεως κόσμου τοῖς ποιήμασιν νοούμενα καθορᾶται, ἥ τε ἀΐδιος αὐτοῦ δύναμις καὶ θειότης, εἰς τὸ εἶναι αὐτοὺς ἀναπολογήτους,
\vs{21}Διότι γνόντες τὸν Θεὸν οὐχ ὡς Θεὸν ἐδόξασαν ἢ ηὐχαρίστησαν, ἀλλὰ ἐματαιώθησαν ἐν τοῖς διαλογισμοῖς αὐτῶν καὶ ἐσκοτίσθη ἡ ἀσύνετος αὐτῶν καρδία.
\vs{22}φάσκοντες εἶναι σοφοὶ ἐμωράνθησαν
\vs{23}καὶ ἤλλαξαν τὴν δόξαν τοῦ ἀφθάρτου Θεοῦ ἐν ὁμοιώματι εἰκόνος φθαρτοῦ ἀνθρώπου καὶ πετεινῶν καὶ τετραπόδων καὶ ἑρπετῶν.
\vs{24}Διὸ παρέδωκεν αὐτοὺς ὁ Θεὸς ἐν ταῖς ἐπιθυμίαις τῶν καρδιῶν αὐτῶν εἰς ἀκαθαρσίαν τοῦ ἀτιμάζεσθαι τὰ σώματα αὐτῶν ἐν αὐτοῖς·
\vs{25}οἵτινες μετήλλαξαν τὴν ἀλήθειαν τοῦ Θεοῦ ἐν τῷ ψεύδει καὶ ἐσεβάσθησαν καὶ ἐλάτρευσαν τῇ κτίσει παρὰ τὸν Κτίσαντα, ὅς ἐστιν εὐλογητὸς εἰς τοὺς αἰῶνας, ἀμήν.
\vs{26}Διὰ τοῦτο παρέδωκεν αὐτοὺς ὁ Θεὸς εἰς πάθη ἀτιμίας, αἵ τε γὰρ θήλειαι αὐτῶν μετήλλαξαν τὴν φυσικὴν χρῆσιν εἰς τὴν παρὰ φύσιν,
\vs{27}ὁμοίως τε καὶ οἱ ἄρσενες ἀφέντες τὴν φυσικὴν χρῆσιν τῆς θηλείας ἐξεκαύθησαν ἐν τῇ ὀρέξει αὐτῶν εἰς ἀλλήλους, ἄρσενες ἐν ἄρσεσιν τὴν ἀσχημοσύνην κατεργαζόμενοι καὶ τὴν ἀντιμισθίαν ἣν ἔδει τῆς πλάνης αὐτῶν ἐν ἑαυτοῖς ἀπολαμβάνοντες.
\vs{28}Καὶ καθὼς οὐκ ἐδοκίμασαν τὸν Θεὸν ἔχειν ἐν ἐπιγνώσει, παρέδωκεν αὐτοὺς ὁ Θεὸς εἰς ἀδόκιμον νοῦν, ποιεῖν τὰ μὴ καθήκοντα,
\vs{29}πεπληρωμένους πάσῃ ἀδικίᾳ πονηρίᾳ πλεονεξίᾳ κακίᾳ, μεστοὺς φθόνου φόνου ἔριδος δόλου κακοηθείας, ψιθυριστάς
\vs{30}καταλάλους θεοστυγεῖς ὑβριστάς ὑπερηφάνους ἀλαζόνας, ἐφευρετὰς κακῶν, γονεῦσιν ἀπειθεῖς,
\vs{31}ἀσυνέτους ἀσυνθέτους ἀστόργους ἀνελεήμονας·
\vs{32}οἵτινες τὸ δικαίωμα τοῦ Θεοῦ ἐπιγνόντες ὅτι οἱ τὰ τοιαῦτα πράσσοντες ἄξιοι θανάτου εἰσίν, οὐ μόνον αὐτὰ ποιοῦσιν ἀλλὰ καὶ συνευδοκοῦσιν τοῖς πράσσουσιν.

\ch{2}
Διὸ ἀναπολόγητος εἶ, ὦ ἄνθρωπε πᾶς ὁ κρίνων· ἐν ᾧ γὰρ κρίνεις τὸν ἕτερον, σεαυτὸν κατακρίνεις, τὰ γὰρ αὐτὰ πράσσεις ὁ κρίνων.
\vs{2}οἴδαμεν δὲ ὅτι τὸ κρίμα τοῦ Θεοῦ ἐστιν κατὰ ἀλήθειαν ἐπὶ τοὺς τὰ τοιαῦτα πράσσοντας.
\vs{3}λογίζῃ δὲ τοῦτο, ὦ ἄνθρωπε ὁ κρίνων τοὺς τὰ τοιαῦτα πράσσοντας καὶ ποιῶν αὐτά, ὅτι σὺ ἐκφεύξῃ τὸ κρίμα τοῦ Θεοῦ;
\vs{4}ἢ τοῦ πλούτου τῆς χρηστότητος αὐτοῦ καὶ τῆς ἀνοχῆς καὶ τῆς μακροθυμίας καταφρονεῖς, ἀγνοῶν ὅτι τὸ χρηστὸν τοῦ Θεοῦ εἰς μετάνοιάν σε ἄγει;
\vs{5}Κατὰ δὲ τὴν σκληρότητά σου καὶ ἀμετανόητον καρδίαν θησαυρίζεις σεαυτῷ ὀργὴν ἐν ἡμέρᾳ ὀργῆς καὶ ἀποκαλύψεως δικαιοκρισίας τοῦ Θεοῦ
\vs{6}ὃς Ἀποδώσει ἑκάστῳ κατὰ τὰ ἔργα αὐτοῦ·
\vs{7}τοῖς μὲν καθ᾽ ὑπομονὴν ἔργου ἀγαθοῦ δόξαν καὶ τιμὴν καὶ ἀφθαρσίαν ζητοῦσιν ζωὴν αἰώνιον,
\vs{8}τοῖς δὲ ἐξ ἐριθείας καὶ ἀπειθοῦσι τῇ ἀληθείᾳ πειθομένοις δὲ τῇ ἀδικίᾳ ὀργὴ καὶ θυμός.
\vs{9}θλῖψις καὶ στενοχωρία ἐπὶ πᾶσαν ψυχὴν ἀνθρώπου τοῦ κατεργαζομένου τὸ κακόν, Ἰουδαίου τε πρῶτον καὶ Ἕλληνος·
\vs{10}δόξα δὲ καὶ τιμὴ καὶ εἰρήνη παντὶ τῷ ἐργαζομένῳ τὸ ἀγαθόν, Ἰουδαίῳ τε πρῶτον καὶ Ἕλληνι·
\vs{11}οὐ γάρ ἐστιν προσωπολημψία παρὰ τῷ Θεῷ.

\vs{12}Ὅσοι γὰρ ἀνόμως ἥμαρτον, ἀνόμως καὶ ἀπολοῦνται, καὶ ὅσοι ἐν νόμῳ ἥμαρτον, διὰ νόμου κριθήσονται·
\vs{13}οὐ γὰρ οἱ ἀκροαταὶ νόμου δίκαιοι παρὰ τῷ Θεῷ, ἀλλ᾽ οἱ ποιηταὶ νόμου δικαιωθήσονται.
\vs{14}Ὅταν γὰρ ἔθνη τὰ μὴ νόμον ἔχοντα φύσει τὰ τοῦ νόμου ποιῶσιν, οὗτοι νόμον μὴ ἔχοντες ἑαυτοῖς εἰσιν νόμος·
\vs{15}οἵτινες ἐνδείκνυνται τὸ ἔργον τοῦ νόμου γραπτὸν ἐν ταῖς καρδίαις αὐτῶν, συμμαρτυρούσης αὐτῶν τῆς συνειδήσεως καὶ μεταξὺ ἀλλήλων τῶν λογισμῶν κατηγορούντων ἢ καὶ ἀπολογουμένων,
\vs{16}ἐν ἡμέρᾳ ὅτε κρίνει ὁ Θεὸς τὰ κρυπτὰ τῶν ἀνθρώπων κατὰ τὸ εὐαγγέλιόν μου διὰ Χριστοῦ Ἰησοῦ.

\vs{17}Εἰ δὲ σὺ Ἰουδαῖος ἐπονομάζῃ καὶ ἐπαναπαύῃ νόμῳ καὶ καυχᾶσαι ἐν Θεῷ
\vs{18}καὶ γινώσκεις τὸ θέλημα καὶ δοκιμάζεις τὰ διαφέροντα κατηχούμενος ἐκ τοῦ νόμου,
\vs{19}πέποιθάς τε σεαυτὸν ὁδηγὸν εἶναι τυφλῶν, φῶς τῶν ἐν σκότει,
\vs{20}παιδευτὴν ἀφρόνων, διδάσκαλον νηπίων, ἔχοντα τὴν μόρφωσιν τῆς γνώσεως καὶ τῆς ἀληθείας ἐν τῷ νόμῳ·
\vs{21}ὁ οὖν διδάσκων ἕτερον σεαυτὸν οὐ διδάσκεις; ὁ κηρύσσων μὴ κλέπτειν κλέπτεις;
\vs{22}ὁ λέγων μὴ μοιχεύειν μοιχεύεις; ὁ βδελυσσόμενος τὰ εἴδωλα ἱεροσυλεῖς;
\vs{23}ὃς ἐν νόμῳ καυχᾶσαι, διὰ τῆς παραβάσεως τοῦ νόμου τὸν Θεὸν ἀτιμάζεις·
\vs{24}τὸ γὰρ Ὄνομα τοῦ Θεοῦ δι᾽ ὑμᾶς βλασφημεῖται ἐν τοῖς ἔθνεσιν, καθὼς γέγραπται.

\vs{25}Περιτομὴ μὲν γὰρ ὠφελεῖ ἐὰν νόμον πράσσῃς· ἐὰν δὲ παραβάτης νόμου ᾖς, ἡ περιτομή σου ἀκροβυστία γέγονεν.
\vs{26}ἐὰν οὖν ἡ ἀκροβυστία τὰ δικαιώματα τοῦ νόμου φυλάσσῃ, οὐχ ἡ ἀκροβυστία αὐτοῦ εἰς περιτομὴν λογισθήσεται;
\vs{27}καὶ κρινεῖ ἡ ἐκ φύσεως ἀκροβυστία τὸν νόμον τελοῦσα σὲ τὸν διὰ γράμματος καὶ περιτομῆς παραβάτην νόμου.
\vs{28}Οὐ γὰρ ὁ ἐν τῷ φανερῷ Ἰουδαῖός ἐστιν οὐδὲ ἡ ἐν τῷ φανερῷ ἐν σαρκὶ περιτομή,
\vs{29}ἀλλ᾽ ὁ ἐν τῷ κρυπτῷ Ἰουδαῖος, καὶ περιτομὴ καρδίας ἐν πνεύματι οὐ γράμματι, οὗ ὁ ἔπαινος οὐκ ἐξ ἀνθρώπων ἀλλ᾽ ἐκ τοῦ Θεοῦ.

\ch{3}
Τί οὖν τὸ περισσὸν τοῦ Ἰουδαίου ἢ τίς ἡ ὠφέλεια τῆς περιτομῆς;
\vs{2}πολὺ κατὰ πάντα τρόπον. πρῶτον μὲν γὰρ ὅτι ἐπιστεύθησαν τὰ λόγια τοῦ Θεοῦ.
\vs{3}Τί γάρ; εἰ ἠπίστησάν τινες, μὴ ἡ ἀπιστία αὐτῶν τὴν πίστιν τοῦ Θεοῦ καταργήσει;
\vs{4}μὴ γένοιτο· γινέσθω δὲ ὁ Θεὸς ἀληθής, πᾶς δὲ ἄνθρωπος ψεύστης, καθὼς γέγραπται· 
\begin{poetryblock}

\begin{quote}Ὅπως ἂν δικαιωθῇς ἐν τοῖς λόγοις σου\end{quote} 

\begin{quote}καὶ νικήσεις ἐν τῷ κρίνεσθαί σε.\end{quote}
\end{poetryblock}

\vs{5}Εἰ δὲ ἡ ἀδικία ἡμῶν Θεοῦ δικαιοσύνην συνίστησιν, τί ἐροῦμεν; μὴ ἄδικος ὁ Θεὸς ὁ ἐπιφέρων τὴν ὀργήν; κατὰ ἄνθρωπον λέγω.
\vs{6}μὴ γένοιτο· ἐπεὶ πῶς κρινεῖ ὁ Θεὸς τὸν κόσμον;
\vs{7}εἰ δὲ ἡ ἀλήθεια τοῦ Θεοῦ ἐν τῷ ἐμῷ ψεύσματι ἐπερίσσευσεν εἰς τὴν δόξαν αὐτοῦ, τί ἔτι κἀγὼ ὡς ἁμαρτωλὸς κρίνομαι;
\vs{8}καὶ μὴ καθὼς βλασφημούμεθα καὶ καθώς φασίν τινες ἡμᾶς λέγειν ὅτι Ποιήσωμεν τὰ κακὰ, ἵνα ἔλθῃ τὰ ἀγαθά; ὧν τὸ κρίμα ἔνδικόν ἐστιν.

\vs{9}Τί οὖν; προεχόμεθα; οὐ πάντως· προῃτιασάμεθα γὰρ Ἰουδαίους τε καὶ Ἕλληνας πάντας ὑφ᾽ ἁμαρτίαν εἶναι,
\vs{10}καθὼς γέγραπται ὅτι 
\begin{poetryblock}

\begin{quote}Οὐκ ἔστιν δίκαιος οὐδὲ εἷς,\end{quote}

\begin{quote} \vs{11}οὐκ ἔστιν ὁ συνίων,\end{quote} 

\begin{quote}οὐκ ἔστιν ὁ ἐκζητῶν τὸν Θεόν.\end{quote}

\begin{quote} \vs{12}πάντες ἐξέκλιναν ἅμα ἠχρεώθησαν·\end{quote} 

\begin{quote}οὐκ ἔστιν ὁ ποιῶν χρηστότητα,\end{quote} 

\begin{quote}οὐκ ἔστιν ἕως ἑνός.\end{quote}

\begin{quote} \vs{13}τάφος ἀνεῳγμένος ὁ λάρυγξ αὐτῶν,\end{quote} 

\begin{quote}ταῖς γλώσσαις αὐτῶν ἐδολιοῦσαν,\end{quote} 

\begin{quote}ἰὸς ἀσπίδων ὑπὸ τὰ χείλη αὐτῶν·\end{quote}

\begin{quote} \vs{14}ὧν τὸ στόμα ἀρᾶς καὶ πικρίας γέμει,\end{quote}

\begin{quote} \vs{15}ὀξεῖς οἱ πόδες αὐτῶν ἐκχέαι αἷμα,\end{quote}

\begin{quote} \vs{16}σύντριμμα καὶ ταλαιπωρία ἐν ταῖς ὁδοῖς αὐτῶν,\end{quote}

\begin{quote} \vs{17}καὶ ὁδὸν εἰρήνης οὐκ ἔγνωσαν.\end{quote}

\begin{quote} \vs{18}οὐκ ἔστιν φόβος Θεοῦ ἀπέναντι τῶν ὀφθαλμῶν αὐτῶν.\end{quote}
\end{poetryblock}

\vs{19}Οἴδαμεν δὲ ὅτι ὅσα ὁ νόμος λέγει τοῖς ἐν τῷ νόμῳ λαλεῖ, ἵνα πᾶν στόμα φραγῇ καὶ ὑπόδικος γένηται πᾶς ὁ κόσμος τῷ Θεῷ·
\vs{20}διότι ἐξ ἔργων νόμου οὐ δικαιωθήσεται πᾶσα σὰρξ ἐνώπιον αὐτοῦ, διὰ γὰρ νόμου ἐπίγνωσις ἁμαρτίας.

\vs{21}Νυνὶ δὲ χωρὶς νόμου δικαιοσύνη Θεοῦ πεφανέρωται μαρτυρουμένη ὑπὸ τοῦ νόμου καὶ τῶν προφητῶν,
\vs{22}δικαιοσύνη δὲ Θεοῦ διὰ πίστεως Ἰησοῦ Χριστοῦ εἰς πάντας τοὺς πιστεύοντας. οὐ γάρ ἐστιν διαστολή,
\vs{23}πάντες γὰρ ἥμαρτον καὶ ὑστεροῦνται τῆς δόξης τοῦ Θεοῦ
\vs{24}δικαιούμενοι δωρεὰν τῇ αὐτοῦ χάριτι διὰ τῆς ἀπολυτρώσεως τῆς ἐν Χριστῷ Ἰησοῦ·
\vs{25}ὃν προέθετο ὁ Θεὸς ἱλαστήριον διὰ τῆς πίστεως ἐν τῷ αὐτοῦ αἵματι εἰς ἔνδειξιν τῆς δικαιοσύνης αὐτοῦ διὰ τὴν πάρεσιν τῶν προγεγονότων ἁμαρτημάτων
\vs{26}ἐν τῇ ἀνοχῇ τοῦ Θεοῦ, πρὸς τὴν ἔνδειξιν τῆς δικαιοσύνης αὐτοῦ ἐν τῷ νῦν καιρῷ, εἰς τὸ εἶναι αὐτὸν δίκαιον καὶ δικαιοῦντα τὸν ἐκ πίστεως Ἰησοῦ.

\vs{27}Ποῦ οὖν ἡ καύχησις; ἐξεκλείσθη. διὰ ποίου νόμου; τῶν ἔργων; οὐχί, ἀλλὰ διὰ νόμου πίστεως.
\vs{28}λογιζόμεθα γὰρ δικαιοῦσθαι πίστει ἄνθρωπον χωρὶς ἔργων νόμου.
\vs{29}Ἢ Ἰουδαίων ὁ Θεὸς μόνον; οὐχὶ καὶ ἐθνῶν; ναὶ καὶ ἐθνῶν,
\vs{30}εἴπερ εἷς ὁ Θεός ὃς δικαιώσει περιτομὴν ἐκ πίστεως καὶ ἀκροβυστίαν διὰ τῆς πίστεως.
\vs{31}Νόμον οὖν καταργοῦμεν διὰ τῆς πίστεως; μὴ γένοιτο· ἀλλὰ νόμον ἱστάνομεν.

\ch{4}
Τί οὖν ἐροῦμεν εὑρηκέναι Ἀβραὰμ τὸν προπάτορα ἡμῶν κατὰ σάρκα;
\vs{2}εἰ γὰρ Ἀβραὰμ ἐξ ἔργων ἐδικαιώθη, ἔχει καύχημα, ἀλλ᾽ οὐ πρὸς Θεόν.
\vs{3}τί γὰρ ἡ γραφὴ λέγει; Ἐπίστευσεν δὲ Ἀβραὰμ τῷ Θεῷ καὶ ἐλογίσθη αὐτῷ εἰς δικαιοσύνην.
\vs{4}Τῷ δὲ ἐργαζομένῳ ὁ μισθὸς οὐ λογίζεται κατὰ χάριν ἀλλὰ κατὰ ὀφείλημα,
\vs{5}τῷ δὲ μὴ ἐργαζομένῳ πιστεύοντι δὲ ἐπὶ τὸν δικαιοῦντα τὸν ἀσεβῆ λογίζεται ἡ πίστις αὐτοῦ εἰς δικαιοσύνην·
\vs{6}καθάπερ καὶ Δαυὶδ λέγει τὸν μακαρισμὸν τοῦ ἀνθρώπου ᾧ ὁ Θεὸς λογίζεται δικαιοσύνην χωρὶς ἔργων·
\begin{poetryblock}

\begin{quote} \vs{7}Μακάριοι ὧν ἀφέθησαν αἱ ἀνομίαι\end{quote} 

\begin{quote}καὶ ὧν ἐπεκαλύφθησαν αἱ ἁμαρτίαι·\end{quote}

\begin{quote} \vs{8}μακάριος ἀνὴρ οὗ οὐ μὴ λογίσηται Κύριος ἁμαρτίαν.\end{quote}
\end{poetryblock}

\vs{9}Ὁ μακαρισμὸς οὖν οὗτος ἐπὶ τὴν περιτομὴν ἢ καὶ ἐπὶ τὴν ἀκροβυστίαν; λέγομεν γάρ· Ἐλογίσθη τῷ Ἀβραὰμ ἡ πίστις εἰς δικαιοσύνην.
\vs{10}πῶς οὖν ἐλογίσθη; ἐν περιτομῇ ὄντι ἢ ἐν ἀκροβυστίᾳ; οὐκ ἐν περιτομῇ ἀλλ᾽ ἐν ἀκροβυστίᾳ·
\vs{11}Καὶ σημεῖον ἔλαβεν περιτομῆς σφραγῖδα τῆς δικαιοσύνης τῆς πίστεως τῆς ἐν τῇ ἀκροβυστίᾳ, εἰς τὸ εἶναι αὐτὸν πατέρα πάντων τῶν πιστευόντων δι᾽ ἀκροβυστίας, εἰς τὸ λογισθῆναι καὶ αὐτοῖς τὴν δικαιοσύνην,
\vs{12}καὶ πατέρα περιτομῆς τοῖς οὐκ ἐκ περιτομῆς μόνον ἀλλὰ καὶ τοῖς στοιχοῦσιν τοῖς ἴχνεσιν τῆς ἐν ἀκροβυστίᾳ πίστεως τοῦ πατρὸς ἡμῶν Ἀβραάμ.

\vs{13}Οὐ γὰρ διὰ νόμου ἡ ἐπαγγελία τῷ Ἀβραὰμ ἢ τῷ σπέρματι αὐτοῦ, τὸ κληρονόμον αὐτὸν εἶναι κόσμου, ἀλλὰ διὰ δικαιοσύνης πίστεως.
\vs{14}εἰ γὰρ οἱ ἐκ νόμου κληρονόμοι, κεκένωται ἡ πίστις καὶ κατήργηται ἡ ἐπαγγελία·
\vs{15}ὁ γὰρ νόμος ὀργὴν κατεργάζεται· οὗ δὲ οὐκ ἔστιν νόμος οὐδὲ παράβασις.
\vs{16}Διὰ τοῦτο ἐκ πίστεως, ἵνα κατὰ χάριν, εἰς τὸ εἶναι βεβαίαν τὴν ἐπαγγελίαν παντὶ τῷ σπέρματι, οὐ τῷ ἐκ τοῦ νόμου μόνον ἀλλὰ καὶ τῷ ἐκ πίστεως Ἀβραάμ, ὅς ἐστιν πατὴρ πάντων ἡμῶν,
\vs{17}καθὼς γέγραπται ὅτι Πατέρα πολλῶν ἐθνῶν τέθεικά σε, κατέναντι οὗ ἐπίστευσεν Θεοῦ τοῦ ζωοποιοῦντος τοὺς νεκροὺς καὶ καλοῦντος τὰ μὴ ὄντα ὡς ὄντα.
\vs{18}ὃς παρ᾽ ἐλπίδα ἐπ᾽ ἐλπίδι ἐπίστευσεν εἰς τὸ γενέσθαι αὐτὸν πατέρα πολλῶν ἐθνῶν κατὰ τὸ εἰρημένον· Οὕτως ἔσται τὸ σπέρμα σου,
\vs{19}καὶ μὴ ἀσθενήσας τῇ πίστει κατενόησεν τὸ ἑαυτοῦ σῶμα ἤδη νενεκρωμένον, ἑκατονταετής που ὑπάρχων, καὶ τὴν νέκρωσιν τῆς μήτρας Σάρρας·
\vs{20}εἰς δὲ τὴν ἐπαγγελίαν τοῦ Θεοῦ οὐ διεκρίθη τῇ ἀπιστίᾳ ἀλλὰ ἐνεδυναμώθη τῇ πίστει, δοὺς δόξαν τῷ Θεῷ
\vs{21}καὶ πληροφορηθεὶς ὅτι ὃ ἐπήγγελται δυνατός ἐστιν καὶ ποιῆσαι.
\vs{22}διὸ καὶ Ἐλογίσθη αὐτῷ εἰς δικαιοσύνην.
\vs{23}Οὐκ ἐγράφη δὲ δι᾽ αὐτὸν μόνον ὅτι Ἐλογίσθη αὐτῷ
\vs{24}ἀλλὰ καὶ δι᾽ ἡμᾶς, οἷς μέλλει λογίζεσθαι, τοῖς πιστεύουσιν ἐπὶ τὸν ἐγείραντα Ἰησοῦν τὸν Κύριον ἡμῶν ἐκ νεκρῶν,
\vs{25}ὃς παρεδόθη διὰ τὰ παραπτώματα ἡμῶν καὶ ἠγέρθη διὰ τὴν δικαίωσιν ἡμῶν.

\ch{5}
Δικαιωθέντες οὖν ἐκ πίστεως εἰρήνην ἔχομεν πρὸς τὸν Θεὸν διὰ τοῦ Κυρίου ἡμῶν Ἰησοῦ Χριστοῦ
\vs{2}δι᾽ οὗ καὶ τὴν προσαγωγὴν ἐσχήκαμεν τῇ πίστει εἰς τὴν χάριν ταύτην ἐν ᾗ ἑστήκαμεν καὶ καυχώμεθα ἐπ᾽ ἐλπίδι τῆς δόξης τοῦ Θεοῦ.
\vs{3}Οὐ μόνον δέ, ἀλλὰ καὶ καυχώμεθα ἐν ταῖς θλίψεσιν, εἰδότες ὅτι ἡ θλῖψις ὑπομονὴν κατεργάζεται,
\vs{4}ἡ δὲ ὑπομονὴ δοκιμήν, ἡ δὲ δοκιμὴ ἐλπίδα.
\vs{5}ἡ δὲ ἐλπὶς οὐ καταισχύνει, ὅτι ἡ ἀγάπη τοῦ Θεοῦ ἐκκέχυται ἐν ταῖς καρδίαις ἡμῶν διὰ Πνεύματος Ἁγίου τοῦ δοθέντος ἡμῖν.
\vs{6}Ἔτι γὰρ Χριστὸς ὄντων ἡμῶν ἀσθενῶν ἔτι κατὰ καιρὸν ὑπὲρ ἀσεβῶν ἀπέθανεν.
\vs{7}μόλις γὰρ ὑπὲρ δικαίου τις ἀποθανεῖται· ὑπὲρ γὰρ τοῦ ἀγαθοῦ τάχα τις καὶ τολμᾷ ἀποθανεῖν·
\vs{8}συνίστησιν δὲ τὴν ἑαυτοῦ ἀγάπην εἰς ἡμᾶς ὁ Θεὸς, ὅτι ἔτι ἁμαρτωλῶν ὄντων ἡμῶν Χριστὸς ὑπὲρ ἡμῶν ἀπέθανεν.
\vs{9}Πολλῷ οὖν μᾶλλον δικαιωθέντες νῦν ἐν τῷ αἵματι αὐτοῦ σωθησόμεθα δι᾽ αὐτοῦ ἀπὸ τῆς ὀργῆς.
\vs{10}εἰ γὰρ ἐχθροὶ ὄντες κατηλλάγημεν τῷ Θεῷ διὰ τοῦ θανάτου τοῦ Υἱοῦ αὐτοῦ, πολλῷ μᾶλλον καταλλαγέντες σωθησόμεθα ἐν τῇ ζωῇ αὐτοῦ·
\vs{11}οὐ μόνον δέ, ἀλλὰ καὶ καυχώμενοι ἐν τῷ Θεῷ διὰ τοῦ Κυρίου ἡμῶν Ἰησοῦ Χριστοῦ δι᾽ οὗ νῦν τὴν καταλλαγὴν ἐλάβομεν.

\vs{12}Διὰ τοῦτο ὥσπερ δι᾽ ἑνὸς ἀνθρώπου ἡ ἁμαρτία εἰς τὸν κόσμον εἰσῆλθεν καὶ διὰ τῆς ἁμαρτίας ὁ θάνατος, καὶ οὕτως εἰς πάντας ἀνθρώπους ὁ θάνατος διῆλθεν, ἐφ᾽ ᾧ πάντες ἥμαρτον·
\vs{13}ἄχρι γὰρ νόμου ἁμαρτία ἦν ἐν κόσμῳ, ἁμαρτία δὲ οὐκ ἐλλογεῖται μὴ ὄντος νόμου,
\vs{14}ἀλλὰ ἐβασίλευσεν ὁ θάνατος ἀπὸ Ἀδὰμ μέχρι Μωϋσέως καὶ ἐπὶ τοὺς μὴ ἁμαρτήσαντας ἐπὶ τῷ ὁμοιώματι τῆς παραβάσεως Ἀδάμ ὅς ἐστιν τύπος τοῦ μέλλοντος.
\vs{15}Ἀλλ᾽ οὐχ ὡς τὸ παράπτωμα, οὕτως καὶ τὸ χάρισμα· εἰ γὰρ τῷ τοῦ ἑνὸς παραπτώματι οἱ πολλοὶ ἀπέθανον, πολλῷ μᾶλλον ἡ χάρις τοῦ Θεοῦ καὶ ἡ δωρεὰ ἐν χάριτι τῇ τοῦ ἑνὸς ἀνθρώπου Ἰησοῦ Χριστοῦ εἰς τοὺς πολλοὺς ἐπερίσσευσεν.
\vs{16}καὶ οὐχ ὡς δι᾽ ἑνὸς ἁμαρτήσαντος τὸ δώρημα· τὸ μὲν γὰρ κρίμα ἐξ ἑνὸς εἰς κατάκριμα, τὸ δὲ χάρισμα ἐκ πολλῶν παραπτωμάτων εἰς δικαίωμα.
\vs{17}εἰ γὰρ τῷ τοῦ ἑνὸς παραπτώματι ὁ θάνατος ἐβασίλευσεν διὰ τοῦ ἑνός, πολλῷ μᾶλλον οἱ τὴν περισσείαν τῆς χάριτος καὶ τῆς δωρεᾶς τῆς δικαιοσύνης λαμβάνοντες ἐν ζωῇ βασιλεύσουσιν διὰ τοῦ ἑνὸς Ἰησοῦ Χριστοῦ.
\vs{18}Ἄρα οὖν ὡς δι᾽ ἑνὸς παραπτώματος εἰς πάντας ἀνθρώπους εἰς κατάκριμα, οὕτως καὶ δι᾽ ἑνὸς δικαιώματος εἰς πάντας ἀνθρώπους εἰς δικαίωσιν ζωῆς·
\vs{19}ὥσπερ γὰρ διὰ τῆς παρακοῆς τοῦ ἑνὸς ἀνθρώπου ἁμαρτωλοὶ κατεστάθησαν οἱ πολλοί, οὕτως καὶ διὰ τῆς ὑπακοῆς τοῦ ἑνὸς δίκαιοι κατασταθήσονται οἱ πολλοί.
\vs{20}Νόμος δὲ παρεισῆλθεν, ἵνα πλεονάσῃ τὸ παράπτωμα· οὗ δὲ ἐπλεόνασεν ἡ ἁμαρτία, ὑπερεπερίσσευσεν ἡ χάρις,
\vs{21}ἵνα ὥσπερ ἐβασίλευσεν ἡ ἁμαρτία ἐν τῷ θανάτῳ, οὕτως καὶ ἡ χάρις βασιλεύσῃ διὰ δικαιοσύνης εἰς ζωὴν αἰώνιον διὰ Ἰησοῦ Χριστοῦ τοῦ Κυρίου ἡμῶν.

\ch{6}
Τί οὖν ἐροῦμεν; ἐπιμένωμεν τῇ ἁμαρτίᾳ, ἵνα ἡ χάρις πλεονάσῃ;
\vs{2}μὴ γένοιτο. οἵτινες ἀπεθάνομεν τῇ ἁμαρτίᾳ, πῶς ἔτι ζήσομεν ἐν αὐτῇ;
\vs{3}ἢ ἀγνοεῖτε ὅτι, ὅσοι ἐβαπτίσθημεν εἰς Χριστὸν Ἰησοῦν, εἰς τὸν θάνατον αὐτοῦ ἐβαπτίσθημεν;
\vs{4}συνετάφημεν οὖν αὐτῷ διὰ τοῦ βαπτίσματος εἰς τὸν θάνατον, ἵνα ὥσπερ ἠγέρθη Χριστὸς ἐκ νεκρῶν διὰ τῆς δόξης τοῦ Πατρός, οὕτως καὶ ἡμεῖς ἐν καινότητι ζωῆς περιπατήσωμεν.
\vs{5}Εἰ γὰρ σύμφυτοι γεγόναμεν τῷ ὁμοιώματι τοῦ θανάτου αὐτοῦ, ἀλλὰ καὶ τῆς ἀναστάσεως ἐσόμεθα·
\vs{6}τοῦτο γινώσκοντες ὅτι ὁ παλαιὸς ἡμῶν ἄνθρωπος συνεσταυρώθη, ἵνα καταργηθῇ τὸ σῶμα τῆς ἁμαρτίας, τοῦ μηκέτι δουλεύειν ἡμᾶς τῇ ἁμαρτίᾳ·
\vs{7}ὁ γὰρ ἀποθανὼν δεδικαίωται ἀπὸ τῆς ἁμαρτίας.
\vs{8}Εἰ δὲ ἀπεθάνομεν σὺν Χριστῷ, πιστεύομεν ὅτι καὶ συζήσομεν αὐτῷ,
\vs{9}εἰδότες ὅτι Χριστὸς ἐγερθεὶς ἐκ νεκρῶν οὐκέτι ἀποθνῄσκει, θάνατος αὐτοῦ οὐκέτι κυριεύει.
\vs{10}ὃ γὰρ ἀπέθανεν, τῇ ἁμαρτίᾳ ἀπέθανεν ἐφάπαξ· ὃ δὲ ζῇ, ζῇ τῷ Θεῷ.
\vs{11}οὕτως καὶ ὑμεῖς λογίζεσθε ἑαυτοὺς εἶναι νεκροὺς μὲν τῇ ἁμαρτίᾳ ζῶντας δὲ τῷ Θεῷ ἐν Χριστῷ Ἰησοῦ.
\vs{12}Μὴ οὖν βασιλευέτω ἡ ἁμαρτία ἐν τῷ θνητῷ ὑμῶν σώματι εἰς τὸ ὑπακούειν ταῖς ἐπιθυμίαις αὐτοῦ,
\vs{13}μηδὲ παριστάνετε τὰ μέλη ὑμῶν ὅπλα ἀδικίας τῇ ἁμαρτίᾳ, ἀλλὰ παραστήσατε ἑαυτοὺς τῷ Θεῷ ὡσεὶ ἐκ νεκρῶν ζῶντας καὶ τὰ μέλη ὑμῶν ὅπλα δικαιοσύνης τῷ Θεῷ.
\vs{14}ἁμαρτία γὰρ ὑμῶν οὐ κυριεύσει· οὐ γάρ ἐστε ὑπὸ νόμον ἀλλὰ ὑπὸ χάριν.
\vs{15}Τί οὖν; ἁμαρτήσωμεν, ὅτι οὐκ ἐσμὲν ὑπὸ νόμον ἀλλὰ ὑπὸ χάριν; μὴ γένοιτο.
\vs{16}οὐκ οἴδατε ὅτι ᾧ παριστάνετε ἑαυτοὺς δούλους εἰς ὑπακοήν, δοῦλοί ἐστε ᾧ ὑπακούετε, ἤτοι ἁμαρτίας εἰς θάνατον ἢ ὑπακοῆς εἰς δικαιοσύνην;
\vs{17}χάρις δὲ τῷ Θεῷ ὅτι ἦτε δοῦλοι τῆς ἁμαρτίας ὑπηκούσατε δὲ ἐκ καρδίας εἰς ὃν παρεδόθητε τύπον διδαχῆς,
\vs{18}ἐλευθερωθέντες δὲ ἀπὸ τῆς ἁμαρτίας ἐδουλώθητε τῇ δικαιοσύνῃ.
\vs{19}Ἀνθρώπινον λέγω διὰ τὴν ἀσθένειαν τῆς σαρκὸς ὑμῶν. ὥσπερ γὰρ παρεστήσατε τὰ μέλη ὑμῶν δοῦλα τῇ ἀκαθαρσίᾳ καὶ τῇ ἀνομίᾳ εἰς τὴν ἀνομίαν, οὕτως νῦν παραστήσατε τὰ μέλη ὑμῶν δοῦλα τῇ δικαιοσύνῃ εἰς ἁγιασμόν.
\vs{20}Ὅτε γὰρ δοῦλοι ἦτε τῆς ἁμαρτίας, ἐλεύθεροι ἦτε τῇ δικαιοσύνῃ.
\vs{21}τίνα οὖν καρπὸν εἴχετε τότε; ἐφ᾽ οἷς νῦν ἐπαισχύνεσθε, τὸ γὰρ τέλος ἐκείνων θάνατος.
\vs{22}νυνὶ δέ ἐλευθερωθέντες ἀπὸ τῆς ἁμαρτίας δουλωθέντες δὲ τῷ Θεῷ ἔχετε τὸν καρπὸν ὑμῶν εἰς ἁγιασμόν, τὸ δὲ τέλος ζωὴν αἰώνιον.
\vs{23}τὰ γὰρ ὀψώνια τῆς ἁμαρτίας θάνατος, τὸ δὲ χάρισμα τοῦ Θεοῦ ζωὴ αἰώνιος ἐν Χριστῷ Ἰησοῦ τῷ Κυρίῳ ἡμῶν.

\ch{7}
Ἢ ἀγνοεῖτε, ἀδελφοί, γινώσκουσιν γὰρ νόμον λαλῶ, ὅτι ὁ νόμος κυριεύει τοῦ ἀνθρώπου ἐφ᾽ ὅσον χρόνον ζῇ;
\vs{2}ἡ γὰρ ὕπανδρος γυνὴ τῷ ζῶντι ἀνδρὶ δέδεται νόμῳ· ἐὰν δὲ ἀποθάνῃ ὁ ἀνήρ, κατήργηται ἀπὸ τοῦ νόμου τοῦ ἀνδρός.
\vs{3}ἄρα οὖν ζῶντος τοῦ ἀνδρὸς μοιχαλὶς χρηματίσει ἐὰν γένηται ἀνδρὶ ἑτέρῳ· ἐὰν δὲ ἀποθάνῃ ὁ ἀνήρ, ἐλευθέρα ἐστὶν ἀπὸ τοῦ νόμου, τοῦ μὴ εἶναι αὐτὴν μοιχαλίδα γενομένην ἀνδρὶ ἑτέρῳ.
\vs{4}Ὥστε, ἀδελφοί μου, καὶ ὑμεῖς ἐθανατώθητε τῷ νόμῳ διὰ τοῦ σώματος τοῦ Χριστοῦ, εἰς τὸ γενέσθαι ὑμᾶς ἑτέρῳ, τῷ ἐκ νεκρῶν ἐγερθέντι, ἵνα καρποφορήσωμεν τῷ Θεῷ.
\vs{5}ὅτε γὰρ ἦμεν ἐν τῇ σαρκί, τὰ παθήματα τῶν ἁμαρτιῶν τὰ διὰ τοῦ νόμου ἐνηργεῖτο ἐν τοῖς μέλεσιν ἡμῶν, εἰς τὸ καρποφορῆσαι τῷ θανάτῳ·
\vs{6}νυνὶ δὲ κατηργήθημεν ἀπὸ τοῦ νόμου ἀποθανόντες ἐν ᾧ κατειχόμεθα, ὥστε δουλεύειν ἡμᾶς ἐν καινότητι πνεύματος καὶ οὐ παλαιότητι γράμματος.

\vs{7}Τί οὖν ἐροῦμεν; ὁ νόμος ἁμαρτία; μὴ γένοιτο· ἀλλὰ τὴν ἁμαρτίαν οὐκ ἔγνων εἰ μὴ διὰ νόμου· τήν τε γὰρ ἐπιθυμίαν οὐκ ᾔδειν εἰ μὴ ὁ νόμος ἔλεγεν· Οὐκ ἐπιθυμήσεις.
\vs{8}ἀφορμὴν δὲ λαβοῦσα ἡ ἁμαρτία διὰ τῆς ἐντολῆς κατειργάσατο ἐν ἐμοὶ πᾶσαν ἐπιθυμίαν· χωρὶς γὰρ νόμου ἁμαρτία νεκρά.
\vs{9}Ἐγὼ δὲ ἔζων χωρὶς νόμου ποτέ, ἐλθούσης δὲ τῆς ἐντολῆς ἡ ἁμαρτία ἀνέζησεν,
\vs{10}ἐγὼ δὲ ἀπέθανον καὶ εὑρέθη μοι ἡ ἐντολὴ ἡ εἰς ζωὴν, αὕτη εἰς θάνατον·
\vs{11}ἡ γὰρ ἁμαρτία ἀφορμὴν λαβοῦσα διὰ τῆς ἐντολῆς ἐξηπάτησέν με καὶ δι᾽ αὐτῆς ἀπέκτεινεν.
\vs{12}Ὥστε ὁ μὲν νόμος ἅγιος καὶ ἡ ἐντολὴ ἁγία καὶ δικαία καὶ ἀγαθή.
\vs{13}Τὸ οὖν ἀγαθὸν ἐμοὶ ἐγένετο θάνατος; μὴ γένοιτο· ἀλλὰ ἡ ἁμαρτία, ἵνα φανῇ ἁμαρτία, διὰ τοῦ ἀγαθοῦ μοι κατεργαζομένη θάνατον, ἵνα γένηται καθ᾽ ὑπερβολὴν ἁμαρτωλὸς ἡ ἁμαρτία διὰ τῆς ἐντολῆς.

\vs{14}Οἴδαμεν γὰρ ὅτι ὁ νόμος πνευματικός ἐστιν, ἐγὼ δὲ σάρκινός εἰμι πεπραμένος ὑπὸ τὴν ἁμαρτίαν.
\vs{15}ὃ γὰρ κατεργάζομαι οὐ γινώσκω· οὐ γὰρ ὃ θέλω τοῦτο πράσσω, ἀλλ᾽ ὃ μισῶ τοῦτο ποιῶ.
\vs{16}εἰ δὲ ὃ οὐ θέλω τοῦτο ποιῶ, σύμφημι τῷ νόμῳ ὅτι καλός.
\vs{17}νυνὶ δὲ οὐκέτι ἐγὼ κατεργάζομαι αὐτὸ ἀλλὰ ἡ οἰκοῦσα ἐν ἐμοὶ ἁμαρτία.
\vs{18}Οἶδα γὰρ ὅτι οὐκ οἰκεῖ ἐν ἐμοί, τοῦτ᾽ ἔστιν ἐν τῇ σαρκί μου, ἀγαθόν· τὸ γὰρ θέλειν παράκειταί μοι, τὸ δὲ κατεργάζεσθαι τὸ καλὸν οὔ·
\vs{19}οὐ γὰρ ὃ θέλω ποιῶ ἀγαθόν, ἀλλὰ ὃ οὐ θέλω κακὸν τοῦτο πράσσω.
\vs{20}εἰ δὲ ὃ οὐ θέλω ἐγὼ τοῦτο ποιῶ, οὐκέτι ἐγὼ κατεργάζομαι αὐτὸ ἀλλὰ ἡ οἰκοῦσα ἐν ἐμοὶ ἁμαρτία.
\vs{21}Εὑρίσκω ἄρα τὸν νόμον, τῷ θέλοντι ἐμοὶ ποιεῖν τὸ καλὸν, ὅτι ἐμοὶ τὸ κακὸν παράκειται·
\vs{22}συνήδομαι γὰρ τῷ νόμῳ τοῦ Θεοῦ κατὰ τὸν ἔσω ἄνθρωπον,
\vs{23}βλέπω δὲ ἕτερον νόμον ἐν τοῖς μέλεσίν μου ἀντιστρατευόμενον τῷ νόμῳ τοῦ νοός μου καὶ αἰχμαλωτίζοντά με ἐν τῷ νόμῳ τῆς ἁμαρτίας τῷ ὄντι ἐν τοῖς μέλεσίν μου.
\vs{24}Ταλαίπωρος ἐγὼ ἄνθρωπος· τίς με ῥύσεται ἐκ τοῦ σώματος τοῦ θανάτου τούτου;
\vs{25}χάρις δὲ τῷ Θεῷ διὰ Ἰησοῦ Χριστοῦ τοῦ Κυρίου ἡμῶν. Ἄρα οὖν αὐτὸς ἐγὼ τῷ μὲν νοῒ δουλεύω νόμῳ Θεοῦ τῇ δὲ σαρκὶ νόμῳ ἁμαρτίας.

\ch{8}
Οὐδὲν ἄρα νῦν κατάκριμα τοῖς ἐν Χριστῷ Ἰησοῦ.
\vs{2}ὁ γὰρ νόμος τοῦ Πνεύματος τῆς ζωῆς ἐν Χριστῷ Ἰησοῦ ἠλευθέρωσέν σε ἀπὸ τοῦ νόμου τῆς ἁμαρτίας καὶ τοῦ θανάτου.
\vs{3}τὸ γὰρ ἀδύνατον τοῦ νόμου ἐν ᾧ ἠσθένει διὰ τῆς σαρκός, ὁ Θεὸς τὸν ἑαυτοῦ Υἱὸν πέμψας ἐν ὁμοιώματι σαρκὸς ἁμαρτίας καὶ περὶ ἁμαρτίας κατέκρινεν τὴν ἁμαρτίαν ἐν τῇ σαρκί,
\vs{4}ἵνα τὸ δικαίωμα τοῦ νόμου πληρωθῇ ἐν ἡμῖν τοῖς μὴ κατὰ σάρκα περιπατοῦσιν ἀλλὰ κατὰ πνεῦμα.
\vs{5}Οἱ γὰρ κατὰ σάρκα ὄντες τὰ τῆς σαρκὸς φρονοῦσιν, οἱ δὲ κατὰ πνεῦμα τὰ τοῦ πνεύματος.
\vs{6}τὸ γὰρ φρόνημα τῆς σαρκὸς θάνατος, τὸ δὲ φρόνημα τοῦ πνεύματος ζωὴ καὶ εἰρήνη·
\vs{7}διότι τὸ φρόνημα τῆς σαρκὸς ἔχθρα εἰς Θεόν, τῷ γὰρ νόμῳ τοῦ Θεοῦ οὐχ ὑποτάσσεται, οὐδὲ γὰρ δύναται·
\vs{8}οἱ δὲ ἐν σαρκὶ ὄντες Θεῷ ἀρέσαι οὐ δύνανται.
\vs{9}Ὑμεῖς δὲ οὐκ ἐστὲ ἐν σαρκὶ ἀλλὰ ἐν πνεύματι, εἴπερ Πνεῦμα Θεοῦ οἰκεῖ ἐν ὑμῖν. εἰ δέ τις Πνεῦμα Χριστοῦ οὐκ ἔχει, οὗτος οὐκ ἔστιν αὐτοῦ.
\vs{10}εἰ δὲ Χριστὸς ἐν ὑμῖν, τὸ μὲν σῶμα νεκρὸν διὰ ἁμαρτίαν τὸ δὲ πνεῦμα ζωὴ διὰ δικαιοσύνην.
\vs{11}εἰ δὲ τὸ Πνεῦμα τοῦ ἐγείραντος τὸν Ἰησοῦν ἐκ νεκρῶν οἰκεῖ ἐν ὑμῖν, ὁ ἐγείρας Χριστὸν ἐκ νεκρῶν ζωοποιήσει καὶ τὰ θνητὰ σώματα ὑμῶν διὰ τοῦ ἐνοικοῦντος αὐτοῦ Πνεύματος ἐν ὑμῖν.

\vs{12}Ἄρα οὖν, ἀδελφοί, ὀφειλέται ἐσμέν οὐ τῇ σαρκὶ τοῦ κατὰ σάρκα ζῆν,
\vs{13}εἰ γὰρ κατὰ σάρκα ζῆτε, μέλλετε ἀποθνήσκειν· εἰ δὲ πνεύματι τὰς πράξεις τοῦ σώματος θανατοῦτε, ζήσεσθε.
\vs{14}ὅσοι γὰρ Πνεύματι Θεοῦ ἄγονται, οὗτοι υἱοί Θεοῦ εἰσιν.
\vs{15}Οὐ γὰρ ἐλάβετε πνεῦμα δουλείας πάλιν εἰς φόβον ἀλλὰ ἐλάβετε πνεῦμα υἱοθεσίας ἐν ᾧ κράζομεν· Ἀββᾶ ὁ Πατήρ.
\vs{16}αὐτὸ τὸ Πνεῦμα συμμαρτυρεῖ τῷ πνεύματι ἡμῶν ὅτι ἐσμὲν τέκνα Θεοῦ.
\vs{17}εἰ δὲ τέκνα, καὶ κληρονόμοι· κληρονόμοι μὲν Θεοῦ, συνκληρονόμοι δὲ Χριστοῦ, εἴπερ συμπάσχομεν ἵνα καὶ συνδοξασθῶμεν.

\vs{18}Λογίζομαι γὰρ ὅτι οὐκ ἄξια τὰ παθήματα τοῦ νῦν καιροῦ πρὸς τὴν μέλλουσαν δόξαν ἀποκαλυφθῆναι εἰς ἡμᾶς.
\vs{19}ἡ γὰρ ἀποκαραδοκία τῆς κτίσεως τὴν ἀποκάλυψιν τῶν υἱῶν τοῦ Θεοῦ ἀπεκδέχεται.
\vs{20}τῇ γὰρ ματαιότητι ἡ κτίσις ὑπετάγη, οὐχ ἑκοῦσα ἀλλὰ διὰ τὸν ὑποτάξαντα, ἐφ᾽ ἑλπίδι
\vs{21}ὅτι καὶ αὐτὴ ἡ κτίσις ἐλευθερωθήσεται ἀπὸ τῆς δουλείας τῆς φθορᾶς εἰς τὴν ἐλευθερίαν τῆς δόξης τῶν τέκνων τοῦ Θεοῦ.
\vs{22}Οἴδαμεν γὰρ ὅτι πᾶσα ἡ κτίσις συστενάζει καὶ συνωδίνει ἄχρι τοῦ νῦν·
\vs{23}οὐ μόνον δέ, ἀλλὰ καὶ αὐτοὶ τὴν ἀπαρχὴν τοῦ Πνεύματος ἔχοντες, ἡμεῖς καὶ αὐτοὶ ἐν ἑαυτοῖς στενάζομεν υἱοθεσίαν ἀπεκδεχόμενοι, τὴν ἀπολύτρωσιν τοῦ σώματος ἡμῶν.
\vs{24}τῇ γὰρ ἐλπίδι ἐσώθημεν· ἐλπὶς δὲ βλεπομένη οὐκ ἔστιν ἐλπίς· ὃ γὰρ βλέπει τις ἐλπίζει;
\vs{25}εἰ δὲ ὃ οὐ βλέπομεν ἐλπίζομεν, δι᾽ ὑπομονῆς ἀπεκδεχόμεθα.
\vs{26}Ὡσαύτως δὲ καὶ τὸ Πνεῦμα συναντιλαμβάνεται τῇ ἀσθενείᾳ ἡμῶν· τὸ γὰρ τί προσευξώμεθα καθὸ δεῖ οὐκ οἴδαμεν, ἀλλὰ αὐτὸ τὸ Πνεῦμα ὑπερεντυγχάνει στεναγμοῖς ἀλαλήτοις·
\vs{27}ὁ δὲ ἐραυνῶν τὰς καρδίας οἶδεν τί τὸ φρόνημα τοῦ Πνεύματος, ὅτι κατὰ Θεὸν ἐντυγχάνει ὑπὲρ ἁγίων.
\vs{28}Οἴδαμεν δὲ ὅτι τοῖς ἀγαπῶσιν τὸν Θεὸν πάντα συνεργεῖ εἰς ἀγαθόν, τοῖς κατὰ πρόθεσιν κλητοῖς οὖσιν.
\vs{29}ὅτι οὓς προέγνω, καὶ προώρισεν συμμόρφους τῆς εἰκόνος τοῦ Υἱοῦ αὐτοῦ, εἰς τὸ εἶναι αὐτὸν πρωτότοκον ἐν πολλοῖς ἀδελφοῖς·
\vs{30}οὓς δὲ προώρισεν, τούτους καὶ ἐκάλεσεν· καὶ οὓς ἐκάλεσεν, τούτους καὶ ἐδικαίωσεν· οὓς δὲ ἐδικαίωσεν, τούτους καὶ ἐδόξασεν.

\vs{31}Τί οὖν ἐροῦμεν πρὸς ταῦτα; εἰ ὁ Θεὸς ὑπὲρ ἡμῶν, τίς καθ᾽ ἡμῶν;
\vs{32}ὅς γε τοῦ ἰδίου Υἱοῦ οὐκ ἐφείσατο ἀλλὰ ὑπὲρ ἡμῶν πάντων παρέδωκεν αὐτόν, πῶς οὐχὶ καὶ σὺν αὐτῷ τὰ πάντα ἡμῖν χαρίσεται;
\vs{33}τίς ἐγκαλέσει κατὰ ἐκλεκτῶν Θεοῦ; Θεὸς ὁ δικαιῶν·
\vs{34}τίς ὁ κατακρινῶν; Χριστὸς Ἰησοῦς ὁ ἀποθανών, μᾶλλον δὲ ἐγερθείς, ὅς καί ἐστιν ἐν δεξιᾷ τοῦ Θεοῦ, ὃς καὶ ἐντυγχάνει ὑπὲρ ἡμῶν.
\vs{35}Τίς ἡμᾶς χωρίσει ἀπὸ τῆς ἀγάπης τοῦ Χριστοῦ; θλῖψις ἢ στενοχωρία ἢ διωγμὸς ἢ λιμὸς ἢ γυμνότης ἢ κίνδυνος ἢ μάχαιρα;
\vs{36}καθὼς γέγραπται ὅτι 
\begin{poetryblock}

\begin{quote}Ἕνεκεν σοῦ θανατούμεθα ὅλην τὴν ἡμέραν,\end{quote} 

\begin{quote}ἐλογίσθημεν ὡς πρόβατα σφαγῆς.\end{quote}
\end{poetryblock}

\vs{37}Ἀλλ᾽ ἐν τούτοις πᾶσιν ὑπερνικῶμεν διὰ τοῦ ἀγαπήσαντος ἡμᾶς.
\vs{38}πέπεισμαι γὰρ ὅτι οὔτε θάνατος οὔτε ζωὴ οὔτε ἄγγελοι οὔτε ἀρχαὶ οὔτε ἐνεστῶτα οὔτε μέλλοντα οὔτε δυνάμεις
\vs{39}οὔτε ὕψωμα οὔτε βάθος οὔτε τις κτίσις ἑτέρα δυνήσεται ἡμᾶς χωρίσαι ἀπὸ τῆς ἀγάπης τοῦ Θεοῦ τῆς ἐν Χριστῷ Ἰησοῦ τῷ Κυρίῳ ἡμῶν.

\ch{9}
Ἀλήθειαν λέγω ἐν Χριστῷ, οὐ ψεύδομαι, συμμαρτυρούσης μοι τῆς συνειδήσεώς μου ἐν Πνεύματι Ἁγίῳ,
\vs{2}ὅτι λύπη μοί ἐστιν μεγάλη καὶ ἀδιάλειπτος ὀδύνη τῇ καρδίᾳ μου.
\vs{3}ηὐχόμην γὰρ ἀνάθεμα εἶναι αὐτὸς ἐγὼ ἀπὸ τοῦ Χριστοῦ ὑπὲρ τῶν ἀδελφῶν μου τῶν συγγενῶν μου κατὰ σάρκα,
\vs{4}οἵτινές εἰσιν Ἰσραηλῖται, ὧν ἡ υἱοθεσία καὶ ἡ δόξα καὶ αἱ διαθῆκαι καὶ ἡ νομοθεσία καὶ ἡ λατρεία καὶ αἱ ἐπαγγελίαι,
\vs{5}ὧν οἱ πατέρες καὶ ἐξ ὧν ὁ Χριστὸς τὸ κατὰ σάρκα, ὁ ὢν ἐπὶ πάντων Θεὸς εὐλογητὸς εἰς τοὺς αἰῶνας, ἀμήν.

\vs{6}Οὐχ οἷον δὲ ὅτι ἐκπέπτωκεν ὁ λόγος τοῦ Θεοῦ. οὐ γὰρ πάντες οἱ ἐξ Ἰσραήλ οὗτοι Ἰσραήλ·
\vs{7}οὐδ᾽ ὅτι εἰσὶν σπέρμα Ἀβραάμ πάντες τέκνα, ἀλλ᾽· Ἐν Ἰσαὰκ κληθήσεταί σοι σπέρμα.
\vs{8}τοῦτ᾽ ἔστιν, οὐ τὰ τέκνα τῆς σαρκὸς ταῦτα τέκνα τοῦ Θεοῦ ἀλλὰ τὰ τέκνα τῆς ἐπαγγελίας λογίζεται εἰς σπέρμα.
\vs{9}ἐπαγγελίας γὰρ ὁ λόγος οὗτος· Κατὰ τὸν καιρὸν τοῦτον ἐλεύσομαι καὶ ἔσται τῇ Σάρρᾳ υἱός.

\vs{10}Οὐ μόνον δέ, ἀλλὰ καὶ Ῥεβέκκα ἐξ ἑνὸς κοίτην ἔχουσα, Ἰσαὰκ τοῦ πατρὸς ἡμῶν·
\vs{11}μήπω γὰρ γεννηθέντων μηδὲ πραξάντων τι ἀγαθὸν ἢ φαῦλον, ἵνα ἡ κατ᾽ ἐκλογὴν πρόθεσις τοῦ Θεοῦ μένῃ,
\vs{12}οὐκ ἐξ ἔργων ἀλλ᾽ ἐκ τοῦ καλοῦντος, ἐρρέθη αὐτῇ ὅτι Ὁ μείζων δουλεύσει τῷ ἐλάσσονι,
\vs{13}καθὼς γέγραπται· Τὸν Ἰακὼβ ἠγάπησα, τὸν δὲ Ἠσαῦ ἐμίσησα.

\vs{14}Τί οὖν ἐροῦμεν; μὴ ἀδικία παρὰ τῷ θεῷ; μὴ γένοιτο.
\vs{15}τῷ Μωϋσεῖ γὰρ λέγει· Ἐλεήσω ὃν ἂν ἐλεῶ καὶ οἰκτιρήσω ὃν ἂν οἰκτίρω.
\vs{16}Ἄρα οὖν οὐ τοῦ θέλοντος οὐδὲ τοῦ τρέχοντος ἀλλὰ τοῦ ἐλεῶντος Θεοῦ.
\vs{17}λέγει γὰρ ἡ γραφὴ τῷ Φαραὼ ὅτι Εἰς αὐτὸ τοῦτο ἐξήγειρά σε ὅπως ἐνδείξωμαι ἐν σοὶ τὴν δύναμίν μου καὶ ὅπως διαγγελῇ τὸ ὄνομά μου ἐν πάσῃ τῇ γῇ.
\vs{18}ἄρα οὖν ὃν θέλει ἐλεεῖ, ὃν δὲ θέλει σκληρύνει.
\vs{19}Ἐρεῖς μοι οὖν· Τί οὖν ἔτι μέμφεται; τῷ γὰρ βουλήματι αὐτοῦ τίς ἀνθέστηκεν;
\vs{20}ὦ ἄνθρωπε, μενοῦνγε σὺ τίς εἶ ὁ ἀνταποκρινόμενος τῷ Θεῷ; μὴ ἐρεῖ τὸ πλάσμα τῷ πλάσαντι· Τί με ἐποίησας οὕτως;
\vs{21}ἢ οὐκ ἔχει ἐξουσίαν ὁ κεραμεὺς τοῦ πηλοῦ ἐκ τοῦ αὐτοῦ φυράματος ποιῆσαι ὃ μὲν εἰς τιμὴν σκεῦος ὃ δὲ εἰς ἀτιμίαν;
\vs{22}Εἰ δὲ θέλων ὁ Θεὸς ἐνδείξασθαι τὴν ὀργὴν καὶ γνωρίσαι τὸ δυνατὸν αὐτοῦ ἤνεγκεν ἐν πολλῇ μακροθυμίᾳ σκεύη ὀργῆς κατηρτισμένα εἰς ἀπώλειαν,
\vs{23}καὶ ἵνα γνωρίσῃ τὸν πλοῦτον τῆς δόξης αὐτοῦ ἐπὶ σκεύη ἐλέους ἃ προητοίμασεν εἰς δόξαν;
\vs{24}οὓς καὶ ἐκάλεσεν ἡμᾶς οὐ μόνον ἐξ Ἰουδαίων ἀλλὰ καὶ ἐξ ἐθνῶν,
\vs{25}ὡς καὶ ἐν τῷ Ὡσηὲ λέγει· 
\begin{poetryblock}

\begin{quote}Καλέσω τὸν οὐ λαόν μου λαόν μου\end{quote} 

\begin{quote}καὶ τὴν οὐκ ἠγαπημένην ἠγαπημένην·\end{quote}

\begin{quote} \vs{26}Καὶ Ἔσται ἐν τῷ τόπῳ οὗ ἐρρέθη αὐτοῖς·\end{quote} 

\begin{quote}Οὐ λαός μου ὑμεῖς,\end{quote} 

\begin{quote}ἐκεῖ κληθήσονται Υἱοὶ Θεοῦ ζῶντος.\end{quote}
\end{poetryblock}

\vs{27}Ἠσαΐας δὲ κράζει ὑπὲρ τοῦ Ἰσραήλ· 
\begin{poetryblock}

\begin{quote}Ἐὰν ᾖ ὁ ἀριθμὸς τῶν υἱῶν Ἰσραὴλ ὡς ἡ ἄμμος τῆς θαλάσσης,\end{quote} 

\begin{quote}τὸ ὑπόλειμμα σωθήσεται·\end{quote}

\begin{quote} \vs{28}λόγον γὰρ συντελῶν καὶ συντέμνων\end{quote} 

\begin{quote}ποιήσει Κύριος ἐπὶ τῆς γῆς.\end{quote}
\end{poetryblock}

\vs{29}Καὶ καθὼς προείρηκεν Ἠσαΐας· 
\begin{poetryblock}

\begin{quote}Εἰ μὴ Κύριος Σαβαὼθ ἐγκατέλιπεν ἡμῖν σπέρμα,\end{quote} 

\begin{quote}ὡς Σόδομα ἂν ἐγενήθημεν καὶ ὡς Γόμορρα ἂν ὡμοιώθημεν.\end{quote}
\end{poetryblock}

\vs{30}Τί οὖν ἐροῦμεν; ὅτι ἔθνη τὰ μὴ διώκοντα δικαιοσύνην κατέλαβεν δικαιοσύνην, δικαιοσύνην δὲ τὴν ἐκ πίστεως,
\vs{31}Ἰσραὴλ δὲ διώκων νόμον δικαιοσύνης εἰς νόμον οὐκ ἔφθασεν.
\vs{32}διὰ τί; ὅτι οὐκ ἐκ πίστεως ἀλλ᾽ ὡς ἐξ ἔργων· προσέκοψαν τῷ λίθῳ τοῦ προσκόμματος,
\vs{33}καθὼς γέγραπται· 
\begin{poetryblock}

\begin{quote}Ἰδοὺ τίθημι ἐν Σιὼν λίθον προσκόμματος καὶ πέτραν σκανδάλου,\end{quote} 

\begin{quote}καὶ ὁ πιστεύων ἐπ᾽ αὐτῷ οὐ καταισχυνθήσεται.\end{quote}
\end{poetryblock}

\ch{10}
Ἀδελφοί, ἡ μὲν εὐδοκία τῆς ἐμῆς καρδίας καὶ ἡ δέησις πρὸς τὸν Θεὸν ὑπὲρ αὐτῶν εἰς σωτηρίαν.
\vs{2}μαρτυρῶ γὰρ αὐτοῖς ὅτι ζῆλον Θεοῦ ἔχουσιν ἀλλ᾽ οὐ κατ᾽ ἐπίγνωσιν·
\vs{3}ἀγνοοῦντες γὰρ τὴν τοῦ Θεοῦ δικαιοσύνην καὶ τὴν ἰδίαν δικαιοσύνην ζητοῦντες στῆσαι, τῇ δικαιοσύνῃ τοῦ Θεοῦ οὐχ ὑπετάγησαν.
\vs{4}τέλος γὰρ νόμου Χριστὸς εἰς δικαιοσύνην παντὶ τῷ πιστεύοντι.
\vs{5}Μωϋσῆς γὰρ γράφει τὴν δικαιοσύνην τὴν ἐκ τοῦ νόμου ὅτι Ὁ ποιήσας αὐτὰ ἄνθρωπος ζήσεται ἐν αὐτῇ.
\vs{6}ἡ δὲ ἐκ πίστεως δικαιοσύνη οὕτως λέγει· Μὴ εἴπῃς ἐν τῇ καρδίᾳ σου· Τίς ἀναβήσεται εἰς τὸν οὐρανόν; τοῦτ᾽ ἔστιν Χριστὸν καταγαγεῖν·
\vs{7}ἤ· Τίς καταβήσεται εἰς τὴν ἄβυσσον; τοῦτ᾽ ἔστιν Χριστὸν ἐκ νεκρῶν ἀναγαγεῖν.
\vs{8}Ἀλλὰ τί λέγει; Ἐγγύς σου τὸ ῥῆμά ἐστιν ἐν τῷ στόματί σου καὶ ἐν τῇ καρδίᾳ σου, τοῦτ᾽ ἔστιν τὸ ῥῆμα τῆς πίστεως ὃ κηρύσσομεν.
\vs{9}ὅτι ἐὰν ὁμολογήσῃς ἐν τῷ στόματί σου Κύριον Ἰησοῦν καὶ πιστεύσῃς ἐν τῇ καρδίᾳ σου ὅτι ὁ Θεὸς αὐτὸν ἤγειρεν ἐκ νεκρῶν, σωθήσῃ·
\vs{10}καρδίᾳ γὰρ πιστεύεται εἰς δικαιοσύνην, στόματι δὲ ὁμολογεῖται εἰς σωτηρίαν.
\vs{11}Λέγει γὰρ ἡ γραφή· Πᾶς ὁ πιστεύων ἐπ᾽ αὐτῷ οὐ καταισχυνθήσεται.
\vs{12}οὐ γάρ ἐστιν διαστολὴ Ἰουδαίου τε καὶ Ἕλληνος, ὁ γὰρ αὐτὸς Κύριος πάντων, πλουτῶν εἰς πάντας τοὺς ἐπικαλουμένους αὐτόν·
\vs{13}Πᾶς γὰρ ὃς ἂν ἐπικαλέσηται τὸ ὄνομα Κυρίου σωθήσεται.

\vs{14}Πῶς οὖν ἐπικαλέσωνται εἰς ὃν οὐκ ἐπίστευσαν; πῶς δὲ πιστεύσωσιν οὗ οὐκ ἤκουσαν; πῶς δὲ ἀκούσωσιν χωρὶς κηρύσσοντος;
\vs{15}πῶς δὲ κηρύξωσιν ἐὰν μὴ ἀποσταλῶσιν; καθὼς γέγραπται· Ὡς ὡραῖοι οἱ πόδες τῶν εὐαγγελιζομένων τὰ ἀγαθά.
\vs{16}Ἀλλ᾽ οὐ πάντες ὑπήκουσαν τῷ εὐαγγελίῳ. Ἠσαΐας γὰρ λέγει· Κύριε, τίς ἐπίστευσεν τῇ ἀκοῇ ἡμῶν;
\vs{17}ἄρα ἡ πίστις ἐξ ἀκοῆς, ἡ δὲ ἀκοὴ διὰ ῥήματος Χριστοῦ.
\vs{18}Ἀλλὰ λέγω, μὴ οὐκ ἤκουσαν; μενοῦνγε· 
\begin{poetryblock}

\begin{quote}Εἰς πᾶσαν τὴν γῆν ἐξῆλθεν ὁ φθόγγος αὐτῶν\end{quote} 

\begin{quote}καὶ εἰς τὰ πέρατα τῆς οἰκουμένης τὰ ῥήματα αὐτῶν.\end{quote}
\end{poetryblock}

\vs{19}Ἀλλὰ λέγω, μὴ Ἰσραὴλ οὐκ ἔγνω; πρῶτος Μωϋσῆς λέγει· 
\begin{poetryblock}

\begin{quote}Ἐγὼ παραζηλώσω ὑμᾶς ἐπ᾽ οὐκ ἔθνει,\end{quote} 

\begin{quote}ἐπ᾽ ἔθνει ἀσυνέτῳ παροργιῶ ὑμᾶς.\end{quote}
\end{poetryblock}

\vs{20}Ἠσαΐας δὲ ἀποτολμᾷ καὶ λέγει· 
\begin{poetryblock}

\begin{quote}Εὑρέθην ἐν τοῖς ἐμὲ μὴ ζητοῦσιν,\end{quote} 

\begin{quote}ἐμφανὴς ἐγενόμην τοῖς ἐμὲ μὴ ἐπερωτῶσιν.\end{quote}
\end{poetryblock}

\vs{21}Πρὸς δὲ τὸν Ἰσραὴλ λέγει· 
\begin{poetryblock}

\begin{quote}Ὅλην τὴν ἡμέραν ἐξεπέτασα τὰς χεῖράς μου\end{quote} 

\begin{quote}πρὸς λαὸν ἀπειθοῦντα καὶ ἀντιλέγοντα.\end{quote}
\end{poetryblock}

\ch{11}
Λέγω οὖν, μὴ ἀπώσατο ὁ Θεὸς τὸν λαὸν αὐτοῦ; μὴ γένοιτο· καὶ γὰρ ἐγὼ Ἰσραηλίτης εἰμί, ἐκ σπέρματος Ἀβραάμ, φυλῆς Βενιαμίν.
\vs{2}οὐκ ἀπώσατο ὁ Θεὸς τὸν λαὸν αὐτοῦ ὃν προέγνω. ἢ οὐκ οἴδατε ἐν Ἠλίᾳ τί λέγει ἡ γραφή, ὡς ἐντυγχάνει τῷ Θεῷ κατὰ τοῦ Ἰσραήλ;
\vs{3}Κύριε, τοὺς προφήτας σου ἀπέκτειναν, τὰ θυσιαστήριά σου κατέσκαψαν, κἀγὼ ὑπελείφθην μόνος καὶ ζητοῦσιν τὴν ψυχήν μου.
\vs{4}Ἀλλὰ τί λέγει αὐτῷ ὁ χρηματισμός; Κατέλιπον ἐμαυτῷ ἑπτακισχιλίους ἄνδρας, οἵτινες οὐκ ἔκαμψαν γόνυ τῇ Βάαλ.
\vs{5}Οὕτως οὖν καὶ ἐν τῷ νῦν καιρῷ λεῖμμα κατ᾽ ἐκλογὴν χάριτος γέγονεν·
\vs{6}εἰ δὲ χάριτι, οὐκέτι ἐξ ἔργων, ἐπεὶ ἡ χάρις οὐκέτι γίνεται χάρις.
\vs{7}Τί οὖν; ὃ ἐπιζητεῖ Ἰσραήλ, τοῦτο οὐκ ἐπέτυχεν, ἡ δὲ ἐκλογὴ ἐπέτυχεν· οἱ δὲ λοιποὶ ἐπωρώθησαν,
\vs{8}καθὼς γέγραπται· 
\begin{poetryblock}

\begin{quote}Ἔδωκεν αὐτοῖς ὁ Θεὸς πνεῦμα κατανύξεως,\end{quote} 

\begin{quote}ὀφθαλμοὺς τοῦ μὴ βλέπειν καὶ ὦτα τοῦ μὴ ἀκούειν,\end{quote} 

\begin{quote}ἕως τῆς σήμερον ἡμέρας.\end{quote}
\end{poetryblock}

\vs{9}Καὶ Δαυὶδ λέγει· 
\begin{poetryblock}

\begin{quote}Γενηθήτω ἡ τράπεζα αὐτῶν εἰς παγίδα καὶ εἰς θήραν\end{quote} 

\begin{quote}καὶ εἰς σκάνδαλον καὶ εἰς ἀνταπόδομα αὐτοῖς,\end{quote}

\begin{quote} \vs{10}σκοτισθήτωσαν οἱ ὀφθαλμοὶ αὐτῶν τοῦ μὴ βλέπειν\end{quote} 

\begin{quote}καὶ τὸν νῶτον αὐτῶν διὰ παντὸς σύνκαμψον.\end{quote}
\end{poetryblock}

\vs{11}Λέγω οὖν, μὴ ἔπταισαν ἵνα πέσωσιν; μὴ γένοιτο· ἀλλὰ τῷ αὐτῶν παραπτώματι ἡ σωτηρία τοῖς ἔθνεσιν εἰς τὸ παραζηλῶσαι αὐτούς.
\vs{12}εἰ δὲ τὸ παράπτωμα αὐτῶν πλοῦτος κόσμου καὶ τὸ ἥττημα αὐτῶν πλοῦτος ἐθνῶν, πόσῳ μᾶλλον τὸ πλήρωμα αὐτῶν.
\vs{13}Ὑμῖν δὲ λέγω τοῖς ἔθνεσιν· ἐφ᾽ ὅσον μὲν οὖν εἰμι ἐγὼ ἐθνῶν ἀπόστολος, τὴν διακονίαν μου δοξάζω,
\vs{14}εἴ πως παραζηλώσω μου τὴν σάρκα καὶ σώσω τινὰς ἐξ αὐτῶν.
\vs{15}εἰ γὰρ ἡ ἀποβολὴ αὐτῶν καταλλαγὴ κόσμου, τίς ἡ πρόσλημψις εἰ μὴ ζωὴ ἐκ νεκρῶν;
\vs{16}εἰ δὲ ἡ ἀπαρχὴ ἁγία, καὶ τὸ φύραμα· καὶ εἰ ἡ ῥίζα ἁγία, καὶ οἱ κλάδοι.
\vs{17}Εἰ δέ τινες τῶν κλάδων ἐξεκλάσθησαν, σὺ δὲ ἀγριέλαιος ὢν ἐνεκεντρίσθης ἐν αὐτοῖς καὶ συνκοινωνὸς τῆς ῥίζης τῆς πιότητος τῆς ἐλαίας ἐγένου,
\vs{18}μὴ κατακαυχῶ τῶν κλάδων· εἰ δὲ κατακαυχᾶσαι οὐ σὺ τὴν ῥίζαν βαστάζεις ἀλλὰ ἡ ῥίζα σέ.
\vs{19}Ἐρεῖς οὖν· Ἐξεκλάσθησαν κλάδοι ἵνα ἐγὼ ἐγκεντρισθῶ.
\vs{20}καλῶς· τῇ ἀπιστίᾳ ἐξεκλάσθησαν, σὺ δὲ τῇ πίστει ἕστηκας. μὴ ὑψηλὰ φρόνει ἀλλὰ φοβοῦ·
\vs{21}εἰ γὰρ ὁ Θεὸς τῶν κατὰ φύσιν κλάδων οὐκ ἐφείσατο, μή πως οὐδὲ σοῦ φείσεται.
\vs{22}ἴδε οὖν χρηστότητα καὶ ἀποτομίαν Θεοῦ· ἐπὶ μὲν τοὺς πεσόντας ἀποτομία, ἐπὶ δὲ σὲ χρηστότης Θεοῦ, ἐὰν ἐπιμένῃς τῇ χρηστότητι, ἐπεὶ καὶ σὺ ἐκκοπήσῃ.
\vs{23}κἀκεῖνοι δέ, ἐὰν μὴ ἐπιμένωσιν τῇ ἀπιστίᾳ, ἐνκεντρισθήσονται· δυνατὸς γάρ ἐστιν ὁ Θεὸς πάλιν ἐνκεντρίσαι αὐτούς.
\vs{24}εἰ γὰρ σὺ ἐκ τῆς κατὰ φύσιν ἐξεκόπης ἀγριελαίου καὶ παρὰ φύσιν ἐνεκεντρίσθης εἰς καλλιέλαιον, πόσῳ μᾶλλον οὗτοι οἱ κατὰ φύσιν ἐνκεντρισθήσονται τῇ ἰδίᾳ ἐλαίᾳ.

\vs{25}Οὐ γὰρ θέλω ὑμᾶς ἀγνοεῖν, ἀδελφοί, τὸ μυστήριον τοῦτο, ἵνα μὴ ἦτε παρ᾽ ἑαυτοῖς φρόνιμοι, ὅτι πώρωσις ἀπὸ μέρους τῷ Ἰσραὴλ γέγονεν ἄχρι οὗ τὸ πλήρωμα τῶν ἐθνῶν εἰσέλθῃ
\vs{26}καὶ οὕτως πᾶς Ἰσραὴλ σωθήσεται, καθὼς γέγραπται· 
\begin{poetryblock}

\begin{quote}Ἥξει ἐκ Σιὼν ὁ Ῥυόμενος,\end{quote} 

\begin{quote}ἀποστρέψει ἀσεβείας ἀπὸ Ἰακώβ.\end{quote}

\begin{quote} \vs{27}καὶ αὕτη αὐτοῖς ἡ παρ᾽ ἐμοῦ διαθήκη,\end{quote} 

\begin{quote}ὅταν ἀφέλωμαι τὰς ἁμαρτίας αὐτῶν.\end{quote}
\end{poetryblock}

\vs{28}Κατὰ μὲν τὸ εὐαγγέλιον ἐχθροὶ δι᾽ ὑμᾶς, κατὰ δὲ τὴν ἐκλογὴν ἀγαπητοὶ διὰ τοὺς πατέρας·
\vs{29}ἀμεταμέλητα γὰρ τὰ χαρίσματα καὶ ἡ κλῆσις τοῦ Θεοῦ.
\vs{30}Ὥσπερ γὰρ ὑμεῖς ποτε ἠπειθήσατε τῷ Θεῷ, νῦν δὲ ἠλεήθητε τῇ τούτων ἀπειθείᾳ,
\vs{31}οὕτως καὶ οὗτοι νῦν ἠπείθησαν τῷ ὑμετέρῳ ἐλέει, ἵνα καὶ αὐτοὶ νῦν ἐλεηθῶσιν.
\vs{32}συνέκλεισεν γὰρ ὁ Θεὸς τοὺς πάντας εἰς ἀπείθειαν, ἵνα τοὺς πάντας ἐλεήσῃ.
\begin{poetryblock}

\begin{quote} \vs{33}Ὦ βάθος πλούτου\end{quote} 

\begin{quote}καὶ σοφίας καὶ γνώσεως Θεοῦ·\end{quote} 

\begin{quote}ὡς ἀνεξεραύνητα τὰ κρίματα αὐτοῦ\end{quote} 

\begin{quote}καὶ ἀνεξιχνίαστοι αἱ ὁδοὶ αὐτοῦ.\end{quote}

\begin{quote} \vs{34}Τίς γὰρ ἔγνω νοῦν Κυρίου;\end{quote} 

\begin{quote}ἢ τίς σύμβουλος αὐτοῦ ἐγένετο;\end{quote}

\begin{quote} \vs{35}Ἢ τίς προέδωκεν αὐτῷ,\end{quote} 

\begin{quote}καὶ ἀνταποδοθήσεται αὐτῷ;\end{quote}

\begin{quote} \vs{36}ὅτι ἐξ αὐτοῦ καὶ δι᾽ αὐτοῦ καὶ εἰς αὐτὸν τὰ πάντα·\end{quote} 

\begin{quote}αὐτῷ ἡ δόξα εἰς τοὺς αἰῶνας, ἀμήν.\end{quote}
\end{poetryblock}

\ch{12}
Παρακαλῶ οὖν ὑμᾶς, ἀδελφοί, διὰ τῶν οἰκτιρμῶν τοῦ Θεοῦ παραστῆσαι τὰ σώματα ὑμῶν θυσίαν ζῶσαν ἁγίαν εὐάρεστον τῷ Θεῷ, τὴν λογικὴν λατρείαν ὑμῶν·
\vs{2}καὶ μὴ συσχηματίζεσθε τῷ αἰῶνι τούτῳ, ἀλλὰ μεταμορφοῦσθε τῇ ἀνακαινώσει τοῦ νοός εἰς τὸ δοκιμάζειν ὑμᾶς τί τὸ θέλημα τοῦ Θεοῦ, τὸ ἀγαθὸν καὶ εὐάρεστον καὶ τέλειον.

\vs{3}Λέγω γὰρ διὰ τῆς χάριτος τῆς δοθείσης μοι παντὶ τῷ ὄντι ἐν ὑμῖν μὴ ὑπερφρονεῖν παρ᾽ ὃ δεῖ φρονεῖν ἀλλὰ φρονεῖν εἰς τὸ σωφρονεῖν, ἑκάστῳ ὡς ὁ Θεὸς ἐμέρισεν μέτρον πίστεως.
\vs{4}καθάπερ γὰρ ἐν ἑνὶ σώματι πολλὰ μέλη ἔχομεν, τὰ δὲ μέλη πάντα οὐ τὴν αὐτὴν ἔχει πρᾶξιν,
\vs{5}οὕτως οἱ πολλοὶ ἓν σῶμά ἐσμεν ἐν Χριστῷ, τὸ δὲ καθ᾽ εἷς ἀλλήλων μέλη.
\vs{6}Ἔχοντες δὲ χαρίσματα κατὰ τὴν χάριν τὴν δοθεῖσαν ἡμῖν διάφορα, εἴτε προφητείαν κατὰ τὴν ἀναλογίαν τῆς πίστεως,
\vs{7}εἴτε διακονίαν ἐν τῇ διακονίᾳ, εἴτε ὁ διδάσκων ἐν τῇ διδασκαλίᾳ,
\vs{8}εἴτε ὁ παρακαλῶν ἐν τῇ παρακλήσει· ὁ μεταδιδοὺς ἐν ἁπλότητι, ὁ προϊστάμενος ἐν σπουδῇ, ὁ ἐλεῶν ἐν ἱλαρότητι.

\vs{9}Ἡ ἀγάπη ἀνυπόκριτος. ἀποστυγοῦντες τὸ πονηρόν, κολλώμενοι τῷ ἀγαθῷ,
\vs{10}τῇ φιλαδελφίᾳ εἰς ἀλλήλους φιλόστοργοι, τῇ τιμῇ ἀλλήλους προηγούμενοι,
\vs{11}τῇ σπουδῇ μὴ ὀκνηροί, τῷ πνεύματι ζέοντες, τῷ Κυρίῳ δουλεύοντες,
\vs{12}τῇ ἐλπίδι χαίροντες, τῇ θλίψει ὑπομένοντες, τῇ προσευχῇ προσκαρτεροῦντες,
\vs{13}ταῖς χρείαις τῶν ἁγίων κοινωνοῦντες, τὴν φιλοξενίαν διώκοντες.
\vs{14}Εὐλογεῖτε τοὺς διώκοντας ὑμᾶς, εὐλογεῖτε καὶ μὴ καταρᾶσθε.
\vs{15}χαίρειν μετὰ χαιρόντων, κλαίειν μετὰ κλαιόντων.
\vs{16}τὸ αὐτὸ εἰς ἀλλήλους φρονοῦντες, μὴ τὰ ὑψηλὰ φρονοῦντες ἀλλὰ τοῖς ταπεινοῖς συναπαγόμενοι. μὴ γίνεσθε φρόνιμοι παρ᾽ ἑαυτοῖς.
\vs{17}μηδενὶ κακὸν ἀντὶ κακοῦ ἀποδιδόντες, προνοούμενοι καλὰ ἐνώπιον πάντων ἀνθρώπων·
\vs{18}εἰ δυνατόν τὸ ἐξ ὑμῶν, μετὰ πάντων ἀνθρώπων εἰρηνεύοντες·
\vs{19}μὴ ἑαυτοὺς ἐκδικοῦντες, ἀγαπητοί, ἀλλὰ δότε τόπον τῇ ὀργῇ, γέγραπται γάρ· Ἐμοὶ ἐκδίκησις, ἐγὼ ἀνταποδώσω, λέγει Κύριος.
\vs{20}Ἀλλὰ Ἐὰν πεινᾷ ὁ ἐχθρός σου, ψώμιζε αὐτόν· ἐὰν διψᾷ, πότιζε αὐτόν· τοῦτο γὰρ ποιῶν ἄνθρακας πυρὸς σωρεύσεις ἐπὶ τὴν κεφαλὴν αὐτοῦ.
\vs{21}Μὴ νικῶ ὑπὸ τοῦ κακοῦ ἀλλὰ νίκα ἐν τῷ ἀγαθῷ τὸ κακόν.

\ch{13}
Πᾶσα ψυχὴ ἐξουσίαις ὑπερεχούσαις ὑποτασσέσθω. οὐ γὰρ ἔστιν ἐξουσία εἰ μὴ ὑπὸ Θεοῦ, αἱ δὲ οὖσαι ὑπὸ Θεοῦ τεταγμέναι εἰσίν.
\vs{2}ὥστε ὁ ἀντιτασσόμενος τῇ ἐξουσίᾳ τῇ τοῦ Θεοῦ διαταγῇ ἀνθέστηκεν, οἱ δὲ ἀνθεστηκότες ἑαυτοῖς κρίμα λήμψονται.
\vs{3}οἱ γὰρ ἄρχοντες οὐκ εἰσὶν φόβος τῷ ἀγαθῷ ἔργῳ ἀλλὰ τῷ κακῷ. θέλεις δὲ μὴ φοβεῖσθαι τὴν ἐξουσίαν· τὸ ἀγαθὸν ποίει, καὶ ἕξεις ἔπαινον ἐξ αὐτῆς·
\vs{4}Θεοῦ γὰρ διάκονός ἐστιν σοὶ εἰς τὸ ἀγαθόν. ἐὰν δὲ τὸ κακὸν ποιῇς, φοβοῦ· οὐ γὰρ εἰκῇ τὴν μάχαιραν φορεῖ· Θεοῦ γὰρ διάκονός ἐστιν ἔκδικος εἰς ὀργὴν τῷ τὸ κακὸν πράσσοντι.
\vs{5}Διὸ ἀνάγκη ὑποτάσσεσθαι, οὐ μόνον διὰ τὴν ὀργὴν ἀλλὰ καὶ διὰ τὴν συνείδησιν.
\vs{6}διὰ τοῦτο γὰρ καὶ φόρους τελεῖτε· λειτουργοὶ γὰρ Θεοῦ εἰσιν εἰς αὐτὸ τοῦτο προσκαρτεροῦντες.
\vs{7}ἀπόδοτε πᾶσιν τὰς ὀφειλάς, τῷ τὸν φόρον τὸν φόρον, τῷ τὸ τέλος τὸ τέλος, τῷ τὸν φόβον τὸν φόβον, τῷ τὴν τιμὴν τὴν τιμήν.

\vs{8}Μηδενὶ μηδὲν ὀφείλετε εἰ μὴ τὸ ἀλλήλους ἀγαπᾶν· ὁ γὰρ ἀγαπῶν τὸν ἕτερον νόμον πεπλήρωκεν.
\vs{9}τὸ γάρ Οὐ μοιχεύσεις, Οὐ φονεύσεις, Οὐ κλέψεις, Οὐκ ἐπιθυμήσεις, καὶ εἴ τις ἑτέρα ἐντολή, ἐν τῷ λόγῳ τούτῳ ἀνακεφαλαιοῦται ἐν τῷ· Ἀγαπήσεις τὸν πλησίον σου ὡς σεαυτόν.
\vs{10}ἡ ἀγάπη τῷ πλησίον κακὸν οὐκ ἐργάζεται· πλήρωμα οὖν νόμου ἡ ἀγάπη.
\vs{11}Καὶ τοῦτο εἰδότες τὸν καιρόν, ὅτι ὥρα ἤδη ὑμᾶς ἐξ ὕπνου ἐγερθῆναι, νῦν γὰρ ἐγγύτερον ἡμῶν ἡ σωτηρία ἢ ὅτε ἐπιστεύσαμεν.
\vs{12}ἡ νὺξ προέκοψεν, ἡ δὲ ἡμέρα ἤγγικεν. ἀποθώμεθα οὖν τὰ ἔργα τοῦ σκότους, ἐνδυσώμεθα δὲ τὰ ὅπλα τοῦ φωτός.
\vs{13}ὡς ἐν ἡμέρᾳ εὐσχημόνως περιπατήσωμεν, μὴ κώμοις καὶ μέθαις, μὴ κοίταις καὶ ἀσελγείαις, μὴ ἔριδι καὶ ζήλῳ,
\vs{14}ἀλλὰ ἐνδύσασθε τὸν Κύριον Ἰησοῦν Χριστόν καὶ τῆς σαρκὸς πρόνοιαν μὴ ποιεῖσθε εἰς ἐπιθυμίας.

\ch{14}
Τὸν δὲ ἀσθενοῦντα τῇ πίστει προσλαμβάνεσθε, μὴ εἰς διακρίσεις διαλογισμῶν.
\vs{2}ὃς μὲν πιστεύει φαγεῖν πάντα, ὁ δὲ ἀσθενῶν λάχανα ἐσθίει.
\vs{3}ὁ ἐσθίων τὸν μὴ ἐσθίοντα μὴ ἐξουθενείτω, ὁ δὲ μὴ ἐσθίων τὸν ἐσθίοντα μὴ κρινέτω, ὁ Θεὸς γὰρ αὐτὸν προσελάβετο.
\vs{4}σὺ τίς εἶ ὁ κρίνων ἀλλότριον οἰκέτην; τῷ ἰδίῳ κυρίῳ στήκει ἢ πίπτει· σταθήσεται δέ, δυνατεῖ γὰρ ὁ Κύριος στῆσαι αὐτόν.
\vs{5}Ὃς μὲν γὰρ κρίνει ἡμέραν παρ᾽ ἡμέραν, ὃς δὲ κρίνει πᾶσαν ἡμέραν· ἕκαστος ἐν τῷ ἰδίῳ νοῒ πληροφορείσθω.
\vs{6}ὁ φρονῶν τὴν ἡμέραν Κυρίῳ φρονεῖ· καὶ ὁ ἐσθίων Κυρίῳ ἐσθίει, εὐχαριστεῖ γὰρ τῷ Θεῷ· καὶ ὁ μὴ ἐσθίων Κυρίῳ οὐκ ἐσθίει καὶ εὐχαριστεῖ τῷ Θεῷ.
\vs{7}Οὐδεὶς γὰρ ἡμῶν ἑαυτῷ ζῇ καὶ οὐδεὶς ἑαυτῷ ἀποθνῄσκει·
\vs{8}ἐάν τε γὰρ ζῶμεν, τῷ Κυρίῳ ζῶμεν, ἐάν τε ἀποθνήσκωμεν, τῷ Κυρίῳ ἀποθνήσκομεν. ἐάν τε οὖν ζῶμεν ἐάν τε ἀποθνήσκωμεν, τοῦ Κυρίου ἐσμέν.
\vs{9}εἰς τοῦτο γὰρ Χριστὸς ἀπέθανεν καὶ ἔζησεν, ἵνα καὶ νεκρῶν καὶ ζώντων κυριεύσῃ.
\vs{10}Σὺ δὲ τί κρίνεις τὸν ἀδελφόν σου; ἢ καὶ σὺ τί ἐξουθενεῖς τὸν ἀδελφόν σου; πάντες γὰρ παραστησόμεθα τῷ βήματι τοῦ Θεοῦ,
\vs{11}γέγραπται γάρ· 
\begin{poetryblock}

\begin{quote}Ζῶ ἐγώ, λέγει Κύριος, ὅτι ἐμοὶ κάμψει πᾶν γόνυ\end{quote} 

\begin{quote}καὶ πᾶσα γλῶσσα ἐξομολογήσεται τῷ Θεῷ.\end{quote}
\end{poetryblock}

\vs{12}Ἄρα οὖν ἕκαστος ἡμῶν περὶ ἑαυτοῦ λόγον δώσει τῷ Θεῷ.

\vs{13}Μηκέτι οὖν ἀλλήλους κρίνωμεν· ἀλλὰ τοῦτο κρίνατε μᾶλλον, τὸ μὴ τιθέναι πρόσκομμα τῷ ἀδελφῷ ἢ σκάνδαλον.
\vs{14}Οἶδα καὶ πέπεισμαι ἐν Κυρίῳ Ἰησοῦ ὅτι οὐδὲν κοινὸν δι᾽ ἑαυτοῦ, εἰ μὴ τῷ λογιζομένῳ τι κοινὸν εἶναι, ἐκείνῳ κοινόν.
\vs{15}εἰ γὰρ διὰ βρῶμα ὁ ἀδελφός σου λυπεῖται, οὐκέτι κατὰ ἀγάπην περιπατεῖς· μὴ τῷ βρώματί σου ἐκεῖνον ἀπόλλυε ὑπὲρ οὗ Χριστὸς ἀπέθανεν.
\vs{16}Μὴ βλασφημείσθω οὖν ὑμῶν τὸ ἀγαθόν.
\vs{17}οὐ γάρ ἐστιν ἡ βασιλεία τοῦ Θεοῦ βρῶσις καὶ πόσις ἀλλὰ δικαιοσύνη καὶ εἰρήνη καὶ χαρὰ ἐν Πνεύματι Ἁγίῳ·
\vs{18}ὁ γὰρ ἐν τούτῳ δουλεύων τῷ Χριστῷ εὐάρεστος τῷ Θεῷ καὶ δόκιμος τοῖς ἀνθρώποις.
\vs{19}Ἄρα οὖν τὰ τῆς εἰρήνης διώκωμεν καὶ τὰ τῆς οἰκοδομῆς τῆς εἰς ἀλλήλους.
\vs{20}μὴ ἕνεκεν βρώματος κατάλυε τὸ ἔργον τοῦ Θεοῦ. πάντα μὲν καθαρά, ἀλλὰ κακὸν τῷ ἀνθρώπῳ τῷ διὰ προσκόμματος ἐσθίοντι.
\vs{21}καλὸν τὸ μὴ φαγεῖν κρέα μηδὲ πιεῖν οἶνον μηδὲ ἐν ᾧ ὁ ἀδελφός σου προσκόπτει.
\vs{22}Σὺ πίστιν ἣν ἔχεις κατὰ σεαυτὸν ἔχε ἐνώπιον τοῦ Θεοῦ. μακάριος ὁ μὴ κρίνων ἑαυτὸν ἐν ᾧ δοκιμάζει·
\vs{23}ὁ δὲ διακρινόμενος ἐὰν φάγῃ κατακέκριται, ὅτι οὐκ ἐκ πίστεως· πᾶν δὲ ὃ οὐκ ἐκ πίστεως ἁμαρτία ἐστίν.

\ch{15}
Ὀφείλομεν δὲ ἡμεῖς οἱ δυνατοὶ τὰ ἀσθενήματα τῶν ἀδυνάτων βαστάζειν καὶ μὴ ἑαυτοῖς ἀρέσκειν.
\vs{2}ἕκαστος ἡμῶν τῷ πλησίον ἀρεσκέτω εἰς τὸ ἀγαθὸν πρὸς οἰκοδομήν·
\vs{3}καὶ γὰρ ὁ Χριστὸς οὐχ ἑαυτῷ ἤρεσεν, ἀλλὰ καθὼς γέγραπται· Οἱ ὀνειδισμοὶ τῶν ὀνειδιζόντων σε ἐπέπεσαν ἐπ᾽ ἐμέ.
\vs{4}ὅσα γὰρ προεγράφη, εἰς τὴν ἡμετέραν διδασκαλίαν ἐγράφη, ἵνα διὰ τῆς ὑπομονῆς καὶ διὰ τῆς παρακλήσεως τῶν γραφῶν τὴν ἐλπίδα ἔχωμεν.
\vs{5}Ὁ δὲ Θεὸς τῆς ὑπομονῆς καὶ τῆς παρακλήσεως δῴη ὑμῖν τὸ αὐτὸ φρονεῖν ἐν ἀλλήλοις κατὰ Χριστὸν Ἰησοῦν,
\vs{6}ἵνα ὁμοθυμαδὸν ἐν ἑνὶ στόματι δοξάζητε τὸν Θεὸν καὶ Πατέρα τοῦ Κυρίου ἡμῶν Ἰησοῦ Χριστοῦ.

\vs{7}Διὸ προσλαμβάνεσθε ἀλλήλους, καθὼς καὶ ὁ Χριστὸς προσελάβετο ὑμᾶς εἰς δόξαν τοῦ Θεοῦ.
\vs{8}λέγω γὰρ Χριστὸν διάκονον γεγενῆσθαι περιτομῆς ὑπὲρ ἀληθείας Θεοῦ, εἰς τὸ βεβαιῶσαι τὰς ἐπαγγελίας τῶν πατέρων,
\vs{9}τὰ δὲ ἔθνη ὑπὲρ ἐλέους δοξάσαι τὸν Θεόν, καθὼς γέγραπται· 
\begin{poetryblock}

\begin{quote}Διὰ τοῦτο ἐξομολογήσομαί σοι ἐν ἔθνεσιν\end{quote} 

\begin{quote}καὶ τῷ ὀνόματί σου ψαλῶ.\end{quote}
\end{poetryblock}

\vs{10}Καὶ πάλιν λέγει· 
\begin{poetryblock}

\begin{quote}Εὐφράνθητε, ἔθνη, μετὰ τοῦ λαοῦ αὐτοῦ.\end{quote}
\end{poetryblock}

\vs{11}Καὶ πάλιν· Αἰνεῖτε, πάντα τὰ ἔθνη, τὸν Κύριον καὶ ἐπαινεσάτωσαν αὐτὸν πάντες οἱ λαοί.
\vs{12}Καὶ πάλιν Ἠσαΐας λέγει· 
\begin{poetryblock}

\begin{quote}Ἔσται ἡ ῥίζα τοῦ Ἰεσσαί\end{quote} 

\begin{quote}καὶ ὁ ἀνιστάμενος ἄρχειν ἐθνῶν,\end{quote} 

\begin{quote}ἐπ᾽ αὐτῷ ἔθνη ἐλπιοῦσιν.\end{quote}
\end{poetryblock}

\vs{13}Ὁ δὲ Θεὸς τῆς ἐλπίδος πληρώσαι ὑμᾶς πάσης χαρᾶς καὶ εἰρήνης ἐν τῷ πιστεύειν, εἰς τὸ περισσεύειν ὑμᾶς ἐν τῇ ἐλπίδι ἐν δυνάμει Πνεύματος Ἁγίου.

\vs{14}Πέπεισμαι δέ, ἀδελφοί μου, καὶ αὐτὸς ἐγὼ περὶ ὑμῶν ὅτι καὶ αὐτοὶ μεστοί ἐστε ἀγαθωσύνης, πεπληρωμένοι πάσης τῆς γνώσεως, δυνάμενοι καὶ ἀλλήλους νουθετεῖν.
\vs{15}τολμηρότερον δὲ ἔγραψα ὑμῖν ἀπὸ μέρους ὡς ἐπαναμιμνήσκων ὑμᾶς διὰ τὴν χάριν τὴν δοθεῖσάν μοι ὑπὸ τοῦ Θεοῦ
\vs{16}εἰς τὸ εἶναί με λειτουργὸν Χριστοῦ Ἰησοῦ εἰς τὰ ἔθνη, ἱερουργοῦντα τὸ εὐαγγέλιον τοῦ Θεοῦ, ἵνα γένηται ἡ προσφορὰ τῶν ἐθνῶν εὐπρόσδεκτος, ἡγιασμένη ἐν Πνεύματι Ἁγίῳ.
\vs{17}Ἔχω οὖν τὴν καύχησιν ἐν Χριστῷ Ἰησοῦ τὰ πρὸς τὸν Θεόν·
\vs{18}οὐ γὰρ τολμήσω τι λαλεῖν ὧν οὐ κατειργάσατο Χριστὸς δι᾽ ἐμοῦ εἰς ὑπακοὴν ἐθνῶν, λόγῳ καὶ ἔργῳ,
\vs{19}ἐν δυνάμει σημείων καὶ τεράτων, ἐν δυνάμει Πνεύματος θεοῦ· ὥστε με ἀπὸ Ἰερουσαλὴμ καὶ κύκλῳ μέχρι τοῦ Ἰλλυρικοῦ πεπληρωκέναι τὸ εὐαγγέλιον τοῦ Χριστοῦ,
\vs{20}οὕτως δὲ φιλοτιμούμενον εὐαγγελίζεσθαι οὐχ ὅπου ὠνομάσθη Χριστός, ἵνα μὴ ἐπ᾽ ἀλλότριον θεμέλιον οἰκοδομῶ,
\vs{21}ἀλλὰ καθὼς γέγραπται· 
\begin{poetryblock}

\begin{quote}Οἷς οὐκ ἀνηγγέλη περὶ αὐτοῦ ὄψονται,\end{quote} 

\begin{quote}καὶ οἳ οὐκ ἀκηκόασιν συνήσουσιν.\end{quote}
\end{poetryblock}

\vs{22}Διὸ καὶ ἐνεκοπτόμην τὰ πολλὰ τοῦ ἐλθεῖν πρὸς ὑμᾶς·
\vs{23}Νυνὶ δὲ μηκέτι τόπον ἔχων ἐν τοῖς κλίμασι τούτοις, ἐπιποθίαν δὲ ἔχων τοῦ ἐλθεῖν πρὸς ὑμᾶς ἀπὸ ἱκανῶν ἐτῶν,
\vs{24}ὡς ἂν πορεύωμαι εἰς τὴν Σπανίαν· ἐλπίζω γὰρ διαπορευόμενος θεάσασθαι ὑμᾶς καὶ ὑφ᾽ ὑμῶν προπεμφθῆναι ἐκεῖ ἐὰν ὑμῶν πρῶτον ἀπὸ μέρους ἐμπλησθῶ.
\vs{25}Νυνὶ δὲ πορεύομαι εἰς Ἰερουσαλὴμ διακονῶν τοῖς ἁγίοις.
\vs{26}εὐδόκησαν γὰρ Μακεδονία καὶ Ἀχαΐα κοινωνίαν τινὰ ποιήσασθαι εἰς τοὺς πτωχοὺς τῶν ἁγίων τῶν ἐν Ἰερουσαλήμ.
\vs{27}εὐδόκησαν γάρ καὶ ὀφειλέται εἰσὶν αὐτῶν· εἰ γὰρ τοῖς πνευματικοῖς αὐτῶν ἐκοινώνησαν τὰ ἔθνη, ὀφείλουσιν καὶ ἐν τοῖς σαρκικοῖς λειτουργῆσαι αὐτοῖς.
\vs{28}Τοῦτο οὖν ἐπιτελέσας καὶ σφραγισάμενος αὐτοῖς τὸν καρπὸν τοῦτον, ἀπελεύσομαι δι᾽ ὑμῶν εἰς Σπανίαν·
\vs{29}οἶδα δὲ ὅτι ἐρχόμενος πρὸς ὑμᾶς ἐν πληρώματι εὐλογίας Χριστοῦ ἐλεύσομαι.

\vs{30}Παρακαλῶ δὲ ὑμᾶς, ἀδελφοί, διὰ τοῦ Κυρίου ἡμῶν Ἰησοῦ Χριστοῦ καὶ διὰ τῆς ἀγάπης τοῦ Πνεύματος συναγωνίσασθαί μοι ἐν ταῖς προσευχαῖς ὑπὲρ ἐμοῦ πρὸς τὸν Θεόν,
\vs{31}ἵνα ῥυσθῶ ἀπὸ τῶν ἀπειθούντων ἐν τῇ Ἰουδαίᾳ καὶ ἡ διακονία μου ἡ εἰς Ἰερουσαλὴμ εὐπρόσδεκτος τοῖς ἁγίοις γένηται,
\vs{32}ἵνα ἐν χαρᾷ ἐλθὼν πρὸς ὑμᾶς διὰ θελήματος Θεοῦ συναναπαύσωμαι ὑμῖν.
\vs{33}Ὁ δὲ Θεὸς τῆς εἰρήνης μετὰ πάντων ὑμῶν, ἀμήν.

\ch{16}
Συνίστημι δὲ ὑμῖν Φοίβην τὴν ἀδελφὴν ἡμῶν, οὖσαν καὶ διάκονον τῆς ἐκκλησίας τῆς ἐν Κενχρεαῖς,
\vs{2}ἵνα αὐτὴν προσδέξησθε ἐν Κυρίῳ ἀξίως τῶν ἁγίων καὶ παραστῆτε αὐτῇ ἐν ᾧ ἂν ὑμῶν χρῄζῃ πράγματι· καὶ γὰρ αὐτὴ προστάτις πολλῶν ἐγενήθη καὶ ἐμοῦ αὐτοῦ.

\vs{3}Ἀσπάσασθε Πρίσκαν καὶ Ἀκύλαν τοὺς συνεργούς μου ἐν Χριστῷ Ἰησοῦ,
\vs{4}οἵτινες ὑπὲρ τῆς ψυχῆς μου τὸν ἑαυτῶν τράχηλον ὑπέθηκαν, οἷς οὐκ ἐγὼ μόνος εὐχαριστῶ ἀλλὰ καὶ πᾶσαι αἱ ἐκκλησίαι τῶν ἐθνῶν,
\vs{5}καὶ τὴν κατ᾽ οἶκον αὐτῶν ἐκκλησίαν. Ἀσπάσασθε Ἐπαίνετον τὸν ἀγαπητόν μου, ὅς ἐστιν ἀπαρχὴ τῆς Ἀσίας εἰς Χριστόν.
\vs{6}Ἀσπάσασθε Μαριάν, ἥτις πολλὰ ἐκοπίασεν εἰς ὑμᾶς.
\vs{7}Ἀσπάσασθε Ἀνδρόνικον καὶ Ἰουνίαν τοὺς συγγενεῖς μου καὶ συναιχμαλώτους μου, οἵτινές εἰσιν ἐπίσημοι ἐν τοῖς ἀποστόλοις, οἳ καὶ πρὸ ἐμοῦ γέγοναν ἐν Χριστῷ.
\vs{8}Ἀσπάσασθε Ἀμπλιᾶτον τὸν ἀγαπητόν μου ἐν Κυρίῳ.
\vs{9}Ἀσπάσασθε Οὐρβανὸν τὸν συνεργὸν ἡμῶν ἐν Χριστῷ καὶ Στάχυν τὸν ἀγαπητόν μου.
\vs{10}Ἀσπάσασθε Ἀπελλῆν τὸν δόκιμον ἐν Χριστῷ. Ἀσπάσασθε τοὺς ἐκ τῶν Ἀριστοβούλου.
\vs{11}Ἀσπάσασθε Ἡρῳδίωνα τὸν συγγενῆ μου. Ἀσπάσασθε τοὺς ἐκ τῶν Ναρκίσσου τοὺς ὄντας ἐν Κυρίῳ.
\vs{12}Ἀσπάσασθε Τρύφαιναν καὶ Τρυφῶσαν τὰς κοπιώσας ἐν Κυρίῳ. Ἀσπάσασθε Περσίδα τὴν ἀγαπητήν, ἥτις πολλὰ ἐκοπίασεν ἐν Κυρίῳ.
\vs{13}Ἀσπάσασθε Ῥοῦφον τὸν ἐκλεκτὸν ἐν Κυρίῳ καὶ τὴν μητέρα αὐτοῦ καὶ ἐμοῦ.
\vs{14}Ἀσπάσασθε Ἀσύνκριτον, Φλέγοντα, Ἑρμῆν, Πατρόβαν, Ἑρμᾶν καὶ τοὺς σὺν αὐτοῖς ἀδελφούς.
\vs{15}Ἀσπάσασθε Φιλόλογον καὶ Ἰουλίαν, Νηρέα καὶ τὴν ἀδελφὴν αὐτοῦ, καὶ Ὀλυμπᾶν καὶ τοὺς σὺν αὐτοῖς πάντας ἁγίους.
\vs{16}Ἀσπάσασθε ἀλλήλους ἐν φιλήματι ἁγίῳ. Ἀσπάζονται ὑμᾶς αἱ ἐκκλησίαι πᾶσαι τοῦ Χριστοῦ.

\vs{17}Παρακαλῶ δὲ ὑμᾶς, ἀδελφοί, σκοπεῖν τοὺς τὰς διχοστασίας καὶ τὰ σκάνδαλα παρὰ τὴν διδαχὴν ἣν ὑμεῖς ἐμάθετε ποιοῦντας, καὶ ἐκκλίνετε ἀπ᾽ αὐτῶν·
\vs{18}οἱ γὰρ τοιοῦτοι τῷ Κυρίῳ ἡμῶν Χριστῷ οὐ δουλεύουσιν ἀλλὰ τῇ ἑαυτῶν κοιλίᾳ, καὶ διὰ τῆς χρηστολογίας καὶ εὐλογίας ἐξαπατῶσιν τὰς καρδίας τῶν ἀκάκων.
\vs{19}Ἡ γὰρ ὑμῶν ὑπακοὴ εἰς πάντας ἀφίκετο· ἐφ᾽ ὑμῖν οὖν χαίρω, θέλω δὲ ὑμᾶς σοφοὺς εἶναι εἰς τὸ ἀγαθόν, ἀκεραίους δὲ εἰς τὸ κακόν.
\vs{20}Ὁ δὲ Θεὸς τῆς εἰρήνης συντρίψει τὸν Σατανᾶν ὑπὸ τοὺς πόδας ὑμῶν ἐν τάχει. Ἡ χάρις τοῦ Κυρίου ἡμῶν Ἰησοῦ μεθ᾽ ὑμῶν.

\vs{21}Ἀσπάζεται ὑμᾶς Τιμόθεος ὁ συνεργός μου καὶ Λούκιος καὶ Ἰάσων καὶ Σωσίπατρος οἱ συγγενεῖς μου.
\vs{22}Ἀσπάζομαι ὑμᾶς ἐγὼ Τέρτιος ὁ γράψας τὴν ἐπιστολὴν ἐν Κυρίῳ.
\vs{23}Ἀσπάζεται ὑμᾶς Γάϊος ὁ ξένος μου καὶ ὅλης τῆς ἐκκλησίας. Ἀσπάζεται ὑμᾶς Ἔραστος ὁ οἰκονόμος τῆς πόλεως καὶ Κούαρτος ὁ ἀδελφός.

\vs{25}Τῷ δὲ δυναμένῳ ὑμᾶς στηρίξαι κατὰ τὸ εὐαγγέλιόν μου καὶ τὸ κήρυγμα Ἰησοῦ Χριστοῦ, κατὰ ἀποκάλυψιν μυστηρίου χρόνοις αἰωνίοις σεσιγημένου,
\vs{26}φανερωθέντος δὲ νῦν διά τε γραφῶν προφητικῶν κατ᾽ ἐπιταγὴν τοῦ αἰωνίου Θεοῦ εἰς ὑπακοὴν πίστεως εἰς πάντα τὰ ἔθνη γνωρισθέντος,
\vs{27}μόνῳ σοφῷ Θεῷ, διὰ Ἰησοῦ Χριστοῦ, ᾧ ἡ δόξα εἰς τοὺς αἰῶνας, ἀμήν.


\def\book{ΠΡΟΣ ΚΟΡΙΝΘΙΟΥΣ Α}
\biblebook{ΠΡΟΣ ΚΟΡΙΝΘΙΟΥΣ Α}


\lettrine[lines=2, loversize=0.2, nindent=0em, findent=.25em]{\textcolor{bookheadingcolor}{Π}}{αῦλος} κλητὸς ἀπόστολος Χριστοῦ Ἰησοῦ διὰ θελήματος Θεοῦ καὶ Σωσθένης ὁ ἀδελφὸς
\vs{2}Τῇ ἐκκλησίᾳ τοῦ Θεοῦ τῇ οὔσῃ ἐν Κορίνθῳ, ἡγιασμένοις ἐν Χριστῷ Ἰησοῦ, κλητοῖς ἁγίοις, σὺν πᾶσιν τοῖς ἐπικαλουμένοις τὸ ὄνομα τοῦ Κυρίου ἡμῶν Ἰησοῦ Χριστοῦ ἐν παντὶ τόπῳ, αὐτῶν καὶ ἡμῶν·
\vs{3}Χάρις ὑμῖν καὶ εἰρήνη ἀπὸ Θεοῦ Πατρὸς ἡμῶν καὶ Κυρίου Ἰησοῦ Χριστοῦ.

\vs{4}Εὐχαριστῶ τῷ Θεῷ μου πάντοτε περὶ ὑμῶν ἐπὶ τῇ χάριτι τοῦ Θεοῦ τῇ δοθείσῃ ὑμῖν ἐν Χριστῷ Ἰησοῦ,
\vs{5}ὅτι ἐν παντὶ ἐπλουτίσθητε ἐν αὐτῷ, ἐν παντὶ λόγῳ καὶ πάσῃ γνώσει,
\vs{6}καθὼς τὸ μαρτύριον τοῦ Χριστοῦ ἐβεβαιώθη ἐν ὑμῖν,
\vs{7}ὥστε ὑμᾶς μὴ ὑστερεῖσθαι ἐν μηδενὶ χαρίσματι ἀπεκδεχομένους τὴν ἀποκάλυψιν τοῦ Κυρίου ἡμῶν Ἰησοῦ Χριστοῦ·
\vs{8}ὃς καὶ βεβαιώσει ὑμᾶς ἕως τέλους ἀνεγκλήτους ἐν τῇ ἡμέρᾳ τοῦ Κυρίου ἡμῶν Ἰησοῦ Χριστοῦ.
\vs{9}πιστὸς ὁ Θεὸς, δι᾽ οὗ ἐκλήθητε εἰς κοινωνίαν τοῦ Υἱοῦ αὐτοῦ Ἰησοῦ Χριστοῦ τοῦ Κυρίου ἡμῶν.

\vs{10}Παρακαλῶ δὲ ὑμᾶς, ἀδελφοί, διὰ τοῦ ὀνόματος τοῦ Κυρίου ἡμῶν Ἰησοῦ Χριστοῦ, ἵνα τὸ αὐτὸ λέγητε πάντες καὶ μὴ ᾖ ἐν ὑμῖν σχίσματα, ἦτε δὲ κατηρτισμένοι ἐν τῷ αὐτῷ νοῒ καὶ ἐν τῇ αὐτῇ γνώμῃ.
\vs{11}ἐδηλώθη γάρ μοι περὶ ὑμῶν, ἀδελφοί μου, ὑπὸ τῶν Χλόης ὅτι ἔριδες ἐν ὑμῖν εἰσιν.
\vs{12}λέγω δὲ τοῦτο ὅτι ἕκαστος ὑμῶν λέγει· Ἐγὼ μέν εἰμι Παύλου, Ἐγὼ δὲ Ἀπολλῶ, Ἐγὼ δὲ Κηφᾶ, Ἐγὼ δὲ Χριστοῦ.
\vs{13}Μεμέρισται ὁ Χριστός; μὴ Παῦλος ἐσταυρώθη ὑπὲρ ὑμῶν, ἢ εἰς τὸ ὄνομα Παύλου ἐβαπτίσθητε;
\vs{14}εὐχαριστῶ τῷ θεῷ ὅτι οὐδένα ὑμῶν ἐβάπτισα εἰ μὴ Κρίσπον καὶ Γάϊον,
\vs{15}ἵνα μή τις εἴπῃ ὅτι εἰς τὸ ἐμὸν ὄνομα ἐβαπτίσθητε.
\vs{16}ἐβάπτισα δὲ καὶ τὸν Στεφανᾶ οἶκον, λοιπὸν οὐκ οἶδα εἴ τινα ἄλλον ἐβάπτισα.
\vs{17}οὐ γὰρ ἀπέστειλέν με Χριστὸς βαπτίζειν ἀλλὰ εὐαγγελίζεσθαι, οὐκ ἐν σοφίᾳ λόγου, ἵνα μὴ κενωθῇ ὁ σταυρὸς τοῦ Χριστοῦ.

\vs{18}Ὁ λόγος γὰρ ὁ τοῦ σταυροῦ τοῖς μὲν ἀπολλυμένοις μωρία ἐστίν, τοῖς δὲ σῳζομένοις ἡμῖν δύναμις Θεοῦ ἐστιν.
\vs{19}γέγραπται γάρ· 
\begin{poetryblock}

\begin{quote}Ἀπολῶ τὴν σοφίαν τῶν σοφῶν\end{quote} 

\begin{quote}καὶ τὴν σύνεσιν τῶν συνετῶν ἀθετήσω.\end{quote}
\end{poetryblock}

\vs{20}Ποῦ σοφός; ποῦ γραμματεύς; ποῦ συζητητὴς τοῦ αἰῶνος τούτου; οὐχὶ ἐμώρανεν ὁ Θεὸς τὴν σοφίαν τοῦ κόσμου;
\vs{21}ἐπειδὴ γὰρ ἐν τῇ σοφίᾳ τοῦ Θεοῦ οὐκ ἔγνω ὁ κόσμος διὰ τῆς σοφίας τὸν Θεόν, εὐδόκησεν ὁ Θεὸς διὰ τῆς μωρίας τοῦ κηρύγματος σῶσαι τοὺς πιστεύοντας·
\vs{22}Ἐπειδὴ καὶ Ἰουδαῖοι σημεῖα αἰτοῦσιν καὶ Ἕλληνες σοφίαν ζητοῦσιν,
\vs{23}ἡμεῖς δὲ κηρύσσομεν Χριστὸν ἐσταυρωμένον, Ἰουδαίοις μὲν σκάνδαλον, ἔθνεσιν δὲ μωρίαν,
\vs{24}αὐτοῖς δὲ τοῖς κλητοῖς, Ἰουδαίοις τε καὶ Ἕλλησιν, Χριστὸν Θεοῦ δύναμιν καὶ Θεοῦ σοφίαν·
\vs{25}Ὅτι τὸ μωρὸν τοῦ Θεοῦ σοφώτερον τῶν ἀνθρώπων ἐστίν καὶ τὸ ἀσθενὲς τοῦ Θεοῦ ἰσχυρότερον τῶν ἀνθρώπων.

\vs{26}Βλέπετε γὰρ τὴν κλῆσιν ὑμῶν, ἀδελφοί, ὅτι οὐ πολλοὶ σοφοὶ κατὰ σάρκα, οὐ πολλοὶ δυνατοί, οὐ πολλοὶ εὐγενεῖς·
\vs{27}ἀλλὰ τὰ μωρὰ τοῦ κόσμου ἐξελέξατο ὁ Θεός, ἵνα καταισχύνῃ τοὺς σοφούς, καὶ τὰ ἀσθενῆ τοῦ κόσμου ἐξελέξατο ὁ Θεός, ἵνα καταισχύνῃ τὰ ἰσχυρά,
\vs{28}καὶ τὰ ἀγενῆ τοῦ κόσμου καὶ τὰ ἐξουθενημένα ἐξελέξατο ὁ Θεός, τὰ μὴ ὄντα, ἵνα τὰ ὄντα καταργήσῃ,
\vs{29}ὅπως μὴ καυχήσηται πᾶσα σὰρξ ἐνώπιον τοῦ Θεοῦ.
\vs{30}Ἐξ αὐτοῦ δὲ ὑμεῖς ἐστε ἐν Χριστῷ Ἰησοῦ, ὃς ἐγενήθη σοφία ἡμῖν ἀπὸ Θεοῦ, δικαιοσύνη τε καὶ ἁγιασμὸς καὶ ἀπολύτρωσις,
\vs{31}ἵνα καθὼς γέγραπται· Ὁ καυχώμενος ἐν Κυρίῳ καυχάσθω.

\ch{2}
Κἀγὼ ἐλθὼν πρὸς ὑμᾶς, ἀδελφοί, ἦλθον οὐ καθ᾽ ὑπεροχὴν λόγου ἢ σοφίας καταγγέλλων ὑμῖν τὸ μυστήριον τοῦ Θεοῦ.
\vs{2}οὐ γὰρ ἔκρινά τι εἰδέναι ἐν ὑμῖν εἰ μὴ Ἰησοῦν Χριστὸν καὶ τοῦτον ἐσταυρωμένον.
\vs{3}κἀγὼ ἐν ἀσθενείᾳ καὶ ἐν φόβῳ καὶ ἐν τρόμῳ πολλῷ ἐγενόμην πρὸς ὑμᾶς,
\vs{4}καὶ ὁ λόγος μου καὶ τὸ κήρυγμά μου οὐκ ἐν πειθοῖς σοφίας λόγοις ἀλλ᾽ ἐν ἀποδείξει Πνεύματος καὶ δυνάμεως,
\vs{5}ἵνα ἡ πίστις ὑμῶν μὴ ᾖ ἐν σοφίᾳ ἀνθρώπων ἀλλ᾽ ἐν δυνάμει Θεοῦ.

\vs{6}Σοφίαν δὲ λαλοῦμεν ἐν τοῖς τελείοις, σοφίαν δὲ οὐ τοῦ αἰῶνος τούτου οὐδὲ τῶν ἀρχόντων τοῦ αἰῶνος τούτου τῶν καταργουμένων·
\vs{7}ἀλλὰ λαλοῦμεν Θεοῦ σοφίαν ἐν μυστηρίῳ τὴν ἀποκεκρυμμένην, ἣν προώρισεν ὁ Θεὸς πρὸ τῶν αἰώνων εἰς δόξαν ἡμῶν,
\vs{8}ἣν οὐδεὶς τῶν ἀρχόντων τοῦ αἰῶνος τούτου ἔγνωκεν· εἰ γὰρ ἔγνωσαν, οὐκ ἂν τὸν Κύριον τῆς δόξης ἐσταύρωσαν.
\vs{9}ἀλλὰ καθὼς γέγραπται· 
\begin{poetryblock}

\begin{quote}Ἃ ὀφθαλμὸς οὐκ εἶδεν καὶ οὖς οὐκ ἤκουσεν\end{quote} 

\begin{quote}καὶ ἐπὶ καρδίαν ἀνθρώπου οὐκ ἀνέβη,\end{quote} 

\begin{quote}ἃ ἡτοίμασεν ὁ Θεὸς τοῖς ἀγαπῶσιν αὐτόν.\end{quote}
\end{poetryblock}

\vs{10}Ἡμῖν γὰρ ἀπεκάλυψεν ὁ Θεὸς διὰ τοῦ Πνεύματος· Τὸ γὰρ Πνεῦμα πάντα ἐραυνᾷ, καὶ τὰ βάθη τοῦ Θεοῦ.
\vs{11}τίς γὰρ οἶδεν ἀνθρώπων τὰ τοῦ ἀνθρώπου εἰ μὴ τὸ πνεῦμα τοῦ ἀνθρώπου τὸ ἐν αὐτῷ; οὕτως καὶ τὰ τοῦ Θεοῦ οὐδεὶς ἔγνωκεν εἰ μὴ τὸ Πνεῦμα τοῦ Θεοῦ.
\vs{12}ἡμεῖς δὲ οὐ τὸ πνεῦμα τοῦ κόσμου ἐλάβομεν ἀλλὰ τὸ πνεῦμα τὸ ἐκ τοῦ Θεοῦ, ἵνα εἰδῶμεν τὰ ὑπὸ τοῦ Θεοῦ χαρισθέντα ἡμῖν·
\vs{13}ἃ καὶ λαλοῦμεν οὐκ ἐν διδακτοῖς ἀνθρωπίνης σοφίας λόγοις ἀλλ᾽ ἐν διδακτοῖς Πνεύματος, πνευματικοῖς πνευματικὰ συνκρίνοντες.
\vs{14}Ψυχικὸς δὲ ἄνθρωπος οὐ δέχεται τὰ τοῦ Πνεύματος τοῦ Θεοῦ· μωρία γὰρ αὐτῷ ἐστίν καὶ οὐ δύναται γνῶναι, ὅτι πνευματικῶς ἀνακρίνεται.
\vs{15}ὁ δὲ πνευματικὸς ἀνακρίνει τὰ πάντα, αὐτὸς δὲ ὑπ᾽ οὐδενὸς ἀνακρίνεται.
\vs{16}Τίς γὰρ ἔγνω νοῦν Κυρίου, ὃς συμβιβάσει αὐτόν; ἡμεῖς δὲ νοῦν Χριστοῦ ἔχομεν.

\ch{3}
Κἀγώ, ἀδελφοί, οὐκ ἠδυνήθην λαλῆσαι ὑμῖν ὡς πνευματικοῖς ἀλλ᾽ ὡς σαρκίνοις, ὡς νηπίοις ἐν Χριστῷ.
\vs{2}γάλα ὑμᾶς ἐπότισα, οὐ βρῶμα· οὔπω γὰρ ἐδύνασθε. Ἀλλ᾽ οὐδὲ ἔτι νῦν δύνασθε,
\vs{3}ἔτι γὰρ σαρκικοί ἐστε. ὅπου γὰρ ἐν ὑμῖν ζῆλος καὶ ἔρις, οὐχὶ σαρκικοί ἐστε καὶ κατὰ ἄνθρωπον περιπατεῖτε;
\vs{4}ὅταν γὰρ λέγῃ τις· Ἐγὼ μέν εἰμι Παύλου, ἕτερος δέ· Ἐγὼ Ἀπολλῶ, οὐκ ἄνθρωποί ἐστε;
\vs{5}Τί οὖν ἐστιν Ἀπολλῶς; τί δέ ἐστιν Παῦλος; διάκονοι δι᾽ ὧν ἐπιστεύσατε, καὶ ἑκάστῳ ὡς ὁ Κύριος ἔδωκεν.
\vs{6}ἐγὼ ἐφύτευσα, Ἀπολλῶς ἐπότισεν, ἀλλὰ ὁ Θεὸς ηὔξανεν·
\vs{7}ὥστε οὔτε ὁ φυτεύων ἐστίν τι οὔτε ὁ ποτίζων ἀλλ᾽ ὁ αὐξάνων Θεός.
\vs{8}ὁ φυτεύων δὲ καὶ ὁ ποτίζων ἕν εἰσιν, ἕκαστος δὲ τὸν ἴδιον μισθὸν λήμψεται κατὰ τὸν ἴδιον κόπον·
\vs{9}Θεοῦ γάρ ἐσμεν συνεργοί, Θεοῦ γεώργιον, Θεοῦ οἰκοδομή ἐστε.
\vs{10}Κατὰ τὴν χάριν τοῦ Θεοῦ τὴν δοθεῖσάν μοι ὡς σοφὸς ἀρχιτέκτων θεμέλιον ἔθηκα, ἄλλος δὲ ἐποικοδομεῖ. ἕκαστος δὲ βλεπέτω πῶς ἐποικοδομεῖ.
\vs{11}θεμέλιον γὰρ ἄλλον οὐδεὶς δύναται θεῖναι παρὰ τὸν κείμενον, ὅς ἐστιν Ἰησοῦς Χριστός.
\vs{12}Εἰ δέ τις ἐποικοδομεῖ ἐπὶ τὸν θεμέλιον χρυσόν, ἄργυρον, λίθους τιμίους, ξύλα, χόρτον, καλάμην,
\vs{13}ἑκάστου τὸ ἔργον φανερὸν γενήσεται, ἡ γὰρ ἡμέρα δηλώσει, ὅτι ἐν πυρὶ ἀποκαλύπτεται· καὶ ἑκάστου τὸ ἔργον ὁποῖόν ἐστιν τὸ πῦρ αὐτὸ δοκιμάσει.
\vs{14}εἴ τινος τὸ ἔργον μενεῖ ὃ ἐποικοδόμησεν, μισθὸν λήμψεται·
\vs{15}εἴ τινος τὸ ἔργον κατακαήσεται, ζημιωθήσεται, αὐτὸς δὲ σωθήσεται, οὕτως δὲ ὡς διὰ πυρός.
\vs{16}Οὐκ οἴδατε ὅτι ναὸς Θεοῦ ἐστε καὶ τὸ Πνεῦμα τοῦ Θεοῦ οἰκεῖ ἐν ὑμῖν;
\vs{17}εἴ τις τὸν ναὸν τοῦ Θεοῦ φθείρει, φθερεῖ τοῦτον ὁ Θεός· ὁ γὰρ ναὸς τοῦ Θεοῦ ἅγιός ἐστιν, οἵτινές ἐστε ὑμεῖς.

\vs{18}Μηδεὶς ἑαυτὸν ἐξαπατάτω· εἴ τις δοκεῖ σοφὸς εἶναι ἐν ὑμῖν ἐν τῷ αἰῶνι τούτῳ, μωρὸς γενέσθω, ἵνα γένηται σοφός.
\vs{19}ἡ γὰρ σοφία τοῦ κόσμου τούτου μωρία παρὰ τῷ Θεῷ ἐστιν. γέγραπται γάρ· Ὁ δρασσόμενος τοὺς σοφοὺς ἐν τῇ πανουργίᾳ αὐτῶν·
\vs{20}καὶ πάλιν· Κύριος γινώσκει τοὺς διαλογισμοὺς τῶν σοφῶν ὅτι εἰσὶν μάταιοι.
\vs{21}Ὥστε μηδεὶς καυχάσθω ἐν ἀνθρώποις· πάντα γὰρ ὑμῶν ἐστιν,
\vs{22}εἴτε Παῦλος εἴτε Ἀπολλῶς εἴτε Κηφᾶς, εἴτε κόσμος εἴτε ζωὴ εἴτε θάνατος, εἴτε ἐνεστῶτα εἴτε μέλλοντα· πάντα ὑμῶν,
\vs{23}ὑμεῖς δὲ Χριστοῦ, Χριστὸς δὲ Θεοῦ.

\ch{4}
Οὕτως ἡμᾶς λογιζέσθω ἄνθρωπος ὡς ὑπηρέτας Χριστοῦ καὶ οἰκονόμους μυστηρίων Θεοῦ.
\vs{2}ὧδε λοιπὸν ζητεῖται ἐν τοῖς οἰκονόμοις, ἵνα πιστός τις εὑρεθῇ.
\vs{3}Ἐμοὶ δὲ εἰς ἐλάχιστόν ἐστιν, ἵνα ὑφ᾽ ὑμῶν ἀνακριθῶ ἢ ὑπὸ ἀνθρωπίνης ἡμέρας· ἀλλ᾽ οὐδὲ ἐμαυτὸν ἀνακρίνω.
\vs{4}οὐδὲν γὰρ ἐμαυτῷ σύνοιδα, ἀλλ᾽ οὐκ ἐν τούτῳ δεδικαίωμαι, ὁ δὲ ἀνακρίνων με Κύριός ἐστιν.
\vs{5}Ὥστε μὴ πρὸ καιροῦ τι κρίνετε ἕως ἂν ἔλθῃ ὁ Κύριος, ὃς καὶ φωτίσει τὰ κρυπτὰ τοῦ σκότους καὶ φανερώσει τὰς βουλὰς τῶν καρδιῶν· καὶ τότε ὁ ἔπαινος γενήσεται ἑκάστῳ ἀπὸ τοῦ Θεοῦ.

\vs{6}Ταῦτα δέ, ἀδελφοί, μετεσχημάτισα εἰς ἐμαυτὸν καὶ Ἀπολλῶν δι᾽ ὑμᾶς, ἵνα ἐν ἡμῖν μάθητε τό Μὴ ὑπὲρ ἃ γέγραπται, ἵνα μὴ εἷς ὑπὲρ τοῦ ἑνὸς φυσιοῦσθε κατὰ τοῦ ἑτέρου.
\vs{7}τίς γάρ σε διακρίνει; τί δὲ ἔχεις ὃ οὐκ ἔλαβες; εἰ δὲ καὶ ἔλαβες, τί καυχᾶσαι ὡς μὴ λαβών;
\vs{8}ἤδη κεκορεσμένοι ἐστέ, ἤδη ἐπλουτήσατε, χωρὶς ἡμῶν ἐβασιλεύσατε· καὶ ὄφελόν γε ἐβασιλεύσατε, ἵνα καὶ ἡμεῖς ὑμῖν συμβασιλεύσωμεν.
\vs{9}δοκῶ γάρ, ὁ Θεὸς ἡμᾶς τοὺς ἀποστόλους ἐσχάτους ἀπέδειξεν ὡς ἐπιθανατίους, ὅτι θέατρον ἐγενήθημεν τῷ κόσμῳ καὶ ἀγγέλοις καὶ ἀνθρώποις.
\vs{10}Ἡμεῖς μωροὶ διὰ Χριστόν, ὑμεῖς δὲ φρόνιμοι ἐν Χριστῷ· ἡμεῖς ἀσθενεῖς, ὑμεῖς δὲ ἰσχυροί· ὑμεῖς ἔνδοξοι, ἡμεῖς δὲ ἄτιμοι.
\vs{11}ἄχρι τῆς ἄρτι ὥρας καὶ πεινῶμεν καὶ διψῶμεν καὶ γυμνιτεύομεν καὶ κολαφιζόμεθα καὶ ἀστατοῦμεν
\vs{12}καὶ κοπιῶμεν ἐργαζόμενοι ταῖς ἰδίαις χερσίν· λοιδορούμενοι εὐλογοῦμεν, διωκόμενοι ἀνεχόμεθα,
\vs{13}δυσφημούμενοι παρακαλοῦμεν· ὡς περικαθάρματα τοῦ κόσμου ἐγενήθημεν, πάντων περίψημα ἕως ἄρτι.

\vs{14}Οὐκ ἐντρέπων ὑμᾶς γράφω ταῦτα ἀλλ᾽ ὡς τέκνα μου ἀγαπητὰ νουθετῶν.
\vs{15}ἐὰν γὰρ μυρίους παιδαγωγοὺς ἔχητε ἐν Χριστῷ ἀλλ᾽ οὐ πολλοὺς πατέρας· ἐν γὰρ Χριστῷ Ἰησοῦ διὰ τοῦ εὐαγγελίου ἐγὼ ὑμᾶς ἐγέννησα.
\vs{16}παρακαλῶ οὖν ὑμᾶς, μιμηταί μου γίνεσθε.
\vs{17}Διὰ τοῦτο ἔπεμψα ὑμῖν Τιμόθεον, ὅς ἐστίν μου τέκνον ἀγαπητὸν καὶ πιστὸν ἐν Κυρίῳ, ὃς ὑμᾶς ἀναμνήσει τὰς ὁδούς μου τὰς ἐν Χριστῷ Ἰησοῦ, καθὼς πανταχοῦ ἐν πάσῃ ἐκκλησίᾳ διδάσκω.
\vs{18}Ὡς μὴ ἐρχομένου δέ μου πρὸς ὑμᾶς ἐφυσιώθησάν τινες·
\vs{19}ἐλεύσομαι δὲ ταχέως πρὸς ὑμᾶς ἐὰν ὁ Κύριος θελήσῃ, καὶ γνώσομαι οὐ τὸν λόγον τῶν πεφυσιωμένων ἀλλὰ τὴν δύναμιν·
\vs{20}οὐ γὰρ ἐν λόγῳ ἡ βασιλεία τοῦ Θεοῦ ἀλλ᾽ ἐν δυνάμει.
\vs{21}τί θέλετε; ἐν ῥάβδῳ ἔλθω πρὸς ὑμᾶς ἢ ἐν ἀγάπῃ πνεύματί τε πραΰτητος;

\ch{5}
Ὅλως ἀκούεται ἐν ὑμῖν πορνεία, καὶ τοιαύτη πορνεία ἥτις οὐδὲ ἐν τοῖς ἔθνεσιν, ὥστε γυναῖκά τινα τοῦ πατρὸς ἔχειν.
\vs{2}καὶ ὑμεῖς πεφυσιωμένοι ἐστέ καὶ οὐχὶ μᾶλλον ἐπενθήσατε, ἵνα ἀρθῇ ἐκ μέσου ὑμῶν ὁ τὸ ἔργον τοῦτο πράξας;
\vs{3}Ἐγὼ μὲν γάρ, ἀπὼν τῷ σώματι παρὼν δὲ τῷ πνεύματι, ἤδη κέκρικα ὡς παρὼν τὸν οὕτως τοῦτο κατεργασάμενον·
\vs{4}ἐν τῷ ὀνόματι τοῦ Κυρίου ἡμῶν Ἰησοῦ συναχθέντων ὑμῶν καὶ τοῦ ἐμοῦ πνεύματος σὺν τῇ δυνάμει τοῦ Κυρίου ἡμῶν Ἰησοῦ,
\vs{5}παραδοῦναι τὸν τοιοῦτον τῷ Σατανᾷ εἰς ὄλεθρον τῆς σαρκός, ἵνα τὸ πνεῦμα σωθῇ ἐν τῇ ἡμέρᾳ τοῦ Κυρίου.
\vs{6}Οὐ καλὸν τὸ καύχημα ὑμῶν. οὐκ οἴδατε ὅτι μικρὰ ζύμη ὅλον τὸ φύραμα ζυμοῖ;
\vs{7}ἐκκαθάρατε τὴν παλαιὰν ζύμην, ἵνα ἦτε νέον φύραμα, καθώς ἐστε ἄζυμοι· καὶ γὰρ τὸ πάσχα ἡμῶν ἐτύθη Χριστός.
\vs{8}ὥστε ἑορτάζωμεν μὴ ἐν ζύμῃ παλαιᾷ μηδὲ ἐν ζύμῃ κακίας καὶ πονηρίας ἀλλ᾽ ἐν ἀζύμοις εἰλικρινείας καὶ ἀληθείας.
\vs{9}Ἔγραψα ὑμῖν ἐν τῇ ἐπιστολῇ μὴ συναναμίγνυσθαι πόρνοις,
\vs{10}οὐ πάντως τοῖς πόρνοις τοῦ κόσμου τούτου ἢ τοῖς πλεονέκταις καὶ ἅρπαξιν ἢ εἰδωλολάτραις, ἐπεὶ ὠφείλετε ἄρα ἐκ τοῦ κόσμου ἐξελθεῖν.
\vs{11}νῦν δὲ ἔγραψα ὑμῖν μὴ συναναμίγνυσθαι ἐάν τις ἀδελφὸς ὀνομαζόμενος ᾖ πόρνος ἢ πλεονέκτης ἢ εἰδωλολάτρης ἢ λοίδορος ἢ μέθυσος ἢ ἅρπαξ, τῷ τοιούτῳ μηδὲ συνεσθίειν.
\vs{12}Τί γάρ μοι τοὺς ἔξω κρίνειν; οὐχὶ τοὺς ἔσω ὑμεῖς κρίνετε;
\vs{13}τοὺς δὲ ἔξω ὁ Θεὸς κρίνει. Ἐξάρατε τὸν πονηρὸν ἐξ ὑμῶν αὐτῶν.

\ch{6}
Τολμᾷ τις ὑμῶν πρᾶγμα ἔχων πρὸς τὸν ἕτερον κρίνεσθαι ἐπὶ τῶν ἀδίκων καὶ οὐχὶ ἐπὶ τῶν ἁγίων;
\vs{2}ἢ οὐκ οἴδατε ὅτι οἱ ἅγιοι τὸν κόσμον κρινοῦσιν; καὶ εἰ ἐν ὑμῖν κρίνεται ὁ κόσμος, ἀνάξιοί ἐστε κριτηρίων ἐλαχίστων;
\vs{3}οὐκ οἴδατε ὅτι ἀγγέλους κρινοῦμεν, μήτι γε βιωτικά;
\vs{4}Βιωτικὰ μὲν οὖν κριτήρια ἐὰν ἔχητε, τοὺς ἐξουθενημένους ἐν τῇ ἐκκλησίᾳ, τούτους καθίζετε;
\vs{5}πρὸς ἐντροπὴν ὑμῖν λέγω. οὕτως οὐκ ἔνι ἐν ὑμῖν οὐδεὶς σοφὸς, ὃς δυνήσεται διακρῖναι ἀνὰ μέσον τοῦ ἀδελφοῦ αὐτοῦ;
\vs{6}ἀλλὰ ἀδελφὸς μετὰ ἀδελφοῦ κρίνεται καὶ τοῦτο ἐπὶ ἀπίστων;
\vs{7}ἤδη μὲν οὖν ὅλως ἥττημα ὑμῖν ἐστιν ὅτι κρίματα ἔχετε μεθ᾽ ἑαυτῶν. διὰ τί οὐχὶ μᾶλλον ἀδικεῖσθε; διὰ τί οὐχὶ μᾶλλον ἀποστερεῖσθε;
\vs{8}ἀλλὰ ὑμεῖς ἀδικεῖτε καὶ ἀποστερεῖτε, καὶ τοῦτο ἀδελφούς.
\vs{9}Ἢ οὐκ οἴδατε ὅτι ἄδικοι Θεοῦ βασιλείαν οὐ κληρονομήσουσιν; μὴ πλανᾶσθε· οὔτε πόρνοι οὔτε εἰδωλολάτραι οὔτε μοιχοὶ οὔτε μαλακοὶ οὔτε ἀρσενοκοῖται
\vs{10}οὔτε κλέπται οὔτε πλεονέκται, οὐ μέθυσοι, οὐ λοίδοροι, οὐχ ἅρπαγες βασιλείαν Θεοῦ κληρονομήσουσιν.
\vs{11}καὶ ταῦτά τινες ἦτε· ἀλλὰ ἀπελούσασθε, ἀλλὰ ἡγιάσθητε, ἀλλὰ ἐδικαιώθητε ἐν τῷ ὀνόματι τοῦ Κυρίου Ἰησοῦ Χριστοῦ καὶ ἐν τῷ Πνεύματι τοῦ Θεοῦ ἡμῶν.

\vs{12}Πάντα μοι ἔξεστιν ἀλλ᾽ οὐ πάντα συμφέρει· Πάντα μοι ἔξεστιν ἀλλ᾽ οὐκ ἐγὼ ἐξουσιασθήσομαι ὑπό τινος.
\vs{13}Τὰ βρώματα τῇ κοιλίᾳ καὶ ἡ κοιλία τοῖς βρώμασιν, ὁ δὲ Θεὸς καὶ ταύτην καὶ ταῦτα καταργήσει. τὸ δὲ σῶμα οὐ τῇ πορνείᾳ ἀλλὰ τῷ Κυρίῳ, καὶ ὁ Κύριος τῷ σώματι·
\vs{14}ὁ δὲ Θεὸς καὶ τὸν Κύριον ἤγειρεν καὶ ἡμᾶς ἐξεγερεῖ διὰ τῆς δυνάμεως αὐτοῦ.
\vs{15}Οὐκ οἴδατε ὅτι τὰ σώματα ὑμῶν μέλη Χριστοῦ ἐστιν; ἄρας οὖν τὰ μέλη τοῦ Χριστοῦ ποιήσω πόρνης μέλη; μὴ γένοιτο.
\vs{16}ἢ οὐκ οἴδατε ὅτι ὁ κολλώμενος τῇ πόρνῃ ἓν σῶμά ἐστιν; Ἔσονται γάρ, φησίν, Οἱ δύο εἰς σάρκα μίαν.
\vs{17}ὁ δὲ κολλώμενος τῷ Κυρίῳ ἓν πνεῦμά ἐστιν.
\vs{18}Φεύγετε τὴν πορνείαν. πᾶν ἁμάρτημα ὃ ἐὰν ποιήσῃ ἄνθρωπος ἐκτὸς τοῦ σώματός ἐστιν· ὁ δὲ πορνεύων εἰς τὸ ἴδιον σῶμα ἁμαρτάνει.
\vs{19}ἢ οὐκ οἴδατε ὅτι τὸ σῶμα ὑμῶν ναὸς τοῦ ἐν ὑμῖν Ἁγίου Πνεύματός ἐστιν οὗ ἔχετε ἀπὸ Θεοῦ, καὶ οὐκ ἐστὲ ἑαυτῶν;
\vs{20}ἠγοράσθητε γὰρ τιμῆς· δοξάσατε δὴ τὸν Θεὸν ἐν τῷ σώματι ὑμῶν.

\ch{7}
Περὶ δὲ ὧν ἐγράψατε, καλὸν ἀνθρώπῳ γυναικὸς μὴ ἅπτεσθαι·
\vs{2}διὰ δὲ τὰς πορνείας ἕκαστος τὴν ἑαυτοῦ γυναῖκα ἐχέτω καὶ ἑκάστη τὸν ἴδιον ἄνδρα ἐχέτω.
\vs{3}Τῇ γυναικὶ ὁ ἀνὴρ τὴν ὀφειλὴν ἀποδιδότω, ὁμοίως δὲ καὶ ἡ γυνὴ τῷ ἀνδρί.
\vs{4}ἡ γυνὴ τοῦ ἰδίου σώματος οὐκ ἐξουσιάζει ἀλλὰ ὁ ἀνήρ, ὁμοίως δὲ καὶ ὁ ἀνὴρ τοῦ ἰδίου σώματος οὐκ ἐξουσιάζει ἀλλὰ ἡ γυνή.
\vs{5}Μὴ ἀποστερεῖτε ἀλλήλους, εἰ μήτι ἂν ἐκ συμφώνου πρὸς καιρὸν, ἵνα σχολάσητε τῇ προσευχῇ καὶ πάλιν ἐπὶ τὸ αὐτὸ ἦτε, ἵνα μὴ πειράζῃ ὑμᾶς ὁ Σατανᾶς διὰ τὴν ἀκρασίαν ὑμῶν.
\vs{6}τοῦτο δὲ λέγω κατὰ συνγνώμην οὐ κατ᾽ ἐπιταγήν.
\vs{7}θέλω δὲ πάντας ἀνθρώπους εἶναι ὡς καὶ ἐμαυτόν· ἀλλὰ ἕκαστος ἴδιον ἔχει χάρισμα ἐκ Θεοῦ, ὁ μὲν οὕτως, ὁ δὲ οὕτως.

\vs{8}Λέγω δὲ τοῖς ἀγάμοις καὶ ταῖς χήραις, καλὸν αὐτοῖς ἐὰν μείνωσιν ὡς κἀγώ·
\vs{9}εἰ δὲ οὐκ ἐγκρατεύονται, γαμησάτωσαν, κρεῖττον γάρ ἐστιν γαμῆσαι ἢ πυροῦσθαι.
\vs{10}Τοῖς δὲ γεγαμηκόσιν παραγγέλλω, οὐκ ἐγὼ ἀλλὰ ὁ Κύριος, γυναῖκα ἀπὸ ἀνδρὸς μὴ χωρισθῆναι,—
\vs{11}ἐὰν δὲ καὶ χωρισθῇ, μενέτω ἄγαμος ἢ τῷ ἀνδρὶ καταλλαγήτω,— καὶ ἄνδρα γυναῖκα μὴ ἀφιέναι.
\vs{12}Τοῖς δὲ λοιποῖς λέγω ἐγώ οὐχ ὁ Κύριος· εἴ τις ἀδελφὸς γυναῖκα ἔχει ἄπιστον καὶ αὕτη συνευδοκεῖ οἰκεῖν μετ᾽ αὐτοῦ, μὴ ἀφιέτω αὐτήν·
\vs{13}καὶ γυνὴ εἴ τις ἔχει ἄνδρα ἄπιστον καὶ οὗτος συνευδοκεῖ οἰκεῖν μετ᾽ αὐτῆς, μὴ ἀφιέτω τὸν ἄνδρα.
\vs{14}ἡγίασται γὰρ ὁ ἀνὴρ ὁ ἄπιστος ἐν τῇ γυναικί καὶ ἡγίασται ἡ γυνὴ ἡ ἄπιστος ἐν τῷ ἀδελφῷ· ἐπεὶ ἄρα τὰ τέκνα ὑμῶν ἀκάθαρτά ἐστιν, νῦν δὲ ἅγιά ἐστιν.
\vs{15}Εἰ δὲ ὁ ἄπιστος χωρίζεται, χωριζέσθω· οὐ δεδούλωται ὁ ἀδελφὸς ἢ ἡ ἀδελφὴ ἐν τοῖς τοιούτοις· ἐν δὲ εἰρήνῃ κέκληκεν ὑμᾶς ὁ Θεός.
\vs{16}τί γὰρ οἶδας, γύναι, εἰ τὸν ἄνδρα σώσεις; ἢ τί οἶδας, ἄνερ, εἰ τὴν γυναῖκα σώσεις;

\vs{17}Εἰ μὴ ἑκάστῳ ὡς ἐμέρισεν ὁ Κύριος, ἕκαστον ὡς κέκληκεν ὁ Θεός, οὕτως περιπατείτω. καὶ οὕτως ἐν ταῖς ἐκκλησίαις πάσαις διατάσσομαι.
\vs{18}περιτετμημένος τις ἐκλήθη, μὴ ἐπισπάσθω· ἐν ἀκροβυστίᾳ κέκληταί τις, μὴ περιτεμνέσθω.
\vs{19}ἡ περιτομὴ οὐδέν ἐστιν καὶ ἡ ἀκροβυστία οὐδέν ἐστιν, ἀλλὰ τήρησις ἐντολῶν Θεοῦ.
\vs{20}Ἕκαστος ἐν τῇ κλήσει ᾗ ἐκλήθη, ἐν ταύτῃ μενέτω.
\vs{21}δοῦλος ἐκλήθης, μή σοι μελέτω· ἀλλ᾽ εἰ καὶ δύνασαι ἐλεύθερος γενέσθαι, μᾶλλον χρῆσαι.
\vs{22}ὁ γὰρ ἐν Κυρίῳ κληθεὶς δοῦλος ἀπελεύθερος Κυρίου ἐστίν, ὁμοίως ὁ ἐλεύθερος κληθεὶς δοῦλός ἐστιν Χριστοῦ.
\vs{23}Τιμῆς ἠγοράσθητε· μὴ γίνεσθε δοῦλοι ἀνθρώπων.
\vs{24}ἕκαστος ἐν ᾧ ἐκλήθη, ἀδελφοί, ἐν τούτῳ μενέτω παρὰ Θεῷ.

\vs{25}Περὶ δὲ τῶν παρθένων ἐπιταγὴν Κυρίου οὐκ ἔχω, γνώμην δὲ δίδωμι ὡς ἠλεημένος ὑπὸ Κυρίου πιστὸς εἶναι.
\vs{26}Νομίζω οὖν τοῦτο καλὸν ὑπάρχειν διὰ τὴν ἐνεστῶσαν ἀνάγκην, ὅτι καλὸν ἀνθρώπῳ τὸ οὕτως εἶναι.
\vs{27}δέδεσαι γυναικί, μὴ ζήτει λύσιν· λέλυσαι ἀπὸ γυναικός, μὴ ζήτει γυναῖκα.
\vs{28}ἐὰν δὲ καὶ γαμήσῃς, οὐχ ἥμαρτες, καὶ ἐὰν γήμῃ ἡ παρθένος, οὐχ ἥμαρτεν· θλῖψιν δὲ τῇ σαρκὶ ἕξουσιν οἱ τοιοῦτοι, ἐγὼ δὲ ὑμῶν φείδομαι.
\vs{29}Τοῦτο δέ φημι, ἀδελφοί, ὁ καιρὸς συνεσταλμένος ἐστίν· τὸ λοιπὸν, ἵνα καὶ οἱ ἔχοντες γυναῖκας ὡς μὴ ἔχοντες ὦσιν
\vs{30}καὶ οἱ κλαίοντες ὡς μὴ κλαίοντες καὶ οἱ χαίροντες ὡς μὴ χαίροντες καὶ οἱ ἀγοράζοντες ὡς μὴ κατέχοντες,
\vs{31}καὶ οἱ χρώμενοι τὸν κόσμον ὡς μὴ καταχρώμενοι· παράγει γὰρ τὸ σχῆμα τοῦ κόσμου τούτου.
\vs{32}Θέλω δὲ ὑμᾶς ἀμερίμνους εἶναι. ὁ ἄγαμος μεριμνᾷ τὰ τοῦ Κυρίου, πῶς ἀρέσῃ τῷ Κυρίῳ·
\vs{33}ὁ δὲ γαμήσας μεριμνᾷ τὰ τοῦ κόσμου, πῶς ἀρέσῃ τῇ γυναικί,
\vs{34}καὶ μεμέρισται. καὶ ἡ γυνὴ ἡ ἄγαμος καὶ ἡ παρθένος μεριμνᾷ τὰ τοῦ Κυρίου, ἵνα ᾖ ἁγία καὶ τῷ σώματι καὶ τῷ πνεύματι· ἡ δὲ γαμήσασα μεριμνᾷ τὰ τοῦ κόσμου, πῶς ἀρέσῃ τῷ ἀνδρί.
\vs{35}Τοῦτο δὲ πρὸς τὸ ὑμῶν αὐτῶν σύμφορον λέγω, οὐχ ἵνα βρόχον ὑμῖν ἐπιβάλω ἀλλὰ πρὸς τὸ εὔσχημον καὶ εὐπάρεδρον τῷ Κυρίῳ ἀπερισπάστως.
\vs{36}Εἰ δέ τις ἀσχημονεῖν ἐπὶ τὴν παρθένον αὐτοῦ νομίζει, ἐὰν ᾖ ὑπέρακμος καὶ οὕτως ὀφείλει γίνεσθαι, ὃ θέλει ποιείτω, οὐχ ἁμαρτάνει, γαμείτωσαν.
\vs{37}ὃς δὲ ἕστηκεν ἐν τῇ καρδίᾳ αὐτοῦ ἑδραῖος μὴ ἔχων ἀνάγκην, ἐξουσίαν δὲ ἔχει περὶ τοῦ ἰδίου θελήματος καὶ τοῦτο κέκρικεν ἐν τῇ ἰδίᾳ καρδίᾳ, τηρεῖν τὴν ἑαυτοῦ παρθένον, καλῶς ποιήσει.
\vs{38}Ὥστε καὶ ὁ γαμίζων τὴν ἑαυτοῦ παρθένον καλῶς ποιεῖ καὶ ὁ μὴ γαμίζων κρεῖσσον ποιήσει.
\vs{39}Γυνὴ δέδεται ἐφ᾽ ὅσον χρόνον ζῇ ὁ ἀνὴρ αὐτῆς· ἐὰν δὲ κοιμηθῇ ὁ ἀνήρ, ἐλευθέρα ἐστὶν ᾧ θέλει γαμηθῆναι, μόνον ἐν Κυρίῳ.
\vs{40}μακαριωτέρα δέ ἐστιν ἐὰν οὕτως μείνῃ, κατὰ τὴν ἐμὴν γνώμην· δοκῶ δὲ κἀγὼ Πνεῦμα Θεοῦ ἔχειν.

\ch{8}
Περὶ δὲ τῶν εἰδωλοθύτων, οἴδαμεν ὅτι πάντες γνῶσιν ἔχομεν. ἡ γνῶσις φυσιοῖ, ἡ δὲ ἀγάπη οἰκοδομεῖ·
\vs{2}εἴ τις δοκεῖ ἐγνωκέναι τι, οὔπω ἔγνω καθὼς δεῖ γνῶναι·
\vs{3}εἰ δέ τις ἀγαπᾷ τὸν Θεόν, οὗτος ἔγνωσται ὑπ᾽ αὐτοῦ.
\vs{4}Περὶ τῆς βρώσεως οὖν τῶν εἰδωλοθύτων, οἴδαμεν ὅτι οὐδὲν εἴδωλον ἐν κόσμῳ καὶ ὅτι οὐδεὶς Θεὸς εἰ μὴ εἷς.
\vs{5}καὶ γὰρ εἴπερ εἰσὶν λεγόμενοι θεοὶ εἴτε ἐν οὐρανῷ εἴτε ἐπὶ γῆς, ὥσπερ εἰσὶν θεοὶ πολλοὶ καὶ κύριοι πολλοί,
\begin{poetryblock}

\begin{quote} \vs{6}ἀλλ᾽ ἡμῖν εἷς Θεὸς ὁ Πατήρ\end{quote} 

\begin{quote}ἐξ οὗ τὰ πάντα καὶ ἡμεῖς εἰς αὐτόν,\end{quote} 

\begin{quote}καὶ εἷς Κύριος Ἰησοῦς Χριστός\end{quote} 

\begin{quote}δι᾽ οὗ τὰ πάντα καὶ ἡμεῖς δι᾽ αὐτοῦ.\end{quote}
\end{poetryblock}
\vs{7}Ἀλλ᾽ οὐκ ἐν πᾶσιν ἡ γνῶσις· τινὲς δὲ τῇ συνηθείᾳ ἕως ἄρτι τοῦ εἰδώλου ὡς εἰδωλόθυτον ἐσθίουσιν, καὶ ἡ συνείδησις αὐτῶν ἀσθενὴς οὖσα μολύνεται.
\vs{8}βρῶμα δὲ ἡμᾶς οὐ παραστήσει τῷ Θεῷ· οὔτε ἐὰν μὴ φάγωμεν ὑστερούμεθα, οὔτε ἐὰν φάγωμεν περισσεύομεν.
\vs{9}Βλέπετε δὲ μή πως ἡ ἐξουσία ὑμῶν αὕτη πρόσκομμα γένηται τοῖς ἀσθενέσιν.
\vs{10}ἐὰν γάρ τις ἴδῃ σὲ τὸν ἔχοντα γνῶσιν ἐν εἰδωλείῳ κατακείμενον, οὐχὶ ἡ συνείδησις αὐτοῦ ἀσθενοῦς ὄντος οἰκοδομηθήσεται εἰς τὸ τὰ εἰδωλόθυτα ἐσθίειν;
\vs{11}ἀπόλλυται γὰρ ὁ ἀσθενῶν ἐν τῇ σῇ γνώσει, ὁ ἀδελφὸς δι᾽ ὃν Χριστὸς ἀπέθανεν.
\vs{12}οὕτως δὲ ἁμαρτάνοντες εἰς τοὺς ἀδελφοὺς καὶ τύπτοντες αὐτῶν τὴν συνείδησιν ἀσθενοῦσαν εἰς Χριστὸν ἁμαρτάνετε.
\vs{13}Διόπερ εἰ βρῶμα σκανδαλίζει τὸν ἀδελφόν μου, οὐ μὴ φάγω κρέα εἰς τὸν αἰῶνα, ἵνα μὴ τὸν ἀδελφόν μου σκανδαλίσω.

\ch{9}
Οὐκ εἰμὶ ἐλεύθερος; οὐκ εἰμὶ ἀπόστολος; οὐχὶ Ἰησοῦν τὸν Κύριον ἡμῶν ἑόρακα; οὐ τὸ ἔργον μου ὑμεῖς ἐστε ἐν Κυρίῳ;
\vs{2}εἰ ἄλλοις οὐκ εἰμὶ ἀπόστολος, ἀλλά γε ὑμῖν εἰμι· ἡ γὰρ σφραγίς μου τῆς ἀποστολῆς ὑμεῖς ἐστε ἐν Κυρίῳ.
\vs{3}Ἡ ἐμὴ ἀπολογία τοῖς ἐμὲ ἀνακρίνουσίν ἐστιν αὕτη.
\vs{4}μὴ οὐκ ἔχομεν ἐξουσίαν φαγεῖν καὶ πεῖν;
\vs{5}μὴ οὐκ ἔχομεν ἐξουσίαν ἀδελφὴν γυναῖκα περιάγειν ὡς καὶ οἱ λοιποὶ ἀπόστολοι καὶ οἱ ἀδελφοὶ τοῦ Κυρίου καὶ Κηφᾶς;
\vs{6}ἢ μόνος ἐγὼ καὶ Βαρνάβας οὐκ ἔχομεν ἐξουσίαν μὴ ἐργάζεσθαι;
\vs{7}Τίς στρατεύεται ἰδίοις ὀψωνίοις ποτέ; τίς φυτεύει ἀμπελῶνα καὶ τὸν καρπὸν αὐτοῦ οὐκ ἐσθίει; ἢ τίς ποιμαίνει ποίμνην καὶ ἐκ τοῦ γάλακτος τῆς ποίμνης οὐκ ἐσθίει;
\vs{8}Μὴ κατὰ ἄνθρωπον ταῦτα λαλῶ ἢ καὶ ὁ νόμος ταῦτα οὐ λέγει;
\vs{9}ἐν γὰρ τῷ Μωϋσέως νόμῳ γέγραπται· Οὐ κημώσεις βοῦν ἀλοῶντα. μὴ τῶν βοῶν μέλει τῷ Θεῷ
\vs{10}ἢ δι᾽ ἡμᾶς πάντως λέγει; δι᾽ ἡμᾶς γὰρ ἐγράφη ὅτι ὀφείλει ἐπ᾽ ἐλπίδι ὁ ἀροτριῶν ἀροτριᾶν καὶ ὁ ἀλοῶν ἐπ᾽ ἐλπίδι τοῦ μετέχειν.
\vs{11}Εἰ ἡμεῖς ὑμῖν τὰ πνευματικὰ ἐσπείραμεν, μέγα εἰ ἡμεῖς ὑμῶν τὰ σαρκικὰ θερίσομεν;
\vs{12}εἰ ἄλλοι τῆς ὑμῶν ἐξουσίας μετέχουσιν, οὐ μᾶλλον ἡμεῖς; ἀλλ᾽ οὐκ ἐχρησάμεθα τῇ ἐξουσίᾳ ταύτῃ, ἀλλὰ πάντα στέγομεν, ἵνα μή τινα ἐνκοπὴν δῶμεν τῷ εὐαγγελίῳ τοῦ Χριστοῦ.
\vs{13}Οὐκ οἴδατε ὅτι οἱ τὰ ἱερὰ ἐργαζόμενοι τὰ ἐκ τοῦ ἱεροῦ ἐσθίουσιν, οἱ τῷ θυσιαστηρίῳ παρεδρεύοντες τῷ θυσιαστηρίῳ συμμερίζονται;
\vs{14}οὕτως καὶ ὁ Κύριος διέταξεν τοῖς τὸ εὐαγγέλιον καταγγέλλουσιν ἐκ τοῦ εὐαγγελίου ζῆν.
\vs{15}ἐγὼ δὲ οὐ κέχρημαι οὐδενὶ τούτων. οὐκ ἔγραψα δὲ ταῦτα, ἵνα οὕτως γένηται ἐν ἐμοί· καλὸν γάρ μοι μᾶλλον ἀποθανεῖν ἢ— τὸ καύχημά μου οὐδεὶς κενώσει.
\vs{16}Ἐὰν γὰρ εὐαγγελίζωμαι, οὐκ ἔστιν μοι καύχημα· ἀνάγκη γάρ μοι ἐπίκειται· οὐαὶ γάρ μοί ἐστιν ἐὰν μὴ εὐαγγελίσωμαι.
\vs{17}εἰ γὰρ ἑκὼν τοῦτο πράσσω, μισθὸν ἔχω· εἰ δὲ ἄκων, οἰκονομίαν πεπίστευμαι·
\vs{18}τίς οὖν μού ἐστιν ὁ μισθός; ἵνα εὐαγγελιζόμενος ἀδάπανον θήσω τὸ εὐαγγέλιον εἰς τὸ μὴ καταχρήσασθαι τῇ ἐξουσίᾳ μου ἐν τῷ εὐαγγελίῳ.

\vs{19}Ἐλεύθερος γὰρ ὢν ἐκ πάντων πᾶσιν ἐμαυτὸν ἐδούλωσα, ἵνα τοὺς πλείονας κερδήσω·
\vs{20}καὶ ἐγενόμην τοῖς Ἰουδαίοις ὡς Ἰουδαῖος, ἵνα Ἰουδαίους κερδήσω· τοῖς ὑπὸ νόμον ὡς ὑπὸ νόμον, μὴ ὢν αὐτὸς ὑπὸ νόμον, ἵνα τοὺς ὑπὸ νόμον κερδήσω·
\vs{21}τοῖς ἀνόμοις ὡς ἄνομος, μὴ ὢν ἄνομος Θεοῦ ἀλλ᾽ ἔννομος Χριστοῦ, ἵνα κερδάνω τοὺς ἀνόμους·
\vs{22}ἐγενόμην τοῖς ἀσθενέσιν ἀσθενής, ἵνα τοὺς ἀσθενεῖς κερδήσω· τοῖς πᾶσιν γέγονα πάντα, ἵνα πάντως τινὰς σώσω.
\vs{23}Πάντα δὲ ποιῶ διὰ τὸ εὐαγγέλιον, ἵνα συνκοινωνὸς αὐτοῦ γένωμαι.

\vs{24}Οὐκ οἴδατε ὅτι οἱ ἐν σταδίῳ τρέχοντες πάντες μὲν τρέχουσιν, εἷς δὲ λαμβάνει τὸ βραβεῖον; οὕτως τρέχετε ἵνα καταλάβητε.
\vs{25}πᾶς δὲ ὁ ἀγωνιζόμενος πάντα ἐγκρατεύεται, ἐκεῖνοι μὲν οὖν ἵνα φθαρτὸν στέφανον λάβωσιν, ἡμεῖς δὲ ἄφθαρτον.
\vs{26}ἐγὼ τοίνυν οὕτως τρέχω ὡς οὐκ ἀδήλως, οὕτως πυκτεύω ὡς οὐκ ἀέρα δέρων·
\vs{27}ἀλλὰ ὑπωπιάζω μου τὸ σῶμα καὶ δουλαγωγῶ, μή πως ἄλλοις κηρύξας αὐτὸς ἀδόκιμος γένωμαι.

\ch{10}
Οὐ θέλω γὰρ ὑμᾶς ἀγνοεῖν, ἀδελφοί, ὅτι οἱ πατέρες ἡμῶν πάντες ὑπὸ τὴν νεφέλην ἦσαν καὶ πάντες διὰ τῆς θαλάσσης διῆλθον
\vs{2}καὶ πάντες εἰς τὸν Μωϋσῆν ἐβαπτίσαντο ἐν τῇ νεφέλῃ καὶ ἐν τῇ θαλάσσῃ
\vs{3}καὶ πάντες τὸ αὐτὸ πνευματικὸν βρῶμα ἔφαγον
\vs{4}καὶ πάντες τὸ αὐτὸ πνευματικὸν ἔπιον πόμα· ἔπινον γὰρ ἐκ πνευματικῆς ἀκολουθούσης πέτρας, ἡ πέτρα δὲ ἦν ὁ Χριστός.
\vs{5}ἀλλ᾽ οὐκ ἐν τοῖς πλείοσιν αὐτῶν εὐδόκησεν ὁ Θεός, κατεστρώθησαν γὰρ ἐν τῇ ἐρήμῳ.
\vs{6}Ταῦτα δὲ τύποι ἡμῶν ἐγενήθησαν, εἰς τὸ μὴ εἶναι ἡμᾶς ἐπιθυμητὰς κακῶν, καθὼς κἀκεῖνοι ἐπεθύμησαν.
\vs{7}μηδὲ εἰδωλολάτραι γίνεσθε καθώς τινες αὐτῶν, ὥσπερ γέγραπται· Ἐκάθισεν ὁ λαὸς φαγεῖν καὶ πεῖν καὶ ἀνέστησαν παίζειν.
\vs{8}μηδὲ πορνεύωμεν, καθώς τινες αὐτῶν ἐπόρνευσαν καὶ ἔπεσαν μιᾷ ἡμέρᾳ εἰκοσι τρεῖς χιλιάδες.
\vs{9}μηδὲ ἐκπειράζωμεν τὸν Χριστόν, καθώς τινες αὐτῶν ἐπείρασαν καὶ ὑπὸ τῶν ὄφεων ἀπώλλυντο.
\vs{10}μηδὲ γογγύζετε, καθάπερ τινὲς αὐτῶν ἐγόγγυσαν καὶ ἀπώλοντο ὑπὸ τοῦ ὀλοθρευτοῦ.
\vs{11}Ταῦτα δὲ τυπικῶς συνέβαινεν ἐκείνοις, ἐγράφη δὲ πρὸς νουθεσίαν ἡμῶν, εἰς οὓς τὰ τέλη τῶν αἰώνων κατήντηκεν.
\vs{12}Ὥστε ὁ δοκῶν ἑστάναι βλεπέτω μὴ πέσῃ.
\vs{13}πειρασμὸς ὑμᾶς οὐκ εἴληφεν εἰ μὴ ἀνθρώπινος· πιστὸς δὲ ὁ Θεός, ὃς οὐκ ἐάσει ὑμᾶς πειρασθῆναι ὑπὲρ ὃ δύνασθε ἀλλὰ ποιήσει σὺν τῷ πειρασμῷ καὶ τὴν ἔκβασιν τοῦ δύνασθαι ὑπενεγκεῖν.

\vs{14}Διόπερ, ἀγαπητοί μου, φεύγετε ἀπὸ τῆς εἰδωλολατρίας.
\vs{15}ὡς φρονίμοις λέγω· κρίνατε ὑμεῖς ὅ φημι.
\vs{16}Τὸ ποτήριον τῆς εὐλογίας ὃ εὐλογοῦμεν, οὐχὶ κοινωνία ἐστὶν τοῦ αἵματος τοῦ Χριστοῦ; τὸν ἄρτον ὃν κλῶμεν, οὐχὶ κοινωνία τοῦ σώματος τοῦ Χριστοῦ ἐστιν;
\vs{17}ὅτι εἷς ἄρτος, ἓν σῶμα οἱ πολλοί ἐσμεν, οἱ γὰρ πάντες ἐκ τοῦ ἑνὸς ἄρτου μετέχομεν.
\vs{18}βλέπετε τὸν Ἰσραὴλ κατὰ σάρκα· οὐχ οἱ ἐσθίοντες τὰς θυσίας κοινωνοὶ τοῦ θυσιαστηρίου εἰσίν;
\vs{19}Τί οὖν φημι; ὅτι εἰδωλόθυτόν τί ἐστιν ἢ ὅτι εἴδωλόν τί ἐστιν;
\vs{20}ἀλλ᾽ ὅτι ἃ θύουσιν, δαιμονίοις καὶ οὐ Θεῷ θύουσιν· οὐ θέλω δὲ ὑμᾶς κοινωνοὺς τῶν δαιμονίων γίνεσθαι.
\vs{21}οὐ δύνασθε ποτήριον Κυρίου πίνειν καὶ ποτήριον δαιμονίων, οὐ δύνασθε τραπέζης Κυρίου μετέχειν καὶ τραπέζης δαιμονίων.
\vs{22}ἢ παραζηλοῦμεν τὸν Κύριον; μὴ ἰσχυρότεροι αὐτοῦ ἐσμεν;

\vs{23}Πάντα ἔξεστιν ἀλλ᾽ οὐ πάντα συμφέρει· Πάντα ἔξεστιν ἀλλ᾽ οὐ πάντα οἰκοδομεῖ.
\vs{24}μηδεὶς τὸ ἑαυτοῦ ζητείτω ἀλλὰ τὸ τοῦ ἑτέρου.
\vs{25}Πᾶν τὸ ἐν μακέλλῳ πωλούμενον ἐσθίετε μηδὲν ἀνακρίνοντες διὰ τὴν συνείδησιν·
\vs{26}Τοῦ Κυρίου γὰρ Ἡ γῆ καὶ τὸ πλήρωμα αὐτῆς.
\vs{27}Εἴ τις καλεῖ ὑμᾶς τῶν ἀπίστων καὶ θέλετε πορεύεσθαι, πᾶν τὸ παρατιθέμενον ὑμῖν ἐσθίετε μηδὲν ἀνακρίνοντες διὰ τὴν συνείδησιν.
\vs{28}ἐὰν δέ τις ὑμῖν εἴπῃ· Τοῦτο ἱερόθυτόν ἐστιν, μὴ ἐσθίετε δι᾽ ἐκεῖνον τὸν μηνύσαντα καὶ τὴν συνείδησιν·
\vs{29}συνείδησιν δὲ λέγω οὐχὶ τὴν ἑαυτοῦ ἀλλὰ τὴν τοῦ ἑτέρου. ἵνατί γὰρ ἡ ἐλευθερία μου κρίνεται ὑπὸ ἄλλης συνειδήσεως;
\vs{30}εἰ ἐγὼ χάριτι μετέχω, τί βλασφημοῦμαι ὑπὲρ οὗ ἐγὼ εὐχαριστῶ;
\vs{31}Εἴτε οὖν ἐσθίετε εἴτε πίνετε εἴτε τι ποιεῖτε, πάντα εἰς δόξαν Θεοῦ ποιεῖτε.
\vs{32}ἀπρόσκοποι καὶ Ἰουδαίοις γίνεσθε καὶ Ἕλλησιν καὶ τῇ ἐκκλησίᾳ τοῦ Θεοῦ,
\vs{33}καθὼς κἀγὼ πάντα πᾶσιν ἀρέσκω μὴ ζητῶν τὸ ἐμαυτοῦ σύμφορον ἀλλὰ τὸ τῶν πολλῶν, ἵνα σωθῶσιν.
\inparch{11} \vs{1}Μιμηταί μου γίνεσθε καθὼς κἀγὼ Χριστοῦ.

\vs{2}Ἐπαινῶ δὲ ὑμᾶς ὅτι πάντα μου μέμνησθε καὶ, καθὼς παρέδωκα ὑμῖν, τὰς παραδόσεις κατέχετε.
\vs{3}Θέλω δὲ ὑμᾶς εἰδέναι ὅτι παντὸς ἀνδρὸς ἡ κεφαλὴ ὁ Χριστός ἐστιν, κεφαλὴ δὲ γυναικὸς ὁ ἀνήρ, κεφαλὴ δὲ τοῦ Χριστοῦ ὁ Θεός.
\vs{4}Πᾶς ἀνὴρ προσευχόμενος ἢ προφητεύων κατὰ κεφαλῆς ἔχων καταισχύνει τὴν κεφαλὴν αὐτοῦ.
\vs{5}πᾶσα δὲ γυνὴ προσευχομένη ἢ προφητεύουσα ἀκατακαλύπτῳ τῇ κεφαλῇ καταισχύνει τὴν κεφαλὴν αὐτῆς· ἓν γάρ ἐστιν καὶ τὸ αὐτὸ τῇ ἐξυρημένῃ.
\vs{6}εἰ γὰρ οὐ κατακαλύπτεται γυνή, καὶ κειράσθω· εἰ δὲ αἰσχρὸν γυναικὶ τὸ κείρασθαι ἢ ξυρᾶσθαι, κατακαλυπτέσθω.
\vs{7}Ἀνὴρ μὲν γὰρ οὐκ ὀφείλει κατακαλύπτεσθαι τὴν κεφαλήν εἰκὼν καὶ δόξα Θεοῦ ὑπάρχων· ἡ γυνὴ δὲ δόξα ἀνδρός ἐστιν.
\vs{8}οὐ γάρ ἐστιν ἀνὴρ ἐκ γυναικός ἀλλὰ γυνὴ ἐξ ἀνδρός·
\vs{9}καὶ γὰρ οὐκ ἐκτίσθη ἀνὴρ διὰ τὴν γυναῖκα ἀλλὰ γυνὴ διὰ τὸν ἄνδρα.
\vs{10}διὰ τοῦτο ὀφείλει ἡ γυνὴ ἐξουσίαν ἔχειν ἐπὶ τῆς κεφαλῆς διὰ τοὺς ἀγγέλους.
\vs{11}Πλὴν οὔτε γυνὴ χωρὶς ἀνδρὸς οὔτε ἀνὴρ χωρὶς γυναικὸς ἐν Κυρίῳ·
\vs{12}ὥσπερ γὰρ ἡ γυνὴ ἐκ τοῦ ἀνδρός, οὕτως καὶ ὁ ἀνὴρ διὰ τῆς γυναικός· τὰ δὲ πάντα ἐκ τοῦ Θεοῦ.
\vs{13}Ἐν ὑμῖν αὐτοῖς κρίνατε· πρέπον ἐστὶν γυναῖκα ἀκατακάλυπτον τῷ Θεῷ προσεύχεσθαι;
\vs{14}οὐδὲ ἡ φύσις αὐτὴ διδάσκει ὑμᾶς ὅτι ἀνὴρ μὲν ἐὰν κομᾷ ἀτιμία αὐτῷ ἐστιν,
\vs{15}γυνὴ δὲ ἐὰν κομᾷ δόξα αὐτῇ ἐστιν; ὅτι ἡ κόμη ἀντὶ περιβολαίου δέδοται αὐτῇ.
\vs{16}Εἰ δέ τις δοκεῖ φιλόνεικος εἶναι, ἡμεῖς τοιαύτην συνήθειαν οὐκ ἔχομεν οὐδὲ αἱ ἐκκλησίαι τοῦ Θεοῦ.

\vs{17}Τοῦτο δὲ παραγγέλλων οὐκ ἐπαινῶ ὅτι οὐκ εἰς τὸ κρεῖσσον ἀλλὰ εἰς τὸ ἧσσον συνέρχεσθε.
\vs{18}πρῶτον μὲν γὰρ συνερχομένων ὑμῶν ἐν ἐκκλησίᾳ ἀκούω σχίσματα ἐν ὑμῖν ὑπάρχειν καὶ μέρος τι πιστεύω.
\vs{19}δεῖ γὰρ καὶ αἱρέσεις ἐν ὑμῖν εἶναι, ἵνα καὶ οἱ δόκιμοι φανεροὶ γένωνται ἐν ὑμῖν.
\vs{20}Συνερχομένων οὖν ὑμῶν ἐπὶ τὸ αὐτὸ οὐκ ἔστιν κυριακὸν δεῖπνον φαγεῖν·
\vs{21}ἕκαστος γὰρ τὸ ἴδιον δεῖπνον προλαμβάνει ἐν τῷ φαγεῖν, καὶ ὃς μὲν πεινᾷ ὃς δὲ μεθύει.
\vs{22}μὴ γὰρ οἰκίας οὐκ ἔχετε εἰς τὸ ἐσθίειν καὶ πίνειν; ἢ τῆς ἐκκλησίας τοῦ Θεοῦ καταφρονεῖτε, καὶ καταισχύνετε τοὺς μὴ ἔχοντας; τί εἴπω ὑμῖν; ἐπαινέσω ὑμᾶς; ἐν τούτῳ οὐκ ἐπαινῶ.

\vs{23}Ἐγὼ γὰρ παρέλαβον ἀπὸ τοῦ Κυρίου, ὃ καὶ παρέδωκα ὑμῖν, ὅτι ὁ Κύριος Ἰησοῦς ἐν τῇ νυκτὶ ᾗ παρεδίδετο ἔλαβεν ἄρτον
\vs{24}καὶ εὐχαριστήσας ἔκλασεν καὶ εἶπεν· Τοῦτό μού ἐστιν τὸ σῶμα τὸ ὑπὲρ ὑμῶν· τοῦτο ποιεῖτε εἰς τὴν ἐμὴν ἀνάμνησιν.
\vs{25}ὡσαύτως καὶ τὸ ποτήριον μετὰ τὸ δειπνῆσαι λέγων· Τοῦτο τὸ ποτήριον ἡ καινὴ διαθήκη ἐστὶν ἐν τῷ ἐμῷ αἵματι· τοῦτο ποιεῖτε, ὁσάκις ἐὰν πίνητε, εἰς τὴν ἐμὴν ἀνάμνησιν.
\vs{26}ὁσάκις γὰρ ἐὰν ἐσθίητε τὸν ἄρτον τοῦτον καὶ τὸ ποτήριον πίνητε, τὸν θάνατον τοῦ Κυρίου καταγγέλλετε ἄχρι οὗ ἔλθῃ.

\vs{27}Ὥστε ὃς ἂν ἐσθίῃ τὸν ἄρτον ἢ πίνῃ τὸ ποτήριον τοῦ Κυρίου ἀναξίως, ἔνοχος ἔσται τοῦ σώματος καὶ τοῦ αἵματος τοῦ Κυρίου.
\vs{28}δοκιμαζέτω δὲ ἄνθρωπος ἑαυτόν καὶ οὕτως ἐκ τοῦ ἄρτου ἐσθιέτω καὶ ἐκ τοῦ ποτηρίου πινέτω·
\vs{29}ὁ γὰρ ἐσθίων καὶ πίνων κρίμα ἑαυτῷ ἐσθίει καὶ πίνει μὴ διακρίνων τὸ σῶμα.
\vs{30}διὰ τοῦτο ἐν ὑμῖν πολλοὶ ἀσθενεῖς καὶ ἄρρωστοι καὶ κοιμῶνται ἱκανοί.
\vs{31}Εἰ δὲ ἑαυτοὺς διεκρίνομεν, οὐκ ἂν ἐκρινόμεθα·
\vs{32}κρινόμενοι δὲ ὑπὸ τοῦ Κυρίου παιδευόμεθα, ἵνα μὴ σὺν τῷ κόσμῳ κατακριθῶμεν.
\vs{33}Ὥστε, ἀδελφοί μου, συνερχόμενοι εἰς τὸ φαγεῖν ἀλλήλους ἐκδέχεσθε.
\vs{34}εἴ τις πεινᾷ, ἐν οἴκῳ ἐσθιέτω, ἵνα μὴ εἰς κρίμα συνέρχησθε. Τὰ δὲ λοιπὰ ὡς ἂν ἔλθω διατάξομαι.

\ch{12}
Περὶ δὲ τῶν πνευματικῶν, ἀδελφοί, οὐ θέλω ὑμᾶς ἀγνοεῖν.
\vs{2}Οἴδατε ὅτι ὅτε ἔθνη ἦτε πρὸς τὰ εἴδωλα τὰ ἄφωνα ὡς ἂν ἤγεσθε ἀπαγόμενοι.
\vs{3}διὸ γνωρίζω ὑμῖν ὅτι οὐδεὶς ἐν Πνεύματι Θεοῦ λαλῶν λέγει· Αναθεμα ΙΗΣΟΥΣ, καὶ οὐδεὶς δύναται εἰπεῖν· Κυριος ΙΗΣΟΥΣ, εἰ μὴ ἐν Πνεύματι Ἁγίῳ.

\vs{4}Διαιρέσεις δὲ χαρισμάτων εἰσίν, τὸ δὲ αὐτὸ Πνεῦμα·
\vs{5}καὶ διαιρέσεις διακονιῶν εἰσιν, καὶ ὁ αὐτὸς Κύριος·
\vs{6}καὶ διαιρέσεις ἐνεργημάτων εἰσίν, ὁ δὲ αὐτὸς Θεός ὁ ἐνεργῶν τὰ πάντα ἐν πᾶσιν.
\vs{7}Ἑκάστῳ δὲ δίδοται ἡ φανέρωσις τοῦ Πνεύματος πρὸς τὸ συμφέρον.
\vs{8}ᾧ μὲν γὰρ διὰ τοῦ Πνεύματος δίδοται λόγος σοφίας, ἄλλῳ δὲ λόγος γνώσεως κατὰ τὸ αὐτὸ Πνεῦμα,
\vs{9}ἑτέρῳ πίστις ἐν τῷ αὐτῷ Πνεύματι, ἄλλῳ δὲ χαρίσματα ἰαμάτων ἐν τῷ ἑνὶ Πνεύματι,
\vs{10}ἄλλῳ δὲ ἐνεργήματα δυνάμεων, ἄλλῳ δὲ προφητεία, ἄλλῳ δὲ διακρίσεις πνευμάτων, ἑτέρῳ γένη γλωσσῶν, ἄλλῳ δὲ ἑρμηνεία γλωσσῶν·
\vs{11}πάντα δὲ ταῦτα ἐνεργεῖ τὸ ἓν καὶ τὸ αὐτὸ Πνεῦμα διαιροῦν ἰδίᾳ ἑκάστῳ καθὼς βούλεται.

\vs{12}Καθάπερ γὰρ τὸ σῶμα ἕν ἐστιν καὶ μέλη πολλὰ ἔχει, πάντα δὲ τὰ μέλη τοῦ σώματος πολλὰ ὄντα ἕν ἐστιν σῶμα, οὕτως καὶ ὁ Χριστός·
\vs{13}καὶ γὰρ ἐν ἑνὶ Πνεύματι ἡμεῖς πάντες εἰς ἓν σῶμα ἐβαπτίσθημεν, εἴτε Ἰουδαῖοι εἴτε Ἕλληνες εἴτε δοῦλοι εἴτε ἐλεύθεροι, καὶ πάντες ἓν Πνεῦμα ἐποτίσθημεν.
\vs{14}Καὶ γὰρ τὸ σῶμα οὐκ ἔστιν ἓν μέλος ἀλλὰ πολλά.
\vs{15}ἐὰν εἴπῃ ὁ πούς· Ὅτι οὐκ εἰμὶ χείρ, οὐκ εἰμὶ ἐκ τοῦ σώματος, οὐ παρὰ τοῦτο οὐκ ἔστιν ἐκ τοῦ σώματος;
\vs{16}καὶ ἐὰν εἴπῃ τὸ οὖς· Ὅτι οὐκ εἰμὶ ὀφθαλμός, οὐκ εἰμὶ ἐκ τοῦ σώματος, οὐ παρὰ τοῦτο οὐκ ἔστιν ἐκ τοῦ σώματος;
\vs{17}εἰ ὅλον τὸ σῶμα ὀφθαλμός, ποῦ ἡ ἀκοή; εἰ ὅλον ἀκοή, ποῦ ἡ ὄσφρησις;
\vs{18}Νυνὶ δὲ ὁ Θεὸς ἔθετο τὰ μέλη, ἓν ἕκαστον αὐτῶν ἐν τῷ σώματι καθὼς ἠθέλησεν.
\vs{19}εἰ δὲ ἦν τὰ πάντα ἓν μέλος, ποῦ τὸ σῶμα;
\vs{20}νῦν δὲ πολλὰ μὲν μέλη, ἓν δὲ σῶμα.
\vs{21}Οὐ δύναται δὲ ὁ ὀφθαλμὸς εἰπεῖν τῇ χειρί· Χρείαν σου οὐκ ἔχω, ἢ πάλιν ἡ κεφαλὴ τοῖς ποσίν· Χρείαν ὑμῶν οὐκ ἔχω·
\vs{22}ἀλλὰ πολλῷ μᾶλλον τὰ δοκοῦντα μέλη τοῦ σώματος ἀσθενέστερα ὑπάρχειν ἀναγκαῖά ἐστιν,
\vs{23}καὶ ἃ δοκοῦμεν ἀτιμότερα εἶναι τοῦ σώματος τούτοις τιμὴν περισσοτέραν περιτίθεμεν, καὶ τὰ ἀσχήμονα ἡμῶν εὐσχημοσύνην περισσοτέραν ἔχει,
\vs{24}τὰ δὲ εὐσχήμονα ἡμῶν οὐ χρείαν ἔχει. Ἀλλὰ ὁ θεὸς συνεκέρασεν τὸ σῶμα τῷ ὑστερουμένῳ περισσοτέραν δοὺς τιμήν,
\vs{25}ἵνα μὴ ᾖ σχίσμα ἐν τῷ σώματι ἀλλὰ τὸ αὐτὸ ὑπὲρ ἀλλήλων μεριμνῶσιν τὰ μέλη.
\vs{26}καὶ εἴτε πάσχει ἓν μέλος, συμπάσχει πάντα τὰ μέλη· εἴτε δοξάζεται ἓν μέλος, συνχαίρει πάντα τὰ μέλη.
\vs{27}Ὑμεῖς δέ ἐστε σῶμα Χριστοῦ καὶ μέλη ἐκ μέρους.
\vs{28}Καὶ οὓς μὲν ἔθετο ὁ Θεὸς ἐν τῇ ἐκκλησίᾳ πρῶτον ἀποστόλους, δεύτερον προφήτας, τρίτον διδασκάλους, ἔπειτα δυνάμεις, ἔπειτα χαρίσματα ἰαμάτων, ἀντιλήμψεις, κυβερνήσεις, γένη γλωσσῶν.
\vs{29}μὴ πάντες ἀπόστολοι; μὴ πάντες προφῆται; μὴ πάντες διδάσκαλοι; μὴ πάντες δυνάμεις;
\vs{30}μὴ πάντες χαρίσματα ἔχουσιν ἰαμάτων; μὴ πάντες γλώσσαις λαλοῦσιν; μὴ πάντες διερμηνεύουσιν;
\vs{31}ζηλοῦτε δὲ τὰ χαρίσματα τὰ μείζονα.

Καὶ ἔτι καθ᾽ ὑπερβολὴν ὁδὸν ὑμῖν δείκνυμι.

\ch{13}
Ἐὰν ταῖς γλώσσαις τῶν ἀνθρώπων λαλῶ καὶ τῶν ἀγγέλων, ἀγάπην δὲ μὴ ἔχω, γέγονα χαλκὸς ἠχῶν ἢ κύμβαλον ἀλαλάζον.
\vs{2}καὶ ἐὰν ἔχω προφητείαν καὶ εἰδῶ τὰ μυστήρια πάντα καὶ πᾶσαν τὴν γνῶσιν καὶ ἐὰν ἔχω πᾶσαν τὴν πίστιν ὥστε ὄρη μεθιστάναι, ἀγάπην δὲ μὴ ἔχω, οὐθέν εἰμι.
\vs{3}κἂν ψωμίσω πάντα τὰ ὑπάρχοντά μου καὶ ἐὰν παραδῶ τὸ σῶμά μου ἵνα καυχήσωμαι, ἀγάπην δὲ μὴ ἔχω, οὐδὲν ὠφελοῦμαι.

\vs{4}Ἡ ἀγάπη μακροθυμεῖ, χρηστεύεται ἡ ἀγάπη, οὐ ζηλοῖ, ἡ ἀγάπη οὐ περπερεύεται, οὐ φυσιοῦται,
\vs{5}οὐκ ἀσχημονεῖ, οὐ ζητεῖ τὰ ἑαυτῆς, οὐ παροξύνεται, οὐ λογίζεται τὸ κακόν,
\vs{6}οὐ χαίρει ἐπὶ τῇ ἀδικίᾳ, συνχαίρει δὲ τῇ ἀληθείᾳ·
\vs{7}πάντα στέγει, πάντα πιστεύει, πάντα ἐλπίζει, πάντα ὑπομένει.

\vs{8}Ἡ ἀγάπη οὐδέποτε πίπτει· εἴτε δὲ προφητεῖαι, καταργηθήσονται· εἴτε γλῶσσαι, παύσονται· εἴτε γνῶσις, καταργηθήσεται.
\vs{9}ἐκ μέρους γὰρ γινώσκομεν καὶ ἐκ μέρους προφητεύομεν·
\vs{10}ὅταν δὲ ἔλθῃ τὸ τέλειον, τὸ ἐκ μέρους καταργηθήσεται.
\vs{11}Ὅτε ἤμην νήπιος, ἐλάλουν ὡς νήπιος, ἐφρόνουν ὡς νήπιος, ἐλογιζόμην ὡς νήπιος· ὅτε γέγονα ἀνήρ, κατήργηκα τὰ τοῦ νηπίου.
\vs{12}βλέπομεν γὰρ ἄρτι δι᾽ ἐσόπτρου ἐν αἰνίγματι, τότε δὲ πρόσωπον πρὸς πρόσωπον· ἄρτι γινώσκω ἐκ μέρους, τότε δὲ ἐπιγνώσομαι καθὼς καὶ ἐπεγνώσθην.
\vs{13}Νυνὶ δὲ μένει πίστις, ἐλπίς, ἀγάπη, τὰ τρία ταῦτα· μείζων δὲ τούτων ἡ ἀγάπη.

\ch{14}
Διώκετε τὴν ἀγάπην, ζηλοῦτε δὲ τὰ πνευματικά, μᾶλλον δὲ ἵνα προφητεύητε.
\vs{2}ὁ γὰρ λαλῶν γλώσσῃ οὐκ ἀνθρώποις λαλεῖ ἀλλὰ Θεῷ· οὐδεὶς γὰρ ἀκούει, πνεύματι δὲ λαλεῖ μυστήρια·
\vs{3}ὁ δὲ προφητεύων ἀνθρώποις λαλεῖ οἰκοδομὴν καὶ παράκλησιν καὶ παραμυθίαν.
\vs{4}ὁ λαλῶν γλώσσῃ ἑαυτὸν οἰκοδομεῖ· ὁ δὲ προφητεύων ἐκκλησίαν οἰκοδομεῖ.
\vs{5}Θέλω δὲ πάντας ὑμᾶς λαλεῖν γλώσσαις, μᾶλλον δὲ ἵνα προφητεύητε· μείζων δὲ ὁ προφητεύων ἢ ὁ λαλῶν γλώσσαις ἐκτὸς εἰ μὴ διερμηνεύῃ, ἵνα ἡ ἐκκλησία οἰκοδομὴν λάβῃ.

\vs{6}Νῦν δέ, ἀδελφοί, ἐὰν ἔλθω πρὸς ὑμᾶς γλώσσαις λαλῶν, τί ὑμᾶς ὠφελήσω ἐὰν μὴ ὑμῖν λαλήσω ἢ ἐν ἀποκαλύψει ἢ ἐν γνώσει ἢ ἐν προφητείᾳ ἢ ἐν διδαχῇ;
\vs{7}ὅμως τὰ ἄψυχα φωνὴν διδόντα, εἴτε αὐλὸς εἴτε κιθάρα, ἐὰν διαστολὴν τοῖς φθόγγοις μὴ δῷ, πῶς γνωσθήσεται τὸ αὐλούμενον ἢ τὸ κιθαριζόμενον;
\vs{8}Καὶ γὰρ ἐὰν ἄδηλον σάλπιγξ φωνὴν δῷ, τίς παρασκευάσεται εἰς πόλεμον;
\vs{9}οὕτως καὶ ὑμεῖς διὰ τῆς γλώσσης ἐὰν μὴ εὔσημον λόγον δῶτε, πῶς γνωσθήσεται τὸ λαλούμενον; ἔσεσθε γὰρ εἰς ἀέρα λαλοῦντες.
\vs{10}Τοσαῦτα εἰ τύχοι γένη φωνῶν εἰσιν ἐν κόσμῳ καὶ οὐδὲν ἄφωνον·
\vs{11}ἐὰν οὖν μὴ εἰδῶ τὴν δύναμιν τῆς φωνῆς, ἔσομαι τῷ λαλοῦντι βάρβαρος καὶ ὁ λαλῶν ἐν ἐμοὶ βάρβαρος.
\vs{12}Οὕτως καὶ ὑμεῖς, ἐπεὶ ζηλωταί ἐστε πνευμάτων, πρὸς τὴν οἰκοδομὴν τῆς ἐκκλησίας ζητεῖτε ἵνα περισσεύητε.

\vs{13}Διὸ ὁ λαλῶν γλώσσῃ προσευχέσθω ἵνα διερμηνεύῃ.
\vs{14}ἐὰν γὰρ προσεύχωμαι γλώσσῃ, τὸ πνεῦμά μου προσεύχεται, ὁ δὲ νοῦς μου ἄκαρπός ἐστιν.
\vs{15}Τί οὖν ἐστιν; προσεύξομαι τῷ πνεύματι, προσεύξομαι δὲ καὶ τῷ νοΐ· ψαλῶ τῷ πνεύματι, ψαλῶ δὲ καὶ τῷ νοΐ.
\vs{16}ἐπεὶ ἐὰν εὐλογῇς ἐν πνεύματι, ὁ ἀναπληρῶν τὸν τόπον τοῦ ἰδιώτου πῶς ἐρεῖ τὸ Ἀμήν ἐπὶ τῇ σῇ εὐχαριστίᾳ; ἐπειδὴ τί λέγεις οὐκ οἶδεν·
\vs{17}σὺ μὲν γὰρ καλῶς εὐχαριστεῖς ἀλλ᾽ ὁ ἕτερος οὐκ οἰκοδομεῖται.
\vs{18}Εὐχαριστῶ τῷ Θεῷ, πάντων ὑμῶν μᾶλλον γλώσσαις λαλῶ·
\vs{19}ἀλλὰ ἐν ἐκκλησίᾳ θέλω πέντε λόγους τῷ νοΐ μου λαλῆσαι, ἵνα καὶ ἄλλους κατηχήσω, ἢ μυρίους λόγους ἐν γλώσσῃ.

\vs{20}Ἀδελφοί, μὴ παιδία γίνεσθε ταῖς φρεσίν ἀλλὰ τῇ κακίᾳ νηπιάζετε, ταῖς δὲ φρεσὶν τέλειοι γίνεσθε.
\vs{21}ἐν τῷ νόμῳ γέγραπται ὅτι 
\begin{poetryblock}

\begin{quote}Ἐν ἑτερογλώσσοις καὶ ἐν χείλεσιν ἑτέρων λαλήσω τῷ λαῷ τούτῳ\end{quote} 

\begin{quote}καὶ οὐδ᾽ οὕτως εἰσακούσονταί μου, λέγει Κύριος.\end{quote}
\end{poetryblock}

\vs{22}Ὥστε αἱ γλῶσσαι εἰς σημεῖόν εἰσιν οὐ τοῖς πιστεύουσιν ἀλλὰ τοῖς ἀπίστοις, ἡ δὲ προφητεία οὐ τοῖς ἀπίστοις ἀλλὰ τοῖς πιστεύουσιν.
\vs{23}Ἐὰν οὖν συνέλθῃ ἡ ἐκκλησία ὅλη ἐπὶ τὸ αὐτὸ καὶ πάντες λαλῶσιν γλώσσαις, εἰσέλθωσιν δὲ ἰδιῶται ἢ ἄπιστοι, οὐκ ἐροῦσιν ὅτι μαίνεσθε;
\vs{24}ἐὰν δὲ πάντες προφητεύωσιν, εἰσέλθῃ δέ τις ἄπιστος ἢ ἰδιώτης, ἐλέγχεται ὑπὸ πάντων, ἀνακρίνεται ὑπὸ πάντων,
\vs{25}τὰ κρυπτὰ τῆς καρδίας αὐτοῦ φανερὰ γίνεται, καὶ οὕτως πεσὼν ἐπὶ πρόσωπον προσκυνήσει τῷ Θεῷ ἀπαγγέλλων ὅτι Ὄντως ὁ Θεὸς ἐν ὑμῖν ἐστιν.

\vs{26}Τί οὖν ἐστιν, ἀδελφοί; ὅταν συνέρχησθε, ἕκαστος ψαλμὸν ἔχει, διδαχὴν ἔχει, ἀποκάλυψιν ἔχει, γλῶσσαν ἔχει, ἑρμηνείαν ἔχει· πάντα πρὸς οἰκοδομὴν γινέσθω.
\vs{27}Εἴτε γλώσσῃ τις λαλεῖ, κατὰ δύο ἢ τὸ πλεῖστον τρεῖς καὶ ἀνὰ μέρος, καὶ εἷς διερμηνευέτω·
\vs{28}ἐὰν δὲ μὴ ᾖ διερμηνευτής, σιγάτω ἐν ἐκκλησίᾳ, ἑαυτῷ δὲ λαλείτω καὶ τῷ Θεῷ.
\vs{29}Προφῆται δὲ δύο ἢ τρεῖς λαλείτωσαν καὶ οἱ ἄλλοι διακρινέτωσαν·
\vs{30}ἐὰν δὲ ἄλλῳ ἀποκαλυφθῇ καθημένῳ, ὁ πρῶτος σιγάτω.
\vs{31}δύνασθε γὰρ καθ᾽ ἕνα πάντες προφητεύειν, ἵνα πάντες μανθάνωσιν καὶ πάντες παρακαλῶνται.
\vs{32}καὶ πνεύματα προφητῶν προφήταις ὑποτάσσεται,
\vs{33}οὐ γάρ ἐστιν ἀκαταστασίας ὁ Θεὸς ἀλλὰ εἰρήνης.

Ὡς ἐν πάσαις ταῖς ἐκκλησίαις τῶν ἁγίων
\vs{34}αἱ γυναῖκες ἐν ταῖς ἐκκλησίαις σιγάτωσαν· οὐ γὰρ ἐπιτρέπεται αὐταῖς λαλεῖν, ἀλλὰ ὑποτασσέσθωσαν, καθὼς καὶ ὁ νόμος λέγει.
\vs{35}εἰ δέ τι μαθεῖν θέλουσιν, ἐν οἴκῳ τοὺς ἰδίους ἄνδρας ἐπερωτάτωσαν· αἰσχρὸν γάρ ἐστιν γυναικὶ λαλεῖν ἐν ἐκκλησίᾳ.
\vs{36}Ἢ ἀφ᾽ ὑμῶν ὁ λόγος τοῦ Θεοῦ ἐξῆλθεν, ἢ εἰς ὑμᾶς μόνους κατήντησεν;

\vs{37}Εἴ τις δοκεῖ προφήτης εἶναι ἢ πνευματικός, ἐπιγινωσκέτω ἃ γράφω ὑμῖν ὅτι Κυρίου ἐστὶν ἐντολή·
\vs{38}εἰ δέ τις ἀγνοεῖ, ἀγνοεῖται.
\vs{39}Ὥστε, ἀδελφοί μου, ζηλοῦτε τὸ προφητεύειν καὶ τὸ λαλεῖν μὴ κωλύετε γλώσσαις·
\vs{40}πάντα δὲ εὐσχημόνως καὶ κατὰ τάξιν γινέσθω.

\ch{15}
Γνωρίζω δὲ ὑμῖν, ἀδελφοί, τὸ εὐαγγέλιον ὃ εὐηγγελισάμην ὑμῖν, ὃ καὶ παρελάβετε, ἐν ᾧ καὶ ἑστήκατε,
\vs{2}δι᾽ οὗ καὶ σῴζεσθε, τίνι λόγῳ εὐηγγελισάμην ὑμῖν εἰ κατέχετε, ἐκτὸς εἰ μὴ εἰκῇ ἐπιστεύσατε.
\vs{3}Παρέδωκα γὰρ ὑμῖν ἐν πρώτοις, ὃ καὶ παρέλαβον, ὅτι Χριστὸς ἀπέθανεν ὑπὲρ τῶν ἁμαρτιῶν ἡμῶν κατὰ τὰς γραφάς
\vs{4}καὶ ὅτι ἐτάφη καὶ ὅτι ἐγήγερται τῇ ἡμέρᾳ τῇ τρίτῃ κατὰ τὰς γραφάς
\vs{5}καὶ ὅτι ὤφθη Κηφᾷ εἶτα τοῖς δώδεκα·
\vs{6}ἔπειτα ὤφθη ἐπάνω πεντακοσίοις ἀδελφοῖς ἐφάπαξ, ἐξ ὧν οἱ πλείονες μένουσιν ἕως ἄρτι, τινὲς δὲ ἐκοιμήθησαν·
\vs{7}ἔπειτα ὤφθη Ἰακώβῳ εἶτα τοῖς ἀποστόλοις πᾶσιν·
\vs{8}ἔσχατον δὲ πάντων ὡσπερεὶ τῷ ἐκτρώματι ὤφθη κἀμοί.
\vs{9}Ἐγὼ γάρ εἰμι ὁ ἐλάχιστος τῶν ἀποστόλων ὃς οὐκ εἰμὶ ἱκανὸς καλεῖσθαι ἀπόστολος, διότι ἐδίωξα τὴν ἐκκλησίαν τοῦ Θεοῦ·
\vs{10}χάριτι δὲ Θεοῦ εἰμι ὅ εἰμι, καὶ ἡ χάρις αὐτοῦ ἡ εἰς ἐμὲ οὐ κενὴ ἐγενήθη, ἀλλὰ περισσότερον αὐτῶν πάντων ἐκοπίασα, οὐκ ἐγὼ δὲ ἀλλὰ ἡ χάρις τοῦ Θεοῦ ἡ σὺν ἐμοί.
\vs{11}εἴτε οὖν ἐγὼ εἴτε ἐκεῖνοι, οὕτως κηρύσσομεν καὶ οὕτως ἐπιστεύσατε.

\vs{12}Εἰ δὲ Χριστὸς κηρύσσεται ὅτι ἐκ νεκρῶν ἐγήγερται, πῶς λέγουσιν ἐν ὑμῖν τινες ὅτι ἀνάστασις νεκρῶν οὐκ ἔστιν;
\vs{13}εἰ δὲ ἀνάστασις νεκρῶν οὐκ ἔστιν, οὐδὲ Χριστὸς ἐγήγερται·
\vs{14}εἰ δὲ Χριστὸς οὐκ ἐγήγερται, κενὸν ἄρα καὶ τὸ κήρυγμα ἡμῶν, κενὴ καὶ ἡ πίστις ὑμῶν·
\vs{15}εὑρισκόμεθα δὲ καὶ ψευδομάρτυρες τοῦ Θεοῦ, ὅτι ἐμαρτυρήσαμεν κατὰ τοῦ Θεοῦ ὅτι ἤγειρεν τὸν Χριστόν, ὃν οὐκ ἤγειρεν εἴπερ ἄρα νεκροὶ οὐκ ἐγείρονται.
\vs{16}Εἰ γὰρ νεκροὶ οὐκ ἐγείρονται, οὐδὲ Χριστὸς ἐγήγερται·
\vs{17}εἰ δὲ Χριστὸς οὐκ ἐγήγερται, ματαία ἡ πίστις ὑμῶν, ἔτι ἐστὲ ἐν ταῖς ἁμαρτίαις ὑμῶν,
\vs{18}ἄρα καὶ οἱ κοιμηθέντες ἐν Χριστῷ ἀπώλοντο.
\vs{19}εἰ ἐν τῇ ζωῇ ταύτῃ ἐν Χριστῷ ἠλπικότες ἐσμὲν μόνον, ἐλεεινότεροι πάντων ἀνθρώπων ἐσμέν.

\vs{20}Νυνὶ δὲ Χριστὸς ἐγήγερται ἐκ νεκρῶν ἀπαρχὴ τῶν κεκοιμημένων.
\vs{21}ἐπειδὴ γὰρ δι᾽ ἀνθρώπου θάνατος, καὶ δι᾽ ἀνθρώπου ἀνάστασις νεκρῶν.
\vs{22}ὥσπερ γὰρ ἐν τῷ Ἀδὰμ πάντες ἀποθνήσκουσιν, οὕτως καὶ ἐν τῷ Χριστῷ πάντες ζωοποιηθήσονται.
\vs{23}Ἕκαστος δὲ ἐν τῷ ἰδίῳ τάγματι· ἀπαρχὴ Χριστός, ἔπειτα οἱ τοῦ Χριστοῦ ἐν τῇ παρουσίᾳ αὐτοῦ,
\vs{24}εἶτα τὸ τέλος, ὅταν παραδιδῷ τὴν βασιλείαν τῷ Θεῷ καὶ Πατρί, ὅταν καταργήσῃ πᾶσαν ἀρχὴν καὶ πᾶσαν ἐξουσίαν καὶ δύναμιν.
\vs{25}δεῖ γὰρ αὐτὸν βασιλεύειν ἄχρι οὗ θῇ πάντας τοὺς ἐχθροὺς ὑπὸ τοὺς πόδας αὐτοῦ.
\vs{26}ἔσχατος ἐχθρὸς καταργεῖται ὁ θάνατος·
\vs{27}Πάντα γὰρ Ὑπέταξεν ὑπὸ τοὺς πόδας αὐτοῦ. ὅταν δὲ εἴπῃ ὅτι πάντα ὑποτέτακται, δῆλον ὅτι ἐκτὸς τοῦ ὑποτάξαντος αὐτῷ τὰ πάντα.
\vs{28}ὅταν δὲ ὑποταγῇ αὐτῷ τὰ πάντα, τότε καὶ αὐτὸς ὁ Υἱὸς ὑποταγήσεται τῷ ὑποτάξαντι αὐτῷ τὰ πάντα, ἵνα ᾖ ὁ Θεὸς τὰ πάντα ἐν πᾶσιν.

\vs{29}Ἐπεὶ τί ποιήσουσιν οἱ βαπτιζόμενοι ὑπὲρ τῶν νεκρῶν; εἰ ὅλως νεκροὶ οὐκ ἐγείρονται, τί καὶ βαπτίζονται ὑπὲρ αὐτῶν;
\vs{30}τί καὶ ἡμεῖς κινδυνεύομεν πᾶσαν ὥραν;
\vs{31}καθ᾽ ἡμέραν ἀποθνῄσκω, νὴ τὴν ὑμετέραν καύχησιν, ἀδελφοί, ἣν ἔχω ἐν Χριστῷ Ἰησοῦ τῷ Κυρίῳ ἡμῶν.
\vs{32}εἰ κατὰ ἄνθρωπον ἐθηριομάχησα ἐν Ἐφέσῳ, τί μοι τὸ ὄφελος; εἰ νεκροὶ οὐκ ἐγείρονται, Φάγωμεν καὶ πίωμεν, αὔριον γὰρ ἀποθνήσκομεν.
\vs{33}Μὴ πλανᾶσθε· 
\begin{poetryblock}

\begin{quote}Φθείρουσιν ἤθη χρηστὰ ὁμιλίαι κακαί.\end{quote}
\end{poetryblock}

\vs{34}ἐκνήψατε δικαίως καὶ μὴ ἁμαρτάνετε, ἀγνωσίαν γὰρ Θεοῦ τινες ἔχουσιν, πρὸς ἐντροπὴν ὑμῖν λαλῶ.

\vs{35}Ἀλλὰ ἐρεῖ τις· Πῶς ἐγείρονται οἱ νεκροί; ποίῳ δὲ σώματι ἔρχονται;
\vs{36}ἄφρων, σὺ ὃ σπείρεις, οὐ ζωοποιεῖται ἐὰν μὴ ἀποθάνῃ·
\vs{37}καὶ ὃ σπείρεις, οὐ τὸ σῶμα τὸ γενησόμενον σπείρεις ἀλλὰ γυμνὸν κόκκον εἰ τύχοι σίτου ἤ τινος τῶν λοιπῶν·
\vs{38}ὁ δὲ Θεὸς δίδωσιν αὐτῷ σῶμα καθὼς ἠθέλησεν, καὶ ἑκάστῳ τῶν σπερμάτων ἴδιον σῶμα.
\vs{39}Οὐ πᾶσα σὰρξ ἡ αὐτὴ σάρξ ἀλλὰ ἄλλη μὲν ἀνθρώπων, ἄλλη δὲ σὰρξ κτηνῶν, ἄλλη δὲ σὰρξ πτηνῶν, ἄλλη δὲ ἰχθύων.
\vs{40}καὶ σώματα ἐπουράνια, καὶ σώματα ἐπίγεια· ἀλλὰ ἑτέρα μὲν ἡ τῶν ἐπουρανίων δόξα, ἑτέρα δὲ ἡ τῶν ἐπιγείων.
\vs{41}ἄλλη δόξα ἡλίου, καὶ ἄλλη δόξα σελήνης, καὶ ἄλλη δόξα ἀστέρων· ἀστὴρ γὰρ ἀστέρος διαφέρει ἐν δόξῃ.
\vs{42}Οὕτως καὶ ἡ ἀνάστασις τῶν νεκρῶν. σπείρεται ἐν φθορᾷ, ἐγείρεται ἐν ἀφθαρσίᾳ·
\vs{43}σπείρεται ἐν ἀτιμίᾳ, ἐγείρεται ἐν δόξῃ· σπείρεται ἐν ἀσθενείᾳ, ἐγείρεται ἐν δυνάμει·
\vs{44}σπείρεται σῶμα ψυχικόν, ἐγείρεται σῶμα πνευματικόν. Εἰ ἔστιν σῶμα ψυχικόν, ἔστιν καὶ πνευματικόν.
\vs{45}οὕτως καὶ γέγραπται· Ἐγένετο ὁ πρῶτος ἄνθρωπος Ἀδὰμ εἰς ψυχὴν ζῶσαν, ὁ ἔσχατος Ἀδὰμ εἰς πνεῦμα ζωοποιοῦν.
\vs{46}Ἀλλ᾽ οὐ πρῶτον τὸ πνευματικὸν ἀλλὰ τὸ ψυχικόν, ἔπειτα τὸ πνευματικόν.
\vs{47}ὁ πρῶτος ἄνθρωπος ἐκ γῆς χοϊκός, ὁ δεύτερος ἄνθρωπος ἐξ οὐρανοῦ.
\vs{48}οἷος ὁ χοϊκός, τοιοῦτοι καὶ οἱ χοϊκοί, καὶ οἷος ὁ ἐπουράνιος, τοιοῦτοι καὶ οἱ ἐπουράνιοι·
\vs{49}καὶ καθὼς ἐφορέσαμεν τὴν εἰκόνα τοῦ χοϊκοῦ, φορέσομεν καὶ τὴν εἰκόνα τοῦ ἐπουρανίου.

\vs{50}Τοῦτο δέ φημι, ἀδελφοί, ὅτι σὰρξ καὶ αἷμα βασιλείαν Θεοῦ κληρονομῆσαι οὐ δύναται οὐδὲ ἡ φθορὰ τὴν ἀφθαρσίαν κληρονομεῖ.
\vs{51}Ἰδοὺ μυστήριον ὑμῖν λέγω· πάντες οὐ κοιμηθησόμεθα, πάντες δὲ ἀλλαγησόμεθα,
\vs{52}ἐν ἀτόμῳ, ἐν ῥιπῇ ὀφθαλμοῦ, ἐν τῇ ἐσχάτῃ σάλπιγγι· σαλπίσει γάρ καὶ οἱ νεκροὶ ἐγερθήσονται ἄφθαρτοι καὶ ἡμεῖς ἀλλαγησόμεθα.
\vs{53}δεῖ γὰρ τὸ φθαρτὸν τοῦτο ἐνδύσασθαι ἀφθαρσίαν καὶ τὸ θνητὸν τοῦτο ἐνδύσασθαι ἀθανασίαν.
\vs{54}Ὅταν δὲ τὸ φθαρτὸν τοῦτο ἐνδύσηται ἀφθαρσίαν καὶ τὸ θνητὸν τοῦτο ἐνδύσηται ἀθανασίαν, τότε γενήσεται ὁ λόγος ὁ γεγραμμένος· 
\begin{poetryblock}

\begin{quote}Κατεπόθη ὁ θάνατος εἰς νῖκος.\end{quote}
\end{poetryblock}

\vs{55}Ποῦ σου, θάνατε, τὸ νῖκος; 
\begin{poetryblock}

\begin{quote}ποῦ σου, θάνατε, τὸ κέντρον;\end{quote}
\end{poetryblock}

\vs{56}Τὸ δὲ κέντρον τοῦ θανάτου ἡ ἁμαρτία, ἡ δὲ δύναμις τῆς ἁμαρτίας ὁ νόμος·
\vs{57}τῷ δὲ Θεῷ χάρις τῷ διδόντι ἡμῖν τὸ νῖκος διὰ τοῦ Κυρίου ἡμῶν Ἰησοῦ Χριστοῦ.
\vs{58}Ὥστε, ἀδελφοί μου ἀγαπητοί, ἑδραῖοι γίνεσθε, ἀμετακίνητοι, περισσεύοντες ἐν τῷ ἔργῳ τοῦ Κυρίου πάντοτε, εἰδότες ὅτι ὁ κόπος ὑμῶν οὐκ ἔστιν κενὸς ἐν Κυρίῳ.

\ch{16}
Περὶ δὲ τῆς λογείας τῆς εἰς τοὺς ἁγίους ὥσπερ διέταξα ταῖς ἐκκλησίαις τῆς Γαλατίας, οὕτως καὶ ὑμεῖς ποιήσατε.
\vs{2}κατὰ μίαν σαββάτου ἕκαστος ὑμῶν παρ᾽ ἑαυτῷ τιθέτω θησαυρίζων ὅ τι ἐὰν εὐοδῶται, ἵνα μὴ ὅταν ἔλθω τότε λογεῖαι γίνωνται.
\vs{3}ὅταν δὲ παραγένωμαι, οὓς ἐὰν δοκιμάσητε, δι᾽ ἐπιστολῶν τούτους πέμψω ἀπενεγκεῖν τὴν χάριν ὑμῶν εἰς Ἰερουσαλήμ·
\vs{4}ἐὰν δὲ ἄξιον ᾖ τοῦ κἀμὲ πορεύεσθαι, σὺν ἐμοὶ πορεύσονται.

\vs{5}Ἐλεύσομαι δὲ πρὸς ὑμᾶς ὅταν Μακεδονίαν διέλθω· Μακεδονίαν γὰρ διέρχομαι,
\vs{6}πρὸς ὑμᾶς δὲ τυχὸν παραμενῶ ἢ καὶ παραχειμάσω, ἵνα ὑμεῖς με προπέμψητε οὗ ἐὰν πορεύωμαι.
\vs{7}οὐ θέλω γὰρ ὑμᾶς ἄρτι ἐν παρόδῳ ἰδεῖν, ἐλπίζω γὰρ χρόνον τινὰ ἐπιμεῖναι πρὸς ὑμᾶς ἐὰν ὁ Κύριος ἐπιτρέψῃ.
\vs{8}ἐπιμενῶ δὲ ἐν Ἐφέσῳ ἕως τῆς Πεντηκοστῆς·
\vs{9}θύρα γάρ μοι ἀνέῳγεν μεγάλη καὶ ἐνεργής, καὶ ἀντικείμενοι πολλοί.

\vs{10}Ἐὰν δὲ ἔλθῃ Τιμόθεος, βλέπετε, ἵνα ἀφόβως γένηται πρὸς ὑμᾶς· τὸ γὰρ ἔργον Κυρίου ἐργάζεται ὡς κἀγώ·
\vs{11}μή τις οὖν αὐτὸν ἐξουθενήσῃ. προπέμψατε δὲ αὐτὸν ἐν εἰρήνῃ, ἵνα ἔλθῃ πρός με· ἐκδέχομαι γὰρ αὐτὸν μετὰ τῶν ἀδελφῶν.
\vs{12}Περὶ δὲ Ἀπολλῶ τοῦ ἀδελφοῦ, πολλὰ παρεκάλεσα αὐτὸν, ἵνα ἔλθῃ πρὸς ὑμᾶς μετὰ τῶν ἀδελφῶν· καὶ πάντως οὐκ ἦν θέλημα ἵνα νῦν ἔλθῃ· ἐλεύσεται δὲ ὅταν εὐκαιρήσῃ.

\vs{13}Γρηγορεῖτε, στήκετε ἐν τῇ πίστει, ἀνδρίζεσθε, κραταιοῦσθε.
\vs{14}πάντα ὑμῶν ἐν ἀγάπῃ γινέσθω.

\vs{15}Παρακαλῶ δὲ ὑμᾶς, ἀδελφοί· οἴδατε τὴν οἰκίαν Στεφανᾶ, ὅτι ἐστὶν ἀπαρχὴ τῆς Ἀχαΐας καὶ εἰς διακονίαν τοῖς ἁγίοις ἔταξαν ἑαυτούς·
\vs{16}ἵνα καὶ ὑμεῖς ὑποτάσσησθε τοῖς τοιούτοις καὶ παντὶ τῷ συνεργοῦντι καὶ κοπιῶντι.
\vs{17}Χαίρω δὲ ἐπὶ τῇ παρουσίᾳ Στεφανᾶ καὶ Φορτουνάτου καὶ Ἀχαϊκοῦ, ὅτι τὸ ὑμέτερον ὑστέρημα οὗτοι ἀνεπλήρωσαν·
\vs{18}ἀνέπαυσαν γὰρ τὸ ἐμὸν πνεῦμα καὶ τὸ ὑμῶν. ἐπιγινώσκετε οὖν τοὺς τοιούτους.

\vs{19}Ἀσπάζονται ὑμᾶς αἱ ἐκκλησίαι τῆς Ἀσίας. Ἀσπάζεται ὑμᾶς ἐν Κυρίῳ πολλὰ Ἀκύλας καὶ Πρίσκα σὺν τῇ κατ᾽ οἶκον αὐτῶν ἐκκλησίᾳ.
\vs{20}Ἀσπάζονται ὑμᾶς οἱ ἀδελφοὶ πάντες. Ἀσπάσασθε ἀλλήλους ἐν φιλήματι ἁγίῳ.

\vs{21}Ὁ ἀσπασμὸς τῇ ἐμῇ χειρὶ Παύλου.
\vs{22}Εἴ τις οὐ φιλεῖ τὸν Κύριον, ἤτω ἀνάθεμα. Μαράνα θά.
\vs{23}Ἡ χάρις τοῦ Κυρίου Ἰησοῦ μεθ᾽ ὑμῶν.
\vs{24}Ἡ ἀγάπη μου μετὰ πάντων ὑμῶν ἐν Χριστῷ Ἰησοῦ.


\def\book{ΠΡΟΣ ΚΟΡΙΝΘΙΟΥΣ Β}
\biblebook{ΠΡΟΣ ΚΟΡΙΝΘΙΟΥΣ Β}


\lettrine[lines=2, loversize=0.2, nindent=0em, findent=.25em]{\textcolor{bookheadingcolor}{Π}}{αῦλος} ἀπόστολος Χριστοῦ Ἰησοῦ διὰ θελήματος Θεοῦ καὶ Τιμόθεος ὁ ἀδελφὸς Τῇ ἐκκλησίᾳ τοῦ Θεοῦ τῇ οὔσῃ ἐν Κορίνθῳ σὺν τοῖς ἁγίοις πᾶσιν τοῖς οὖσιν ἐν ὅλῃ τῇ Ἀχαΐᾳ,
\vs{2}Χάρις ὑμῖν καὶ εἰρήνη ἀπὸ Θεοῦ Πατρὸς ἡμῶν καὶ Κυρίου Ἰησοῦ Χριστοῦ.

\vs{3}Εὐλογητὸς ὁ Θεὸς καὶ Πατὴρ τοῦ Κυρίου ἡμῶν Ἰησοῦ Χριστοῦ, ὁ Πατὴρ τῶν οἰκτιρμῶν καὶ Θεὸς πάσης παρακλήσεως,
\vs{4}ὁ παρακαλῶν ἡμᾶς ἐπὶ πάσῃ τῇ θλίψει ἡμῶν εἰς τὸ δύνασθαι ἡμᾶς παρακαλεῖν τοὺς ἐν πάσῃ θλίψει διὰ τῆς παρακλήσεως ἧς παρακαλούμεθα αὐτοὶ ὑπὸ τοῦ Θεοῦ.
\vs{5}ὅτι καθὼς περισσεύει τὰ παθήματα τοῦ Χριστοῦ εἰς ἡμᾶς, οὕτως διὰ τοῦ Χριστοῦ περισσεύει καὶ ἡ παράκλησις ἡμῶν.
\vs{6}Εἴτε δὲ θλιβόμεθα, ὑπὲρ τῆς ὑμῶν παρακλήσεως καὶ σωτηρίας· εἴτε παρακαλούμεθα, ὑπὲρ τῆς ὑμῶν παρακλήσεως τῆς ἐνεργουμένης ἐν ὑπομονῇ τῶν αὐτῶν παθημάτων ὧν καὶ ἡμεῖς πάσχομεν.
\vs{7}καὶ ἡ ἐλπὶς ἡμῶν βεβαία ὑπὲρ ὑμῶν εἰδότες ὅτι ὡς κοινωνοί ἐστε τῶν παθημάτων, οὕτως καὶ τῆς παρακλήσεως.

\vs{8}Οὐ γὰρ θέλομεν ὑμᾶς ἀγνοεῖν, ἀδελφοί, ὑπὲρ τῆς θλίψεως ἡμῶν τῆς γενομένης ἐν τῇ Ἀσίᾳ, ὅτι καθ᾽ ὑπερβολὴν ὑπὲρ δύναμιν ἐβαρήθημεν ὥστε ἐξαπορηθῆναι ἡμᾶς καὶ τοῦ ζῆν·
\vs{9}ἀλλὰ αὐτοὶ ἐν ἑαυτοῖς τὸ ἀπόκριμα τοῦ θανάτου ἐσχήκαμεν, ἵνα μὴ πεποιθότες ὦμεν ἐφ᾽ ἑαυτοῖς ἀλλ᾽ ἐπὶ τῷ Θεῷ τῷ ἐγείροντι τοὺς νεκρούς·
\vs{10}ὃς ἐκ τηλικούτου θανάτου ἐρρύσατο ἡμᾶς καὶ ῥύσεται, εἰς ὃν ἠλπίκαμεν ὅτι καὶ ἔτι ῥύσεται,
\vs{11}συνυπουργούντων καὶ ὑμῶν ὑπὲρ ἡμῶν τῇ δεήσει, ἵνα ἐκ πολλῶν προσώπων τὸ εἰς ἡμᾶς χάρισμα διὰ πολλῶν εὐχαριστηθῇ ὑπὲρ ἡμῶν.

\vs{12}Ἡ γὰρ καύχησις ἡμῶν αὕτη ἐστίν, τὸ μαρτύριον τῆς συνειδήσεως ἡμῶν, ὅτι ἐν ἁγιότητι καὶ εἰλικρινείᾳ τοῦ Θεοῦ, καὶ οὐκ ἐν σοφίᾳ σαρκικῇ ἀλλ᾽ ἐν χάριτι Θεοῦ, ἀνεστράφημεν ἐν τῷ κόσμῳ, περισσοτέρως δὲ πρὸς ὑμᾶς.
\vs{13}οὐ γὰρ ἄλλα γράφομεν ὑμῖν ἀλλ᾽ ἢ ἃ ἀναγινώσκετε ἢ καὶ ἐπιγινώσκετε· ἐλπίζω δὲ ὅτι ἕως τέλους ἐπιγνώσεσθε,
\vs{14}καθὼς καὶ ἐπέγνωτε ἡμᾶς ἀπὸ μέρους, ὅτι καύχημα ὑμῶν ἐσμεν καθάπερ καὶ ὑμεῖς ἡμῶν ἐν τῇ ἡμέρᾳ τοῦ Κυρίου ἡμῶν Ἰησοῦ.

\vs{15}Καὶ ταύτῃ τῇ πεποιθήσει ἐβουλόμην πρότερον πρὸς ὑμᾶς ἐλθεῖν, ἵνα δευτέραν χάριν σχῆτε,
\vs{16}καὶ δι᾽ ὑμῶν διελθεῖν εἰς Μακεδονίαν καὶ πάλιν ἀπὸ Μακεδονίας ἐλθεῖν πρὸς ὑμᾶς καὶ ὑφ᾽ ὑμῶν προπεμφθῆναι εἰς τὴν Ἰουδαίαν.
\vs{17}Τοῦτο οὖν βουλόμενος μήτι ἄρα τῇ ἐλαφρίᾳ ἐχρησάμην; ἢ ἃ βουλεύομαι κατὰ σάρκα βουλεύομαι, ἵνα ᾖ παρ᾽ ἐμοὶ τό Ναί ναὶ καὶ τὸ Οὔ οὔ;
\vs{18}πιστὸς δὲ ὁ Θεὸς ὅτι ὁ λόγος ἡμῶν ὁ πρὸς ὑμᾶς οὐκ ἔστιν Ναί καὶ Οὔ.
\vs{19}ὁ τοῦ Θεοῦ γὰρ Υἱὸς Ἰησοῦς Χριστὸς ὁ ἐν ὑμῖν δι᾽ ἡμῶν κηρυχθείς, δι᾽ ἐμοῦ καὶ Σιλουανοῦ καὶ Τιμοθέου, οὐκ ἐγένετο Ναί καὶ Οὔ ἀλλὰ Ναί ἐν αὐτῷ γέγονεν.
\vs{20}ὅσαι γὰρ ἐπαγγελίαι Θεοῦ, ἐν αὐτῷ τὸ Ναί· διὸ καὶ δι᾽ αὐτοῦ τὸ Ἀμὴν τῷ Θεῷ πρὸς δόξαν δι᾽ ἡμῶν.
\vs{21}Ὁ δὲ βεβαιῶν ἡμᾶς σὺν ὑμῖν εἰς Χριστὸν καὶ χρίσας ἡμᾶς Θεός,
\vs{22}ὁ καὶ σφραγισάμενος ἡμᾶς καὶ δοὺς τὸν ἀρραβῶνα τοῦ Πνεύματος ἐν ταῖς καρδίαις ἡμῶν.

\vs{23}Ἐγὼ δὲ μάρτυρα τὸν Θεὸν ἐπικαλοῦμαι ἐπὶ τὴν ἐμὴν ψυχήν, ὅτι φειδόμενος ὑμῶν οὐκέτι ἦλθον εἰς Κόρινθον.
\vs{24}οὐχ ὅτι κυριεύομεν ὑμῶν τῆς πίστεως ἀλλὰ συνεργοί ἐσμεν τῆς χαρᾶς ὑμῶν· τῇ γὰρ πίστει ἑστήκατε.
\inparch{2} \vs{1}Ἔκρινα γὰρ ἐμαυτῷ τοῦτο τὸ μὴ πάλιν ἐν λύπῃ πρὸς ὑμᾶς ἐλθεῖν.
\vs{2}εἰ γὰρ ἐγὼ λυπῶ ὑμᾶς, καὶ τίς ὁ εὐφραίνων με εἰ μὴ ὁ λυπούμενος ἐξ ἐμοῦ;
\vs{3}καὶ ἔγραψα τοῦτο αὐτὸ, ἵνα μὴ ἐλθὼν λύπην σχῶ ἀφ᾽ ὧν ἔδει με χαίρειν, πεποιθὼς ἐπὶ πάντας ὑμᾶς ὅτι ἡ ἐμὴ χαρὰ πάντων ὑμῶν ἐστιν.
\vs{4}ἐκ γὰρ πολλῆς θλίψεως καὶ συνοχῆς καρδίας ἔγραψα ὑμῖν διὰ πολλῶν δακρύων, οὐχ ἵνα λυπηθῆτε ἀλλὰ τὴν ἀγάπην ἵνα γνῶτε ἣν ἔχω περισσοτέρως εἰς ὑμᾶς.

\vs{5}Εἰ δέ τις λελύπηκεν, οὐκ ἐμὲ λελύπηκεν, ἀλλὰ ἀπὸ μέρους, ἵνα μὴ ἐπιβαρῶ, πάντας ὑμᾶς.
\vs{6}ἱκανὸν τῷ τοιούτῳ ἡ ἐπιτιμία αὕτη ἡ ὑπὸ τῶν πλειόνων,
\vs{7}ὥστε τοὐναντίον μᾶλλον ὑμᾶς χαρίσασθαι καὶ παρακαλέσαι, μή πως τῇ περισσοτέρᾳ λύπῃ καταποθῇ ὁ τοιοῦτος.
\vs{8}διὸ παρακαλῶ ὑμᾶς κυρῶσαι εἰς αὐτὸν ἀγάπην·
\vs{9}Εἰς τοῦτο γὰρ καὶ ἔγραψα, ἵνα γνῶ τὴν δοκιμὴν ὑμῶν, εἰ εἰς πάντα ὑπήκοοί ἐστε.
\vs{10}ᾧ δέ τι χαρίζεσθε, κἀγώ· καὶ γὰρ ἐγὼ ὃ κεχάρισμαι, εἴ τι κεχάρισμαι, δι᾽ ὑμᾶς ἐν προσώπῳ Χριστοῦ,
\vs{11}ἵνα μὴ πλεονεκτηθῶμεν ὑπὸ τοῦ Σατανᾶ· οὐ γὰρ αὐτοῦ τὰ νοήματα ἀγνοοῦμεν.

\vs{12}Ἐλθὼν δὲ εἰς τὴν Τρῳάδα εἰς τὸ εὐαγγέλιον τοῦ Χριστοῦ καὶ θύρας μοι ἀνεῳγμένης ἐν Κυρίῳ,
\vs{13}οὐκ ἔσχηκα ἄνεσιν τῷ πνεύματί μου τῷ μὴ εὑρεῖν με Τίτον τὸν ἀδελφόν μου, ἀλλὰ ἀποταξάμενος αὐτοῖς ἐξῆλθον εἰς Μακεδονίαν.

\vs{14}Τῷ δὲ Θεῷ χάρις τῷ πάντοτε θριαμβεύοντι ἡμᾶς ἐν τῷ Χριστῷ καὶ τὴν ὀσμὴν τῆς γνώσεως αὐτοῦ φανεροῦντι δι᾽ ἡμῶν ἐν παντὶ τόπῳ·
\vs{15}ὅτι Χριστοῦ εὐωδία ἐσμὲν τῷ Θεῷ ἐν τοῖς σωζομένοις καὶ ἐν τοῖς ἀπολλυμένοις,
\vs{16}οἷς μὲν ὀσμὴ ἐκ θανάτου εἰς θάνατον, οἷς δὲ ὀσμὴ ἐκ ζωῆς εἰς ζωήν. καὶ πρὸς ταῦτα τίς ἱκανός;
\vs{17}Οὐ γάρ ἐσμεν ὡς οἱ πολλοὶ καπηλεύοντες τὸν λόγον τοῦ Θεοῦ, ἀλλ᾽ ὡς ἐξ εἰλικρινείας, ἀλλ᾽ ὡς ἐκ Θεοῦ κατέναντι Θεοῦ ἐν Χριστῷ λαλοῦμεν.

\ch{3}
Ἀρχόμεθα πάλιν ἑαυτοὺς συνιστάνειν; ἢ μὴ χρῄζομεν ὥς τινες συστατικῶν ἐπιστολῶν πρὸς ὑμᾶς ἢ ἐξ ὑμῶν;
\vs{2}ἡ ἐπιστολὴ ἡμῶν ὑμεῖς ἐστε, ἐνγεγραμμένη ἐν ταῖς καρδίαις ἡμῶν, γινωσκομένη καὶ ἀναγινωσκομένη ὑπὸ πάντων ἀνθρώπων,
\vs{3}φανερούμενοι ὅτι ἐστὲ ἐπιστολὴ Χριστοῦ διακονηθεῖσα ὑφ᾽ ἡμῶν, ἐνγεγραμμένη οὐ μέλανι ἀλλὰ Πνεύματι Θεοῦ ζῶντος, οὐκ ἐν πλαξὶν λιθίναις ἀλλ᾽ ἐν πλαξὶν καρδίαις σαρκίναις.

\vs{4}Πεποίθησιν δὲ τοιαύτην ἔχομεν διὰ τοῦ Χριστοῦ πρὸς τὸν Θεόν.
\vs{5}οὐχ ὅτι ἀφ᾽ ἑαυτῶν ἱκανοί ἐσμεν λογίσασθαί τι ὡς ἐξ ἑαυτῶν, ἀλλ᾽ ἡ ἱκανότης ἡμῶν ἐκ τοῦ Θεοῦ,
\vs{6}ὃς καὶ ἱκάνωσεν ἡμᾶς διακόνους καινῆς διαθήκης, οὐ γράμματος ἀλλὰ πνεύματος· τὸ γὰρ γράμμα ἀποκτέννει, τὸ δὲ πνεῦμα ζωοποιεῖ.
\vs{7}Εἰ δὲ ἡ διακονία τοῦ θανάτου ἐν γράμμασιν ἐντετυπωμένη λίθοις ἐγενήθη ἐν δόξῃ, ὥστε μὴ δύνασθαι ἀτενίσαι τοὺς υἱοὺς Ἰσραὴλ εἰς τὸ πρόσωπον Μωϋσέως διὰ τὴν δόξαν τοῦ προσώπου αὐτοῦ τὴν καταργουμένην,
\vs{8}πῶς οὐχὶ μᾶλλον ἡ διακονία τοῦ πνεύματος ἔσται ἐν δόξῃ;
\vs{9}εἰ γὰρ τῇ διακονία τῆς κατακρίσεως δόξα, πολλῷ μᾶλλον περισσεύει ἡ διακονία τῆς δικαιοσύνης δόξῃ.
\vs{10}καὶ γὰρ οὐ δεδόξασται τὸ δεδοξασμένον ἐν τούτῳ τῷ μέρει εἵνεκεν τῆς ὑπερβαλλούσης δόξης.
\vs{11}εἰ γὰρ τὸ καταργούμενον διὰ δόξης, πολλῷ μᾶλλον τὸ μένον ἐν δόξῃ.

\vs{12}Ἔχοντες οὖν τοιαύτην ἐλπίδα πολλῇ παρρησίᾳ χρώμεθα
\vs{13}καὶ οὐ καθάπερ Μωϋσῆς ἐτίθει κάλυμμα ἐπὶ τὸ πρόσωπον αὐτοῦ πρὸς τὸ μὴ ἀτενίσαι τοὺς υἱοὺς Ἰσραὴλ εἰς τὸ τέλος τοῦ καταργουμένου.
\vs{14}Ἀλλὰ ἐπωρώθη τὰ νοήματα αὐτῶν. ἄχρι γὰρ τῆς σήμερον ἡμέρας τὸ αὐτὸ κάλυμμα ἐπὶ τῇ ἀναγνώσει τῆς παλαιᾶς διαθήκης μένει, μὴ ἀνακαλυπτόμενον ὅτι ἐν Χριστῷ καταργεῖται·
\vs{15}ἀλλ᾽ ἕως σήμερον ἡνίκα ἂν ἀναγινώσκηται Μωϋσῆς, κάλυμμα ἐπὶ τὴν καρδίαν αὐτῶν κεῖται·
\vs{16}ἡνίκα δὲ ἐὰν ἐπιστρέψῃ πρὸς Κύριον, περιαιρεῖται τὸ κάλυμμα.
\vs{17}Ὁ δὲ Κύριος τὸ Πνεῦμά ἐστιν· οὗ δὲ τὸ Πνεῦμα Κυρίου, ἐλευθερία.
\vs{18}ἡμεῖς δὲ πάντες ἀνακεκαλυμμένῳ προσώπῳ τὴν δόξαν Κυρίου κατοπτριζόμενοι τὴν αὐτὴν εἰκόνα μεταμορφούμεθα ἀπὸ δόξης εἰς δόξαν καθάπερ ἀπὸ Κυρίου Πνεύματος.

\ch{4}
Διὰ τοῦτο, ἔχοντες τὴν διακονίαν ταύτην καθὼς ἠλεήθημεν, οὐκ ἐγκακοῦμεν
\vs{2}ἀλλὰ ἀπειπάμεθα τὰ κρυπτὰ τῆς αἰσχύνης, μὴ περιπατοῦντες ἐν πανουργίᾳ μηδὲ δολοῦντες τὸν λόγον τοῦ Θεοῦ ἀλλὰ τῇ φανερώσει τῆς ἀληθείας συνιστάνοντες ἑαυτοὺς πρὸς πᾶσαν συνείδησιν ἀνθρώπων ἐνώπιον τοῦ Θεοῦ.
\vs{3}Εἰ δὲ καὶ ἔστιν κεκαλυμμένον τὸ εὐαγγέλιον ἡμῶν, ἐν τοῖς ἀπολλυμένοις ἐστὶν κεκαλυμμένον,
\vs{4}ἐν οἷς ὁ θεὸς τοῦ αἰῶνος τούτου ἐτύφλωσεν τὰ νοήματα τῶν ἀπίστων εἰς τὸ μὴ αὐγάσαι τὸν φωτισμὸν τοῦ εὐαγγελίου τῆς δόξης τοῦ Χριστοῦ, ὅς ἐστιν εἰκὼν τοῦ Θεοῦ.
\vs{5}Οὐ γὰρ ἑαυτοὺς κηρύσσομεν ἀλλὰ Ἰησοῦν Χριστὸν Κύριον, ἑαυτοὺς δὲ δούλους ὑμῶν διὰ Ἰησοῦν.
\vs{6}ὅτι ὁ Θεὸς ὁ εἰπών· Ἐκ σκότους φῶς λάμψει, ὃς ἔλαμψεν ἐν ταῖς καρδίαις ἡμῶν πρὸς φωτισμὸν τῆς γνώσεως τῆς δόξης τοῦ Θεοῦ ἐν προσώπῳ Ἰησοῦ Χριστοῦ.

\vs{7}Ἔχομεν δὲ τὸν θησαυρὸν τοῦτον ἐν ὀστρακίνοις σκεύεσιν, ἵνα ἡ ὑπερβολὴ τῆς δυνάμεως ᾖ τοῦ Θεοῦ καὶ μὴ ἐξ ἡμῶν·
\vs{8}ἐν παντὶ θλιβόμενοι ἀλλ᾽ οὐ στενοχωρούμενοι, ἀπορούμενοι ἀλλ᾽ οὐκ ἐξαπορούμενοι,
\vs{9}διωκόμενοι ἀλλ᾽ οὐκ ἐγκαταλειπόμενοι, καταβαλλόμενοι ἀλλ᾽ οὐκ ἀπολλύμενοι,
\vs{10}πάντοτε τὴν νέκρωσιν τοῦ Ἰησοῦ ἐν τῷ σώματι περιφέροντες, ἵνα καὶ ἡ ζωὴ τοῦ Ἰησοῦ ἐν τῷ σώματι ἡμῶν φανερωθῇ.
\vs{11}ἀεὶ γὰρ ἡμεῖς οἱ ζῶντες εἰς θάνατον παραδιδόμεθα διὰ Ἰησοῦν, ἵνα καὶ ἡ ζωὴ τοῦ Ἰησοῦ φανερωθῇ ἐν τῇ θνητῇ σαρκὶ ἡμῶν.
\vs{12}ὥστε ὁ θάνατος ἐν ἡμῖν ἐνεργεῖται, ἡ δὲ ζωὴ ἐν ὑμῖν.
\vs{13}Ἔχοντες δὲ τὸ αὐτὸ πνεῦμα τῆς πίστεως κατὰ τὸ γεγραμμένον· Ἐπίστευσα, διὸ ἐλάλησα, καὶ ἡμεῖς πιστεύομεν, διὸ καὶ λαλοῦμεν,
\vs{14}εἰδότες ὅτι ὁ ἐγείρας τὸν Κύριον Ἰησοῦν καὶ ἡμᾶς σὺν Ἰησοῦ ἐγερεῖ καὶ παραστήσει σὺν ὑμῖν.
\vs{15}τὰ γὰρ πάντα δι᾽ ὑμᾶς, ἵνα ἡ χάρις πλεονάσασα διὰ τῶν πλειόνων τὴν εὐχαριστίαν περισσεύσῃ εἰς τὴν δόξαν τοῦ Θεοῦ.

\vs{16}Διὸ οὐκ ἐγκακοῦμεν, ἀλλ᾽ εἰ καὶ ὁ ἔξω ἡμῶν ἄνθρωπος διαφθείρεται, ἀλλ᾽ ὁ ἔσω ἡμῶν ἀνακαινοῦται ἡμέρᾳ καὶ ἡμέρᾳ.
\vs{17}τὸ γὰρ παραυτίκα ἐλαφρὸν τῆς θλίψεως ἡμῶν καθ᾽ ὑπερβολὴν εἰς ὑπερβολὴν αἰώνιον βάρος δόξης κατεργάζεται ἡμῖν,
\vs{18}μὴ σκοπούντων ἡμῶν τὰ βλεπόμενα ἀλλὰ τὰ μὴ βλεπόμενα· τὰ γὰρ βλεπόμενα πρόσκαιρα, τὰ δὲ μὴ βλεπόμενα αἰώνια.

\ch{5}
Οἴδαμεν γὰρ ὅτι ἐὰν ἡ ἐπίγειος ἡμῶν οἰκία τοῦ σκήνους καταλυθῇ, οἰκοδομὴν ἐκ Θεοῦ ἔχομεν, οἰκίαν ἀχειροποίητον αἰώνιον ἐν τοῖς οὐρανοῖς.
\vs{2}καὶ γὰρ ἐν τούτῳ στενάζομεν τὸ οἰκητήριον ἡμῶν τὸ ἐξ οὐρανοῦ ἐπενδύσασθαι ἐπιποθοῦντες,
\vs{3}εἴ γε καὶ ἐνδυσάμενοι οὐ γυμνοὶ εὑρεθησόμεθα.
\vs{4}καὶ γὰρ οἱ ὄντες ἐν τῷ σκήνει στενάζομεν βαρούμενοι, ἐφ᾽ ᾧ οὐ θέλομεν ἐκδύσασθαι ἀλλ᾽ ἐπενδύσασθαι, ἵνα καταποθῇ τὸ θνητὸν ὑπὸ τῆς ζωῆς.
\vs{5}ὁ δὲ κατεργασάμενος ἡμᾶς εἰς αὐτὸ τοῦτο Θεός, ὁ δοὺς ἡμῖν τὸν ἀρραβῶνα τοῦ Πνεύματος.
\vs{6}Θαρροῦντες οὖν πάντοτε καὶ εἰδότες ὅτι ἐνδημοῦντες ἐν τῷ σώματι ἐκδημοῦμεν ἀπὸ τοῦ Κυρίου·
\vs{7}διὰ πίστεως γὰρ περιπατοῦμεν, οὐ διὰ εἴδους·
\vs{8}Θαρροῦμεν δὲ καὶ εὐδοκοῦμεν μᾶλλον ἐκδημῆσαι ἐκ τοῦ σώματος καὶ ἐνδημῆσαι πρὸς τὸν Κύριον.
\vs{9}διὸ καὶ φιλοτιμούμεθα, εἴτε ἐνδημοῦντες εἴτε ἐκδημοῦντες, εὐάρεστοι αὐτῷ εἶναι.
\vs{10}τοὺς γὰρ πάντας ἡμᾶς φανερωθῆναι δεῖ ἔμπροσθεν τοῦ βήματος τοῦ Χριστοῦ, ἵνα κομίσηται ἕκαστος τὰ διὰ τοῦ σώματος πρὸς ἃ ἔπραξεν, εἴτε ἀγαθὸν εἴτε φαῦλον.

\vs{11}Εἰδότες οὖν τὸν φόβον τοῦ Κυρίου ἀνθρώπους πείθομεν, Θεῷ δὲ πεφανερώμεθα· ἐλπίζω δὲ καὶ ἐν ταῖς συνειδήσεσιν ὑμῶν πεφανερῶσθαι.
\vs{12}οὐ πάλιν ἑαυτοὺς συνιστάνομεν ὑμῖν ἀλλὰ ἀφορμὴν διδόντες ὑμῖν καυχήματος ὑπὲρ ἡμῶν, ἵνα ἔχητε πρὸς τοὺς ἐν προσώπῳ καυχωμένους καὶ μὴ ἐν καρδίᾳ.
\vs{13}Εἴτε γὰρ ἐξέστημεν, Θεῷ· εἴτε σωφρονοῦμεν, ὑμῖν.
\vs{14}ἡ γὰρ ἀγάπη τοῦ Χριστοῦ συνέχει ἡμᾶς, κρίναντας τοῦτο, ὅτι εἷς ὑπὲρ πάντων ἀπέθανεν, ἄρα οἱ πάντες ἀπέθανον·
\vs{15}καὶ ὑπὲρ πάντων ἀπέθανεν, ἵνα οἱ ζῶντες μηκέτι ἑαυτοῖς ζῶσιν ἀλλὰ τῷ ὑπὲρ αὐτῶν ἀποθανόντι καὶ ἐγερθέντι.
\vs{16}Ὥστε ἡμεῖς ἀπὸ τοῦ νῦν οὐδένα οἴδαμεν κατὰ σάρκα· εἰ καὶ ἐγνώκαμεν κατὰ σάρκα Χριστόν, ἀλλὰ νῦν οὐκέτι γινώσκομεν.
\vs{17}ὥστε εἴ τις ἐν Χριστῷ, καινὴ κτίσις· τὰ ἀρχαῖα παρῆλθεν, ἰδοὺ γέγονεν καινά.
\vs{18}Τὰ δὲ πάντα ἐκ τοῦ Θεοῦ τοῦ καταλλάξαντος ἡμᾶς ἑαυτῷ διὰ Χριστοῦ καὶ δόντος ἡμῖν τὴν διακονίαν τῆς καταλλαγῆς,
\vs{19}ὡς ὅτι Θεὸς ἦν ἐν Χριστῷ κόσμον καταλλάσσων ἑαυτῷ, μὴ λογιζόμενος αὐτοῖς τὰ παραπτώματα αὐτῶν καὶ θέμενος ἐν ἡμῖν τὸν λόγον τῆς καταλλαγῆς.
\vs{20}Ὑπὲρ Χριστοῦ οὖν πρεσβεύομεν ὡς τοῦ Θεοῦ παρακαλοῦντος δι᾽ ἡμῶν· δεόμεθα ὑπὲρ Χριστοῦ, καταλλάγητε τῷ Θεῷ.
\vs{21}τὸν μὴ γνόντα ἁμαρτίαν ὑπὲρ ἡμῶν ἁμαρτίαν ἐποίησεν, ἵνα ἡμεῖς γενώμεθα δικαιοσύνη Θεοῦ ἐν αὐτῷ.

\ch{6}
Συνεργοῦντες δὲ καὶ παρακαλοῦμεν μὴ εἰς κενὸν τὴν χάριν τοῦ Θεοῦ δέξασθαι ὑμᾶς·
\vs{2}λέγει γάρ· 
\begin{poetryblock}

\begin{quote}Καιρῷ δεκτῷ ἐπήκουσά σου\end{quote} 

\begin{quote}καὶ ἐν ἡμέρᾳ σωτηρίας ἐβοήθησά σοι.\end{quote}
\end{poetryblock}

Ἰδοὺ νῦν καιρὸς εὐπρόσδεκτος, ἰδοὺ νῦν ἡμέρα σωτηρίας.

\vs{3}Μηδεμίαν ἐν μηδενὶ διδόντες προσκοπήν, ἵνα μὴ μωμηθῇ ἡ διακονία,
\vs{4}ἀλλ᾽ ἐν παντὶ συνιστάντες ἑαυτοὺς ὡς Θεοῦ διάκονοι, ἐν ὑπομονῇ πολλῇ, ἐν θλίψεσιν, ἐν ἀνάγκαις, ἐν στενοχωρίαις,
\vs{5}ἐν πληγαῖς, ἐν φυλακαῖς, ἐν ἀκαταστασίαις, ἐν κόποις, ἐν ἀγρυπνίαις, ἐν νηστείαις,
\vs{6}ἐν ἁγνότητι, ἐν γνώσει, ἐν μακροθυμίᾳ, ἐν χρηστότητι, ἐν Πνεύματι Ἁγίῳ, ἐν ἀγάπῃ ἀνυποκρίτῳ,
\vs{7}ἐν λόγῳ ἀληθείας, ἐν δυνάμει Θεοῦ· διὰ τῶν ὅπλων τῆς δικαιοσύνης τῶν δεξιῶν καὶ ἀριστερῶν,
\vs{8}διὰ δόξης καὶ ἀτιμίας, διὰ δυσφημίας καὶ εὐφημίας· ὡς πλάνοι καὶ ἀληθεῖς,
\vs{9}ὡς ἀγνοούμενοι καὶ ἐπιγινωσκόμενοι, ὡς ἀποθνήσκοντες καὶ ἰδοὺ ζῶμεν, ὡς παιδευόμενοι καὶ μὴ θανατούμενοι,
\vs{10}ὡς λυπούμενοι ἀεὶ δὲ χαίροντες, ὡς πτωχοὶ πολλοὺς δὲ πλουτίζοντες, ὡς μηδὲν ἔχοντες καὶ πάντα κατέχοντες.

\vs{11}Τὸ στόμα ἡμῶν ἀνέῳγεν πρὸς ὑμᾶς, Κορίνθιοι, ἡ καρδία ἡμῶν πεπλάτυνται·
\vs{12}οὐ στενοχωρεῖσθε ἐν ἡμῖν, στενοχωρεῖσθε δὲ ἐν τοῖς σπλάγχνοις ὑμῶν·
\vs{13}τὴν δὲ αὐτὴν ἀντιμισθίαν, ὡς τέκνοις λέγω, πλατύνθητε καὶ ὑμεῖς.

\vs{14}Μὴ γίνεσθε ἑτεροζυγοῦντες ἀπίστοις· τίς γὰρ μετοχὴ δικαιοσύνῃ καὶ ἀνομίᾳ, ἢ τίς κοινωνία φωτὶ πρὸς σκότος;
\vs{15}τίς δὲ συμφώνησις Χριστοῦ πρὸς Βελιάρ, ἢ τίς μερὶς πιστῷ μετὰ ἀπίστου;
\vs{16}τίς δὲ συνκατάθεσις ναῷ Θεοῦ μετὰ εἰδώλων; ἡμεῖς γὰρ ναὸς Θεοῦ ἐσμεν ζῶντος, καθὼς εἶπεν ὁ Θεὸς ὅτι 
\begin{poetryblock}

\begin{quote}Ἐνοικήσω ἐν αὐτοῖς καὶ ἐμπεριπατήσω\end{quote} 

\begin{quote}καὶ ἔσομαι αὐτῶν Θεός καὶ αὐτοὶ ἔσονταί μου λαός.\end{quote}

\begin{quote} \vs{17}Διὸ ἐξέλθατε ἐκ μέσου αὐτῶν\end{quote} 

\begin{quote}καὶ ἀφορίσθητε, λέγει Κύριος,\end{quote} 

\begin{quote}καὶ ἀκαθάρτου μὴ ἅπτεσθε·\end{quote} 

\begin{quote}κἀγὼ εἰσδέξομαι ὑμᾶς\end{quote}

\begin{quote} \vs{18}Καὶ Ἔσομαι ὑμῖν εἰς Πατέρα\end{quote} 

\begin{quote}καὶ ὑμεῖς ἔσεσθέ μοι εἰς υἱοὺς καὶ θυγατέρας,\end{quote} 

\begin{quote}λέγει Κύριος Παντοκράτωρ.\end{quote}
\end{poetryblock}

\ch{7}
Ταύτας οὖν ἔχοντες τὰς ἐπαγγελίας, ἀγαπητοί, καθαρίσωμεν ἑαυτοὺς ἀπὸ παντὸς μολυσμοῦ σαρκὸς καὶ πνεύματος, ἐπιτελοῦντες ἁγιωσύνην ἐν φόβῳ Θεοῦ.

\vs{2}Χωρήσατε ἡμᾶς· οὐδένα ἠδικήσαμεν, οὐδένα ἐφθείραμεν, οὐδένα ἐπλεονεκτήσαμεν.
\vs{3}πρὸς κατάκρισιν οὐ λέγω· προείρηκα γὰρ ὅτι ἐν ταῖς καρδίαις ἡμῶν ἐστε εἰς τὸ συναποθανεῖν καὶ συζῆν.
\vs{4}πολλή μοι παρρησία πρὸς ὑμᾶς, πολλή μοι καύχησις ὑπὲρ ὑμῶν· πεπλήρωμαι τῇ παρακλήσει, ὑπερπερισσεύομαι τῇ χαρᾷ ἐπὶ πάσῃ τῇ θλίψει ἡμῶν.

\vs{5}Καὶ γὰρ ἐλθόντων ἡμῶν εἰς Μακεδονίαν οὐδεμίαν ἔσχηκεν ἄνεσιν ἡ σὰρξ ἡμῶν ἀλλ᾽ ἐν παντὶ θλιβόμενοι· ἔξωθεν μάχαι, ἔσωθεν φόβοι.
\vs{6}ἀλλ᾽ ὁ παρακαλῶν τοὺς ταπεινοὺς παρεκάλεσεν ἡμᾶς ὁ Θεὸς ἐν τῇ παρουσίᾳ Τίτου,
\vs{7}οὐ μόνον δὲ ἐν τῇ παρουσίᾳ αὐτοῦ ἀλλὰ καὶ ἐν τῇ παρακλήσει ᾗ παρεκλήθη ἐφ᾽ ὑμῖν, ἀναγγέλλων ἡμῖν τὴν ὑμῶν ἐπιπόθησιν, τὸν ὑμῶν ὀδυρμόν, τὸν ὑμῶν ζῆλον ὑπὲρ ἐμοῦ ὥστε με μᾶλλον χαρῆναι.
\vs{8}Ὅτι εἰ καὶ ἐλύπησα ὑμᾶς ἐν τῇ ἐπιστολῇ, οὐ μεταμέλομαι· εἰ καὶ μετεμελόμην, βλέπω γὰρ ὅτι ἡ ἐπιστολὴ ἐκείνη εἰ καὶ πρὸς ὥραν ἐλύπησεν ὑμᾶς,
\vs{9}νῦν χαίρω, οὐχ ὅτι ἐλυπήθητε ἀλλ᾽ ὅτι ἐλυπήθητε εἰς μετάνοιαν· ἐλυπήθητε γὰρ κατὰ Θεόν, ἵνα ἐν μηδενὶ ζημιωθῆτε ἐξ ἡμῶν.
\vs{10}ἡ γὰρ κατὰ Θεὸν λύπη μετάνοιαν εἰς σωτηρίαν ἀμεταμέλητον ἐργάζεται· ἡ δὲ τοῦ κόσμου λύπη θάνατον κατεργάζεται.
\vs{11}Ἰδοὺ γὰρ αὐτὸ τοῦτο τὸ κατὰ Θεὸν λυπηθῆναι πόσην κατειργάσατο ὑμῖν σπουδήν, ἀλλὰ ἀπολογίαν, ἀλλὰ ἀγανάκτησιν, ἀλλὰ φόβον, ἀλλὰ ἐπιπόθησιν, ἀλλὰ ζῆλον, ἀλλὰ ἐκδίκησιν. ἐν παντὶ συνεστήσατε ἑαυτοὺς ἁγνοὺς εἶναι τῷ πράγματι.
\vs{12}ἄρα εἰ καὶ ἔγραψα ὑμῖν, οὐχ ἕνεκεν τοῦ ἀδικήσαντος οὐδὲ ἕνεκεν τοῦ ἀδικηθέντος ἀλλ᾽ ἕνεκεν τοῦ φανερωθῆναι τὴν σπουδὴν ὑμῶν τὴν ὑπὲρ ἡμῶν πρὸς ὑμᾶς ἐνώπιον τοῦ Θεοῦ.
\vs{13}διὰ τοῦτο παρακεκλήμεθα. Ἐπὶ δὲ τῇ παρακλήσει ἡμῶν περισσοτέρως μᾶλλον ἐχάρημεν ἐπὶ τῇ χαρᾷ Τίτου, ὅτι ἀναπέπαυται τὸ πνεῦμα αὐτοῦ ἀπὸ πάντων ὑμῶν·
\vs{14}ὅτι εἴ τι αὐτῷ ὑπὲρ ὑμῶν κεκαύχημαι, οὐ κατῃσχύνθην, ἀλλ᾽ ὡς πάντα ἐν ἀληθείᾳ ἐλαλήσαμεν ὑμῖν, οὕτως καὶ ἡ καύχησις ἡμῶν ἡ ἐπὶ Τίτου ἀλήθεια ἐγενήθη.
\vs{15}καὶ τὰ σπλάγχνα αὐτοῦ περισσοτέρως εἰς ὑμᾶς ἐστιν ἀναμιμνῃσκομένου τὴν πάντων ὑμῶν ὑπακοήν, ὡς μετὰ φόβου καὶ τρόμου ἐδέξασθε αὐτόν.
\vs{16}χαίρω ὅτι ἐν παντὶ θαρρῶ ἐν ὑμῖν.

\ch{8}
Γνωρίζομεν δὲ ὑμῖν, ἀδελφοί, τὴν χάριν τοῦ Θεοῦ τὴν δεδομένην ἐν ταῖς ἐκκλησίαις τῆς Μακεδονίας,
\vs{2}ὅτι ἐν πολλῇ δοκιμῇ θλίψεως ἡ περισσεία τῆς χαρᾶς αὐτῶν καὶ ἡ κατὰ βάθους πτωχεία αὐτῶν ἐπερίσσευσεν εἰς τὸ πλοῦτος τῆς ἁπλότητος αὐτῶν·
\vs{3}ὅτι κατὰ δύναμιν, μαρτυρῶ, καὶ παρὰ δύναμιν, αὐθαίρετοι
\vs{4}μετὰ πολλῆς παρακλήσεως δεόμενοι ἡμῶν τὴν χάριν καὶ τὴν κοινωνίαν τῆς διακονίας τῆς εἰς τοὺς ἁγίους,
\vs{5}καὶ οὐ καθὼς ἠλπίσαμεν ἀλλ᾽ ἑαυτοὺς ἔδωκαν πρῶτον τῷ Κυρίῳ καὶ ἡμῖν διὰ θελήματος Θεοῦ
\vs{6}Εἰς τὸ παρακαλέσαι ἡμᾶς Τίτον, ἵνα καθὼς προενήρξατο οὕτως καὶ ἐπιτελέσῃ εἰς ὑμᾶς καὶ τὴν χάριν ταύτην.
\vs{7}ἀλλ᾽ ὥσπερ ἐν παντὶ περισσεύετε, πίστει καὶ λόγῳ καὶ γνώσει καὶ πάσῃ σπουδῇ καὶ τῇ ἐξ ἡμῶν ἐν ὑμῖν ἀγάπῃ, ἵνα καὶ ἐν ταύτῃ τῇ χάριτι περισσεύητε.
\vs{8}Οὐ κατ᾽ ἐπιταγὴν λέγω ἀλλὰ διὰ τῆς ἑτέρων σπουδῆς καὶ τὸ τῆς ὑμετέρας ἀγάπης γνήσιον δοκιμάζων·
\vs{9}Γινώσκετε γὰρ τὴν χάριν τοῦ Κυρίου ἡμῶν Ἰησοῦ Χριστοῦ, ὅτι δι᾽ ὑμᾶς ἐπτώχευσεν πλούσιος ὤν, ἵνα ὑμεῖς τῇ ἐκείνου πτωχείᾳ πλουτήσητε.
\vs{10}καὶ γνώμην ἐν τούτῳ δίδωμι· τοῦτο γὰρ ὑμῖν συμφέρει, οἵτινες οὐ μόνον τὸ ποιῆσαι ἀλλὰ καὶ τὸ θέλειν προενήρξασθε ἀπὸ πέρυσι·
\vs{11}νυνὶ δὲ καὶ τὸ ποιῆσαι ἐπιτελέσατε, ὅπως καθάπερ ἡ προθυμία τοῦ θέλειν, οὕτως καὶ τὸ ἐπιτελέσαι ἐκ τοῦ ἔχειν.
\vs{12}εἰ γὰρ ἡ προθυμία πρόκειται, καθὸ ἐὰν ἔχῃ εὐπρόσδεκτος, οὐ καθὸ οὐκ ἔχει.
\vs{13}Οὐ γὰρ ἵνα ἄλλοις ἄνεσις, ὑμῖν θλῖψις, ἀλλ᾽ ἐξ ἰσότητος·
\vs{14}ἐν τῷ νῦν καιρῷ τὸ ὑμῶν περίσσευμα εἰς τὸ ἐκείνων ὑστέρημα, ἵνα καὶ τὸ ἐκείνων περίσσευμα γένηται εἰς τὸ ὑμῶν ὑστέρημα, ὅπως γένηται ἰσότης,
\vs{15}καθὼς γέγραπται· Ὁ τὸ πολὺ οὐκ ἐπλεόνασεν, καὶ ὁ τὸ ὀλίγον οὐκ ἠλαττόνησεν.

\vs{16}Χάρις δὲ τῷ Θεῷ τῷ διδόντι τὴν αὐτὴν σπουδὴν ὑπὲρ ὑμῶν ἐν τῇ καρδίᾳ Τίτου,
\vs{17}ὅτι τὴν μὲν παράκλησιν ἐδέξατο, σπουδαιότερος δὲ ὑπάρχων αὐθαίρετος ἐξῆλθεν πρὸς ὑμᾶς.
\vs{18}Συνεπέμψαμεν δὲ μετ᾽ αὐτοῦ τὸν ἀδελφὸν οὗ ὁ ἔπαινος ἐν τῷ εὐαγγελίῳ διὰ πασῶν τῶν ἐκκλησιῶν,
\vs{19}οὐ μόνον δὲ, ἀλλὰ καὶ χειροτονηθεὶς ὑπὸ τῶν ἐκκλησιῶν συνέκδημος ἡμῶν σὺν τῇ χάριτι ταύτῃ τῇ διακονουμένῃ ὑφ᾽ ἡμῶν πρὸς τὴν αὐτοῦ τοῦ Κυρίου δόξαν καὶ προθυμίαν ἡμῶν,
\vs{20}στελλόμενοι τοῦτο, μή τις ἡμᾶς μωμήσηται ἐν τῇ ἁδρότητι ταύτῃ τῇ διακονουμένῃ ὑφ᾽ ἡμῶν·
\vs{21}προνοοῦμεν γὰρ καλὰ οὐ μόνον ἐνώπιον Κυρίου ἀλλὰ καὶ ἐνώπιον ἀνθρώπων.
\vs{22}Συνεπέμψαμεν δὲ αὐτοῖς τὸν ἀδελφὸν ἡμῶν ὃν ἐδοκιμάσαμεν ἐν πολλοῖς πολλάκις σπουδαῖον ὄντα, νυνὶ δὲ πολὺ σπουδαιότερον πεποιθήσει πολλῇ τῇ εἰς ὑμᾶς.
\vs{23}εἴτε ὑπὲρ Τίτου, κοινωνὸς ἐμὸς καὶ εἰς ὑμᾶς συνεργός· εἴτε ἀδελφοὶ ἡμῶν, ἀπόστολοι ἐκκλησιῶν, δόξα Χριστοῦ.
\vs{24}τὴν οὖν ἔνδειξιν τῆς ἀγάπης ὑμῶν καὶ ἡμῶν καυχήσεως ὑπὲρ ὑμῶν εἰς αὐτοὺς ἐνδεικνύμενοι εἰς πρόσωπον τῶν ἐκκλησιῶν.

\ch{9}
Περὶ μὲν γὰρ τῆς διακονίας τῆς εἰς τοὺς ἁγίους περισσόν μοί ἐστιν τὸ γράφειν ὑμῖν·
\vs{2}οἶδα γὰρ τὴν προθυμίαν ὑμῶν ἣν ὑπὲρ ὑμῶν καυχῶμαι Μακεδόσιν, ὅτι Ἀχαΐα παρεσκεύασται ἀπὸ πέρυσι, καὶ τὸ ὑμῶν ζῆλος ἠρέθισεν τοὺς πλείονας.
\vs{3}Ἔπεμψα δὲ τοὺς ἀδελφούς, ἵνα μὴ τὸ καύχημα ἡμῶν τὸ ὑπὲρ ὑμῶν κενωθῇ ἐν τῷ μέρει τούτῳ, ἵνα καθὼς ἔλεγον παρεσκευασμένοι ἦτε,
\vs{4}μή πως ἐὰν ἔλθωσιν σὺν ἐμοὶ Μακεδόνες καὶ εὕρωσιν ὑμᾶς ἀπαρασκευάστους καταισχυνθῶμεν ἡμεῖς, ἵνα μὴ λέγωμεν ὑμεῖς, ἐν τῇ ὑποστάσει ταύτῃ.
\vs{5}ἀναγκαῖον οὖν ἡγησάμην παρακαλέσαι τοὺς ἀδελφοὺς, ἵνα προέλθωσιν εἰς ὑμᾶς καὶ προκαταρτίσωσιν τὴν προεπηγγελμένην εὐλογίαν ὑμῶν, ταύτην ἑτοίμην εἶναι οὕτως ὡς εὐλογίαν καὶ μὴ ὡς πλεονεξίαν.

\vs{6}Τοῦτο δέ, ὁ σπείρων φειδομένως φειδομένως καὶ θερίσει, καὶ ὁ σπείρων ἐπ᾽ εὐλογίαις ἐπ᾽ εὐλογίαις καὶ θερίσει.
\vs{7}ἕκαστος καθὼς προῄρηται τῇ καρδίᾳ, μὴ ἐκ λύπης ἢ ἐξ ἀνάγκης· ἱλαρὸν γὰρ δότην ἀγαπᾷ ὁ Θεός.
\vs{8}δυνατεῖ δὲ ὁ Θεὸς πᾶσαν χάριν περισσεῦσαι εἰς ὑμᾶς, ἵνα ἐν παντὶ πάντοτε πᾶσαν αὐτάρκειαν ἔχοντες περισσεύητε εἰς πᾶν ἔργον ἀγαθόν,
\vs{9}καθὼς γέγραπται· 
\begin{poetryblock}

\begin{quote}Ἐσκόρπισεν, ἔδωκεν τοῖς πένησιν,\end{quote} 

\begin{quote}ἡ δικαιοσύνη αὐτοῦ μένει εἰς τὸν αἰῶνα.\end{quote}
\end{poetryblock}
\vs{10}Ὁ δὲ ἐπιχορηγῶν σπόρον τῷ σπείροντι καὶ ἄρτον εἰς βρῶσιν χορηγήσει καὶ πληθυνεῖ τὸν σπόρον ὑμῶν καὶ αὐξήσει τὰ γενήματα τῆς δικαιοσύνης ὑμῶν.
\vs{11}ἐν παντὶ πλουτιζόμενοι εἰς πᾶσαν ἁπλότητα, ἥτις κατεργάζεται δι᾽ ἡμῶν εὐχαριστίαν τῷ Θεῷ·
\vs{12}ὅτι ἡ διακονία τῆς λειτουργίας ταύτης οὐ μόνον ἐστὶν προσαναπληροῦσα τὰ ὑστερήματα τῶν ἁγίων, ἀλλὰ καὶ περισσεύουσα διὰ πολλῶν εὐχαριστιῶν τῷ Θεῷ.
\vs{13}διὰ τῆς δοκιμῆς τῆς διακονίας ταύτης δοξάζοντες τὸν Θεὸν ἐπὶ τῇ ὑποταγῇ τῆς ὁμολογίας ὑμῶν εἰς τὸ εὐαγγέλιον τοῦ Χριστοῦ καὶ ἁπλότητι τῆς κοινωνίας εἰς αὐτοὺς καὶ εἰς πάντας,
\vs{14}καὶ αὐτῶν δεήσει ὑπὲρ ὑμῶν ἐπιποθούντων ὑμᾶς διὰ τὴν ὑπερβάλλουσαν χάριν τοῦ Θεοῦ ἐφ᾽ ὑμῖν.
\vs{15}Χάρις τῷ Θεῷ ἐπὶ τῇ ἀνεκδιηγήτῳ αὐτοῦ δωρεᾷ.

\ch{10}
Αὐτὸς δὲ ἐγὼ Παῦλος παρακαλῶ ὑμᾶς διὰ τῆς πραΰτητος καὶ ἐπιεικείας τοῦ Χριστοῦ, ὃς κατὰ πρόσωπον μὲν ταπεινὸς ἐν ὑμῖν, ἀπὼν δὲ θαρρῶ εἰς ὑμᾶς·
\vs{2}δέομαι δὲ τὸ μὴ παρὼν θαρρῆσαι τῇ πεποιθήσει ᾗ λογίζομαι τολμῆσαι ἐπί τινας τοὺς λογιζομένους ἡμᾶς ὡς κατὰ σάρκα περιπατοῦντας.
\vs{3}Ἐν σαρκὶ γὰρ περιπατοῦντες οὐ κατὰ σάρκα στρατευόμεθα,
\vs{4}τὰ γὰρ ὅπλα τῆς στρατείας ἡμῶν οὐ σαρκικὰ ἀλλὰ δυνατὰ τῷ Θεῷ πρὸς καθαίρεσιν ὀχυρωμάτων, λογισμοὺς καθαιροῦντες
\vs{5}καὶ πᾶν ὕψωμα ἐπαιρόμενον κατὰ τῆς γνώσεως τοῦ Θεοῦ, καὶ αἰχμαλωτίζοντες πᾶν νόημα εἰς τὴν ὑπακοὴν τοῦ Χριστοῦ,
\vs{6}καὶ ἐν ἑτοίμῳ ἔχοντες ἐκδικῆσαι πᾶσαν παρακοήν, ὅταν πληρωθῇ ὑμῶν ἡ ὑπακοή.

\vs{7}Τὰ κατὰ πρόσωπον βλέπετε. εἴ τις πέποιθεν ἑαυτῷ Χριστοῦ εἶναι, τοῦτο λογιζέσθω πάλιν ἐφ᾽ ἑαυτοῦ, ὅτι καθὼς αὐτὸς Χριστοῦ, οὕτως καὶ ἡμεῖς.
\vs{8}ἐάν τε γὰρ περισσότερόν τι καυχήσωμαι περὶ τῆς ἐξουσίας ἡμῶν ἧς ἔδωκεν ὁ Κύριος εἰς οἰκοδομὴν καὶ οὐκ εἰς καθαίρεσιν ὑμῶν, οὐκ αἰσχυνθήσομαι.
\vs{9}ἵνα μὴ δόξω ὡς ἂν ἐκφοβεῖν ὑμᾶς διὰ τῶν ἐπιστολῶν·
\vs{10}Ὅτι Αἱ ἐπιστολαὶ μέν, φησίν, Βαρεῖαι καὶ ἰσχυραί, ἡ δὲ παρουσία τοῦ σώματος ἀσθενὴς καὶ ὁ λόγος ἐξουθενημένος.
\vs{11}τοῦτο λογιζέσθω ὁ τοιοῦτος, ὅτι οἷοί ἐσμεν τῷ λόγῳ δι᾽ ἐπιστολῶν ἀπόντες, τοιοῦτοι καὶ παρόντες τῷ ἔργῳ.

\vs{12}Οὐ γὰρ τολμῶμεν ἐνκρῖναι ἢ συνκρῖναι ἑαυτούς τισιν τῶν ἑαυτοὺς συνιστανόντων, ἀλλὰ αὐτοὶ ἐν ἑαυτοῖς ἑαυτοὺς μετροῦντες καὶ συνκρίνοντες ἑαυτοὺς ἑαυτοῖς οὐ συνιᾶσιν.
\vs{13}ἡμεῖς δὲ οὐκ εἰς τὰ ἄμετρα καυχησόμεθα ἀλλὰ κατὰ τὸ μέτρον τοῦ κανόνος οὗ ἐμέρισεν ἡμῖν ὁ Θεὸς μέτρου, ἐφικέσθαι ἄχρι καὶ ὑμῶν.
\vs{14}οὐ γὰρ ὡς μὴ ἐφικνούμενοι εἰς ὑμᾶς ὑπερεκτείνομεν ἑαυτούς, ἄχρι γὰρ καὶ ὑμῶν ἐφθάσαμεν ἐν τῷ εὐαγγελίῳ τοῦ Χριστοῦ,
\vs{15}οὐκ εἰς τὰ ἄμετρα καυχώμενοι ἐν ἀλλοτρίοις κόποις, ἐλπίδα δὲ ἔχοντες αὐξανομένης τῆς πίστεως ὑμῶν ἐν ὑμῖν μεγαλυνθῆναι κατὰ τὸν κανόνα ἡμῶν εἰς περισσείαν
\vs{16}εἰς τὰ ὑπερέκεινα ὑμῶν εὐαγγελίσασθαι, οὐκ ἐν ἀλλοτρίῳ κανόνι εἰς τὰ ἕτοιμα καυχήσασθαι.
\vs{17}Ὁ δὲ καυχώμενος ἐν Κυρίῳ καυχάσθω·
\vs{18}οὐ γὰρ ὁ ἑαυτὸν συνιστάνων, ἐκεῖνός ἐστιν δόκιμος, ἀλλὰ ὃν ὁ Κύριος συνίστησιν.

\ch{11}
Ὄφελον ἀνείχεσθέ μου μικρόν τι ἀφροσύνης· ἀλλὰ καὶ ἀνέχεσθέ μου.
\vs{2}ζηλῶ γὰρ ὑμᾶς Θεοῦ ζήλῳ, ἡρμοσάμην γὰρ ὑμᾶς ἑνὶ ἀνδρὶ παρθένον ἁγνὴν παραστῆσαι τῷ Χριστῷ·
\vs{3}Φοβοῦμαι δὲ μή πως, ὡς ὁ ὄφις ἐξηπάτησεν Εὕαν ἐν τῇ πανουργίᾳ αὐτοῦ, φθαρῇ τὰ νοήματα ὑμῶν ἀπὸ τῆς ἁπλότητος καὶ τῆς ἁγνότητος τῆς εἰς τὸν Χριστόν.
\vs{4}εἰ μὲν γὰρ ὁ ἐρχόμενος ἄλλον Ἰησοῦν κηρύσσει ὃν οὐκ ἐκηρύξαμεν, ἢ πνεῦμα ἕτερον λαμβάνετε ὃ οὐκ ἐλάβετε, ἢ εὐαγγέλιον ἕτερον ὃ οὐκ ἐδέξασθε, καλῶς ἀνέχεσθε.

\vs{5}Λογίζομαι γὰρ μηδὲν ὑστερηκέναι τῶν Ὑπερλίαν ἀποστόλων.
\vs{6}εἰ δὲ καὶ ἰδιώτης τῷ λόγῳ, ἀλλ᾽ οὐ τῇ γνώσει, ἀλλ᾽ ἐν παντὶ φανερώσαντες ἐν πᾶσιν εἰς ὑμᾶς.
\vs{7}Ἢ ἁμαρτίαν ἐποίησα ἐμαυτὸν ταπεινῶν ἵνα ὑμεῖς ὑψωθῆτε, ὅτι δωρεὰν τὸ τοῦ Θεοῦ εὐαγγέλιον εὐηγγελισάμην ὑμῖν;
\vs{8}ἄλλας ἐκκλησίας ἐσύλησα λαβὼν ὀψώνιον πρὸς τὴν ὑμῶν διακονίαν,
\vs{9}καὶ παρὼν πρὸς ὑμᾶς καὶ ὑστερηθεὶς οὐ κατενάρκησα οὐθενός· τὸ γὰρ ὑστέρημά μου προσανεπλήρωσαν οἱ ἀδελφοὶ ἐλθόντες ἀπὸ Μακεδονίας, καὶ ἐν παντὶ ἀβαρῆ ἐμαυτὸν ὑμῖν ἐτήρησα καὶ τηρήσω.
\vs{10}ἔστιν ἀλήθεια Χριστοῦ ἐν ἐμοὶ ὅτι ἡ καύχησις αὕτη οὐ φραγήσεται εἰς ἐμὲ ἐν τοῖς κλίμασιν τῆς Ἀχαΐας.
\vs{11}διὰ τί; ὅτι οὐκ ἀγαπῶ ὑμᾶς; ὁ Θεὸς οἶδεν.

\vs{12}Ὃ δὲ ποιῶ, καὶ ποιήσω, ἵνα ἐκκόψω τὴν ἀφορμὴν τῶν θελόντων ἀφορμήν, ἵνα ἐν ᾧ καυχῶνται εὑρεθῶσιν καθὼς καὶ ἡμεῖς.
\vs{13}οἱ γὰρ τοιοῦτοι ψευδαπόστολοι, ἐργάται δόλιοι, μετασχηματιζόμενοι εἰς ἀποστόλους Χριστοῦ.
\vs{14}καὶ οὐ θαῦμα· αὐτὸς γὰρ ὁ Σατανᾶς μετασχηματίζεται εἰς ἄγγελον φωτός.
\vs{15}οὐ μέγα οὖν εἰ καὶ οἱ διάκονοι αὐτοῦ μετασχηματίζονται ὡς διάκονοι δικαιοσύνης· ὧν τὸ τέλος ἔσται κατὰ τὰ ἔργα αὐτῶν.

\vs{16}Πάλιν λέγω, μή τίς με δόξῃ ἄφρονα εἶναι· εἰ δὲ μή γε, κἂν ὡς ἄφρονα δέξασθέ με, ἵνα κἀγὼ μικρόν τι καυχήσωμαι.
\vs{17}ὃ λαλῶ, οὐ κατὰ Κύριον λαλῶ ἀλλ᾽ ὡς ἐν ἀφροσύνῃ, ἐν ταύτῃ τῇ ὑποστάσει τῆς καυχήσεως.
\vs{18}ἐπεὶ πολλοὶ καυχῶνται κατὰ σάρκα, κἀγὼ καυχήσομαι.
\vs{19}ἡδέως γὰρ ἀνέχεσθε τῶν ἀφρόνων φρόνιμοι ὄντες·
\vs{20}ἀνέχεσθε γὰρ εἴ τις ὑμᾶς καταδουλοῖ, εἴ τις κατεσθίει, εἴ τις λαμβάνει, εἴ τις ἐπαίρεται, εἴ τις εἰς πρόσωπον ὑμᾶς δέρει.
\vs{21}κατὰ ἀτιμίαν λέγω, ὡς ὅτι ἡμεῖς ἠσθενήκαμεν.

Ἐν ᾧ δ᾽ ἄν τις τολμᾷ, ἐν ἀφροσύνῃ λέγω, τολμῶ κἀγώ.
\vs{22}Ἑβραῖοί εἰσιν; κἀγώ. Ἰσραηλῖταί εἰσιν; κἀγώ. σπέρμα Ἀβραάμ εἰσιν; κἀγώ.
\vs{23}διάκονοι Χριστοῦ εἰσιν; παραφρονῶν λαλῶ, ὑπὲρ ἐγώ· ἐν κόποις περισσοτέρως, ἐν φυλακαῖς περισσοτέρως, ἐν πληγαῖς ὑπερβαλλόντως, ἐν θανάτοις πολλάκις.
\vs{24}Ὑπὸ Ἰουδαίων πεντάκις τεσσεράκοντα παρὰ μίαν ἔλαβον,
\vs{25}τρὶς ἐραβδίσθην, ἅπαξ ἐλιθάσθην, τρὶς ἐναυάγησα, νυχθήμερον ἐν τῷ βυθῷ πεποίηκα·
\vs{26}ὁδοιπορίαις πολλάκις, κινδύνοις ποταμῶν, κινδύνοις λῃστῶν, κινδύνοις ἐκ γένους, κινδύνοις ἐξ ἐθνῶν, κινδύνοις ἐν πόλει, κινδύνοις ἐν ἐρημίᾳ, κινδύνοις ἐν θαλάσσῃ, κινδύνοις ἐν ψευδαδέλφοις,
\vs{27}κόπῳ καὶ μόχθῳ, ἐν ἀγρυπνίαις πολλάκις, ἐν λιμῷ καὶ δίψει, ἐν νηστείαις πολλάκις, ἐν ψύχει καὶ γυμνότητι·
\vs{28}Χωρὶς τῶν παρεκτὸς ἡ ἐπίστασίς μοι ἡ καθ᾽ ἡμέραν, ἡ μέριμνα πασῶν τῶν ἐκκλησιῶν.
\vs{29}τίς ἀσθενεῖ καὶ οὐκ ἀσθενῶ; τίς σκανδαλίζεται καὶ οὐκ ἐγὼ πυροῦμαι;
\vs{30}Εἰ καυχᾶσθαι δεῖ, τὰ τῆς ἀσθενείας μου καυχήσομαι.
\vs{31}ὁ Θεὸς καὶ Πατὴρ τοῦ Κυρίου Ἰησοῦ οἶδεν, ὁ ὢν εὐλογητὸς εἰς τοὺς αἰῶνας, ὅτι οὐ ψεύδομαι.
\vs{32}ἐν Δαμασκῷ ὁ ἐθνάρχης Ἁρέτα τοῦ βασιλέως ἐφρούρει τὴν πόλιν Δαμασκηνῶν πιάσαι με,
\vs{33}καὶ διὰ θυρίδος ἐν σαργάνῃ ἐχαλάσθην διὰ τοῦ τείχους καὶ ἐξέφυγον τὰς χεῖρας αὐτοῦ.

\ch{12}
Καυχᾶσθαι δεῖ, οὐ συμφέρον μέν, ἐλεύσομαι δὲ εἰς ὀπτασίας καὶ ἀποκαλύψεις Κυρίου.
\vs{2}οἶδα ἄνθρωπον ἐν Χριστῷ πρὸ ἐτῶν δεκατεσσάρων, εἴτε ἐν σώματι οὐκ οἶδα, εἴτε ἐκτὸς τοῦ σώματος οὐκ οἶδα, ὁ Θεὸς οἶδεν, ἁρπαγέντα τὸν τοιοῦτον ἕως τρίτου οὐρανοῦ.
\vs{3}καὶ οἶδα τὸν τοιοῦτον ἄνθρωπον, εἴτε ἐν σώματι εἴτε χωρὶς τοῦ σώματος οὐκ οἶδα, ὁ Θεὸς οἶδεν,
\vs{4}ὅτι ἡρπάγη εἰς τὸν Παράδεισον καὶ ἤκουσεν ἄρρητα ῥήματα ἃ οὐκ ἐξὸν ἀνθρώπῳ λαλῆσαι.
\vs{5}Ὑπὲρ τοῦ τοιούτου καυχήσομαι, ὑπὲρ δὲ ἐμαυτοῦ οὐ καυχήσομαι εἰ μὴ ἐν ταῖς ἀσθενείαις.
\vs{6}ἐὰν γὰρ θελήσω καυχήσασθαι, οὐκ ἔσομαι ἄφρων, ἀλήθειαν γὰρ ἐρῶ· φείδομαι δέ, μή τις εἰς ἐμὲ λογίσηται ὑπὲρ ὃ βλέπει με ἢ ἀκούει τι ἐξ ἐμοῦ
\vs{7}καὶ τῇ ὑπερβολῇ τῶν ἀποκαλύψεων. Διὸ ἵνα μὴ ὑπεραίρωμαι, ἐδόθη μοι σκόλοψ τῇ σαρκί, ἄγγελος Σατανᾶ, ἵνα με κολαφίζῃ, ἵνα μὴ ὑπεραίρωμαι.
\vs{8}ὑπὲρ τούτου τρὶς τὸν Κύριον παρεκάλεσα ἵνα ἀποστῇ ἀπ᾽ ἐμοῦ.
\vs{9}καὶ εἴρηκέν μοι· Ἀρκεῖ σοι ἡ χάρις μου, ἡ γὰρ δύναμις ἐν ἀσθενείᾳ τελεῖται. Ἥδιστα οὖν μᾶλλον καυχήσομαι ἐν ταῖς ἀσθενείαις μου, ἵνα ἐπισκηνώσῃ ἐπ᾽ ἐμὲ ἡ δύναμις τοῦ Χριστοῦ.
\vs{10}διὸ εὐδοκῶ ἐν ἀσθενείαις, ἐν ὕβρεσιν, ἐν ἀνάγκαις, ἐν διωγμοῖς καὶ στενοχωρίαις, ὑπὲρ Χριστοῦ· ὅταν γὰρ ἀσθενῶ, τότε δυνατός εἰμι.

\vs{11}Γέγονα ἄφρων, ὑμεῖς με ἠναγκάσατε. ἐγὼ γὰρ ὤφειλον ὑφ᾽ ὑμῶν συνίστασθαι· οὐδὲν γὰρ ὑστέρησα τῶν Ὑπερλίαν ἀποστόλων εἰ καὶ οὐδέν εἰμι.
\vs{12}τὰ μὲν σημεῖα τοῦ ἀποστόλου κατειργάσθη ἐν ὑμῖν ἐν πάσῃ ὑπομονῇ, σημείοις τε καὶ τέρασιν καὶ δυνάμεσιν.
\vs{13}τί γάρ ἐστιν ὃ ἡσσώθητε ὑπὲρ τὰς λοιπὰς ἐκκλησίας, εἰ μὴ ὅτι αὐτὸς ἐγὼ οὐ κατενάρκησα ὑμῶν; χαρίσασθέ μοι τὴν ἀδικίαν ταύτην.

\vs{14}Ἰδοὺ τρίτον τοῦτο ἑτοίμως ἔχω ἐλθεῖν πρὸς ὑμᾶς, καὶ οὐ καταναρκήσω· οὐ γὰρ ζητῶ τὰ ὑμῶν ἀλλὰ ὑμᾶς. οὐ γὰρ ὀφείλει τὰ τέκνα τοῖς γονεῦσιν θησαυρίζειν ἀλλὰ οἱ γονεῖς τοῖς τέκνοις.
\vs{15}ἐγὼ δὲ ἥδιστα δαπανήσω καὶ ἐκδαπανηθήσομαι ὑπὲρ τῶν ψυχῶν ὑμῶν. εἰ περισσοτέρως ὑμᾶς ἀγαπῶν, ἧσσον ἀγαπῶμαι;
\vs{16}Ἔστω δέ, ἐγὼ οὐ κατεβάρησα ὑμᾶς· ἀλλὰ ὑπάρχων πανοῦργος δόλῳ ὑμᾶς ἔλαβον.
\vs{17}μή τινα ὧν ἀπέσταλκα πρὸς ὑμᾶς, δι᾽ αὐτοῦ ἐπλεονέκτησα ὑμᾶς;
\vs{18}παρεκάλεσα Τίτον καὶ συναπέστειλα τὸν ἀδελφόν· μήτι ἐπλεονέκτησεν ὑμᾶς Τίτος; οὐ τῷ αὐτῷ Πνεύματι περιεπατήσαμεν; οὐ τοῖς αὐτοῖς ἴχνεσιν;

\vs{19}Πάλαι δοκεῖτε ὅτι ὑμῖν ἀπολογούμεθα. κατέναντι Θεοῦ ἐν Χριστῷ λαλοῦμεν· τὰ δὲ πάντα, ἀγαπητοί, ὑπὲρ τῆς ὑμῶν οἰκοδομῆς.
\vs{20}φοβοῦμαι γὰρ μή πως ἐλθὼν οὐχ οἵους θέλω εὕρω ὑμᾶς κἀγὼ εὑρεθῶ ὑμῖν οἷον οὐ θέλετε· μή πως ἔρις, ζῆλος, θυμοί, ἐριθεῖαι, καταλαλιαί, ψιθυρισμοί, φυσιώσεις, ἀκαταστασίαι·
\vs{21}μὴ πάλιν ἐλθόντος μου ταπεινώσῃ με ὁ Θεός μου πρὸς ὑμᾶς καὶ πενθήσω πολλοὺς τῶν προημαρτηκότων καὶ μὴ μετανοησάντων ἐπὶ τῇ ἀκαθαρσίᾳ καὶ πορνείᾳ καὶ ἀσελγείᾳ ᾗ ἔπραξαν.

\ch{13}
Τρίτον τοῦτο ἔρχομαι πρὸς ὑμᾶς· Ἐπὶ στόματος δύο μαρτύρων καὶ τριῶν σταθήσεται πᾶν ῥῆμα.
\vs{2}Προείρηκα καὶ προλέγω, ὡς παρὼν τὸ δεύτερον καὶ ἀπὼν νῦν, τοῖς προημαρτηκόσιν καὶ τοῖς λοιποῖς πᾶσιν, ὅτι ἐὰν ἔλθω εἰς τὸ πάλιν οὐ φείσομαι,
\vs{3}ἐπεὶ δοκιμὴν ζητεῖτε τοῦ ἐν ἐμοὶ λαλοῦντος Χριστοῦ, ὃς εἰς ὑμᾶς οὐκ ἀσθενεῖ ἀλλὰ δυνατεῖ ἐν ὑμῖν.
\vs{4}καὶ γὰρ ἐσταυρώθη ἐξ ἀσθενείας, ἀλλὰ ζῇ ἐκ δυνάμεως Θεοῦ. καὶ γὰρ ἡμεῖς ἀσθενοῦμεν ἐν αὐτῷ, ἀλλὰ ζήσομεν σὺν αὐτῷ ἐκ δυνάμεως Θεοῦ εἰς ὑμᾶς.

\vs{5}Ἑαυτοὺς πειράζετε εἰ ἐστὲ ἐν τῇ πίστει, ἑαυτοὺς δοκιμάζετε· ἢ οὐκ ἐπιγινώσκετε ἑαυτοὺς ὅτι Ἰησοῦς Χριστὸς ἐν ὑμῖν; εἰ μήτι ἀδόκιμοί ἐστε.
\vs{6}ἐλπίζω δὲ ὅτι γνώσεσθε ὅτι ἡμεῖς οὐκ ἐσμὲν ἀδόκιμοι.
\vs{7}Εὐχόμεθα δὲ πρὸς τὸν Θεὸν μὴ ποιῆσαι ὑμᾶς κακὸν μηδέν, οὐχ ἵνα ἡμεῖς δόκιμοι φανῶμεν, ἀλλ᾽ ἵνα ὑμεῖς τὸ καλὸν ποιῆτε, ἡμεῖς δὲ ὡς ἀδόκιμοι ὦμεν.
\vs{8}οὐ γὰρ δυνάμεθά τι κατὰ τῆς ἀληθείας ἀλλὰ ὑπὲρ τῆς ἀληθείας.
\vs{9}χαίρομεν γὰρ ὅταν ἡμεῖς ἀσθενῶμεν, ὑμεῖς δὲ δυνατοὶ ἦτε· τοῦτο καὶ εὐχόμεθα, τὴν ὑμῶν κατάρτισιν.
\vs{10}Διὰ τοῦτο ταῦτα ἀπὼν γράφω, ἵνα παρὼν μὴ ἀποτόμως χρήσωμαι κατὰ τὴν ἐξουσίαν ἣν ὁ Κύριος ἔδωκέν μοι εἰς οἰκοδομὴν καὶ οὐκ εἰς καθαίρεσιν.

\vs{11}Λοιπόν, ἀδελφοί, χαίρετε, καταρτίζεσθε, παρακαλεῖσθε, τὸ αὐτὸ φρονεῖτε, εἰρηνεύετε, καὶ ὁ Θεὸς τῆς ἀγάπης καὶ εἰρήνης ἔσται μεθ᾽ ὑμῶν.
\vs{12}Ἀσπάσασθε ἀλλήλους ἐν ἁγίῳ φιλήματι. Ἀσπάζονται ὑμᾶς οἱ ἅγιοι πάντες.

\vs{13}Ἡ χάρις τοῦ Κυρίου Ἰησοῦ Χριστοῦ καὶ ἡ ἀγάπη τοῦ Θεοῦ καὶ ἡ κοινωνία τοῦ Ἁγίου Πνεύματος μετὰ πάντων ὑμῶν.


\def\book{ΠΡΟΣ ΓΑΛΑΤΑΣ}
\biblebook{ΠΡΟΣ ΓΑΛΑΤΑΣ}


\lettrine[lines=2, loversize=0.2, nindent=0em, findent=.25em]{\textcolor{bookheadingcolor}{Π}}{αῦλος} ἀπόστολος οὐκ ἀπ᾽ ἀνθρώπων οὐδὲ δι᾽ ἀνθρώπου ἀλλὰ διὰ Ἰησοῦ Χριστοῦ καὶ Θεοῦ Πατρὸς τοῦ ἐγείραντος αὐτὸν ἐκ νεκρῶν,
\vs{2}καὶ οἱ σὺν ἐμοὶ πάντες ἀδελφοί Ταῖς ἐκκλησίαις τῆς Γαλατίας,
\vs{3}Χάρις ὑμῖν καὶ εἰρήνη ἀπὸ Θεοῦ Πατρὸς ἡμῶν καὶ Κυρίου Ἰησοῦ Χριστοῦ
\vs{4}τοῦ δόντος ἑαυτὸν ὑπὲρ τῶν ἁμαρτιῶν ἡμῶν, ὅπως ἐξέληται ἡμᾶς ἐκ τοῦ αἰῶνος τοῦ ἐνεστῶτος πονηροῦ κατὰ τὸ θέλημα τοῦ Θεοῦ καὶ Πατρὸς ἡμῶν,
\vs{5}ᾧ ἡ δόξα εἰς τοὺς αἰῶνας τῶν αἰώνων, ἀμήν.

\vs{6}Θαυμάζω ὅτι οὕτως ταχέως μετατίθεσθε ἀπὸ τοῦ καλέσαντος ὑμᾶς ἐν χάριτι Χριστοῦ εἰς ἕτερον εὐαγγέλιον,
\vs{7}ὃ οὐκ ἔστιν ἄλλο, εἰ μή τινές εἰσιν οἱ ταράσσοντες ὑμᾶς καὶ θέλοντες μεταστρέψαι τὸ εὐαγγέλιον τοῦ Χριστοῦ.
\vs{8}Ἀλλὰ καὶ ἐὰν ἡμεῖς ἢ ἄγγελος ἐξ οὐρανοῦ εὐαγγελίζηται ὑμῖν παρ᾽ ὃ εὐηγγελισάμεθα ὑμῖν, ἀνάθεμα ἔστω.
\vs{9}ὡς προειρήκαμεν καὶ ἄρτι πάλιν λέγω· εἴ τις ὑμᾶς εὐαγγελίζεται παρ᾽ ὃ παρελάβετε, ἀνάθεμα ἔστω.

\vs{10}Ἄρτι γὰρ ἀνθρώπους πείθω ἢ τὸν Θεόν; ἢ ζητῶ ἀνθρώποις ἀρέσκειν; εἰ ἔτι ἀνθρώποις ἤρεσκον, Χριστοῦ δοῦλος οὐκ ἂν ἤμην.
\vs{11}γνωρίζω γὰρ ὑμῖν, ἀδελφοί, τὸ εὐαγγέλιον τὸ εὐαγγελισθὲν ὑπ᾽ ἐμοῦ ὅτι οὐκ ἔστιν κατὰ ἄνθρωπον·
\vs{12}οὐδὲ γὰρ ἐγὼ παρὰ ἀνθρώπου παρέλαβον αὐτό οὔτε ἐδιδάχθην, ἀλλὰ δι᾽ ἀποκαλύψεως Ἰησοῦ Χριστοῦ.
\vs{13}Ἠκούσατε γὰρ τὴν ἐμὴν ἀναστροφήν ποτε ἐν τῷ Ἰουδαϊσμῷ, ὅτι καθ᾽ ὑπερβολὴν ἐδίωκον τὴν ἐκκλησίαν τοῦ Θεοῦ καὶ ἐπόρθουν αὐτήν,
\vs{14}καὶ προέκοπτον ἐν τῷ Ἰουδαϊσμῷ ὑπὲρ πολλοὺς συνηλικιώτας ἐν τῷ γένει μου, περισσοτέρως ζηλωτὴς ὑπάρχων τῶν πατρικῶν μου παραδόσεων.
\vs{15}Ὅτε δὲ εὐδόκησεν ὁ θεὸς ὁ ἀφορίσας με ἐκ κοιλίας μητρός μου καὶ καλέσας διὰ τῆς χάριτος αὐτοῦ
\vs{16}ἀποκαλύψαι τὸν Υἱὸν αὐτοῦ ἐν ἐμοὶ, ἵνα εὐαγγελίζωμαι αὐτὸν ἐν τοῖς ἔθνεσιν, εὐθέως οὐ προσανεθέμην σαρκὶ καὶ αἵματι
\vs{17}οὐδὲ ἀνῆλθον εἰς Ἱεροσόλυμα πρὸς τοὺς πρὸ ἐμοῦ ἀποστόλους, ἀλλὰ ἀπῆλθον εἰς Ἀραβίαν καὶ πάλιν ὑπέστρεψα εἰς Δαμασκόν.
\vs{18}Ἔπειτα μετὰ ἔτη τρία ἀνῆλθον εἰς Ἱεροσόλυμα ἱστορῆσαι Κηφᾶν καὶ ἐπέμεινα πρὸς αὐτὸν ἡμέρας δεκαπέντε,
\vs{19}ἕτερον δὲ τῶν ἀποστόλων οὐκ εἶδον εἰ μὴ Ἰάκωβον τὸν ἀδελφὸν τοῦ Κυρίου.
\vs{20}ἃ δὲ γράφω ὑμῖν, ἰδοὺ ἐνώπιον τοῦ Θεοῦ ὅτι οὐ ψεύδομαι.
\vs{21}Ἔπειτα ἦλθον εἰς τὰ κλίματα τῆς Συρίας καὶ τῆς Κιλικίας·
\vs{22}ἤμην δὲ ἀγνοούμενος τῷ προσώπῳ ταῖς ἐκκλησίαις τῆς Ἰουδαίας ταῖς ἐν Χριστῷ.
\vs{23}μόνον δὲ ἀκούοντες ἦσαν ὅτι Ὁ διώκων ἡμᾶς ποτε νῦν εὐαγγελίζεται τὴν πίστιν ἥν ποτε ἐπόρθει,
\vs{24}καὶ ἐδόξαζον ἐν ἐμοὶ τὸν Θεόν.

\ch{2}
Ἔπειτα διὰ δεκατεσσάρων ἐτῶν πάλιν ἀνέβην εἰς Ἱεροσόλυμα μετὰ Βαρνάβα συμπαραλαβὼν καὶ Τίτον·
\vs{2}ἀνέβην δὲ κατὰ ἀποκάλυψιν· καὶ ἀνεθέμην αὐτοῖς τὸ εὐαγγέλιον ὃ κηρύσσω ἐν τοῖς ἔθνεσιν, κατ᾽ ἰδίαν δὲ τοῖς δοκοῦσιν, μή πως εἰς κενὸν τρέχω ἢ ἔδραμον.
\vs{3}ἀλλ᾽ οὐδὲ Τίτος ὁ σὺν ἐμοί, Ἕλλην ὤν, ἠναγκάσθη περιτμηθῆναι·
\vs{4}διὰ δὲ τοὺς παρεισάκτους ψευδαδέλφους, οἵτινες παρεισῆλθον κατασκοπῆσαι τὴν ἐλευθερίαν ἡμῶν ἣν ἔχομεν ἐν Χριστῷ Ἰησοῦ, ἵνα ἡμᾶς καταδουλώσουσιν,
\vs{5}οἷς οὐδὲ πρὸς ὥραν εἴξαμεν τῇ ὑποταγῇ, ἵνα ἡ ἀλήθεια τοῦ εὐαγγελίου διαμείνῃ πρὸς ὑμᾶς.
\vs{6}Ἀπὸ δὲ τῶν δοκούντων εἶναί τι,— ὁποῖοί ποτε ἦσαν οὐδέν μοι διαφέρει· πρόσωπον ὁ Θεὸς ἀνθρώπου οὐ λαμβάνει— ἐμοὶ γὰρ οἱ δοκοῦντες οὐδὲν προσανέθεντο,
\vs{7}ἀλλὰ τοὐναντίον ἰδόντες ὅτι πεπίστευμαι τὸ εὐαγγέλιον τῆς ἀκροβυστίας καθὼς Πέτρος τῆς περιτομῆς,
\vs{8}ὁ γὰρ ἐνεργήσας Πέτρῳ εἰς ἀποστολὴν τῆς περιτομῆς ἐνήργησεν καὶ ἐμοὶ εἰς τὰ ἔθνη,
\vs{9}καὶ γνόντες τὴν χάριν τὴν δοθεῖσάν μοι, Ἰάκωβος καὶ Κηφᾶς καὶ Ἰωάννης, οἱ δοκοῦντες στῦλοι εἶναι, δεξιὰς ἔδωκαν ἐμοὶ καὶ Βαρνάβα κοινωνίας, ἵνα ἡμεῖς εἰς τὰ ἔθνη, αὐτοὶ δὲ εἰς τὴν περιτομήν·
\vs{10}μόνον τῶν πτωχῶν ἵνα μνημονεύωμεν, ὃ καὶ ἐσπούδασα αὐτὸ τοῦτο ποιῆσαι.

\vs{11}Ὅτε δὲ ἦλθεν Κηφᾶς εἰς Ἀντιόχειαν, κατὰ πρόσωπον αὐτῷ ἀντέστην, ὅτι κατεγνωσμένος ἦν.
\vs{12}πρὸ τοῦ γὰρ ἐλθεῖν τινας ἀπὸ Ἰακώβου μετὰ τῶν ἐθνῶν συνήσθιεν· ὅτε δὲ ἦλθον, ὑπέστελλεν καὶ ἀφώριζεν ἑαυτόν φοβούμενος τοὺς ἐκ περιτομῆς.
\vs{13}καὶ συνυπεκρίθησαν αὐτῷ καὶ οἱ λοιποὶ Ἰουδαῖοι, ὥστε καὶ Βαρνάβας συναπήχθη αὐτῶν τῇ ὑποκρίσει.
\vs{14}Ἀλλ᾽ ὅτε εἶδον ὅτι οὐκ ὀρθοποδοῦσιν πρὸς τὴν ἀλήθειαν τοῦ εὐαγγελίου, εἶπον τῷ Κηφᾷ ἔμπροσθεν πάντων· Εἰ σὺ Ἰουδαῖος ὑπάρχων ἐθνικῶς καὶ οὐκ Ἰουδαϊκῶς ζῇς, πῶς τὰ ἔθνη ἀναγκάζεις ἰουδαΐζειν;
\vs{15}Ἡμεῖς φύσει Ἰουδαῖοι καὶ οὐκ ἐξ ἐθνῶν Ἁμαρτωλοί·
\vs{16}εἰδότες δὲ ὅτι οὐ δικαιοῦται ἄνθρωπος ἐξ ἔργων νόμου ἐὰν μὴ διὰ πίστεως Ἰησοῦ Χριστοῦ, καὶ ἡμεῖς εἰς Χριστὸν Ἰησοῦν ἐπιστεύσαμεν, ἵνα δικαιωθῶμεν ἐκ πίστεως Χριστοῦ καὶ οὐκ ἐξ ἔργων νόμου, ὅτι ἐξ ἔργων νόμου οὐ δικαιωθήσεται πᾶσα σάρξ.
\vs{17}Εἰ δὲ ζητοῦντες δικαιωθῆναι ἐν Χριστῷ εὑρέθημεν καὶ αὐτοὶ ἁμαρτωλοί, ἆρα Χριστὸς ἁμαρτίας διάκονος; μὴ γένοιτο.
\vs{18}εἰ γὰρ ἃ κατέλυσα ταῦτα πάλιν οἰκοδομῶ, παραβάτην ἐμαυτὸν συνιστάνω.
\vs{19}ἐγὼ γὰρ διὰ νόμου νόμῳ ἀπέθανον, ἵνα Θεῷ ζήσω. Χριστῷ συνεσταύρωμαι·
\vs{20}ζῶ δὲ οὐκέτι ἐγώ, ζῇ δὲ ἐν ἐμοὶ Χριστός· ὃ δὲ νῦν ζῶ ἐν σαρκί, ἐν πίστει ζῶ τῇ τοῦ Υἱοῦ τοῦ Θεοῦ τοῦ ἀγαπήσαντός με καὶ παραδόντος ἑαυτὸν ὑπὲρ ἐμοῦ.
\vs{21}Οὐκ ἀθετῶ τὴν χάριν τοῦ Θεοῦ· εἰ γὰρ διὰ νόμου δικαιοσύνη, ἄρα Χριστὸς δωρεὰν ἀπέθανεν.

\ch{3}
Ὦ ἀνόητοι Γαλάται, τίς ὑμᾶς ἐβάσκανεν, οἷς κατ᾽ ὀφθαλμοὺς Ἰησοῦς Χριστὸς προεγράφη ἐσταυρωμένος;
\vs{2}τοῦτο μόνον θέλω μαθεῖν ἀφ᾽ ὑμῶν· ἐξ ἔργων νόμου τὸ Πνεῦμα ἐλάβετε ἢ ἐξ ἀκοῆς πίστεως;
\vs{3}Οὕτως ἀνόητοί ἐστε, ἐναρξάμενοι Πνεύματι νῦν σαρκὶ ἐπιτελεῖσθε;
\vs{4}τοσαῦτα ἐπάθετε εἰκῇ; εἴ γε καὶ εἰκῇ.
\vs{5}ὁ οὖν ἐπιχορηγῶν ὑμῖν τὸ Πνεῦμα καὶ ἐνεργῶν δυνάμεις ἐν ὑμῖν, ἐξ ἔργων νόμου ἢ ἐξ ἀκοῆς πίστεως;

\vs{6}Καθὼς Ἀβραὰμ ἐπίστευσεν τῷ Θεῷ, καὶ ἐλογίσθη αὐτῷ εἰς δικαιοσύνην·
\vs{7}Γινώσκετε ἄρα ὅτι οἱ ἐκ πίστεως, οὗτοι υἱοί εἰσιν Ἀβραάμ.
\vs{8}προϊδοῦσα δὲ ἡ γραφὴ ὅτι ἐκ πίστεως δικαιοῖ τὰ ἔθνη ὁ Θεὸς, προευηγγελίσατο τῷ Ἀβραὰμ ὅτι Ἐνευλογηθήσονται ἐν σοὶ πάντα τὰ ἔθνη·
\vs{9}ὥστε οἱ ἐκ πίστεως εὐλογοῦνται σὺν τῷ πιστῷ Ἀβραάμ.

\vs{10}Ὅσοι γὰρ ἐξ ἔργων νόμου εἰσὶν, ὑπὸ κατάραν εἰσίν· γέγραπται γὰρ ὅτι Ἐπικατάρατος πᾶς ὃς οὐκ ἐμμένει πᾶσιν τοῖς γεγραμμένοις ἐν τῷ βιβλίῳ τοῦ νόμου τοῦ ποιῆσαι αὐτά.
\vs{11}ὅτι δὲ ἐν νόμῳ οὐδεὶς δικαιοῦται παρὰ τῷ Θεῷ δῆλον, ὅτι Ὁ δίκαιος ἐκ πίστεως ζήσεται·
\vs{12}ὁ δὲ νόμος οὐκ ἔστιν ἐκ πίστεως, ἀλλ᾽ Ὁ ποιήσας αὐτὰ ζήσεται ἐν αὐτοῖς.
\vs{13}Χριστὸς ἡμᾶς ἐξηγόρασεν ἐκ τῆς κατάρας τοῦ νόμου γενόμενος ὑπὲρ ἡμῶν κατάρα, ὅτι γέγραπται· Ἐπικατάρατος πᾶς ὁ κρεμάμενος ἐπὶ ξύλου,
\vs{14}ἵνα εἰς τὰ ἔθνη ἡ εὐλογία τοῦ Ἀβραὰμ γένηται ἐν Χριστῷ Ἰησοῦ, ἵνα τὴν ἐπαγγελίαν τοῦ Πνεύματος λάβωμεν διὰ τῆς πίστεως.

\vs{15}Ἀδελφοί, κατὰ ἄνθρωπον λέγω· ὅμως ἀνθρώπου κεκυρωμένην διαθήκην οὐδεὶς ἀθετεῖ ἢ ἐπιδιατάσσεται.
\vs{16}τῷ δὲ Ἀβραὰμ ἐρρέθησαν αἱ ἐπαγγελίαι καὶ τῷ σπέρματι αὐτοῦ. οὐ λέγει· Καὶ τοῖς σπέρμασιν, ὡς ἐπὶ πολλῶν ἀλλ᾽ ὡς ἐφ᾽ ἑνός· Καὶ τῷ σπέρματί σου, ὅς ἐστιν Χριστός.
\vs{17}Τοῦτο δὲ λέγω· διαθήκην προκεκυρωμένην ὑπὸ τοῦ Θεοῦ ὁ μετὰ τετρακόσια καὶ τριάκοντα ἔτη γεγονὼς νόμος οὐκ ἀκυροῖ εἰς τὸ καταργῆσαι τὴν ἐπαγγελίαν.
\vs{18}εἰ γὰρ ἐκ νόμου ἡ κληρονομία, οὐκέτι ἐξ ἐπαγγελίας· τῷ δὲ Ἀβραὰμ δι᾽ ἐπαγγελίας κεχάρισται ὁ Θεός.

\vs{19}Τί οὖν ὁ νόμος; τῶν παραβάσεων χάριν προσετέθη, ἄχρις οὗ ἔλθῃ τὸ σπέρμα ᾧ ἐπήγγελται, διαταγεὶς δι᾽ ἀγγέλων ἐν χειρὶ μεσίτου.
\vs{20}ὁ δὲ μεσίτης ἑνὸς οὐκ ἔστιν, ὁ δὲ Θεὸς εἷς ἐστιν.
\vs{21}Ὁ οὖν νόμος κατὰ τῶν ἐπαγγελιῶν τοῦ Θεοῦ; μὴ γένοιτο. εἰ γὰρ ἐδόθη νόμος ὁ δυνάμενος ζωοποιῆσαι, ὄντως ἐκ νόμου ἂν ἦν ἡ δικαιοσύνη·
\vs{22}ἀλλὰ συνέκλεισεν ἡ γραφὴ τὰ πάντα ὑπὸ ἁμαρτίαν, ἵνα ἡ ἐπαγγελία ἐκ πίστεως Ἰησοῦ Χριστοῦ δοθῇ τοῖς πιστεύουσιν.

\vs{23}Πρὸ τοῦ δὲ ἐλθεῖν τὴν πίστιν ὑπὸ νόμον ἐφρουρούμεθα συνκλειόμενοι εἰς τὴν μέλλουσαν πίστιν ἀποκαλυφθῆναι,
\vs{24}ὥστε ὁ νόμος παιδαγωγὸς ἡμῶν γέγονεν εἰς Χριστόν, ἵνα ἐκ πίστεως δικαιωθῶμεν·
\vs{25}ἐλθούσης δὲ τῆς πίστεως οὐκέτι ὑπὸ παιδαγωγόν ἐσμεν.
\vs{26}Πάντες γὰρ υἱοὶ Θεοῦ ἐστε διὰ τῆς πίστεως ἐν Χριστῷ Ἰησοῦ·
\vs{27}ὅσοι γὰρ εἰς Χριστὸν ἐβαπτίσθητε, Χριστὸν ἐνεδύσασθε.
\vs{28}οὐκ ἔνι Ἰουδαῖος οὐδὲ Ἕλλην, οὐκ ἔνι δοῦλος οὐδὲ ἐλεύθερος, οὐκ ἔνι ἄρσεν καὶ θῆλυ· πάντες γὰρ ὑμεῖς εἷς ἐστε ἐν Χριστῷ Ἰησοῦ.
\vs{29}εἰ δὲ ὑμεῖς Χριστοῦ, ἄρα τοῦ Ἀβραὰμ σπέρμα ἐστέ, κατ᾽ ἐπαγγελίαν κληρονόμοι.

\ch{4}
Λέγω δέ, ἐφ᾽ ὅσον χρόνον ὁ κληρονόμος νήπιός ἐστιν, οὐδὲν διαφέρει δούλου κύριος πάντων ὤν,
\vs{2}ἀλλὰ ὑπὸ ἐπιτρόπους ἐστὶν καὶ οἰκονόμους ἄχρι τῆς προθεσμίας τοῦ πατρός.
\vs{3}οὕτως καὶ ἡμεῖς, ὅτε ἦμεν νήπιοι, ὑπὸ τὰ στοιχεῖα τοῦ κόσμου ἤμεθα δεδουλωμένοι·
\vs{4}Ὅτε δὲ ἦλθεν τὸ πλήρωμα τοῦ χρόνου, ἐξαπέστειλεν ὁ Θεὸς τὸν Υἱὸν αὐτοῦ, γενόμενον ἐκ γυναικός, γενόμενον ὑπὸ νόμον,
\vs{5}ἵνα τοὺς ὑπὸ νόμον ἐξαγοράσῃ, ἵνα τὴν υἱοθεσίαν ἀπολάβωμεν.
\vs{6}Ὅτι δέ ἐστε υἱοί, ἐξαπέστειλεν ὁ Θεὸς τὸ Πνεῦμα τοῦ Υἱοῦ αὐτοῦ εἰς τὰς καρδίας ἡμῶν κρᾶζον· Ἀββᾶ ὁ Πατήρ.
\vs{7}ὥστε οὐκέτι εἶ δοῦλος ἀλλὰ υἱός· εἰ δὲ υἱός, καὶ κληρονόμος διὰ Θεοῦ.

\vs{8}Ἀλλὰ τότε μὲν οὐκ εἰδότες Θεὸν ἐδουλεύσατε τοῖς φύσει μὴ οὖσιν θεοῖς·
\vs{9}νῦν δὲ γνόντες Θεόν, μᾶλλον δὲ γνωσθέντες ὑπὸ Θεοῦ, πῶς ἐπιστρέφετε πάλιν ἐπὶ τὰ ἀσθενῆ καὶ πτωχὰ στοιχεῖα οἷς πάλιν ἄνωθεν δουλεύειν θέλετε;
\vs{10}ἡμέρας παρατηρεῖσθε καὶ μῆνας καὶ καιροὺς καὶ ἐνιαυτούς,
\vs{11}φοβοῦμαι ὑμᾶς μή πως εἰκῇ κεκοπίακα εἰς ὑμᾶς.

\vs{12}Γίνεσθε ὡς ἐγώ, ὅτι κἀγὼ ὡς ὑμεῖς, ἀδελφοί, δέομαι ὑμῶν. οὐδέν με ἠδικήσατε·
\vs{13}Οἴδατε δὲ ὅτι δι᾽ ἀσθένειαν τῆς σαρκὸς εὐηγγελισάμην ὑμῖν τὸ πρότερον,
\vs{14}καὶ τὸν πειρασμὸν ὑμῶν ἐν τῇ σαρκί μου οὐκ ἐξουθενήσατε οὐδὲ ἐξεπτύσατε, ἀλλὰ ὡς ἄγγελον Θεοῦ ἐδέξασθέ με, ὡς Χριστὸν Ἰησοῦν.
\vs{15}ποῦ οὖν ὁ μακαρισμὸς ὑμῶν; μαρτυρῶ γὰρ ὑμῖν ὅτι εἰ δυνατὸν τοὺς ὀφθαλμοὺς ὑμῶν ἐξορύξαντες ἐδώκατέ μοι.
\vs{16}ὥστε ἐχθρὸς ὑμῶν γέγονα ἀληθεύων ὑμῖν;
\vs{17}Ζηλοῦσιν ὑμᾶς οὐ καλῶς, ἀλλὰ ἐκκλεῖσαι ὑμᾶς θέλουσιν, ἵνα αὐτοὺς ζηλοῦτε·
\vs{18}καλὸν δὲ ζηλοῦσθαι ἐν καλῷ πάντοτε καὶ μὴ μόνον ἐν τῷ παρεῖναί με πρὸς ὑμᾶς.
\vs{19}Τέκνα μου, οὓς πάλιν ὠδίνω μέχρις οὗ μορφωθῇ Χριστὸς ἐν ὑμῖν·
\vs{20}ἤθελον δὲ παρεῖναι πρὸς ὑμᾶς ἄρτι καὶ ἀλλάξαι τὴν φωνήν μου, ὅτι ἀποροῦμαι ἐν ὑμῖν.

\vs{21}Λέγετέ μοι, οἱ ὑπὸ νόμον θέλοντες εἶναι, τὸν νόμον οὐκ ἀκούετε;
\vs{22}γέγραπται γὰρ ὅτι Ἀβραὰμ δύο υἱοὺς ἔσχεν, ἕνα ἐκ τῆς παιδίσκης καὶ ἕνα ἐκ τῆς ἐλευθέρας.
\vs{23}ἀλλ᾽ ὁ μὲν ἐκ τῆς παιδίσκης κατὰ σάρκα γεγέννηται, ὁ δὲ ἐκ τῆς ἐλευθέρας δι᾽ ἐπαγγελίας.
\vs{24}ἅτινά ἐστιν ἀλληγορούμενα· αὗται γάρ εἰσιν δύο διαθῆκαι, μία μὲν ἀπὸ ὄρους Σινᾶ εἰς δουλείαν γεννῶσα, ἥτις ἐστὶν Ἁγάρ.
\vs{25}τὸ δὲ Ἁγὰρ Σινᾶ ὄρος ἐστὶν ἐν τῇ Ἀραβίᾳ· συστοιχεῖ δὲ τῇ νῦν Ἰερουσαλήμ, δουλεύει γὰρ μετὰ τῶν τέκνων αὐτῆς.
\vs{26}ἡ δὲ ἄνω Ἰερουσαλὴμ ἐλευθέρα ἐστίν, ἥτις ἐστὶν μήτηρ ἡμῶν·
\vs{27}γέγραπται γάρ· 
\begin{poetryblock}

\begin{quote}Εὐφράνθητι, στεῖρα ἡ οὐ τίκτουσα,\end{quote} 

\begin{quote}ῥῆξον καὶ βόησον, ἡ οὐκ ὠδίνουσα·\end{quote} 

\begin{quote}ὅτι πολλὰ τὰ τέκνα τῆς ἐρήμου\end{quote} 

\begin{quote}μᾶλλον ἢ τῆς ἐχούσης τὸν ἄνδρα.\end{quote}
\end{poetryblock}

\vs{28}Ὑμεῖς δέ, ἀδελφοί, κατὰ Ἰσαὰκ ἐπαγγελίας τέκνα ἐστέ.
\vs{29}ἀλλ᾽ ὥσπερ τότε ὁ κατὰ σάρκα γεννηθεὶς ἐδίωκεν τὸν κατὰ Πνεῦμα, οὕτως καὶ νῦν.
\vs{30}Ἀλλὰ τί λέγει ἡ γραφή; Ἔκβαλε τὴν παιδίσκην καὶ τὸν υἱὸν αὐτῆς· οὐ γὰρ μὴ κληρονομήσει ὁ υἱὸς τῆς παιδίσκης μετὰ τοῦ υἱοῦ τῆς ἐλευθέρας.
\vs{31}διό, ἀδελφοί, οὐκ ἐσμὲν παιδίσκης τέκνα ἀλλὰ τῆς ἐλευθέρας.

\ch{5}
Τῇ ἐλευθερίᾳ ἡμᾶς Χριστὸς ἠλευθέρωσεν· στήκετε οὖν καὶ μὴ πάλιν ζυγῷ δουλείας ἐνέχεσθε.
\vs{2}Ἴδε ἐγὼ Παῦλος λέγω ὑμῖν ὅτι ἐὰν περιτέμνησθε, Χριστὸς ὑμᾶς οὐδὲν ὠφελήσει.
\vs{3}μαρτύρομαι δὲ πάλιν παντὶ ἀνθρώπῳ περιτεμνομένῳ ὅτι ὀφειλέτης ἐστὶν ὅλον τὸν νόμον ποιῆσαι.
\vs{4}κατηργήθητε ἀπὸ Χριστοῦ, οἵτινες ἐν νόμῳ δικαιοῦσθε, τῆς χάριτος ἐξεπέσατε.
\vs{5}Ἡμεῖς γὰρ Πνεύματι ἐκ πίστεως ἐλπίδα δικαιοσύνης ἀπεκδεχόμεθα.
\vs{6}ἐν γὰρ Χριστῷ Ἰησοῦ οὔτε περιτομή τι ἰσχύει οὔτε ἀκροβυστία ἀλλὰ πίστις δι᾽ ἀγάπης ἐνεργουμένη.

\vs{7}Ἐτρέχετε καλῶς· τίς ὑμᾶς ἐνέκοψεν τῇ ἀληθείᾳ μὴ πείθεσθαι;
\vs{8}ἡ πεισμονὴ οὐκ ἐκ τοῦ καλοῦντος ὑμᾶς.
\vs{9}μικρὰ ζύμη ὅλον τὸ φύραμα ζυμοῖ.
\vs{10}ἐγὼ πέποιθα εἰς ὑμᾶς ἐν Κυρίῳ ὅτι οὐδὲν ἄλλο φρονήσετε· ὁ δὲ ταράσσων ὑμᾶς βαστάσει τὸ κρίμα, ὅστις ἐὰν ᾖ.
\vs{11}Ἐγὼ δέ, ἀδελφοί, εἰ περιτομὴν ἔτι κηρύσσω, τί ἔτι διώκομαι; ἄρα κατήργηται τὸ σκάνδαλον τοῦ σταυροῦ.
\vs{12}Ὄφελον καὶ ἀποκόψονται οἱ ἀναστατοῦντες ὑμᾶς.

\vs{13}Ὑμεῖς γὰρ ἐπ᾽ ἐλευθερίᾳ ἐκλήθητε, ἀδελφοί· μόνον μὴ τὴν ἐλευθερίαν εἰς ἀφορμὴν τῇ σαρκί, ἀλλὰ διὰ τῆς ἀγάπης δουλεύετε ἀλλήλοις.
\vs{14}ὁ γὰρ πᾶς νόμος ἐν ἑνὶ λόγῳ πεπλήρωται, ἐν τῷ· Ἀγαπήσεις τὸν πλησίον σου ὡς σεαυτόν.
\vs{15}εἰ δὲ ἀλλήλους δάκνετε καὶ κατεσθίετε, βλέπετε μὴ ὑπ᾽ ἀλλήλων ἀναλωθῆτε.
\vs{16}Λέγω δέ, Πνεύματι περιπατεῖτε καὶ ἐπιθυμίαν σαρκὸς οὐ μὴ τελέσητε.
\vs{17}ἡ γὰρ σὰρξ ἐπιθυμεῖ κατὰ τοῦ Πνεύματος, τὸ δὲ Πνεῦμα κατὰ τῆς σαρκός, ταῦτα γὰρ ἀλλήλοις ἀντίκειται, ἵνα μὴ ἃ ἐὰν θέλητε ταῦτα ποιῆτε.
\vs{18}εἰ δὲ Πνεύματι ἄγεσθε, οὐκ ἐστὲ ὑπὸ νόμον.
\vs{19}Φανερὰ δέ ἐστιν τὰ ἔργα τῆς σαρκός, ἅτινά ἐστιν πορνεία, ἀκαθαρσία, ἀσέλγεια,
\vs{20}εἰδωλολατρία, φαρμακεία, ἔχθραι, ἔρις, ζῆλος, θυμοί, ἐριθεῖαι, διχοστασίαι, αἱρέσεις,
\vs{21}φθόνοι, μέθαι, κῶμοι καὶ τὰ ὅμοια τούτοις, ἃ προλέγω ὑμῖν, καθὼς προεῖπον ὅτι οἱ τὰ τοιαῦτα πράσσοντες βασιλείαν Θεοῦ οὐ κληρονομήσουσιν.
\vs{22}Ὁ δὲ καρπὸς τοῦ Πνεύματός ἐστιν ἀγάπη χαρά εἰρήνη, μακροθυμία χρηστότης ἀγαθωσύνη, πίστις
\vs{23}πραΰτης ἐγκράτεια· κατὰ τῶν τοιούτων οὐκ ἔστιν νόμος.
\vs{24}Οἱ δὲ τοῦ Χριστοῦ Ἰησοῦ τὴν σάρκα ἐσταύρωσαν σὺν τοῖς παθήμασιν καὶ ταῖς ἐπιθυμίαις.
\vs{25}Εἰ ζῶμεν Πνεύματι, Πνεύματι καὶ στοιχῶμεν.
\vs{26}μὴ γινώμεθα κενόδοξοι, ἀλλήλους προκαλούμενοι, ἀλλήλοις φθονοῦντες.

\ch{6}
Ἀδελφοί, ἐὰν καὶ προλημφθῇ ἄνθρωπος ἔν τινι παραπτώματι, ὑμεῖς οἱ πνευματικοὶ καταρτίζετε τὸν τοιοῦτον ἐν πνεύματι πραΰτητος, σκοπῶν σεαυτόν μὴ καὶ σὺ πειρασθῇς.
\vs{2}Ἀλλήλων τὰ βάρη βαστάζετε καὶ οὕτως ἀναπληρώσετε τὸν νόμον τοῦ Χριστοῦ.
\vs{3}εἰ γὰρ δοκεῖ τις εἶναί τι μηδὲν ὤν, φρεναπατᾷ ἑαυτόν.
\vs{4}Τὸ δὲ ἔργον ἑαυτοῦ δοκιμαζέτω ἕκαστος, καὶ τότε εἰς ἑαυτὸν μόνον τὸ καύχημα ἕξει καὶ οὐκ εἰς τὸν ἕτερον·
\vs{5}ἕκαστος γὰρ τὸ ἴδιον φορτίον βαστάσει.

\vs{6}Κοινωνείτω δὲ ὁ κατηχούμενος τὸν λόγον τῷ κατηχοῦντι ἐν πᾶσιν ἀγαθοῖς.
\vs{7}Μὴ πλανᾶσθε, Θεὸς οὐ μυκτηρίζεται. ὃ γὰρ ἐὰν σπείρῃ ἄνθρωπος, τοῦτο καὶ θερίσει·
\vs{8}ὅτι ὁ σπείρων εἰς τὴν σάρκα ἑαυτοῦ ἐκ τῆς σαρκὸς θερίσει φθοράν, ὁ δὲ σπείρων εἰς τὸ Πνεῦμα ἐκ τοῦ Πνεύματος θερίσει ζωὴν αἰώνιον.
\vs{9}Τὸ δὲ καλὸν ποιοῦντες μὴ ἐνκακῶμεν, καιρῷ γὰρ ἰδίῳ θερίσομεν μὴ ἐκλυόμενοι.
\vs{10}Ἄρα οὖν ὡς καιρὸν ἔχομεν, ἐργαζώμεθα τὸ ἀγαθὸν πρὸς πάντας, μάλιστα δὲ πρὸς τοὺς οἰκείους τῆς πίστεως.

\vs{11}Ἴδετε πηλίκοις ὑμῖν γράμμασιν ἔγραψα τῇ ἐμῇ χειρί.
\vs{12}Ὅσοι θέλουσιν εὐπροσωπῆσαι ἐν σαρκί, οὗτοι ἀναγκάζουσιν ὑμᾶς περιτέμνεσθαι, μόνον ἵνα τῷ σταυρῷ τοῦ Χριστοῦ μὴ διώκωνται.
\vs{13}οὐδὲ γὰρ οἱ περιτεμνόμενοι αὐτοὶ νόμον φυλάσσουσιν ἀλλὰ θέλουσιν ὑμᾶς περιτέμνεσθαι, ἵνα ἐν τῇ ὑμετέρᾳ σαρκὶ καυχήσωνται.
\vs{14}Ἐμοὶ δὲ μὴ γένοιτο καυχᾶσθαι εἰ μὴ ἐν τῷ σταυρῷ τοῦ Κυρίου ἡμῶν Ἰησοῦ Χριστοῦ, δι᾽ οὗ ἐμοὶ κόσμος ἐσταύρωται κἀγὼ κόσμῳ.
\vs{15}οὔτε γὰρ περιτομή τί ἐστιν οὔτε ἀκροβυστία ἀλλὰ καινὴ κτίσις.
\vs{16}Καὶ ὅσοι τῷ κανόνι τούτῳ στοιχήσουσιν, εἰρήνη ἐπ᾽ αὐτοὺς καὶ ἔλεος καὶ ἐπὶ τὸν Ἰσραὴλ τοῦ Θεοῦ.

\vs{17}Τοῦ λοιποῦ κόπους μοι μηδεὶς παρεχέτω· ἐγὼ γὰρ τὰ στίγματα τοῦ Ἰησοῦ ἐν τῷ σώματί μου βαστάζω.

\vs{18}Ἡ χάρις τοῦ Κυρίου ἡμῶν Ἰησοῦ Χριστοῦ μετὰ τοῦ πνεύματος ὑμῶν, ἀδελφοί· Ἀμήν.


\def\book{ΠΡΟΣ ΕΦΕΣΙΟΥΣ}
\biblebook{ΠΡΟΣ ΕΦΕΣΙΟΥΣ}


\lettrine[lines=2, loversize=0.2, nindent=0em, findent=.25em]{\textcolor{bookheadingcolor}{Π}}{αῦλος} ἀπόστολος Χριστοῦ Ἰησοῦ διὰ θελήματος Θεοῦ Τοῖς ἁγίοις τοῖς οὖσιν ἐν Ἐφέσῳ καὶ πιστοῖς ἐν Χριστῷ Ἰησοῦ,
\vs{2}Χάρις ὑμῖν καὶ εἰρήνη ἀπὸ Θεοῦ Πατρὸς ἡμῶν καὶ Κυρίου Ἰησοῦ Χριστοῦ.

\vs{3}Εὐλογητὸς ὁ Θεὸς καὶ Πατὴρ τοῦ Κυρίου ἡμῶν Ἰησοῦ Χριστοῦ, ὁ εὐλογήσας ἡμᾶς ἐν πάσῃ εὐλογίᾳ πνευματικῇ ἐν τοῖς ἐπουρανίοις ἐν Χριστῷ,
\vs{4}καθὼς ἐξελέξατο ἡμᾶς ἐν αὐτῷ πρὸ καταβολῆς κόσμου εἶναι ἡμᾶς ἁγίους καὶ ἀμώμους κατενώπιον αὐτοῦ ἐν ἀγάπῃ,
\vs{5}προορίσας ἡμᾶς εἰς υἱοθεσίαν διὰ Ἰησοῦ Χριστοῦ εἰς αὐτόν, κατὰ τὴν εὐδοκίαν τοῦ θελήματος αὐτοῦ,
\vs{6}εἰς ἔπαινον δόξης τῆς χάριτος αὐτοῦ ἧς ἐχαρίτωσεν ἡμᾶς ἐν τῷ Ἠγαπημένῳ.
\vs{7}ἐν ᾧ ἔχομεν τὴν ἀπολύτρωσιν διὰ τοῦ αἵματος αὐτοῦ, τὴν ἄφεσιν τῶν παραπτωμάτων, κατὰ τὸ πλοῦτος τῆς χάριτος αὐτοῦ
\vs{8}ἧς ἐπερίσσευσεν εἰς ἡμᾶς, ἐν πάσῃ σοφίᾳ καὶ φρονήσει,
\vs{9}γνωρίσας ἡμῖν τὸ μυστήριον τοῦ θελήματος αὐτοῦ, κατὰ τὴν εὐδοκίαν αὐτοῦ ἣν προέθετο ἐν αὐτῷ
\vs{10}εἰς οἰκονομίαν τοῦ πληρώματος τῶν καιρῶν, ἀνακεφαλαιώσασθαι τὰ πάντα ἐν τῷ Χριστῷ, τὰ ἐπὶ τοῖς οὐρανοῖς καὶ τὰ ἐπὶ τῆς γῆς ἐν αὐτῷ.
\vs{11}ἐν ᾧ καὶ ἐκληρώθημεν προορισθέντες κατὰ πρόθεσιν τοῦ τὰ πάντα ἐνεργοῦντος κατὰ τὴν βουλὴν τοῦ θελήματος αὐτοῦ
\vs{12}εἰς τὸ εἶναι ἡμᾶς εἰς ἔπαινον δόξης αὐτοῦ τοὺς προηλπικότας ἐν τῷ Χριστῷ.
\vs{13}ἐν ᾧ καὶ ὑμεῖς ἀκούσαντες τὸν λόγον τῆς ἀληθείας, τὸ εὐαγγέλιον τῆς σωτηρίας ὑμῶν, ἐν ᾧ καὶ πιστεύσαντες ἐσφραγίσθητε τῷ Πνεύματι τῆς ἐπαγγελίας τῷ Ἁγίῳ,
\vs{14}ὅ ἐστιν ἀρραβὼν τῆς κληρονομίας ἡμῶν, εἰς ἀπολύτρωσιν τῆς περιποιήσεως, εἰς ἔπαινον τῆς δόξης αὐτοῦ.

\vs{15}Διὰ τοῦτο κἀγώ ἀκούσας τὴν καθ᾽ ὑμᾶς πίστιν ἐν τῷ Κυρίῳ Ἰησοῦ καὶ τὴν ἀγάπην τὴν εἰς πάντας τοὺς ἁγίους
\vs{16}οὐ παύομαι εὐχαριστῶν ὑπὲρ ὑμῶν μνείαν ποιούμενος ἐπὶ τῶν προσευχῶν μου,
\vs{17}ἵνα ὁ Θεὸς τοῦ Κυρίου ἡμῶν Ἰησοῦ Χριστοῦ, ὁ Πατὴρ τῆς δόξης, δώῃ ὑμῖν πνεῦμα σοφίας καὶ ἀποκαλύψεως ἐν ἐπιγνώσει αὐτοῦ,
\vs{18}πεφωτισμένους τοὺς ὀφθαλμοὺς τῆς καρδίας ὑμῶν εἰς τὸ εἰδέναι ὑμᾶς τίς ἐστιν ἡ ἐλπὶς τῆς κλήσεως αὐτοῦ, τίς ὁ πλοῦτος τῆς δόξης τῆς κληρονομίας αὐτοῦ ἐν τοῖς ἁγίοις,
\vs{19}καὶ τί τὸ ὑπερβάλλον μέγεθος τῆς δυνάμεως αὐτοῦ εἰς ἡμᾶς τοὺς πιστεύοντας κατὰ τὴν ἐνέργειαν τοῦ κράτους τῆς ἰσχύος αὐτοῦ.
\vs{20}ἣν ἐνήργηκεν ἐν τῷ Χριστῷ ἐγείρας αὐτὸν ἐκ νεκρῶν καὶ καθίσας ἐν δεξιᾷ αὐτοῦ ἐν τοῖς ἐπουρανίοις
\vs{21}ὑπεράνω πάσης ἀρχῆς καὶ ἐξουσίας καὶ δυνάμεως καὶ κυριότητος καὶ παντὸς ὀνόματος ὀνομαζομένου, οὐ μόνον ἐν τῷ αἰῶνι τούτῳ ἀλλὰ καὶ ἐν τῷ μέλλοντι·
\vs{22}καὶ πάντα ὑπέταξεν ὑπὸ τοὺς πόδας αὐτοῦ καὶ αὐτὸν ἔδωκεν κεφαλὴν ὑπὲρ πάντα τῇ ἐκκλησίᾳ,
\vs{23}ἥτις ἐστὶν τὸ σῶμα αὐτοῦ, τὸ πλήρωμα τοῦ τὰ πάντα ἐν πᾶσιν πληρουμένου.

\ch{2}
Καὶ ὑμᾶς ὄντας νεκροὺς τοῖς παραπτώμασιν καὶ ταῖς ἁμαρτίαις ὑμῶν,
\vs{2}ἐν αἷς ποτε περιεπατήσατε κατὰ τὸν αἰῶνα τοῦ κόσμου τούτου, κατὰ τὸν ἄρχοντα τῆς ἐξουσίας τοῦ ἀέρος, τοῦ πνεύματος τοῦ νῦν ἐνεργοῦντος ἐν τοῖς υἱοῖς τῆς ἀπειθείας·
\vs{3}ἐν οἷς καὶ ἡμεῖς πάντες ἀνεστράφημέν ποτε ἐν ταῖς ἐπιθυμίαις τῆς σαρκὸς ἡμῶν ποιοῦντες τὰ θελήματα τῆς σαρκὸς καὶ τῶν διανοιῶν, καὶ ἤμεθα τέκνα φύσει ὀργῆς ὡς καὶ οἱ λοιποί·
\vs{4}Ὁ δὲ Θεὸς πλούσιος ὢν ἐν ἐλέει, διὰ τὴν πολλὴν ἀγάπην αὐτοῦ ἣν ἠγάπησεν ἡμᾶς,
\vs{5}καὶ ὄντας ἡμᾶς νεκροὺς τοῖς παραπτώμασιν συνεζωοποίησεν τῷ Χριστῷ,— χάριτί ἐστε σεσῳσμένοι—
\vs{6}καὶ συνήγειρεν καὶ συνεκάθισεν ἐν τοῖς ἐπουρανίοις ἐν Χριστῷ Ἰησοῦ,
\vs{7}ἵνα ἐνδείξηται ἐν τοῖς αἰῶσιν τοῖς ἐπερχομένοις τὸ ὑπερβάλλον πλοῦτος τῆς χάριτος αὐτοῦ ἐν χρηστότητι ἐφ᾽ ἡμᾶς ἐν Χριστῷ Ἰησοῦ.
\vs{8}Τῇ γὰρ χάριτί ἐστε σεσῳσμένοι διὰ πίστεως· καὶ τοῦτο οὐκ ἐξ ὑμῶν, Θεοῦ τὸ δῶρον·
\vs{9}οὐκ ἐξ ἔργων, ἵνα μή τις καυχήσηται.
\vs{10}αὐτοῦ γάρ ἐσμεν ποίημα, κτισθέντες ἐν Χριστῷ Ἰησοῦ ἐπὶ ἔργοις ἀγαθοῖς οἷς προητοίμασεν ὁ Θεὸς, ἵνα ἐν αὐτοῖς περιπατήσωμεν.
\vs{11}Διὸ μνημονεύετε ὅτι ποτὲ ὑμεῖς τὰ ἔθνη ἐν σαρκί, οἱ λεγόμενοι ἀκροβυστία ὑπὸ τῆς λεγομένης περιτομῆς ἐν σαρκὶ χειροποιήτου,
\vs{12}ὅτι ἦτε τῷ καιρῷ ἐκείνῳ χωρὶς Χριστοῦ, ἀπηλλοτριωμένοι τῆς πολιτείας τοῦ Ἰσραὴλ καὶ ξένοι τῶν διαθηκῶν τῆς ἐπαγγελίας, ἐλπίδα μὴ ἔχοντες καὶ ἄθεοι ἐν τῷ κόσμῳ.
\vs{13}νυνὶ δὲ ἐν Χριστῷ Ἰησοῦ ὑμεῖς οἵ ποτε ὄντες μακρὰν ἐγενήθητε ἐγγὺς ἐν τῷ αἵματι τοῦ Χριστοῦ.
\vs{14}Αὐτὸς γάρ ἐστιν ἡ εἰρήνη ἡμῶν, ὁ ποιήσας τὰ ἀμφότερα ἓν καὶ τὸ μεσότοιχον τοῦ φραγμοῦ λύσας, τὴν ἔχθραν ἐν τῇ σαρκὶ αὐτοῦ,
\vs{15}τὸν νόμον τῶν ἐντολῶν ἐν δόγμασιν καταργήσας, ἵνα τοὺς δύο κτίσῃ ἐν αὑτῷ εἰς ἕνα καινὸν ἄνθρωπον ποιῶν εἰρήνην
\vs{16}καὶ ἀποκαταλλάξῃ τοὺς ἀμφοτέρους ἐν ἑνὶ σώματι τῷ Θεῷ διὰ τοῦ σταυροῦ, ἀποκτείνας τὴν ἔχθραν ἐν αὐτῷ.
\vs{17}Καὶ ἐλθὼν εὐηγγελίσατο εἰρήνην ὑμῖν τοῖς μακρὰν καὶ εἰρήνην τοῖς ἐγγύς·
\vs{18}ὅτι δι᾽ αὐτοῦ ἔχομεν τὴν προσαγωγὴν οἱ ἀμφότεροι ἐν ἑνὶ Πνεύματι πρὸς τὸν Πατέρα.
\vs{19}Ἄρα οὖν οὐκέτι ἐστὲ ξένοι καὶ πάροικοι ἀλλὰ ἐστὲ συμπολῖται τῶν ἁγίων καὶ οἰκεῖοι τοῦ Θεοῦ,
\vs{20}ἐποικοδομηθέντες ἐπὶ τῷ θεμελίῳ τῶν ἀποστόλων καὶ προφητῶν, ὄντος ἀκρογωνιαίου αὐτοῦ Χριστοῦ Ἰησοῦ,
\vs{21}ἐν ᾧ πᾶσα οἰκοδομὴ συναρμολογουμένη αὔξει εἰς ναὸν ἅγιον ἐν Κυρίῳ,
\vs{22}ἐν ᾧ καὶ ὑμεῖς συνοικοδομεῖσθε εἰς κατοικητήριον τοῦ Θεοῦ ἐν Πνεύματι.

\ch{3}
Τούτου χάριν ἐγὼ Παῦλος ὁ δέσμιος τοῦ Χριστοῦ Ἰησοῦ ὑπὲρ ὑμῶν τῶν ἐθνῶν—
\vs{2}Εἴ γε ἠκούσατε τὴν οἰκονομίαν τῆς χάριτος τοῦ Θεοῦ τῆς δοθείσης μοι εἰς ὑμᾶς,
\vs{3}ὅτι κατὰ ἀποκάλυψιν ἐγνωρίσθη μοι τὸ μυστήριον, καθὼς προέγραψα ἐν ὀλίγῳ,
\vs{4}πρὸς ὃ δύνασθε ἀναγινώσκοντες νοῆσαι τὴν σύνεσίν μου ἐν τῷ μυστηρίῳ τοῦ Χριστοῦ,
\vs{5}ὃ ἑτέραις γενεαῖς οὐκ ἐγνωρίσθη τοῖς υἱοῖς τῶν ἀνθρώπων ὡς νῦν ἀπεκαλύφθη τοῖς ἁγίοις ἀποστόλοις αὐτοῦ καὶ προφήταις ἐν Πνεύματι,
\vs{6}εἶναι τὰ ἔθνη συνκληρονόμα καὶ σύσσωμα καὶ συμμέτοχα τῆς ἐπαγγελίας ἐν Χριστῷ Ἰησοῦ διὰ τοῦ εὐαγγελίου,
\vs{7}οὗ ἐγενήθην διάκονος κατὰ τὴν δωρεὰν τῆς χάριτος τοῦ Θεοῦ τῆς δοθείσης μοι κατὰ τὴν ἐνέργειαν τῆς δυνάμεως αὐτοῦ.

\vs{8}Ἐμοὶ τῷ ἐλαχιστοτέρῳ πάντων ἁγίων ἐδόθη ἡ χάρις αὕτη, τοῖς ἔθνεσιν εὐαγγελίσασθαι τὸ ἀνεξιχνίαστον πλοῦτος τοῦ Χριστοῦ
\vs{9}καὶ φωτίσαι πάντας τίς ἡ οἰκονομία τοῦ μυστηρίου τοῦ ἀποκεκρυμμένου ἀπὸ τῶν αἰώνων ἐν τῷ Θεῷ τῷ τὰ πάντα κτίσαντι,
\vs{10}ἵνα γνωρισθῇ νῦν ταῖς ἀρχαῖς καὶ ταῖς ἐξουσίαις ἐν τοῖς ἐπουρανίοις διὰ τῆς ἐκκλησίας ἡ πολυποίκιλος σοφία τοῦ Θεοῦ,
\vs{11}κατὰ πρόθεσιν τῶν αἰώνων ἣν ἐποίησεν ἐν τῷ Χριστῷ Ἰησοῦ τῷ Κυρίῳ ἡμῶν,
\vs{12}ἐν ᾧ ἔχομεν τὴν παρρησίαν καὶ προσαγωγὴν ἐν πεποιθήσει διὰ τῆς πίστεως αὐτοῦ.
\vs{13}Διὸ αἰτοῦμαι μὴ ἐνκακεῖν ἐν ταῖς θλίψεσίν μου ὑπὲρ ὑμῶν, ἥτις ἐστὶν δόξα ὑμῶν.

\vs{14}Τούτου χάριν κάμπτω τὰ γόνατά μου πρὸς τὸν Πατέρα,
\vs{15}ἐξ οὗ πᾶσα πατριὰ ἐν οὐρανοῖς καὶ ἐπὶ γῆς ὀνομάζεται,
\vs{16}ἵνα δῷ ὑμῖν κατὰ τὸ πλοῦτος τῆς δόξης αὐτοῦ δυνάμει κραταιωθῆναι διὰ τοῦ Πνεύματος αὐτοῦ εἰς τὸν ἔσω ἄνθρωπον,
\vs{17}κατοικῆσαι τὸν Χριστὸν διὰ τῆς πίστεως ἐν ταῖς καρδίαις ὑμῶν, ἐν ἀγάπῃ ἐρριζωμένοι καὶ τεθεμελιωμένοι,
\vs{18}ἵνα ἐξισχύσητε καταλαβέσθαι σὺν πᾶσιν τοῖς ἁγίοις τί τὸ πλάτος καὶ μῆκος καὶ ὕψος καὶ βάθος,
\vs{19}γνῶναί τε τὴν ὑπερβάλλουσαν τῆς γνώσεως ἀγάπην τοῦ Χριστοῦ, ἵνα πληρωθῆτε εἰς πᾶν τὸ πλήρωμα τοῦ Θεοῦ.
\vs{20}Τῷ δὲ δυναμένῳ ὑπὲρ πάντα ποιῆσαι ὑπερεκπερισσοῦ ὧν αἰτούμεθα ἢ νοοῦμεν κατὰ τὴν δύναμιν τὴν ἐνεργουμένην ἐν ἡμῖν,
\vs{21}αὐτῷ ἡ δόξα ἐν τῇ ἐκκλησίᾳ καὶ ἐν Χριστῷ Ἰησοῦ εἰς πάσας τὰς γενεὰς τοῦ αἰῶνος τῶν αἰώνων, ἀμήν.

\ch{4}
Παρακαλῶ οὖν ὑμᾶς ἐγὼ ὁ δέσμιος ἐν Κυρίῳ ἀξίως περιπατῆσαι τῆς κλήσεως ἧς ἐκλήθητε,
\vs{2}μετὰ πάσης ταπεινοφροσύνης καὶ πραΰτητος, μετὰ μακροθυμίας, ἀνεχόμενοι ἀλλήλων ἐν ἀγάπῃ,
\vs{3}σπουδάζοντες τηρεῖν τὴν ἑνότητα τοῦ Πνεύματος ἐν τῷ συνδέσμῳ τῆς εἰρήνης·
\vs{4}ἓν σῶμα καὶ ἓν Πνεῦμα, καθὼς καὶ ἐκλήθητε ἐν μιᾷ ἐλπίδι τῆς κλήσεως ὑμῶν·
\begin{poetryblock}

\begin{quote} \vs{5}εἷς Κύριος, μία πίστις, ἓν βάπτισμα,\end{quote}

\begin{quote} \vs{6}εἷς Θεὸς καὶ Πατὴρ πάντων,\end{quote} 

\begin{quote}ὁ ἐπὶ πάντων καὶ διὰ πάντων καὶ ἐν πᾶσιν.\end{quote}
\end{poetryblock}
\vs{7}Ἑνὶ δὲ ἑκάστῳ ἡμῶν ἐδόθη ἡ χάρις κατὰ τὸ μέτρον τῆς δωρεᾶς τοῦ Χριστοῦ.
\vs{8}διὸ λέγει· 
\begin{poetryblock}

\begin{quote}Ἀναβὰς εἰς ὕψος ᾐχμαλώτευσεν αἰχμαλωσίαν,\end{quote} 

\begin{quote}ἔδωκεν δόματα τοῖς ἀνθρώποις.\end{quote}
\end{poetryblock}

\vs{9}Τὸ δὲ Ἀνέβη τί ἐστιν, εἰ μὴ ὅτι καὶ κατέβη εἰς τὰ κατώτερα μέρη τῆς γῆς;
\vs{10}ὁ καταβὰς αὐτός ἐστιν καὶ ὁ ἀναβὰς ὑπεράνω πάντων τῶν οὐρανῶν, ἵνα πληρώσῃ τὰ πάντα.
\vs{11}Καὶ αὐτὸς ἔδωκεν τοὺς μὲν ἀποστόλους, τοὺς δὲ προφήτας, τοὺς δὲ εὐαγγελιστάς, τοὺς δὲ ποιμένας καὶ διδασκάλους,
\vs{12}πρὸς τὸν καταρτισμὸν τῶν ἁγίων εἰς ἔργον διακονίας, εἰς οἰκοδομὴν τοῦ σώματος τοῦ Χριστοῦ,
\vs{13}μέχρι καταντήσωμεν οἱ πάντες εἰς τὴν ἑνότητα τῆς πίστεως καὶ τῆς ἐπιγνώσεως τοῦ Υἱοῦ τοῦ Θεοῦ, εἰς ἄνδρα τέλειον, εἰς μέτρον ἡλικίας τοῦ πληρώματος τοῦ Χριστοῦ,
\vs{14}ἵνα μηκέτι ὦμεν νήπιοι, κλυδωνιζόμενοι καὶ περιφερόμενοι παντὶ ἀνέμῳ τῆς διδασκαλίας ἐν τῇ κυβείᾳ τῶν ἀνθρώπων, ἐν πανουργίᾳ πρὸς τὴν μεθοδείαν τῆς πλάνης,
\vs{15}ἀληθεύοντες δὲ ἐν ἀγάπῃ αὐξήσωμεν εἰς αὐτὸν τὰ πάντα, ὅς ἐστιν ἡ κεφαλή, Χριστός,
\vs{16}ἐξ οὗ πᾶν τὸ σῶμα συναρμολογούμενον καὶ συμβιβαζόμενον διὰ πάσης ἁφῆς τῆς ἐπιχορηγίας κατ᾽ ἐνέργειαν ἐν μέτρῳ ἑνὸς ἑκάστου μέρους τὴν αὔξησιν τοῦ σώματος ποιεῖται εἰς οἰκοδομὴν ἑαυτοῦ ἐν ἀγάπῃ.

\vs{17}Τοῦτο οὖν λέγω καὶ μαρτύρομαι ἐν Κυρίῳ, μηκέτι ὑμᾶς περιπατεῖν, καθὼς καὶ τὰ ἔθνη περιπατεῖ ἐν ματαιότητι τοῦ νοὸς αὐτῶν,
\vs{18}ἐσκοτωμένοι τῇ διανοίᾳ ὄντες, ἀπηλλοτριωμένοι τῆς ζωῆς τοῦ Θεοῦ διὰ τὴν ἄγνοιαν τὴν οὖσαν ἐν αὐτοῖς, διὰ τὴν πώρωσιν τῆς καρδίας αὐτῶν,
\vs{19}οἵτινες ἀπηλγηκότες ἑαυτοὺς παρέδωκαν τῇ ἀσελγείᾳ εἰς ἐργασίαν ἀκαθαρσίας πάσης ἐν πλεονεξίᾳ.

\vs{20}Ὑμεῖς δὲ οὐχ οὕτως ἐμάθετε τὸν Χριστόν,
\vs{21}εἴ γε αὐτὸν ἠκούσατε καὶ ἐν αὐτῷ ἐδιδάχθητε, καθώς ἐστιν ἀλήθεια ἐν τῷ Ἰησοῦ,
\vs{22}ἀποθέσθαι ὑμᾶς κατὰ τὴν προτέραν ἀναστροφὴν τὸν παλαιὸν ἄνθρωπον τὸν φθειρόμενον κατὰ τὰς ἐπιθυμίας τῆς ἀπάτης,
\vs{23}ἀνανεοῦσθαι δὲ τῷ πνεύματι τοῦ νοὸς ὑμῶν
\vs{24}καὶ ἐνδύσασθαι τὸν καινὸν ἄνθρωπον τὸν κατὰ Θεὸν κτισθέντα ἐν δικαιοσύνῃ καὶ ὁσιότητι τῆς ἀληθείας.

\vs{25}Διὸ ἀποθέμενοι τὸ ψεῦδος λαλεῖτε ἀλήθειαν ἕκαστος μετὰ τοῦ πλησίον αὐτοῦ, ὅτι ἐσμὲν ἀλλήλων μέλη.
\vs{26}Ὀργίζεσθε καὶ μὴ ἁμαρτάνετε· ὁ ἥλιος μὴ ἐπιδυέτω ἐπὶ τῷ παροργισμῷ ὑμῶν,
\vs{27}μηδὲ δίδοτε τόπον τῷ διαβόλῳ.
\vs{28}Ὁ κλέπτων μηκέτι κλεπτέτω, μᾶλλον δὲ κοπιάτω ἐργαζόμενος ταῖς ἰδίαις χερσὶν τὸ ἀγαθόν, ἵνα ἔχῃ μεταδιδόναι τῷ χρείαν ἔχοντι.
\vs{29}Πᾶς λόγος σαπρὸς ἐκ τοῦ στόματος ὑμῶν μὴ ἐκπορευέσθω, ἀλλὰ εἴ τις ἀγαθὸς πρὸς οἰκοδομὴν τῆς χρείας, ἵνα δῷ χάριν τοῖς ἀκούουσιν.
\vs{30}Καὶ μὴ λυπεῖτε τὸ Πνεῦμα τὸ Ἅγιον τοῦ Θεοῦ, ἐν ᾧ ἐσφραγίσθητε εἰς ἡμέραν ἀπολυτρώσεως.
\vs{31}Πᾶσα πικρία καὶ θυμὸς καὶ ὀργὴ καὶ κραυγὴ καὶ βλασφημία ἀρθήτω ἀφ᾽ ὑμῶν σὺν πάσῃ κακίᾳ.
\vs{32}γίνεσθε δὲ εἰς ἀλλήλους χρηστοί, εὔσπλαγχνοι, χαριζόμενοι ἑαυτοῖς, καθὼς καὶ ὁ Θεὸς ἐν Χριστῷ ἐχαρίσατο ὑμῖν.

\ch{5}
Γίνεσθε οὖν μιμηταὶ τοῦ Θεοῦ ὡς τέκνα ἀγαπητά
\vs{2}καὶ περιπατεῖτε ἐν ἀγάπῃ, καθὼς καὶ ὁ Χριστὸς ἠγάπησεν ἡμᾶς καὶ παρέδωκεν ἑαυτὸν ὑπὲρ ἡμῶν προσφορὰν καὶ θυσίαν τῷ Θεῷ εἰς ὀσμὴν εὐωδίας.

\vs{3}Πορνεία δὲ καὶ ἀκαθαρσία πᾶσα ἢ πλεονεξία μηδὲ ὀνομαζέσθω ἐν ὑμῖν, καθὼς πρέπει ἁγίοις,
\vs{4}καὶ αἰσχρότης καὶ μωρολογία ἢ εὐτραπελία, ἃ οὐκ ἀνῆκεν, ἀλλὰ μᾶλλον εὐχαριστία.
\vs{5}τοῦτο γὰρ ἴστε γινώσκοντες, ὅτι πᾶς πόρνος ἢ ἀκάθαρτος ἢ πλεονέκτης, ὅ ἐστιν εἰδωλολάτρης, οὐκ ἔχει κληρονομίαν ἐν τῇ βασιλείᾳ τοῦ Χριστοῦ καὶ Θεοῦ.
\vs{6}Μηδεὶς ὑμᾶς ἀπατάτω κενοῖς λόγοις· διὰ ταῦτα γὰρ ἔρχεται ἡ ὀργὴ τοῦ Θεοῦ ἐπὶ τοὺς υἱοὺς τῆς ἀπειθείας.
\vs{7}μὴ οὖν γίνεσθε συμμέτοχοι αὐτῶν·
\vs{8}Ἦτε γάρ ποτε σκότος, νῦν δὲ φῶς ἐν Κυρίῳ· ὡς τέκνα φωτὸς περιπατεῖτε—
\vs{9}ὁ γὰρ καρπὸς τοῦ φωτὸς ἐν πάσῃ ἀγαθωσύνῃ καὶ δικαιοσύνῃ καὶ ἀληθείᾳ—
\vs{10}δοκιμάζοντες τί ἐστιν εὐάρεστον τῷ Κυρίῳ,
\vs{11}Καὶ μὴ συνκοινωνεῖτε τοῖς ἔργοις τοῖς ἀκάρποις τοῦ σκότους, μᾶλλον δὲ καὶ ἐλέγχετε.
\vs{12}τὰ γὰρ κρυφῇ γινόμενα ὑπ᾽ αὐτῶν αἰσχρόν ἐστιν καὶ λέγειν,
\vs{13}τὰ δὲ πάντα ἐλεγχόμενα ὑπὸ τοῦ φωτὸς φανεροῦται,
\vs{14}πᾶν γὰρ τὸ φανερούμενον φῶς ἐστιν. διὸ λέγει· 
\begin{poetryblock}

\begin{quote}Ἔγειρε, ὁ καθεύδων,\end{quote} 

\begin{quote}καὶ ἀνάστα ἐκ τῶν νεκρῶν,\end{quote} 

\begin{quote}καὶ ἐπιφαύσει σοι ὁ Χριστός.\end{quote}
\end{poetryblock}

\vs{15}Βλέπετε οὖν ἀκριβῶς πῶς περιπατεῖτε μὴ ὡς ἄσοφοι ἀλλ᾽ ὡς σοφοί,
\vs{16}ἐξαγοραζόμενοι τὸν καιρόν, ὅτι αἱ ἡμέραι πονηραί εἰσιν.
\vs{17}διὰ τοῦτο μὴ γίνεσθε ἄφρονες, ἀλλὰ συνίετε τί τὸ θέλημα τοῦ Κυρίου.
\vs{18}καὶ μὴ μεθύσκεσθε οἴνῳ, ἐν ᾧ ἐστιν ἀσωτία, ἀλλὰ πληροῦσθε ἐν Πνεύματι,
\vs{19}λαλοῦντες ἑαυτοῖς ἐν ψαλμοῖς καὶ ὕμνοις καὶ ᾠδαῖς πνευματικαῖς, ᾄδοντες καὶ ψάλλοντες τῇ καρδίᾳ ὑμῶν τῷ Κυρίῳ,
\vs{20}εὐχαριστοῦντες πάντοτε ὑπὲρ πάντων ἐν ὀνόματι τοῦ Κυρίου ἡμῶν Ἰησοῦ Χριστοῦ τῷ Θεῷ καὶ Πατρί.
\vs{21}Ὑποτασσόμενοι ἀλλήλοις ἐν φόβῳ Χριστοῦ,
\vs{22}Αἱ γυναῖκες τοῖς ἰδίοις ἀνδράσιν ὡς τῷ Κυρίῳ,
\vs{23}ὅτι ἀνήρ ἐστιν κεφαλὴ τῆς γυναικὸς ὡς καὶ ὁ Χριστὸς κεφαλὴ τῆς ἐκκλησίας, αὐτὸς σωτὴρ τοῦ σώματος·
\vs{24}ἀλλὰ ὡς ἡ ἐκκλησία ὑποτάσσεται τῷ Χριστῷ, οὕτως καὶ αἱ γυναῖκες τοῖς ἀνδράσιν ἐν παντί.

\vs{25}Οἱ ἄνδρες, ἀγαπᾶτε τὰς γυναῖκας, καθὼς καὶ ὁ Χριστὸς ἠγάπησεν τὴν ἐκκλησίαν καὶ ἑαυτὸν παρέδωκεν ὑπὲρ αὐτῆς,
\vs{26}ἵνα αὐτὴν ἁγιάσῃ καθαρίσας τῷ λουτρῷ τοῦ ὕδατος ἐν ῥήματι,
\vs{27}ἵνα παραστήσῃ αὐτὸς ἑαυτῷ ἔνδοξον τὴν ἐκκλησίαν, μὴ ἔχουσαν σπίλον ἢ ῥυτίδα ἤ τι τῶν τοιούτων, ἀλλ᾽ ἵνα ᾖ ἁγία καὶ ἄμωμος.
\vs{28}Οὕτως ὀφείλουσιν καὶ οἱ ἄνδρες ἀγαπᾶν τὰς ἑαυτῶν γυναῖκας ὡς τὰ ἑαυτῶν σώματα. ὁ ἀγαπῶν τὴν ἑαυτοῦ γυναῖκα ἑαυτὸν ἀγαπᾷ.
\vs{29}οὐδεὶς γάρ ποτε τὴν ἑαυτοῦ σάρκα ἐμίσησεν ἀλλὰ ἐκτρέφει καὶ θάλπει αὐτήν, καθὼς καὶ ὁ Χριστὸς τὴν ἐκκλησίαν,
\vs{30}ὅτι μέλη ἐσμὲν τοῦ σώματος αὐτοῦ.
\vs{31}Ἀντὶ τούτου καταλείψει ἄνθρωπος τὸν πατέρα καὶ τὴν μητέρα καὶ προσκολληθήσεται πρὸς τὴν γυναῖκα αὐτοῦ, καὶ ἔσονται οἱ δύο εἰς σάρκα μίαν.
\vs{32}τὸ μυστήριον τοῦτο μέγα ἐστίν· ἐγὼ δὲ λέγω εἰς Χριστὸν καὶ εἰς τὴν ἐκκλησίαν.
\vs{33}πλὴν καὶ ὑμεῖς οἱ καθ᾽ ἕνα, ἕκαστος τὴν ἑαυτοῦ γυναῖκα οὕτως ἀγαπάτω ὡς ἑαυτόν, ἡ δὲ γυνὴ ἵνα φοβῆται τὸν ἄνδρα.

\ch{6}
Τὰ τέκνα, ὑπακούετε τοῖς γονεῦσιν ὑμῶν ἐν Κυρίῳ· τοῦτο γάρ ἐστιν δίκαιον.
\vs{2}Τίμα τὸν πατέρα σου καὶ τὴν μητέρα, ἥτις ἐστὶν ἐντολὴ πρώτη ἐν ἐπαγγελίᾳ,
\vs{3}Ἵνα εὖ σοι γένηται καὶ ἔσῃ μακροχρόνιος ἐπὶ τῆς γῆς.
\vs{4}Καὶ οἱ πατέρες, μὴ παροργίζετε τὰ τέκνα ὑμῶν ἀλλὰ ἐκτρέφετε αὐτὰ ἐν παιδείᾳ καὶ νουθεσίᾳ Κυρίου.

\vs{5}Οἱ δοῦλοι, ὑπακούετε τοῖς κατὰ σάρκα κυρίοις μετὰ φόβου καὶ τρόμου ἐν ἁπλότητι τῆς καρδίας ὑμῶν ὡς τῷ Χριστῷ,
\vs{6}μὴ κατ᾽ ὀφθαλμοδουλίαν ὡς ἀνθρωπάρεσκοι ἀλλ᾽ ὡς δοῦλοι Χριστοῦ ποιοῦντες τὸ θέλημα τοῦ Θεοῦ ἐκ ψυχῆς,
\vs{7}μετ᾽ εὐνοίας δουλεύοντες ὡς τῷ Κυρίῳ καὶ οὐκ ἀνθρώποις,
\vs{8}εἰδότες ὅτι ἕκαστος ἐάν τι ποιήσῃ ἀγαθόν, τοῦτο κομίσεται παρὰ Κυρίου εἴτε δοῦλος εἴτε ἐλεύθερος.
\vs{9}Καὶ οἱ κύριοι, τὰ αὐτὰ ποιεῖτε πρὸς αὐτούς, ἀνιέντες τὴν ἀπειλήν, εἰδότες ὅτι καὶ αὐτῶν καὶ ὑμῶν ὁ Κύριός ἐστιν ἐν οὐρανοῖς καὶ προσωπολημψία οὐκ ἔστιν παρ᾽ αὐτῷ.

\vs{10}Τοῦ λοιποῦ, ἐνδυναμοῦσθε ἐν Κυρίῳ καὶ ἐν τῷ κράτει τῆς ἰσχύος αὐτοῦ.
\vs{11}ἐνδύσασθε τὴν πανοπλίαν τοῦ Θεοῦ πρὸς τὸ δύνασθαι ὑμᾶς στῆναι πρὸς τὰς μεθοδείας τοῦ διαβόλου·
\vs{12}ὅτι οὐκ ἔστιν ἡμῖν ἡ πάλη πρὸς αἷμα καὶ σάρκα ἀλλὰ πρὸς τὰς ἀρχάς, πρὸς τὰς ἐξουσίας, πρὸς τοὺς κοσμοκράτορας τοῦ σκότους τούτου, πρὸς τὰ πνευματικὰ τῆς πονηρίας ἐν τοῖς ἐπουρανίοις.
\vs{13}Διὰ τοῦτο ἀναλάβετε τὴν πανοπλίαν τοῦ Θεοῦ, ἵνα δυνηθῆτε ἀντιστῆναι ἐν τῇ ἡμέρᾳ τῇ πονηρᾷ καὶ ἅπαντα κατεργασάμενοι στῆναι.
\vs{14}στῆτε οὖν περιζωσάμενοι τὴν ὀσφὺν ὑμῶν ἐν ἀληθείᾳ καὶ ἐνδυσάμενοι τὸν θώρακα τῆς δικαιοσύνης
\vs{15}καὶ ὑποδησάμενοι τοὺς πόδας ἐν ἑτοιμασίᾳ τοῦ εὐαγγελίου τῆς εἰρήνης,
\vs{16}ἐν πᾶσιν ἀναλαβόντες τὸν θυρεὸν τῆς πίστεως, ἐν ᾧ δυνήσεσθε πάντα τὰ βέλη τοῦ πονηροῦ τὰ πεπυρωμένα σβέσαι·
\vs{17}καὶ τὴν περικεφαλαίαν τοῦ σωτηρίου δέξασθε καὶ τὴν μάχαιραν τοῦ Πνεύματος, ὅ ἐστιν ῥῆμα Θεοῦ.
\vs{18}διὰ πάσης προσευχῆς καὶ δεήσεως προσευχόμενοι ἐν παντὶ καιρῷ ἐν Πνεύματι, καὶ εἰς αὐτὸ ἀγρυπνοῦντες ἐν πάσῃ προσκαρτερήσει καὶ δεήσει περὶ πάντων τῶν ἁγίων
\vs{19}καὶ ὑπὲρ ἐμοῦ, ἵνα μοι δοθῇ λόγος ἐν ἀνοίξει τοῦ στόματός μου, ἐν παρρησίᾳ γνωρίσαι τὸ μυστήριον τοῦ εὐαγγελίου,
\vs{20}ὑπὲρ οὗ πρεσβεύω ἐν ἁλύσει, ἵνα ἐν αὐτῷ παρρησιάσωμαι ὡς δεῖ με λαλῆσαι.

\vs{21}Ἵνα δὲ εἰδῆτε καὶ ὑμεῖς τὰ κατ᾽ ἐμέ, τί πράσσω, πάντα γνωρίσει ὑμῖν Τυχικὸς ὁ ἀγαπητὸς ἀδελφὸς καὶ πιστὸς διάκονος ἐν Κυρίῳ,
\vs{22}ὃν ἔπεμψα πρὸς ὑμᾶς εἰς αὐτὸ τοῦτο, ἵνα γνῶτε τὰ περὶ ἡμῶν καὶ παρακαλέσῃ τὰς καρδίας ὑμῶν.

\vs{23}Εἰρήνη τοῖς ἀδελφοῖς καὶ ἀγάπη μετὰ πίστεως ἀπὸ Θεοῦ Πατρὸς καὶ Κυρίου Ἰησοῦ Χριστοῦ.
\vs{24}Ἡ χάρις μετὰ πάντων τῶν ἀγαπώντων τὸν Κύριον ἡμῶν Ἰησοῦν Χριστὸν ἐν ἀφθαρσίᾳ.


\def\book{ΠΡΟΣ ΦΙΛΙΠΠΗΣΙΟΥΣ}
\biblebook{ΠΡΟΣ ΦΙΛΙΠΠΗΣΙΟΥΣ}


\lettrine[lines=2, loversize=0.2, nindent=0em, findent=.25em]{\textcolor{bookheadingcolor}{Π}}{αῦλος} καὶ Τιμόθεος δοῦλοι Χριστοῦ Ἰησοῦ Πᾶσιν τοῖς ἁγίοις ἐν Χριστῷ Ἰησοῦ τοῖς οὖσιν ἐν Φιλίπποις σὺν ἐπισκόποις καὶ διακόνοις,
\vs{2}Χάρις ὑμῖν καὶ εἰρήνη ἀπὸ Θεοῦ Πατρὸς ἡμῶν καὶ Κυρίου Ἰησοῦ Χριστοῦ.

\vs{3}Εὐχαριστῶ τῷ Θεῷ μου ἐπὶ πάσῃ τῇ μνείᾳ ὑμῶν
\vs{4}πάντοτε ἐν πάσῃ δεήσει μου ὑπὲρ πάντων ὑμῶν, μετὰ χαρᾶς τὴν δέησιν ποιούμενος,
\vs{5}ἐπὶ τῇ κοινωνίᾳ ὑμῶν εἰς τὸ εὐαγγέλιον ἀπὸ τῆς πρώτης ἡμέρας ἄχρι τοῦ νῦν,
\vs{6}πεποιθὼς αὐτὸ τοῦτο, ὅτι ὁ ἐναρξάμενος ἐν ὑμῖν ἔργον ἀγαθὸν ἐπιτελέσει ἄχρι ἡμέρας Χριστοῦ Ἰησοῦ·
\vs{7}Καθώς ἐστιν δίκαιον ἐμοὶ τοῦτο φρονεῖν ὑπὲρ πάντων ὑμῶν διὰ τὸ ἔχειν με ἐν τῇ καρδίᾳ ὑμᾶς, ἔν τε τοῖς δεσμοῖς μου καὶ ἐν τῇ ἀπολογίᾳ καὶ βεβαιώσει τοῦ εὐαγγελίου συνκοινωνούς μου τῆς χάριτος πάντας ὑμᾶς ὄντας.
\vs{8}μάρτυς γάρ μου ὁ Θεός ὡς ἐπιποθῶ πάντας ὑμᾶς ἐν σπλάγχνοις Χριστοῦ Ἰησοῦ.
\vs{9}Καὶ τοῦτο προσεύχομαι, ἵνα ἡ ἀγάπη ὑμῶν ἔτι μᾶλλον καὶ μᾶλλον περισσεύῃ ἐν ἐπιγνώσει καὶ πάσῃ αἰσθήσει
\vs{10}εἰς τὸ δοκιμάζειν ὑμᾶς τὰ διαφέροντα, ἵνα ἦτε εἰλικρινεῖς καὶ ἀπρόσκοποι εἰς ἡμέραν Χριστοῦ,
\vs{11}πεπληρωμένοι καρπὸν δικαιοσύνης τὸν διὰ Ἰησοῦ Χριστοῦ εἰς δόξαν καὶ ἔπαινον Θεοῦ.

\vs{12}Γινώσκειν δὲ ὑμᾶς βούλομαι, ἀδελφοί, ὅτι τὰ κατ᾽ ἐμὲ μᾶλλον εἰς προκοπὴν τοῦ εὐαγγελίου ἐλήλυθεν,
\vs{13}ὥστε τοὺς δεσμούς μου φανεροὺς ἐν Χριστῷ γενέσθαι ἐν ὅλῳ τῷ πραιτωρίῳ καὶ τοῖς λοιποῖς πᾶσιν,
\vs{14}καὶ τοὺς πλείονας τῶν ἀδελφῶν ἐν Κυρίῳ πεποιθότας τοῖς δεσμοῖς μου περισσοτέρως τολμᾶν ἀφόβως τὸν λόγον λαλεῖν.
\vs{15}Τινὲς μὲν καὶ διὰ φθόνον καὶ ἔριν, τινὲς δὲ καὶ δι᾽ εὐδοκίαν τὸν Χριστὸν κηρύσσουσιν·
\vs{16}οἱ μὲν ἐξ ἀγάπης, εἰδότες ὅτι εἰς ἀπολογίαν τοῦ εὐαγγελίου κεῖμαι,
\vs{17}οἱ δὲ ἐξ ἐριθείας τὸν Χριστὸν καταγγέλλουσιν, οὐχ ἁγνῶς, οἰόμενοι θλῖψιν ἐγείρειν τοῖς δεσμοῖς μου.
\vs{18}Τί γάρ; πλὴν ὅτι παντὶ τρόπῳ, εἴτε προφάσει εἴτε ἀληθείᾳ, Χριστὸς καταγγέλλεται, καὶ ἐν τούτῳ χαίρω.

ἀλλὰ καὶ χαρήσομαι,
\vs{19}οἶδα γὰρ ὅτι τοῦτό μοι ἀποβήσεται εἰς σωτηρίαν διὰ τῆς ὑμῶν δεήσεως καὶ ἐπιχορηγίας τοῦ Πνεύματος Ἰησοῦ Χριστοῦ
\vs{20}κατὰ τὴν ἀποκαραδοκίαν καὶ ἐλπίδα μου, ὅτι ἐν οὐδενὶ αἰσχυνθήσομαι ἀλλ᾽ ἐν πάσῃ παρρησίᾳ ὡς πάντοτε καὶ νῦν μεγαλυνθήσεται Χριστὸς ἐν τῷ σώματί μου, εἴτε διὰ ζωῆς εἴτε διὰ θανάτου.
\vs{21}Ἐμοὶ γὰρ τὸ ζῆν Χριστὸς καὶ τὸ ἀποθανεῖν κέρδος.
\vs{22}εἰ δὲ τὸ ζῆν ἐν σαρκί, τοῦτό μοι καρπὸς ἔργου, καὶ τί αἱρήσομαι οὐ γνωρίζω.
\vs{23}συνέχομαι δὲ ἐκ τῶν δύο, τὴν ἐπιθυμίαν ἔχων εἰς τὸ ἀναλῦσαι καὶ σὺν Χριστῷ εἶναι, πολλῷ γὰρ μᾶλλον κρεῖσσον·
\vs{24}τὸ δὲ ἐπιμένειν ἐν τῇ σαρκὶ ἀναγκαιότερον δι᾽ ὑμᾶς.
\vs{25}Καὶ τοῦτο πεποιθὼς οἶδα ὅτι μενῶ καὶ παραμενῶ πᾶσιν ὑμῖν εἰς τὴν ὑμῶν προκοπὴν καὶ χαρὰν τῆς πίστεως,
\vs{26}ἵνα τὸ καύχημα ὑμῶν περισσεύῃ ἐν Χριστῷ Ἰησοῦ ἐν ἐμοὶ διὰ τῆς ἐμῆς παρουσίας πάλιν πρὸς ὑμᾶς.

\vs{27}Μόνον ἀξίως τοῦ εὐαγγελίου τοῦ Χριστοῦ πολιτεύεσθε, ἵνα εἴτε ἐλθὼν καὶ ἰδὼν ὑμᾶς εἴτε ἀπὼν ἀκούω τὰ περὶ ὑμῶν, ὅτι στήκετε ἐν ἑνὶ πνεύματι, μιᾷ ψυχῇ συναθλοῦντες τῇ πίστει τοῦ εὐαγγελίου
\vs{28}καὶ μὴ πτυρόμενοι ἐν μηδενὶ ὑπὸ τῶν ἀντικειμένων, ἥτις ἐστὶν αὐτοῖς ἔνδειξις ἀπωλείας, ὑμῶν δὲ σωτηρίας, καὶ τοῦτο ἀπὸ Θεοῦ·
\vs{29}ὅτι ὑμῖν ἐχαρίσθη τὸ ὑπὲρ Χριστοῦ, οὐ μόνον τὸ εἰς αὐτὸν πιστεύειν ἀλλὰ καὶ τὸ ὑπὲρ αὐτοῦ πάσχειν,
\vs{30}τὸν αὐτὸν ἀγῶνα ἔχοντες, οἷον εἴδετε ἐν ἐμοὶ καὶ νῦν ἀκούετε ἐν ἐμοί.

\ch{2}
Εἴ τις οὖν παράκλησις ἐν Χριστῷ, εἴ τι παραμύθιον ἀγάπης, εἴ τις κοινωνία Πνεύματος, εἴ τις σπλάγχνα καὶ οἰκτιρμοί,
\vs{2}πληρώσατέ μου τὴν χαρὰν ἵνα τὸ αὐτὸ φρονῆτε, τὴν αὐτὴν ἀγάπην ἔχοντες, σύμψυχοι, τὸ ἓν φρονοῦντες,
\vs{3}μηδὲν κατ᾽ ἐριθείαν μηδὲ κατὰ κενοδοξίαν ἀλλὰ τῇ ταπεινοφροσύνῃ ἀλλήλους ἡγούμενοι ὑπερέχοντας ἑαυτῶν,
\vs{4}μὴ τὰ ἑαυτῶν ἕκαστος σκοποῦντες ἀλλὰ καὶ τὰ ἑτέρων ἕκαστοι.

\vs{5}Τοῦτο φρονεῖτε ἐν ὑμῖν ὃ καὶ ἐν Χριστῷ Ἰησοῦ,
\begin{poetryblock}

\begin{quote} \vs{6}Ὃς ἐν μορφῇ Θεοῦ ὑπάρχων\end{quote} 

\begin{quote}οὐχ ἁρπαγμὸν ἡγήσατο\end{quote} 

\begin{quote}τὸ εἶναι ἴσα Θεῷ,\end{quote}

\begin{quote} \vs{7}ἀλλὰ ἑαυτὸν ἐκένωσεν\end{quote} 

\begin{quote}μορφὴν δούλου λαβών,\end{quote} 

\begin{quote}ἐν ὁμοιώματι ἀνθρώπων γενόμενος·\end{quote} 

\begin{quote}καὶ σχήματι εὑρεθεὶς ὡς ἄνθρωπος\end{quote}

\begin{quote} \vs{8}ἐταπείνωσεν ἑαυτὸν\end{quote} 

\begin{quote}γενόμενος ὑπήκοος μέχρι θανάτου,\end{quote} 

\begin{quote}θανάτου δὲ σταυροῦ.\end{quote}

\begin{quote} \vs{9}Διὸ καὶ ὁ Θεὸς αὐτὸν ὑπερύψωσεν\end{quote} 

\begin{quote}καὶ ἐχαρίσατο αὐτῷ τὸ ὄνομα\end{quote} 

\begin{quote}τὸ ὑπὲρ πᾶν ὄνομα,\end{quote}

\begin{quote} \vs{10}ἵνα ἐν τῷ ὀνόματι Ἰησοῦ\end{quote} 

\begin{quote}πᾶν γόνυ κάμψῃ\end{quote} 

\begin{quote}ἐπουρανίων καὶ ἐπιγείων καὶ καταχθονίων\end{quote}

\begin{quote} \vs{11}καὶ πᾶσα γλῶσσα ἐξομολογήσηται ὅτι\end{quote} 

\begin{quote}ΚΥΡΙΟΣ ΙΗΣΟΥΣ ΧΡΙΣΤΟΣ\end{quote} 

\begin{quote}εἰς δόξαν Θεοῦ Πατρός.\end{quote}
\end{poetryblock}

\vs{12}Ὥστε, ἀγαπητοί μου, καθὼς πάντοτε ὑπηκούσατε, μὴ ὡς ἐν τῇ παρουσίᾳ μου μόνον ἀλλὰ νῦν πολλῷ μᾶλλον ἐν τῇ ἀπουσίᾳ μου, μετὰ φόβου καὶ τρόμου τὴν ἑαυτῶν σωτηρίαν κατεργάζεσθε·
\vs{13}Θεὸς γάρ ἐστιν ὁ ἐνεργῶν ἐν ὑμῖν καὶ τὸ θέλειν καὶ τὸ ἐνεργεῖν ὑπὲρ τῆς εὐδοκίας.
\vs{14}Πάντα ποιεῖτε χωρὶς γογγυσμῶν καὶ διαλογισμῶν,
\vs{15}ἵνα γένησθε ἄμεμπτοι καὶ ἀκέραιοι, τέκνα Θεοῦ ἄμωμα μέσον γενεᾶς σκολιᾶς καὶ διεστραμμένης, ἐν οἷς φαίνεσθε ὡς φωστῆρες ἐν κόσμῳ,
\vs{16}λόγον ζωῆς ἐπέχοντες, εἰς καύχημα ἐμοὶ εἰς ἡμέραν Χριστοῦ, ὅτι οὐκ εἰς κενὸν ἔδραμον οὐδὲ εἰς κενὸν ἐκοπίασα.
\vs{17}Ἀλλὰ εἰ καὶ σπένδομαι ἐπὶ τῇ θυσίᾳ καὶ λειτουργίᾳ τῆς πίστεως ὑμῶν, χαίρω καὶ συνχαίρω πᾶσιν ὑμῖν·
\vs{18}τὸ δὲ αὐτὸ καὶ ὑμεῖς χαίρετε καὶ συνχαίρετέ μοι.

\vs{19}Ἐλπίζω δὲ ἐν Κυρίῳ Ἰησοῦ Τιμόθεον ταχέως πέμψαι ὑμῖν, ἵνα κἀγὼ εὐψυχῶ γνοὺς τὰ περὶ ὑμῶν.
\vs{20}οὐδένα γὰρ ἔχω ἰσόψυχον, ὅστις γνησίως τὰ περὶ ὑμῶν μεριμνήσει·
\vs{21}οἱ πάντες γὰρ τὰ ἑαυτῶν ζητοῦσιν, οὐ τὰ Ἰησοῦ Χριστοῦ.
\vs{22}τὴν δὲ δοκιμὴν αὐτοῦ γινώσκετε, ὅτι ὡς πατρὶ τέκνον σὺν ἐμοὶ ἐδούλευσεν εἰς τὸ εὐαγγέλιον.
\vs{23}Τοῦτον μὲν οὖν ἐλπίζω πέμψαι ὡς ἂν ἀφίδω τὰ περὶ ἐμὲ ἐξαυτῆς·
\vs{24}πέποιθα δὲ ἐν Κυρίῳ ὅτι καὶ αὐτὸς ταχέως ἐλεύσομαι.
\vs{25}Ἀναγκαῖον δὲ ἡγησάμην Ἐπαφρόδιτον τὸν ἀδελφὸν καὶ συνεργὸν καὶ συστρατιώτην μου, ὑμῶν δὲ ἀπόστολον καὶ λειτουργὸν τῆς χρείας μου, πέμψαι πρὸς ὑμᾶς,
\vs{26}ἐπειδὴ ἐπιποθῶν ἦν πάντας ὑμᾶς καὶ ἀδημονῶν, διότι ἠκούσατε ὅτι ἠσθένησεν.
\vs{27}καὶ γὰρ ἠσθένησεν παραπλήσιον θανάτῳ· ἀλλὰ ὁ Θεὸς ἠλέησεν αὐτόν, οὐκ αὐτὸν δὲ μόνον ἀλλὰ καὶ ἐμέ, ἵνα μὴ λύπην ἐπὶ λύπην σχῶ.
\vs{28}Σπουδαιοτέρως οὖν ἔπεμψα αὐτὸν, ἵνα ἰδόντες αὐτὸν πάλιν χαρῆτε κἀγὼ ἀλυπότερος ὦ.
\vs{29}προσδέχεσθε οὖν αὐτὸν ἐν Κυρίῳ μετὰ πάσης χαρᾶς καὶ τοὺς τοιούτους ἐντίμους ἔχετε,
\vs{30}ὅτι διὰ τὸ ἔργον Χριστοῦ μέχρι θανάτου ἤγγισεν παραβολευσάμενος τῇ ψυχῇ, ἵνα ἀναπληρώσῃ τὸ ὑμῶν ὑστέρημα τῆς πρός με λειτουργίας.

\ch{3}
Τὸ λοιπόν, ἀδελφοί μου, χαίρετε ἐν Κυρίῳ. τὰ αὐτὰ γράφειν ὑμῖν ἐμοὶ μὲν οὐκ ὀκνηρόν, ὑμῖν δὲ ἀσφαλές.

\vs{2}Βλέπετε τοὺς κύνας, βλέπετε τοὺς κακοὺς ἐργάτας, βλέπετε τὴν κατατομήν.
\vs{3}ἡμεῖς γάρ ἐσμεν ἡ περιτομή, οἱ Πνεύματι Θεοῦ λατρεύοντες καὶ καυχώμενοι ἐν Χριστῷ Ἰησοῦ καὶ οὐκ ἐν σαρκὶ πεποιθότες,
\vs{4}καίπερ ἐγὼ ἔχων πεποίθησιν καὶ ἐν σαρκί. Εἴ τις δοκεῖ ἄλλος πεποιθέναι ἐν σαρκί, ἐγὼ μᾶλλον·
\vs{5}περιτομῇ ὀκταήμερος, ἐκ γένους Ἰσραήλ, φυλῆς Βενιαμίν, Ἑβραῖος ἐξ Ἑβραίων, κατὰ νόμον Φαρισαῖος,
\vs{6}κατὰ ζῆλος διώκων τὴν ἐκκλησίαν, κατὰ δικαιοσύνην τὴν ἐν νόμῳ γενόμενος ἄμεμπτος.
\vs{7}Ἀλλὰ ἅτινα ἦν μοι κέρδη, ταῦτα ἥγημαι διὰ τὸν Χριστὸν ζημίαν.
\vs{8}ἀλλὰ μενοῦνγε καὶ ἡγοῦμαι πάντα ζημίαν εἶναι διὰ τὸ ὑπερέχον τῆς γνώσεως Χριστοῦ Ἰησοῦ τοῦ Κυρίου μου, δι᾽ ὃν τὰ πάντα ἐζημιώθην, καὶ ἡγοῦμαι σκύβαλα, ἵνα Χριστὸν κερδήσω
\vs{9}καὶ εὑρεθῶ ἐν αὐτῷ, μὴ ἔχων ἐμὴν δικαιοσύνην τὴν ἐκ νόμου ἀλλὰ τὴν διὰ πίστεως Χριστοῦ, τὴν ἐκ Θεοῦ δικαιοσύνην ἐπὶ τῇ πίστει,
\vs{10}τοῦ γνῶναι αὐτὸν καὶ τὴν δύναμιν τῆς ἀναστάσεως αὐτοῦ καὶ τὴν κοινωνίαν τῶν παθημάτων αὐτοῦ, συμμορφιζόμενος τῷ θανάτῳ αὐτοῦ,
\vs{11}εἴ πως καταντήσω εἰς τὴν ἐξανάστασιν τὴν ἐκ νεκρῶν.

\vs{12}Οὐχ ὅτι ἤδη ἔλαβον ἢ ἤδη τετελείωμαι, διώκω δὲ εἰ καὶ καταλάβω, ἐφ᾽ ᾧ καὶ κατελήμφθην ὑπὸ Χριστοῦ Ἰησοῦ.
\vs{13}ἀδελφοί, ἐγὼ ἐμαυτὸν οὐ λογίζομαι κατειληφέναι· ἓν δέ, τὰ μὲν ὀπίσω ἐπιλανθανόμενος τοῖς δὲ ἔμπροσθεν ἐπεκτεινόμενος,
\vs{14}κατὰ σκοπὸν διώκω εἰς τὸ βραβεῖον τῆς ἄνω κλήσεως τοῦ Θεοῦ ἐν Χριστῷ Ἰησοῦ.
\vs{15}Ὅσοι οὖν τέλειοι, τοῦτο φρονῶμεν· καὶ εἴ τι ἑτέρως φρονεῖτε, καὶ τοῦτο ὁ Θεὸς ὑμῖν ἀποκαλύψει·
\vs{16}πλὴν εἰς ὃ ἐφθάσαμεν, τῷ αὐτῷ στοιχεῖν.

\vs{17}Συμμιμηταί μου γίνεσθε, ἀδελφοί, καὶ σκοπεῖτε τοὺς οὕτω περιπατοῦντας καθὼς ἔχετε τύπον ἡμᾶς.
\vs{18}πολλοὶ γὰρ περιπατοῦσιν οὓς πολλάκις ἔλεγον ὑμῖν, νῦν δὲ καὶ κλαίων λέγω, τοὺς ἐχθροὺς τοῦ σταυροῦ τοῦ Χριστοῦ,
\vs{19}ὧν τὸ τέλος ἀπώλεια, ὧν ὁ θεὸς ἡ κοιλία καὶ ἡ δόξα ἐν τῇ αἰσχύνῃ αὐτῶν, οἱ τὰ ἐπίγεια φρονοῦντες.
\vs{20}Ἡμῶν γὰρ τὸ πολίτευμα ἐν οὐρανοῖς ὑπάρχει, ἐξ οὗ καὶ Σωτῆρα ἀπεκδεχόμεθα Κύριον Ἰησοῦν Χριστόν,
\vs{21}ὃς μετασχηματίσει τὸ σῶμα τῆς ταπεινώσεως ἡμῶν σύμμορφον τῷ σώματι τῆς δόξης αὐτοῦ κατὰ τὴν ἐνέργειαν τοῦ δύνασθαι αὐτὸν καὶ ὑποτάξαι αὑτῷ τὰ πάντα.

\ch{4}
Ὥστε, ἀδελφοί μου ἀγαπητοὶ καὶ ἐπιπόθητοι, χαρὰ καὶ στέφανός μου, οὕτως στήκετε ἐν Κυρίῳ, ἀγαπητοί.

\vs{2}Εὐοδίαν παρακαλῶ καὶ Συντύχην παρακαλῶ τὸ αὐτὸ φρονεῖν ἐν Κυρίῳ.
\vs{3}ναὶ ἐρωτῶ καὶ σέ, γνήσιε σύζυγε, συλλαμβάνου αὐταῖς, αἵτινες ἐν τῷ εὐαγγελίῳ συνήθλησάν μοι μετὰ καὶ Κλήμεντος καὶ τῶν λοιπῶν συνεργῶν μου, ὧν τὰ ὀνόματα ἐν βίβλῳ ζωῆς.

\vs{4}Χαίρετε ἐν Κυρίῳ πάντοτε· πάλιν ἐρῶ, χαίρετε.
\vs{5}τὸ ἐπιεικὲς ὑμῶν γνωσθήτω πᾶσιν ἀνθρώποις. ὁ Κύριος ἐγγύς.
\vs{6}Μηδὲν μεριμνᾶτε, ἀλλ᾽ ἐν παντὶ τῇ προσευχῇ καὶ τῇ δεήσει μετὰ εὐχαριστίας τὰ αἰτήματα ὑμῶν γνωριζέσθω πρὸς τὸν Θεόν.
\vs{7}καὶ ἡ εἰρήνη τοῦ Θεοῦ ἡ ὑπερέχουσα πάντα νοῦν φρουρήσει τὰς καρδίας ὑμῶν καὶ τὰ νοήματα ὑμῶν ἐν Χριστῷ Ἰησοῦ.

\vs{8}Τὸ λοιπόν, ἀδελφοί, ὅσα ἐστὶν ἀληθῆ, ὅσα σεμνά, ὅσα δίκαια, ὅσα ἁγνά, ὅσα προσφιλῆ, ὅσα εὔφημα, εἴ τις ἀρετὴ καὶ εἴ τις ἔπαινος, ταῦτα λογίζεσθε·
\vs{9}ἃ καὶ ἐμάθετε καὶ παρελάβετε καὶ ἠκούσατε καὶ εἴδετε ἐν ἐμοί, ταῦτα πράσσετε· καὶ ὁ Θεὸς τῆς εἰρήνης ἔσται μεθ᾽ ὑμῶν.

\vs{10}Ἐχάρην δὲ ἐν Κυρίῳ μεγάλως ὅτι ἤδη ποτὲ ἀνεθάλετε τὸ ὑπὲρ ἐμοῦ φρονεῖν, ἐφ᾽ ᾧ καὶ ἐφρονεῖτε, ἠκαιρεῖσθε δέ.
\vs{11}οὐχ ὅτι καθ᾽ ὑστέρησιν λέγω, ἐγὼ γὰρ ἔμαθον ἐν οἷς εἰμι αὐτάρκης εἶναι.
\vs{12}οἶδα καὶ ταπεινοῦσθαι, οἶδα καὶ περισσεύειν· ἐν παντὶ καὶ ἐν πᾶσιν μεμύημαι, καὶ χορτάζεσθαι καὶ πεινᾶν καὶ περισσεύειν καὶ ὑστερεῖσθαι·
\vs{13}πάντα ἰσχύω ἐν τῷ ἐνδυναμοῦντί με.
\vs{14}Πλὴν καλῶς ἐποιήσατε συνκοινωνήσαντές μου τῇ θλίψει.
\vs{15}οἴδατε δὲ καὶ ὑμεῖς, Φιλιππήσιοι, ὅτι ἐν ἀρχῇ τοῦ εὐαγγελίου, ὅτε ἐξῆλθον ἀπὸ Μακεδονίας, οὐδεμία μοι ἐκκλησία ἐκοινώνησεν εἰς λόγον δόσεως καὶ λήμψεως εἰ μὴ ὑμεῖς μόνοι,
\vs{16}ὅτι καὶ ἐν Θεσσαλονίκῃ καὶ ἅπαξ καὶ δὶς εἰς τὴν χρείαν μοι ἐπέμψατε.
\vs{17}Οὐχ ὅτι ἐπιζητῶ τὸ δόμα, ἀλλὰ ἐπιζητῶ τὸν καρπὸν τὸν πλεονάζοντα εἰς λόγον ὑμῶν.
\vs{18}ἀπέχω δὲ πάντα καὶ περισσεύω· πεπλήρωμαι δεξάμενος παρὰ Ἐπαφροδίτου τὰ παρ᾽ ὑμῶν, ὀσμὴν εὐωδίας, θυσίαν δεκτήν, εὐάρεστον τῷ Θεῷ.
\vs{19}Ὁ δὲ Θεός μου πληρώσει πᾶσαν χρείαν ὑμῶν κατὰ τὸ πλοῦτος αὐτοῦ ἐν δόξῃ ἐν Χριστῷ Ἰησοῦ.
\vs{20}τῷ δὲ Θεῷ καὶ Πατρὶ ἡμῶν ἡ δόξα εἰς τοὺς αἰῶνας τῶν αἰώνων, ἀμήν.

\vs{21}Ἀσπάσασθε πάντα ἅγιον ἐν Χριστῷ Ἰησοῦ. Ἀσπάζονται ὑμᾶς οἱ σὺν ἐμοὶ ἀδελφοί.
\vs{22}Ἀσπάζονται ὑμᾶς πάντες οἱ ἅγιοι, μάλιστα δὲ οἱ ἐκ τῆς Καίσαρος οἰκίας.

\vs{23}Ἡ χάρις τοῦ Κυρίου Ἰησοῦ Χριστοῦ μετὰ τοῦ πνεύματος ὑμῶν.


\def\book{ΠΡΟΣ ΚΟΛΟΣΣΑΕΙΣ}
\biblebook{ΠΡΟΣ ΚΟΛΟΣΣΑΕΙΣ}


\lettrine[lines=2, loversize=0.2, nindent=0em, findent=.25em]{\textcolor{bookheadingcolor}{Π}}{αῦλος} ἀπόστολος Χριστοῦ Ἰησοῦ διὰ θελήματος Θεοῦ καὶ Τιμόθεος ὁ ἀδελφὸς
\vs{2}Τοῖς ἐν Κολοσσαῖς ἁγίοις καὶ πιστοῖς ἀδελφοῖς ἐν Χριστῷ, Χάρις ὑμῖν καὶ εἰρήνη ἀπὸ Θεοῦ Πατρὸς ἡμῶν.
\vs{3}Εὐχαριστοῦμεν τῷ Θεῷ Πατρὶ τοῦ Κυρίου ἡμῶν Ἰησοῦ Χριστοῦ πάντοτε περὶ ὑμῶν προσευχόμενοι,
\vs{4}ἀκούσαντες τὴν πίστιν ὑμῶν ἐν Χριστῷ Ἰησοῦ καὶ τὴν ἀγάπην ἣν ἔχετε εἰς πάντας τοὺς ἁγίους
\vs{5}διὰ τὴν ἐλπίδα τὴν ἀποκειμένην ὑμῖν ἐν τοῖς οὐρανοῖς, ἣν προηκούσατε ἐν τῷ λόγῳ τῆς ἀληθείας τοῦ εὐαγγελίου
\vs{6}τοῦ παρόντος εἰς ὑμᾶς, καθὼς καὶ ἐν παντὶ τῷ κόσμῳ ἐστὶν καρποφορούμενον καὶ αὐξανόμενον καθὼς καὶ ἐν ὑμῖν, ἀφ᾽ ἧς ἡμέρας ἠκούσατε καὶ ἐπέγνωτε τὴν χάριν τοῦ Θεοῦ ἐν ἀληθείᾳ·
\vs{7}καθὼς ἐμάθετε ἀπὸ Ἐπαφρᾶ τοῦ ἀγαπητοῦ συνδούλου ἡμῶν, ὅς ἐστιν πιστὸς ὑπὲρ ἡμῶν διάκονος τοῦ Χριστοῦ,
\vs{8}ὁ καὶ δηλώσας ἡμῖν τὴν ὑμῶν ἀγάπην ἐν Πνεύματι.

\vs{9}Διὰ τοῦτο καὶ ἡμεῖς, ἀφ᾽ ἧς ἡμέρας ἠκούσαμεν, οὐ παυόμεθα ὑπὲρ ὑμῶν προσευχόμενοι καὶ αἰτούμενοι, ἵνα πληρωθῆτε τὴν ἐπίγνωσιν τοῦ θελήματος αὐτοῦ ἐν πάσῃ σοφίᾳ καὶ συνέσει πνευματικῇ,
\vs{10}περιπατῆσαι ἀξίως τοῦ Κυρίου εἰς πᾶσαν ἀρεσκείαν, ἐν παντὶ ἔργῳ ἀγαθῷ καρποφοροῦντες καὶ αὐξανόμενοι τῇ ἐπιγνώσει τοῦ Θεοῦ,
\vs{11}ἐν πάσῃ δυνάμει δυναμούμενοι κατὰ τὸ κράτος τῆς δόξης αὐτοῦ εἰς πᾶσαν ὑπομονὴν καὶ μακροθυμίαν.

μετὰ χαρᾶς
\vs{12}εὐχαριστοῦντες τῷ Πατρὶ τῷ ἱκανώσαντι ὑμᾶς εἰς τὴν μερίδα τοῦ κλήρου τῶν ἁγίων ἐν τῷ φωτί·
\vs{13}ὃς ἐρρύσατο ἡμᾶς ἐκ τῆς ἐξουσίας τοῦ σκότους καὶ μετέστησεν εἰς τὴν βασιλείαν τοῦ Υἱοῦ τῆς ἀγάπης αὐτοῦ,
\vs{14}ἐν ᾧ ἔχομεν τὴν ἀπολύτρωσιν, τὴν ἄφεσιν τῶν ἁμαρτιῶν·
\begin{poetryblock}

\begin{quote} \vs{15}Ὅς ἐστιν εἰκὼν τοῦ Θεοῦ τοῦ ἀοράτου,\end{quote} 

\begin{quote}πρωτότοκος πάσης κτίσεως,\end{quote}

\begin{quote} \vs{16}ὅτι ἐν αὐτῷ ἐκτίσθη τὰ πάντα\end{quote} 

\begin{quote}ἐν τοῖς οὐρανοῖς καὶ ἐπὶ τῆς γῆς,\end{quote} 

\begin{quote}τὰ ὁρατὰ καὶ τὰ ἀόρατα,\end{quote} 

\begin{quote}εἴτε θρόνοι εἴτε κυριότητες\end{quote} 

\begin{quote}εἴτε ἀρχαὶ εἴτε ἐξουσίαι·\end{quote} 

\begin{quote}τὰ πάντα δι᾽ αὐτοῦ καὶ εἰς αὐτὸν ἔκτισται·\end{quote}

\begin{quote} \vs{17}Καὶ αὐτός ἐστιν πρὸ πάντων\end{quote} 

\begin{quote}καὶ τὰ πάντα ἐν αὐτῷ συνέστηκεν,\end{quote}

\begin{quote} \vs{18}καὶ αὐτός ἐστιν ἡ κεφαλὴ τοῦ σώματος τῆς ἐκκλησίας·\end{quote} 

\begin{quote}ὅς ἐστιν ἀρχή,\end{quote} 

\begin{quote}πρωτότοκος ἐκ τῶν νεκρῶν,\end{quote} 

\begin{quote}ἵνα γένηται ἐν πᾶσιν αὐτὸς πρωτεύων,\end{quote}

\begin{quote} \vs{19}ὅτι ἐν αὐτῷ εὐδόκησεν πᾶν τὸ πλήρωμα κατοικῆσαι\end{quote}

\begin{quote} \vs{20}καὶ δι᾽ αὐτοῦ ἀποκαταλλάξαι τὰ πάντα εἰς αὐτόν,\end{quote} 

\begin{quote}εἰρηνοποιήσας διὰ τοῦ αἵματος τοῦ σταυροῦ αὐτοῦ,\end{quote} 

\begin{quote}δι᾽ αὐτοῦ εἴτε τὰ ἐπὶ τῆς γῆς\end{quote} 

\begin{quote}εἴτε τὰ ἐν τοῖς οὐρανοῖς.\end{quote}
\end{poetryblock}

\vs{21}Καὶ ὑμᾶς ποτε ὄντας ἀπηλλοτριωμένους καὶ ἐχθροὺς τῇ διανοίᾳ ἐν τοῖς ἔργοις τοῖς πονηροῖς,
\vs{22}νυνὶ δὲ ἀποκατήλλαξεν ἐν τῷ σώματι τῆς σαρκὸς αὐτοῦ διὰ τοῦ θανάτου παραστῆσαι ὑμᾶς ἁγίους καὶ ἀμώμους καὶ ἀνεγκλήτους κατενώπιον αὐτοῦ,
\vs{23}εἴ γε ἐπιμένετε τῇ πίστει τεθεμελιωμένοι καὶ ἑδραῖοι καὶ μὴ μετακινούμενοι ἀπὸ τῆς ἐλπίδος τοῦ εὐαγγελίου οὗ ἠκούσατε, τοῦ κηρυχθέντος ἐν πάσῃ κτίσει τῇ ὑπὸ τὸν οὐρανόν, οὗ ἐγενόμην ἐγὼ Παῦλος διάκονος.

\vs{24}Νῦν χαίρω ἐν τοῖς παθήμασιν ὑπὲρ ὑμῶν καὶ ἀνταναπληρῶ τὰ ὑστερήματα τῶν θλίψεων τοῦ Χριστοῦ ἐν τῇ σαρκί μου ὑπὲρ τοῦ σώματος αὐτοῦ, ὅ ἐστιν ἡ ἐκκλησία,
\vs{25}ἧς ἐγενόμην ἐγὼ διάκονος κατὰ τὴν οἰκονομίαν τοῦ Θεοῦ τὴν δοθεῖσάν μοι εἰς ὑμᾶς πληρῶσαι τὸν λόγον τοῦ Θεοῦ,
\vs{26}τὸ μυστήριον τὸ ἀποκεκρυμμένον ἀπὸ τῶν αἰώνων καὶ ἀπὸ τῶν γενεῶν— νῦν δὲ ἐφανερώθη τοῖς ἁγίοις αὐτοῦ,
\vs{27}οἷς ἠθέλησεν ὁ Θεὸς γνωρίσαι τί τὸ πλοῦτος τῆς δόξης τοῦ μυστηρίου τούτου ἐν τοῖς ἔθνεσιν, ὅ ἐστιν Χριστὸς ἐν ὑμῖν, ἡ ἐλπὶς τῆς δόξης·
\vs{28}ὃν ἡμεῖς καταγγέλλομεν νουθετοῦντες πάντα ἄνθρωπον καὶ διδάσκοντες πάντα ἄνθρωπον ἐν πάσῃ σοφίᾳ, ἵνα παραστήσωμεν πάντα ἄνθρωπον τέλειον ἐν Χριστῷ·
\vs{29}Εἰς ὃ καὶ κοπιῶ ἀγωνιζόμενος κατὰ τὴν ἐνέργειαν αὐτοῦ τὴν ἐνεργουμένην ἐν ἐμοὶ ἐν δυνάμει.

\ch{2}
Θέλω γὰρ ὑμᾶς εἰδέναι ἡλίκον ἀγῶνα ἔχω ὑπὲρ ὑμῶν καὶ τῶν ἐν Λαοδικείᾳ καὶ ὅσοι οὐχ ἑόρακαν τὸ πρόσωπόν μου ἐν σαρκί,
\vs{2}ἵνα παρακληθῶσιν αἱ καρδίαι αὐτῶν συμβιβασθέντες ἐν ἀγάπῃ καὶ εἰς πᾶν πλοῦτος τῆς πληροφορίας τῆς συνέσεως, εἰς ἐπίγνωσιν τοῦ μυστηρίου τοῦ Θεοῦ, Χριστοῦ,
\vs{3}ἐν ᾧ εἰσιν πάντες οἱ θησαυροὶ τῆς σοφίας καὶ γνώσεως ἀπόκρυφοι.
\vs{4}Τοῦτο λέγω, ἵνα μηδεὶς ὑμᾶς παραλογίζηται ἐν πιθανολογίᾳ.
\vs{5}εἰ γὰρ καὶ τῇ σαρκὶ ἄπειμι, ἀλλὰ τῷ πνεύματι σὺν ὑμῖν εἰμι, χαίρων καὶ βλέπων ὑμῶν τὴν τάξιν καὶ τὸ στερέωμα τῆς εἰς Χριστὸν πίστεως ὑμῶν.

\vs{6}Ὡς οὖν παρελάβετε τὸν Χριστὸν Ἰησοῦν τὸν Κύριον, ἐν αὐτῷ περιπατεῖτε,
\vs{7}ἐρριζωμένοι καὶ ἐποικοδομούμενοι ἐν αὐτῷ καὶ βεβαιούμενοι τῇ πίστει καθὼς ἐδιδάχθητε, περισσεύοντες ἐν εὐχαριστίᾳ.
\vs{8}Βλέπετε μή τις ὑμᾶς ἔσται ὁ συλαγωγῶν διὰ τῆς φιλοσοφίας καὶ κενῆς ἀπάτης κατὰ τὴν παράδοσιν τῶν ἀνθρώπων, κατὰ τὰ στοιχεῖα τοῦ κόσμου καὶ οὐ κατὰ Χριστόν·
\vs{9}ὅτι ἐν αὐτῷ κατοικεῖ πᾶν τὸ πλήρωμα τῆς Θεότητος σωματικῶς,
\vs{10}καὶ ἐστὲ ἐν αὐτῷ πεπληρωμένοι, ὅς ἐστιν ἡ κεφαλὴ πάσης ἀρχῆς καὶ ἐξουσίας.
\vs{11}ἐν ᾧ καὶ περιετμήθητε περιτομῇ ἀχειροποιήτῳ ἐν τῇ ἀπεκδύσει τοῦ σώματος τῆς σαρκός, ἐν τῇ περιτομῇ τοῦ Χριστοῦ,
\vs{12}συνταφέντες αὐτῷ ἐν τῷ βαπτισμῷ, ἐν ᾧ καὶ συνηγέρθητε διὰ τῆς πίστεως τῆς ἐνεργείας τοῦ Θεοῦ τοῦ ἐγείραντος αὐτὸν ἐκ νεκρῶν·
\vs{13}Καὶ ὑμᾶς νεκροὺς ὄντας ἐν τοῖς παραπτώμασιν καὶ τῇ ἀκροβυστίᾳ τῆς σαρκὸς ὑμῶν, συνεζωοποίησεν ὑμᾶς σὺν αὐτῷ, χαρισάμενος ἡμῖν πάντα τὰ παραπτώματα.
\vs{14}ἐξαλείψας τὸ καθ᾽ ἡμῶν χειρόγραφον τοῖς δόγμασιν ὃ ἦν ὑπεναντίον ἡμῖν, καὶ αὐτὸ ἦρκεν ἐκ τοῦ μέσου προσηλώσας αὐτὸ τῷ σταυρῷ·
\vs{15}ἀπεκδυσάμενος τὰς ἀρχὰς καὶ τὰς ἐξουσίας ἐδειγμάτισεν ἐν παρρησίᾳ, θριαμβεύσας αὐτοὺς ἐν αὐτῷ.

\vs{16}Μὴ οὖν τις ὑμᾶς κρινέτω ἐν βρώσει καὶ ἐν πόσει ἢ ἐν μέρει ἑορτῆς ἢ νεομηνίας ἢ σαββάτων·
\vs{17}ἅ ἐστιν σκιὰ τῶν μελλόντων, τὸ δὲ σῶμα τοῦ Χριστοῦ.
\vs{18}μηδεὶς ὑμᾶς καταβραβευέτω θέλων ἐν ταπεινοφροσύνῃ καὶ θρησκείᾳ τῶν ἀγγέλων, ἃ ἑόρακεν ἐμβατεύων, εἰκῇ φυσιούμενος ὑπὸ τοῦ νοὸς τῆς σαρκὸς αὐτοῦ,
\vs{19}καὶ οὐ κρατῶν τὴν Κεφαλήν, ἐξ οὗ πᾶν τὸ σῶμα διὰ τῶν ἁφῶν καὶ συνδέσμων ἐπιχορηγούμενον καὶ συμβιβαζόμενον αὔξει τὴν αὔξησιν τοῦ Θεοῦ.

\vs{20}Εἰ ἀπεθάνετε σὺν Χριστῷ ἀπὸ τῶν στοιχείων τοῦ κόσμου, τί ὡς ζῶντες ἐν κόσμῳ δογματίζεσθε;
\vs{21}Μὴ ἅψῃ μηδὲ γεύσῃ μηδὲ θίγῃς,
\vs{22}ἅ ἐστιν πάντα εἰς φθορὰν τῇ ἀποχρήσει, κατὰ τὰ ἐντάλματα καὶ διδασκαλίας τῶν ἀνθρώπων,
\vs{23}ἅτινά ἐστιν λόγον μὲν ἔχοντα σοφίας ἐν ἐθελοθρησκίᾳ καὶ ταπεινοφροσύνῃ καὶ ἀφειδίᾳ σώματος, οὐκ ἐν τιμῇ τινι πρὸς πλησμονὴν τῆς σαρκός.

\ch{3}
Εἰ οὖν συνηγέρθητε τῷ Χριστῷ, τὰ ἄνω ζητεῖτε, οὗ ὁ Χριστός ἐστιν ἐν δεξιᾷ τοῦ Θεοῦ καθήμενος·
\vs{2}τὰ ἄνω φρονεῖτε, μὴ τὰ ἐπὶ τῆς γῆς.
\vs{3}ἀπεθάνετε γάρ καὶ ἡ ζωὴ ὑμῶν κέκρυπται σὺν τῷ Χριστῷ ἐν τῷ Θεῷ·
\vs{4}ὅταν ὁ Χριστὸς φανερωθῇ, ἡ ζωὴ ὑμῶν, τότε καὶ ὑμεῖς σὺν αὐτῷ φανερωθήσεσθε ἐν δόξῃ.

\vs{5}Νεκρώσατε οὖν τὰ μέλη τὰ ἐπὶ τῆς γῆς, πορνείαν ἀκαθαρσίαν πάθος ἐπιθυμίαν κακήν, καὶ τὴν πλεονεξίαν, ἥτις ἐστὶν εἰδωλολατρία,
\vs{6}δι᾽ ἃ ἔρχεται ἡ ὀργὴ τοῦ Θεοῦ ἐπὶ τοὺς υἱοὺς τῆς ἀπειθείας.
\vs{7}ἐν οἷς καὶ ὑμεῖς περιεπατήσατέ ποτε, ὅτε ἐζῆτε ἐν τούτοις·
\vs{8}νυνὶ δὲ ἀπόθεσθε καὶ ὑμεῖς τὰ πάντα, ὀργήν, θυμόν, κακίαν, βλασφημίαν, αἰσχρολογίαν ἐκ τοῦ στόματος ὑμῶν·
\vs{9}Μὴ ψεύδεσθε εἰς ἀλλήλους, ἀπεκδυσάμενοι τὸν παλαιὸν ἄνθρωπον σὺν ταῖς πράξεσιν αὐτοῦ
\vs{10}καὶ ἐνδυσάμενοι τὸν νέον τὸν ἀνακαινούμενον εἰς ἐπίγνωσιν κατ᾽ εἰκόνα τοῦ κτίσαντος αὐτόν,
\vs{11}ὅπου οὐκ ἔνι Ἕλλην καὶ Ἰουδαῖος, περιτομὴ καὶ ἀκροβυστία, βάρβαρος, Σκύθης, δοῦλος, ἐλεύθερος, ἀλλὰ τὰ πάντα καὶ ἐν πᾶσιν Χριστός.

\vs{12}Ἐνδύσασθε οὖν, ὡς ἐκλεκτοὶ τοῦ Θεοῦ ἅγιοι καὶ ἠγαπημένοι, σπλάγχνα οἰκτιρμοῦ χρηστότητα ταπεινοφροσύνην πραΰτητα μακροθυμίαν,
\vs{13}ἀνεχόμενοι ἀλλήλων καὶ χαριζόμενοι ἑαυτοῖς ἐάν τις πρός τινα ἔχῃ μομφήν· καθὼς καὶ ὁ Κύριος ἐχαρίσατο ὑμῖν, οὕτως καὶ ὑμεῖς·
\vs{14}ἐπὶ πᾶσιν δὲ τούτοις τὴν ἀγάπην, ὅ ἐστιν σύνδεσμος τῆς τελειότητος.
\vs{15}καὶ ἡ εἰρήνη τοῦ Χριστοῦ βραβευέτω ἐν ταῖς καρδίαις ὑμῶν, εἰς ἣν καὶ ἐκλήθητε ἐν ἑνὶ σώματι· καὶ εὐχάριστοι γίνεσθε.
\vs{16}Ὁ λόγος τοῦ Χριστοῦ ἐνοικείτω ἐν ὑμῖν πλουσίως, ἐν πάσῃ σοφίᾳ διδάσκοντες καὶ νουθετοῦντες ἑαυτοὺς, ψαλμοῖς ὕμνοις ᾠδαῖς πνευματικαῖς ἐν τῇ χάριτι ᾄδοντες ἐν ταῖς καρδίαις ὑμῶν τῷ Θεῷ·
\vs{17}καὶ πᾶν ὅ τι ἐὰν ποιῆτε ἐν λόγῳ ἢ ἐν ἔργῳ, πάντα ἐν ὀνόματι Κυρίου Ἰησοῦ, εὐχαριστοῦντες τῷ Θεῷ Πατρὶ δι᾽ αὐτοῦ.

\vs{18}Αἱ γυναῖκες, ὑποτάσσεσθε τοῖς ἀνδράσιν ὡς ἀνῆκεν ἐν Κυρίῳ.
\vs{19}Οἱ ἄνδρες, ἀγαπᾶτε τὰς γυναῖκας καὶ μὴ πικραίνεσθε πρὸς αὐτάς.
\vs{20}Τὰ τέκνα, ὑπακούετε τοῖς γονεῦσιν κατὰ πάντα, τοῦτο γὰρ εὐάρεστόν ἐστιν ἐν Κυρίῳ.
\vs{21}Οἱ πατέρες, μὴ ἐρεθίζετε τὰ τέκνα ὑμῶν, ἵνα μὴ ἀθυμῶσιν.

\vs{22}Οἱ δοῦλοι, ὑπακούετε κατὰ πάντα τοῖς κατὰ σάρκα κυρίοις, μὴ ἐν ὀφθαλμοδουλίαις ὡς ἀνθρωπάρεσκοι, ἀλλ᾽ ἐν ἁπλότητι καρδίας φοβούμενοι τὸν Κύριον.
\vs{23}Ὃ ἐὰν ποιῆτε, ἐκ ψυχῆς ἐργάζεσθε ὡς τῷ Κυρίῳ καὶ οὐκ ἀνθρώποις,
\vs{24}εἰδότες ὅτι ἀπὸ Κυρίου ἀπολήμψεσθε τὴν ἀνταπόδοσιν τῆς κληρονομίας. τῷ Κυρίῳ Χριστῷ δουλεύετε·
\vs{25}ὁ γὰρ ἀδικῶν κομίσεται ὃ ἠδίκησεν, καὶ οὐκ ἔστιν προσωπολημψία.
\inparch{4} \vs{1}Οἱ κύριοι, τὸ δίκαιον καὶ τὴν ἰσότητα τοῖς δούλοις παρέχεσθε, εἰδότες ὅτι καὶ ὑμεῖς ἔχετε Κύριον ἐν οὐρανῷ.

\vs{2}Τῇ προσευχῇ προσκαρτερεῖτε, γρηγοροῦντες ἐν αὐτῇ ἐν εὐχαριστίᾳ,
\vs{3}προσευχόμενοι ἅμα καὶ περὶ ἡμῶν, ἵνα ὁ Θεὸς ἀνοίξῃ ἡμῖν θύραν τοῦ λόγου λαλῆσαι τὸ μυστήριον τοῦ Χριστοῦ, δι᾽ ὃ καὶ δέδεμαι,
\vs{4}ἵνα φανερώσω αὐτὸ ὡς δεῖ με λαλῆσαι.
\vs{5}Ἐν σοφίᾳ περιπατεῖτε πρὸς τοὺς ἔξω τὸν καιρὸν ἐξαγοραζόμενοι.
\vs{6}ὁ λόγος ὑμῶν πάντοτε ἐν χάριτι, ἅλατι ἠρτυμένος, εἰδέναι πῶς δεῖ ὑμᾶς ἑνὶ ἑκάστῳ ἀποκρίνεσθαι.

\vs{7}Τὰ κατ᾽ ἐμὲ πάντα γνωρίσει ὑμῖν Τυχικὸς ὁ ἀγαπητὸς ἀδελφὸς καὶ πιστὸς διάκονος καὶ σύνδουλος ἐν Κυρίῳ,
\vs{8}ὃν ἔπεμψα πρὸς ὑμᾶς εἰς αὐτὸ τοῦτο, ἵνα γνῶτε τὰ περὶ ἡμῶν καὶ παρακαλέσῃ τὰς καρδίας ὑμῶν,
\vs{9}σὺν Ὀνησίμῳ τῷ πιστῷ καὶ ἀγαπητῷ ἀδελφῷ, ὅς ἐστιν ἐξ ὑμῶν· πάντα ὑμῖν γνωρίσουσιν τὰ ὧδε.
\vs{10}Ἀσπάζεται ὑμᾶς Ἀρίσταρχος ὁ συναιχμάλωτός μου καὶ Μᾶρκος ὁ ἀνεψιὸς Βαρνάβα περὶ οὗ ἐλάβετε ἐντολάς, ἐὰν ἔλθῃ πρὸς ὑμᾶς, δέξασθε αὐτόν
\vs{11}καὶ Ἰησοῦς ὁ λεγόμενος Ἰοῦστος, οἱ ὄντες ἐκ περιτομῆς, οὗτοι μόνοι συνεργοὶ εἰς τὴν βασιλείαν τοῦ Θεοῦ, οἵτινες ἐγενήθησάν μοι παρηγορία.
\vs{12}Ἀσπάζεται ὑμᾶς Ἐπαφρᾶς ὁ ἐξ ὑμῶν, δοῦλος Χριστοῦ Ἰησοῦ, πάντοτε ἀγωνιζόμενος ὑπὲρ ὑμῶν ἐν ταῖς προσευχαῖς, ἵνα σταθῆτε τέλειοι καὶ πεπληροφορημένοι ἐν παντὶ θελήματι τοῦ Θεοῦ.
\vs{13}μαρτυρῶ γὰρ αὐτῷ ὅτι ἔχει πολὺν πόνον ὑπὲρ ὑμῶν καὶ τῶν ἐν Λαοδικείᾳ καὶ τῶν ἐν Ἱεραπόλει.
\vs{14}Ἀσπάζεται ὑμᾶς Λουκᾶς ὁ ἰατρὸς ὁ ἀγαπητὸς καὶ Δημᾶς.

\vs{15}Ἀσπάσασθε τοὺς ἐν Λαοδικείᾳ ἀδελφοὺς καὶ Νύμφαν καὶ τὴν κατ᾽ οἶκον αὐτῆς ἐκκλησίαν.
\vs{16}Καὶ ὅταν ἀναγνωσθῇ παρ᾽ ὑμῖν ἡ ἐπιστολή, ποιήσατε ἵνα καὶ ἐν τῇ Λαοδικέων ἐκκλησίᾳ ἀναγνωσθῇ, καὶ τὴν ἐκ Λαοδικείας ἵνα καὶ ὑμεῖς ἀναγνῶτε.
\vs{17}Καὶ εἴπατε Ἀρχίππῳ· Βλέπε τὴν διακονίαν ἣν παρέλαβες ἐν Κυρίῳ, ἵνα αὐτὴν πληροῖς.

\vs{18}Ὁ ἀσπασμὸς τῇ ἐμῇ χειρὶ Παύλου. Μνημονεύετέ μου τῶν δεσμῶν. Ἡ χάρις μεθ᾽ ὑμῶν.


\def\book{ΑΠΟΚΑΛΥΨΙΣ ΙΩΑΝΝΟΥ}
\biblebook{ΑΠΟΚΑΛΥΨΙΣ ΙΩΑΝΝΟΥ}


\lettrine[lines=2, loversize=0.2, nindent=0em, findent=.25em]{\textcolor{bookheadingcolor}{Ἀ}}{ποκάλυψις} Ἰησοῦ Χριστοῦ ἣν ἔδωκεν αὐτῷ ὁ Θεός δεῖξαι τοῖς δούλοις αὐτοῦ ἃ δεῖ γενέσθαι ἐν τάχει, καὶ ἐσήμανεν ἀποστείλας διὰ τοῦ ἀγγέλου αὐτοῦ τῷ δούλῳ αὐτοῦ Ἰωάννῃ,
\vs{2}ὃς ἐμαρτύρησεν τὸν λόγον τοῦ Θεοῦ καὶ τὴν μαρτυρίαν Ἰησοῦ Χριστοῦ ὅσα εἶδεν.
\vs{3}Μακάριος ὁ ἀναγινώσκων καὶ οἱ ἀκούοντες τοὺς λόγους τῆς προφητείας καὶ τηροῦντες τὰ ἐν αὐτῇ γεγραμμένα, ὁ γὰρ καιρὸς ἐγγύς.

\vs{4}Ἰωάννης Ταῖς ἑπτὰ ἐκκλησίαις ταῖς ἐν τῇ Ἀσίᾳ· Χάρις ὑμῖν καὶ εἰρήνη ἀπὸ ὁ ὢν καὶ ὁ ἦν καὶ ὁ ἐρχόμενος καὶ ἀπὸ τῶν ἑπτὰ Πνευμάτων ἃ ἐνώπιον τοῦ θρόνου αὐτοῦ
\vs{5}καὶ ἀπὸ Ἰησοῦ Χριστοῦ, ὁ μάρτυς, ὁ πιστός, ὁ πρωτότοκος τῶν νεκρῶν καὶ ὁ ἄρχων τῶν βασιλέων τῆς γῆς. Τῷ ἀγαπῶντι ἡμᾶς καὶ λύσαντι ἡμᾶς ἐκ τῶν ἁμαρτιῶν ἡμῶν ἐν τῷ αἵματι αὐτοῦ,
\vs{6}καὶ ἐποίησεν ἡμᾶς βασιλείαν, ἱερεῖς τῷ Θεῷ καὶ Πατρὶ αὐτοῦ, αὐτῷ ἡ δόξα καὶ τὸ κράτος εἰς τοὺς αἰῶνας τῶν αἰώνων· ἀμήν.

\vs{7}Ἰδοὺ ἔρχεται μετὰ τῶν νεφελῶν, 
\begin{poetryblock}

\begin{quote}καὶ ὄψεται αὐτὸν πᾶς ὀφθαλμὸς\end{quote} 

\begin{quote}καὶ οἵτινες αὐτὸν ἐξεκέντησαν,\end{quote} 

\begin{quote}καὶ κόψονται ἐπ᾽ αὐτὸν πᾶσαι αἱ φυλαὶ τῆς γῆς.\end{quote}
\end{poetryblock}

ναί, ἀμήν.

\vs{8}Ἐγώ εἰμι τὸ Ἄλφα καὶ τὸ Ὦ, λέγει Κύριος ὁ Θεός, ὁ ὢν καὶ ὁ ἦν καὶ ὁ ἐρχόμενος, ὁ Παντοκράτωρ.

\vs{9}Ἐγὼ Ἰωάννης, ὁ ἀδελφὸς ὑμῶν καὶ συνκοινωνὸς ἐν τῇ θλίψει καὶ βασιλείᾳ καὶ ὑπομονῇ ἐν Ἰησοῦ, ἐγενόμην ἐν τῇ νήσῳ τῇ καλουμένῃ Πάτμῳ διὰ τὸν λόγον τοῦ Θεοῦ καὶ τὴν μαρτυρίαν Ἰησοῦ.
\vs{10}ἐγενόμην ἐν Πνεύματι ἐν τῇ κυριακῇ ἡμέρᾳ καὶ ἤκουσα ὀπίσω μου φωνὴν μεγάλην ὡς σάλπιγγος
\vs{11}λεγούσης· Ὃ βλέπεις γράψον εἰς βιβλίον καὶ πέμψον ταῖς ἑπτὰ ἐκκλησίαις, εἰς Ἔφεσον καὶ εἰς Σμύρναν καὶ εἰς Πέργαμον καὶ εἰς Θυάτειρα καὶ εἰς Σάρδεις καὶ εἰς Φιλαδέλφειαν καὶ εἰς Λαοδίκειαν.

\vs{12}Καὶ ἐπέστρεψα βλέπειν τὴν φωνὴν ἥτις ἐλάλει μετ᾽ ἐμοῦ, καὶ ἐπιστρέψας εἶδον ἑπτὰ λυχνίας χρυσᾶς
\vs{13}καὶ ἐν μέσῳ τῶν λυχνιῶν ὅμοιον υἱὸν ἀνθρώπου ἐνδεδυμένον ποδήρη καὶ περιεζωσμένον πρὸς τοῖς μαστοῖς ζώνην χρυσᾶν.
\vs{14}ἡ δὲ κεφαλὴ αὐτοῦ καὶ αἱ τρίχες λευκαὶ ὡς ἔριον λευκόν ὡς χιών καὶ οἱ ὀφθαλμοὶ αὐτοῦ ὡς φλὸξ πυρός
\vs{15}καὶ οἱ πόδες αὐτοῦ ὅμοιοι χαλκολιβάνῳ ὡς ἐν καμίνῳ πεπυρωμένης καὶ ἡ φωνὴ αὐτοῦ ὡς φωνὴ ὑδάτων πολλῶν,
\vs{16}καὶ ἔχων ἐν τῇ δεξιᾷ χειρὶ αὐτοῦ ἀστέρας ἑπτά καὶ ἐκ τοῦ στόματος αὐτοῦ ῥομφαία δίστομος ὀξεῖα ἐκπορευομένη καὶ ἡ ὄψις αὐτοῦ ὡς ὁ ἥλιος φαίνει ἐν τῇ δυνάμει αὐτοῦ.

\vs{17}Καὶ ὅτε εἶδον αὐτόν, ἔπεσα πρὸς τοὺς πόδας αὐτοῦ ὡς νεκρός, καὶ ἔθηκεν τὴν δεξιὰν αὐτοῦ ἐπ᾽ ἐμὲ λέγων·

Μὴ φοβοῦ· ἐγώ εἰμι ὁ πρῶτος καὶ ὁ ἔσχατος
\vs{18}καὶ ὁ Ζῶν, καὶ ἐγενόμην νεκρὸς καὶ ἰδοὺ ζῶν εἰμι εἰς τοὺς αἰῶνας τῶν αἰώνων καὶ ἔχω τὰς κλεῖς τοῦ θανάτου καὶ τοῦ ᾅδου.
\vs{19}Γράψον οὖν ἃ εἶδες καὶ ἃ εἰσὶν καὶ ἃ μέλλει γενέσθαι μετὰ ταῦτα.
\vs{20}τὸ μυστήριον τῶν ἑπτὰ ἀστέρων οὓς εἶδες ἐπὶ τῆς δεξιᾶς μου καὶ τὰς ἑπτὰ λυχνίας τὰς χρυσᾶς· οἱ ἑπτὰ ἀστέρες ἄγγελοι τῶν ἑπτὰ ἐκκλησιῶν εἰσίν καὶ αἱ λυχνίαι αἱ ἑπτὰ ἑπτὰ ἐκκλησίαι εἰσίν.

\ch{2}
Τῷ ἀγγέλῳ τῆς ἐν Ἐφέσῳ ἐκκλησίας γράψον·

Τάδε λέγει ὁ κρατῶν τοὺς ἑπτὰ ἀστέρας ἐν τῇ δεξιᾷ αὐτοῦ, ὁ περιπατῶν ἐν μέσῳ τῶν ἑπτὰ λυχνιῶν τῶν χρυσῶν·
\vs{2}Οἶδα τὰ ἔργα σου καὶ τὸν κόπον καὶ τὴν ὑπομονήν σου καὶ ὅτι οὐ δύνῃ βαστάσαι κακούς, καὶ ἐπείρασας τοὺς λέγοντας ἑαυτοὺς ἀποστόλους καὶ οὐκ εἰσίν καὶ εὗρες αὐτοὺς ψευδεῖς,
\vs{3}καὶ ὑπομονὴν ἔχεις καὶ ἐβάστασας διὰ τὸ ὄνομά μου καὶ οὐ κεκοπίακες.
\vs{4}Ἀλλὰ ἔχω κατὰ σοῦ ὅτι τὴν ἀγάπην σου τὴν πρώτην ἀφῆκες.
\vs{5}μνημόνευε οὖν πόθεν πέπτωκας καὶ μετανόησον καὶ τὰ πρῶτα ἔργα ποίησον· εἰ δὲ μή, ἔρχομαί σοι καὶ κινήσω τὴν λυχνίαν σου ἐκ τοῦ τόπου αὐτῆς, ἐὰν μὴ μετανοήσῃς.
\vs{6}Ἀλλὰ τοῦτο ἔχεις, ὅτι μισεῖς τὰ ἔργα τῶν Νικολαϊτῶν ἃ κἀγὼ μισῶ.

\vs{7}Ὁ ἔχων οὖς ἀκουσάτω τί τὸ Πνεῦμα λέγει ταῖς ἐκκλησίαις. Τῷ νικῶντι δώσω αὐτῷ φαγεῖν ἐκ τοῦ ξύλου τῆς ζωῆς, ὅ ἐστιν ἐν τῷ Παραδείσῳ τοῦ Θεοῦ.
\vs{8}Καὶ τῷ ἀγγέλῳ τῆς ἐν Σμύρνῃ ἐκκλησίας γράψον·

Τάδε λέγει ὁ πρῶτος καὶ ὁ ἔσχατος, ὃς ἐγένετο νεκρὸς καὶ ἔζησεν·
\vs{9}Οἶδά σου τὴν θλῖψιν καὶ τὴν πτωχείαν, ἀλλὰ πλούσιος εἶ, καὶ τὴν βλασφημίαν ἐκ τῶν λεγόντων Ἰουδαίους εἶναι ἑαυτούς καὶ οὐκ εἰσίν ἀλλὰ συναγωγὴ τοῦ Σατανᾶ.
\vs{10}Μηδὲν φοβοῦ ἃ μέλλεις πάσχειν. ἰδοὺ μέλλει βάλλειν ὁ διάβολος ἐξ ὑμῶν εἰς φυλακὴν ἵνα πειρασθῆτε καὶ ἕξετε θλῖψιν ἡμερῶν δέκα. γίνου πιστὸς ἄχρι θανάτου, καὶ δώσω σοι τὸν στέφανον τῆς ζωῆς.

\vs{11}Ὁ ἔχων οὖς ἀκουσάτω τί τὸ Πνεῦμα λέγει ταῖς ἐκκλησίαις. Ὁ νικῶν οὐ μὴ ἀδικηθῇ ἐκ τοῦ θανάτου τοῦ δευτέρου.

\vs{12}Καὶ τῷ ἀγγέλῳ τῆς ἐν Περγάμῳ ἐκκλησίας γράψον·

Τάδε λέγει ὁ ἔχων τὴν ῥομφαίαν τὴν δίστομον τὴν ὀξεῖαν·
\vs{13}Οἶδα ποῦ κατοικεῖς, ὅπου ὁ θρόνος τοῦ Σατανᾶ, καὶ κρατεῖς τὸ ὄνομά μου καὶ οὐκ ἠρνήσω τὴν πίστιν μου καὶ ἐν ταῖς ἡμέραις Ἀντιπᾶς ὁ μάρτυς μου ὁ πιστός μου, ὃς ἀπεκτάνθη παρ᾽ ὑμῖν, ὅπου ὁ Σατανᾶς κατοικεῖ.
\vs{14}Ἀλλ᾽ ἔχω κατὰ σοῦ ὀλίγα ὅτι ἔχεις ἐκεῖ κρατοῦντας τὴν διδαχὴν Βαλαάμ, ὃς ἐδίδασκεν τῷ Βαλὰκ βαλεῖν σκάνδαλον ἐνώπιον τῶν υἱῶν Ἰσραήλ φαγεῖν εἰδωλόθυτα καὶ πορνεῦσαι.
\vs{15}οὕτως ἔχεις καὶ σὺ κρατοῦντας τὴν διδαχὴν τῶν Νικολαϊτῶν ὁμοίως.
\vs{16}μετανόησον οὖν· εἰ δὲ μή, ἔρχομαί σοι ταχύ καὶ πολεμήσω μετ᾽ αὐτῶν ἐν τῇ ῥομφαίᾳ τοῦ στόματός μου.

\vs{17}Ὁ ἔχων οὖς ἀκουσάτω τί τὸ Πνεῦμα λέγει ταῖς ἐκκλησίαις. Τῷ νικῶντι δώσω αὐτῷ τοῦ μάννα τοῦ κεκρυμμένου καὶ δώσω αὐτῷ ψῆφον λευκήν, καὶ ἐπὶ τὴν ψῆφον ὄνομα καινὸν γεγραμμένον ὃ οὐδεὶς οἶδεν εἰ μὴ ὁ λαμβάνων.

\vs{18}Καὶ τῷ ἀγγέλῳ τῆς ἐν Θυατείροις ἐκκλησίας γράψον·

Τάδε λέγει ὁ Υἱὸς τοῦ Θεοῦ, ὁ ἔχων τοὺς ὀφθαλμοὺς αὐτοῦ ὡς φλόγα πυρός καὶ οἱ πόδες αὐτοῦ ὅμοιοι χαλκολιβάνῳ·
\vs{19}Οἶδά σου τὰ ἔργα καὶ τὴν ἀγάπην καὶ τὴν πίστιν καὶ τὴν διακονίαν καὶ τὴν ὑπομονήν σου, καὶ τὰ ἔργα σου τὰ ἔσχατα πλείονα τῶν πρώτων.
\vs{20}Ἀλλὰ ἔχω κατὰ σοῦ ὅτι ἀφεῖς τὴν γυναῖκα Ἰεζάβελ, ἡ λέγουσα ἑαυτὴν προφῆτιν καὶ διδάσκει καὶ πλανᾷ τοὺς ἐμοὺς δούλους πορνεῦσαι καὶ φαγεῖν εἰδωλόθυτα.
\vs{21}καὶ ἔδωκα αὐτῇ χρόνον ἵνα μετανοήσῃ, καὶ οὐ θέλει μετανοῆσαι ἐκ τῆς πορνείας αὐτῆς.
\vs{22}Ἰδοὺ βάλλω αὐτὴν εἰς κλίνην καὶ τοὺς μοιχεύοντας μετ᾽ αὐτῆς εἰς θλῖψιν μεγάλην, ἐὰν μὴ μετανοήσωσιν ἐκ τῶν ἔργων αὐτῆς,
\vs{23}καὶ τὰ τέκνα αὐτῆς ἀποκτενῶ ἐν θανάτῳ. καὶ γνώσονται πᾶσαι αἱ ἐκκλησίαι ὅτι ἐγώ εἰμι ὁ ἐραυνῶν νεφροὺς καὶ καρδίας, καὶ δώσω ὑμῖν ἑκάστῳ κατὰ τὰ ἔργα ὑμῶν.
\vs{24}Ὑμῖν δὲ λέγω τοῖς λοιποῖς τοῖς ἐν Θυατείροις, ὅσοι οὐκ ἔχουσιν τὴν διδαχὴν ταύτην, οἵτινες οὐκ ἔγνωσαν τὰ βαθέα τοῦ Σατανᾶ ὡς λέγουσιν· οὐ βάλλω ἐφ᾽ ὑμᾶς ἄλλο βάρος,
\vs{25}πλὴν ὃ ἔχετε κρατήσατε ἄχρι οὗ ἂν ἥξω.

\vs{26}Καὶ ὁ νικῶν καὶ ὁ τηρῶν ἄχρι τέλους τὰ ἔργα μου, δώσω αὐτῷ ἐξουσίαν ἐπὶ τῶν ἐθνῶν
\vs{27}καὶ ποιμανεῖ αὐτοὺς ἐν ῥάβδῳ σιδηρᾷ ὡς τὰ σκεύη τὰ κεραμικὰ συντρίβεται,
\vs{28}ὡς κἀγὼ εἴληφα παρὰ τοῦ Πατρός μου, καὶ δώσω αὐτῷ τὸν ἀστέρα τὸν πρωϊνόν.
\vs{29}Ὁ ἔχων οὖς ἀκουσάτω τί τὸ Πνεῦμα λέγει ταῖς ἐκκλησίαις.

\ch{3}
Καὶ τῷ ἀγγέλῳ τῆς ἐν Σάρδεσιν ἐκκλησίας γράψον·

Τάδε λέγει ὁ ἔχων τὰ ἑπτὰ Πνεύματα τοῦ Θεοῦ καὶ τοὺς ἑπτὰ ἀστέρας· Οἶδά σου τὰ ἔργα ὅτι ὄνομα ἔχεις ὅτι ζῇς, καὶ νεκρὸς εἶ.
\vs{2}γίνου γρηγορῶν καὶ στήρισον τὰ λοιπὰ ἃ ἔμελλον ἀποθανεῖν, οὐ γὰρ εὕρηκά σου τὰ ἔργα πεπληρωμένα ἐνώπιον τοῦ Θεοῦ μου.
\vs{3}μνημόνευε οὖν πῶς εἴληφας καὶ ἤκουσας καὶ τήρει καὶ μετανόησον. ἐὰν οὖν μὴ γρηγορήσῃς, ἥξω ὡς κλέπτης, καὶ οὐ μὴ γνῷς ποίαν ὥραν ἥξω ἐπὶ σέ.
\vs{4}Ἀλλὰ ἔχεις ὀλίγα ὀνόματα ἐν Σάρδεσιν ἃ οὐκ ἐμόλυναν τὰ ἱμάτια αὐτῶν, καὶ περιπατήσουσιν μετ᾽ ἐμοῦ ἐν λευκοῖς, ὅτι ἄξιοί εἰσιν.

\vs{5}Ὁ νικῶν οὕτως περιβαλεῖται ἐν ἱματίοις λευκοῖς καὶ οὐ μὴ ἐξαλείψω τὸ ὄνομα αὐτοῦ ἐκ τῆς βίβλου τῆς ζωῆς καὶ ὁμολογήσω τὸ ὄνομα αὐτοῦ ἐνώπιον τοῦ Πατρός μου καὶ ἐνώπιον τῶν ἀγγέλων αὐτοῦ.
\vs{6}Ὁ ἔχων οὖς ἀκουσάτω τί τὸ Πνεῦμα λέγει ταῖς ἐκκλησίαις.

\vs{7}Καὶ τῷ ἀγγέλῳ τῆς ἐν Φιλαδελφείᾳ ἐκκλησίας γράψον·

Τάδε λέγει ὁ ἅγιος, ὁ ἀληθινός, ὁ ἔχων τὴν κλεῖν Δαυίδ, ὁ ἀνοίγων καὶ οὐδεὶς κλείσει καὶ κλείων καὶ οὐδεὶς ἀνοίγει·
\vs{8}Οἶδά σου τὰ ἔργα, ἰδοὺ δέδωκα ἐνώπιόν σου θύραν ἠνεῳγμένην, ἣν οὐδεὶς δύναται κλεῖσαι αὐτήν, ὅτι μικρὰν ἔχεις δύναμιν καὶ ἐτήρησάς μου τὸν λόγον καὶ οὐκ ἠρνήσω τὸ ὄνομά μου.
\vs{9}ἰδοὺ διδῶ ἐκ τῆς συναγωγῆς τοῦ Σατανᾶ τῶν λεγόντων ἑαυτοὺς Ἰουδαίους εἶναι, καὶ οὐκ εἰσὶν ἀλλὰ ψεύδονται. ἰδοὺ ποιήσω αὐτοὺς ἵνα ἥξουσιν καὶ προσκυνήσουσιν ἐνώπιον τῶν ποδῶν σου καὶ γνῶσιν ὅτι ἐγὼ ἠγάπησά σε.
\vs{10}Ὅτι ἐτήρησας τὸν λόγον τῆς ὑπομονῆς μου, κἀγώ σε τηρήσω ἐκ τῆς ὥρας τοῦ πειρασμοῦ τῆς μελλούσης ἔρχεσθαι ἐπὶ τῆς οἰκουμένης ὅλης πειράσαι τοὺς κατοικοῦντας ἐπὶ τῆς γῆς.
\vs{11}ἔρχομαι ταχύ· κράτει ὃ ἔχεις, ἵνα μηδεὶς λάβῃ τὸν στέφανόν σου.

\vs{12}Ὁ νικῶν ποιήσω αὐτὸν στῦλον ἐν τῷ ναῷ τοῦ Θεοῦ μου καὶ ἔξω οὐ μὴ ἐξέλθῃ ἔτι καὶ γράψω ἐπ᾽ αὐτὸν τὸ ὄνομα τοῦ Θεοῦ μου καὶ τὸ ὄνομα τῆς πόλεως τοῦ Θεοῦ μου, τῆς καινῆς Ἰερουσαλήμ ἡ καταβαίνουσα ἐκ τοῦ οὐρανοῦ ἀπὸ τοῦ Θεοῦ μου, καὶ τὸ ὄνομά μου τὸ καινόν.
\vs{13}Ὁ ἔχων οὖς ἀκουσάτω τί τὸ Πνεῦμα λέγει ταῖς ἐκκλησίαις.

\vs{14}Καὶ τῷ ἀγγέλῳ τῆς ἐν Λαοδικείᾳ ἐκκλησίας γράψον·

Τάδε λέγει ὁ Ἀμήν, ὁ μάρτυς ὁ πιστὸς καὶ ἀληθινός, ἡ ἀρχὴ τῆς κτίσεως τοῦ Θεοῦ·
\vs{15}Οἶδά σου τὰ ἔργα ὅτι οὔτε ψυχρὸς εἶ οὔτε ζεστός. ὄφελον ψυχρὸς ἦς ἢ ζεστός.
\vs{16}οὕτως ὅτι χλιαρὸς εἶ καὶ οὔτε ζεστὸς οὔτε ψυχρός, μέλλω σε ἐμέσαι ἐκ τοῦ στόματός μου.
\vs{17}Ὅτι λέγεις ὅτι Πλούσιός εἰμι καὶ πεπλούτηκα καὶ οὐδὲν χρείαν ἔχω, καὶ οὐκ οἶδας ὅτι σὺ εἶ ὁ ταλαίπωρος καὶ ἐλεεινὸς καὶ πτωχὸς καὶ τυφλὸς καὶ γυμνός,
\vs{18}συμβουλεύω σοι ἀγοράσαι παρ᾽ ἐμοῦ χρυσίον πεπυρωμένον ἐκ πυρὸς ἵνα πλουτήσῃς, καὶ ἱμάτια λευκὰ ἵνα περιβάλῃ καὶ μὴ φανερωθῇ ἡ αἰσχύνη τῆς γυμνότητός σου, καὶ κολλούριον ἐγχρῖσαι τοὺς ὀφθαλμούς σου ἵνα βλέπῃς.
\vs{19}ἐγὼ ὅσους ἐὰν φιλῶ ἐλέγχω καὶ παιδεύω· ζήλευε οὖν καὶ μετανόησον.
\vs{20}Ἰδοὺ ἕστηκα ἐπὶ τὴν θύραν καὶ κρούω· ἐάν τις ἀκούσῃ τῆς φωνῆς μου καὶ ἀνοίξῃ τὴν θύραν, καὶ εἰσελεύσομαι πρὸς αὐτὸν καὶ δειπνήσω μετ᾽ αὐτοῦ καὶ αὐτὸς μετ᾽ ἐμοῦ.

\vs{21}Ὁ νικῶν δώσω αὐτῷ καθίσαι μετ᾽ ἐμοῦ ἐν τῷ θρόνῳ μου, ὡς κἀγὼ ἐνίκησα καὶ ἐκάθισα μετὰ τοῦ Πατρός μου ἐν τῷ θρόνῳ αὐτοῦ.
\vs{22}Ὁ ἔχων οὖς ἀκουσάτω τί τὸ Πνεῦμα λέγει ταῖς ἐκκλησίαις.

\ch{4}
Μετὰ ταῦτα εἶδον, καὶ ἰδοὺ θύρα ἠνεῳγμένη ἐν τῷ οὐρανῷ, καὶ ἡ φωνὴ ἡ πρώτη ἣν ἤκουσα ὡς σάλπιγγος λαλούσης μετ᾽ ἐμοῦ λέγων· Ἀνάβα ὧδε, καὶ δείξω σοι ἃ δεῖ γενέσθαι μετὰ ταῦτα.

\vs{2}εὐθέως ἐγενόμην ἐν Πνεύματι, καὶ ἰδοὺ θρόνος ἔκειτο ἐν τῷ οὐρανῷ, καὶ ἐπὶ τὸν θρόνον καθήμενος,
\vs{3}καὶ ὁ καθήμενος ὅμοιος ὁράσει λίθῳ ἰάσπιδι καὶ σαρδίῳ, καὶ ἶρις κυκλόθεν τοῦ θρόνου ὅμοιος ὁράσει σμαραγδίνῳ.
\vs{4}καὶ κυκλόθεν τοῦ θρόνου θρόνους εἴκοσι τέσσαρες, καὶ ἐπὶ τοὺς θρόνους εἴκοσι τέσσαρας πρεσβυτέρους καθημένους περιβεβλημένους ἐν ἱματίοις λευκοῖς καὶ ἐπὶ τὰς κεφαλὰς αὐτῶν στεφάνους χρυσοῦς.
\vs{5}Καὶ ἐκ τοῦ θρόνου ἐκπορεύονται ἀστραπαὶ καὶ φωναὶ καὶ βρονταί, καὶ ἑπτὰ λαμπάδες πυρὸς καιόμεναι ἐνώπιον τοῦ θρόνου, ἅ εἰσιν τὰ ἑπτὰ Πνεύματα τοῦ Θεοῦ,
\vs{6}καὶ ἐνώπιον τοῦ θρόνου ὡς θάλασσα ὑαλίνη ὁμοία κρυστάλλῳ. καὶ ἐν μέσῳ τοῦ θρόνου καὶ κύκλῳ τοῦ θρόνου τέσσαρα ζῷα γέμοντα ὀφθαλμῶν ἔμπροσθεν καὶ ὄπισθεν.
\vs{7}καὶ τὸ ζῷον τὸ πρῶτον ὅμοιον λέοντι καὶ τὸ δεύτερον ζῷον ὅμοιον μόσχῳ καὶ τὸ τρίτον ζῷον ἔχων τὸ πρόσωπον ὡς ἀνθρώπου καὶ τὸ τέταρτον ζῷον ὅμοιον ἀετῷ πετομένῳ.
\vs{8}καὶ τὰ τέσσαρα ζῷα, ἓν καθ᾽ ἓν αὐτῶν ἔχων ἀνὰ πτέρυγας ἕξ, κυκλόθεν καὶ ἔσωθεν γέμουσιν ὀφθαλμῶν, καὶ ἀνάπαυσιν οὐκ ἔχουσιν ἡμέρας καὶ νυκτὸς λέγοντες· 
\begin{poetryblock}

\begin{quote}Ἅγιος ἅγιος ἅγιος Κύριος ὁ Θεός ὁ Παντοκράτωρ,\end{quote} 

\begin{quote}ὁ ἦν καὶ ὁ ὢν καὶ ὁ ἐρχόμενος.\end{quote}
\end{poetryblock}

\vs{9}Καὶ ὅταν δώσουσιν τὰ ζῷα δόξαν καὶ τιμὴν καὶ εὐχαριστίαν τῷ καθημένῳ ἐπὶ τῷ θρόνῳ τῷ ζῶντι εἰς τοὺς αἰῶνας τῶν αἰώνων,
\vs{10}πεσοῦνται οἱ εἴκοσι τέσσαρες πρεσβύτεροι ἐνώπιον τοῦ καθημένου ἐπὶ τοῦ θρόνου καὶ προσκυνήσουσιν τῷ ζῶντι εἰς τοὺς αἰῶνας τῶν αἰώνων καὶ βαλοῦσιν τοὺς στεφάνους αὐτῶν ἐνώπιον τοῦ θρόνου λέγοντες·
\begin{poetryblock}

\begin{quote} \vs{11}Ἄξιος εἶ, ὁ Κύριος καὶ ὁ Θεὸς ἡμῶν, λαβεῖν τὴν δόξαν καὶ τὴν τιμὴν καὶ τὴν δύναμιν, ὅτι σὺ ἔκτισας τὰ πάντα καὶ διὰ τὸ θέλημά σου ἦσαν καὶ ἐκτίσθησαν.\end{quote}
\end{poetryblock}

\ch{5}
Καὶ εἶδον ἐπὶ τὴν δεξιὰν τοῦ καθημένου ἐπὶ τοῦ θρόνου βιβλίον γεγραμμένον ἔσωθεν καὶ ὄπισθεν κατεσφραγισμένον σφραγῖσιν ἑπτά.
\vs{2}καὶ εἶδον ἄγγελον ἰσχυρὸν κηρύσσοντα ἐν φωνῇ μεγάλῃ· Τίς ἄξιος ἀνοῖξαι τὸ βιβλίον καὶ λῦσαι τὰς σφραγῖδας αὐτοῦ;
\vs{3}Καὶ οὐδεὶς ἐδύνατο ἐν τῷ οὐρανῷ οὐδὲ ἐπὶ τῆς γῆς οὐδὲ ὑποκάτω τῆς γῆς ἀνοῖξαι τὸ βιβλίον οὔτε βλέπειν αὐτό.
\vs{4}καὶ ἔκλαιον πολὺ, ὅτι οὐδεὶς ἄξιος εὑρέθη ἀνοῖξαι τὸ βιβλίον οὔτε βλέπειν αὐτό.
\vs{5}Καὶ εἷς ἐκ τῶν πρεσβυτέρων λέγει μοι· Μὴ κλαῖε, ἰδοὺ ἐνίκησεν ὁ Λέων ὁ ἐκ τῆς φυλῆς Ἰούδα, ἡ Ῥίζα Δαυίδ, ἀνοῖξαι τὸ βιβλίον καὶ τὰς ἑπτὰ σφραγῖδας αὐτοῦ.

\vs{6}Καὶ εἶδον ἐν μέσῳ τοῦ θρόνου καὶ τῶν τεσσάρων ζῴων καὶ ἐν μέσῳ τῶν πρεσβυτέρων Ἀρνίον ἑστηκὸς ὡς ἐσφαγμένον ἔχων κέρατα ἑπτὰ καὶ ὀφθαλμοὺς ἑπτά οἵ εἰσιν τὰ ἑπτὰ Πνεύματα τοῦ Θεοῦ ἀπεσταλμένοι εἰς πᾶσαν τὴν γῆν.
\vs{7}καὶ ἦλθεν καὶ εἴληφεν ἐκ τῆς δεξιᾶς τοῦ καθημένου ἐπὶ τοῦ θρόνου.

\vs{8}Καὶ ὅτε ἔλαβεν τὸ βιβλίον, τὰ τέσσαρα ζῷα καὶ οἱ εἴκοσι τέσσαρες πρεσβύτεροι ἔπεσαν ἐνώπιον τοῦ Ἀρνίου ἔχοντες ἕκαστος κιθάραν καὶ φιάλας χρυσᾶς γεμούσας θυμιαμάτων, αἵ εἰσιν αἱ προσευχαὶ τῶν ἁγίων,
\vs{9}καὶ ᾄδουσιν ᾠδὴν καινὴν λέγοντες· 
\begin{poetryblock}

\begin{quote}Ἄξιος εἶ λαβεῖν τὸ βιβλίον καὶ ἀνοῖξαι τὰς σφραγῖδας αὐτοῦ,\end{quote} 

\begin{quote}ὅτι ἐσφάγης καὶ ἠγόρασας τῷ Θεῷ ἐν τῷ αἵματί σου\end{quote} 

\begin{quote}ἐκ πάσης φυλῆς καὶ γλώσσης καὶ λαοῦ καὶ ἔθνους\end{quote}

\begin{quote} \vs{10}καὶ ἐποίησας αὐτοὺς τῷ Θεῷ ἡμῶν βασιλείαν καὶ ἱερεῖς,\end{quote} 

\begin{quote}καὶ βασιλεύσουσιν ἐπὶ τῆς γῆς.\end{quote}
\end{poetryblock}

\vs{11}Καὶ εἶδον, καὶ ἤκουσα φωνὴν ἀγγέλων πολλῶν κύκλῳ τοῦ θρόνου καὶ τῶν ζῴων καὶ τῶν πρεσβυτέρων, καὶ ἦν ὁ ἀριθμὸς αὐτῶν μυριάδες μυριάδων καὶ χιλιάδες χιλιάδων
\vs{12}λέγοντες φωνῇ μεγάλῃ· 
\begin{poetryblock}

\begin{quote}Ἄξιόν ἐστιν τὸ Ἀρνίον τὸ ἐσφαγμένον λαβεῖν\end{quote} 

\begin{quote}τὴν δύναμιν καὶ πλοῦτον καὶ σοφίαν\end{quote} 

\begin{quote}καὶ ἰσχὺν καὶ τιμὴν καὶ δόξαν καὶ εὐλογίαν.\end{quote}
\end{poetryblock}

\vs{13}Καὶ πᾶν κτίσμα ὃ ἐν τῷ οὐρανῷ καὶ ἐπὶ τῆς γῆς καὶ ὑποκάτω τῆς γῆς καὶ ἐπὶ τῆς θαλάσσης καὶ τὰ ἐν αὐτοῖς πάντα ἤκουσα λέγοντας· 
\begin{poetryblock}

\begin{quote}Τῷ καθημένῳ ἐπὶ τῷ θρόνῳ καὶ τῷ Ἀρνίῳ\end{quote} 

\begin{quote}ἡ εὐλογία καὶ ἡ τιμὴ καὶ ἡ δόξα καὶ τὸ κράτος\end{quote} 

\begin{quote}εἰς τοὺς αἰῶνας τῶν αἰώνων.\end{quote}
\end{poetryblock}
\vs{14}Καὶ τὰ τέσσαρα ζῷα ἔλεγον· Ἀμήν. καὶ οἱ πρεσβύτεροι ἔπεσαν καὶ προσεκύνησαν.

\ch{6}
Καὶ εἶδον ὅτε ἤνοιξεν τὸ Ἀρνίον μίαν ἐκ τῶν ἑπτὰ σφραγίδων, καὶ ἤκουσα ἑνὸς ἐκ τῶν τεσσάρων ζῴων λέγοντος ὡς φωνῇ βροντῆς· Ἔρχου.
\vs{2}Καὶ εἶδον, καὶ ἰδοὺ ἵππος λευκός, καὶ ὁ καθήμενος ἐπ᾽ αὐτὸν ἔχων τόξον καὶ ἐδόθη αὐτῷ στέφανος καὶ ἐξῆλθεν νικῶν καὶ ἵνα νικήσῃ.

\vs{3}Καὶ ὅτε ἤνοιξεν τὴν σφραγῖδα τὴν δευτέραν, ἤκουσα τοῦ δευτέρου ζῴου λέγοντος· Ἔρχου.
\vs{4}Καὶ ἐξῆλθεν ἄλλος ἵππος πυρρός, καὶ τῷ καθημένῳ ἐπ᾽ αὐτὸν ἐδόθη αὐτῷ λαβεῖν τὴν εἰρήνην ἐκ τῆς γῆς καὶ ἵνα ἀλλήλους σφάξουσιν καὶ ἐδόθη αὐτῷ μάχαιρα μεγάλη.

\vs{5}Καὶ ὅτε ἤνοιξεν τὴν σφραγῖδα τὴν τρίτην, ἤκουσα τοῦ τρίτου ζῴου λέγοντος· Ἔρχου. Καὶ εἶδον, καὶ ἰδοὺ ἵππος μέλας, καὶ ὁ καθήμενος ἐπ᾽ αὐτὸν ἔχων ζυγὸν ἐν τῇ χειρὶ αὐτοῦ.
\vs{6}καὶ ἤκουσα ὡς φωνὴν ἐν μέσῳ τῶν τεσσάρων ζῴων λέγουσαν· Χοῖνιξ σίτου δηναρίου καὶ τρεῖς χοίνικες κριθῶν δηναρίου, καὶ τὸ ἔλαιον καὶ τὸν οἶνον μὴ ἀδικήσῃς.

\vs{7}Καὶ ὅτε ἤνοιξεν τὴν σφραγῖδα τὴν τετάρτην, ἤκουσα φωνὴν τοῦ τετάρτου ζῴου λέγοντος· Ἔρχου.
\vs{8}Καὶ εἶδον, καὶ ἰδοὺ ἵππος χλωρός, καὶ ὁ καθήμενος ἐπάνω αὐτοῦ ὄνομα αὐτῷ Ὁ Θάνατος, καὶ ὁ ᾅδης ἠκολούθει μετ᾽ αὐτοῦ καὶ ἐδόθη αὐτοῖς ἐξουσία ἐπὶ τὸ τέταρτον τῆς γῆς ἀποκτεῖναι ἐν ῥομφαίᾳ καὶ ἐν λιμῷ καὶ ἐν θανάτῳ καὶ ὑπὸ τῶν θηρίων τῆς γῆς.

\vs{9}Καὶ ὅτε ἤνοιξεν τὴν πέμπτην σφραγῖδα, εἶδον ὑποκάτω τοῦ θυσιαστηρίου τὰς ψυχὰς τῶν ἐσφαγμένων διὰ τὸν λόγον τοῦ Θεοῦ καὶ διὰ τὴν μαρτυρίαν ἣν εἶχον.
\vs{10}καὶ ἔκραξαν φωνῇ μεγάλῃ λέγοντες· Ἕως πότε, ὁ Δεσπότης ὁ ἅγιος καὶ ἀληθινός, οὐ κρίνεις καὶ ἐκδικεῖς τὸ αἷμα ἡμῶν ἐκ τῶν κατοικούντων ἐπὶ τῆς γῆς;
\vs{11}Καὶ ἐδόθη αὐτοῖς ἑκάστῳ στολὴ λευκή καὶ ἐρρέθη αὐτοῖς ἵνα ἀναπαύσονται ἔτι χρόνον μικρόν, ἕως πληρωθῶσιν καὶ οἱ σύνδουλοι αὐτῶν καὶ οἱ ἀδελφοὶ αὐτῶν οἱ μέλλοντες ἀποκτέννεσθαι ὡς καὶ αὐτοί.

\vs{12}Καὶ εἶδον ὅτε ἤνοιξεν τὴν σφραγῖδα τὴν ἕκτην, καὶ σεισμὸς μέγας ἐγένετο καὶ ὁ ἥλιος ἐγένετο μέλας ὡς σάκκος τρίχινος καὶ ἡ σελήνη ὅλη ἐγένετο ὡς αἷμα
\vs{13}καὶ οἱ ἀστέρες τοῦ οὐρανοῦ ἔπεσαν εἰς τὴν γῆν, ὡς συκῆ βάλλει τοὺς ὀλύνθους αὐτῆς ὑπὸ ἀνέμου μεγάλου σειομένη,
\vs{14}καὶ ὁ οὐρανὸς ἀπεχωρίσθη ὡς βιβλίον ἑλισσόμενον καὶ πᾶν ὄρος καὶ νῆσος ἐκ τῶν τόπων αὐτῶν ἐκινήθησαν.
\vs{15}Καὶ οἱ βασιλεῖς τῆς γῆς καὶ οἱ μεγιστᾶνες καὶ οἱ χιλίαρχοι καὶ οἱ πλούσιοι καὶ οἱ ἰσχυροὶ καὶ πᾶς δοῦλος καὶ ἐλεύθερος ἔκρυψαν ἑαυτοὺς εἰς τὰ σπήλαια καὶ εἰς τὰς πέτρας τῶν ὀρέων
\vs{16}καὶ λέγουσιν τοῖς ὄρεσιν καὶ ταῖς πέτραις· Πέσετε ἐφ᾽ ἡμᾶς καὶ κρύψατε ἡμᾶς ἀπὸ προσώπου τοῦ καθημένου ἐπὶ τοῦ θρόνου καὶ ἀπὸ τῆς ὀργῆς τοῦ Ἀρνίου,
\vs{17}ὅτι ἦλθεν ἡ ἡμέρα ἡ μεγάλη τῆς ὀργῆς αὐτῶν, καὶ τίς δύναται σταθῆναι;

\ch{7}
Μετὰ τοῦτο εἶδον τέσσαρας ἀγγέλους ἑστῶτας ἐπὶ τὰς τέσσαρας γωνίας τῆς γῆς, κρατοῦντας τοὺς τέσσαρας ἀνέμους τῆς γῆς ἵνα μὴ πνέῃ ἄνεμος ἐπὶ τῆς γῆς μήτε ἐπὶ τῆς θαλάσσης μήτε ἐπὶ πᾶν δένδρον.
\vs{2}καὶ εἶδον ἄλλον ἄγγελον ἀναβαίνοντα ἀπὸ ἀνατολῆς ἡλίου ἔχοντα σφραγῖδα Θεοῦ ζῶντος, καὶ ἔκραξεν φωνῇ μεγάλῃ τοῖς τέσσαρσιν ἀγγέλοις οἷς ἐδόθη αὐτοῖς ἀδικῆσαι τὴν γῆν καὶ τὴν θάλασσαν
\vs{3}λέγων· Μὴ ἀδικήσητε τὴν γῆν μήτε τὴν θάλασσαν μήτε τὰ δένδρα, ἄχρι σφραγίσωμεν τοὺς δούλους τοῦ Θεοῦ ἡμῶν ἐπὶ τῶν μετώπων αὐτῶν.

\vs{4}Καὶ ἤκουσα τὸν ἀριθμὸν τῶν ἐσφραγισμένων, ἑκατὸν τεσσεράκοντα τέσσαρες χιλιάδες, ἐσφραγισμένοι ἐκ πάσης φυλῆς υἱῶν Ἰσραήλ·

\vs{5}Ἐκ φυλῆς Ἰούδα δώδεκα χιλιάδες ἐσφραγισμένοι, 
\begin{poetryblock}

\begin{quote}ἐκ φυλῆς Ῥουβὴν δώδεκα χιλιάδες,\end{quote} 

\begin{quote}ἐκ φυλῆς Γὰδ δώδεκα χιλιάδες,\end{quote}

\begin{quote} \vs{6}Ἐκ φυλῆς Ἀσὴρ δώδεκα χιλιάδες,\end{quote} 

\begin{quote}ἐκ φυλῆς Νεφθαλὶμ δώδεκα χιλιάδες,\end{quote} 

\begin{quote}ἐκ φυλῆς Μανασσῆ δώδεκα χιλιάδες,\end{quote}

\begin{quote} \vs{7}Ἐκ φυλῆς Συμεὼν δώδεκα χιλιάδες,\end{quote} 

\begin{quote}ἐκ φυλῆς Λευὶ δώδεκα χιλιάδες,\end{quote} 

\begin{quote}ἐκ φυλῆς Ἰσσαχὰρ δώδεκα χιλιάδες,\end{quote}

\begin{quote} \vs{8}Ἐκ φυλῆς Ζαβουλὼν δώδεκα χιλιάδες,\end{quote} 

\begin{quote}ἐκ φυλῆς Ἰωσὴφ δώδεκα χιλιάδες,\end{quote} 

\begin{quote}ἐκ φυλῆς Βενιαμὶν δώδεκα χιλιάδες ἐσφραγισμένοι.\end{quote}
\end{poetryblock}

\vs{9}Μετὰ ταῦτα εἶδον, καὶ ἰδοὺ ὄχλος πολύς, ὃν ἀριθμῆσαι αὐτὸν οὐδεὶς ἐδύνατο, ἐκ παντὸς ἔθνους καὶ φυλῶν καὶ λαῶν καὶ γλωσσῶν ἑστῶτες ἐνώπιον τοῦ θρόνου καὶ ἐνώπιον τοῦ Ἀρνίου περιβεβλημένους στολὰς λευκάς καὶ φοίνικες ἐν ταῖς χερσὶν αὐτῶν,
\vs{10}καὶ κράζουσιν φωνῇ μεγάλῃ λέγοντες· 
\begin{poetryblock}

\begin{quote}Ἡ σωτηρία τῷ Θεῷ ἡμῶν τῷ καθημένῳ ἐπὶ τῷ θρόνῳ καὶ τῷ Ἀρνίῳ.\end{quote}
\end{poetryblock}
\vs{11}Καὶ πάντες οἱ ἄγγελοι εἱστήκεισαν κύκλῳ τοῦ θρόνου καὶ τῶν πρεσβυτέρων καὶ τῶν τεσσάρων ζῴων καὶ ἔπεσαν ἐνώπιον τοῦ θρόνου ἐπὶ τὰ πρόσωπα αὐτῶν καὶ προσεκύνησαν τῷ Θεῷ
\vs{12}λέγοντες· 
\begin{poetryblock}

\begin{quote}Ἀμήν, ἡ εὐλογία καὶ ἡ δόξα καὶ ἡ σοφία καὶ ἡ εὐχαριστία καὶ ἡ τιμὴ καὶ ἡ δύναμις καὶ ἡ ἰσχὺς τῷ Θεῷ ἡμῶν εἰς τοὺς αἰῶνας τῶν αἰώνων· ἀμήν.\end{quote}
\end{poetryblock}

\vs{13}Καὶ ἀπεκρίθη εἷς ἐκ τῶν πρεσβυτέρων λέγων μοι· Οὗτοι οἱ περιβεβλημένοι τὰς στολὰς τὰς λευκὰς τίνες εἰσὶν καὶ πόθεν ἦλθον;
\vs{14}Καὶ εἴρηκα αὐτῷ· Κύριέ μου, σὺ οἶδας. Καὶ εἶπέν μοι· 
\begin{poetryblock}

\begin{quote}Οὗτοί εἰσιν οἱ ἐρχόμενοι ἐκ τῆς θλίψεως τῆς μεγάλης\end{quote} 

\begin{quote}καὶ ἔπλυναν τὰς στολὰς αὐτῶν\end{quote} 

\begin{quote}καὶ ἐλεύκαναν αὐτὰς ἐν τῷ αἵματι τοῦ Ἀρνίου.\end{quote}

\begin{quote} \vs{15}διὰ τοῦτό Εἰσιν ἐνώπιον τοῦ θρόνου τοῦ Θεοῦ\end{quote} 

\begin{quote}καὶ λατρεύουσιν αὐτῷ ἡμέρας καὶ νυκτὸς ἐν τῷ ναῷ αὐτοῦ,\end{quote} 

\begin{quote}καὶ ὁ καθήμενος ἐπὶ τοῦ θρόνου σκηνώσει ἐπ᾽ αὐτούς.\end{quote}

\begin{quote} \vs{16}οὐ πεινάσουσιν ἔτι οὐδὲ διψήσουσιν ἔτι\end{quote} 

\begin{quote}οὐδὲ μὴ πέσῃ ἐπ᾽ αὐτοὺς ὁ ἥλιος οὐδὲ πᾶν καῦμα,\end{quote}

\begin{quote} \vs{17}ὅτι τὸ Ἀρνίον τὸ ἀνὰ μέσον τοῦ θρόνου ποιμανεῖ αὐτούς\end{quote} 

\begin{quote}καὶ ὁδηγήσει αὐτοὺς ἐπὶ ζωῆς πηγὰς ὑδάτων,\end{quote} 

\begin{quote}καὶ ἐξαλείψει ὁ Θεὸς πᾶν δάκρυον ἐκ τῶν ὀφθαλμῶν αὐτῶν.\end{quote}
\end{poetryblock}

\ch{8}
Καὶ ὅταν ἤνοιξεν τὴν σφραγῖδα τὴν ἑβδόμην, ἐγένετο σιγὴ ἐν τῷ οὐρανῷ ὡς ἡμιώριον.
\vs{2}καὶ εἶδον τοὺς ἑπτὰ ἀγγέλους οἳ ἐνώπιον τοῦ Θεοῦ ἑστήκασιν, καὶ ἐδόθησαν αὐτοῖς ἑπτὰ σάλπιγγες.
\vs{3}Καὶ ἄλλος ἄγγελος ἦλθεν καὶ ἐστάθη ἐπὶ τοῦ θυσιαστηρίου ἔχων λιβανωτὸν χρυσοῦν, καὶ ἐδόθη αὐτῷ θυμιάματα πολλὰ, ἵνα δώσει ταῖς προσευχαῖς τῶν ἁγίων πάντων ἐπὶ τὸ θυσιαστήριον τὸ χρυσοῦν τὸ ἐνώπιον τοῦ θρόνου.
\vs{4}καὶ ἀνέβη ὁ καπνὸς τῶν θυμιαμάτων ταῖς προσευχαῖς τῶν ἁγίων ἐκ χειρὸς τοῦ ἀγγέλου ἐνώπιον τοῦ Θεοῦ.
\vs{5}Καὶ εἴληφεν ὁ ἄγγελος τὸν λιβανωτόν καὶ ἐγέμισεν αὐτὸν ἐκ τοῦ πυρὸς τοῦ θυσιαστηρίου καὶ ἔβαλεν εἰς τὴν γῆν, καὶ ἐγένοντο βρονταὶ καὶ φωναὶ καὶ ἀστραπαὶ καὶ σεισμός.

\vs{6}Καὶ οἱ ἑπτὰ ἄγγελοι οἱ ἔχοντες τὰς ἑπτὰ σάλπιγγας ἡτοίμασαν αὑτοὺς ἵνα σαλπίσωσιν.

\vs{7}Καὶ ὁ πρῶτος ἐσάλπισεν· καὶ ἐγένετο χάλαζα καὶ πῦρ μεμιγμένα ἐν αἵματι καὶ ἐβλήθη εἰς τὴν γῆν, καὶ τὸ τρίτον τῆς γῆς κατεκάη καὶ τὸ τρίτον τῶν δένδρων κατεκάη καὶ πᾶς χόρτος χλωρὸς κατεκάη.

\vs{8}Καὶ ὁ δεύτερος ἄγγελος ἐσάλπισεν· καὶ ὡς ὄρος μέγα πυρὶ καιόμενον ἐβλήθη εἰς τὴν θάλασσαν, καὶ ἐγένετο τὸ τρίτον τῆς θαλάσσης αἷμα
\vs{9}καὶ ἀπέθανεν τὸ τρίτον τῶν κτισμάτων τῶν ἐν τῇ θαλάσσῃ τὰ ἔχοντα ψυχάς καὶ τὸ τρίτον τῶν πλοίων διεφθάρησαν.

\vs{10}Καὶ ὁ τρίτος ἄγγελος ἐσάλπισεν· καὶ ἔπεσεν ἐκ τοῦ οὐρανοῦ ἀστὴρ μέγας καιόμενος ὡς λαμπάς καὶ ἔπεσεν ἐπὶ τὸ τρίτον τῶν ποταμῶν καὶ ἐπὶ τὰς πηγὰς τῶν ὑδάτων,
\vs{11}καὶ τὸ ὄνομα τοῦ ἀστέρος λέγεται Ὁ Ἄψινθος, καὶ ἐγένετο τὸ τρίτον τῶν ὑδάτων εἰς ἄψινθον καὶ πολλοὶ τῶν ἀνθρώπων ἀπέθανον ἐκ τῶν ὑδάτων ὅτι ἐπικράνθησαν.

\vs{12}Καὶ ὁ τέταρτος ἄγγελος ἐσάλπισεν· καὶ ἐπλήγη τὸ τρίτον τοῦ ἡλίου καὶ τὸ τρίτον τῆς σελήνης καὶ τὸ τρίτον τῶν ἀστέρων, ἵνα σκοτισθῇ τὸ τρίτον αὐτῶν καὶ ἡ ἡμέρα μὴ φάνῃ τὸ τρίτον αὐτῆς καὶ ἡ νὺξ ὁμοίως.

\vs{13}Καὶ εἶδον, καὶ ἤκουσα ἑνὸς ἀετοῦ πετομένου ἐν μεσουρανήματι λέγοντος φωνῇ μεγάλῃ· Οὐαὶ οὐαὶ οὐαὶ τοὺς κατοικοῦντας ἐπὶ τῆς γῆς ἐκ τῶν λοιπῶν φωνῶν τῆς σάλπιγγος τῶν τριῶν ἀγγέλων τῶν μελλόντων σαλπίζειν.

\ch{9}
Καὶ ὁ πέμπτος ἄγγελος ἐσάλπισεν· καὶ εἶδον ἀστέρα ἐκ τοῦ οὐρανοῦ πεπτωκότα εἰς τὴν γῆν, καὶ ἐδόθη αὐτῷ ἡ κλεὶς τοῦ φρέατος τῆς ἀβύσσου
\vs{2}καὶ ἤνοιξεν τὸ φρέαρ τῆς ἀβύσσου, καὶ ἀνέβη καπνὸς ἐκ τοῦ φρέατος ὡς καπνὸς καμίνου μεγάλης, καὶ ἐσκοτώθη ὁ ἥλιος καὶ ὁ ἀὴρ ἐκ τοῦ καπνοῦ τοῦ φρέατος.
\vs{3}Καὶ ἐκ τοῦ καπνοῦ ἐξῆλθον ἀκρίδες εἰς τὴν γῆν, καὶ ἐδόθη αὐταῖς ἐξουσία ὡς ἔχουσιν ἐξουσίαν οἱ σκορπίοι τῆς γῆς.
\vs{4}καὶ ἐρρέθη αὐταῖς ἵνα μὴ ἀδικήσουσιν τὸν χόρτον τῆς γῆς οὐδὲ πᾶν χλωρὸν οὐδὲ πᾶν δένδρον, εἰ μὴ τοὺς ἀνθρώπους οἵτινες οὐκ ἔχουσι τὴν σφραγῖδα τοῦ Θεοῦ ἐπὶ τῶν μετώπων.
\vs{5}καὶ ἐδόθη αὐτοῖς ἵνα μὴ ἀποκτείνωσιν αὐτούς, ἀλλ᾽ ἵνα βασανισθήσονται μῆνας πέντε, καὶ ὁ βασανισμὸς αὐτῶν ὡς βασανισμὸς σκορπίου ὅταν παίσῃ ἄνθρωπον.
\vs{6}καὶ ἐν ταῖς ἡμέραις ἐκείναις ζητήσουσιν οἱ ἄνθρωποι τὸν θάνατον καὶ οὐ μὴ εὑρήσουσιν αὐτόν, καὶ ἐπιθυμήσουσιν ἀποθανεῖν καὶ φεύγει ὁ θάνατος ἀπ᾽ αὐτῶν.

\vs{7}Καὶ τὰ ὁμοιώματα τῶν ἀκρίδων ὅμοια ἵπποις ἡτοιμασμένοις εἰς πόλεμον, καὶ ἐπὶ τὰς κεφαλὰς αὐτῶν ὡς στέφανοι ὅμοιοι χρυσῷ, καὶ τὰ πρόσωπα αὐτῶν ὡς πρόσωπα ἀνθρώπων,
\vs{8}καὶ εἶχον τρίχας ὡς τρίχας γυναικῶν, καὶ οἱ ὀδόντες αὐτῶν ὡς λεόντων ἦσαν,
\vs{9}καὶ εἶχον θώρακας ὡς θώρακας σιδηροῦς, καὶ ἡ φωνὴ τῶν πτερύγων αὐτῶν ὡς φωνὴ ἁρμάτων ἵππων πολλῶν τρεχόντων εἰς πόλεμον,
\vs{10}καὶ ἔχουσιν οὐρὰς ὁμοίας σκορπίοις καὶ κέντρα, καὶ ἐν ταῖς οὐραῖς αὐτῶν ἡ ἐξουσία αὐτῶν ἀδικῆσαι τοὺς ἀνθρώπους μῆνας πέντε,
\vs{11}ἔχουσιν ἐπ᾽ αὐτῶν βασιλέα τὸν ἄγγελον τῆς ἀβύσσου, ὄνομα αὐτῷ Ἑβραϊστί Ἀβαδδών, καὶ ἐν τῇ Ἑλληνικῇ ὄνομα ἔχει Ἀπολλύων.
\vs{12}Ἡ Οὐαὶ ἡ μία ἀπῆλθεν· ἰδοὺ ἔρχεται ἔτι δύο Οὐαὶ μετὰ ταῦτα.

\vs{13}Καὶ ὁ ἕκτος ἄγγελος ἐσάλπισεν· καὶ ἤκουσα φωνὴν μίαν ἐκ τῶν τεσσάρων κεράτων τοῦ θυσιαστηρίου τοῦ χρυσοῦ τοῦ ἐνώπιον τοῦ Θεοῦ,
\vs{14}λέγοντα τῷ ἕκτῳ ἀγγέλῳ, ὁ ἔχων τὴν σάλπιγγα· Λῦσον τοὺς τέσσαρας ἀγγέλους τοὺς δεδεμένους ἐπὶ τῷ ποταμῷ τῷ μεγάλῳ Εὐφράτῃ.
\vs{15}καὶ ἐλύθησαν οἱ τέσσαρες ἄγγελοι οἱ ἡτοιμασμένοι εἰς τὴν ὥραν καὶ ἡμέραν καὶ μῆνα καὶ ἐνιαυτόν, ἵνα ἀποκτείνωσιν τὸ τρίτον τῶν ἀνθρώπων.
\vs{16}καὶ ὁ ἀριθμὸς τῶν στρατευμάτων τοῦ ἱππικοῦ δισμυριάδες μυριάδων, ἤκουσα τὸν ἀριθμὸν αὐτῶν.
\vs{17}Καὶ οὕτως εἶδον τοὺς ἵππους ἐν τῇ ὁράσει καὶ τοὺς καθημένους ἐπ᾽ αὐτῶν, ἔχοντας θώρακας πυρίνους καὶ ὑακινθίνους καὶ θειώδεις, καὶ αἱ κεφαλαὶ τῶν ἵππων ὡς κεφαλαὶ λεόντων, καὶ ἐκ τῶν στομάτων αὐτῶν ἐκπορεύεται πῦρ καὶ καπνὸς καὶ θεῖον.
\vs{18}ἀπὸ τῶν τριῶν πληγῶν τούτων ἀπεκτάνθησαν τὸ τρίτον τῶν ἀνθρώπων, ἐκ τοῦ πυρὸς καὶ τοῦ καπνοῦ καὶ τοῦ θείου τοῦ ἐκπορευομένου ἐκ τῶν στομάτων αὐτῶν.
\vs{19}ἡ γὰρ ἐξουσία τῶν ἵππων ἐν τῷ στόματι αὐτῶν ἐστιν καὶ ἐν ταῖς οὐραῖς αὐτῶν, αἱ γὰρ οὐραὶ αὐτῶν ὅμοιαι ὄφεσιν, ἔχουσαι κεφαλάς καὶ ἐν αὐταῖς ἀδικοῦσιν.

\vs{20}Καὶ οἱ λοιποὶ τῶν ἀνθρώπων, οἳ οὐκ ἀπεκτάνθησαν ἐν ταῖς πληγαῖς ταύταις, οὐδὲ μετενόησαν ἐκ τῶν ἔργων τῶν χειρῶν αὐτῶν, ἵνα μὴ προσκυνήσουσιν τὰ δαιμόνια καὶ τὰ εἴδωλα τὰ χρυσᾶ καὶ τὰ ἀργυρᾶ καὶ τὰ χαλκᾶ καὶ τὰ λίθινα καὶ τὰ ξύλινα, ἃ οὔτε βλέπειν δύνανται οὔτε ἀκούειν οὔτε περιπατεῖν,
\vs{21}καὶ οὐ μετενόησαν ἐκ τῶν φόνων αὐτῶν οὔτε ἐκ τῶν φαρμάκων αὐτῶν οὔτε ἐκ τῆς πορνείας αὐτῶν οὔτε ἐκ τῶν κλεμμάτων αὐτῶν.

\ch{10}
Καὶ εἶδον ἄλλον ἄγγελον ἰσχυρὸν καταβαίνοντα ἐκ τοῦ οὐρανοῦ περιβεβλημένον νεφέλην, καὶ ἡ ἶρις ἐπὶ τῆς κεφαλῆς αὐτοῦ καὶ τὸ πρόσωπον αὐτοῦ ὡς ὁ ἥλιος καὶ οἱ πόδες αὐτοῦ ὡς στῦλοι πυρός,
\vs{2}καὶ ἔχων ἐν τῇ χειρὶ αὐτοῦ βιβλαρίδιον ἠνεῳγμένον. καὶ ἔθηκεν τὸν πόδα αὐτοῦ τὸν δεξιὸν ἐπὶ τῆς θαλάσσης, τὸν δὲ εὐώνυμον ἐπὶ τῆς γῆς,
\vs{3}καὶ ἔκραξεν φωνῇ μεγάλῃ ὥσπερ λέων μυκᾶται. καὶ ὅτε ἔκραξεν, ἐλάλησαν αἱ ἑπτὰ βρονταὶ τὰς ἑαυτῶν φωνάς.
\vs{4}Καὶ ὅτε ἐλάλησαν αἱ ἑπτὰ βρονταί, ἤμελλον γράφειν, καὶ ἤκουσα φωνὴν ἐκ τοῦ οὐρανοῦ λέγουσαν· Σφράγισον ἃ ἐλάλησαν αἱ ἑπτὰ βρονταί, καὶ μὴ αὐτὰ γράψῃς.

\vs{5}Καὶ ὁ ἄγγελος, ὃν εἶδον ἑστῶτα ἐπὶ τῆς θαλάσσης καὶ ἐπὶ τῆς γῆς, ἦρεν τὴν χεῖρα αὐτοῦ τὴν δεξιὰν εἰς τὸν οὐρανόν
\vs{6}καὶ ὤμοσεν ἐν τῷ ζῶντι εἰς τοὺς αἰῶνας τῶν αἰώνων, ὃς ἔκτισεν τὸν οὐρανὸν καὶ τὰ ἐν αὐτῷ καὶ τὴν γῆν καὶ τὰ ἐν αὐτῇ καὶ τὴν θάλασσαν καὶ τὰ ἐν αὐτῇ, ὅτι Χρόνος οὐκέτι ἔσται,
\vs{7}ἀλλ᾽ ἐν ταῖς ἡμέραις τῆς φωνῆς τοῦ ἑβδόμου ἀγγέλου, ὅταν μέλλῃ σαλπίζειν, καὶ ἐτελέσθη τὸ μυστήριον τοῦ Θεοῦ, ὡς εὐηγγέλισεν τοὺς ἑαυτοῦ δούλους τοὺς προφήτας.

\vs{8}Καὶ ἡ φωνὴ ἣν ἤκουσα ἐκ τοῦ οὐρανοῦ πάλιν λαλοῦσαν μετ᾽ ἐμοῦ καὶ λέγουσαν· Ὕπαγε λάβε τὸ βιβλίον τὸ ἠνεῳγμένον ἐν τῇ χειρὶ τοῦ ἀγγέλου τοῦ ἑστῶτος ἐπὶ τῆς θαλάσσης καὶ ἐπὶ τῆς γῆς.
\vs{9}Καὶ ἀπῆλθα πρὸς τὸν ἄγγελον λέγων αὐτῷ Δοῦναί μοι τὸ βιβλαρίδιον. Καὶ λέγει μοι· Λάβε καὶ κατάφαγε αὐτό, καὶ πικρανεῖ σου τὴν κοιλίαν, ἀλλ᾽ ἐν τῷ στόματί σου ἔσται γλυκὺ ὡς μέλι.

\vs{10}Καὶ ἔλαβον τὸ βιβλαρίδιον ἐκ τῆς χειρὸς τοῦ ἀγγέλου καὶ κατέφαγον αὐτό, καὶ ἦν ἐν τῷ στόματί μου ὡς μέλι γλυκύ καὶ ὅτε ἔφαγον αὐτό, ἐπικράνθη ἡ κοιλία μου.
\vs{11}Καὶ λέγουσίν μοι· Δεῖ σε πάλιν προφητεῦσαι ἐπὶ λαοῖς καὶ ἔθνεσιν καὶ γλώσσαις καὶ βασιλεῦσιν πολλοῖς.

\ch{11}
Καὶ ἐδόθη μοι κάλαμος ὅμοιος ῥάβδῳ, λέγων· Ἔγειρε καὶ μέτρησον τὸν ναὸν τοῦ Θεοῦ καὶ τὸ θυσιαστήριον καὶ τοὺς προσκυνοῦντας ἐν αὐτῷ.
\vs{2}καὶ τὴν αὐλὴν τὴν ἔξωθεν τοῦ ναοῦ ἔκβαλε ἔξωθεν καὶ μὴ αὐτὴν μετρήσῃς, ὅτι ἐδόθη τοῖς ἔθνεσιν, καὶ τὴν πόλιν τὴν ἁγίαν πατήσουσιν μῆνας τεσσεράκοντα καὶ δύο.

\vs{3}καὶ δώσω τοῖς δυσὶν μάρτυσίν μου καὶ προφητεύσουσιν ἡμέρας χιλίας διακοσίας ἑξήκοντα περιβεβλημένοι σάκκους.
\vs{4}Οὗτοί εἰσιν αἱ δύο ἐλαῖαι καὶ αἱ δύο λυχνίαι αἱ ἐνώπιον τοῦ Κυρίου τῆς γῆς ἑστῶτες.
\vs{5}καὶ εἴ τις αὐτοὺς θέλει ἀδικῆσαι πῦρ ἐκπορεύεται ἐκ τοῦ στόματος αὐτῶν καὶ κατεσθίει τοὺς ἐχθροὺς αὐτῶν· καὶ εἴ τις θελήσῃ αὐτοὺς ἀδικῆσαι, οὕτως δεῖ αὐτὸν ἀποκτανθῆναι.
\vs{6}οὗτοι ἔχουσιν τὴν ἐξουσίαν κλεῖσαι τὸν οὐρανόν, ἵνα μὴ ὑετὸς βρέχῃ τὰς ἡμέρας τῆς προφητείας αὐτῶν, καὶ ἐξουσίαν ἔχουσιν ἐπὶ τῶν ὑδάτων στρέφειν αὐτὰ εἰς αἷμα καὶ πατάξαι τὴν γῆν ἐν πάσῃ πληγῇ ὁσάκις ἐὰν θελήσωσιν.

\vs{7}Καὶ ὅταν τελέσωσιν τὴν μαρτυρίαν αὐτῶν, τὸ θηρίον τὸ ἀναβαῖνον ἐκ τῆς ἀβύσσου ποιήσει μετ᾽ αὐτῶν πόλεμον καὶ νικήσει αὐτοὺς καὶ ἀποκτενεῖ αὐτούς.
\vs{8}καὶ τὸ πτῶμα αὐτῶν ἐπὶ τῆς πλατείας τῆς πόλεως τῆς μεγάλης, ἥτις καλεῖται πνευματικῶς Σόδομα καὶ Αἴγυπτος, ὅπου καὶ ὁ Κύριος αὐτῶν ἐσταυρώθη.
\vs{9}καὶ βλέπουσιν ἐκ τῶν λαῶν καὶ φυλῶν καὶ γλωσσῶν καὶ ἐθνῶν τὸ πτῶμα αὐτῶν ἡμέρας τρεῖς καὶ ἥμισυ καὶ τὰ πτώματα αὐτῶν οὐκ ἀφίουσιν τεθῆναι εἰς μνῆμα.
\vs{10}καὶ οἱ κατοικοῦντες ἐπὶ τῆς γῆς χαίρουσιν ἐπ᾽ αὐτοῖς καὶ εὐφραίνονται καὶ δῶρα πέμψουσιν ἀλλήλοις, ὅτι οὗτοι οἱ δύο προφῆται ἐβασάνισαν τοὺς κατοικοῦντας ἐπὶ τῆς γῆς.

\vs{11}Καὶ μετὰ τὰς τρεῖς ἡμέρας καὶ ἥμισυ πνεῦμα ζωῆς ἐκ τοῦ Θεοῦ εἰσῆλθεν ἐν αὐτοῖς, καὶ ἔστησαν ἐπὶ τοὺς πόδας αὐτῶν, καὶ φόβος μέγας ἐπέπεσεν ἐπὶ τοὺς θεωροῦντας αὐτούς.
\vs{12}καὶ ἤκουσαν φωνῆς μεγάλης ἐκ τοῦ οὐρανοῦ λεγούσης αὐτοῖς· Ἀνάβατε ὧδε. καὶ ἀνέβησαν εἰς τὸν οὐρανὸν ἐν τῇ νεφέλῃ, καὶ ἐθεώρησαν αὐτοὺς οἱ ἐχθροὶ αὐτῶν.
\vs{13}Καὶ ἐν ἐκείνῃ τῇ ὥρᾳ ἐγένετο σεισμὸς μέγας καὶ τὸ δέκατον τῆς πόλεως ἔπεσεν καὶ ἀπεκτάνθησαν ἐν τῷ σεισμῷ ὀνόματα ἀνθρώπων χιλιάδες ἑπτά καὶ οἱ λοιποὶ ἔμφοβοι ἐγένοντο καὶ ἔδωκαν δόξαν τῷ Θεῷ τοῦ οὐρανοῦ.

\vs{14}Ἡ Οὐαὶ ἡ δευτέρα ἀπῆλθεν· ἰδοὺ ἡ Οὐαὶ ἡ τρίτη ἔρχεται ταχύ.

\vs{15}Καὶ ὁ ἕβδομος ἄγγελος ἐσάλπισεν· καὶ ἐγένοντο φωναὶ μεγάλαι ἐν τῷ οὐρανῷ λέγοντες· 
\begin{poetryblock}

\begin{quote}Ἐγένετο ἡ βασιλεία τοῦ κόσμου τοῦ Κυρίου ἡμῶν\end{quote} 

\begin{quote}καὶ τοῦ Χριστοῦ αὐτοῦ,\end{quote} 

\begin{quote}καὶ βασιλεύσει εἰς τοὺς αἰῶνας τῶν αἰώνων.\end{quote}
\end{poetryblock}

\vs{16}Καὶ οἱ εἴκοσι τέσσαρες πρεσβύτεροι οἱ ἐνώπιον τοῦ Θεοῦ καθήμενοι ἐπὶ τοὺς θρόνους αὐτῶν ἔπεσαν ἐπὶ τὰ πρόσωπα αὐτῶν καὶ προσεκύνησαν τῷ Θεῷ
\vs{17}λέγοντες· 
\begin{poetryblock}

\begin{quote}Εὐχαριστοῦμέν σοι, Κύριε ὁ Θεός ὁ Παντοκράτωρ,\end{quote} 

\begin{quote}ὁ ὢν καὶ ὁ ἦν,\end{quote} 

\begin{quote}ὅτι εἴληφας τὴν δύναμίν σου τὴν μεγάλην\end{quote} 

\begin{quote}καὶ ἐβασίλευσας.\end{quote}

\begin{quote} \vs{18}καὶ τὰ ἔθνη ὠργίσθησαν,\end{quote} 

\begin{quote}καὶ ἦλθεν ἡ ὀργή σου\end{quote} 

\begin{quote}καὶ ὁ καιρὸς τῶν νεκρῶν κριθῆναι\end{quote} 

\begin{quote}καὶ δοῦναι τὸν μισθὸν τοῖς δούλοις σου τοῖς προφήταις\end{quote} 

\begin{quote}καὶ τοῖς ἁγίοις καὶ τοῖς φοβουμένοις τὸ ὄνομά σου,\end{quote} 

\begin{quote}τοὺς μικροὺς καὶ τοὺς μεγάλους,\end{quote} 

\begin{quote}καὶ διαφθεῖραι τοὺς διαφθείροντας τὴν γῆν.\end{quote}
\end{poetryblock}

\vs{19}Καὶ ἠνοίγη ὁ ναὸς τοῦ Θεοῦ ὁ ἐν τῷ οὐρανῷ καὶ ὤφθη ἡ κιβωτὸς τῆς διαθήκης αὐτοῦ ἐν τῷ ναῷ αὐτοῦ, καὶ ἐγένοντο ἀστραπαὶ καὶ φωναὶ καὶ βρονταὶ καὶ σεισμὸς καὶ χάλαζα μεγάλη.

\ch{12}
Καὶ σημεῖον μέγα ὤφθη ἐν τῷ οὐρανῷ, γυνὴ περιβεβλημένη τὸν ἥλιον, καὶ ἡ σελήνη ὑποκάτω τῶν ποδῶν αὐτῆς καὶ ἐπὶ τῆς κεφαλῆς αὐτῆς στέφανος ἀστέρων δώδεκα,
\vs{2}καὶ ἐν γαστρὶ ἔχουσα, καὶ κράζει ὠδίνουσα καὶ βασανιζομένη τεκεῖν.
\vs{3}Καὶ ὤφθη ἄλλο σημεῖον ἐν τῷ οὐρανῷ, καὶ ἰδοὺ δράκων μέγας πυρρός ἔχων κεφαλὰς ἑπτὰ καὶ κέρατα δέκα καὶ ἐπὶ τὰς κεφαλὰς αὐτοῦ ἑπτὰ διαδήματα,
\vs{4}καὶ ἡ οὐρὰ αὐτοῦ σύρει τὸ τρίτον τῶν ἀστέρων τοῦ οὐρανοῦ καὶ ἔβαλεν αὐτοὺς εἰς τὴν γῆν. καὶ ὁ δράκων ἕστηκεν ἐνώπιον τῆς γυναικὸς τῆς μελλούσης τεκεῖν, ἵνα ὅταν τέκῃ τὸ τέκνον αὐτῆς καταφάγῃ.
\vs{5}Καὶ ἔτεκεν υἱόν ἄρσεν, ὃς μέλλει ποιμαίνειν πάντα τὰ ἔθνη ἐν ῥάβδῳ σιδηρᾷ. καὶ ἡρπάσθη τὸ τέκνον αὐτῆς πρὸς τὸν Θεὸν καὶ πρὸς τὸν θρόνον αὐτοῦ.
\vs{6}καὶ ἡ γυνὴ ἔφυγεν εἰς τὴν ἔρημον, ὅπου ἔχει ἐκεῖ τόπον ἡτοιμασμένον ἀπὸ τοῦ Θεοῦ, ἵνα ἐκεῖ τρέφωσιν αὐτὴν ἡμέρας χιλίας διακοσίας ἑξήκοντα.
\vs{7}Καὶ ἐγένετο πόλεμος ἐν τῷ οὐρανῷ, ὁ Μιχαὴλ καὶ οἱ ἄγγελοι αὐτοῦ τοῦ πολεμῆσαι μετὰ τοῦ δράκοντος. καὶ ὁ δράκων ἐπολέμησεν καὶ οἱ ἄγγελοι αὐτοῦ,
\vs{8}καὶ οὐκ ἴσχυσεν οὐδὲ τόπος εὑρέθη αὐτῶν ἔτι ἐν τῷ οὐρανῷ.
\vs{9}καὶ ἐβλήθη ὁ δράκων ὁ μέγας, ὁ ὄφις ὁ ἀρχαῖος, ὁ καλούμενος Διάβολος καὶ Ὁ Σατανᾶς, ὁ πλανῶν τὴν οἰκουμένην ὅλην, ἐβλήθη εἰς τὴν γῆν, καὶ οἱ ἄγγελοι αὐτοῦ μετ᾽ αὐτοῦ ἐβλήθησαν.
\vs{10}Καὶ ἤκουσα φωνὴν μεγάλην ἐν τῷ οὐρανῷ λέγουσαν· 
\begin{poetryblock}

\begin{quote}Ἄρτι ἐγένετο ἡ σωτηρία καὶ ἡ δύναμις\end{quote} 

\begin{quote}καὶ ἡ βασιλεία τοῦ Θεοῦ ἡμῶν\end{quote} 

\begin{quote}καὶ ἡ ἐξουσία τοῦ Χριστοῦ αὐτοῦ,\end{quote} 

\begin{quote}ὅτι ἐβλήθη ὁ κατήγωρ τῶν ἀδελφῶν ἡμῶν,\end{quote} 

\begin{quote}ὁ κατηγορῶν αὐτοὺς ἐνώπιον τοῦ Θεοῦ ἡμῶν ἡμέρας καὶ νυκτός.\end{quote}

\begin{quote} \vs{11}καὶ αὐτοὶ ἐνίκησαν αὐτὸν διὰ τὸ αἷμα τοῦ Ἀρνίου\end{quote} 

\begin{quote}καὶ διὰ τὸν λόγον τῆς μαρτυρίας αὐτῶν\end{quote} 

\begin{quote}καὶ οὐκ ἠγάπησαν τὴν ψυχὴν αὐτῶν ἄχρι θανάτου.\end{quote}

\begin{quote} \vs{12}διὰ τοῦτο εὐφραίνεσθε, οἱ οὐρανοὶ\end{quote} 

\begin{quote}καὶ οἱ ἐν αὐτοῖς σκηνοῦντες.\end{quote} 

\begin{quote}οὐαὶ τὴν γῆν καὶ τὴν θάλασσαν,\end{quote} 

\begin{quote}ὅτι κατέβη ὁ διάβολος πρὸς ὑμᾶς\end{quote} 

\begin{quote}ἔχων θυμὸν μέγαν,\end{quote} 

\begin{quote}εἰδὼς ὅτι ὀλίγον καιρὸν ἔχει.\end{quote}
\end{poetryblock}

\vs{13}Καὶ ὅτε εἶδεν ὁ δράκων ὅτι ἐβλήθη εἰς τὴν γῆν, ἐδίωξεν τὴν γυναῖκα ἥτις ἔτεκεν τὸν ἄρσενα.
\vs{14}καὶ ἐδόθησαν τῇ γυναικὶ αἱ δύο πτέρυγες τοῦ ἀετοῦ τοῦ μεγάλου, ἵνα πέτηται εἰς τὴν ἔρημον εἰς τὸν τόπον αὐτῆς, ὅπου τρέφεται ἐκεῖ καιρὸν καὶ καιροὺς καὶ ἥμισυ καιροῦ ἀπὸ προσώπου τοῦ ὄφεως.
\vs{15}Καὶ ἔβαλεν ὁ ὄφις ἐκ τοῦ στόματος αὐτοῦ ὀπίσω τῆς γυναικὸς ὕδωρ ὡς ποταμόν, ἵνα αὐτὴν ποταμοφόρητον ποιήσῃ.
\vs{16}καὶ ἐβοήθησεν ἡ γῆ τῇ γυναικί καὶ ἤνοιξεν ἡ γῆ τὸ στόμα αὐτῆς καὶ κατέπιεν τὸν ποταμὸν ὃν ἔβαλεν ὁ δράκων ἐκ τοῦ στόματος αὐτοῦ.
\vs{17}καὶ ὠργίσθη ὁ δράκων ἐπὶ τῇ γυναικί καὶ ἀπῆλθεν ποιῆσαι πόλεμον μετὰ τῶν λοιπῶν τοῦ σπέρματος αὐτῆς τῶν τηρούντων τὰς ἐντολὰς τοῦ Θεοῦ καὶ ἐχόντων τὴν μαρτυρίαν Ἰησοῦ.

\vs{18}Καὶ ἐστάθη ἐπὶ τὴν ἄμμον τῆς θαλάσσης.

\ch{13}
Καὶ εἶδον ἐκ τῆς θαλάσσης θηρίον ἀναβαῖνον, ἔχον κέρατα δέκα καὶ κεφαλὰς ἑπτά καὶ ἐπὶ τῶν κεράτων αὐτοῦ δέκα διαδήματα καὶ ἐπὶ τὰς κεφαλὰς αὐτοῦ ὀνόματα βλασφημίας.
\vs{2}καὶ τὸ θηρίον ὃ εἶδον ἦν ὅμοιον παρδάλει καὶ οἱ πόδες αὐτοῦ ὡς ἄρκου καὶ τὸ στόμα αὐτοῦ ὡς στόμα λέοντος. καὶ ἔδωκεν αὐτῷ ὁ δράκων τὴν δύναμιν αὐτοῦ καὶ τὸν θρόνον αὐτοῦ καὶ ἐξουσίαν μεγάλην.
\vs{3}Καὶ μίαν ἐκ τῶν κεφαλῶν αὐτοῦ ὡς ἐσφαγμένην εἰς θάνατον, καὶ ἡ πληγὴ τοῦ θανάτου αὐτοῦ ἐθεραπεύθη.

καὶ ἐθαυμάσθη ὅλη ἡ γῆ ὀπίσω τοῦ θηρίου
\vs{4}καὶ προσεκύνησαν τῷ δράκοντι, ὅτι ἔδωκεν τὴν ἐξουσίαν τῷ θηρίῳ, καὶ προσεκύνησαν τῷ θηρίῳ λέγοντες· Τίς ὅμοιος τῷ θηρίῳ καὶ τίς δύναται πολεμῆσαι μετ᾽ αὐτοῦ;

\vs{5}Καὶ ἐδόθη αὐτῷ στόμα λαλοῦν μεγάλα καὶ βλασφημίας καὶ ἐδόθη αὐτῷ ἐξουσία ποιῆσαι μῆνας τεσσεράκοντα καὶ δύο.
\vs{6}καὶ ἤνοιξεν τὸ στόμα αὐτοῦ εἰς βλασφημίας πρὸς τὸν Θεόν βλασφημῆσαι τὸ ὄνομα αὐτοῦ καὶ τὴν σκηνὴν αὐτοῦ, τοὺς ἐν τῷ οὐρανῷ σκηνοῦντας.
\vs{7}Καὶ ἐδόθη αὐτῷ ποιῆσαι πόλεμον μετὰ τῶν ἁγίων καὶ νικῆσαι αὐτούς, καὶ ἐδόθη αὐτῷ ἐξουσία ἐπὶ πᾶσαν φυλὴν καὶ λαὸν καὶ γλῶσσαν καὶ ἔθνος.
\vs{8}καὶ προσκυνήσουσιν αὐτὸν πάντες οἱ κατοικοῦντες ἐπὶ τῆς γῆς, οὗ οὐ γέγραπται τὸ ὄνομα αὐτοῦ ἐν τῷ βιβλίῳ τῆς ζωῆς τοῦ Ἀρνίου τοῦ ἐσφαγμένου ἀπὸ καταβολῆς κόσμου.

\vs{9}Εἴ τις ἔχει οὖς ἀκουσάτω.

\vs{10}Εἴ τις εἰς αἰχμαλωσίαν, εἰς αἰχμαλωσίαν ὑπάγει· 
\begin{poetryblock}

\begin{quote}εἴ τις ἐν μαχαίρῃ ἀποκτανθῆναι αὐτὸν ἐν μαχαίρῃ ἀποκτανθῆναι.\end{quote}
\end{poetryblock}

Ὧδέ ἐστιν ἡ ὑπομονὴ καὶ ἡ πίστις τῶν ἁγίων.

\vs{11}Καὶ εἶδον ἄλλο θηρίον ἀναβαῖνον ἐκ τῆς γῆς, καὶ εἶχεν κέρατα δύο ὅμοια ἀρνίῳ καὶ ἐλάλει ὡς δράκων.
\vs{12}καὶ τὴν ἐξουσίαν τοῦ πρώτου θηρίου πᾶσαν ποιεῖ ἐνώπιον αὐτοῦ, καὶ ποιεῖ τὴν γῆν καὶ τοὺς ἐν αὐτῇ κατοικοῦντας ἵνα προσκυνήσουσιν τὸ θηρίον τὸ πρῶτον, οὗ ἐθεραπεύθη ἡ πληγὴ τοῦ θανάτου αὐτοῦ.
\vs{13}Καὶ ποιεῖ σημεῖα μεγάλα, ἵνα καὶ πῦρ ποιῇ ἐκ τοῦ οὐρανοῦ καταβαίνειν εἰς τὴν γῆν ἐνώπιον τῶν ἀνθρώπων,
\vs{14}καὶ πλανᾷ τοὺς κατοικοῦντας ἐπὶ τῆς γῆς διὰ τὰ σημεῖα ἃ ἐδόθη αὐτῷ ποιῆσαι ἐνώπιον τοῦ θηρίου, λέγων τοῖς κατοικοῦσιν ἐπὶ τῆς γῆς ποιῆσαι εἰκόνα τῷ θηρίῳ, ὃς ἔχει τὴν πληγὴν τῆς μαχαίρης καὶ ἔζησεν.

\vs{15}καὶ ἐδόθη αὐτῷ δοῦναι πνεῦμα τῇ εἰκόνι τοῦ θηρίου, ἵνα καὶ λαλήσῃ ἡ εἰκὼν τοῦ θηρίου καὶ ποιήσῃ ἵνα ὅσοι ἐὰν μὴ προσκυνήσωσιν τῇ εἰκόνι τοῦ θηρίου ἀποκτανθῶσιν.
\vs{16}Καὶ ποιεῖ πάντας, τοὺς μικροὺς καὶ τοὺς μεγάλους, καὶ τοὺς πλουσίους καὶ τοὺς πτωχούς, καὶ τοὺς ἐλευθέρους καὶ τοὺς δούλους, ἵνα δῶσιν αὐτοῖς χάραγμα ἐπὶ τῆς χειρὸς αὐτῶν τῆς δεξιᾶς ἢ ἐπὶ τὸ μέτωπον αὐτῶν
\vs{17}καὶ ἵνα μή τις δύνηται ἀγοράσαι ἢ πωλῆσαι εἰ μὴ ὁ ἔχων τὸ χάραγμα τὸ ὄνομα τοῦ θηρίου ἢ τὸν ἀριθμὸν τοῦ ὀνόματος αὐτοῦ.

\vs{18}Ὧδε ἡ σοφία ἐστίν. ὁ ἔχων νοῦν ψηφισάτω τὸν ἀριθμὸν τοῦ θηρίου, ἀριθμὸς γὰρ ἀνθρώπου ἐστίν, καὶ ὁ ἀριθμὸς αὐτοῦ ἑξακόσιοι ἑξήκοντα ἕξ.

\ch{14}
Καὶ εἶδον, καὶ ἰδοὺ τὸ Ἀρνίον ἑστὸς ἐπὶ τὸ ὄρος Σιών καὶ μετ᾽ αὐτοῦ ἑκατὸν τεσσεράκοντα τέσσαρες χιλιάδες ἔχουσαι τὸ ὄνομα αὐτοῦ καὶ τὸ ὄνομα τοῦ Πατρὸς αὐτοῦ γεγραμμένον ἐπὶ τῶν μετώπων αὐτῶν.
\vs{2}καὶ ἤκουσα φωνὴν ἐκ τοῦ οὐρανοῦ ὡς φωνὴν ὑδάτων πολλῶν καὶ ὡς φωνὴν βροντῆς μεγάλης, καὶ ἡ φωνὴ ἣν ἤκουσα ὡς κιθαρῳδῶν κιθαριζόντων ἐν ταῖς κιθάραις αὐτῶν.
\vs{3}Καὶ ᾄδουσιν ὡς ᾠδὴν καινὴν ἐνώπιον τοῦ θρόνου καὶ ἐνώπιον τῶν τεσσάρων ζῴων καὶ τῶν πρεσβυτέρων, καὶ οὐδεὶς ἐδύνατο μαθεῖν τὴν ᾠδὴν εἰ μὴ αἱ ἑκατὸν τεσσεράκοντα τέσσαρες χιλιάδες, οἱ ἠγορασμένοι ἀπὸ τῆς γῆς.

\vs{4}οὗτοί εἰσιν οἳ μετὰ γυναικῶν οὐκ ἐμολύνθησαν, παρθένοι γάρ εἰσιν, οὗτοι οἱ ἀκολουθοῦντες τῷ Ἀρνίῳ ὅπου ἂν ὑπάγῃ. οὗτοι ἠγοράσθησαν ἀπὸ τῶν ἀνθρώπων ἀπαρχὴ τῷ Θεῷ καὶ τῷ Ἀρνίῳ,
\vs{5}καὶ ἐν τῷ στόματι αὐτῶν οὐχ εὑρέθη ψεῦδος, ἄμωμοί εἰσιν.

\vs{6}Καὶ εἶδον ἄλλον ἄγγελον πετόμενον ἐν μεσουρανήματι, ἔχοντα εὐαγγέλιον αἰώνιον εὐαγγελίσαι ἐπὶ τοὺς καθημένους ἐπὶ τῆς γῆς καὶ ἐπὶ πᾶν ἔθνος καὶ φυλὴν καὶ γλῶσσαν καὶ λαόν,
\vs{7}λέγων ἐν φωνῇ μεγάλῃ·

Φοβήθητε τὸν Θεὸν καὶ δότε αὐτῷ δόξαν, ὅτι ἦλθεν ἡ ὥρα τῆς κρίσεως αὐτοῦ, καὶ προσκυνήσατε τῷ ποιήσαντι τὸν οὐρανὸν καὶ τὴν γῆν καὶ θάλασσαν καὶ πηγὰς ὑδάτων.

\vs{8}Καὶ ἄλλος ἄγγελος δεύτερος ἠκολούθησεν λέγων· 
\begin{poetryblock}

\begin{quote}Ἔπεσεν ἔπεσεν Βαβυλὼν ἡ μεγάλη ἣ ἐκ τοῦ οἴνου τοῦ θυμοῦ τῆς πορνείας αὐτῆς πεπότικεν πάντα τὰ ἔθνη.\end{quote}
\end{poetryblock}

\vs{9}Καὶ ἄλλος ἄγγελος τρίτος ἠκολούθησεν αὐτοῖς λέγων ἐν φωνῇ μεγάλῃ·

Εἴ τις προσκυνεῖ τὸ θηρίον καὶ τὴν εἰκόνα αὐτοῦ καὶ λαμβάνει χάραγμα ἐπὶ τοῦ μετώπου αὐτοῦ ἢ ἐπὶ τὴν χεῖρα αὐτοῦ,
\vs{10}καὶ αὐτὸς πίεται ἐκ τοῦ οἴνου τοῦ θυμοῦ τοῦ Θεοῦ τοῦ κεκερασμένου ἀκράτου ἐν τῷ ποτηρίῳ τῆς ὀργῆς αὐτοῦ καὶ βασανισθήσεται ἐν πυρὶ καὶ θείῳ ἐνώπιον ἀγγέλων ἁγίων καὶ ἐνώπιον τοῦ Ἀρνίου.
\vs{11}καὶ ὁ καπνὸς τοῦ βασανισμοῦ αὐτῶν εἰς αἰῶνας αἰώνων ἀναβαίνει, καὶ οὐκ ἔχουσιν ἀνάπαυσιν ἡμέρας καὶ νυκτός οἱ προσκυνοῦντες τὸ θηρίον καὶ τὴν εἰκόνα αὐτοῦ καὶ εἴ τις λαμβάνει τὸ χάραγμα τοῦ ὀνόματος αὐτοῦ.
\vs{12}Ὧδε ἡ ὑπομονὴ τῶν ἁγίων ἐστίν, οἱ τηροῦντες τὰς ἐντολὰς τοῦ Θεοῦ καὶ τὴν πίστιν Ἰησοῦ.

\vs{13}Καὶ ἤκουσα φωνῆς ἐκ τοῦ οὐρανοῦ λεγούσης· Γράψον· 
\begin{poetryblock}

\begin{quote}Μακάριοι οἱ νεκροὶ οἱ ἐν Κυρίῳ ἀποθνῄσκοντες ἀπ᾽ ἄρτι. Ναί, λέγει τὸ Πνεῦμα, Ἵνα ἀναπαήσονται ἐκ τῶν κόπων αὐτῶν, τὰ γὰρ ἔργα αὐτῶν ἀκολουθεῖ μετ᾽ αὐτῶν.\end{quote}
\end{poetryblock}

\vs{14}Καὶ εἶδον, καὶ ἰδοὺ νεφέλη λευκή, καὶ ἐπὶ τὴν νεφέλην καθήμενον ὅμοιον υἱὸν ἀνθρώπου, ἔχων ἐπὶ τῆς κεφαλῆς αὐτοῦ στέφανον χρυσοῦν καὶ ἐν τῇ χειρὶ αὐτοῦ δρέπανον ὀξύ.
\vs{15}Καὶ ἄλλος ἄγγελος ἐξῆλθεν ἐκ τοῦ ναοῦ κράζων ἐν φωνῇ μεγάλῃ τῷ καθημένῳ ἐπὶ τῆς νεφέλης·

Πέμψον τὸ δρέπανόν σου καὶ θέρισον, ὅτι ἦλθεν ἡ ὥρα θερίσαι, ὅτι ἐξηράνθη ὁ θερισμὸς τῆς γῆς.
\vs{16}καὶ ἔβαλεν ὁ καθήμενος ἐπὶ τῆς νεφέλης τὸ δρέπανον αὐτοῦ ἐπὶ τὴν γῆν καὶ ἐθερίσθη ἡ γῆ.

\vs{17}Καὶ ἄλλος ἄγγελος ἐξῆλθεν ἐκ τοῦ ναοῦ τοῦ ἐν τῷ οὐρανῷ ἔχων καὶ αὐτὸς δρέπανον ὀξύ.
\vs{18}Καὶ ἄλλος ἄγγελος ἐξῆλθεν ἐκ τοῦ θυσιαστηρίου ὁ ἔχων ἐξουσίαν ἐπὶ τοῦ πυρός, καὶ ἐφώνησεν φωνῇ μεγάλῃ τῷ ἔχοντι τὸ δρέπανον τὸ ὀξὺ λέγων· Πέμψον σου τὸ δρέπανον τὸ ὀξὺ καὶ τρύγησον τοὺς βότρυας τῆς ἀμπέλου τῆς γῆς, ὅτι ἤκμασαν αἱ σταφυλαὶ αὐτῆς.
\vs{19}Καὶ ἔβαλεν ὁ ἄγγελος τὸ δρέπανον αὐτοῦ εἰς τὴν γῆν καὶ ἐτρύγησεν τὴν ἄμπελον τῆς γῆς καὶ ἔβαλεν εἰς τὴν ληνὸν τοῦ θυμοῦ τοῦ Θεοῦ τὸν μέγαν.
\vs{20}καὶ ἐπατήθη ἡ ληνὸς ἔξωθεν τῆς πόλεως καὶ ἐξῆλθεν αἷμα ἐκ τῆς ληνοῦ ἄχρι τῶν χαλινῶν τῶν ἵππων ἀπὸ σταδίων χιλίων ἑξακοσίων.

\ch{15}
Καὶ εἶδον ἄλλο σημεῖον ἐν τῷ οὐρανῷ μέγα καὶ θαυμαστόν, ἀγγέλους ἑπτὰ ἔχοντας πληγὰς ἑπτὰ τὰς ἐσχάτας, ὅτι ἐν αὐταῖς ἐτελέσθη ὁ θυμὸς τοῦ Θεοῦ.

\vs{2}Καὶ εἶδον ὡς θάλασσαν ὑαλίνην μεμιγμένην πυρί καὶ τοὺς νικῶντας ἐκ τοῦ θηρίου καὶ ἐκ τῆς εἰκόνος αὐτοῦ καὶ ἐκ τοῦ ἀριθμοῦ τοῦ ὀνόματος αὐτοῦ ἑστῶτας ἐπὶ τὴν θάλασσαν τὴν ὑαλίνην ἔχοντας κιθάρας τοῦ Θεοῦ.
\vs{3}καὶ ᾄδουσιν τὴν ᾠδὴν Μωϋσέως τοῦ δούλου τοῦ Θεοῦ καὶ τὴν ᾠδὴν τοῦ Ἀρνίου λέγοντες· 
\begin{poetryblock}

\begin{quote}Μεγάλα καὶ θαυμαστὰ τὰ ἔργα σου,\end{quote} 

\begin{quote}Κύριε ὁ Θεός ὁ Παντοκράτωρ·\end{quote} 

\begin{quote}δίκαιαι καὶ ἀληθιναὶ αἱ ὁδοί σου,\end{quote} 

\begin{quote}ὁ Βασιλεὺς τῶν ἐθνῶν·\end{quote}

\begin{quote} \vs{4}τίς οὐ μὴ φοβηθῇ, Κύριε,\end{quote} 

\begin{quote}καὶ δοξάσει τὸ ὄνομά σου;\end{quote} 

\begin{quote}ὅτι μόνος ὅσιος,\end{quote} 

\begin{quote}ὅτι πάντα τὰ ἔθνη ἥξουσιν\end{quote} 

\begin{quote}καὶ προσκυνήσουσιν ἐνώπιόν σου,\end{quote} 

\begin{quote}ὅτι τὰ δικαιώματά σου ἐφανερώθησαν.\end{quote}
\end{poetryblock}

\vs{5}Καὶ μετὰ ταῦτα εἶδον, καὶ ἠνοίγη ὁ ναὸς τῆς σκηνῆς τοῦ μαρτυρίου ἐν τῷ οὐρανῷ,
\vs{6}καὶ ἐξῆλθον οἱ ἑπτὰ ἄγγελοι οἱ ἔχοντες τὰς ἑπτὰ πληγὰς ἐκ τοῦ ναοῦ ἐνδεδυμένοι λίνον καθαρὸν λαμπρὸν καὶ περιεζωσμένοι περὶ τὰ στήθη ζώνας χρυσᾶς.
\vs{7}Καὶ ἓν ἐκ τῶν τεσσάρων ζῴων ἔδωκεν τοῖς ἑπτὰ ἀγγέλοις ἑπτὰ φιάλας χρυσᾶς γεμούσας τοῦ θυμοῦ τοῦ Θεοῦ τοῦ ζῶντος εἰς τοὺς αἰῶνας τῶν αἰώνων.
\vs{8}καὶ ἐγεμίσθη ὁ ναὸς καπνοῦ ἐκ τῆς δόξης τοῦ Θεοῦ καὶ ἐκ τῆς δυνάμεως αὐτοῦ, καὶ οὐδεὶς ἐδύνατο εἰσελθεῖν εἰς τὸν ναὸν ἄχρι τελεσθῶσιν αἱ ἑπτὰ πληγαὶ τῶν ἑπτὰ ἀγγέλων.

\ch{16}
Καὶ ἤκουσα μεγάλης φωνῆς ἐκ τοῦ ναοῦ λεγούσης τοῖς ἑπτὰ ἀγγέλοις· Ὑπάγετε καὶ ἐκχέετε τὰς ἑπτὰ φιάλας τοῦ θυμοῦ τοῦ Θεοῦ εἰς τὴν γῆν.

\vs{2}Καὶ ἀπῆλθεν ὁ πρῶτος καὶ ἐξέχεεν τὴν φιάλην αὐτοῦ εἰς τὴν γῆν, καὶ ἐγένετο ἕλκος κακὸν καὶ πονηρὸν ἐπὶ τοὺς ἀνθρώπους τοὺς ἔχοντας τὸ χάραγμα τοῦ θηρίου καὶ τοὺς προσκυνοῦντας τῇ εἰκόνι αὐτοῦ.

\vs{3}Καὶ ὁ δεύτερος ἐξέχεεν τὴν φιάλην αὐτοῦ εἰς τὴν θάλασσαν, καὶ ἐγένετο αἷμα ὡς νεκροῦ, καὶ πᾶσα ψυχὴ ζωῆς ἀπέθανεν τὰ ἐν τῇ θαλάσσῃ.

\vs{4}Καὶ ὁ τρίτος ἐξέχεεν τὴν φιάλην αὐτοῦ εἰς τοὺς ποταμοὺς καὶ τὰς πηγὰς τῶν ὑδάτων, καὶ ἐγένετο αἷμα.
\vs{5}Καὶ ἤκουσα τοῦ ἀγγέλου τῶν ὑδάτων λέγοντος· 
\begin{poetryblock}

\begin{quote}Δίκαιος εἶ, ὁ ὢν καὶ ὁ ἦν, ὁ Ὅσιος,\end{quote} 

\begin{quote}ὅτι ταῦτα ἔκρινας,\end{quote}

\begin{quote} \vs{6}ὅτι αἷμα ἁγίων καὶ προφητῶν ἐξέχεαν\end{quote} 

\begin{quote}καὶ αἷμα αὐτοῖς δέδωκας πιεῖν,\end{quote} 

\begin{quote}ἄξιοί εἰσιν.\end{quote}
\end{poetryblock}

\vs{7}Καὶ ἤκουσα τοῦ θυσιαστηρίου λέγοντος· 
\begin{poetryblock}

\begin{quote}Ναί Κύριε ὁ Θεός ὁ Παντοκράτωρ,\end{quote} 

\begin{quote}ἀληθιναὶ καὶ δίκαιαι αἱ κρίσεις σου.\end{quote}
\end{poetryblock}

\vs{8}Καὶ ὁ τέταρτος ἐξέχεεν τὴν φιάλην αὐτοῦ ἐπὶ τὸν ἥλιον, καὶ ἐδόθη αὐτῷ καυματίσαι τοὺς ἀνθρώπους ἐν πυρί.
\vs{9}καὶ ἐκαυματίσθησαν οἱ ἄνθρωποι καῦμα μέγα καὶ ἐβλασφήμησαν τὸ ὄνομα τοῦ Θεοῦ τοῦ ἔχοντος τὴν ἐξουσίαν ἐπὶ τὰς πληγὰς ταύτας καὶ οὐ μετενόησαν δοῦναι αὐτῷ δόξαν.

\vs{10}Καὶ ὁ πέμπτος ἐξέχεεν τὴν φιάλην αὐτοῦ ἐπὶ τὸν θρόνον τοῦ θηρίου, καὶ ἐγένετο ἡ βασιλεία αὐτοῦ ἐσκοτωμένη, καὶ ἐμασῶντο τὰς γλώσσας αὐτῶν ἐκ τοῦ πόνου,
\vs{11}καὶ ἐβλασφήμησαν τὸν Θεὸν τοῦ οὐρανοῦ ἐκ τῶν πόνων αὐτῶν καὶ ἐκ τῶν ἑλκῶν αὐτῶν καὶ οὐ μετενόησαν ἐκ τῶν ἔργων αὐτῶν.

\vs{12}Καὶ ὁ ἕκτος ἐξέχεεν τὴν φιάλην αὐτοῦ ἐπὶ τὸν ποταμὸν τὸν μέγαν τὸν Εὐφράτην, καὶ ἐξηράνθη τὸ ὕδωρ αὐτοῦ, ἵνα ἑτοιμασθῇ ἡ ὁδὸς τῶν βασιλέων τῶν ἀπὸ ἀνατολῆς ἡλίου.
\vs{13}Καὶ εἶδον ἐκ τοῦ στόματος τοῦ δράκοντος καὶ ἐκ τοῦ στόματος τοῦ θηρίου καὶ ἐκ τοῦ στόματος τοῦ ψευδοπροφήτου πνεύματα τρία ἀκάθαρτα ὡς βάτραχοι·
\vs{14}εἰσὶν γὰρ πνεύματα δαιμονίων ποιοῦντα σημεῖα, ἃ ἐκπορεύεται ἐπὶ τοὺς βασιλεῖς τῆς οἰκουμένης ὅλης συναγαγεῖν αὐτοὺς εἰς τὸν πόλεμον τῆς ἡμέρας τῆς μεγάλης τοῦ Θεοῦ τοῦ Παντοκράτορος.
\vs{15}Ἰδοὺ ἔρχομαι ὡς κλέπτης. μακάριος ὁ γρηγορῶν καὶ τηρῶν τὰ ἱμάτια αὐτοῦ, ἵνα μὴ γυμνὸς περιπατῇ καὶ βλέπωσιν τὴν ἀσχημοσύνην αὐτοῦ.
\vs{16}Καὶ συνήγαγεν αὐτοὺς εἰς τὸν τόπον τὸν καλούμενον Ἑβραϊστὶ Ἁρμαγεδών.

\vs{17}Καὶ ὁ ἕβδομος ἐξέχεεν τὴν φιάλην αὐτοῦ ἐπὶ τὸν ἀέρα, καὶ ἐξῆλθεν φωνὴ μεγάλη ἐκ τοῦ ναοῦ ἀπὸ τοῦ θρόνου λέγουσα· Γέγονεν.
\vs{18}Καὶ ἐγένοντο ἀστραπαὶ καὶ φωναὶ καὶ βρονταί καὶ σεισμὸς ἐγένετο μέγας, οἷος οὐκ ἐγένετο ἀφ᾽ οὗ ἄνθρωπος ἐγένετο ἐπὶ τῆς γῆς τηλικοῦτος σεισμὸς οὕτω μέγας.
\vs{19}καὶ ἐγένετο ἡ πόλις ἡ μεγάλη εἰς τρία μέρη καὶ αἱ πόλεις τῶν ἐθνῶν ἔπεσαν. καὶ Βαβυλὼν ἡ μεγάλη ἐμνήσθη ἐνώπιον τοῦ Θεοῦ δοῦναι αὐτῇ τὸ ποτήριον τοῦ οἴνου τοῦ θυμοῦ τῆς ὀργῆς αὐτοῦ.
\vs{20}Καὶ πᾶσα νῆσος ἔφυγεν καὶ ὄρη οὐχ εὑρέθησαν.
\vs{21}καὶ χάλαζα μεγάλη ὡς ταλαντιαία καταβαίνει ἐκ τοῦ οὐρανοῦ ἐπὶ τοὺς ἀνθρώπους, καὶ ἐβλασφήμησαν οἱ ἄνθρωποι τὸν Θεὸν ἐκ τῆς πληγῆς τῆς χαλάζης, ὅτι μεγάλη ἐστὶν ἡ πληγὴ αὐτῆς σφόδρα.

\ch{17}
Καὶ ἦλθεν εἷς ἐκ τῶν ἑπτὰ ἀγγέλων τῶν ἐχόντων τὰς ἑπτὰ φιάλας καὶ ἐλάλησεν μετ᾽ ἐμοῦ λέγων· Δεῦρο, δείξω σοι τὸ κρίμα τῆς πόρνης τῆς μεγάλης τῆς καθημένης ἐπὶ ὑδάτων πολλῶν,
\vs{2}μεθ᾽ ἧς ἐπόρνευσαν οἱ βασιλεῖς τῆς γῆς καὶ ἐμεθύσθησαν οἱ κατοικοῦντες τὴν γῆν ἐκ τοῦ οἴνου τῆς πορνείας αὐτῆς.
\vs{3}Καὶ ἀπήνεγκέν με εἰς ἔρημον ἐν Πνεύματι.

καὶ εἶδον γυναῖκα καθημένην ἐπὶ θηρίον κόκκινον, γέμοντα ὀνόματα βλασφημίας, ἔχων κεφαλὰς ἑπτὰ καὶ κέρατα δέκα.
\vs{4}καὶ ἡ γυνὴ ἦν περιβεβλημένη πορφυροῦν καὶ κόκκινον καὶ κεχρυσωμένη χρυσίῳ καὶ λίθῳ τιμίῳ καὶ μαργαρίταις, ἔχουσα ποτήριον χρυσοῦν ἐν τῇ χειρὶ αὐτῆς γέμον βδελυγμάτων καὶ τὰ ἀκάθαρτα τῆς πορνείας αὐτῆς
\vs{5}καὶ ἐπὶ τὸ μέτωπον αὐτῆς ὄνομα γεγραμμένον, μυστήριον, ΒΑΒΥΛΩΝ ἡ ΜΕΓΑΛΗ, ἡ ΜΗΤΗΡ ΤΩΝ ΠΟΡΝΩΝ ΚΑΙ ΤΩΝ ΒΔΕΛΥΓΜΑΤΩΝ ΤΗΣ ΓΗΣ.
\vs{6}Καὶ εἶδον τὴν γυναῖκα μεθύουσαν ἐκ τοῦ αἵματος τῶν ἁγίων καὶ ἐκ τοῦ αἵματος τῶν μαρτύρων Ἰησοῦ. Καὶ ἐθαύμασα ἰδὼν αὐτὴν θαῦμα μέγα.

\vs{7}Καὶ εἶπέν μοι ὁ ἄγγελος· Διὰ τί ἐθαύμασας; ἐγὼ ἐρῶ σοι τὸ μυστήριον τῆς γυναικὸς καὶ τοῦ θηρίου τοῦ βαστάζοντος αὐτήν τοῦ ἔχοντος τὰς ἑπτὰ κεφαλὰς καὶ τὰ δέκα κέρατα.

\vs{8}Τὸ θηρίον ὃ εἶδες ἦν καὶ οὐκ ἔστιν καὶ μέλλει ἀναβαίνειν ἐκ τῆς ἀβύσσου καὶ εἰς ἀπώλειαν ὑπάγει, καὶ θαυμασθήσονται οἱ κατοικοῦντες ἐπὶ τῆς γῆς, ὧν οὐ γέγραπται τὸ ὄνομα ἐπὶ τὸ βιβλίον τῆς ζωῆς ἀπὸ καταβολῆς κόσμου, βλεπόντων τὸ θηρίον ὅτι ἦν καὶ οὐκ ἔστιν καὶ παρέσται.
\vs{9}Ὧδε ὁ νοῦς ὁ ἔχων σοφίαν. αἱ ἑπτὰ κεφαλαὶ ἑπτὰ ὄρη εἰσίν, ὅπου ἡ γυνὴ κάθηται ἐπ᾽ αὐτῶν. καὶ βασιλεῖς ἑπτά εἰσιν·
\vs{10}οἱ πέντε ἔπεσαν, ὁ εἷς ἔστιν, ὁ ἄλλος οὔπω ἦλθεν, καὶ ὅταν ἔλθῃ ὀλίγον αὐτὸν δεῖ μεῖναι.
\vs{11}Καὶ τὸ θηρίον ὃ ἦν καὶ οὐκ ἔστιν καὶ αὐτὸς ὄγδοός ἐστιν καὶ ἐκ τῶν ἑπτά ἐστιν, καὶ εἰς ἀπώλειαν ὑπάγει.
\vs{12}καὶ τὰ δέκα κέρατα ἃ εἶδες δέκα βασιλεῖς εἰσιν, οἵτινες βασιλείαν οὔπω ἔλαβον, ἀλλὰ ἐξουσίαν ὡς βασιλεῖς μίαν ὥραν λαμβάνουσιν μετὰ τοῦ θηρίου.
\vs{13}οὗτοι μίαν γνώμην ἔχουσιν καὶ τὴν δύναμιν καὶ ἐξουσίαν αὐτῶν τῷ θηρίῳ διδόασιν.
\vs{14}Οὗτοι μετὰ τοῦ Ἀρνίου πολεμήσουσιν καὶ τὸ Ἀρνίον νικήσει αὐτούς, ὅτι Κύριος κυρίων ἐστὶν καὶ Βασιλεὺς βασιλέων καὶ οἱ μετ᾽ αὐτοῦ κλητοὶ καὶ ἐκλεκτοὶ καὶ πιστοί.

\vs{15}Καὶ λέγει μοι· Τὰ ὕδατα ἃ εἶδες οὗ ἡ πόρνη κάθηται, λαοὶ καὶ ὄχλοι εἰσὶν καὶ ἔθνη καὶ γλῶσσαι.
\vs{16}καὶ τὰ δέκα κέρατα ἃ εἶδες καὶ τὸ θηρίον οὗτοι μισήσουσιν τὴν πόρνην καὶ ἠρημωμένην ποιήσουσιν αὐτὴν καὶ γυμνήν καὶ τὰς σάρκας αὐτῆς φάγονται καὶ αὐτὴν κατακαύσουσιν ἐν πυρί.
\vs{17}ὁ γὰρ Θεὸς ἔδωκεν εἰς τὰς καρδίας αὐτῶν ποιῆσαι τὴν γνώμην αὐτοῦ καὶ ποιῆσαι μίαν γνώμην καὶ δοῦναι τὴν βασιλείαν αὐτῶν τῷ θηρίῳ ἄχρι τελεσθήσονται οἱ λόγοι τοῦ Θεοῦ.
\vs{18}καὶ ἡ γυνὴ ἣν εἶδες ἔστιν ἡ πόλις ἡ μεγάλη ἡ ἔχουσα βασιλείαν ἐπὶ τῶν βασιλέων τῆς γῆς.

\ch{18}
Μετὰ ταῦτα εἶδον ἄλλον ἄγγελον καταβαίνοντα ἐκ τοῦ οὐρανοῦ ἔχοντα ἐξουσίαν μεγάλην, καὶ ἡ γῆ ἐφωτίσθη ἐκ τῆς δόξης αὐτοῦ.
\vs{2}καὶ ἔκραξεν ἐν ἰσχυρᾷ φωνῇ λέγων·

Ἔπεσεν ἔπεσεν Βαβυλὼν ἡ μεγάλη, καὶ ἐγένετο κατοικητήριον δαιμονίων καὶ φυλακὴ παντὸς πνεύματος ἀκαθάρτου καὶ φυλακὴ παντὸς ὀρνέου ἀκαθάρτου καὶ φυλακὴ παντὸς θηρίου ἀκαθάρτου καὶ μεμισημένου,
\vs{3}ὅτι ἐκ τοῦ οἴνου τοῦ θυμοῦ τῆς πορνείας αὐτῆς πέπωκαν πάντα τὰ ἔθνη καὶ οἱ βασιλεῖς τῆς γῆς μετ᾽ αὐτῆς ἐπόρνευσαν καὶ οἱ ἔμποροι τῆς γῆς ἐκ τῆς δυνάμεως τοῦ στρήνους αὐτῆς ἐπλούτησαν.

\vs{4}Καὶ ἤκουσα ἄλλην φωνὴν ἐκ τοῦ οὐρανοῦ λέγουσαν·

Ἐξέλθατε ὁ λαός μου ἐξ αὐτῆς ἵνα μὴ συνκοινωνήσητε ταῖς ἁμαρτίαις αὐτῆς, καὶ ἐκ τῶν πληγῶν αὐτῆς ἵνα μὴ λάβητε,
\vs{5}ὅτι ἐκολλήθησαν αὐτῆς αἱ ἁμαρτίαι ἄχρι τοῦ οὐρανοῦ καὶ ἐμνημόνευσεν ὁ Θεὸς τὰ ἀδικήματα αὐτῆς.
\vs{6}ἀπόδοτε αὐτῇ ὡς καὶ αὐτὴ ἀπέδωκεν καὶ διπλώσατε τὰ διπλᾶ κατὰ τὰ ἔργα αὐτῆς, ἐν τῷ ποτηρίῳ ᾧ ἐκέρασεν κεράσατε αὐτῇ διπλοῦν,
\vs{7}ὅσα ἐδόξασεν αὑτὴν καὶ ἐστρηνίασεν, τοσοῦτον δότε αὐτῇ βασανισμὸν καὶ πένθος. ὅτι ἐν τῇ καρδίᾳ αὐτῆς λέγει ὅτι Κάθημαι βασίλισσα καὶ χήρα οὐκ εἰμί καὶ πένθος οὐ μὴ ἴδω.
\vs{8}διὰ τοῦτο ἐν μιᾷ ἡμέρᾳ ἥξουσιν αἱ πληγαὶ αὐτῆς, θάνατος καὶ πένθος καὶ λιμός, καὶ ἐν πυρὶ κατακαυθήσεται, ὅτι ἰσχυρὸς Κύριος ὁ Θεὸς ὁ κρίνας αὐτήν.

\vs{9}Καὶ κλαύσουσιν καὶ κόψονται ἐπ᾽ αὐτὴν οἱ βασιλεῖς τῆς γῆς οἱ μετ᾽ αὐτῆς πορνεύσαντες καὶ στρηνιάσαντες, ὅταν βλέπωσιν τὸν καπνὸν τῆς πυρώσεως αὐτῆς,
\vs{10}ἀπὸ μακρόθεν ἑστηκότες διὰ τὸν φόβον τοῦ βασανισμοῦ αὐτῆς λέγοντες· 
\begin{poetryblock}

\begin{quote}Οὐαὶ οὐαί, ἡ πόλις ἡ μεγάλη,\end{quote} 

\begin{quote}Βαβυλὼν ἡ πόλις ἡ ἰσχυρά,\end{quote} 

\begin{quote}ὅτι μιᾷ ὥρᾳ ἦλθεν ἡ κρίσις σου.\end{quote}
\end{poetryblock}

\vs{11}Καὶ οἱ ἔμποροι τῆς γῆς κλαίουσιν καὶ πενθοῦσιν ἐπ᾽ αὐτήν, ὅτι τὸν γόμον αὐτῶν οὐδεὶς ἀγοράζει οὐκέτι
\vs{12}γόμον χρυσοῦ καὶ ἀργύρου καὶ λίθου τιμίου καὶ μαργαριτῶν καὶ βυσσίνου καὶ πορφύρας καὶ σιρικοῦ καὶ κοκκίνου, καὶ πᾶν ξύλον θύϊνον καὶ πᾶν σκεῦος ἐλεφάντινον καὶ πᾶν σκεῦος ἐκ ξύλου τιμιωτάτου καὶ χαλκοῦ καὶ σιδήρου καὶ μαρμάρου,
\vs{13}καὶ κιννάμωμον καὶ ἄμωμον καὶ θυμιάματα καὶ μύρον καὶ λίβανον καὶ οἶνον καὶ ἔλαιον καὶ σεμίδαλιν καὶ σῖτον καὶ κτήνη καὶ πρόβατα, καὶ ἵππων καὶ ῥεδῶν καὶ σωμάτων, καὶ ψυχὰς ἀνθρώπων.
\begin{poetryblock}

\begin{quote} \vs{14}Καὶ ἡ ὀπώρα σου τῆς ἐπιθυμίας τῆς ψυχῆς ἀπῆλθεν ἀπὸ σοῦ,\end{quote} 

\begin{quote}καὶ πάντα τὰ λιπαρὰ καὶ τὰ λαμπρὰ ἀπώλετο ἀπὸ σοῦ\end{quote} 

\begin{quote}καὶ οὐκέτι οὐ μὴ αὐτὰ εὑρήσουσιν.\end{quote}
\end{poetryblock}

\vs{15}Οἱ ἔμποροι τούτων οἱ πλουτήσαντες ἀπ᾽ αὐτῆς ἀπὸ μακρόθεν στήσονται διὰ τὸν φόβον τοῦ βασανισμοῦ αὐτῆς κλαίοντες καὶ πενθοῦντες
\vs{16}λέγοντες· 
\begin{poetryblock}

\begin{quote}Οὐαὶ οὐαί, ἡ πόλις ἡ μεγάλη,\end{quote} 

\begin{quote}ἡ περιβεβλημένη βύσσινον καὶ πορφυροῦν καὶ κόκκινον\end{quote} 

\begin{quote}καὶ κεχρυσωμένη ἐν χρυσίῳ καὶ λίθῳ τιμίῳ καὶ μαργαρίτῃ,\end{quote}
\end{poetryblock}

\vs{17}ὅτι μιᾷ ὥρᾳ ἠρημώθη ὁ τοσοῦτος πλοῦτος.

Καὶ πᾶς κυβερνήτης καὶ πᾶς ὁ ἐπὶ τόπον πλέων καὶ ναῦται καὶ ὅσοι τὴν θάλασσαν ἐργάζονται, ἀπὸ μακρόθεν ἔστησαν
\vs{18}καὶ ἔκραζον βλέποντες τὸν καπνὸν τῆς πυρώσεως αὐτῆς λέγοντες· Τίς ὁμοία τῇ πόλει τῇ μεγάλῃ;
\vs{19}Καὶ ἔβαλον χοῦν ἐπὶ τὰς κεφαλὰς αὐτῶν καὶ ἔκραζον κλαίοντες καὶ πενθοῦντες λέγοντες· 
\begin{poetryblock}

\begin{quote}Οὐαὶ οὐαί, ἡ πόλις ἡ μεγάλη,\end{quote} 

\begin{quote}ἐν ᾗ ἐπλούτησαν πάντες οἱ ἔχοντες τὰ πλοῖα ἐν τῇ θαλάσσῃ ἐκ τῆς τιμιότητος αὐτῆς,\end{quote} 

\begin{quote}ὅτι μιᾷ ὥρᾳ ἠρημώθη.\end{quote}

\begin{quote} \vs{20}Εὐφραίνου ἐπ᾽ αὐτῇ, οὐρανέ\end{quote} 

\begin{quote}καὶ οἱ ἅγιοι καὶ οἱ ἀπόστολοι καὶ οἱ προφῆται,\end{quote} 

\begin{quote}ὅτι ἔκρινεν ὁ Θεὸς τὸ κρίμα ὑμῶν ἐξ αὐτῆς.\end{quote}
\end{poetryblock}

\vs{21}Καὶ ἦρεν εἷς ἄγγελος ἰσχυρὸς λίθον ὡς μύλινον μέγαν καὶ ἔβαλεν εἰς τὴν θάλασσαν λέγων· 
\begin{poetryblock}

\begin{quote}Οὕτως ὁρμήματι βληθήσεται Βαβυλὼν ἡ μεγάλη πόλις\end{quote} 

\begin{quote}καὶ οὐ μὴ εὑρεθῇ ἔτι.\end{quote}

\begin{quote} \vs{22}καὶ φωνὴ κιθαρῳδῶν καὶ μουσικῶν καὶ αὐλητῶν καὶ σαλπιστῶν\end{quote} 

\begin{quote}οὐ μὴ ἀκουσθῇ ἐν σοὶ ἔτι,\end{quote} 

\begin{quote}καὶ πᾶς τεχνίτης πάσης τέχνης\end{quote} 

\begin{quote}οὐ μὴ εὑρεθῇ ἐν σοὶ ἔτι,\end{quote} 

\begin{quote}καὶ φωνὴ μύλου\end{quote} 

\begin{quote}οὐ μὴ ἀκουσθῇ ἐν σοὶ ἔτι,\end{quote}

\begin{quote} \vs{23}καὶ φῶς λύχνου\end{quote} 

\begin{quote}οὐ μὴ φάνῃ ἐν σοὶ ἔτι,\end{quote} 

\begin{quote}καὶ φωνὴ νυμφίου καὶ νύμφης\end{quote} 

\begin{quote}οὐ μὴ ἀκουσθῇ ἐν σοὶ ἔτι·\end{quote} 

\begin{quote}ὅτι οἱ ἔμποροί σου ἦσαν οἱ μεγιστᾶνες τῆς γῆς,\end{quote} 

\begin{quote}ὅτι ἐν τῇ φαρμακείᾳ σου ἐπλανήθησαν πάντα τὰ ἔθνη,\end{quote}

\begin{quote} \vs{24}Καὶ ἐν αὐτῇ αἷμα προφητῶν καὶ ἁγίων εὑρέθη\end{quote} 

\begin{quote}καὶ πάντων τῶν ἐσφαγμένων ἐπὶ τῆς γῆς.\end{quote}
\end{poetryblock}

\ch{19}
Μετὰ ταῦτα ἤκουσα ὡς φωνὴν μεγάλην ὄχλου πολλοῦ ἐν τῷ οὐρανῷ λεγόντων· 
\begin{poetryblock}

\begin{quote}Ἁλληλουϊά·\end{quote} 

\begin{quote}ἡ σωτηρία καὶ ἡ δόξα καὶ ἡ δύναμις τοῦ Θεοῦ ἡμῶν,\end{quote}

\begin{quote} \vs{2}ὅτι ἀληθιναὶ καὶ δίκαιαι αἱ κρίσεις αὐτοῦ·\end{quote} 

\begin{quote}ὅτι ἔκρινεν τὴν πόρνην τὴν μεγάλην\end{quote} 

\begin{quote}ἥτις ἔφθειρεν τὴν γῆν ἐν τῇ πορνείᾳ αὐτῆς,\end{quote} 

\begin{quote}καὶ ἐξεδίκησεν τὸ αἷμα τῶν δούλων αὐτοῦ ἐκ χειρὸς αὐτῆς.\end{quote}
\end{poetryblock}

\vs{3}Καὶ δεύτερον εἴρηκαν· 
\begin{poetryblock}

\begin{quote}Ἁλληλουϊά·\end{quote} 

\begin{quote}καὶ ὁ καπνὸς αὐτῆς ἀναβαίνει εἰς τοὺς αἰῶνας τῶν αἰώνων.\end{quote}
\end{poetryblock}

\vs{4}Καὶ ἔπεσαν οἱ πρεσβύτεροι οἱ εἴκοσι τέσσαρες καὶ τὰ τέσσαρα ζῷα καὶ προσεκύνησαν τῷ Θεῷ τῷ καθημένῳ ἐπὶ τῷ θρόνῳ λέγοντες· 
\begin{poetryblock}

\begin{quote}Ἀμήν Ἁλληλουϊά.\end{quote}
\end{poetryblock}

\vs{5}Καὶ φωνὴ ἀπὸ τοῦ θρόνου ἐξῆλθεν λέγουσα· 
\begin{poetryblock}

\begin{quote}Αἰνεῖτε τῷ Θεῷ ἡμῶν\end{quote} 

\begin{quote}πάντες οἱ δοῦλοι αὐτοῦ\end{quote} 

\begin{quote}καὶ οἱ φοβούμενοι αὐτόν,\end{quote} 

\begin{quote}οἱ μικροὶ καὶ οἱ μεγάλοι.\end{quote}
\end{poetryblock}

\vs{6}Καὶ ἤκουσα ὡς φωνὴν ὄχλου πολλοῦ καὶ ὡς φωνὴν ὑδάτων πολλῶν καὶ ὡς φωνὴν βροντῶν ἰσχυρῶν λεγόντων· 
\begin{poetryblock}

\begin{quote}Ἁλληλουϊά,\end{quote} 

\begin{quote}ὅτι ἐβασίλευσεν Κύριος ὁ Θεός ἡμῶν ὁ Παντοκράτωρ.\end{quote}

\begin{quote} \vs{7}χαίρωμεν καὶ ἀγαλλιῶμεν καὶ δώσομεν τὴν δόξαν αὐτῷ,\end{quote} 

\begin{quote}ὅτι ἦλθεν ὁ γάμος τοῦ Ἀρνίου καὶ ἡ γυνὴ αὐτοῦ ἡτοίμασεν ἑαυτήν\end{quote}
\end{poetryblock}

\vs{8}καὶ ἐδόθη αὐτῇ ἵνα περιβάληται βύσσινον λαμπρὸν καθαρόν·

Τὸ γὰρ βύσσινον τὰ δικαιώματα τῶν ἁγίων ἐστίν.

\vs{9}Καὶ λέγει μοι· Γράψον· Μακάριοι οἱ εἰς τὸ δεῖπνον τοῦ γάμου τοῦ Ἀρνίου κεκλημένοι. καὶ λέγει μοι· Οὗτοι οἱ λόγοι ἀληθινοὶ τοῦ Θεοῦ εἰσιν.
\vs{10}Καὶ ἔπεσα ἔμπροσθεν τῶν ποδῶν αὐτοῦ προσκυνῆσαι αὐτῷ. καὶ λέγει μοι· Ὅρα μή· σύνδουλός σού εἰμι καὶ τῶν ἀδελφῶν σου τῶν ἐχόντων τὴν μαρτυρίαν Ἰησοῦ· τῷ Θεῷ προσκύνησον. ἡ γὰρ μαρτυρία Ἰησοῦ ἐστιν τὸ πνεῦμα τῆς προφητείας.

\vs{11}Καὶ εἶδον τὸν οὐρανὸν ἠνεῳγμένον, καὶ ἰδοὺ ἵππος λευκός καὶ ὁ καθήμενος ἐπ᾽ αὐτὸν καλούμενος Πιστὸς καὶ Ἀληθινός, καὶ ἐν δικαιοσύνῃ κρίνει καὶ πολεμεῖ.
\vs{12}οἱ δὲ ὀφθαλμοὶ αὐτοῦ ὡς φλὸξ πυρός, καὶ ἐπὶ τὴν κεφαλὴν αὐτοῦ διαδήματα πολλά, ἔχων ὄνομα γεγραμμένον ὃ οὐδεὶς οἶδεν εἰ μὴ αὐτός,
\vs{13}καὶ περιβεβλημένος ἱμάτιον βεβαμμένον αἵματι, καὶ κέκληται τὸ ὄνομα αὐτοῦ Ὁ Λόγος τοῦ Θεοῦ.

\vs{14}Καὶ τὰ στρατεύματα τὰ ἐν τῷ οὐρανῷ ἠκολούθει αὐτῷ ἐφ᾽ ἵπποις λευκοῖς, ἐνδεδυμένοι βύσσινον λευκὸν καθαρόν.
\vs{15}καὶ ἐκ τοῦ στόματος αὐτοῦ ἐκπορεύεται ῥομφαία ὀξεῖα, ἵνα ἐν αὐτῇ πατάξῃ τὰ ἔθνη, καὶ αὐτὸς ποιμανεῖ αὐτοὺς ἐν ῥάβδῳ σιδηρᾷ, καὶ αὐτὸς πατεῖ τὴν ληνὸν τοῦ οἴνου τοῦ θυμοῦ τῆς ὀργῆς τοῦ Θεοῦ τοῦ Παντοκράτορος,
\vs{16}καὶ ἔχει ἐπὶ τὸ ἱμάτιον καὶ ἐπὶ τὸν μηρὸν αὐτοῦ ὄνομα γεγραμμένον· ΒΑΣΙΛΕΥΣ ΒΑΣΙΛΕΩΝ ΚΑΙ ΚΥΡΙΟΣ ΚΥΡΙΩΝ.

\vs{17}Καὶ εἶδον ἕνα ἄγγελον ἑστῶτα ἐν τῷ ἡλίῳ καὶ ἔκραξεν ἐν φωνῇ μεγάλῃ λέγων πᾶσιν τοῖς ὀρνέοις τοῖς πετομένοις ἐν μεσουρανήματι·

Δεῦτε συνάχθητε εἰς τὸ δεῖπνον τὸ μέγα τοῦ Θεοῦ
\vs{18}ἵνα φάγητε σάρκας βασιλέων καὶ σάρκας χιλιάρχων καὶ σάρκας ἰσχυρῶν καὶ σάρκας ἵππων καὶ τῶν καθημένων ἐπ᾽ αὐτῶν καὶ σάρκας πάντων ἐλευθέρων τε καὶ δούλων καὶ μικρῶν καὶ μεγάλων.

\vs{19}Καὶ εἶδον τὸ θηρίον καὶ τοὺς βασιλεῖς τῆς γῆς καὶ τὰ στρατεύματα αὐτῶν συνηγμένα ποιῆσαι τὸν πόλεμον μετὰ τοῦ καθημένου ἐπὶ τοῦ ἵππου καὶ μετὰ τοῦ στρατεύματος αὐτοῦ.
\vs{20}καὶ ἐπιάσθη τὸ θηρίον καὶ μετ᾽ αὐτοῦ ὁ ψευδοπροφήτης ὁ ποιήσας τὰ σημεῖα ἐνώπιον αὐτοῦ, ἐν οἷς ἐπλάνησεν τοὺς λαβόντας τὸ χάραγμα τοῦ θηρίου καὶ τοὺς προσκυνοῦντας τῇ εἰκόνι αὐτοῦ· ζῶντες ἐβλήθησαν οἱ δύο εἰς τὴν λίμνην τοῦ πυρὸς τῆς καιομένης ἐν θείῳ.
\vs{21}καὶ οἱ λοιποὶ ἀπεκτάνθησαν ἐν τῇ ῥομφαίᾳ τοῦ καθημένου ἐπὶ τοῦ ἵππου τῇ ἐξελθούσῃ ἐκ τοῦ στόματος αὐτοῦ, Καὶ πάντα τὰ ὄρνεα ἐχορτάσθησαν ἐκ τῶν σαρκῶν αὐτῶν.

\ch{20}
Καὶ εἶδον ἄγγελον καταβαίνοντα ἐκ τοῦ οὐρανοῦ ἔχοντα τὴν κλεῖν τῆς ἀβύσσου καὶ ἅλυσιν μεγάλην ἐπὶ τὴν χεῖρα αὐτοῦ.
\vs{2}καὶ ἐκράτησεν τὸν δράκοντα, ὁ ὄφις ὁ ἀρχαῖος, ὅς ἐστιν Διάβολος καὶ Ὁ Σατανᾶς, καὶ ἔδησεν αὐτὸν χίλια ἔτη
\vs{3}καὶ ἔβαλεν αὐτὸν εἰς τὴν ἄβυσσον καὶ ἔκλεισεν καὶ ἐσφράγισεν ἐπάνω αὐτοῦ, ἵνα μὴ πλανήσῃ ἔτι τὰ ἔθνη ἄχρι τελεσθῇ τὰ χίλια ἔτη. μετὰ ταῦτα δεῖ λυθῆναι αὐτὸν μικρὸν χρόνον.

\vs{4}Καὶ εἶδον θρόνους καὶ ἐκάθισαν ἐπ᾽ αὐτούς καὶ κρίμα ἐδόθη αὐτοῖς, καὶ τὰς ψυχὰς τῶν πεπελεκισμένων διὰ τὴν μαρτυρίαν Ἰησοῦ καὶ διὰ τὸν λόγον τοῦ θεοῦ καὶ οἵτινες οὐ προσεκύνησαν τὸ θηρίον οὐδὲ τὴν εἰκόνα αὐτοῦ καὶ οὐκ ἔλαβον τὸ χάραγμα ἐπὶ τὸ μέτωπον καὶ ἐπὶ τὴν χεῖρα αὐτῶν. καὶ ἔζησαν καὶ ἐβασίλευσαν μετὰ τοῦ χριστοῦ χίλια ἔτη.
\vs{5}Οἱ λοιποὶ τῶν νεκρῶν οὐκ ἔζησαν ἄχρι τελεσθῇ τὰ χίλια ἔτη.

αὕτη ἡ ἀνάστασις ἡ πρώτη.
\vs{6}μακάριος καὶ ἅγιος ὁ ἔχων μέρος ἐν τῇ ἀναστάσει τῇ πρώτῃ· ἐπὶ τούτων ὁ δεύτερος θάνατος οὐκ ἔχει ἐξουσίαν, ἀλλ᾽ ἔσονται ἱερεῖς τοῦ Θεοῦ καὶ τοῦ Χριστοῦ καὶ βασιλεύσουσιν μετ᾽ αὐτοῦ τὰ χίλια ἔτη.

\vs{7}Καὶ ὅταν τελεσθῇ τὰ χίλια ἔτη, λυθήσεται ὁ Σατανᾶς ἐκ τῆς φυλακῆς αὐτοῦ
\vs{8}καὶ ἐξελεύσεται πλανῆσαι τὰ ἔθνη τὰ ἐν ταῖς τέσσαρσιν γωνίαις τῆς γῆς, τὸν Γὼγ καὶ Μαγώγ, συναγαγεῖν αὐτοὺς εἰς τὸν πόλεμον, ὧν ὁ ἀριθμὸς αὐτῶν ὡς ἡ ἄμμος τῆς θαλάσσης.
\vs{9}Καὶ ἀνέβησαν ἐπὶ τὸ πλάτος τῆς γῆς καὶ ἐκύκλευσαν τὴν παρεμβολὴν τῶν ἁγίων καὶ τὴν πόλιν τὴν ἠγαπημένην, καὶ κατέβη πῦρ ἐκ τοῦ οὐρανοῦ καὶ κατέφαγεν αὐτούς.
\vs{10}καὶ ὁ διάβολος ὁ πλανῶν αὐτοὺς ἐβλήθη εἰς τὴν λίμνην τοῦ πυρὸς καὶ θείου ὅπου καὶ τὸ θηρίον καὶ ὁ ψευδοπροφήτης, καὶ βασανισθήσονται ἡμέρας καὶ νυκτὸς εἰς τοὺς αἰῶνας τῶν αἰώνων.

\vs{11}Καὶ εἶδον θρόνον μέγαν λευκὸν καὶ τὸν καθήμενον ἐπ᾽ αὐτόν, οὗ ἀπὸ τοῦ προσώπου ἔφυγεν ἡ γῆ καὶ ὁ οὐρανός καὶ τόπος οὐχ εὑρέθη αὐτοῖς.
\vs{12}καὶ εἶδον τοὺς νεκρούς, τοὺς μεγάλους καὶ τοὺς μικρούς, ἑστῶτας ἐνώπιον τοῦ θρόνου. καὶ βιβλία ἠνοίχθησαν, Καὶ ἄλλο βιβλίον ἠνοίχθη, ὅ ἐστιν τῆς ζωῆς, καὶ ἐκρίθησαν οἱ νεκροὶ ἐκ τῶν γεγραμμένων ἐν τοῖς βιβλίοις κατὰ τὰ ἔργα αὐτῶν.
\vs{13}καὶ ἔδωκεν ἡ θάλασσα τοὺς νεκροὺς τοὺς ἐν αὐτῇ καὶ ὁ θάνατος καὶ ὁ ᾅδης ἔδωκαν τοὺς νεκροὺς τοὺς ἐν αὐτοῖς, καὶ ἐκρίθησαν ἕκαστος κατὰ τὰ ἔργα αὐτῶν.
\vs{14}Καὶ ὁ θάνατος καὶ ὁ ᾅδης ἐβλήθησαν εἰς τὴν λίμνην τοῦ πυρός. οὗτος ὁ θάνατος ὁ δεύτερός ἐστιν, ἡ λίμνη τοῦ πυρός.
\vs{15}καὶ εἴ τις οὐχ εὑρέθη ἐν τῇ βίβλῳ τῆς ζωῆς γεγραμμένος, ἐβλήθη εἰς τὴν λίμνην τοῦ πυρός.

\ch{21}
Καὶ εἶδον οὐρανὸν καινὸν καὶ γῆν καινήν. ὁ γὰρ πρῶτος οὐρανὸς καὶ ἡ πρώτη γῆ ἀπῆλθαν καὶ ἡ θάλασσα οὐκ ἔστιν ἔτι.
\vs{2}καὶ τὴν πόλιν τὴν ἁγίαν Ἰερουσαλὴμ καινὴν εἶδον καταβαίνουσαν ἐκ τοῦ οὐρανοῦ ἀπὸ τοῦ Θεοῦ ἡτοιμασμένην ὡς νύμφην κεκοσμημένην τῷ ἀνδρὶ αὐτῆς.
\vs{3}Καὶ ἤκουσα φωνῆς μεγάλης ἐκ τοῦ θρόνου λεγούσης·

Ἰδοὺ ἡ σκηνὴ τοῦ Θεοῦ μετὰ τῶν ἀνθρώπων, καὶ σκηνώσει μετ᾽ αὐτῶν, καὶ αὐτοὶ λαοὶ αὐτοῦ ἔσονται, καὶ αὐτὸς ὁ Θεὸς μετ᾽ αὐτῶν ἔσται αὐτῶν θεός,
\vs{4}καὶ ἐξαλείψει πᾶν δάκρυον ἐκ τῶν ὀφθαλμῶν αὐτῶν, καὶ ὁ θάνατος οὐκ ἔσται ἔτι οὔτε πένθος οὔτε κραυγὴ οὔτε πόνος οὐκ ἔσται ἔτι, ὅτι τὰ πρῶτα ἀπῆλθαν.

\vs{5}Καὶ εἶπεν ὁ καθήμενος ἐπὶ τῷ θρόνῳ· Ἰδοὺ καινὰ ποιῶ πάντα καὶ λέγει· Γράψον, ὅτι οὗτοι οἱ λόγοι πιστοὶ καὶ ἀληθινοί εἰσιν.
\vs{6}καὶ εἶπέν μοι· Γέγοναν. ἐγὼ εἰμι τὸ Ἄλφα καὶ τὸ Ὦ, ἡ ἀρχὴ καὶ τὸ τέλος. ἐγὼ τῷ διψῶντι δώσω ἐκ τῆς πηγῆς τοῦ ὕδατος τῆς ζωῆς δωρεάν.
\vs{7}ὁ νικῶν κληρονομήσει ταῦτα καὶ ἔσομαι αὐτῷ Θεὸς καὶ αὐτὸς ἔσται μοι υἱός.
\vs{8}Τοῖς δὲ δειλοῖς καὶ ἀπίστοις καὶ ἐβδελυγμένοις καὶ φονεῦσιν καὶ πόρνοις καὶ φαρμάκοις καὶ εἰδωλολάτραις καὶ πᾶσιν τοῖς ψευδέσιν τὸ μέρος αὐτῶν ἐν τῇ λίμνῃ τῇ καιομένῃ πυρὶ καὶ θείῳ, ὅ ἐστιν ὁ θάνατος ὁ δεύτερος.

\vs{9}Καὶ ἦλθεν εἷς ἐκ τῶν ἑπτὰ ἀγγέλων τῶν ἐχόντων τὰς ἑπτὰ φιάλας τῶν γεμόντων τῶν ἑπτὰ πληγῶν τῶν ἐσχάτων καὶ ἐλάλησεν μετ᾽ ἐμοῦ λέγων· Δεῦρο, δείξω σοι τὴν νύμφην τὴν γυναῖκα τοῦ ἀρνίου.
\vs{10}Καὶ ἀπήνεγκέν με ἐν Πνεύματι ἐπὶ ὄρος μέγα καὶ ὑψηλόν, καὶ ἔδειξέν μοι τὴν πόλιν τὴν ἁγίαν Ἰερουσαλὴμ καταβαίνουσαν ἐκ τοῦ οὐρανοῦ ἀπὸ τοῦ Θεοῦ
\vs{11}ἔχουσαν τὴν δόξαν τοῦ Θεοῦ, ὁ φωστὴρ αὐτῆς ὅμοιος λίθῳ τιμιωτάτῳ ὡς λίθῳ ἰάσπιδι κρυσταλλίζοντι.
\vs{12}ἔχουσα τεῖχος μέγα καὶ ὑψηλόν, ἔχουσα πυλῶνας δώδεκα καὶ ἐπὶ τοῖς πυλῶσιν ἀγγέλους δώδεκα καὶ ὀνόματα ἐπιγεγραμμένα, ἅ ἐστιν τὰ ὀνόματα τῶν δώδεκα φυλῶν υἱῶν Ἰσραήλ·
\vs{13}ἀπὸ ἀνατολῆς πυλῶνες τρεῖς καὶ ἀπὸ βορρᾶ πυλῶνες τρεῖς καὶ ἀπὸ νότου πυλῶνες τρεῖς καὶ ἀπὸ δυσμῶν πυλῶνες τρεῖς.
\vs{14}καὶ τὸ τεῖχος τῆς πόλεως ἔχων θεμελίους δώδεκα καὶ ἐπ᾽ αὐτῶν δώδεκα ὀνόματα τῶν δώδεκα ἀποστόλων τοῦ Ἀρνίου.

\vs{15}Καὶ ὁ λαλῶν μετ᾽ ἐμοῦ εἶχεν μέτρον κάλαμον χρυσοῦν, ἵνα μετρήσῃ τὴν πόλιν καὶ τοὺς πυλῶνας αὐτῆς καὶ τὸ τεῖχος αὐτῆς.
\vs{16}καὶ ἡ πόλις τετράγωνος κεῖται καὶ τὸ μῆκος αὐτῆς ὅσον καὶ τὸ πλάτος. καὶ ἐμέτρησεν τὴν πόλιν τῷ καλάμῳ ἐπὶ σταδίων δώδεκα χιλιάδων, τὸ μῆκος καὶ τὸ πλάτος καὶ τὸ ὕψος αὐτῆς ἴσα ἐστίν.
\vs{17}καὶ ἐμέτρησεν τὸ τεῖχος αὐτῆς ἑκατὸν τεσσεράκοντα τεσσάρων πηχῶν μέτρον ἀνθρώπου, ὅ ἐστιν ἀγγέλου.
\vs{18}Καὶ ἡ ἐνδώμησις τοῦ τείχους αὐτῆς ἴασπις καὶ ἡ πόλις χρυσίον καθαρὸν ὅμοιον ὑάλῳ καθαρῷ.
\vs{19}οἱ θεμέλιοι τοῦ τείχους τῆς πόλεως παντὶ λίθῳ τιμίῳ κεκοσμημένοι· ὁ θεμέλιος ὁ πρῶτος ἴασπις, ὁ δεύτερος σάπφιρος, ὁ τρίτος χαλκηδών, ὁ τέταρτος σμάραγδος,
\vs{20}ὁ πέμπτος σαρδόνυξ, ὁ ἕκτος σάρδιον, ὁ ἕβδομος χρυσόλιθος, ὁ ὄγδοος βήρυλλος, ὁ ἔνατος τοπάζιον, ὁ δέκατος χρυσόπρασος, ὁ ἑνδέκατος ὑάκινθος, ὁ δωδέκατος ἀμέθυστος,
\vs{21}Καὶ οἱ δώδεκα πυλῶνες δώδεκα μαργαρῖται, ἀνὰ εἷς ἕκαστος τῶν πυλώνων ἦν ἐξ ἑνὸς μαργαρίτου. καὶ ἡ πλατεῖα τῆς πόλεως χρυσίον καθαρὸν ὡς ὕαλος διαυγής.

\vs{22}Καὶ ναὸν οὐκ εἶδον ἐν αὐτῇ, ὁ γὰρ Κύριος ὁ Θεὸς ὁ Παντοκράτωρ ναὸς αὐτῆς ἐστιν καὶ τὸ Ἀρνίον.
\vs{23}καὶ ἡ πόλις οὐ χρείαν ἔχει τοῦ ἡλίου οὐδὲ τῆς σελήνης ἵνα φαίνωσιν αὐτῇ, ἡ γὰρ δόξα τοῦ Θεοῦ ἐφώτισεν αὐτήν, καὶ ὁ λύχνος αὐτῆς τὸ Ἀρνίον.
\vs{24}καὶ περιπατήσουσιν τὰ ἔθνη διὰ τοῦ φωτὸς αὐτῆς, καὶ οἱ βασιλεῖς τῆς γῆς φέρουσιν τὴν δόξαν αὐτῶν εἰς αὐτήν,
\vs{25}καὶ οἱ πυλῶνες αὐτῆς οὐ μὴ κλεισθῶσιν ἡμέρας, νὺξ γὰρ οὐκ ἔσται ἐκεῖ,
\vs{26}Καὶ οἴσουσιν τὴν δόξαν καὶ τὴν τιμὴν τῶν ἐθνῶν εἰς αὐτήν.
\vs{27}καὶ οὐ μὴ εἰσέλθῃ εἰς αὐτὴν πᾶν κοινὸν καὶ ὁ ποιῶν βδέλυγμα καὶ ψεῦδος εἰ μὴ οἱ γεγραμμένοι ἐν τῷ βιβλίῳ τῆς ζωῆς τοῦ Ἀρνίου.

\ch{22}
Καὶ ἔδειξέν μοι ποταμὸν ὕδατος ζωῆς λαμπρὸν ὡς κρύσταλλον, ἐκπορευόμενον ἐκ τοῦ θρόνου τοῦ Θεοῦ καὶ τοῦ Ἀρνίου.
\vs{2}ἐν μέσῳ τῆς πλατείας αὐτῆς καὶ τοῦ ποταμοῦ ἐντεῦθεν καὶ ἐκεῖθεν ξύλον ζωῆς ποιοῦν καρποὺς δώδεκα, κατὰ μῆνα ἕκαστον ἀποδιδοῦν τὸν καρπὸν αὐτοῦ, καὶ τὰ φύλλα τοῦ ξύλου εἰς θεραπείαν τῶν ἐθνῶν.
\vs{3}Καὶ πᾶν κατάθεμα οὐκ ἔσται ἔτι. καὶ ὁ θρόνος τοῦ Θεοῦ καὶ τοῦ Ἀρνίου ἐν αὐτῇ ἔσται, καὶ οἱ δοῦλοι αὐτοῦ λατρεύσουσιν αὐτῷ
\vs{4}καὶ ὄψονται τὸ πρόσωπον αὐτοῦ, καὶ τὸ ὄνομα αὐτοῦ ἐπὶ τῶν μετώπων αὐτῶν.
\vs{5}καὶ νὺξ οὐκ ἔσται ἔτι καὶ οὐκ ἔχουσιν χρείαν φωτὸς λύχνου καὶ φωτὸς ἡλίου, ὅτι Κύριος ὁ Θεὸς φωτίσει ἐπ᾽ αὐτούς, καὶ βασιλεύσουσιν εἰς τοὺς αἰῶνας τῶν αἰώνων.

\vs{6}Καὶ εἶπέν μοι· Οὗτοι οἱ λόγοι πιστοὶ καὶ ἀληθινοί, καὶ ὁ Κύριος ὁ Θεὸς τῶν πνευμάτων τῶν προφητῶν ἀπέστειλεν τὸν ἄγγελον αὐτοῦ δεῖξαι τοῖς δούλοις αὐτοῦ ἃ δεῖ γενέσθαι ἐν τάχει.
\vs{7}Καὶ Ἰδοὺ ἔρχομαι ταχύ. μακάριος ὁ τηρῶν τοὺς λόγους τῆς προφητείας τοῦ βιβλίου τούτου.

\vs{8}Κἀγὼ Ἰωάννης ὁ ἀκούων καὶ βλέπων ταῦτα. καὶ ὅτε ἤκουσα καὶ ἔβλεψα, ἔπεσα προσκυνῆσαι ἔμπροσθεν τῶν ποδῶν τοῦ ἀγγέλου τοῦ δεικνύοντός μοι ταῦτα.
\vs{9}καὶ λέγει μοι· Ὅρα μή· σύνδουλός σού εἰμι καὶ τῶν ἀδελφῶν σου τῶν προφητῶν καὶ τῶν τηρούντων τοὺς λόγους τοῦ βιβλίου τούτου· τῷ Θεῷ προσκύνησον.

\vs{10}Καὶ λέγει μοι· Μὴ σφραγίσῃς τοὺς λόγους τῆς προφητείας τοῦ βιβλίου τούτου, ὁ καιρὸς γὰρ ἐγγύς ἐστιν.
\vs{11}ὁ ἀδικῶν ἀδικησάτω ἔτι καὶ ὁ ῥυπαρὸς ῥυπανθήτω ἔτι, καὶ ὁ δίκαιος δικαιοσύνην ποιησάτω ἔτι καὶ ὁ ἅγιος ἁγιασθήτω ἔτι.

\vs{12}Ἰδοὺ ἔρχομαι ταχύ, καὶ ὁ μισθός μου μετ᾽ ἐμοῦ ἀποδοῦναι ἑκάστῳ ὡς τὸ ἔργον ἐστὶν αὐτοῦ.
\vs{13}ἐγὼ τὸ Ἄλφα καὶ τὸ Ὦ, ὁ πρῶτος καὶ ὁ ἔσχατος, ἡ ἀρχὴ καὶ τὸ τέλος.

\vs{14}Μακάριοι οἱ πλύνοντες τὰς στολὰς αὐτῶν, ἵνα ἔσται ἡ ἐξουσία αὐτῶν ἐπὶ τὸ ξύλον τῆς ζωῆς καὶ τοῖς πυλῶσιν εἰσέλθωσιν εἰς τὴν πόλιν.
\vs{15}ἔξω οἱ κύνες καὶ οἱ φάρμακοι καὶ οἱ πόρνοι καὶ οἱ φονεῖς καὶ οἱ εἰδωλολάτραι καὶ πᾶς φιλῶν καὶ ποιῶν ψεῦδος.

\vs{16}Ἐγὼ Ἰησοῦς ἔπεμψα τὸν ἄγγελόν μου μαρτυρῆσαι ὑμῖν ταῦτα ἐπὶ ταῖς ἐκκλησίαις. ἐγώ εἰμι ἡ ῥίζα καὶ τὸ γένος Δαυίδ, ὁ ἀστὴρ ὁ λαμπρός ὁ πρωϊνός.

\vs{17}Καὶ τὸ Πνεῦμα καὶ ἡ νύμφη λέγουσιν· Ἔρχου. καὶ ὁ ἀκούων εἰπάτω· Ἔρχου. καὶ ὁ διψῶν ἐρχέσθω, ὁ θέλων λαβέτω ὕδωρ ζωῆς δωρεάν.

\vs{18}Μαρτυρῶ ἐγὼ παντὶ τῷ ἀκούοντι τοὺς λόγους τῆς προφητείας τοῦ βιβλίου τούτου· ἐάν τις ἐπιθῇ ἐπ᾽ αὐτά, ἐπιθήσει ὁ Θεὸς ἐπ᾽ αὐτὸν τὰς πληγὰς τὰς γεγραμμένας ἐν τῷ βιβλίῳ τούτῳ,
\vs{19}καὶ ἐάν τις ἀφέλῃ ἀπὸ τῶν λόγων τοῦ βιβλίου τῆς προφητείας ταύτης, ἀφελεῖ ὁ Θεὸς τὸ μέρος αὐτοῦ ἀπὸ τοῦ ξύλου τῆς ζωῆς καὶ ἐκ τῆς πόλεως τῆς ἁγίας τῶν γεγραμμένων ἐν τῷ βιβλίῳ τούτῳ.

\vs{20}Λέγει ὁ μαρτυρῶν ταῦτα· Ναί, ἔρχομαι ταχύ. Ἀμήν, ἔρχου Κύριε Ἰησοῦ.
\vs{21}Ἡ χάρις τοῦ Κυρίου Ἰησοῦ μετὰ πάντων.


\end{spacing}
\end{document}