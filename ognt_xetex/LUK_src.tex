\NormalFont\ShortTitle{ΚΑΤΑ ΛΟΥΚΑΝ}
{\MT ΚΑΤΑ ΛΟΥΚΑΝ

\par }\ChapOne{1}{\PP \VerseOne{1}Ἐπειδήπερ πολλοὶ ἐπεχείρησαν ἀνατάξασθαι διήγησιν περὶ τῶν πεπληροφορημένων ἐν ἡμῖν πραγμάτων,
\VS{2}καθὼς παρέδοσαν ἡμῖν οἱ ἀπ᾽ ἀρχῆς αὐτόπται καὶ ὑπηρέται γενόμενοι τοῦ λόγου,
\VS{3}ἔδοξε κἀμοὶ παρηκολουθηκότι ἄνωθεν πᾶσιν ἀκριβῶς καθεξῆς σοι γράψαι, κράτιστε Θεόφιλε,
\VS{4}ἵνα ἐπιγνῷς περὶ ὧν κατηχήθης λόγων τὴν ἀσφάλειαν.
\par }{\PP \VS{5}Ἐγένετο ἐν ταῖς ἡμέραις Ἡρῴδου βασιλέως τῆς Ἰουδαίας ἱερεύς τις ὀνόματι Ζαχαρίας ἐξ ἐφημερίας Ἀβιά, καὶ γυνὴ αὐτῷ ἐκ τῶν θυγατέρων Ἀαρών καὶ τὸ ὄνομα αὐτῆς Ἐλισάβετ.
\VS{6}ἦσαν δὲ δίκαιοι ἀμφότεροι ἐναντίον τοῦ Θεοῦ, πορευόμενοι ἐν πάσαις ταῖς ἐντολαῖς καὶ δικαιώμασιν τοῦ Κυρίου ἄμεμπτοι.
\VS{7}καὶ οὐκ ἦν αὐτοῖς τέκνον, καθότι ἦν ἡ Ἐλισάβετ στεῖρα, καὶ ἀμφότεροι προβεβηκότες ἐν ταῖς ἡμέραις αὐτῶν ἦσαν.
\par }{\PP \VS{8}Ἐγένετο δὲ ἐν τῷ ἱερατεύειν αὐτὸν ἐν τῇ τάξει τῆς ἐφημερίας αὐτοῦ ἔναντι τοῦ Θεοῦ,
\VS{9}κατὰ τὸ ἔθος τῆς ἱερατείας ἔλαχε τοῦ θυμιᾶσαι εἰσελθὼν εἰς τὸν ναὸν τοῦ Κυρίου,
\VS{10}καὶ πᾶν τὸ πλῆθος ἦν τοῦ λαοῦ προσευχόμενον ἔξω τῇ ὥρᾳ τοῦ θυμιάματος.
\VS{11}Ὤφθη δὲ αὐτῷ ἄγγελος Κυρίου ἑστὼς ἐκ δεξιῶν τοῦ θυσιαστηρίου τοῦ θυμιάματος.
\VS{12}καὶ ἐταράχθη Ζαχαρίας ἰδών καὶ φόβος ἐπέπεσεν ἐπ᾽ αὐτόν.
\VS{13}Εἶπεν δὲ πρὸς αὐτὸν ὁ ἄγγελος· Μὴ φοβοῦ, Ζαχαρία, διότι εἰσηκούσθη ἡ δέησίς σου, καὶ ἡ γυνή σου Ἐλισάβετ γεννήσει υἱόν σοι καὶ καλέσεις τὸ ὄνομα αὐτοῦ Ἰωάννην.
\begin{poetryblock}
\par }{\PP \begin{quote} \VS{14}καὶ ἔσται χαρά σοι καὶ ἀγαλλίασις\end{quote} 
\par }{\PP \begin{quote}καὶ πολλοὶ ἐπὶ τῇ γενέσει αὐτοῦ χαρήσονται.\end{quote}
\par }{\PP \begin{quote} \VS{15}ἔσται γὰρ μέγας ἐνώπιον τοῦ Κυρίου,\end{quote} 
\par }{\PP \begin{quote}καὶ οἶνον καὶ σίκερα οὐ μὴ πίῃ,\end{quote} 
\par }{\PP \begin{quote}καὶ Πνεύματος Ἁγίου πλησθήσεται\end{quote} 
\par }{\PP \begin{quote}ἔτι ἐκ κοιλίας μητρὸς αὐτοῦ,\end{quote}
\par }{\PP \begin{quote} \VS{16}καὶ πολλοὺς τῶν υἱῶν Ἰσραὴλ ἐπιστρέψει\end{quote} 
\par }{\PP \begin{quote}ἐπὶ Κύριον τὸν Θεὸν αὐτῶν.\end{quote}
\par }{\PP \begin{quote} \VS{17}καὶ αὐτὸς προελεύσεται ἐνώπιον αὐτοῦ\end{quote} 
\par }{\PP \begin{quote}ἐν πνεύματι καὶ δυνάμει Ἠλίου,\end{quote} 
\par }{\PP \begin{quote}ἐπιστρέψαι καρδίας πατέρων ἐπὶ τέκνα\end{quote} 
\par }{\PP \begin{quote}καὶ ἀπειθεῖς ἐν φρονήσει δικαίων,\end{quote} 
\par }{\PP \begin{quote}ἑτοιμάσαι Κυρίῳ λαὸν κατεσκευασμένον.\end{quote}
\end{poetryblock}
\par }{\PP \VS{18}Καὶ εἶπεν Ζαχαρίας πρὸς τὸν ἄγγελον· Κατὰ τί γνώσομαι τοῦτο; ἐγὼ γάρ εἰμι πρεσβύτης καὶ ἡ γυνή μου προβεβηκυῖα ἐν ταῖς ἡμέραις αὐτῆς.
\VS{19}Καὶ ἀποκριθεὶς ὁ ἄγγελος εἶπεν αὐτῷ· Ἐγώ εἰμι Γαβριὴλ ὁ παρεστηκὼς ἐνώπιον τοῦ Θεοῦ καὶ ἀπεστάλην λαλῆσαι πρὸς σὲ καὶ εὐαγγελίσασθαί σοι ταῦτα·
\VS{20}καὶ ἰδοὺ ἔσῃ σιωπῶν καὶ μὴ δυνάμενος λαλῆσαι ἄχρι ἧς ἡμέρας γένηται ταῦτα, ἀνθ᾽ ὧν οὐκ ἐπίστευσας τοῖς λόγοις μου, οἵτινες πληρωθήσονται εἰς τὸν καιρὸν αὐτῶν.
\par }{\PP \VS{21}Καὶ ἦν ὁ λαὸς προσδοκῶν τὸν Ζαχαρίαν καὶ ἐθαύμαζον ἐν τῷ χρονίζειν ἐν τῷ ναῷ αὐτόν.
\VS{22}ἐξελθὼν δὲ οὐκ ἐδύνατο λαλῆσαι αὐτοῖς, καὶ ἐπέγνωσαν ὅτι ὀπτασίαν ἑώρακεν ἐν τῷ ναῷ· καὶ αὐτὸς ἦν διανεύων αὐτοῖς καὶ διέμενεν κωφός.
\VS{23}καὶ ἐγένετο ὡς ἐπλήσθησαν αἱ ἡμέραι τῆς λειτουργίας αὐτοῦ, ἀπῆλθεν εἰς τὸν οἶκον αὐτοῦ.
\VS{24}Μετὰ δὲ ταύτας τὰς ἡμέρας συνέλαβεν Ἐλισάβετ ἡ γυνὴ αὐτοῦ καὶ περιέκρυβεν ἑαυτὴν μῆνας πέντε λέγουσα
\VS{25}ὅτι Οὕτως μοι πεποίηκεν Κύριος ἐν ἡμέραις αἷς ἐπεῖδεν ἀφελεῖν ὄνειδός μου ἐν ἀνθρώποις.
\VS{26}Ἐν δὲ τῷ μηνὶ τῷ ἕκτῳ ἀπεστάλη ὁ ἄγγελος Γαβριὴλ ἀπὸ τοῦ Θεοῦ εἰς πόλιν τῆς Γαλιλαίας ᾗ ὄνομα Ναζαρὲθ
\VS{27}πρὸς παρθένον ἐμνηστευμένην ἀνδρὶ ᾧ ὄνομα Ἰωσὴφ ἐξ οἴκου Δαυὶδ καὶ τὸ ὄνομα τῆς παρθένου Μαριάμ.
\VS{28}καὶ εἰσελθὼν πρὸς αὐτὴν εἶπεν· Χαῖρε, κεχαριτωμένη, ὁ Κύριος μετὰ σοῦ.
\VS{29}Ἡ δὲ ἐπὶ τῷ λόγῳ διεταράχθη καὶ διελογίζετο ποταπὸς εἴη ὁ ἀσπασμὸς οὗτος.
\VS{30}καὶ εἶπεν ὁ ἄγγελος αὐτῇ· 
\begin{poetryblock}
\par }{\PP \begin{quote}Μὴ φοβοῦ, Μαριάμ, εὗρες γὰρ χάριν παρὰ τῷ Θεῷ.\end{quote}
\par }{\PP \begin{quote} \VS{31}καὶ ἰδοὺ συλλήμψῃ ἐν γαστρὶ καὶ τέξῃ υἱόν\end{quote} 
\par }{\PP \begin{quote}καὶ καλέσεις τὸ ὄνομα αὐτοῦ Ἰησοῦν.\end{quote}
\par }{\PP \begin{quote} \VS{32}οὗτος ἔσται μέγας καὶ Υἱὸς Ὑψίστου κληθήσεται\end{quote} 
\par }{\PP \begin{quote}καὶ δώσει αὐτῷ Κύριος ὁ Θεὸς τὸν θρόνον Δαυὶδ τοῦ πατρὸς αὐτοῦ,\end{quote}
\par }{\PP \begin{quote} \VS{33}καὶ βασιλεύσει ἐπὶ τὸν οἶκον Ἰακὼβ εἰς τοὺς αἰῶνας\end{quote} 
\par }{\PP \begin{quote}καὶ τῆς βασιλείας αὐτοῦ οὐκ ἔσται τέλος.\end{quote}
\end{poetryblock}
\par }{\PP \VS{34}Εἶπεν δὲ Μαριὰμ πρὸς τὸν ἄγγελον· Πῶς ἔσται τοῦτο, ἐπεὶ ἄνδρα οὐ γινώσκω;
\VS{35}Καὶ ἀποκριθεὶς ὁ ἄγγελος εἶπεν αὐτῇ· 
\begin{poetryblock}
\par }{\PP \begin{quote}Πνεῦμα Ἅγιον ἐπελεύσεται ἐπὶ σέ\end{quote} 
\par }{\PP \begin{quote}καὶ δύναμις Ὑψίστου ἐπισκιάσει σοι·\end{quote} 
\par }{\PP \begin{quote}διὸ καὶ τὸ γεννώμενον ἅγιον κληθήσεται Υἱὸς Θεοῦ.\end{quote}
\end{poetryblock}
\par }{\PP \VS{36}καὶ ἰδοὺ Ἐλισάβετ ἡ συγγενίς σου καὶ αὐτὴ συνείληφεν υἱὸν ἐν γήρει αὐτῆς καὶ οὗτος μὴν ἕκτος ἐστὶν αὐτῇ τῇ καλουμένῃ στείρᾳ·
\VS{37}ὅτι οὐκ ἀδυνατήσει παρὰ τοῦ Θεοῦ πᾶν ῥῆμα.
\VS{38}Εἶπεν δὲ Μαριάμ· Ἰδοὺ ἡ δούλη Κυρίου· γένοιτό μοι κατὰ τὸ ῥῆμά σου. καὶ ἀπῆλθεν ἀπ᾽ αὐτῆς ὁ ἄγγελος.
\par }{\PP \VS{39}Ἀναστᾶσα δὲ Μαριὰμ ἐν ταῖς ἡμέραις ταύταις ἐπορεύθη εἰς τὴν ὀρεινὴν μετὰ σπουδῆς εἰς πόλιν Ἰούδα,
\VS{40}καὶ εἰσῆλθεν εἰς τὸν οἶκον Ζαχαρίου καὶ ἠσπάσατο τὴν Ἐλισάβετ.
\VS{41}καὶ ἐγένετο ὡς ἤκουσεν τὸν ἀσπασμὸν τῆς Μαρίας ἡ Ἐλισάβετ, ἐσκίρτησεν τὸ βρέφος ἐν τῇ κοιλίᾳ αὐτῆς, καὶ ἐπλήσθη Πνεύματος Ἁγίου ἡ Ἐλισάβετ,
\VS{42}καὶ ἀνεφώνησεν κραυγῇ μεγάλῃ καὶ εἶπεν· 
\begin{poetryblock}
\par }{\PP \begin{quote}Εὐλογημένη σὺ ἐν γυναιξίν\end{quote} 
\par }{\PP \begin{quote}καὶ εὐλογημένος ὁ καρπὸς τῆς κοιλίας σου.\end{quote}
\end{poetryblock}
\par }{\PP \VS{43}καὶ πόθεν μοι τοῦτο ἵνα ἔλθῃ ἡ μήτηρ τοῦ Κυρίου μου πρὸς ἐμέ;
\VS{44}ἰδοὺ γὰρ ὡς ἐγένετο ἡ φωνὴ τοῦ ἀσπασμοῦ σου εἰς τὰ ὦτά μου, ἐσκίρτησεν ἐν ἀγαλλιάσει τὸ βρέφος ἐν τῇ κοιλίᾳ μου.
\VS{45}καὶ μακαρία ἡ πιστεύσασα ὅτι ἔσται τελείωσις τοῖς λελαλημένοις αὐτῇ παρὰ Κυρίου.
\par }{\PP \VS{46}Καὶ εἶπεν Μαριάμ· 
\begin{poetryblock}
\par }{\PP \begin{quote}Μεγαλύνει ἡ ψυχή μου τὸν Κύριον,\end{quote}
\par }{\PP \begin{quote} \VS{47}καὶ ἠγαλλίασεν τὸ πνεῦμά μου ἐπὶ τῷ Θεῷ τῷ Σωτῆρί μου,\end{quote}
\par }{\PP \begin{quote} \VS{48}ὅτι ἐπέβλεψεν ἐπὶ τὴν ταπείνωσιν τῆς δούλης αὐτοῦ.\end{quote} 
\par }{\PP \begin{quote}ἰδοὺ γὰρ ἀπὸ τοῦ νῦν μακαριοῦσίν με πᾶσαι αἱ γενεαί,\end{quote}
\par }{\PP \begin{quote} \VS{49}ὅτι ἐποίησέν μοι μεγάλα ὁ δυνατός.\end{quote} 
\par }{\PP \begin{quote}καὶ ἅγιον τὸ ὄνομα αὐτοῦ,\end{quote}
\par }{\PP \begin{quote} \VS{50}καὶ τὸ ἔλεος αὐτοῦ εἰς γενεὰς καὶ γενεὰς\end{quote} 
\par }{\PP \begin{quote}τοῖς φοβουμένοις αὐτόν.\end{quote}
\par }{\PP \begin{quote} \VS{51}Ἐποίησεν κράτος ἐν βραχίονι αὐτοῦ,\end{quote} 
\par }{\PP \begin{quote}διεσκόρπισεν ὑπερηφάνους διανοίᾳ καρδίας αὐτῶν·\end{quote}
\par }{\PP \begin{quote} \VS{52}καθεῖλεν δυνάστας ἀπὸ θρόνων\end{quote} 
\par }{\PP \begin{quote}καὶ ὕψωσεν ταπεινούς,\end{quote}
\par }{\PP \begin{quote} \VS{53}πεινῶντας ἐνέπλησεν ἀγαθῶν\end{quote} 
\par }{\PP \begin{quote}καὶ πλουτοῦντας ἐξαπέστειλεν κενούς.\end{quote}
\par }{\PP \begin{quote} \VS{54}ἀντελάβετο Ἰσραὴλ παιδὸς αὐτοῦ,\end{quote} 
\par }{\PP \begin{quote}μνησθῆναι ἐλέους,\end{quote}
\par }{\PP \begin{quote} \VS{55}καθὼς ἐλάλησεν πρὸς τοὺς πατέρας ἡμῶν,\end{quote} 
\par }{\PP \begin{quote}τῷ Ἀβραὰμ καὶ τῷ σπέρματι αὐτοῦ εἰς τὸν αἰῶνα.\end{quote}
\end{poetryblock}
\par }{\PP \VS{56}Ἔμεινεν δὲ Μαριὰμ σὺν αὐτῇ ὡς μῆνας τρεῖς, καὶ ὑπέστρεψεν εἰς τὸν οἶκον αὐτῆς.
\par }{\PP \VS{57}Τῇ δὲ Ἐλισάβετ ἐπλήσθη ὁ χρόνος τοῦ τεκεῖν αὐτήν καὶ ἐγέννησεν υἱόν.
\VS{58}καὶ ἤκουσαν οἱ περίοικοι καὶ οἱ συγγενεῖς αὐτῆς ὅτι ἐμεγάλυνεν Κύριος τὸ ἔλεος αὐτοῦ μετ᾽ αὐτῆς καὶ συνέχαιρον αὐτῇ.
\VS{59}Καὶ ἐγένετο ἐν τῇ ἡμέρᾳ τῇ ὀγδόῃ ἦλθον περιτεμεῖν τὸ παιδίον καὶ ἐκάλουν αὐτὸ ἐπὶ τῷ ὀνόματι τοῦ πατρὸς αὐτοῦ Ζαχαρίαν.
\VS{60}καὶ ἀποκριθεῖσα ἡ μήτηρ αὐτοῦ εἶπεν· Οὐχί, ἀλλὰ κληθήσεται Ἰωάννης.
\VS{61}Καὶ εἶπαν πρὸς αὐτὴν ὅτι Οὐδείς ἐστιν ἐκ τῆς συγγενείας σου ὃς καλεῖται τῷ ὀνόματι τούτῳ.
\VS{62}ἐνένευον δὲ τῷ πατρὶ αὐτοῦ τὸ τί ἂν θέλοι καλεῖσθαι αὐτό.
\VS{63}Καὶ αἰτήσας πινακίδιον ἔγραψεν λέγων· Ἰωάννης ἐστὶν ὄνομα αὐτοῦ. καὶ ἐθαύμασαν πάντες.
\VS{64}ἀνεῴχθη δὲ τὸ στόμα αὐτοῦ παραχρῆμα καὶ ἡ γλῶσσα αὐτοῦ, καὶ ἐλάλει εὐλογῶν τὸν Θεόν.
\VS{65}Καὶ ἐγένετο ἐπὶ πάντας φόβος τοὺς περιοικοῦντας αὐτούς, καὶ ἐν ὅλῃ τῇ ὀρεινῇ τῆς Ἰουδαίας διελαλεῖτο πάντα τὰ ῥήματα ταῦτα,
\VS{66}καὶ ἔθεντο πάντες οἱ ἀκούσαντες ἐν τῇ καρδίᾳ αὐτῶν λέγοντες· Τί ἄρα τὸ παιδίον τοῦτο ἔσται; καὶ γὰρ χεὶρ Κυρίου ἦν μετ᾽ αὐτοῦ.
\par }{\PP \VS{67}Καὶ Ζαχαρίας ὁ πατὴρ αὐτοῦ ἐπλήσθη Πνεύματος Ἁγίου καὶ ἐπροφήτευσεν λέγων·
\begin{poetryblock}
\par }{\PP \begin{quote} \VS{68}Εὐλογητὸς Κύριος ὁ Θεὸς τοῦ Ἰσραήλ,\end{quote} 
\par }{\PP \begin{quote}ὅτι ἐπεσκέψατο καὶ ἐποίησεν λύτρωσιν τῷ λαῷ αὐτοῦ,\end{quote}
\par }{\PP \begin{quote} \VS{69}καὶ ἤγειρεν κέρας σωτηρίας ἡμῖν\end{quote} 
\par }{\PP \begin{quote}ἐν οἴκῳ Δαυὶδ παιδὸς αὐτοῦ,\end{quote}
\par }{\PP \begin{quote} \VS{70}καθὼς ἐλάλησεν διὰ στόματος τῶν ἁγίων ἀπ᾽ αἰῶνος προφητῶν αὐτοῦ,\end{quote}
\par }{\PP \begin{quote} \VS{71}σωτηρίαν ἐξ ἐχθρῶν ἡμῶν καὶ ἐκ χειρὸς πάντων τῶν μισούντων ἡμᾶς,\end{quote}
\par }{\PP \begin{quote} \VS{72}ποιῆσαι ἔλεος μετὰ τῶν πατέρων ἡμῶν\end{quote} 
\par }{\PP \begin{quote}καὶ μνησθῆναι διαθήκης ἁγίας αὐτοῦ,\end{quote}
\par }{\PP \begin{quote} \VS{73}ὅρκον ὃν ὤμοσεν πρὸς Ἀβραὰμ τὸν πατέρα ἡμῶν,\end{quote} 
\par }{\PP \begin{quote}τοῦ δοῦναι ἡμῖν\end{quote}
\end{poetryblock}
\par }{\PP \VS{74}ἀφόβως ἐκ χειρὸς ἐχθρῶν ῥυσθέντας 
\begin{poetryblock}
\par }{\PP \begin{quote}λατρεύειν αὐτῷ\end{quote}
\end{poetryblock}
\par }{\PP \VS{75}ἐν ὁσιότητι καὶ δικαιοσύνῃ 
\begin{poetryblock}
\par }{\PP \begin{quote}ἐνώπιον αὐτοῦ πάσαις ταῖς ἡμέραις ἡμῶν.\end{quote}
\par }{\PP \begin{quote} \VS{76}Καὶ σὺ δέ, παιδίον, προφήτης Ὑψίστου κληθήσῃ·\end{quote} 
\par }{\PP \begin{quote}προπορεύσῃ γὰρ ἐνώπιον Κυρίου ἑτοιμάσαι ὁδοὺς αὐτοῦ,\end{quote}
\par }{\PP \begin{quote} \VS{77}τοῦ δοῦναι γνῶσιν σωτηρίας τῷ λαῷ αὐτοῦ\end{quote} 
\par }{\PP \begin{quote}ἐν ἀφέσει ἁμαρτιῶν αὐτῶν,\end{quote}
\par }{\PP \begin{quote} \VS{78}διὰ σπλάγχνα ἐλέους Θεοῦ ἡμῶν,\end{quote} 
\par }{\PP \begin{quote}ἐν οἷς ἐπισκέψεται ἡμᾶς ἀνατολὴ ἐξ ὕψους,\end{quote}
\par }{\PP \begin{quote} \VS{79}ἐπιφᾶναι τοῖς ἐν σκότει καὶ σκιᾷ θανάτου καθημένοις,\end{quote} 
\par }{\PP \begin{quote}τοῦ κατευθῦναι τοὺς πόδας ἡμῶν εἰς ὁδὸν εἰρήνης.\end{quote}
\end{poetryblock}
\par }{\PP \VS{80}Τὸ δὲ παιδίον ηὔξανεν καὶ ἐκραταιοῦτο πνεύματι, καὶ ἦν ἐν ταῖς ἐρήμοις ἕως ἡμέρας ἀναδείξεως αὐτοῦ πρὸς τὸν Ἰσραήλ.

\par }\Chap{2}{\PP \VerseOne{1}Ἐγένετο δὲ ἐν ταῖς ἡμέραις ἐκείναις ἐξῆλθεν δόγμα παρὰ Καίσαρος Αὐγούστου ἀπογράφεσθαι πᾶσαν τὴν οἰκουμένην.
\VS{2}αὕτη ἀπογραφὴ πρώτη ἐγένετο ἡγεμονεύοντος τῆς Συρίας Κυρηνίου.
\VS{3}καὶ ἐπορεύοντο πάντες ἀπογράφεσθαι, ἕκαστος εἰς τὴν ἑαυτοῦ πόλιν.
\VS{4}Ἀνέβη δὲ καὶ Ἰωσὴφ ἀπὸ τῆς Γαλιλαίας ἐκ πόλεως Ναζαρὲθ εἰς τὴν Ἰουδαίαν εἰς πόλιν Δαυὶδ ἥτις καλεῖται Βηθλεέμ, διὰ τὸ εἶναι αὐτὸν ἐξ οἴκου καὶ πατριᾶς Δαυίδ,
\VS{5}ἀπογράψασθαι σὺν Μαριὰμ τῇ ἐμνηστευμένῃ αὐτῷ, οὔσῃ ἐγκύῳ.
\VS{6}Ἐγένετο δὲ ἐν τῷ εἶναι αὐτοὺς ἐκεῖ ἐπλήσθησαν αἱ ἡμέραι τοῦ τεκεῖν αὐτήν,
\VS{7}καὶ ἔτεκεν τὸν υἱὸν αὐτῆς τὸν πρωτότοκον, καὶ ἐσπαργάνωσεν αὐτὸν καὶ ἀνέκλινεν αὐτὸν ἐν φάτνῃ, διότι οὐκ ἦν αὐτοῖς τόπος ἐν τῷ καταλύματι.
\par }{\PP \VS{8}Καὶ ποιμένες ἦσαν ἐν τῇ χώρᾳ τῇ αὐτῇ ἀγραυλοῦντες καὶ φυλάσσοντες φυλακὰς τῆς νυκτὸς ἐπὶ τὴν ποίμνην αὐτῶν.
\VS{9}καὶ ἄγγελος Κυρίου ἐπέστη αὐτοῖς καὶ δόξα Κυρίου περιέλαμψεν αὐτούς, καὶ ἐφοβήθησαν φόβον μέγαν.
\VS{10}καὶ εἶπεν αὐτοῖς ὁ ἄγγελος· Μὴ φοβεῖσθε, ἰδοὺ γὰρ εὐαγγελίζομαι ὑμῖν χαρὰν μεγάλην ἥτις ἔσται παντὶ τῷ λαῷ,
\VS{11}ὅτι ἐτέχθη ὑμῖν σήμερον Σωτήρ ὅς ἐστιν Χριστὸς Κύριος ἐν πόλει Δαυίδ.
\VS{12}καὶ τοῦτο ὑμῖν τὸ σημεῖον, εὑρήσετε βρέφος ἐσπαργανωμένον καὶ κείμενον ἐν φάτνῃ.
\VS{13}Καὶ ἐξαίφνης ἐγένετο σὺν τῷ ἀγγέλῳ πλῆθος στρατιᾶς οὐρανίου αἰνούντων τὸν Θεὸν καὶ λεγόντων·
\begin{poetryblock}
\par }{\PP \begin{quote} \VS{14}Δόξα ἐν ὑψίστοις Θεῷ\end{quote} 
\par }{\PP \begin{quote}καὶ ἐπὶ γῆς εἰρήνη\end{quote} 
\par }{\PP \begin{quote}ἐν ἀνθρώποις εὐδοκίας.\end{quote}
\end{poetryblock}
\par }{\PP \VS{15}Καὶ ἐγένετο ὡς ἀπῆλθον ἀπ᾽ αὐτῶν εἰς τὸν οὐρανὸν οἱ ἄγγελοι, οἱ ποιμένες ἐλάλουν πρὸς ἀλλήλους· Διέλθωμεν δὴ ἕως Βηθλεὲμ καὶ ἴδωμεν τὸ ῥῆμα τοῦτο τὸ γεγονὸς ὃ ὁ Κύριος ἐγνώρισεν ἡμῖν.
\VS{16}Καὶ ἦλθαν σπεύσαντες καὶ ἀνεῦραν τήν τε Μαριὰμ καὶ τὸν Ἰωσὴφ καὶ τὸ βρέφος κείμενον ἐν τῇ φάτνῃ·
\VS{17}ἰδόντες δὲ ἐγνώρισαν περὶ τοῦ ῥήματος τοῦ λαληθέντος αὐτοῖς περὶ τοῦ παιδίου τούτου.
\VS{18}καὶ πάντες οἱ ἀκούσαντες ἐθαύμασαν περὶ τῶν λαληθέντων ὑπὸ τῶν ποιμένων πρὸς αὐτούς·
\VS{19}ἡ δὲ Μαρία πάντα συνετήρει τὰ ῥήματα ταῦτα συμβάλλουσα ἐν τῇ καρδίᾳ αὐτῆς.
\VS{20}Καὶ ὑπέστρεψαν οἱ ποιμένες δοξάζοντες καὶ αἰνοῦντες τὸν Θεὸν ἐπὶ πᾶσιν οἷς ἤκουσαν καὶ εἶδον καθὼς ἐλαλήθη πρὸς αὐτούς.
\par }{\PP \VS{21}Καὶ ὅτε ἐπλήσθησαν ἡμέραι ὀκτὼ τοῦ περιτεμεῖν αὐτόν καὶ ἐκλήθη τὸ ὄνομα αὐτοῦ Ἰησοῦς, τὸ κληθὲν ὑπὸ τοῦ ἀγγέλου πρὸ τοῦ συλλημφθῆναι αὐτὸν ἐν τῇ κοιλίᾳ.
\par }{\PP \VS{22}Καὶ ὅτε ἐπλήσθησαν αἱ ἡμέραι τοῦ καθαρισμοῦ αὐτῶν κατὰ τὸν νόμον Μωϋσέως, ἀνήγαγον αὐτὸν εἰς Ἱεροσόλυμα παραστῆσαι τῷ Κυρίῳ,
\VS{23}καθὼς γέγραπται ἐν νόμῳ Κυρίου ὅτι Πᾶν ἄρσεν διανοῖγον μήτραν ἅγιον τῷ Κυρίῳ κληθήσεται,
\VS{24}καὶ τοῦ δοῦναι θυσίαν κατὰ τὸ εἰρημένον ἐν τῷ νόμῳ Κυρίου, Ζεῦγος τρυγόνων ἢ δύο νοσσοὺς περιστερῶν.
\par }{\PP \VS{25}Καὶ ἰδοὺ ἄνθρωπος ἦν ἐν Ἰερουσαλὴμ ᾧ ὄνομα Συμεών καὶ ὁ ἄνθρωπος οὗτος δίκαιος καὶ εὐλαβής προσδεχόμενος παράκλησιν τοῦ Ἰσραήλ, καὶ Πνεῦμα ἦν Ἅγιον ἐπ᾽ αὐτόν·
\VS{26}καὶ ἦν αὐτῷ κεχρηματισμένον ὑπὸ τοῦ Πνεύματος τοῦ Ἁγίου μὴ ἰδεῖν θάνατον πρὶν ἢ ἂν ἴδῃ τὸν Χριστὸν Κυρίου.
\VS{27}καὶ ἦλθεν ἐν τῷ Πνεύματι εἰς τὸ ἱερόν· καὶ ἐν τῷ εἰσαγαγεῖν τοὺς γονεῖς τὸ παιδίον Ἰησοῦν τοῦ ποιῆσαι αὐτοὺς κατὰ τὸ εἰθισμένον τοῦ νόμου περὶ αὐτοῦ
\VS{28}καὶ αὐτὸς ἐδέξατο αὐτὸ εἰς τὰς ἀγκάλας καὶ εὐλόγησεν τὸν Θεὸν καὶ εἶπεν·
\begin{poetryblock}
\par }{\PP \begin{quote} \VS{29}Νῦν ἀπολύεις τὸν δοῦλόν σου, Δέσποτα,\end{quote} 
\par }{\PP \begin{quote}κατὰ τὸ ῥῆμά σου ἐν εἰρήνῃ·\end{quote}
\par }{\PP \begin{quote} \VS{30}ὅτι εἶδον οἱ ὀφθαλμοί μου τὸ σωτήριόν σου,\end{quote}
\par }{\PP \begin{quote} \VS{31}ὃ ἡτοίμασας κατὰ πρόσωπον πάντων τῶν λαῶν,\end{quote}
\par }{\PP \begin{quote} \VS{32}φῶς εἰς ἀποκάλυψιν ἐθνῶν\end{quote} 
\par }{\PP \begin{quote}καὶ δόξαν λαοῦ σου Ἰσραήλ.\end{quote}
\end{poetryblock}
\par }{\PP \VS{33}Καὶ ἦν ὁ πατὴρ αὐτοῦ καὶ ἡ μήτηρ θαυμάζοντες ἐπὶ τοῖς λαλουμένοις περὶ αὐτοῦ.
\VS{34}καὶ εὐλόγησεν αὐτοὺς Συμεὼν καὶ εἶπεν πρὸς Μαριὰμ τὴν μητέρα αὐτοῦ· Ἰδοὺ οὗτος κεῖται εἰς πτῶσιν καὶ ἀνάστασιν πολλῶν ἐν τῷ Ἰσραὴλ καὶ εἰς σημεῖον ἀντιλεγόμενον—
\VS{35}καὶ σοῦ δὲ αὐτῆς τὴν ψυχὴν διελεύσεται ῥομφαία— ὅπως ἂν ἀποκαλυφθῶσιν ἐκ πολλῶν καρδιῶν διαλογισμοί.
\VS{36}Καὶ ἦν Ἅννα προφῆτις, θυγάτηρ Φανουήλ, ἐκ φυλῆς Ἀσήρ· αὕτη προβεβηκυῖα ἐν ἡμέραις πολλαῖς, ζήσασα μετὰ ἀνδρὸς ἔτη ἑπτὰ ἀπὸ τῆς παρθενίας αὐτῆς
\VS{37}καὶ αὐτὴ χήρα ἕως ἐτῶν ὀγδοήκοντα τεσσάρων, ἣ οὐκ ἀφίστατο τοῦ ἱεροῦ νηστείαις καὶ δεήσεσιν λατρεύουσα νύκτα καὶ ἡμέραν.
\VS{38}καὶ αὐτῇ τῇ ὥρᾳ ἐπιστᾶσα ἀνθωμολογεῖτο τῷ Θεῷ καὶ ἐλάλει περὶ αὐτοῦ πᾶσιν τοῖς προσδεχομένοις λύτρωσιν Ἰερουσαλήμ.
\par }{\PP \VS{39}Καὶ ὡς ἐτέλεσαν πάντα τὰ κατὰ τὸν νόμον Κυρίου, ἐπέστρεψαν εἰς τὴν Γαλιλαίαν εἰς πόλιν ἑαυτῶν Ναζαρέθ.
\VS{40}Τὸ δὲ παιδίον ηὔξανεν καὶ ἐκραταιοῦτο πληρούμενον σοφίᾳ, καὶ χάρις Θεοῦ ἦν ἐπ᾽ αὐτό.
\par }{\PP \VS{41}Καὶ ἐπορεύοντο οἱ γονεῖς αὐτοῦ κατ᾽ ἔτος εἰς Ἰερουσαλὴμ τῇ ἑορτῇ τοῦ πάσχα.
\VS{42}Καὶ ὅτε ἐγένετο ἐτῶν δώδεκα, ἀναβαινόντων αὐτῶν κατὰ τὸ ἔθος τῆς ἑορτῆς
\VS{43}καὶ τελειωσάντων τὰς ἡμέρας, ἐν τῷ ὑποστρέφειν αὐτοὺς ὑπέμεινεν Ἰησοῦς ὁ παῖς ἐν Ἰερουσαλήμ, καὶ οὐκ ἔγνωσαν οἱ γονεῖς αὐτοῦ.
\VS{44}νομίσαντες δὲ αὐτὸν εἶναι ἐν τῇ συνοδίᾳ ἦλθον ἡμέρας ὁδὸν καὶ ἀνεζήτουν αὐτὸν ἐν τοῖς συγγενεῦσιν καὶ τοῖς γνωστοῖς,
\VS{45}καὶ μὴ εὑρόντες ὑπέστρεψαν εἰς Ἰερουσαλὴμ ἀναζητοῦντες αὐτόν.
\VS{46}Καὶ ἐγένετο μετὰ ἡμέρας τρεῖς εὗρον αὐτὸν ἐν τῷ ἱερῷ καθεζόμενον ἐν μέσῳ τῶν διδασκάλων καὶ ἀκούοντα αὐτῶν καὶ ἐπερωτῶντα αὐτούς·
\VS{47}ἐξίσταντο δὲ πάντες οἱ ἀκούοντες αὐτοῦ ἐπὶ τῇ συνέσει καὶ ταῖς ἀποκρίσεσιν αὐτοῦ.
\VS{48}Καὶ ἰδόντες αὐτὸν ἐξεπλάγησαν, καὶ εἶπεν πρὸς αὐτὸν ἡ μήτηρ αὐτοῦ· Τέκνον, τί ἐποίησας ἡμῖν οὕτως; ἰδοὺ ὁ πατήρ σου κἀγὼ ὀδυνώμενοι ἐζητοῦμέν σε.
\VS{49}Καὶ εἶπεν πρὸς αὐτούς· Τί ὅτι ἐζητεῖτέ με; οὐκ ᾔδειτε ὅτι ἐν τοῖς τοῦ Πατρός μου δεῖ εἶναί με;
\VS{50}καὶ αὐτοὶ οὐ συνῆκαν τὸ ῥῆμα ὃ ἐλάλησεν αὐτοῖς.
\VS{51}Καὶ κατέβη μετ᾽ αὐτῶν καὶ ἦλθεν εἰς Ναζαρὲθ καὶ ἦν ὑποτασσόμενος αὐτοῖς. καὶ ἡ μήτηρ αὐτοῦ διετήρει πάντα τὰ ῥήματα ἐν τῇ καρδίᾳ αὐτῆς.
\VS{52}Καὶ Ἰησοῦς προέκοπτεν ἐν τῇ σοφίᾳ καὶ ἡλικίᾳ καὶ χάριτι παρὰ Θεῷ καὶ ἀνθρώποις.

\par }\Chap{3}{\PP \VerseOne{1}Ἐν ἔτει δὲ πεντεκαιδεκάτῳ τῆς ἡγεμονίας Τιβερίου Καίσαρος, ἡγεμονεύοντος Ποντίου Πιλάτου τῆς Ἰουδαίας, καὶ τετρααρχοῦντος τῆς Γαλιλαίας Ἡρῴδου, Φιλίππου δὲ τοῦ ἀδελφοῦ αὐτοῦ τετρααρχοῦντος τῆς Ἰτουραίας καὶ Τραχωνίτιδος χώρας, καὶ Λυσανίου τῆς Ἀβιληνῆς τετρααρχοῦντος,
\VS{2}ἐπὶ ἀρχιερέως Ἅννα καὶ Καϊάφα, ἐγένετο ῥῆμα Θεοῦ ἐπὶ Ἰωάννην τὸν Ζαχαρίου υἱὸν ἐν τῇ ἐρήμῳ.
\VS{3}Καὶ ἦλθεν εἰς πᾶσαν τὴν περίχωρον τοῦ Ἰορδάνου κηρύσσων βάπτισμα μετανοίας εἰς ἄφεσιν ἁμαρτιῶν,
\VS{4}ὡς γέγραπται ἐν βίβλῳ λόγων Ἠσαΐου τοῦ προφήτου· 
\begin{poetryblock}
\par }{\PP \begin{quote}Φωνὴ βοῶντος ἐν τῇ ἐρήμῳ·\end{quote} 
\par }{\PP \begin{quote}Ἑτοιμάσατε τὴν ὁδὸν Κυρίου,\end{quote} 
\par }{\PP \begin{quote}εὐθείας ποιεῖτε τὰς τρίβους αὐτοῦ·\end{quote}
\par }{\PP \begin{quote} \VS{5}πᾶσα φάραγξ πληρωθήσεται\end{quote} 
\par }{\PP \begin{quote}καὶ πᾶν ὄρος καὶ βουνὸς ταπεινωθήσεται,\end{quote} 
\par }{\PP \begin{quote}καὶ ἔσται τὰ σκολιὰ εἰς εὐθείαν\end{quote} 
\par }{\PP \begin{quote}καὶ αἱ τραχεῖαι εἰς ὁδοὺς λείας·\end{quote}
\par }{\PP \begin{quote} \VS{6}καὶ ὄψεται πᾶσα σὰρξ τὸ σωτήριον τοῦ Θεοῦ.\end{quote}
\end{poetryblock}
\par }{\PP \VS{7}Ἔλεγεν οὖν τοῖς ἐκπορευομένοις ὄχλοις βαπτισθῆναι ὑπ᾽ αὐτοῦ· Γεννήματα ἐχιδνῶν, τίς ὑπέδειξεν ὑμῖν φυγεῖν ἀπὸ τῆς μελλούσης ὀργῆς;
\VS{8}ποιήσατε οὖν καρποὺς ἀξίους τῆς μετανοίας καὶ μὴ ἄρξησθε λέγειν ἐν ἑαυτοῖς· Πατέρα ἔχομεν τὸν Ἀβραάμ. λέγω γὰρ ὑμῖν ὅτι δύναται ὁ Θεὸς ἐκ τῶν λίθων τούτων ἐγεῖραι τέκνα τῷ Ἀβραάμ.
\VS{9}ἤδη δὲ καὶ ἡ ἀξίνη πρὸς τὴν ῥίζαν τῶν δένδρων κεῖται· πᾶν οὖν δένδρον μὴ ποιοῦν καρπὸν καλὸν ἐκκόπτεται καὶ εἰς πῦρ βάλλεται.
\par }{\PP \VS{10}Καὶ ἐπηρώτων αὐτὸν οἱ ὄχλοι λέγοντες· Τί οὖν ποιήσωμεν;
\VS{11}Ἀποκριθεὶς δὲ ἔλεγεν αὐτοῖς· Ὁ ἔχων δύο χιτῶνας μεταδότω τῷ μὴ ἔχοντι, καὶ ὁ ἔχων βρώματα ὁμοίως ποιείτω.
\VS{12}Ἦλθον δὲ καὶ τελῶναι βαπτισθῆναι καὶ εἶπαν πρὸς αὐτόν· Διδάσκαλε, τί ποιήσωμεν;
\VS{13}Ὁ δὲ εἶπεν πρὸς αὐτούς· Μηδὲν πλέον παρὰ τὸ διατεταγμένον ὑμῖν πράσσετε.
\VS{14}Ἐπηρώτων δὲ αὐτὸν καὶ στρατευόμενοι λέγοντες· Τί ποιήσωμεν καὶ ἡμεῖς; Καὶ εἶπεν αὐτοῖς· Μηδένα διασείσητε μηδὲ συκοφαντήσητε καὶ ἀρκεῖσθε τοῖς ὀψωνίοις ὑμῶν.
\par }{\PP \VS{15}Προσδοκῶντος δὲ τοῦ λαοῦ καὶ διαλογιζομένων πάντων ἐν ταῖς καρδίαις αὐτῶν περὶ τοῦ Ἰωάννου, μήποτε αὐτὸς εἴη ὁ Χριστός,
\VS{16}ἀπεκρίνατο λέγων πᾶσιν ὁ Ἰωάννης· Ἐγὼ μὲν ὕδατι βαπτίζω ὑμᾶς· ἔρχεται δὲ ὁ ἰσχυρότερός μου, οὗ οὐκ εἰμὶ ἱκανὸς λῦσαι τὸν ἱμάντα τῶν ὑποδημάτων αὐτοῦ· αὐτὸς ὑμᾶς βαπτίσει ἐν Πνεύματι Ἁγίῳ καὶ πυρί·
\VS{17}οὗ τὸ πτύον ἐν τῇ χειρὶ αὐτοῦ διακαθᾶραι τὴν ἅλωνα αὐτοῦ καὶ συναγαγεῖν τὸν σῖτον εἰς τὴν ἀποθήκην αὐτοῦ, τὸ δὲ ἄχυρον κατακαύσει πυρὶ ἀσβέστῳ.
\par }{\PP \VS{18}Πολλὰ μὲν οὖν καὶ ἕτερα παρακαλῶν εὐηγγελίζετο τὸν λαόν.
\VS{19}ὁ δὲ Ἡρῴδης ὁ τετραάρχης, ἐλεγχόμενος ὑπ᾽ αὐτοῦ περὶ Ἡρῳδιάδος τῆς γυναικὸς τοῦ ἀδελφοῦ αὐτοῦ καὶ περὶ πάντων ὧν ἐποίησεν πονηρῶν ὁ Ἡρῴδης,
\VS{20}προσέθηκεν καὶ τοῦτο ἐπὶ πᾶσιν καὶ κατέκλεισεν τὸν Ἰωάννην ἐν φυλακῇ.
\par }{\PP \VS{21}Ἐγένετο δὲ ἐν τῷ βαπτισθῆναι ἅπαντα τὸν λαὸν καὶ Ἰησοῦ βαπτισθέντος καὶ προσευχομένου ἀνεῳχθῆναι τὸν οὐρανὸν
\VS{22}καὶ καταβῆναι τὸ Πνεῦμα τὸ Ἅγιον σωματικῷ εἴδει ὡς περιστερὰν ἐπ᾽ αὐτόν, καὶ φωνὴν ἐξ οὐρανοῦ γενέσθαι· Σὺ εἶ ὁ Υἱός μου ὁ ἀγαπητός, ἐν σοὶ εὐδόκησα.
\par }{\PP \VS{23}Καὶ αὐτὸς ἦν Ἰησοῦς ἀρχόμενος ὡσεὶ ἐτῶν τριάκοντα, Ὢν υἱός, ὡς ἐνομίζετο, Ἰωσὴφ τοῦ Ἠλὶ
\VS{24}τοῦ Μαθθὰτ τοῦ Λευὶ τοῦ Μελχὶ τοῦ Ἰανναὶ τοῦ Ἰωσὴφ
\VS{25}τοῦ Ματταθίου τοῦ Ἀμὼς τοῦ Ναοὺμ τοῦ Ἑσλὶ τοῦ Ναγγαὶ
\VS{26}τοῦ Μαὰθ τοῦ Ματταθίου τοῦ Σεμεῒν τοῦ Ἰωσὴχ τοῦ Ἰωδὰ
\VS{27}τοῦ Ἰωανὰν τοῦ Ῥησὰ τοῦ Ζοροβάβελ τοῦ Σαλαθιὴλ τοῦ Νηρὶ
\VS{28}τοῦ Μελχὶ τοῦ Ἀδδὶ τοῦ Κωσὰμ τοῦ Ἐλμαδὰμ τοῦ Ἢρ
\VS{29}τοῦ Ἰησοῦ τοῦ Ἐλιέζερ τοῦ Ἰωρὶμ τοῦ Μαθθὰτ τοῦ Λευὶ
\VS{30}τοῦ Συμεὼν τοῦ Ἰούδα τοῦ Ἰωσὴφ τοῦ Ἰωνὰμ τοῦ Ἐλιακὶμ
\VS{31}τοῦ Μελεὰ τοῦ Μεννὰ τοῦ Ματταθὰ τοῦ Ναθὰμ τοῦ Δαυὶδ
\VS{32}τοῦ Ἰεσσαὶ τοῦ Ἰωβὴδ τοῦ Βοὸς τοῦ Σαλὰ τοῦ Ναασσὼν
\VS{33}τοῦ Ἀμιναδὰβ τοῦ Ἀδμὶν τοῦ Ἀρνὶ τοῦ Ἑσρὼμ τοῦ Φαρὲς τοῦ Ἰούδα
\VS{34}τοῦ Ἰακὼβ τοῦ Ἰσαὰκ τοῦ Ἀβραὰμ τοῦ Θάρα τοῦ Ναχὼρ
\VS{35}τοῦ Σεροὺχ τοῦ Ῥαγαῦ τοῦ Φάλεκ τοῦ Ἔβερ τοῦ Σαλὰ
\VS{36}τοῦ Καϊνὰμ τοῦ Ἀρφαξὰδ τοῦ Σὴμ τοῦ Νῶε τοῦ Λάμεχ
\VS{37}τοῦ Μαθουσαλὰ τοῦ Ἑνὼχ τοῦ Ἰάρετ τοῦ Μαλελεὴλ τοῦ Καϊνὰμ
\VS{38}τοῦ Ἐνὼς τοῦ Σὴθ τοῦ Ἀδὰμ τοῦ Θεοῦ.

\par }\Chap{4}{\PP \VerseOne{1}Ἰησοῦς δὲ πλήρης Πνεύματος Ἁγίου ὑπέστρεψεν ἀπὸ τοῦ Ἰορδάνου καὶ ἤγετο ἐν τῷ Πνεύματι ἐν τῇ ἐρήμῳ
\VS{2}ἡμέρας τεσσεράκοντα πειραζόμενος ὑπὸ τοῦ διαβόλου. Καὶ οὐκ ἔφαγεν οὐδὲν ἐν ταῖς ἡμέραις ἐκείναις καὶ συντελεσθεισῶν αὐτῶν ἐπείνασεν.
\VS{3}Εἶπεν δὲ αὐτῷ ὁ διάβολος· Εἰ Υἱὸς εἶ τοῦ Θεοῦ, εἰπὲ τῷ λίθῳ τούτῳ ἵνα γένηται ἄρτος.
\VS{4}Καὶ ἀπεκρίθη πρὸς αὐτὸν ὁ Ἰησοῦς· Γέγραπται ὅτι Οὐκ ἐπ᾽ ἄρτῳ μόνῳ ζήσεται ὁ ἄνθρωπος.
\VS{5}Καὶ ἀναγαγὼν αὐτὸν ἔδειξεν αὐτῷ πάσας τὰς βασιλείας τῆς οἰκουμένης ἐν στιγμῇ χρόνου
\VS{6}καὶ εἶπεν αὐτῷ ὁ διάβολος· Σοὶ δώσω τὴν ἐξουσίαν ταύτην ἅπασαν καὶ τὴν δόξαν αὐτῶν, ὅτι ἐμοὶ παραδέδοται καὶ ᾧ ἐὰν θέλω δίδωμι αὐτήν·
\VS{7}σὺ οὖν ἐὰν προσκυνήσῃς ἐνώπιον ἐμοῦ, ἔσται σοῦ πᾶσα.
\VS{8}Καὶ ἀποκριθεὶς ὁ Ἰησοῦς εἶπεν αὐτῷ· Γέγραπται· Κύριον τὸν Θεόν σου προσκυνήσεις καὶ αὐτῷ μόνῳ λατρεύσεις.
\par }{\PP \VS{9}Ἤγαγεν δὲ αὐτὸν εἰς Ἰερουσαλὴμ καὶ ἔστησεν ἐπὶ τὸ πτερύγιον τοῦ ἱεροῦ καὶ εἶπεν αὐτῷ· Εἰ Υἱὸς εἶ τοῦ Θεοῦ, βάλε σεαυτὸν ἐντεῦθεν κάτω·
\VS{10}γέγραπται γὰρ ὅτι 
\begin{poetryblock}
\par }{\PP \begin{quote}Τοῖς ἀγγέλοις αὐτοῦ ἐντελεῖται περὶ σοῦ\end{quote} 
\par }{\PP \begin{quote}τοῦ διαφυλάξαι σε\end{quote}
\end{poetryblock}
\par }{\PP \VS{11}καὶ ὅτι 
\begin{poetryblock}
\par }{\PP \begin{quote}ἐπὶ χειρῶν ἀροῦσίν σε,\end{quote} 
\par }{\PP \begin{quote}μήποτε προσκόψῃς πρὸς λίθον τὸν πόδα σου.\end{quote}
\end{poetryblock}
\par }{\PP \VS{12}Καὶ ἀποκριθεὶς εἶπεν αὐτῷ ὁ Ἰησοῦς ὅτι Εἴρηται· Οὐκ ἐκπειράσεις Κύριον τὸν Θεόν σου.
\par }{\PP \VS{13}Καὶ συντελέσας πάντα πειρασμὸν ὁ διάβολος ἀπέστη ἀπ᾽ αὐτοῦ ἄχρι καιροῦ.
\par }{\PP \VS{14}Καὶ ὑπέστρεψεν ὁ Ἰησοῦς ἐν τῇ δυνάμει τοῦ Πνεύματος εἰς τὴν Γαλιλαίαν. καὶ φήμη ἐξῆλθεν καθ᾽ ὅλης τῆς περιχώρου περὶ αὐτοῦ.
\VS{15}καὶ αὐτὸς ἐδίδασκεν ἐν ταῖς συναγωγαῖς αὐτῶν δοξαζόμενος ὑπὸ πάντων.
\par }{\PP \VS{16}Καὶ ἦλθεν εἰς Ναζαρά, οὗ ἦν τεθραμμένος, καὶ εἰσῆλθεν κατὰ τὸ εἰωθὸς αὐτῷ ἐν τῇ ἡμέρᾳ τῶν σαββάτων εἰς τὴν συναγωγήν καὶ ἀνέστη ἀναγνῶναι.
\VS{17}καὶ ἐπεδόθη αὐτῷ βιβλίον τοῦ προφήτου Ἠσαΐου καὶ ἀναπτύξας τὸ βιβλίον εὗρεν τὸν τόπον οὗ ἦν γεγραμμένον·
\begin{poetryblock}
\par }{\PP \begin{quote} \VS{18}Πνεῦμα Κυρίου ἐπ᾽ ἐμέ\end{quote} 
\par }{\PP \begin{quote}Οὗ εἵνεκεν ἔχρισέν με\end{quote} 
\par }{\PP \begin{quote}Εὐαγγελίσασθαι πτωχοῖς,\end{quote} 
\par }{\PP \begin{quote}Ἀπέσταλκέν με,\end{quote} 
\par }{\PP \begin{quote}κηρῦξαι αἰχμαλώτοις ἄφεσιν\end{quote} 
\par }{\PP \begin{quote}Καὶ τυφλοῖς ἀνάβλεψιν,\end{quote} 
\par }{\PP \begin{quote}Ἀποστεῖλαι τεθραυσμένους ἐν ἀφέσει,\end{quote}
\par }{\PP \begin{quote} \VS{19}Κηρῦξαι ἐνιαυτὸν Κυρίου δεκτόν.\end{quote}
\end{poetryblock}
\par }{\PP \VS{20}Καὶ πτύξας τὸ βιβλίον ἀποδοὺς τῷ ὑπηρέτῃ ἐκάθισεν· καὶ πάντων οἱ ὀφθαλμοὶ ἐν τῇ συναγωγῇ ἦσαν ἀτενίζοντες αὐτῷ.
\VS{21}ἤρξατο δὲ λέγειν πρὸς αὐτοὺς ὅτι Σήμερον πεπλήρωται ἡ γραφὴ αὕτη ἐν τοῖς ὠσὶν ὑμῶν.
\VS{22}Καὶ πάντες ἐμαρτύρουν αὐτῷ καὶ ἐθαύμαζον ἐπὶ τοῖς λόγοις τῆς χάριτος τοῖς ἐκπορευομένοις ἐκ τοῦ στόματος αὐτοῦ καὶ ἔλεγον· Οὐχὶ υἱός ἐστιν Ἰωσὴφ οὗτος;
\VS{23}Καὶ εἶπεν πρὸς αὐτούς· Πάντως ἐρεῖτέ μοι τὴν παραβολὴν ταύτην· Ἰατρέ, θεράπευσον σεαυτόν· ὅσα ἠκούσαμεν γενόμενα εἰς τὴν Καφαρναοὺμ ποίησον καὶ ὧδε ἐν τῇ πατρίδι σου.
\VS{24}Εἶπεν δέ· Ἀμὴν λέγω ὑμῖν ὅτι οὐδεὶς προφήτης δεκτός ἐστιν ἐν τῇ πατρίδι αὐτοῦ.
\VS{25}ἐπ᾽ ἀληθείας δὲ λέγω ὑμῖν, πολλαὶ χῆραι ἦσαν ἐν ταῖς ἡμέραις Ἠλίου ἐν τῷ Ἰσραήλ, ὅτε ἐκλείσθη ὁ οὐρανὸς ἐπὶ ἔτη τρία καὶ μῆνας ἕξ, ὡς ἐγένετο λιμὸς μέγας ἐπὶ πᾶσαν τὴν γῆν,
\VS{26}καὶ πρὸς οὐδεμίαν αὐτῶν ἐπέμφθη Ἠλίας εἰ μὴ εἰς Σάρεπτα τῆς Σιδωνίας πρὸς γυναῖκα χήραν.
\VS{27}καὶ πολλοὶ λεπροὶ ἦσαν ἐν τῷ Ἰσραὴλ ἐπὶ Ἐλισαίου τοῦ προφήτου, καὶ οὐδεὶς αὐτῶν ἐκαθαρίσθη εἰ μὴ Ναιμὰν ὁ Σύρος.
\VS{28}Καὶ ἐπλήσθησαν πάντες θυμοῦ ἐν τῇ συναγωγῇ ἀκούοντες ταῦτα
\VS{29}καὶ ἀναστάντες ἐξέβαλον αὐτὸν ἔξω τῆς πόλεως καὶ ἤγαγον αὐτὸν ἕως ὀφρύος τοῦ ὄρους ἐφ᾽ οὗ ἡ πόλις ᾠκοδόμητο αὐτῶν ὥστε κατακρημνίσαι αὐτόν·
\VS{30}αὐτὸς δὲ διελθὼν διὰ μέσου αὐτῶν ἐπορεύετο.
\par }{\PP \VS{31}Καὶ κατῆλθεν εἰς Καφαρναοὺμ πόλιν τῆς Γαλιλαίας. καὶ ἦν διδάσκων αὐτοὺς ἐν τοῖς σάββασιν·
\VS{32}καὶ ἐξεπλήσσοντο ἐπὶ τῇ διδαχῇ αὐτοῦ, ὅτι ἐν ἐξουσίᾳ ἦν ὁ λόγος αὐτοῦ.
\par }{\PP \VS{33}Καὶ ἐν τῇ συναγωγῇ ἦν ἄνθρωπος ἔχων πνεῦμα δαιμονίου ἀκαθάρτου καὶ ἀνέκραξεν φωνῇ μεγάλῃ·
\VS{34}Ἔα, τί ἡμῖν καὶ σοί, Ἰησοῦ Ναζαρηνέ; ἦλθες ἀπολέσαι ἡμᾶς; οἶδά σε τίς εἶ, ὁ Ἅγιος τοῦ Θεοῦ.
\VS{35}Καὶ ἐπετίμησεν αὐτῷ ὁ Ἰησοῦς λέγων· Φιμώθητι καὶ ἔξελθε ἀπ᾽ αὐτοῦ. καὶ ῥίψαν αὐτὸν τὸ δαιμόνιον εἰς τὸ μέσον ἐξῆλθεν ἀπ᾽ αὐτοῦ μηδὲν βλάψαν αὐτόν.
\VS{36}Καὶ ἐγένετο θάμβος ἐπὶ πάντας καὶ συνελάλουν πρὸς ἀλλήλους λέγοντες· Τίς ὁ λόγος οὗτος ὅτι ἐν ἐξουσίᾳ καὶ δυνάμει ἐπιτάσσει τοῖς ἀκαθάρτοις πνεύμασιν καὶ ἐξέρχονται;
\VS{37}καὶ ἐξεπορεύετο ἦχος περὶ αὐτοῦ εἰς πάντα τόπον τῆς περιχώρου.
\par }{\PP \VS{38}Ἀναστὰς δὲ ἀπὸ τῆς συναγωγῆς εἰσῆλθεν εἰς τὴν οἰκίαν Σίμωνος. πενθερὰ δὲ τοῦ Σίμωνος ἦν συνεχομένη πυρετῷ μεγάλῳ καὶ ἠρώτησαν αὐτὸν περὶ αὐτῆς.
\VS{39}καὶ ἐπιστὰς ἐπάνω αὐτῆς ἐπετίμησεν τῷ πυρετῷ καὶ ἀφῆκεν αὐτήν· παραχρῆμα δὲ ἀναστᾶσα διηκόνει αὐτοῖς.
\par }{\PP \VS{40}Δύνοντος δὲ τοῦ ἡλίου ἅπαντες ὅσοι εἶχον ἀσθενοῦντας νόσοις ποικίλαις ἤγαγον αὐτοὺς πρὸς αὐτόν· ὁ δὲ ἑνὶ ἑκάστῳ αὐτῶν τὰς χεῖρας ἐπιτιθεὶς ἐθεράπευεν αὐτούς.
\VS{41}ἐξήρχετο δὲ καὶ δαιμόνια ἀπὸ πολλῶν κραυγάζοντα καὶ λέγοντα ὅτι Σὺ εἶ ὁ Υἱὸς τοῦ Θεοῦ. καὶ ἐπιτιμῶν οὐκ εἴα αὐτὰ λαλεῖν, ὅτι ᾔδεισαν τὸν Χριστὸν αὐτὸν εἶναι.
\par }{\PP \VS{42}Γενομένης δὲ ἡμέρας ἐξελθὼν ἐπορεύθη εἰς ἔρημον τόπον· καὶ οἱ ὄχλοι ἐπεζήτουν αὐτόν καὶ ἦλθον ἕως αὐτοῦ καὶ κατεῖχον αὐτὸν τοῦ μὴ πορεύεσθαι ἀπ᾽ αὐτῶν.
\VS{43}ὁ δὲ εἶπεν πρὸς αὐτοὺς ὅτι Καὶ ταῖς ἑτέραις πόλεσιν εὐαγγελίσασθαί με δεῖ τὴν βασιλείαν τοῦ Θεοῦ, ὅτι ἐπὶ τοῦτο ἀπεστάλην.
\VS{44}Καὶ ἦν κηρύσσων εἰς τὰς συναγωγὰς τῆς Ἰουδαίας.

\par }\Chap{5}{\PP \VerseOne{1}Ἐγένετο δὲ ἐν τῷ τὸν ὄχλον ἐπικεῖσθαι αὐτῷ καὶ ἀκούειν τὸν λόγον τοῦ Θεοῦ καὶ αὐτὸς ἦν ἑστὼς παρὰ τὴν λίμνην Γεννησαρέτ
\VS{2}καὶ εἶδεν δύο πλοῖα ἑστῶτα παρὰ τὴν λίμνην· οἱ δὲ ἁλιεῖς ἀπ᾽ αὐτῶν ἀποβάντες ἔπλυνον τὰ δίκτυα.
\VS{3}ἐμβὰς δὲ εἰς ἓν τῶν πλοίων, ὃ ἦν Σίμωνος, ἠρώτησεν αὐτὸν ἀπὸ τῆς γῆς ἐπαναγαγεῖν ὀλίγον· καθίσας δὲ ἐκ τοῦ πλοίου ἐδίδασκεν τοὺς ὄχλους.
\par }{\PP \VS{4}Ὡς δὲ ἐπαύσατο λαλῶν, εἶπεν πρὸς τὸν Σίμωνα· Ἐπανάγαγε εἰς τὸ βάθος καὶ χαλάσατε τὰ δίκτυα ὑμῶν εἰς ἄγραν.
\VS{5}Καὶ ἀποκριθεὶς Σίμων εἶπεν· Ἐπιστάτα, δι᾽ ὅλης νυκτὸς κοπιάσαντες οὐδὲν ἐλάβομεν· ἐπὶ δὲ τῷ ῥήματί σου χαλάσω τὰ δίκτυα.
\VS{6}καὶ τοῦτο ποιήσαντες συνέκλεισαν πλῆθος ἰχθύων πολύ, διερρήσσετο δὲ τὰ δίκτυα αὐτῶν.
\VS{7}καὶ κατένευσαν τοῖς μετόχοις ἐν τῷ ἑτέρῳ πλοίῳ τοῦ ἐλθόντας συλλαβέσθαι αὐτοῖς· καὶ ἦλθον καὶ ἔπλησαν ἀμφότερα τὰ πλοῖα ὥστε βυθίζεσθαι αὐτά.
\VS{8}Ἰδὼν δὲ Σίμων Πέτρος προσέπεσεν τοῖς γόνασιν Ἰησοῦ λέγων· Ἔξελθε ἀπ᾽ ἐμοῦ, ὅτι ἀνὴρ ἁμαρτωλός εἰμι, Κύριε.
\VS{9}θάμβος γὰρ περιέσχεν αὐτὸν καὶ πάντας τοὺς σὺν αὐτῷ ἐπὶ τῇ ἄγρᾳ τῶν ἰχθύων ὧν συνέλαβον,
\VS{10}ὁμοίως δὲ καὶ Ἰάκωβον καὶ Ἰωάννην υἱοὺς Ζεβεδαίου, οἳ ἦσαν κοινωνοὶ τῷ Σίμωνι. Καὶ εἶπεν πρὸς τὸν Σίμωνα ὁ Ἰησοῦς· Μὴ φοβοῦ· ἀπὸ τοῦ νῦν ἀνθρώπους ἔσῃ ζωγρῶν.
\VS{11}καὶ καταγαγόντες τὰ πλοῖα ἐπὶ τὴν γῆν ἀφέντες πάντα ἠκολούθησαν αὐτῷ.
\par }{\PP \VS{12}Καὶ ἐγένετο ἐν τῷ εἶναι αὐτὸν ἐν μιᾷ τῶν πόλεων καὶ ἰδοὺ ἀνὴρ πλήρης λέπρας· ἰδὼν δὲ τὸν Ἰησοῦν, πεσὼν ἐπὶ πρόσωπον ἐδεήθη αὐτοῦ λέγων· Κύριε, ἐὰν θέλῃς δύνασαί με καθαρίσαι.
\VS{13}Καὶ ἐκτείνας τὴν χεῖρα ἥψατο αὐτοῦ λέγων· Θέλω, καθαρίσθητι· καὶ εὐθέως ἡ λέπρα ἀπῆλθεν ἀπ᾽ αὐτοῦ.
\VS{14}Καὶ αὐτὸς παρήγγειλεν αὐτῷ μηδενὶ εἰπεῖν, Ἀλλὰ ἀπελθὼν δεῖξον σεαυτὸν τῷ ἱερεῖ καὶ προσένεγκε περὶ τοῦ καθαρισμοῦ σου καθὼς προσέταξεν Μωϋσῆς, εἰς μαρτύριον αὐτοῖς.
\VS{15}Διήρχετο δὲ μᾶλλον ὁ λόγος περὶ αὐτοῦ, καὶ συνήρχοντο ὄχλοι πολλοὶ ἀκούειν καὶ θεραπεύεσθαι ἀπὸ τῶν ἀσθενειῶν αὐτῶν·
\VS{16}αὐτὸς δὲ ἦν ὑποχωρῶν ἐν ταῖς ἐρήμοις καὶ προσευχόμενος.
\par }{\PP \VS{17}Καὶ ἐγένετο ἐν μιᾷ τῶν ἡμερῶν καὶ αὐτὸς ἦν διδάσκων, καὶ ἦσαν καθήμενοι Φαρισαῖοι καὶ νομοδιδάσκαλοι οἳ ἦσαν ἐληλυθότες ἐκ πάσης κώμης τῆς Γαλιλαίας καὶ Ἰουδαίας καὶ Ἰερουσαλήμ· καὶ δύναμις Κυρίου ἦν εἰς τὸ ἰᾶσθαι αὐτόν.
\VS{18}Καὶ ἰδοὺ ἄνδρες φέροντες ἐπὶ κλίνης ἄνθρωπον ὃς ἦν παραλελυμένος καὶ ἐζήτουν αὐτὸν εἰσενεγκεῖν καὶ θεῖναι αὐτὸν ἐνώπιον αὐτοῦ.
\VS{19}καὶ μὴ εὑρόντες ποίας εἰσενέγκωσιν αὐτὸν διὰ τὸν ὄχλον, ἀναβάντες ἐπὶ τὸ δῶμα διὰ τῶν κεράμων καθῆκαν αὐτὸν σὺν τῷ κλινιδίῳ εἰς τὸ μέσον ἔμπροσθεν τοῦ Ἰησοῦ.
\VS{20}Καὶ ἰδὼν τὴν πίστιν αὐτῶν εἶπεν· Ἄνθρωπε, ἀφέωνταί σοι αἱ ἁμαρτίαι σου.
\VS{21}Καὶ ἤρξαντο διαλογίζεσθαι οἱ γραμματεῖς καὶ οἱ Φαρισαῖοι λέγοντες· Τίς ἐστιν οὗτος ὃς λαλεῖ βλασφημίας; τίς δύναται ἁμαρτίας ἀφεῖναι εἰ μὴ μόνος ὁ Θεός;
\VS{22}Ἐπιγνοὺς δὲ ὁ Ἰησοῦς τοὺς διαλογισμοὺς αὐτῶν ἀποκριθεὶς εἶπεν πρὸς αὐτούς· Τί διαλογίζεσθε ἐν ταῖς καρδίαις ὑμῶν;
\VS{23}τί ἐστιν εὐκοπώτερον, εἰπεῖν· Ἀφέωνταί σοι αἱ ἁμαρτίαι σου, ἢ εἰπεῖν· Ἔγειρε καὶ περιπάτει;
\VS{24}ἵνα δὲ εἰδῆτε ὅτι ὁ Υἱὸς τοῦ ἀνθρώπου ἐξουσίαν ἔχει ἐπὶ τῆς γῆς ἀφιέναι ἁμαρτίας— εἶπεν τῷ παραλελυμένῳ· Σοὶ λέγω, ἔγειρε καὶ ἄρας τὸ κλινίδιόν σου πορεύου εἰς τὸν οἶκόν σου.
\VS{25}Καὶ παραχρῆμα ἀναστὰς ἐνώπιον αὐτῶν, ἄρας ἐφ᾽ ὃ κατέκειτο, ἀπῆλθεν εἰς τὸν οἶκον αὐτοῦ δοξάζων τὸν Θεόν.
\VS{26}καὶ ἔκστασις ἔλαβεν ἅπαντας καὶ ἐδόξαζον τὸν Θεόν καὶ ἐπλήσθησαν φόβου λέγοντες ὅτι Εἴδομεν παράδοξα σήμερον.
\par }{\PP \VS{27}Καὶ μετὰ ταῦτα ἐξῆλθεν καὶ ἐθεάσατο τελώνην ὀνόματι Λευὶν καθήμενον ἐπὶ τὸ τελώνιον, καὶ εἶπεν αὐτῷ· Ἀκολούθει μοι.
\VS{28}καὶ καταλιπὼν πάντα ἀναστὰς ἠκολούθει αὐτῷ.
\VS{29}Καὶ ἐποίησεν δοχὴν μεγάλην Λευὶς αὐτῷ ἐν τῇ οἰκίᾳ αὐτοῦ, καὶ ἦν ὄχλος πολὺς τελωνῶν καὶ ἄλλων οἳ ἦσαν μετ᾽ αὐτῶν κατακείμενοι.
\VS{30}καὶ ἐγόγγυζον οἱ Φαρισαῖοι καὶ οἱ γραμματεῖς αὐτῶν πρὸς τοὺς μαθητὰς αὐτοῦ λέγοντες· Διὰ τί μετὰ τῶν τελωνῶν καὶ ἁμαρτωλῶν ἐσθίετε καὶ πίνετε;
\VS{31}Καὶ ἀποκριθεὶς ὁ Ἰησοῦς εἶπεν πρὸς αὐτούς· Οὐ χρείαν ἔχουσιν οἱ ὑγιαίνοντες ἰατροῦ ἀλλὰ οἱ κακῶς ἔχοντες·
\VS{32}οὐκ ἐλήλυθα καλέσαι δικαίους ἀλλὰ ἁμαρτωλοὺς εἰς μετάνοιαν.
\par }{\PP \VS{33}Οἱ δὲ εἶπαν πρὸς αὐτόν· Οἱ μαθηταὶ Ἰωάννου νηστεύουσιν πυκνὰ καὶ δεήσεις ποιοῦνται ὁμοίως καὶ οἱ τῶν Φαρισαίων, οἱ δὲ σοὶ ἐσθίουσιν καὶ πίνουσιν.
\VS{34}Ὁ δὲ Ἰησοῦς εἶπεν πρὸς αὐτούς· Μὴ δύνασθε τοὺς υἱοὺς τοῦ νυμφῶνος ἐν ᾧ ὁ νυμφίος μετ᾽ αὐτῶν ἐστιν ποιῆσαι νηστεῦσαι;
\VS{35}ἐλεύσονται δὲ ἡμέραι, καὶ ὅταν ἀπαρθῇ ἀπ᾽ αὐτῶν ὁ νυμφίος, τότε νηστεύσουσιν ἐν ἐκείναις ταῖς ἡμέραις.
\par }{\PP \VS{36}Ἔλεγεν δὲ καὶ παραβολὴν πρὸς αὐτοὺς ὅτι Οὐδεὶς ἐπίβλημα ἀπὸ ἱματίου καινοῦ σχίσας ἐπιβάλλει ἐπὶ ἱμάτιον παλαιόν· εἰ δὲ μή γε, καὶ τὸ καινὸν σχίσει καὶ τῷ παλαιῷ οὐ συμφωνήσει τὸ ἐπίβλημα τὸ ἀπὸ τοῦ καινοῦ.
\VS{37}Καὶ οὐδεὶς βάλλει οἶνον νέον εἰς ἀσκοὺς παλαιούς· εἰ δὲ μή γε, ῥήξει ὁ οἶνος ὁ νέος τοὺς ἀσκούς καὶ αὐτὸς ἐκχυθήσεται καὶ οἱ ἀσκοὶ ἀπολοῦνται·
\VS{38}ἀλλὰ οἶνον νέον εἰς ἀσκοὺς καινοὺς βλητέον.
\VS{39}καὶ οὐδεὶς πιὼν παλαιὸν θέλει νέον· λέγει γάρ· Ὁ παλαιὸς χρηστός ἐστιν.

\par }\Chap{6}{\PP \VerseOne{1}Ἐγένετο δὲ ἐν σαββάτῳ διαπορεύεσθαι αὐτὸν διὰ σπορίμων, καὶ ἔτιλλον οἱ μαθηταὶ αὐτοῦ καὶ ἤσθιον τοὺς στάχυας ψώχοντες ταῖς χερσίν.
\VS{2}τινὲς δὲ τῶν Φαρισαίων εἶπαν· Τί ποιεῖτε ὃ οὐκ ἔξεστιν τοῖς σάββασιν;
\VS{3}Καὶ ἀποκριθεὶς πρὸς αὐτοὺς εἶπεν ὁ Ἰησοῦς· Οὐδὲ τοῦτο ἀνέγνωτε ὃ ἐποίησεν Δαυὶδ ὅτε ἐπείνασεν αὐτὸς καὶ οἱ μετ᾽ αὐτοῦ ὄντες,
\VS{4}ὡς εἰσῆλθεν εἰς τὸν οἶκον τοῦ Θεοῦ καὶ τοὺς ἄρτους τῆς προθέσεως λαβὼν ἔφαγεν καὶ ἔδωκεν τοῖς μετ᾽ αὐτοῦ, οὓς οὐκ ἔξεστιν φαγεῖν εἰ μὴ μόνους τοὺς ἱερεῖς;
\VS{5}καὶ ἔλεγεν αὐτοῖς· Κύριός ἐστιν τοῦ σαββάτου ὁ Υἱὸς τοῦ ἀνθρώπου.
\VS{6}Ἐγένετο δὲ ἐν ἑτέρῳ σαββάτῳ εἰσελθεῖν αὐτὸν εἰς τὴν συναγωγὴν καὶ διδάσκειν. καὶ ἦν ἄνθρωπος ἐκεῖ καὶ ἡ χεὶρ αὐτοῦ ἡ δεξιὰ ἦν ξηρά.
\VS{7}παρετηροῦντο δὲ αὐτὸν οἱ γραμματεῖς καὶ οἱ Φαρισαῖοι εἰ ἐν τῷ σαββάτῳ θεραπεύει, ἵνα εὕρωσιν κατηγορεῖν αὐτοῦ.
\VS{8}Αὐτὸς δὲ ᾔδει τοὺς διαλογισμοὺς αὐτῶν, εἶπεν δὲ τῷ ἀνδρὶ τῷ ξηρὰν ἔχοντι τὴν χεῖρα· Ἔγειρε καὶ στῆθι εἰς τὸ μέσον· καὶ ἀναστὰς ἔστη.
\VS{9}εἶπεν δὲ ὁ Ἰησοῦς πρὸς αὐτούς· Ἐπερωτῶ ὑμᾶς εἰ ἔξεστιν τῷ σαββάτῳ ἀγαθοποιῆσαι ἢ κακοποιῆσαι, ψυχὴν σῶσαι ἢ ἀπολέσαι;
\VS{10}Καὶ περιβλεψάμενος πάντας αὐτοὺς εἶπεν αὐτῷ· Ἔκτεινον τὴν χεῖρά σου. ὁ δὲ ἐποίησεν καὶ ἀπεκατεστάθη ἡ χεὶρ αὐτοῦ.
\VS{11}αὐτοὶ δὲ ἐπλήσθησαν ἀνοίας καὶ διελάλουν πρὸς ἀλλήλους τί ἂν ποιήσαιεν τῷ Ἰησοῦ.
\par }{\PP \VS{12}Ἐγένετο δὲ ἐν ταῖς ἡμέραις ταύταις ἐξελθεῖν αὐτὸν εἰς τὸ ὄρος προσεύξασθαι, καὶ ἦν διανυκτερεύων ἐν τῇ προσευχῇ τοῦ Θεοῦ.
\VS{13}καὶ ὅτε ἐγένετο ἡμέρα, προσεφώνησεν τοὺς μαθητὰς αὐτοῦ, καὶ ἐκλεξάμενος ἀπ᾽ αὐτῶν δώδεκα, οὓς καὶ ἀποστόλους ὠνόμασεν·
\VS{14}Σίμωνα ὃν καὶ ὠνόμασεν Πέτρον, καὶ Ἀνδρέαν τὸν ἀδελφὸν αὐτοῦ, καὶ Ἰάκωβον καὶ Ἰωάννην καὶ Φίλιππον καὶ Βαρθολομαῖον
\VS{15}καὶ Μαθθαῖον καὶ Θωμᾶν καὶ Ἰάκωβον Ἁλφαίου καὶ Σίμωνα τὸν καλούμενον Ζηλωτὴν
\VS{16}καὶ Ἰούδαν Ἰακώβου καὶ Ἰούδαν Ἰσκαριὼθ, ὃς ἐγένετο προδότης.
\par }{\PP \VS{17}Καὶ καταβὰς μετ᾽ αὐτῶν ἔστη ἐπὶ τόπου πεδινοῦ, καὶ ὄχλος πολὺς μαθητῶν αὐτοῦ, καὶ πλῆθος πολὺ τοῦ λαοῦ ἀπὸ πάσης τῆς Ἰουδαίας καὶ Ἰερουσαλὴμ καὶ τῆς παραλίου Τύρου καὶ Σιδῶνος,
\VS{18}οἳ ἦλθον ἀκοῦσαι αὐτοῦ καὶ ἰαθῆναι ἀπὸ τῶν νόσων αὐτῶν· καὶ οἱ ἐνοχλούμενοι ἀπὸ πνευμάτων ἀκαθάρτων ἐθεραπεύοντο,
\VS{19}καὶ πᾶς ὁ ὄχλος ἐζήτουν ἅπτεσθαι αὐτοῦ, ὅτι δύναμις παρ᾽ αὐτοῦ ἐξήρχετο καὶ ἰᾶτο πάντας.
\par }{\PP \VS{20}Καὶ αὐτὸς ἐπάρας τοὺς ὀφθαλμοὺς αὐτοῦ εἰς τοὺς μαθητὰς αὐτοῦ ἔλεγεν· 
\begin{poetryblock}
\par }{\PP \begin{quote}Μακάριοι οἱ πτωχοί, Ὅτι ὑμετέρα ἐστὶν ἡ βασιλεία τοῦ Θεοῦ.\end{quote}
\par }{\PP \begin{quote} \VS{21}Μακάριοι οἱ πεινῶντες νῦν,\end{quote} 
\par }{\PP \begin{quote}Ὅτι χορτασθήσεσθε.\end{quote} 
\par }{\PP \begin{quote}Μακάριοι οἱ κλαίοντες νῦν,\end{quote} 
\par }{\PP \begin{quote}Ὅτι γελάσετε.\end{quote}
\par }{\PP \begin{quote} \VS{22}Μακάριοί ἐστε ὅταν μισήσωσιν ὑμᾶς οἱ ἄνθρωποι καὶ ὅταν ἀφορίσωσιν ὑμᾶς καὶ ὀνειδίσωσιν καὶ ἐκβάλωσιν τὸ ὄνομα ὑμῶν ὡς πονηρὸν ἕνεκα τοῦ Υἱοῦ τοῦ ἀνθρώπου·\end{quote}
\end{poetryblock}
\par }{\PP \VS{23}χάρητε ἐν ἐκείνῃ τῇ ἡμέρᾳ καὶ σκιρτήσατε, ἰδοὺ γὰρ ὁ μισθὸς ὑμῶν πολὺς ἐν τῷ οὐρανῷ· κατὰ τὰ αὐτὰ γὰρ ἐποίουν τοῖς προφήταις οἱ πατέρες αὐτῶν.
\begin{poetryblock}
\par }{\PP \begin{quote} \VS{24}Πλὴν οὐαὶ ὑμῖν τοῖς πλουσίοις,\end{quote} 
\par }{\PP \begin{quote}Ὅτι ἀπέχετε τὴν παράκλησιν ὑμῶν.\end{quote}
\par }{\PP \begin{quote} \VS{25}Οὐαὶ ὑμῖν, οἱ ἐμπεπλησμένοι νῦν,\end{quote} 
\par }{\PP \begin{quote}Ὅτι πεινάσετε.\end{quote} 
\par }{\PP \begin{quote}Οὐαί, οἱ γελῶντες νῦν,\end{quote} 
\par }{\PP \begin{quote}Ὅτι πενθήσετε καὶ κλαύσετε.\end{quote}
\par }{\PP \begin{quote} \VS{26}Οὐαὶ ὅταν ὑμᾶς καλῶς εἴπωσιν πάντες οἱ ἄνθρωποι·\end{quote} 
\par }{\PP \begin{quote}Κατὰ τὰ αὐτὰ γὰρ ἐποίουν τοῖς ψευδοπροφήταις οἱ πατέρες αὐτῶν.\end{quote}
\end{poetryblock}
\par }{\PP \VS{27}Ἀλλὰ ὑμῖν λέγω τοῖς ἀκούουσιν· Ἀγαπᾶτε τοὺς ἐχθροὺς ὑμῶν, καλῶς ποιεῖτε τοῖς μισοῦσιν ὑμᾶς,
\VS{28}εὐλογεῖτε τοὺς καταρωμένους ὑμᾶς, προσεύχεσθε περὶ τῶν ἐπηρεαζόντων ὑμᾶς.
\VS{29}τῷ τύπτοντί σε ἐπὶ τὴν σιαγόνα πάρεχε καὶ τὴν ἄλλην, καὶ ἀπὸ τοῦ αἴροντός σου τὸ ἱμάτιον καὶ τὸν χιτῶνα μὴ κωλύσῃς.
\VS{30}παντὶ αἰτοῦντί σε δίδου, καὶ ἀπὸ τοῦ αἴροντος τὰ σὰ μὴ ἀπαίτει.
\par }{\PP \VS{31}καὶ καθὼς θέλετε ἵνα ποιῶσιν ὑμῖν οἱ ἄνθρωποι ποιεῖτε αὐτοῖς ὁμοίως.
\VS{32}Καὶ εἰ ἀγαπᾶτε τοὺς ἀγαπῶντας ὑμᾶς, ποία ὑμῖν χάρις ἐστίν; καὶ γὰρ οἱ ἁμαρτωλοὶ τοὺς ἀγαπῶντας αὐτοὺς ἀγαπῶσιν.
\VS{33}καὶ γὰρ ἐὰν ἀγαθοποιῆτε τοὺς ἀγαθοποιοῦντας ὑμᾶς, ποία ὑμῖν χάρις ἐστίν; καὶ οἱ ἁμαρτωλοὶ τὸ αὐτὸ ποιοῦσιν.
\VS{34}καὶ ἐὰν δανίσητε παρ᾽ ὧν ἐλπίζετε λαβεῖν, ποία ὑμῖν χάρις ἐστίν; καὶ ἁμαρτωλοὶ ἁμαρτωλοῖς δανίζουσιν ἵνα ἀπολάβωσιν τὰ ἴσα.
\VS{35}Πλὴν ἀγαπᾶτε τοὺς ἐχθροὺς ὑμῶν καὶ ἀγαθοποιεῖτε καὶ δανίζετε μηδὲν ἀπελπίζοντες· καὶ ἔσται ὁ μισθὸς ὑμῶν πολύς, καὶ ἔσεσθε υἱοὶ Ὑψίστου, ὅτι αὐτὸς χρηστός ἐστιν ἐπὶ τοὺς ἀχαρίστους καὶ πονηρούς.
\par }{\PP \VS{36}Γίνεσθε οἰκτίρμονες καθὼς καὶ ὁ Πατὴρ ὑμῶν οἰκτίρμων ἐστίν.
\VS{37}Καὶ μὴ κρίνετε, καὶ οὐ μὴ κριθῆτε· καὶ μὴ καταδικάζετε, καὶ οὐ μὴ καταδικασθῆτε. ἀπολύετε, καὶ ἀπολυθήσεσθε·
\VS{38}δίδοτε, καὶ δοθήσεται ὑμῖν· μέτρον καλὸν πεπιεσμένον σεσαλευμένον ὑπερεκχυννόμενον δώσουσιν εἰς τὸν κόλπον ὑμῶν· ᾧ γὰρ μέτρῳ μετρεῖτε ἀντιμετρηθήσεται ὑμῖν.
\par }{\PP \VS{39}Εἶπεν δὲ καὶ παραβολὴν αὐτοῖς· Μήτι δύναται τυφλὸς τυφλὸν ὁδηγεῖν; οὐχὶ ἀμφότεροι εἰς βόθυνον ἐμπεσοῦνται;
\VS{40}οὐκ ἔστιν μαθητὴς ὑπὲρ τὸν διδάσκαλον· κατηρτισμένος δὲ πᾶς ἔσται ὡς ὁ διδάσκαλος αὐτοῦ.
\par }{\PP \VS{41}Τί δὲ βλέπεις τὸ κάρφος τὸ ἐν τῷ ὀφθαλμῷ τοῦ ἀδελφοῦ σου, τὴν δὲ δοκὸν τὴν ἐν τῷ ἰδίῳ ὀφθαλμῷ οὐ κατανοεῖς;
\VS{42}πῶς δύνασαι λέγειν τῷ ἀδελφῷ σου· Ἀδελφέ, ἄφες ἐκβάλω τὸ κάρφος τὸ ἐν τῷ ὀφθαλμῷ σου, αὐτὸς τὴν ἐν τῷ ὀφθαλμῷ σοῦ δοκὸν οὐ βλέπων; ὑποκριτά, ἔκβαλε πρῶτον τὴν δοκὸν ἐκ τοῦ ὀφθαλμοῦ σοῦ, καὶ τότε διαβλέψεις τὸ κάρφος τὸ ἐν τῷ ὀφθαλμῷ τοῦ ἀδελφοῦ σου ἐκβαλεῖν.
\par }{\PP \VS{43}Οὐ γάρ ἐστιν δένδρον καλὸν ποιοῦν καρπὸν σαπρόν, οὐδὲ πάλιν δένδρον σαπρὸν ποιοῦν καρπὸν καλόν.
\VS{44}ἕκαστον γὰρ δένδρον ἐκ τοῦ ἰδίου καρποῦ γινώσκεται· οὐ γὰρ ἐξ ἀκανθῶν συλλέγουσιν σῦκα οὐδὲ ἐκ βάτου σταφυλὴν τρυγῶσιν.
\VS{45}ὁ ἀγαθὸς ἄνθρωπος ἐκ τοῦ ἀγαθοῦ θησαυροῦ τῆς καρδίας προφέρει τὸ ἀγαθόν, καὶ ὁ πονηρὸς ἐκ τοῦ πονηροῦ προφέρει τὸ πονηρόν· ἐκ γὰρ περισσεύματος καρδίας λαλεῖ τὸ στόμα αὐτοῦ.
\par }{\PP \VS{46}Τί δέ με καλεῖτε· Κύριε κύριε, καὶ οὐ ποιεῖτε ἃ λέγω;
\VS{47}Πᾶς ὁ ἐρχόμενος πρός με καὶ ἀκούων μου τῶν λόγων καὶ ποιῶν αὐτούς, ὑποδείξω ὑμῖν τίνι ἐστὶν ὅμοιος·
\VS{48}ὅμοιός ἐστιν ἀνθρώπῳ οἰκοδομοῦντι οἰκίαν ὃς ἔσκαψεν καὶ ἐβάθυνεν καὶ ἔθηκεν θεμέλιον ἐπὶ τὴν πέτραν· πλημμύρης δὲ γενομένης προσέρηξεν ὁ ποταμὸς τῇ οἰκίᾳ ἐκείνῃ, καὶ οὐκ ἴσχυσεν σαλεῦσαι αὐτὴν διὰ τὸ καλῶς οἰκοδομῆσθαι αὐτήν.
\VS{49}ὁ δὲ ἀκούσας καὶ μὴ ποιήσας ὅμοιός ἐστιν ἀνθρώπῳ οἰκοδομήσαντι οἰκίαν ἐπὶ τὴν γῆν χωρὶς θεμελίου, ᾗ προσέρηξεν ὁ ποταμός, καὶ εὐθὺς συνέπεσεν καὶ ἐγένετο τὸ ῥῆγμα τῆς οἰκίας ἐκείνης μέγα.

\par }\Chap{7}{\PP \VerseOne{1}Ἐπειδὴ ἐπλήρωσεν πάντα τὰ ῥήματα αὐτοῦ εἰς τὰς ἀκοὰς τοῦ λαοῦ, εἰσῆλθεν εἰς Καφαρναούμ.
\VS{2}Ἑκατοντάρχου δέ τινος δοῦλος κακῶς ἔχων ἤμελλεν τελευτᾶν, ὃς ἦν αὐτῷ ἔντιμος.
\VS{3}ἀκούσας δὲ περὶ τοῦ Ἰησοῦ ἀπέστειλεν πρὸς αὐτὸν πρεσβυτέρους τῶν Ἰουδαίων ἐρωτῶν αὐτὸν ὅπως ἐλθὼν διασώσῃ τὸν δοῦλον αὐτοῦ.
\VS{4}οἱ δὲ παραγενόμενοι πρὸς τὸν Ἰησοῦν παρεκάλουν αὐτὸν σπουδαίως λέγοντες ὅτι Ἄξιός ἐστιν ᾧ παρέξῃ τοῦτο·
\VS{5}ἀγαπᾷ γὰρ τὸ ἔθνος ἡμῶν καὶ τὴν συναγωγὴν αὐτὸς ᾠκοδόμησεν ἡμῖν.
\VS{6}Ὁ δὲ Ἰησοῦς ἐπορεύετο σὺν αὐτοῖς. ἤδη δὲ αὐτοῦ οὐ μακρὰν ἀπέχοντος ἀπὸ τῆς οἰκίας ἔπεμψεν φίλους ὁ ἑκατοντάρχης λέγων αὐτῷ· Κύριε, μὴ σκύλλου, οὐ γὰρ ἱκανός εἰμι ἵνα ὑπὸ τὴν στέγην μου εἰσέλθῃς·
\VS{7}διὸ οὐδὲ ἐμαυτὸν ἠξίωσα πρὸς σὲ ἐλθεῖν· ἀλλὰ εἰπὲ λόγῳ, καὶ ἰαθήτω ὁ παῖς μου.
\VS{8}καὶ γὰρ ἐγὼ ἄνθρωπός εἰμι ὑπὸ ἐξουσίαν τασσόμενος ἔχων ὑπ᾽ ἐμαυτὸν στρατιώτας, καὶ λέγω τούτῳ· Πορεύθητι, καὶ πορεύεται, καὶ ἄλλῳ· Ἔρχου, καὶ ἔρχεται, καὶ τῷ δούλῳ μου· Ποίησον τοῦτο, καὶ ποιεῖ.
\VS{9}Ἀκούσας δὲ ταῦτα ὁ Ἰησοῦς ἐθαύμασεν αὐτόν καὶ στραφεὶς τῷ ἀκολουθοῦντι αὐτῷ ὄχλῳ εἶπεν· Λέγω ὑμῖν, οὐδὲ ἐν τῷ Ἰσραὴλ τοσαύτην πίστιν εὗρον.
\VS{10}καὶ ὑποστρέψαντες εἰς τὸν οἶκον οἱ πεμφθέντες εὗρον τὸν δοῦλον ὑγιαίνοντα.
\par }{\PP \VS{11}Καὶ ἐγένετο ἐν τῷ ἑξῆς ἐπορεύθη εἰς πόλιν καλουμένην Ναΐν καὶ συνεπορεύοντο αὐτῷ οἱ μαθηταὶ αὐτοῦ καὶ ὄχλος πολύς.
\VS{12}ὡς δὲ ἤγγισεν τῇ πύλῃ τῆς πόλεως, καὶ ἰδοὺ ἐξεκομίζετο τεθνηκὼς μονογενὴς υἱὸς τῇ μητρὶ αὐτοῦ καὶ αὐτὴ ἦν χήρα, καὶ ὄχλος τῆς πόλεως ἱκανὸς ἦν σὺν αὐτῇ.
\VS{13}καὶ ἰδὼν αὐτὴν ὁ Κύριος ἐσπλαγχνίσθη ἐπ᾽ αὐτῇ καὶ εἶπεν αὐτῇ· Μὴ κλαῖε.
\VS{14}Καὶ προσελθὼν ἥψατο τῆς σοροῦ, οἱ δὲ βαστάζοντες ἔστησαν, καὶ εἶπεν· Νεανίσκε, σοὶ λέγω, ἐγέρθητι.
\VS{15}καὶ ἀνεκάθισεν ὁ νεκρὸς καὶ ἤρξατο λαλεῖν, καὶ ἔδωκεν αὐτὸν τῇ μητρὶ αὐτοῦ.
\VS{16}Ἔλαβεν δὲ φόβος πάντας καὶ ἐδόξαζον τὸν Θεὸν λέγοντες ὅτι Προφήτης μέγας ἠγέρθη ἐν ἡμῖν καὶ ὅτι Ἐπεσκέψατο ὁ Θεὸς τὸν λαὸν αὐτοῦ.
\VS{17}καὶ ἐξῆλθεν ὁ λόγος οὗτος ἐν ὅλῃ τῇ Ἰουδαίᾳ περὶ αὐτοῦ καὶ πάσῃ τῇ περιχώρῳ.
\par }{\PP \VS{18}Καὶ ἀπήγγειλαν Ἰωάννῃ οἱ μαθηταὶ αὐτοῦ περὶ πάντων τούτων. καὶ προσκαλεσάμενος δύο τινὰς τῶν μαθητῶν αὐτοῦ ὁ Ἰωάννης
\VS{19}ἔπεμψεν πρὸς τὸν Κύριον λέγων· Σὺ εἶ ὁ ἐρχόμενος ἢ ἄλλον προσδοκῶμεν;
\VS{20}Παραγενόμενοι δὲ πρὸς αὐτὸν οἱ ἄνδρες εἶπαν· Ἰωάννης ὁ Βαπτιστὴς ἀπέστειλεν ἡμᾶς πρὸς σὲ λέγων· Σὺ εἶ ὁ ἐρχόμενος ἢ ἄλλον προσδοκῶμεν;
\VS{21}Ἐν ἐκείνῃ τῇ ὥρᾳ ἐθεράπευσεν πολλοὺς ἀπὸ νόσων καὶ μαστίγων καὶ πνευμάτων πονηρῶν καὶ τυφλοῖς πολλοῖς ἐχαρίσατο βλέπειν.
\VS{22}καὶ ἀποκριθεὶς εἶπεν αὐτοῖς· Πορευθέντες ἀπαγγείλατε Ἰωάννῃ ἃ εἴδετε καὶ ἠκούσατε· 
\begin{poetryblock}
\par }{\PP \begin{quote}τυφλοὶ ἀναβλέπουσιν, χωλοὶ περιπατοῦσιν,\end{quote} 
\par }{\PP \begin{quote}λεπροὶ καθαρίζονται καὶ κωφοὶ ἀκούουσιν,\end{quote} 
\par }{\PP \begin{quote}νεκροὶ ἐγείρονται, πτωχοὶ εὐαγγελίζονται·\end{quote}
\end{poetryblock}
\par }{\PP \VS{23}καὶ μακάριός ἐστιν ὃς ἐὰν μὴ σκανδαλισθῇ ἐν ἐμοί.
\par }{\PP \VS{24}Ἀπελθόντων δὲ τῶν ἀγγέλων Ἰωάννου ἤρξατο λέγειν πρὸς τοὺς ὄχλους περὶ Ἰωάννου· Τί ἐξήλθατε εἰς τὴν ἔρημον θεάσασθαι; κάλαμον ὑπὸ ἀνέμου σαλευόμενον;
\VS{25}ἀλλὰ τί ἐξήλθατε ἰδεῖν; ἄνθρωπον ἐν μαλακοῖς ἱματίοις ἠμφιεσμένον; ἰδοὺ οἱ ἐν ἱματισμῷ ἐνδόξῳ καὶ τρυφῇ ὑπάρχοντες ἐν τοῖς βασιλείοις εἰσίν.
\VS{26}Ἀλλὰ τί ἐξήλθατε ἰδεῖν; προφήτην; ναί λέγω ὑμῖν, καὶ περισσότερον προφήτου.
\VS{27}οὗτός ἐστιν περὶ οὗ γέγραπται· 
\begin{poetryblock}
\par }{\PP \begin{quote}Ἰδοὺ ἀποστέλλω τὸν ἄγγελόν μου πρὸ προσώπου σου,\end{quote} 
\par }{\PP \begin{quote}Ὃς κατασκευάσει τὴν ὁδόν σου ἔμπροσθέν σου.\end{quote}
\end{poetryblock}
\par }{\PP \VS{28}Λέγω ὑμῖν, μείζων ἐν γεννητοῖς γυναικῶν Ἰωάννου οὐδείς ἐστιν· ὁ δὲ μικρότερος ἐν τῇ βασιλείᾳ τοῦ Θεοῦ μείζων αὐτοῦ ἐστιν.
\par }{\PP \VS{29}Καὶ πᾶς ὁ λαὸς ἀκούσας καὶ οἱ τελῶναι ἐδικαίωσαν τὸν Θεόν βαπτισθέντες τὸ βάπτισμα Ἰωάννου·
\VS{30}οἱ δὲ Φαρισαῖοι καὶ οἱ νομικοὶ τὴν βουλὴν τοῦ Θεοῦ ἠθέτησαν εἰς ἑαυτούς μὴ βαπτισθέντες ὑπ᾽ αὐτοῦ.
\par }{\PP \VS{31}Τίνι οὖν ὁμοιώσω τοὺς ἀνθρώπους τῆς γενεᾶς ταύτης καὶ τίνι εἰσὶν ὅμοιοι;
\VS{32}ὅμοιοί εἰσιν παιδίοις τοῖς ἐν ἀγορᾷ καθημένοις καὶ προσφωνοῦσιν ἀλλήλοις ἃ λέγει· 
\begin{poetryblock}
\par }{\PP \begin{quote}Ηὐλήσαμεν ὑμῖν καὶ οὐκ ὠρχήσασθε,\end{quote} 
\par }{\PP \begin{quote}Ἐθρηνήσαμεν καὶ οὐκ ἐκλαύσατε.\end{quote}
\end{poetryblock}
\par }{\PP \VS{33}Ἐλήλυθεν γὰρ Ἰωάννης ὁ Βαπτιστὴς μὴ ἐσθίων ἄρτον μήτε πίνων οἶνον, καὶ λέγετε· Δαιμόνιον ἔχει.
\VS{34}ἐλήλυθεν ὁ Υἱὸς τοῦ ἀνθρώπου ἐσθίων καὶ πίνων, καὶ λέγετε· Ἰδοὺ ἄνθρωπος φάγος καὶ οἰνοπότης, φίλος τελωνῶν καὶ ἁμαρτωλῶν.
\VS{35}καὶ ἐδικαιώθη ἡ σοφία ἀπὸ πάντων τῶν τέκνων αὐτῆς.
\par }{\PP \VS{36}Ἠρώτα δέ τις αὐτὸν τῶν Φαρισαίων ἵνα φάγῃ μετ᾽ αὐτοῦ, καὶ εἰσελθὼν εἰς τὸν οἶκον τοῦ Φαρισαίου κατεκλίθη.
\VS{37}καὶ ἰδοὺ γυνὴ ἥτις ἦν ἐν τῇ πόλει ἁμαρτωλός, καὶ ἐπιγνοῦσα ὅτι κατάκειται ἐν τῇ οἰκίᾳ τοῦ Φαρισαίου, κομίσασα ἀλάβαστρον μύρου
\VS{38}καὶ στᾶσα ὀπίσω παρὰ τοὺς πόδας αὐτοῦ κλαίουσα τοῖς δάκρυσιν ἤρξατο βρέχειν τοὺς πόδας αὐτοῦ καὶ ταῖς θριξὶν τῆς κεφαλῆς αὐτῆς ἐξέμασσεν καὶ κατεφίλει τοὺς πόδας αὐτοῦ καὶ ἤλειφεν τῷ μύρῳ.
\VS{39}Ἰδὼν δὲ ὁ Φαρισαῖος ὁ καλέσας αὐτὸν εἶπεν ἐν ἑαυτῷ λέγων· Οὗτος εἰ ἦν προφήτης, ἐγίνωσκεν ἂν τίς καὶ ποταπὴ ἡ γυνὴ ἥτις ἅπτεται αὐτοῦ, ὅτι ἁμαρτωλός ἐστιν.
\VS{40}Καὶ ἀποκριθεὶς ὁ Ἰησοῦς εἶπεν πρὸς αὐτόν· Σίμων, ἔχω σοί τι εἰπεῖν. Ὁ δέ· Διδάσκαλε, εἰπέ, φησίν.
\VS{41}Δύο χρεοφειλέται ἦσαν δανιστῇ τινι· ὁ εἷς ὤφειλεν δηνάρια πεντακόσια, ὁ δὲ ἕτερος πεντήκοντα.
\VS{42}μὴ ἐχόντων αὐτῶν ἀποδοῦναι ἀμφοτέροις ἐχαρίσατο. τίς οὖν αὐτῶν πλεῖον ἀγαπήσει αὐτόν;
\VS{43}Ἀποκριθεὶς Σίμων εἶπεν· Ὑπολαμβάνω ὅτι ᾧ τὸ πλεῖον ἐχαρίσατο. Ὁ δὲ εἶπεν αὐτῷ· Ὀρθῶς ἔκρινας.
\VS{44}Καὶ στραφεὶς πρὸς τὴν γυναῖκα τῷ Σίμωνι ἔφη· Βλέπεις ταύτην τὴν γυναῖκα; εἰσῆλθόν σου εἰς τὴν οἰκίαν, ὕδωρ μοι ἐπὶ πόδας οὐκ ἔδωκας· αὕτη δὲ τοῖς δάκρυσιν ἔβρεξέν μου τοὺς πόδας καὶ ταῖς θριξὶν αὐτῆς ἐξέμαξεν.
\VS{45}φίλημά μοι οὐκ ἔδωκας· αὕτη δὲ ἀφ᾽ ἧς εἰσῆλθον οὐ διέλιπεν καταφιλοῦσά μου τοὺς πόδας.
\VS{46}ἐλαίῳ τὴν κεφαλήν μου οὐκ ἤλειψας· αὕτη δὲ μύρῳ ἤλειψεν τοὺς πόδας μου.
\VS{47}οὗ χάριν λέγω σοι, ἀφέωνται αἱ ἁμαρτίαι αὐτῆς αἱ πολλαί, ὅτι ἠγάπησεν πολύ· ᾧ δὲ ὀλίγον ἀφίεται, ὀλίγον ἀγαπᾷ.
\VS{48}Εἶπεν δὲ αὐτῇ· Ἀφέωνταί σου αἱ ἁμαρτίαι.
\VS{49}Καὶ ἤρξαντο οἱ συνανακείμενοι λέγειν ἐν ἑαυτοῖς· Τίς οὗτός ἐστιν ὃς καὶ ἁμαρτίας ἀφίησιν;
\VS{50}Εἶπεν δὲ πρὸς τὴν γυναῖκα· Ἡ πίστις σου σέσωκέν σε· πορεύου εἰς εἰρήνην.

\par }\Chap{8}{\PP \VerseOne{1}Καὶ ἐγένετο ἐν τῷ καθεξῆς καὶ αὐτὸς διώδευεν κατὰ πόλιν καὶ κώμην κηρύσσων καὶ εὐαγγελιζόμενος τὴν βασιλείαν τοῦ Θεοῦ καὶ οἱ δώδεκα σὺν αὐτῷ,
\VS{2}καὶ γυναῖκές τινες αἳ ἦσαν τεθεραπευμέναι ἀπὸ πνευμάτων πονηρῶν καὶ ἀσθενειῶν, Μαρία ἡ καλουμένη Μαγδαληνή, ἀφ᾽ ἧς δαιμόνια ἑπτὰ ἐξεληλύθει,
\VS{3}καὶ Ἰωάννα γυνὴ Χουζᾶ ἐπιτρόπου Ἡρῴδου καὶ Σουσάννα καὶ ἕτεραι πολλαί, αἵτινες διηκόνουν αὐτοῖς ἐκ τῶν ὑπαρχόντων αὐταῖς.
\par }{\PP \VS{4}Συνιόντος δὲ ὄχλου πολλοῦ καὶ τῶν κατὰ πόλιν ἐπιπορευομένων πρὸς αὐτὸν εἶπεν διὰ παραβολῆς·
\VS{5}Ἐξῆλθεν ὁ σπείρων τοῦ σπεῖραι τὸν σπόρον αὐτοῦ. καὶ ἐν τῷ σπείρειν αὐτὸν ὃ μὲν ἔπεσεν παρὰ τὴν ὁδόν καὶ κατεπατήθη, καὶ τὰ πετεινὰ τοῦ οὐρανοῦ κατέφαγεν αὐτό.
\VS{6}Καὶ ἕτερον κατέπεσεν ἐπὶ τὴν πέτραν, καὶ φυὲν ἐξηράνθη διὰ τὸ μὴ ἔχειν ἰκμάδα.
\VS{7}Καὶ ἕτερον ἔπεσεν ἐν μέσῳ τῶν ἀκανθῶν, καὶ συμφυεῖσαι αἱ ἄκανθαι ἀπέπνιξαν αὐτό.
\VS{8}Καὶ ἕτερον ἔπεσεν εἰς τὴν γῆν τὴν ἀγαθήν καὶ φυὲν ἐποίησεν καρπὸν ἑκατονταπλασίονα. Ταῦτα λέγων ἐφώνει· Ὁ ἔχων ὦτα ἀκούειν ἀκουέτω.
\par }{\PP \VS{9}Ἐπηρώτων δὲ αὐτὸν οἱ μαθηταὶ αὐτοῦ Τίς αὕτη εἴη ἡ παραβολή.
\VS{10}Ὁ δὲ εἶπεν· Ὑμῖν δέδοται γνῶναι τὰ μυστήρια τῆς βασιλείας τοῦ Θεοῦ, τοῖς δὲ λοιποῖς ἐν παραβολαῖς, ἵνα 
\begin{poetryblock}
\par }{\PP \begin{quote}Βλέποντες μὴ βλέπωσιν\end{quote} 
\par }{\PP \begin{quote}Καὶ ἀκούοντες μὴ συνιῶσιν.\end{quote}
\end{poetryblock}
\par }{\PP \VS{11}Ἔστιν δὲ αὕτη ἡ παραβολή· ὁ σπόρος ἐστὶν ὁ λόγος τοῦ Θεοῦ.
\VS{12}οἱ δὲ παρὰ τὴν ὁδόν εἰσιν οἱ ἀκούσαντες, εἶτα ἔρχεται ὁ διάβολος καὶ αἴρει τὸν λόγον ἀπὸ τῆς καρδίας αὐτῶν, ἵνα μὴ πιστεύσαντες σωθῶσιν.
\VS{13}Οἱ δὲ ἐπὶ τῆς πέτρας οἳ ὅταν ἀκούσωσιν μετὰ χαρᾶς δέχονται τὸν λόγον, καὶ οὗτοι ῥίζαν οὐκ ἔχουσιν, οἳ πρὸς καιρὸν πιστεύουσιν καὶ ἐν καιρῷ πειρασμοῦ ἀφίστανται.
\VS{14}Τὸ δὲ εἰς τὰς ἀκάνθας πεσόν, οὗτοί εἰσιν οἱ ἀκούσαντες, καὶ ὑπὸ μεριμνῶν καὶ πλούτου καὶ ἡδονῶν τοῦ βίου πορευόμενοι συμπνίγονται καὶ οὐ τελεσφοροῦσιν.
\VS{15}Τὸ δὲ ἐν τῇ καλῇ γῇ, οὗτοί εἰσιν οἵτινες ἐν καρδίᾳ καλῇ καὶ ἀγαθῇ ἀκούσαντες τὸν λόγον κατέχουσιν καὶ καρποφοροῦσιν ἐν ὑπομονῇ.
\par }{\PP \VS{16}Οὐδεὶς δὲ λύχνον ἅψας καλύπτει αὐτὸν σκεύει ἢ ὑποκάτω κλίνης τίθησιν, ἀλλ᾽ ἐπὶ λυχνίας τίθησιν, ἵνα οἱ εἰσπορευόμενοι βλέπωσιν τὸ φῶς.
\VS{17}οὐ γάρ ἐστιν κρυπτὸν ὃ οὐ φανερὸν γενήσεται οὐδὲ ἀπόκρυφον ὃ οὐ μὴ γνωσθῇ καὶ εἰς φανερὸν ἔλθῃ.
\par }{\PP \VS{18}Βλέπετε οὖν πῶς ἀκούετε· ὃς ἂν γὰρ ἔχῃ, δοθήσεται αὐτῷ· καὶ ὃς ἂν μὴ ἔχῃ, καὶ ὃ δοκεῖ ἔχειν ἀρθήσεται ἀπ᾽ αὐτοῦ.
\par }{\PP \VS{19}Παρεγένετο δὲ πρὸς αὐτὸν ἡ μήτηρ καὶ οἱ ἀδελφοὶ αὐτοῦ καὶ οὐκ ἠδύναντο συντυχεῖν αὐτῷ διὰ τὸν ὄχλον.
\VS{20}ἀπηγγέλη δὲ αὐτῷ· Ἡ μήτηρ σου καὶ οἱ ἀδελφοί σου ἑστήκασιν ἔξω ἰδεῖν θέλοντές σε.
\VS{21}Ὁ δὲ ἀποκριθεὶς εἶπεν πρὸς αὐτούς· Μήτηρ μου καὶ ἀδελφοί μου οὗτοί εἰσιν οἱ τὸν λόγον τοῦ Θεοῦ ἀκούοντες καὶ ποιοῦντες.
\par }{\PP \VS{22}Ἐγένετο δὲ ἐν μιᾷ τῶν ἡμερῶν καὶ αὐτὸς ἐνέβη εἰς πλοῖον καὶ οἱ μαθηταὶ αὐτοῦ καὶ εἶπεν πρὸς αὐτούς· Διέλθωμεν εἰς τὸ πέραν τῆς λίμνης, καὶ ἀνήχθησαν.
\VS{23}πλεόντων δὲ αὐτῶν ἀφύπνωσεν. καὶ κατέβη λαῖλαψ ἀνέμου εἰς τὴν λίμνην καὶ συνεπληροῦντο καὶ ἐκινδύνευον.
\VS{24}Προσελθόντες δὲ διήγειραν αὐτὸν λέγοντες· Ἐπιστάτα ἐπιστάτα, ἀπολλύμεθα. Ὁ δὲ διεγερθεὶς ἐπετίμησεν τῷ ἀνέμῳ καὶ τῷ κλύδωνι τοῦ ὕδατος· καὶ ἐπαύσαντο καὶ ἐγένετο γαλήνη.
\VS{25}Εἶπεν δὲ αὐτοῖς· Ποῦ ἡ πίστις ὑμῶν; Φοβηθέντες δὲ ἐθαύμασαν λέγοντες πρὸς ἀλλήλους· Τίς ἄρα οὗτός ἐστιν ὅτι καὶ τοῖς ἀνέμοις ἐπιτάσσει καὶ τῷ ὕδατι, καὶ ὑπακούουσιν αὐτῷ;
\par }{\PP \VS{26}Καὶ κατέπλευσαν εἰς τὴν χώραν τῶν Γερασηνῶν, ἥτις ἐστὶν ἀντιπέρα τῆς Γαλιλαίας.
\VS{27}ἐξελθόντι δὲ αὐτῷ ἐπὶ τὴν γῆν ὑπήντησεν ἀνήρ τις ἐκ τῆς πόλεως ἔχων δαιμόνια καὶ χρόνῳ ἱκανῷ οὐκ ἐνεδύσατο ἱμάτιον καὶ ἐν οἰκίᾳ οὐκ ἔμενεν ἀλλ᾽ ἐν τοῖς μνήμασιν.
\VS{28}Ἰδὼν δὲ τὸν Ἰησοῦν ἀνακράξας προσέπεσεν αὐτῷ καὶ φωνῇ μεγάλῃ εἶπεν· Τί ἐμοὶ καὶ σοί, Ἰησοῦ Υἱὲ τοῦ Θεοῦ τοῦ Ὑψίστου; δέομαί σου, μή με βασανίσῃς.
\VS{29}παρήγγειλεν γὰρ τῷ πνεύματι τῷ ἀκαθάρτῳ ἐξελθεῖν ἀπὸ τοῦ ἀνθρώπου. πολλοῖς γὰρ χρόνοις συνηρπάκει αὐτόν καὶ ἐδεσμεύετο ἁλύσεσιν καὶ πέδαις φυλασσόμενος καὶ διαρρήσσων τὰ δεσμὰ ἠλαύνετο ὑπὸ τοῦ δαιμονίου εἰς τὰς ἐρήμους.
\VS{30}Ἐπηρώτησεν δὲ αὐτὸν ὁ Ἰησοῦς· Τί σοι ὄνομά ἐστιν; ὁ Δὲ εἶπεν· Λεγιών, ὅτι εἰσῆλθεν δαιμόνια πολλὰ εἰς αὐτόν.
\VS{31}καὶ παρεκάλουν αὐτὸν ἵνα μὴ ἐπιτάξῃ αὐτοῖς εἰς τὴν ἄβυσσον ἀπελθεῖν.
\VS{32}Ἦν δὲ ἐκεῖ ἀγέλη χοίρων ἱκανῶν βοσκομένη ἐν τῷ ὄρει· καὶ παρεκάλεσαν αὐτὸν ἵνα ἐπιτρέψῃ αὐτοῖς εἰς ἐκείνους εἰσελθεῖν· καὶ ἐπέτρεψεν αὐτοῖς.
\VS{33}Ἐξελθόντα δὲ τὰ δαιμόνια ἀπὸ τοῦ ἀνθρώπου εἰσῆλθον εἰς τοὺς χοίρους, καὶ ὥρμησεν ἡ ἀγέλη κατὰ τοῦ κρημνοῦ εἰς τὴν λίμνην καὶ ἀπεπνίγη.
\par }{\PP \VS{34}Ἰδόντες δὲ οἱ βόσκοντες τὸ γεγονὸς ἔφυγον καὶ ἀπήγγειλαν εἰς τὴν πόλιν καὶ εἰς τοὺς ἀγρούς.
\VS{35}ἐξῆλθον δὲ ἰδεῖν τὸ γεγονὸς καὶ ἦλθον πρὸς τὸν Ἰησοῦν καὶ εὗρον καθήμενον τὸν ἄνθρωπον ἀφ᾽ οὗ τὰ δαιμόνια ἐξῆλθεν ἱματισμένον καὶ σωφρονοῦντα παρὰ τοὺς πόδας τοῦ Ἰησοῦ, καὶ ἐφοβήθησαν.
\VS{36}ἀπήγγειλαν δὲ αὐτοῖς οἱ ἰδόντες πῶς ἐσώθη ὁ δαιμονισθείς.
\VS{37}Καὶ ἠρώτησεν αὐτὸν ἅπαν τὸ πλῆθος τῆς περιχώρου τῶν Γερασηνῶν ἀπελθεῖν ἀπ᾽ αὐτῶν, ὅτι φόβῳ μεγάλῳ συνείχοντο· αὐτὸς δὲ ἐμβὰς εἰς πλοῖον ὑπέστρεψεν.
\VS{38}Ἐδεῖτο δὲ αὐτοῦ ὁ ἀνὴρ ἀφ᾽ οὗ ἐξεληλύθει τὰ δαιμόνια εἶναι σὺν αὐτῷ· ἀπέλυσεν δὲ αὐτὸν λέγων·
\VS{39}Ὑπόστρεφε εἰς τὸν οἶκόν σου καὶ διηγοῦ ὅσα σοι ἐποίησεν ὁ Θεός. καὶ ἀπῆλθεν καθ᾽ ὅλην τὴν πόλιν κηρύσσων ὅσα ἐποίησεν αὐτῷ ὁ Ἰησοῦς.
\par }{\PP \VS{40}Ἐν δὲ τῷ ὑποστρέφειν τὸν Ἰησοῦν ἀπεδέξατο αὐτὸν ὁ ὄχλος· ἦσαν γὰρ πάντες προσδοκῶντες αὐτόν.
\VS{41}καὶ ἰδοὺ ἦλθεν ἀνὴρ ᾧ ὄνομα Ἰάϊρος καὶ οὗτος ἄρχων τῆς συναγωγῆς ὑπῆρχεν, καὶ πεσὼν παρὰ τοὺς πόδας τοῦ Ἰησοῦ παρεκάλει αὐτὸν εἰσελθεῖν εἰς τὸν οἶκον αὐτοῦ,
\VS{42}ὅτι θυγάτηρ μονογενὴς ἦν αὐτῷ ὡς ἐτῶν δώδεκα καὶ αὐτὴ ἀπέθνῃσκεν. Ἐν δὲ τῷ ὑπάγειν αὐτὸν οἱ ὄχλοι συνέπνιγον αὐτόν.
\VS{43}καὶ γυνὴ οὖσα ἐν ῥύσει αἵματος ἀπὸ ἐτῶν δώδεκα, ἥτις ἰατροῖς προσαναλώσασα ὅλον τὸν βίον οὐκ ἴσχυσεν ἀπ᾽ οὐδενὸς θεραπευθῆναι,
\VS{44}προσελθοῦσα ὄπισθεν ἥψατο τοῦ κρασπέδου τοῦ ἱματίου αὐτοῦ καὶ παραχρῆμα ἔστη ἡ ῥύσις τοῦ αἵματος αὐτῆς.
\VS{45}Καὶ εἶπεν ὁ Ἰησοῦς· Τίς ὁ ἁψάμενός μου; Ἀρνουμένων δὲ πάντων εἶπεν ὁ Πέτρος· Ἐπιστάτα, οἱ ὄχλοι συνέχουσίν σε καὶ ἀποθλίβουσιν.
\VS{46}Ὁ δὲ Ἰησοῦς εἶπεν· Ἥψατό μού τις, ἐγὼ γὰρ ἔγνων δύναμιν ἐξεληλυθυῖαν ἀπ᾽ ἐμοῦ.
\VS{47}Ἰδοῦσα δὲ ἡ γυνὴ ὅτι οὐκ ἔλαθεν, τρέμουσα ἦλθεν καὶ προσπεσοῦσα αὐτῷ δι᾽ ἣν αἰτίαν ἥψατο αὐτοῦ ἀπήγγειλεν ἐνώπιον παντὸς τοῦ λαοῦ καὶ ὡς ἰάθη παραχρῆμα.
\VS{48}Ὁ δὲ εἶπεν αὐτῇ· Θυγάτηρ, ἡ πίστις σου σέσωκέν σε· πορεύου εἰς εἰρήνην.
\VS{49}Ἔτι αὐτοῦ λαλοῦντος ἔρχεταί τις παρὰ τοῦ ἀρχισυναγώγου λέγων ὅτι Τέθνηκεν ἡ θυγάτηρ σου· μηκέτι σκύλλε τὸν Διδάσκαλον.
\VS{50}Ὁ δὲ Ἰησοῦς ἀκούσας ἀπεκρίθη αὐτῷ· Μὴ φοβοῦ, μόνον πίστευσον, καὶ σωθήσεται.
\VS{51}Ἐλθὼν δὲ εἰς τὴν οἰκίαν οὐκ ἀφῆκεν εἰσελθεῖν τινα σὺν αὐτῷ εἰ μὴ Πέτρον καὶ Ἰωάννην καὶ Ἰάκωβον καὶ τὸν πατέρα τῆς παιδὸς καὶ τὴν μητέρα.
\VS{52}ἔκλαιον δὲ πάντες καὶ ἐκόπτοντο αὐτήν. ὁ δὲ εἶπεν· Μὴ κλαίετε, οὐ γὰρ ἀπέθανεν ἀλλὰ καθεύδει.
\VS{53}Καὶ κατεγέλων αὐτοῦ εἰδότες ὅτι ἀπέθανεν.
\VS{54}Αὐτὸς δὲ κρατήσας τῆς χειρὸς αὐτῆς ἐφώνησεν λέγων· Ἡ Παῖς, ἔγειρε.
\VS{55}καὶ ἐπέστρεψεν τὸ πνεῦμα αὐτῆς καὶ ἀνέστη παραχρῆμα καὶ διέταξεν αὐτῇ δοθῆναι φαγεῖν.
\VS{56}καὶ ἐξέστησαν οἱ γονεῖς αὐτῆς· ὁ δὲ παρήγγειλεν αὐτοῖς μηδενὶ εἰπεῖν τὸ γεγονός.

\par }\Chap{9}{\PP \VerseOne{1}Συνκαλεσάμενος δὲ τοὺς δώδεκα ἔδωκεν αὐτοῖς δύναμιν καὶ ἐξουσίαν ἐπὶ πάντα τὰ δαιμόνια καὶ νόσους θεραπεύειν
\VS{2}καὶ ἀπέστειλεν αὐτοὺς κηρύσσειν τὴν βασιλείαν τοῦ Θεοῦ καὶ ἰᾶσθαι τοὺς ἀσθενεῖς,
\VS{3}καὶ εἶπεν πρὸς αὐτούς· Μηδὲν αἴρετε εἰς τὴν ὁδόν, μήτε ῥάβδον μήτε πήραν μήτε ἄρτον μήτε ἀργύριον μήτε ἀνὰ δύο χιτῶνας ἔχειν.
\VS{4}καὶ εἰς ἣν ἂν οἰκίαν εἰσέλθητε, ἐκεῖ μένετε καὶ ἐκεῖθεν ἐξέρχεσθε.
\VS{5}καὶ ὅσοι ἂν μὴ δέχωνται ὑμᾶς, ἐξερχόμενοι ἀπὸ τῆς πόλεως ἐκείνης τὸν κονιορτὸν ἀπὸ τῶν ποδῶν ὑμῶν ἀποτινάσσετε εἰς μαρτύριον ἐπ᾽ αὐτούς.
\VS{6}Ἐξερχόμενοι δὲ διήρχοντο κατὰ τὰς κώμας εὐαγγελιζόμενοι καὶ θεραπεύοντες πανταχοῦ.
\par }{\PP \VS{7}Ἤκουσεν δὲ Ἡρῴδης ὁ τετραάρχης τὰ γινόμενα πάντα καὶ διηπόρει διὰ τὸ λέγεσθαι ὑπό τινων ὅτι Ἰωάννης ἠγέρθη ἐκ νεκρῶν,
\VS{8}ὑπό τινων δὲ ὅτι Ἠλίας ἐφάνη, ἄλλων δὲ ὅτι προφήτης τις τῶν ἀρχαίων ἀνέστη.
\VS{9}Εἶπεν δὲ Ἡρῴδης· Ἰωάννην ἐγὼ ἀπεκεφάλισα· τίς δέ ἐστιν οὗτος περὶ οὗ ἀκούω τοιαῦτα; καὶ ἐζήτει ἰδεῖν αὐτόν.
\par }{\PP \VS{10}Καὶ ὑποστρέψαντες οἱ ἀπόστολοι διηγήσαντο αὐτῷ ὅσα ἐποίησαν. Καὶ παραλαβὼν αὐτοὺς ὑπεχώρησεν κατ᾽ ἰδίαν εἰς πόλιν καλουμένην Βηθσαϊδά.
\VS{11}οἱ δὲ ὄχλοι γνόντες ἠκολούθησαν αὐτῷ· καὶ ἀποδεξάμενος αὐτοὺς ἐλάλει αὐτοῖς περὶ τῆς βασιλείας τοῦ Θεοῦ, καὶ τοὺς χρείαν ἔχοντας θεραπείας ἰᾶτο.
\par }{\PP \VS{12}Ἡ δὲ ἡμέρα ἤρξατο κλίνειν· προσελθόντες δὲ οἱ δώδεκα εἶπαν αὐτῷ· Ἀπόλυσον τὸν ὄχλον, ἵνα πορευθέντες εἰς τὰς κύκλῳ κώμας καὶ ἀγροὺς καταλύσωσιν καὶ εὕρωσιν ἐπισιτισμόν, ὅτι ὧδε ἐν ἐρήμῳ τόπῳ ἐσμέν.
\VS{13}Εἶπεν δὲ πρὸς αὐτούς· Δότε αὐτοῖς ὑμεῖς φαγεῖν. Οἱ δὲ εἶπαν· Οὐκ εἰσὶν ἡμῖν πλεῖον ἢ ἄρτοι πέντε καὶ ἰχθύες δύο, εἰ μήτι πορευθέντες ἡμεῖς ἀγοράσωμεν εἰς πάντα τὸν λαὸν τοῦτον βρώματα.
\VS{14}ἦσαν γὰρ ὡσεὶ ἄνδρες πεντακισχίλιοι. Εἶπεν δὲ πρὸς τοὺς μαθητὰς αὐτοῦ· Κατακλίνατε αὐτοὺς κλισίας ὡσεὶ ἀνὰ πεντήκοντα.
\VS{15}καὶ ἐποίησαν οὕτως καὶ κατέκλιναν ἅπαντας.
\VS{16}Λαβὼν δὲ τοὺς πέντε ἄρτους καὶ τοὺς δύο ἰχθύας ἀναβλέψας εἰς τὸν οὐρανὸν εὐλόγησεν αὐτοὺς καὶ κατέκλασεν καὶ ἐδίδου τοῖς μαθηταῖς παραθεῖναι τῷ ὄχλῳ.
\VS{17}Καὶ ἔφαγον καὶ ἐχορτάσθησαν πάντες, καὶ ἤρθη τὸ περισσεῦσαν αὐτοῖς κλασμάτων κόφινοι δώδεκα.
\par }{\PP \VS{18}Καὶ ἐγένετο ἐν τῷ εἶναι αὐτὸν προσευχόμενον κατὰ μόνας συνῆσαν αὐτῷ οἱ μαθηταί, καὶ ἐπηρώτησεν αὐτοὺς λέγων· Τίνα με λέγουσιν οἱ ὄχλοι εἶναι;
\VS{19}Οἱ δὲ ἀποκριθέντες εἶπαν· Ἰωάννην τὸν Βαπτιστήν, ἄλλοι δὲ Ἠλίαν, ἄλλοι δὲ ὅτι προφήτης τις τῶν ἀρχαίων ἀνέστη.
\VS{20}Εἶπεν δὲ αὐτοῖς· Ὑμεῖς δὲ τίνα με λέγετε εἶναι; Πέτρος δὲ ἀποκριθεὶς εἶπεν· Τὸν Χριστὸν τοῦ Θεοῦ.
\VS{21}Ὁ δὲ ἐπιτιμήσας αὐτοῖς παρήγγειλεν μηδενὶ λέγειν τοῦτο
\VS{22}εἰπὼν ὅτι Δεῖ τὸν Υἱὸν τοῦ ἀνθρώπου πολλὰ παθεῖν καὶ ἀποδοκιμασθῆναι ἀπὸ τῶν πρεσβυτέρων καὶ ἀρχιερέων καὶ γραμματέων καὶ ἀποκτανθῆναι καὶ τῇ τρίτῃ ἡμέρᾳ ἐγερθῆναι.
\par }{\PP \VS{23}Ἔλεγεν δὲ πρὸς πάντας· Εἴ τις θέλει ὀπίσω μου ἔρχεσθαι, ἀρνησάσθω ἑαυτὸν καὶ ἀράτω τὸν σταυρὸν αὐτοῦ καθ᾽ ἡμέραν καὶ ἀκολουθείτω μοι.
\VS{24}ὃς γὰρ ἂν θέλῃ τὴν ψυχὴν αὐτοῦ σῶσαι ἀπολέσει αὐτήν· ὃς δ᾽ ἂν ἀπολέσῃ τὴν ψυχὴν αὐτοῦ ἕνεκεν ἐμοῦ οὗτος σώσει αὐτήν.
\VS{25}Τί γὰρ ὠφελεῖται ἄνθρωπος κερδήσας τὸν κόσμον ὅλον ἑαυτὸν δὲ ἀπολέσας ἢ ζημιωθείς;
\VS{26}ὃς γὰρ ἂν ἐπαισχυνθῇ με καὶ τοὺς ἐμοὺς λόγους, τοῦτον ὁ Υἱὸς τοῦ ἀνθρώπου ἐπαισχυνθήσεται, ὅταν ἔλθῃ ἐν τῇ δόξῃ αὐτοῦ καὶ τοῦ Πατρὸς καὶ τῶν ἁγίων ἀγγέλων.
\VS{27}λέγω δὲ ὑμῖν ἀληθῶς, εἰσίν τινες τῶν αὐτοῦ ἑστηκότων οἳ οὐ μὴ γεύσωνται θανάτου ἕως ἂν ἴδωσιν τὴν βασιλείαν τοῦ Θεοῦ.
\par }{\PP \VS{28}Ἐγένετο δὲ μετὰ τοὺς λόγους τούτους ὡσεὶ ἡμέραι ὀκτὼ καὶ παραλαβὼν Πέτρον καὶ Ἰωάννην καὶ Ἰάκωβον ἀνέβη εἰς τὸ ὄρος προσεύξασθαι.
\VS{29}καὶ ἐγένετο ἐν τῷ προσεύχεσθαι αὐτὸν τὸ εἶδος τοῦ προσώπου αὐτοῦ ἕτερον καὶ ὁ ἱματισμὸς αὐτοῦ λευκὸς ἐξαστράπτων.
\VS{30}καὶ ἰδοὺ ἄνδρες δύο συνελάλουν αὐτῷ, οἵτινες ἦσαν Μωϋσῆς καὶ Ἠλίας,
\VS{31}οἳ ὀφθέντες ἐν δόξῃ ἔλεγον τὴν ἔξοδον αὐτοῦ, ἣν ἤμελλεν πληροῦν ἐν Ἰερουσαλήμ.
\VS{32}Ὁ δὲ Πέτρος καὶ οἱ σὺν αὐτῷ ἦσαν βεβαρημένοι ὕπνῳ· διαγρηγορήσαντες δὲ εἶδον τὴν δόξαν αὐτοῦ καὶ τοὺς δύο ἄνδρας τοὺς συνεστῶτας αὐτῷ.
\VS{33}καὶ ἐγένετο ἐν τῷ διαχωρίζεσθαι αὐτοὺς ἀπ᾽ αὐτοῦ εἶπεν ὁ Πέτρος πρὸς τὸν Ἰησοῦν· Ἐπιστάτα, καλόν ἐστιν ἡμᾶς ὧδε εἶναι, καὶ ποιήσωμεν σκηνὰς τρεῖς, μίαν σοὶ καὶ μίαν Μωϋσεῖ καὶ μίαν Ἠλίᾳ, μὴ εἰδὼς ὃ λέγει.
\VS{34}Ταῦτα δὲ αὐτοῦ λέγοντος ἐγένετο νεφέλη καὶ ἐπεσκίαζεν αὐτούς· ἐφοβήθησαν δὲ ἐν τῷ εἰσελθεῖν αὐτοὺς εἰς τὴν νεφέλην.
\VS{35}καὶ φωνὴ ἐγένετο ἐκ τῆς νεφέλης λέγουσα· Οὗτός ἐστιν ὁ Υἱός μου ὁ ἐκλελεγμένος, αὐτοῦ ἀκούετε.
\VS{36}καὶ ἐν τῷ γενέσθαι τὴν φωνὴν εὑρέθη Ἰησοῦς μόνος. καὶ αὐτοὶ ἐσίγησαν καὶ οὐδενὶ ἀπήγγειλαν ἐν ἐκείναις ταῖς ἡμέραις οὐδὲν ὧν ἑώρακαν.
\par }{\PP \VS{37}Ἐγένετο δὲ τῇ ἑξῆς ἡμέρᾳ κατελθόντων αὐτῶν ἀπὸ τοῦ ὄρους συνήντησεν αὐτῷ ὄχλος πολύς.
\VS{38}καὶ ἰδοὺ ἀνὴρ ἀπὸ τοῦ ὄχλου ἐβόησεν λέγων· Διδάσκαλε, δέομαί σου ἐπιβλέψαι ἐπὶ τὸν υἱόν μου, ὅτι μονογενής μοί ἐστιν,
\VS{39}καὶ ἰδοὺ πνεῦμα λαμβάνει αὐτόν καὶ ἐξαίφνης κράζει καὶ σπαράσσει αὐτὸν μετὰ ἀφροῦ καὶ μόγις ἀποχωρεῖ ἀπ᾽ αὐτοῦ συντρῖβον αὐτόν·
\VS{40}καὶ ἐδεήθην τῶν μαθητῶν σου ἵνα ἐκβάλωσιν αὐτό, καὶ οὐκ ἠδυνήθησαν.
\VS{41}Ἀποκριθεὶς δὲ ὁ Ἰησοῦς εἶπεν· Ὦ γενεὰ ἄπιστος καὶ διεστραμμένη, ἕως πότε ἔσομαι πρὸς ὑμᾶς καὶ ἀνέξομαι ὑμῶν; προσάγαγε ὧδε τὸν υἱόν σου.
\VS{42}Ἔτι δὲ προσερχομένου αὐτοῦ ἔρρηξεν αὐτὸν τὸ δαιμόνιον καὶ συνεσπάραξεν· ἐπετίμησεν δὲ ὁ Ἰησοῦς τῷ πνεύματι τῷ ἀκαθάρτῳ καὶ ἰάσατο τὸν παῖδα καὶ ἀπέδωκεν αὐτὸν τῷ πατρὶ αὐτοῦ.
\VS{43}Ἐξεπλήσσοντο δὲ πάντες ἐπὶ τῇ μεγαλειότητι τοῦ Θεοῦ.
\par }{\PP Πάντων δὲ θαυμαζόντων ἐπὶ πᾶσιν οἷς ἐποίει εἶπεν πρὸς τοὺς μαθητὰς αὐτοῦ·
\VS{44}Θέσθε ὑμεῖς εἰς τὰ ὦτα ὑμῶν τοὺς λόγους τούτους· ὁ γὰρ Υἱὸς τοῦ ἀνθρώπου μέλλει παραδίδοσθαι εἰς χεῖρας ἀνθρώπων.
\VS{45}οἱ δὲ ἠγνόουν τὸ ῥῆμα τοῦτο καὶ ἦν παρακεκαλυμμένον ἀπ᾽ αὐτῶν ἵνα μὴ αἴσθωνται αὐτό, καὶ ἐφοβοῦντο ἐρωτῆσαι αὐτὸν περὶ τοῦ ῥήματος τούτου.
\par }{\PP \VS{46}Εἰσῆλθεν δὲ διαλογισμὸς ἐν αὐτοῖς, τὸ τίς ἂν εἴη μείζων αὐτῶν.
\VS{47}ὁ δὲ Ἰησοῦς εἰδὼς τὸν διαλογισμὸν τῆς καρδίας αὐτῶν, ἐπιλαβόμενος παιδίον ἔστησεν αὐτὸ παρ᾽ ἑαυτῷ
\VS{48}καὶ εἶπεν αὐτοῖς· Ὃς ἐὰν δέξηται τοῦτο τὸ παιδίον ἐπὶ τῷ ὀνόματί μου, ἐμὲ δέχεται· καὶ ὃς ἂν ἐμὲ δέξηται, δέχεται τὸν ἀποστείλαντά με· ὁ γὰρ μικρότερος ἐν πᾶσιν ὑμῖν ὑπάρχων οὗτός ἐστιν μέγας.
\par }{\PP \VS{49}Ἀποκριθεὶς δὲ Ἰωάννης εἶπεν· Ἐπιστάτα, εἴδομέν τινα ἐν τῷ ὀνόματί σου ἐκβάλλοντα δαιμόνια καὶ ἐκωλύομεν αὐτὸν, ὅτι οὐκ ἀκολουθεῖ μεθ᾽ ἡμῶν.
\VS{50}Εἶπεν δὲ πρὸς αὐτὸν ὁ Ἰησοῦς· Μὴ κωλύετε· ὃς γὰρ οὐκ ἔστιν καθ᾽ ὑμῶν, ὑπὲρ ὑμῶν ἐστιν.
\par }{\PP \VS{51}Ἐγένετο δὲ ἐν τῷ συμπληροῦσθαι τὰς ἡμέρας τῆς ἀναλήμψεως αὐτοῦ καὶ αὐτὸς τὸ πρόσωπον ἐστήρισεν τοῦ πορεύεσθαι εἰς Ἰερουσαλήμ.
\VS{52}καὶ ἀπέστειλεν ἀγγέλους πρὸ προσώπου αὐτοῦ. καὶ πορευθέντες εἰσῆλθον εἰς κώμην Σαμαριτῶν ὡς ἑτοιμάσαι αὐτῷ·
\VS{53}καὶ οὐκ ἐδέξαντο αὐτόν, ὅτι τὸ πρόσωπον αὐτοῦ ἦν πορευόμενον εἰς Ἰερουσαλήμ.
\VS{54}Ἰδόντες δὲ οἱ μαθηταὶ Ἰάκωβος καὶ Ἰωάννης εἶπαν· Κύριε, θέλεις εἴπωμεν πῦρ καταβῆναι ἀπὸ τοῦ οὐρανοῦ καὶ ἀναλῶσαι αὐτούς;
\VS{55}Στραφεὶς δὲ ἐπετίμησεν αὐτοῖς.
\VS{56}καὶ ἐπορεύθησαν εἰς ἑτέραν κώμην.
\par }{\PP \VS{57}Καὶ πορευομένων αὐτῶν ἐν τῇ ὁδῷ εἶπέν τις πρὸς αὐτόν· Ἀκολουθήσω σοι ὅπου ἐὰν ἀπέρχῃ.
\VS{58}Καὶ εἶπεν αὐτῷ ὁ Ἰησοῦς· Αἱ ἀλώπεκες φωλεοὺς ἔχουσιν καὶ τὰ πετεινὰ τοῦ οὐρανοῦ κατασκηνώσεις, ὁ δὲ Υἱὸς τοῦ ἀνθρώπου οὐκ ἔχει ποῦ τὴν κεφαλὴν κλίνῃ.
\VS{59}Εἶπεν δὲ πρὸς ἕτερον· Ἀκολούθει μοι. ὁ Δὲ εἶπεν· Κύριε, Ἐπίτρεψόν μοι ἀπελθόντι πρῶτον θάψαι τὸν πατέρα μου.
\VS{60}Εἶπεν δὲ αὐτῷ· Ἄφες τοὺς νεκροὺς θάψαι τοὺς ἑαυτῶν νεκρούς, σὺ δὲ ἀπελθὼν διάγγελλε τὴν βασιλείαν τοῦ Θεοῦ.
\VS{61}Εἶπεν δὲ καὶ ἕτερος· Ἀκολουθήσω σοι, Κύριε· πρῶτον δὲ ἐπίτρεψόν μοι ἀποτάξασθαι τοῖς εἰς τὸν οἶκόν μου.
\VS{62}Εἶπεν δὲ πρὸς αὐτὸν ὁ Ἰησοῦς· Οὐδεὶς ἐπιβαλὼν τὴν χεῖρα ἐπ᾽ ἄροτρον καὶ βλέπων εἰς τὰ ὀπίσω εὔθετός ἐστιν τῇ βασιλείᾳ τοῦ Θεοῦ.

\par }\Chap{10}{\PP \VerseOne{1}Μετὰ δὲ ταῦτα ἀνέδειξεν ὁ Κύριος ἑτέρους ἑβδομήκοντα δύο καὶ ἀπέστειλεν αὐτοὺς ἀνὰ δύο δύο πρὸ προσώπου αὐτοῦ εἰς πᾶσαν πόλιν καὶ τόπον οὗ ἤμελλεν αὐτὸς ἔρχεσθαι.
\VS{2}ἔλεγεν δὲ πρὸς αὐτούς· Ὁ μὲν θερισμὸς πολύς, οἱ δὲ ἐργάται ὀλίγοι· δεήθητε οὖν τοῦ Κυρίου τοῦ θερισμοῦ ὅπως ἐργάτας ἐκβάλῃ εἰς τὸν θερισμὸν αὐτοῦ.
\VS{3}Ὑπάγετε· ἰδοὺ ἀποστέλλω ὑμᾶς ὡς ἄρνας ἐν μέσῳ λύκων.
\VS{4}μὴ βαστάζετε βαλλάντιον, μὴ πήραν, μὴ ὑποδήματα, καὶ μηδένα κατὰ τὴν ὁδὸν ἀσπάσησθε.
\VS{5}Εἰς ἣν δ᾽ ἂν εἰσέλθητε οἰκίαν, πρῶτον λέγετε· Εἰρήνη τῷ οἴκῳ τούτῳ.
\VS{6}καὶ ἐὰν ἐκεῖ ᾖ υἱὸς εἰρήνης, ἐπαναπαήσεται ἐπ᾽ αὐτὸν ἡ εἰρήνη ὑμῶν· εἰ δὲ μή γε, ἐφ᾽ ὑμᾶς ἀνακάμψει.
\VS{7}ἐν αὐτῇ δὲ τῇ οἰκίᾳ μένετε ἐσθίοντες καὶ πίνοντες τὰ παρ᾽ αὐτῶν· ἄξιος γὰρ ὁ ἐργάτης τοῦ μισθοῦ αὐτοῦ. μὴ μεταβαίνετε ἐξ οἰκίας εἰς οἰκίαν.
\VS{8}Καὶ εἰς ἣν ἂν πόλιν εἰσέρχησθε καὶ δέχωνται ὑμᾶς, ἐσθίετε τὰ παρατιθέμενα ὑμῖν
\VS{9}καὶ θεραπεύετε τοὺς ἐν αὐτῇ ἀσθενεῖς καὶ λέγετε αὐτοῖς· Ἤγγικεν ἐφ᾽ ὑμᾶς ἡ βασιλεία τοῦ Θεοῦ.
\VS{10}Εἰς ἣν δ᾽ ἂν πόλιν εἰσέλθητε καὶ μὴ δέχωνται ὑμᾶς, ἐξελθόντες εἰς τὰς πλατείας αὐτῆς εἴπατε·
\VS{11}Καὶ τὸν κονιορτὸν τὸν κολληθέντα ἡμῖν ἐκ τῆς πόλεως ὑμῶν εἰς τοὺς πόδας ἀπομασσόμεθα ὑμῖν· πλὴν τοῦτο γινώσκετε ὅτι ἤγγικεν ἡ βασιλεία τοῦ Θεοῦ.
\VS{12}λέγω ὑμῖν ὅτι Σοδόμοις ἐν τῇ ἡμέρᾳ ἐκείνῃ ἀνεκτότερον ἔσται ἢ τῇ πόλει ἐκείνῃ.
\par }{\PP \VS{13}Οὐαί σοι, Χοραζίν, οὐαί σοι, Βηθσαϊδά· ὅτι εἰ ἐν Τύρῳ καὶ Σιδῶνι ἐγενήθησαν αἱ δυνάμεις αἱ γενόμεναι ἐν ὑμῖν, πάλαι ἂν ἐν σάκκῳ καὶ σποδῷ καθήμενοι μετενόησαν.
\VS{14}πλὴν Τύρῳ καὶ Σιδῶνι ἀνεκτότερον ἔσται ἐν τῇ κρίσει ἢ ὑμῖν.
\VS{15}καὶ σύ, Καφαρναούμ, μὴ ἕως οὐρανοῦ ὑψωθήσῃ; ἕως τοῦ ᾅδου καταβήσῃ.
\par }{\PP \VS{16}Ὁ ἀκούων ὑμῶν ἐμοῦ ἀκούει, καὶ ὁ ἀθετῶν ὑμᾶς ἐμὲ ἀθετεῖ· ὁ δὲ ἐμὲ ἀθετῶν ἀθετεῖ τὸν ἀποστείλαντά με.
\par }{\PP \VS{17}Ὑπέστρεψαν δὲ οἱ ἑβδομήκοντα δύο μετὰ χαρᾶς λέγοντες· Κύριε, καὶ τὰ δαιμόνια ὑποτάσσεται ἡμῖν ἐν τῷ ὀνόματί σου.
\VS{18}Εἶπεν δὲ αὐτοῖς· Ἐθεώρουν τὸν Σατανᾶν ὡς ἀστραπὴν ἐκ τοῦ οὐρανοῦ πεσόντα.
\VS{19}ἰδοὺ δέδωκα ὑμῖν τὴν ἐξουσίαν τοῦ πατεῖν ἐπάνω ὄφεων καὶ σκορπίων, καὶ ἐπὶ πᾶσαν τὴν δύναμιν τοῦ ἐχθροῦ, καὶ οὐδὲν ὑμᾶς οὐ μὴ ἀδικήσῃ.
\VS{20}πλὴν ἐν τούτῳ μὴ χαίρετε ὅτι τὰ πνεύματα ὑμῖν ὑποτάσσεται, χαίρετε δὲ ὅτι τὰ ὀνόματα ὑμῶν ἐνγέγραπται ἐν τοῖς οὐρανοῖς.
\par }{\PP \VS{21}Ἐν αὐτῇ τῇ ὥρᾳ ἠγαλλιάσατο ἐν τῷ Πνεύματι τῷ Ἁγίῳ καὶ εἶπεν· Ἐξομολογοῦμαί σοι, Πάτερ, Κύριε τοῦ οὐρανοῦ καὶ τῆς γῆς, ὅτι ἀπέκρυψας ταῦτα ἀπὸ σοφῶν καὶ συνετῶν καὶ ἀπεκάλυψας αὐτὰ νηπίοις· ναί ὁ Πατήρ, ὅτι οὕτως εὐδοκία ἐγένετο ἔμπροσθέν σου.
\VS{22}Πάντα μοι παρεδόθη ὑπὸ τοῦ Πατρός μου, καὶ οὐδεὶς γινώσκει τίς ἐστιν ὁ Υἱὸς εἰ μὴ ὁ Πατήρ, καὶ τίς ἐστιν ὁ Πατὴρ εἰ μὴ ὁ Υἱὸς καὶ ᾧ ἐὰν βούληται ὁ Υἱὸς ἀποκαλύψαι.
\par }{\PP \VS{23}Καὶ στραφεὶς πρὸς τοὺς μαθητὰς κατ᾽ ἰδίαν εἶπεν· Μακάριοι οἱ ὀφθαλμοὶ οἱ βλέποντες ἃ βλέπετε.
\VS{24}λέγω γὰρ ὑμῖν ὅτι πολλοὶ προφῆται καὶ βασιλεῖς ἠθέλησαν ἰδεῖν ἃ ὑμεῖς βλέπετε καὶ οὐκ εἶδαν, καὶ ἀκοῦσαι ἃ ἀκούετε καὶ οὐκ ἤκουσαν.
\par }{\PP \VS{25}Καὶ ἰδοὺ νομικός τις ἀνέστη ἐκπειράζων αὐτὸν λέγων· Διδάσκαλε, τί ποιήσας ζωὴν αἰώνιον κληρονομήσω;
\VS{26}Ὁ δὲ εἶπεν πρὸς αὐτόν· Ἐν τῷ νόμῳ τί γέγραπται; πῶς ἀναγινώσκεις;
\VS{27}Ὁ δὲ ἀποκριθεὶς εἶπεν· Ἀγαπήσεις Κύριον τὸν Θεόν σου ἐξ ὅλης τῆς καρδίας σου καὶ ἐν ὅλῃ τῇ ψυχῇ σου καὶ ἐν ὅλῃ τῇ ἰσχύϊ σου καὶ ἐν ὅλῃ τῇ διανοίᾳ σου, καὶ Τὸν πλησίον σου ὡς σεαυτόν.
\VS{28}Εἶπεν δὲ αὐτῷ· Ὀρθῶς ἀπεκρίθης· τοῦτο ποίει καὶ ζήσῃ.
\VS{29}Ὁ δὲ θέλων δικαιῶσαι ἑαυτὸν εἶπεν πρὸς τὸν Ἰησοῦν· Καὶ τίς ἐστίν μου πλησίον;
\par }{\PP \VS{30}Ὑπολαβὼν ὁ Ἰησοῦς εἶπεν· Ἄνθρωπός τις κατέβαινεν ἀπὸ Ἰερουσαλὴμ εἰς Ἰεριχὼ καὶ λῃσταῖς περιέπεσεν, οἳ καὶ ἐκδύσαντες αὐτὸν καὶ πληγὰς ἐπιθέντες ἀπῆλθον ἀφέντες ἡμιθανῆ.
\VS{31}Κατὰ συγκυρίαν δὲ ἱερεύς τις κατέβαινεν ἐν τῇ ὁδῷ ἐκείνῃ καὶ ἰδὼν αὐτὸν ἀντιπαρῆλθεν·
\VS{32}Ὁμοίως δὲ καὶ Λευίτης γενόμενος κατὰ τὸν τόπον ἐλθὼν καὶ ἰδὼν ἀντιπαρῆλθεν.
\VS{33}Σαμαρίτης δέ τις ὁδεύων ἦλθεν κατ᾽ αὐτὸν καὶ ἰδὼν ἐσπλαγχνίσθη,
\VS{34}καὶ προσελθὼν κατέδησεν τὰ τραύματα αὐτοῦ ἐπιχέων ἔλαιον καὶ οἶνον, ἐπιβιβάσας δὲ αὐτὸν ἐπὶ τὸ ἴδιον κτῆνος ἤγαγεν αὐτὸν εἰς πανδοχεῖον καὶ ἐπεμελήθη αὐτοῦ.
\VS{35}Καὶ ἐπὶ τὴν αὔριον ἐκβαλὼν ἔδωκεν δύο δηνάρια τῷ πανδοχεῖ καὶ εἶπεν· Ἐπιμελήθητι αὐτοῦ, καὶ ὅ τι ἂν προσδαπανήσῃς ἐγὼ ἐν τῷ ἐπανέρχεσθαί με ἀποδώσω σοι.
\VS{36}Τίς τούτων τῶν τριῶν πλησίον δοκεῖ σοι γεγονέναι τοῦ ἐμπεσόντος εἰς τοὺς λῃστάς;
\VS{37}Ὁ δὲ εἶπεν· Ὁ ποιήσας τὸ ἔλεος μετ᾽ αὐτοῦ. Εἶπεν δὲ αὐτῷ ὁ Ἰησοῦς· Πορεύου καὶ σὺ ποίει ὁμοίως.
\par }{\PP \VS{38}Ἐν δὲ τῷ πορεύεσθαι αὐτοὺς αὐτὸς εἰσῆλθεν εἰς κώμην τινά· γυνὴ δέ τις ὀνόματι Μάρθα ὑπεδέξατο αὐτὸν.
\VS{39}καὶ τῇδε ἦν ἀδελφὴ καλουμένη Μαριάμ, ἣ καὶ παρακαθεσθεῖσα πρὸς τοὺς πόδας τοῦ Κυρίου ἤκουεν τὸν λόγον αὐτοῦ.
\VS{40}ἡ δὲ Μάρθα περιεσπᾶτο περὶ πολλὴν διακονίαν· ἐπιστᾶσα δὲ εἶπεν· Κύριε, οὐ μέλει σοι ὅτι ἡ ἀδελφή μου μόνην με κατέλιπεν διακονεῖν; εἰπὲ οὖν αὐτῇ ἵνα μοι συναντιλάβηται.
\VS{41}Ἀποκριθεὶς δὲ εἶπεν αὐτῇ ὁ Κύριος· Μάρθα Μάρθα, μεριμνᾷς καὶ θορυβάζῃ περὶ πολλά,
\VS{42}ἑνός δέ ἐστιν χρεία· Μαριὰμ γὰρ τὴν ἀγαθὴν μερίδα ἐξελέξατο ἥτις οὐκ ἀφαιρεθήσεται αὐτῆς.

\par }\Chap{11}{\PP \VerseOne{1}Καὶ ἐγένετο ἐν τῷ εἶναι αὐτὸν ἐν τόπῳ τινὶ προσευχόμενον, ὡς ἐπαύσατο, εἶπέν τις τῶν μαθητῶν αὐτοῦ πρὸς αὐτόν· Κύριε, δίδαξον ἡμᾶς προσεύχεσθαι, καθὼς καὶ Ἰωάννης ἐδίδαξεν τοὺς μαθητὰς αὐτοῦ.
\VS{2}Εἶπεν δὲ αὐτοῖς· Ὅταν προσεύχησθε λέγετε· 
\begin{poetryblock}
\par }{\PP \begin{quote}Πάτερ, ἁγιασθήτω τὸ ὄνομά σου· Ἐλθέτω ἡ βασιλεία σου·\end{quote}
\par }{\PP \begin{quote} \VS{3}Τὸν ἄρτον ἡμῶν τὸν ἐπιούσιον δίδου ἡμῖν τὸ καθ᾽ ἡμέραν·\end{quote}
\par }{\PP \begin{quote} \VS{4}Καὶ ἄφες ἡμῖν τὰς ἁμαρτίας ἡμῶν,\end{quote} 
\par }{\PP \begin{quote}Καὶ γὰρ αὐτοὶ ἀφίομεν παντὶ ὀφείλοντι ἡμῖν·\end{quote} 
\par }{\PP \begin{quote}Καὶ μὴ εἰσενέγκῃς ἡμᾶς εἰς πειρασμόν.\end{quote}
\end{poetryblock}
\VS{5}Καὶ εἶπεν πρὸς αὐτούς· Τίς ἐξ ὑμῶν ἕξει φίλον καὶ πορεύσεται πρὸς αὐτὸν μεσονυκτίου καὶ εἴπῃ αὐτῷ· Φίλε, χρῆσόν μοι τρεῖς ἄρτους,
\VS{6}ἐπειδὴ φίλος μου παρεγένετο ἐξ ὁδοῦ πρός με καὶ οὐκ ἔχω ὃ παραθήσω αὐτῷ·
\VS{7}Κἀκεῖνος ἔσωθεν ἀποκριθεὶς εἴπῃ· Μή μοι κόπους πάρεχε· ἤδη ἡ θύρα κέκλεισται καὶ τὰ παιδία μου μετ᾽ ἐμοῦ εἰς τὴν κοίτην εἰσίν· οὐ δύναμαι ἀναστὰς δοῦναί σοι.
\VS{8}Λέγω ὑμῖν, εἰ καὶ οὐ δώσει αὐτῷ ἀναστὰς διὰ τὸ εἶναι φίλον αὐτοῦ, διά γε τὴν ἀναίδειαν αὐτοῦ ἐγερθεὶς δώσει αὐτῷ ὅσων χρῄζει.
\VS{9}Κἀγὼ ὑμῖν λέγω, αἰτεῖτε καὶ δοθήσεται ὑμῖν, ζητεῖτε καὶ εὑρήσετε, κρούετε καὶ ἀνοιγήσεται ὑμῖν·
\VS{10}πᾶς γὰρ ὁ αἰτῶν λαμβάνει καὶ ὁ ζητῶν εὑρίσκει καὶ τῷ κρούοντι ἀνοιγήσεται.
\VS{11}Τίνα δὲ ἐξ ὑμῶν τὸν πατέρα αἰτήσει ὁ υἱὸς ἰχθύν, καὶ ἀντὶ ἰχθύος ὄφιν αὐτῷ ἐπιδώσει;
\VS{12}ἢ καὶ αἰτήσει ᾠόν, ἐπιδώσει αὐτῷ σκορπίον;
\VS{13}εἰ οὖν ὑμεῖς πονηροὶ ὑπάρχοντες οἴδατε δόματα ἀγαθὰ διδόναι τοῖς τέκνοις ὑμῶν, πόσῳ μᾶλλον ὁ Πατὴρ ὁ ἐξ οὐρανοῦ δώσει Πνεῦμα Ἅγιον τοῖς αἰτοῦσιν αὐτόν.
\par }{\PP \VS{14}Καὶ ἦν ἐκβάλλων δαιμόνιον καὶ αὐτὸ ἦν κωφόν· ἐγένετο δὲ τοῦ δαιμονίου ἐξελθόντος ἐλάλησεν ὁ κωφός καὶ ἐθαύμασαν οἱ ὄχλοι.
\VS{15}τινὲς δὲ ἐξ αὐτῶν εἶπον· Ἐν Βεελζεβοὺλ τῷ ἄρχοντι τῶν δαιμονίων ἐκβάλλει τὰ δαιμόνια·
\VS{16}ἕτεροι δὲ πειράζοντες σημεῖον ἐξ οὐρανοῦ ἐζήτουν παρ᾽ αὐτοῦ.
\VS{17}Αὐτὸς δὲ εἰδὼς αὐτῶν τὰ διανοήματα εἶπεν αὐτοῖς· Πᾶσα βασιλεία ἐφ᾽ ἑαυτὴν διαμερισθεῖσα ἐρημοῦται καὶ οἶκος ἐπὶ οἶκον πίπτει.
\VS{18}εἰ δὲ καὶ ὁ Σατανᾶς ἐφ᾽ ἑαυτὸν διεμερίσθη, πῶς σταθήσεται ἡ βασιλεία αὐτοῦ; ὅτι λέγετε ἐν Βεελζεβοὺλ ἐκβάλλειν με τὰ δαιμόνια.
\VS{19}εἰ δὲ ἐγὼ ἐν Βεελζεβοὺλ ἐκβάλλω τὰ δαιμόνια, οἱ υἱοὶ ὑμῶν ἐν τίνι ἐκβάλλουσιν; διὰ τοῦτο αὐτοὶ ὑμῶν κριταὶ ἔσονται.
\VS{20}εἰ δὲ ἐν δακτύλῳ Θεοῦ ἐγὼ ἐκβάλλω τὰ δαιμόνια, ἄρα ἔφθασεν ἐφ᾽ ὑμᾶς ἡ βασιλεία τοῦ Θεοῦ.
\VS{21}Ὅταν ὁ ἰσχυρὸς καθωπλισμένος φυλάσσῃ τὴν ἑαυτοῦ αὐλήν, ἐν εἰρήνῃ ἐστὶν τὰ ὑπάρχοντα αὐτοῦ·
\VS{22}ἐπὰν δὲ ἰσχυρότερος αὐτοῦ ἐπελθὼν νικήσῃ αὐτόν, τὴν πανοπλίαν αὐτοῦ αἴρει ἐφ᾽ ᾗ ἐπεποίθει καὶ τὰ σκῦλα αὐτοῦ διαδίδωσιν.
\VS{23}Ὁ μὴ ὢν μετ᾽ ἐμοῦ κατ᾽ ἐμοῦ ἐστιν, καὶ ὁ μὴ συνάγων μετ᾽ ἐμοῦ σκορπίζει.
\VS{24}Ὅταν τὸ ἀκάθαρτον πνεῦμα ἐξέλθῃ ἀπὸ τοῦ ἀνθρώπου, διέρχεται δι᾽ ἀνύδρων τόπων ζητοῦν ἀνάπαυσιν καὶ μὴ εὑρίσκον· τότε λέγει· Ὑποστρέψω εἰς τὸν οἶκόν μου ὅθεν ἐξῆλθον·
\VS{25}καὶ ἐλθὸν εὑρίσκει σεσαρωμένον καὶ κεκοσμημένον.
\VS{26}τότε πορεύεται καὶ παραλαμβάνει ἕτερα πνεύματα πονηρότερα ἑαυτοῦ ἑπτά καὶ εἰσελθόντα κατοικεῖ ἐκεῖ· καὶ γίνεται τὰ ἔσχατα τοῦ ἀνθρώπου ἐκείνου χείρονα τῶν πρώτων.
\par }{\PP \VS{27}Ἐγένετο δὲ ἐν τῷ λέγειν αὐτὸν ταῦτα ἐπάρασά τις φωνὴν γυνὴ ἐκ τοῦ ὄχλου εἶπεν αὐτῷ· Μακαρία ἡ κοιλία ἡ βαστάσασά σε καὶ μαστοὶ οὓς ἐθήλασας.
\VS{28}Αὐτὸς δὲ εἶπεν· Μενοῦν μακάριοι οἱ ἀκούοντες τὸν λόγον τοῦ Θεοῦ καὶ φυλάσσοντες.
\par }{\PP \VS{29}Τῶν δὲ ὄχλων ἐπαθροιζομένων ἤρξατο λέγειν· Ἡ γενεὰ αὕτη γενεὰ πονηρά ἐστιν· σημεῖον ζητεῖ, καὶ σημεῖον οὐ δοθήσεται αὐτῇ εἰ μὴ τὸ σημεῖον Ἰωνᾶ.
\VS{30}καθὼς γὰρ ἐγένετο Ἰωνᾶς τοῖς Νινευίταις σημεῖον, οὕτως ἔσται καὶ ὁ Υἱὸς τοῦ ἀνθρώπου τῇ γενεᾷ ταύτῃ.
\VS{31}Βασίλισσα νότου ἐγερθήσεται ἐν τῇ κρίσει μετὰ τῶν ἀνδρῶν τῆς γενεᾶς ταύτης καὶ κατακρινεῖ αὐτούς, ὅτι ἦλθεν ἐκ τῶν περάτων τῆς γῆς ἀκοῦσαι τὴν σοφίαν Σολομῶνος, καὶ ἰδοὺ πλεῖον Σολομῶνος ὧδε.
\VS{32}ἄνδρες Νινευῖται ἀναστήσονται ἐν τῇ κρίσει μετὰ τῆς γενεᾶς ταύτης καὶ κατακρινοῦσιν αὐτήν· ὅτι μετενόησαν εἰς τὸ κήρυγμα Ἰωνᾶ, καὶ ἰδοὺ πλεῖον Ἰωνᾶ ὧδε.
\par }{\PP \VS{33}Οὐδεὶς λύχνον ἅψας εἰς κρύπτην τίθησιν οὐδὲ ὑπὸ τὸν μόδιον ἀλλ᾽ ἐπὶ τὴν λυχνίαν, ἵνα οἱ εἰσπορευόμενοι τὸ φῶς βλέπωσιν.
\par }{\PP \VS{34}Ὁ λύχνος τοῦ σώματός ἐστιν ὁ ὀφθαλμός σου. ὅταν ὁ ὀφθαλμός σου ἁπλοῦς ᾖ, καὶ ὅλον τὸ σῶμά σου φωτεινόν ἐστιν· ἐπὰν δὲ πονηρὸς ᾖ, καὶ τὸ σῶμά σου σκοτεινόν.
\VS{35}σκόπει οὖν μὴ τὸ φῶς τὸ ἐν σοὶ σκότος ἐστίν.
\VS{36}εἰ οὖν τὸ σῶμά σου ὅλον φωτεινόν, μὴ ἔχον μέρος τι σκοτεινόν, ἔσται φωτεινὸν ὅλον ὡς ὅταν ὁ λύχνος τῇ ἀστραπῇ φωτίζῃ σε.
\par }{\PP \VS{37}Ἐν δὲ τῷ λαλῆσαι ἐρωτᾷ αὐτὸν Φαρισαῖος ὅπως ἀριστήσῃ παρ᾽ αὐτῷ· εἰσελθὼν δὲ ἀνέπεσεν.
\VS{38}ὁ δὲ Φαρισαῖος ἰδὼν ἐθαύμασεν ὅτι οὐ πρῶτον ἐβαπτίσθη πρὸ τοῦ ἀρίστου.
\VS{39}Εἶπεν δὲ ὁ Κύριος πρὸς αὐτόν· Νῦν ὑμεῖς οἱ Φαρισαῖοι τὸ ἔξωθεν τοῦ ποτηρίου καὶ τοῦ πίνακος καθαρίζετε, τὸ δὲ ἔσωθεν ὑμῶν γέμει ἁρπαγῆς καὶ πονηρίας.
\VS{40}ἄφρονες, οὐχ ὁ ποιήσας τὸ ἔξωθεν καὶ τὸ ἔσωθεν ἐποίησεν;
\VS{41}πλὴν τὰ ἐνόντα δότε ἐλεημοσύνην, καὶ ἰδοὺ πάντα καθαρὰ ὑμῖν ἐστιν.
\par }{\PP \VS{42}Ἀλλὰ οὐαὶ ὑμῖν τοῖς Φαρισαίοις, ὅτι ἀποδεκατοῦτε τὸ ἡδύοσμον καὶ τὸ πήγανον καὶ πᾶν λάχανον καὶ παρέρχεσθε τὴν κρίσιν καὶ τὴν ἀγάπην τοῦ Θεοῦ· ταῦτα δὲ ἔδει ποιῆσαι κἀκεῖνα μὴ παρεῖναι.
\par }{\PP \VS{43}Οὐαὶ ὑμῖν τοῖς Φαρισαίοις, ὅτι ἀγαπᾶτε τὴν πρωτοκαθεδρίαν ἐν ταῖς συναγωγαῖς καὶ τοὺς ἀσπασμοὺς ἐν ταῖς ἀγοραῖς.
\VS{44}οὐαὶ ὑμῖν, ὅτι ἐστὲ ὡς τὰ μνημεῖα τὰ ἄδηλα, καὶ οἱ ἄνθρωποι οἱ περιπατοῦντες ἐπάνω οὐκ οἴδασιν.
\par }{\PP \VS{45}Ἀποκριθεὶς δέ τις τῶν νομικῶν λέγει αὐτῷ· Διδάσκαλε, ταῦτα λέγων καὶ ἡμᾶς ὑβρίζεις.
\VS{46}Ὁ δὲ εἶπεν· Καὶ ὑμῖν τοῖς νομικοῖς οὐαί, ὅτι φορτίζετε τοὺς ἀνθρώπους φορτία δυσβάστακτα, καὶ αὐτοὶ ἑνὶ τῶν δακτύλων ὑμῶν οὐ προσψαύετε τοῖς φορτίοις.
\par }{\PP \VS{47}οὐαὶ ὑμῖν, ὅτι οἰκοδομεῖτε τὰ μνημεῖα τῶν προφητῶν, οἱ δὲ πατέρες ὑμῶν ἀπέκτειναν αὐτούς.
\VS{48}ἄρα μάρτυρές ἐστε καὶ συνευδοκεῖτε τοῖς ἔργοις τῶν πατέρων ὑμῶν, ὅτι αὐτοὶ μὲν ἀπέκτειναν αὐτοὺς, ὑμεῖς δὲ οἰκοδομεῖτε.
\VS{49}διὰ τοῦτο καὶ ἡ σοφία τοῦ Θεοῦ εἶπεν· Ἀποστελῶ εἰς αὐτοὺς προφήτας καὶ ἀποστόλους, καὶ ἐξ αὐτῶν ἀποκτενοῦσιν καὶ διώξουσιν,
\VS{50}ἵνα ἐκζητηθῇ τὸ αἷμα πάντων τῶν προφητῶν τὸ ἐκκεχυμένον ἀπὸ καταβολῆς κόσμου ἀπὸ τῆς γενεᾶς ταύτης,
\VS{51}ἀπὸ αἵματος Ἅβελ ἕως αἵματος Ζαχαρίου τοῦ ἀπολομένου μεταξὺ τοῦ θυσιαστηρίου καὶ τοῦ οἴκου· ναί λέγω ὑμῖν, ἐκζητηθήσεται ἀπὸ τῆς γενεᾶς ταύτης.
\par }{\PP \VS{52}Οὐαὶ ὑμῖν τοῖς νομικοῖς, ὅτι ἤρατε τὴν κλεῖδα τῆς γνώσεως· αὐτοὶ οὐκ εἰσήλθατε καὶ τοὺς εἰσερχομένους ἐκωλύσατε.
\par }{\PP \VS{53}Κἀκεῖθεν ἐξελθόντος αὐτοῦ ἤρξαντο οἱ γραμματεῖς καὶ οἱ Φαρισαῖοι δεινῶς ἐνέχειν καὶ ἀποστοματίζειν αὐτὸν περὶ πλειόνων,
\VS{54}ἐνεδρεύοντες αὐτὸν θηρεῦσαί τι ἐκ τοῦ στόματος αὐτοῦ.

\par }\Chap{12}{\PP \VerseOne{1}Ἐν οἷς ἐπισυναχθεισῶν τῶν μυριάδων τοῦ ὄχλου, ὥστε καταπατεῖν ἀλλήλους, ἤρξατο λέγειν πρὸς τοὺς μαθητὰς αὐτοῦ πρῶτον· Προσέχετε ἑαυτοῖς ἀπὸ τῆς ζύμης, ἥτις ἐστὶν ὑπόκρισις, τῶν Φαρισαίων.
\par }{\PP \VS{2}οὐδὲν δὲ συγκεκαλυμμένον ἐστὶν ὃ οὐκ ἀποκαλυφθήσεται καὶ κρυπτὸν ὃ οὐ γνωσθήσεται.
\VS{3}ἀνθ᾽ ὧν ὅσα ἐν τῇ σκοτίᾳ εἴπατε ἐν τῷ φωτὶ ἀκουσθήσεται, καὶ ὃ πρὸς τὸ οὖς ἐλαλήσατε ἐν τοῖς ταμείοις κηρυχθήσεται ἐπὶ τῶν δωμάτων.
\par }{\PP \VS{4}Λέγω δὲ ὑμῖν τοῖς φίλοις μου, μὴ φοβηθῆτε ἀπὸ τῶν ἀποκτεινόντων τὸ σῶμα καὶ μετὰ ταῦτα μὴ ἐχόντων περισσότερόν τι ποιῆσαι.
\VS{5}ὑποδείξω δὲ ὑμῖν τίνα φοβηθῆτε· φοβήθητε τὸν μετὰ τὸ ἀποκτεῖναι ἔχοντα ἐξουσίαν ἐμβαλεῖν εἰς τὴν γέενναν. ναί λέγω ὑμῖν, τοῦτον φοβήθητε.
\VS{6}Οὐχὶ πέντε στρουθία πωλοῦνται ἀσσαρίων δύο; καὶ ἓν ἐξ αὐτῶν οὐκ ἔστιν ἐπιλελησμένον ἐνώπιον τοῦ Θεοῦ.
\VS{7}ἀλλὰ καὶ αἱ τρίχες τῆς κεφαλῆς ὑμῶν πᾶσαι ἠρίθμηνται. μὴ φοβεῖσθε· πολλῶν στρουθίων διαφέρετε.
\VS{8}Λέγω δὲ ὑμῖν, πᾶς ὃς ἂν ὁμολογήσῃ ἐν ἐμοὶ ἔμπροσθεν τῶν ἀνθρώπων, καὶ ὁ Υἱὸς τοῦ ἀνθρώπου ὁμολογήσει ἐν αὐτῷ ἔμπροσθεν τῶν ἀγγέλων τοῦ Θεοῦ·
\VS{9}ὁ δὲ ἀρνησάμενός με ἐνώπιον τῶν ἀνθρώπων ἀπαρνηθήσεται ἐνώπιον τῶν ἀγγέλων τοῦ Θεοῦ.
\par }{\PP \VS{10}καὶ πᾶς ὃς ἐρεῖ λόγον εἰς τὸν Υἱὸν τοῦ ἀνθρώπου, ἀφεθήσεται αὐτῷ· τῷ δὲ εἰς τὸ Ἅγιον Πνεῦμα βλασφημήσαντι οὐκ ἀφεθήσεται.
\par }{\PP \VS{11}Ὅταν δὲ εἰσφέρωσιν ὑμᾶς ἐπὶ τὰς συναγωγὰς καὶ τὰς ἀρχὰς καὶ τὰς ἐξουσίας, μὴ μεριμνήσητε πῶς ἢ τί ἀπολογήσησθε ἢ τί εἴπητε·
\VS{12}τὸ γὰρ Ἅγιον Πνεῦμα διδάξει ὑμᾶς ἐν αὐτῇ τῇ ὥρᾳ ἃ δεῖ εἰπεῖν.
\par }{\PP \VS{13}Εἶπεν δέ τις ἐκ τοῦ ὄχλου αὐτῷ· Διδάσκαλε, εἰπὲ τῷ ἀδελφῷ μου μερίσασθαι μετ᾽ ἐμοῦ τὴν κληρονομίαν.
\VS{14}Ὁ δὲ εἶπεν αὐτῷ· Ἄνθρωπε, τίς με κατέστησεν κριτὴν ἢ μεριστὴν ἐφ᾽ ὑμᾶς;
\VS{15}εἶπεν δὲ πρὸς αὐτούς· Ὁρᾶτε καὶ φυλάσσεσθε ἀπὸ πάσης πλεονεξίας, ὅτι οὐκ ἐν τῷ περισσεύειν τινὶ ἡ ζωὴ αὐτοῦ ἐστιν ἐκ τῶν ὑπαρχόντων αὐτῷ.
\par }{\PP \VS{16}Εἶπεν δὲ παραβολὴν πρὸς αὐτοὺς λέγων· Ἀνθρώπου τινὸς πλουσίου εὐφόρησεν ἡ χώρα.
\VS{17}καὶ διελογίζετο ἐν ἑαυτῷ λέγων· Τί ποιήσω, ὅτι οὐκ ἔχω ποῦ συνάξω τοὺς καρπούς μου;
\VS{18}καὶ εἶπεν· Τοῦτο ποιήσω, καθελῶ μου τὰς ἀποθήκας καὶ μείζονας οἰκοδομήσω καὶ συνάξω ἐκεῖ πάντα τὸν σῖτον καὶ τὰ ἀγαθά μου
\VS{19}καὶ ἐρῶ τῇ ψυχῇ μου· Ψυχή, ἔχεις πολλὰ ἀγαθὰ κείμενα εἰς ἔτη πολλά· ἀναπαύου, φάγε, πίε, εὐφραίνου.
\VS{20}Εἶπεν δὲ αὐτῷ ὁ Θεός· Ἄφρων, ταύτῃ τῇ νυκτὶ τὴν ψυχήν σου ἀπαιτοῦσιν ἀπὸ σοῦ· ἃ δὲ ἡτοίμασας, τίνι ἔσται;
\VS{21}Οὕτως ὁ θησαυρίζων ἑαυτῷ καὶ μὴ εἰς Θεὸν πλουτῶν.
\par }{\PP \VS{22}Εἶπεν δὲ πρὸς τοὺς μαθητὰς αὐτοῦ· Διὰ τοῦτο λέγω ὑμῖν· μὴ μεριμνᾶτε τῇ ψυχῇ τί φάγητε, μηδὲ τῷ σώματι τί ἐνδύσησθε.
\VS{23}ἡ γὰρ ψυχὴ πλεῖόν ἐστιν τῆς τροφῆς καὶ τὸ σῶμα τοῦ ἐνδύματος.
\VS{24}κατανοήσατε τοὺς κόρακας ὅτι οὐ σπείρουσιν οὐδὲ θερίζουσιν, οἷς οὐκ ἔστιν ταμεῖον οὐδὲ ἀποθήκη, καὶ ὁ Θεὸς τρέφει αὐτούς· πόσῳ μᾶλλον ὑμεῖς διαφέρετε τῶν πετεινῶν.
\VS{25}τίς δὲ ἐξ ὑμῶν μεριμνῶν δύναται ἐπὶ τὴν ἡλικίαν αὐτοῦ προσθεῖναι πῆχυν;
\VS{26}εἰ οὖν οὐδὲ ἐλάχιστον δύνασθε, τί περὶ τῶν λοιπῶν μεριμνᾶτε;
\VS{27}Κατανοήσατε τὰ κρίνα πῶς αὐξάνει· οὐ κοπιᾷ οὐδὲ νήθει· λέγω δὲ ὑμῖν, οὐδὲ Σολομὼν ἐν πάσῃ τῇ δόξῃ αὐτοῦ περιεβάλετο ὡς ἓν τούτων.
\VS{28}εἰ δὲ ἐν ἀγρῷ τὸν χόρτον ὄντα σήμερον καὶ αὔριον εἰς κλίβανον βαλλόμενον ὁ Θεὸς οὕτως ἀμφιέζει, πόσῳ μᾶλλον ὑμᾶς, ὀλιγόπιστοι.
\VS{29}Καὶ ὑμεῖς μὴ ζητεῖτε τί φάγητε καὶ τί πίητε καὶ μὴ μετεωρίζεσθε·
\VS{30}ταῦτα γὰρ πάντα τὰ ἔθνη τοῦ κόσμου ἐπιζητοῦσιν, ὑμῶν δὲ ὁ Πατὴρ οἶδεν ὅτι χρῄζετε τούτων.
\VS{31}πλὴν ζητεῖτε τὴν βασιλείαν αὐτοῦ, καὶ ταῦτα προστεθήσεται ὑμῖν.
\VS{32}Μὴ φοβοῦ, τὸ μικρὸν ποίμνιον, ὅτι εὐδόκησεν ὁ Πατὴρ ὑμῶν δοῦναι ὑμῖν τὴν βασιλείαν.
\par }{\PP \VS{33}Πωλήσατε τὰ ὑπάρχοντα ὑμῶν καὶ δότε ἐλεημοσύνην· ποιήσατε ἑαυτοῖς βαλλάντια μὴ παλαιούμενα, θησαυρὸν ἀνέκλειπτον ἐν τοῖς οὐρανοῖς, ὅπου κλέπτης οὐκ ἐγγίζει οὐδὲ σὴς διαφθείρει·
\VS{34}ὅπου γάρ ἐστιν ὁ θησαυρὸς ὑμῶν, ἐκεῖ καὶ ἡ καρδία ὑμῶν ἔσται.
\par }{\PP \VS{35}Ἔστωσαν ὑμῶν αἱ ὀσφύες περιεζωσμέναι καὶ οἱ λύχνοι καιόμενοι·
\VS{36}καὶ ὑμεῖς ὅμοιοι ἀνθρώποις προσδεχομένοις τὸν κύριον ἑαυτῶν πότε ἀναλύσῃ ἐκ τῶν γάμων, ἵνα ἐλθόντος καὶ κρούσαντος εὐθέως ἀνοίξωσιν αὐτῷ.
\VS{37}μακάριοι οἱ δοῦλοι ἐκεῖνοι, οὓς ἐλθὼν ὁ κύριος εὑρήσει γρηγοροῦντας· ἀμὴν λέγω ὑμῖν ὅτι περιζώσεται καὶ ἀνακλινεῖ αὐτοὺς καὶ παρελθὼν διακονήσει αὐτοῖς.
\VS{38}κἂν ἐν τῇ δευτέρᾳ κἂν ἐν τῇ τρίτῃ φυλακῇ ἔλθῃ καὶ εὕρῃ οὕτως, μακάριοί εἰσιν ἐκεῖνοι.
\VS{39}Τοῦτο δὲ γινώσκετε ὅτι εἰ ᾔδει ὁ οἰκοδεσπότης ποίᾳ ὥρᾳ ὁ κλέπτης ἔρχεται, οὐκ ἂν ἀφῆκεν διορυχθῆναι τὸν οἶκον αὐτοῦ.
\VS{40}καὶ ὑμεῖς γίνεσθε ἕτοιμοι, ὅτι ᾗ ὥρᾳ οὐ δοκεῖτε ὁ Υἱὸς τοῦ ἀνθρώπου ἔρχεται.
\par }{\PP \VS{41}Εἶπεν δὲ ὁ Πέτρος· Κύριε, πρὸς ἡμᾶς τὴν παραβολὴν ταύτην λέγεις ἢ καὶ πρὸς πάντας;
\VS{42}Καὶ εἶπεν ὁ Κύριος· Τίς ἄρα ἐστὶν ὁ πιστὸς οἰκονόμος ὁ φρόνιμος, ὃν καταστήσει ὁ κύριος ἐπὶ τῆς θεραπείας αὐτοῦ τοῦ διδόναι ἐν καιρῷ τὸ σιτομέτριον;
\VS{43}μακάριος ὁ δοῦλος ἐκεῖνος, ὃν ἐλθὼν ὁ κύριος αὐτοῦ εὑρήσει ποιοῦντα οὕτως.
\VS{44}ἀληθῶς λέγω ὑμῖν ὅτι ἐπὶ πᾶσιν τοῖς ὑπάρχουσιν αὐτοῦ καταστήσει αὐτόν.
\VS{45}Ἐὰν δὲ εἴπῃ ὁ δοῦλος ἐκεῖνος ἐν τῇ καρδίᾳ αὐτοῦ· Χρονίζει ὁ κύριός μου ἔρχεσθαι, καὶ ἄρξηται τύπτειν τοὺς παῖδας καὶ τὰς παιδίσκας, ἐσθίειν τε καὶ πίνειν καὶ μεθύσκεσθαι,
\VS{46}ἥξει ὁ κύριος τοῦ δούλου ἐκείνου ἐν ἡμέρᾳ ᾗ οὐ προσδοκᾷ καὶ ἐν ὥρᾳ ᾗ οὐ γινώσκει, καὶ διχοτομήσει αὐτὸν καὶ τὸ μέρος αὐτοῦ μετὰ τῶν ἀπίστων θήσει.
\par }{\PP \VS{47}Ἐκεῖνος δὲ ὁ δοῦλος ὁ γνοὺς τὸ θέλημα τοῦ κυρίου αὐτοῦ καὶ μὴ ἑτοιμάσας ἢ ποιήσας πρὸς τὸ θέλημα αὐτοῦ δαρήσεται πολλάς·
\VS{48}ὁ δὲ μὴ γνοὺς, ποιήσας δὲ ἄξια πληγῶν δαρήσεται ὀλίγας. παντὶ δὲ ᾧ ἐδόθη πολύ, πολὺ ζητηθήσεται παρ᾽ αὐτοῦ, καὶ ᾧ παρέθεντο πολύ, περισσότερον αἰτήσουσιν αὐτόν.
\par }{\PP \VS{49}Πῦρ ἦλθον βαλεῖν ἐπὶ τὴν γῆν, καὶ τί θέλω εἰ ἤδη ἀνήφθη.
\VS{50}βάπτισμα δὲ ἔχω βαπτισθῆναι, καὶ πῶς συνέχομαι ἕως ὅτου τελεσθῇ.
\VS{51}Δοκεῖτε ὅτι εἰρήνην παρεγενόμην δοῦναι ἐν τῇ γῇ; οὐχί, λέγω ὑμῖν, ἀλλ᾽ ἢ διαμερισμόν.
\VS{52}ἔσονται γὰρ ἀπὸ τοῦ νῦν πέντε ἐν ἑνὶ οἴκῳ διαμεμερισμένοι, τρεῖς ἐπὶ δυσὶν καὶ δύο ἐπὶ τρισίν,
\VS{53}διαμερισθήσονται πατὴρ ἐπὶ υἱῷ καὶ υἱὸς ἐπὶ πατρί, μήτηρ ἐπὶ τὴν θυγατέρα καὶ θυγάτηρ ἐπὶ τὴν μητέρα, πενθερὰ ἐπὶ τὴν νύμφην αὐτῆς καὶ νύμφη ἐπὶ τὴν πενθεράν.
\VS{54}Ἔλεγεν δὲ καὶ τοῖς ὄχλοις· Ὅταν ἴδητε τὴν νεφέλην ἀνατέλλουσαν ἐπὶ δυσμῶν, εὐθέως λέγετε ὅτι Ὄμβρος ἔρχεται, καὶ γίνεται οὕτως·
\VS{55}καὶ ὅταν νότον πνέοντα, λέγετε ὅτι Καύσων ἔσται, καὶ γίνεται.
\VS{56}ὑποκριταί, τὸ πρόσωπον τῆς γῆς καὶ τοῦ οὐρανοῦ οἴδατε δοκιμάζειν, τὸν καιρὸν δὲ τοῦτον πῶς οὐκ οἴδατε δοκιμάζειν;
\par }{\PP \VS{57}Τί δὲ καὶ ἀφ᾽ ἑαυτῶν οὐ κρίνετε τὸ δίκαιον;
\VS{58}ὡς γὰρ ὑπάγεις μετὰ τοῦ ἀντιδίκου σου ἐπ᾽ ἄρχοντα, ἐν τῇ ὁδῷ δὸς ἐργασίαν ἀπηλλάχθαι ἀπ᾽ αὐτοῦ, μήποτε κατασύρῃ σε πρὸς τὸν κριτήν, καὶ ὁ κριτής σε παραδώσει τῷ πράκτορι, καὶ ὁ πράκτωρ σε βαλεῖ εἰς φυλακήν.
\VS{59}λέγω σοι, οὐ μὴ ἐξέλθῃς ἐκεῖθεν, ἕως καὶ τὸ ἔσχατον λεπτὸν ἀποδῷς.

\par }\Chap{13}{\PP \VerseOne{1}Παρῆσαν δέ τινες ἐν αὐτῷ τῷ καιρῷ ἀπαγγέλλοντες αὐτῷ περὶ τῶν Γαλιλαίων ὧν τὸ αἷμα Πιλᾶτος ἔμιξεν μετὰ τῶν θυσιῶν αὐτῶν.
\VS{2}καὶ ἀποκριθεὶς εἶπεν αὐτοῖς· Δοκεῖτε ὅτι οἱ Γαλιλαῖοι οὗτοι ἁμαρτωλοὶ παρὰ πάντας τοὺς Γαλιλαίους ἐγένοντο, ὅτι ταῦτα πεπόνθασιν;
\VS{3}οὐχί, λέγω ὑμῖν, ἀλλ᾽ ἐὰν μὴ μετανοῆτε πάντες ὁμοίως ἀπολεῖσθε.
\VS{4}ἢ ἐκεῖνοι οἱ δεκαοκτὼ ἐφ᾽ οὓς ἔπεσεν ὁ πύργος ἐν τῷ Σιλωὰμ καὶ ἀπέκτεινεν αὐτούς, δοκεῖτε ὅτι αὐτοὶ ὀφειλέται ἐγένοντο παρὰ πάντας τοὺς ἀνθρώπους τοὺς κατοικοῦντας Ἰερουσαλήμ;
\VS{5}οὐχί, λέγω ὑμῖν, ἀλλ᾽ ἐὰν μὴ μετανοῆτε πάντες ὡσαύτως ἀπολεῖσθε.
\par }{\PP \VS{6}Ἔλεγεν δὲ ταύτην τὴν παραβολήν· Συκῆν εἶχέν τις πεφυτευμένην ἐν τῷ ἀμπελῶνι αὐτοῦ, καὶ ἦλθεν ζητῶν καρπὸν ἐν αὐτῇ καὶ οὐχ εὗρεν.
\VS{7}εἶπεν δὲ πρὸς τὸν ἀμπελουργόν· Ἰδοὺ τρία ἔτη ἀφ᾽ οὗ ἔρχομαι ζητῶν καρπὸν ἐν τῇ συκῇ ταύτῃ καὶ οὐχ εὑρίσκω· ἔκκοψον οὖν αὐτήν, ἵνατί καὶ τὴν γῆν καταργεῖ;
\VS{8}Ὁ δὲ ἀποκριθεὶς λέγει αὐτῷ· Κύριε, ἄφες αὐτὴν καὶ τοῦτο τὸ ἔτος, ἕως ὅτου σκάψω περὶ αὐτὴν καὶ βάλω κόπρια,
\VS{9}κἂν μὲν ποιήσῃ καρπὸν εἰς τὸ μέλλον· εἰ δὲ μή γε, ἐκκόψεις αὐτήν.
\par }{\PP \VS{10}Ἦν δὲ διδάσκων ἐν μιᾷ τῶν συναγωγῶν ἐν τοῖς σάββασιν.
\VS{11}καὶ ἰδοὺ γυνὴ πνεῦμα ἔχουσα ἀσθενείας ἔτη δεκαοκτώ καὶ ἦν συνκύπτουσα καὶ μὴ δυναμένη ἀνακύψαι εἰς τὸ παντελές.
\VS{12}ἰδὼν δὲ αὐτὴν ὁ Ἰησοῦς προσεφώνησεν καὶ εἶπεν αὐτῇ· Γύναι, ἀπολέλυσαι τῆς ἀσθενείας σου,
\VS{13}καὶ ἐπέθηκεν αὐτῇ τὰς χεῖρας· καὶ παραχρῆμα ἀνωρθώθη καὶ ἐδόξαζεν τὸν Θεόν.
\VS{14}Ἀποκριθεὶς δὲ ὁ ἀρχισυνάγωγος, ἀγανακτῶν ὅτι τῷ σαββάτῳ ἐθεράπευσεν ὁ Ἰησοῦς, ἔλεγεν τῷ ὄχλῳ ὅτι Ἓξ ἡμέραι εἰσὶν ἐν αἷς δεῖ ἐργάζεσθαι· ἐν αὐταῖς οὖν ἐρχόμενοι θεραπεύεσθε καὶ μὴ τῇ ἡμέρᾳ τοῦ σαββάτου.
\VS{15}Ἀπεκρίθη δὲ αὐτῷ ὁ Κύριος καὶ εἶπεν· Ὑποκριταί, ἕκαστος ὑμῶν τῷ σαββάτῳ οὐ λύει τὸν βοῦν αὐτοῦ ἢ τὸν ὄνον ἀπὸ τῆς φάτνης καὶ ἀπαγαγὼν ποτίζει;
\VS{16}ταύτην δὲ θυγατέρα Ἀβραὰμ οὖσαν, ἣν ἔδησεν ὁ Σατανᾶς ἰδοὺ δέκα καὶ ὀκτὼ ἔτη, οὐκ ἔδει λυθῆναι ἀπὸ τοῦ δεσμοῦ τούτου τῇ ἡμέρᾳ τοῦ σαββάτου;
\VS{17}Καὶ ταῦτα λέγοντος αὐτοῦ κατῃσχύνοντο πάντες οἱ ἀντικείμενοι αὐτῷ, καὶ πᾶς ὁ ὄχλος ἔχαιρεν ἐπὶ πᾶσιν τοῖς ἐνδόξοις τοῖς γινομένοις ὑπ᾽ αὐτοῦ.
\par }{\PP \VS{18}Ἔλεγεν οὖν· Τίνι ὁμοία ἐστὶν ἡ βασιλεία τοῦ Θεοῦ καὶ τίνι ὁμοιώσω αὐτήν;
\VS{19}ὁμοία ἐστὶν κόκκῳ σινάπεως, ὃν λαβὼν ἄνθρωπος ἔβαλεν εἰς κῆπον ἑαυτοῦ, καὶ ηὔξησεν καὶ ἐγένετο εἰς δένδρον, καὶ τὰ πετεινὰ τοῦ οὐρανοῦ κατεσκήνωσεν ἐν τοῖς κλάδοις αὐτοῦ.
\par }{\PP \VS{20}Καὶ πάλιν εἶπεν· Τίνι ὁμοιώσω τὴν βασιλείαν τοῦ Θεοῦ;
\VS{21}ὁμοία ἐστὶν ζύμῃ, ἣν λαβοῦσα γυνὴ ἐνέκρυψεν εἰς ἀλεύρου σάτα τρία ἕως οὗ ἐζυμώθη ὅλον.
\par }{\PP \VS{22}Καὶ διεπορεύετο κατὰ πόλεις καὶ κώμας διδάσκων καὶ πορείαν ποιούμενος εἰς Ἱεροσόλυμα.
\par }{\PP \VS{23}Εἶπεν δέ τις αὐτῷ· Κύριε, εἰ ὀλίγοι οἱ σῳζόμενοι; Ὁ δὲ εἶπεν πρὸς αὐτούς·
\VS{24}Ἀγωνίζεσθε εἰσελθεῖν διὰ τῆς στενῆς θύρας, ὅτι πολλοί, λέγω ὑμῖν, ζητήσουσιν εἰσελθεῖν καὶ οὐκ ἰσχύσουσιν.
\VS{25}ἀφ᾽ οὗ ἂν ἐγερθῇ ὁ οἰκοδεσπότης καὶ ἀποκλείσῃ τὴν θύραν καὶ ἄρξησθε ἔξω ἑστάναι καὶ κρούειν τὴν θύραν λέγοντες· Κύριε, ἄνοιξον ἡμῖν, Καὶ ἀποκριθεὶς ἐρεῖ ὑμῖν· Οὐκ οἶδα ὑμᾶς πόθεν ἐστέ.
\VS{26}Τότε ἄρξεσθε λέγειν· Ἐφάγομεν ἐνώπιόν σου καὶ ἐπίομεν καὶ ἐν ταῖς πλατείαις ἡμῶν ἐδίδαξας·
\VS{27}Καὶ ἐρεῖ Λέγων ὑμῖν· Οὐκ οἶδα ὑμᾶς πόθεν ἐστέ· ἀπόστητε ἀπ᾽ ἐμοῦ πάντες ἐργάται ἀδικίας.
\VS{28}Ἐκεῖ ἔσται ὁ κλαυθμὸς καὶ ὁ βρυγμὸς τῶν ὀδόντων, ὅταν ὄψησθε Ἀβραὰμ καὶ Ἰσαὰκ καὶ Ἰακὼβ καὶ πάντας τοὺς προφήτας ἐν τῇ βασιλείᾳ τοῦ Θεοῦ, ὑμᾶς δὲ ἐκβαλλομένους ἔξω.
\VS{29}καὶ ἥξουσιν ἀπὸ ἀνατολῶν καὶ δυσμῶν καὶ ἀπὸ βορρᾶ καὶ νότου καὶ ἀνακλιθήσονται ἐν τῇ βασιλείᾳ τοῦ Θεοῦ.
\VS{30}καὶ ἰδοὺ εἰσὶν ἔσχατοι οἳ ἔσονται πρῶτοι καὶ εἰσὶν πρῶτοι οἳ ἔσονται ἔσχατοι.
\par }{\PP \VS{31}Ἐν αὐτῇ τῇ ὥρᾳ προσῆλθάν τινες Φαρισαῖοι λέγοντες αὐτῷ· Ἔξελθε καὶ πορεύου ἐντεῦθεν, ὅτι Ἡρῴδης θέλει σε ἀποκτεῖναι.
\VS{32}Καὶ εἶπεν αὐτοῖς· Πορευθέντες εἴπατε τῇ ἀλώπεκι ταύτῃ· Ἰδοὺ ἐκβάλλω δαιμόνια καὶ ἰάσεις ἀποτελῶ σήμερον καὶ αὔριον καὶ τῇ τρίτῃ τελειοῦμαι.
\VS{33}πλὴν δεῖ με σήμερον καὶ αὔριον καὶ τῇ ἐχομένῃ πορεύεσθαι, ὅτι οὐκ ἐνδέχεται προφήτην ἀπολέσθαι ἔξω Ἰερουσαλήμ.
\par }{\PP \VS{34}Ἰερουσαλὴμ Ἰερουσαλήμ, ἡ ἀποκτείνουσα τοὺς προφήτας καὶ λιθοβολοῦσα τοὺς ἀπεσταλμένους πρὸς αὐτήν, ποσάκις ἠθέλησα ἐπισυνάξαι τὰ τέκνα σου ὃν τρόπον ὄρνις τὴν ἑαυτῆς νοσσιὰν ὑπὸ τὰς πτέρυγας, καὶ οὐκ ἠθελήσατε.
\VS{35}ἰδοὺ ἀφίεται ὑμῖν ὁ οἶκος ὑμῶν. λέγω δὲ ὑμῖν, οὐ μὴ ἴδητέ με ἕως ἥξει ὅτε εἴπητε· 
\begin{poetryblock}
\par }{\PP \begin{quote}Εὐλογημένος ὁ ἐρχόμενος ἐν ὀνόματι Κυρίου.\end{quote}
\end{poetryblock}

\par }\Chap{14}{\PP \VerseOne{1}Καὶ ἐγένετο ἐν τῷ ἐλθεῖν αὐτὸν εἰς οἶκόν τινος τῶν ἀρχόντων τῶν Φαρισαίων σαββάτῳ φαγεῖν ἄρτον καὶ αὐτοὶ ἦσαν παρατηρούμενοι αὐτόν.
\par }{\PP \VS{2}καὶ ἰδοὺ ἄνθρωπός τις ἦν ὑδρωπικὸς ἔμπροσθεν αὐτοῦ.
\VS{3}καὶ ἀποκριθεὶς ὁ Ἰησοῦς εἶπεν πρὸς τοὺς νομικοὺς καὶ Φαρισαίους λέγων· Ἔξεστιν τῷ σαββάτῳ θεραπεῦσαι ἢ οὔ;
\VS{4}Οἱ δὲ ἡσύχασαν. Καὶ ἐπιλαβόμενος ἰάσατο αὐτὸν καὶ ἀπέλυσεν.
\VS{5}καὶ πρὸς αὐτοὺς εἶπεν· Τίνος ὑμῶν υἱὸς ἢ βοῦς εἰς φρέαρ πεσεῖται, καὶ οὐκ εὐθέως ἀνασπάσει αὐτὸν ἐν ἡμέρᾳ τοῦ σαββάτου;
\VS{6}Καὶ οὐκ ἴσχυσαν ἀνταποκριθῆναι πρὸς ταῦτα.
\par }{\PP \VS{7}Ἔλεγεν δὲ πρὸς τοὺς κεκλημένους παραβολήν, ἐπέχων πῶς τὰς πρωτοκλισίας ἐξελέγοντο, λέγων πρὸς αὐτούς·
\VS{8}Ὅταν κληθῇς ὑπό τινος εἰς γάμους, μὴ κατακλιθῇς εἰς τὴν πρωτοκλισίαν, μήποτε ἐντιμότερός σου ᾖ κεκλημένος ὑπ᾽ αὐτοῦ,
\VS{9}καὶ ἐλθὼν ὁ σὲ καὶ αὐτὸν καλέσας ἐρεῖ σοι· Δὸς τούτῳ τόπον, καὶ τότε ἄρξῃ μετὰ αἰσχύνης τὸν ἔσχατον τόπον κατέχειν.
\VS{10}Ἀλλ᾽ ὅταν κληθῇς, πορευθεὶς ἀνάπεσε εἰς τὸν ἔσχατον τόπον, ἵνα ὅταν ἔλθῃ ὁ κεκληκώς σε ἐρεῖ σοι· Φίλε, προσανάβηθι ἀνώτερον· τότε ἔσται σοι δόξα ἐνώπιον πάντων τῶν συνανακειμένων σοι.
\VS{11}ὅτι πᾶς ὁ ὑψῶν ἑαυτὸν ταπεινωθήσεται, καὶ ὁ ταπεινῶν ἑαυτὸν ὑψωθήσεται.
\par }{\PP \VS{12}Ἔλεγεν δὲ καὶ τῷ κεκληκότι αὐτόν· Ὅταν ποιῇς ἄριστον ἢ δεῖπνον, μὴ φώνει τοὺς φίλους σου μηδὲ τοὺς ἀδελφούς σου μηδὲ τοὺς συγγενεῖς σου μηδὲ γείτονας πλουσίους, μήποτε καὶ αὐτοὶ ἀντικαλέσωσίν σε καὶ γένηται ἀνταπόδομά σοι.
\VS{13}ἀλλ᾽ ὅταν δοχὴν ποιῇς, κάλει πτωχούς, ἀναπείρους, χωλούς, τυφλούς·
\VS{14}καὶ μακάριος ἔσῃ, ὅτι οὐκ ἔχουσιν ἀνταποδοῦναί σοι, ἀνταποδοθήσεται γάρ σοι ἐν τῇ ἀναστάσει τῶν δικαίων.
\par }{\PP \VS{15}Ἀκούσας δέ τις τῶν συνανακειμένων ταῦτα εἶπεν αὐτῷ· Μακάριος ὅστις φάγεται ἄρτον ἐν τῇ βασιλείᾳ τοῦ Θεοῦ.
\par }{\PP \VS{16}Ὁ δὲ εἶπεν αὐτῷ· Ἄνθρωπός τις ἐποίει δεῖπνον μέγα, καὶ ἐκάλεσεν πολλούς
\VS{17}καὶ ἀπέστειλεν τὸν δοῦλον αὐτοῦ τῇ ὥρᾳ τοῦ δείπνου εἰπεῖν τοῖς κεκλημένοις· Ἔρχεσθε, ὅτι ἤδη ἕτοιμά ἐστιν.
\VS{18}Καὶ ἤρξαντο ἀπὸ μιᾶς πάντες παραιτεῖσθαι. ὁ πρῶτος εἶπεν αὐτῷ· Ἀγρὸν ἠγόρασα καὶ ἔχω ἀνάγκην ἐξελθὼν ἰδεῖν αὐτόν· ἐρωτῶ σε, ἔχε με παρῃτημένον.
\VS{19}Καὶ ἕτερος εἶπεν· Ζεύγη βοῶν ἠγόρασα πέντε καὶ πορεύομαι δοκιμάσαι αὐτά· ἐρωτῶ σε, ἔχε με παρῃτημένον.
\VS{20}Καὶ ἕτερος εἶπεν· Γυναῖκα ἔγημα καὶ διὰ τοῦτο οὐ δύναμαι ἐλθεῖν.
\VS{21}Καὶ παραγενόμενος ὁ δοῦλος ἀπήγγειλεν τῷ κυρίῳ αὐτοῦ ταῦτα. τότε ὀργισθεὶς ὁ οἰκοδεσπότης εἶπεν τῷ δούλῳ αὐτοῦ· Ἔξελθε ταχέως εἰς τὰς πλατείας καὶ ῥύμας τῆς πόλεως καὶ τοὺς πτωχοὺς καὶ ἀναπείρους καὶ τυφλοὺς καὶ χωλοὺς εἰσάγαγε ὧδε.
\VS{22}Καὶ εἶπεν ὁ δοῦλος· Κύριε, γέγονεν ὃ ἐπέταξας, καὶ ἔτι τόπος ἐστίν.
\VS{23}Καὶ εἶπεν ὁ κύριος πρὸς τὸν δοῦλον· Ἔξελθε εἰς τὰς ὁδοὺς καὶ φραγμοὺς καὶ ἀνάγκασον εἰσελθεῖν, ἵνα γεμισθῇ μου ὁ οἶκος·
\VS{24}λέγω γὰρ ὑμῖν ὅτι οὐδεὶς τῶν ἀνδρῶν ἐκείνων τῶν κεκλημένων γεύσεταί μου τοῦ δείπνου.
\par }{\PP \VS{25}Συνεπορεύοντο δὲ αὐτῷ ὄχλοι πολλοί, καὶ στραφεὶς εἶπεν πρὸς αὐτούς·
\VS{26}Εἴ τις ἔρχεται πρός με καὶ οὐ μισεῖ τὸν πατέρα ἑαυτοῦ καὶ τὴν μητέρα καὶ τὴν γυναῖκα καὶ τὰ τέκνα καὶ τοὺς ἀδελφοὺς καὶ τὰς ἀδελφάς ἔτι τε καὶ τὴν ψυχὴν ἑαυτοῦ, οὐ δύναται εἶναί μου μαθητής.
\VS{27}ὅστις οὐ βαστάζει τὸν σταυρὸν ἑαυτοῦ καὶ ἔρχεται ὀπίσω μου, οὐ δύναται εἶναί μου μαθητής.
\par }{\PP \VS{28}Τίς γὰρ ἐξ ὑμῶν θέλων πύργον οἰκοδομῆσαι οὐχὶ πρῶτον καθίσας ψηφίζει τὴν δαπάνην, εἰ ἔχει εἰς ἀπαρτισμόν;
\VS{29}ἵνα μήποτε θέντος αὐτοῦ θεμέλιον καὶ μὴ ἰσχύοντος ἐκτελέσαι πάντες οἱ θεωροῦντες ἄρξωνται αὐτῷ ἐμπαίζειν
\VS{30}λέγοντες ὅτι Οὗτος ὁ ἄνθρωπος ἤρξατο οἰκοδομεῖν καὶ οὐκ ἴσχυσεν ἐκτελέσαι.
\VS{31}Ἢ τίς βασιλεὺς πορευόμενος ἑτέρῳ βασιλεῖ συμβαλεῖν εἰς πόλεμον οὐχὶ καθίσας πρῶτον βουλεύσεται εἰ δυνατός ἐστιν ἐν δέκα χιλιάσιν ὑπαντῆσαι τῷ μετὰ εἴκοσι χιλιάδων ἐρχομένῳ ἐπ᾽ αὐτόν;
\VS{32}εἰ δὲ μή γε, ἔτι αὐτοῦ πόρρω ὄντος πρεσβείαν ἀποστείλας ἐρωτᾷ τὰ πρὸς εἰρήνην.
\VS{33}Οὕτως οὖν πᾶς ἐξ ὑμῶν ὃς οὐκ ἀποτάσσεται πᾶσιν τοῖς ἑαυτοῦ ὑπάρχουσιν οὐ δύναται εἶναί μου μαθητής.
\par }{\PP \VS{34}Καλὸν οὖν τὸ ἅλας· ἐὰν δὲ καὶ τὸ ἅλας μωρανθῇ, ἐν τίνι ἀρτυθήσεται;
\VS{35}οὔτε εἰς γῆν οὔτε εἰς κοπρίαν εὔθετόν ἐστιν, ἔξω βάλλουσιν αὐτό. Ὁ ἔχων ὦτα ἀκούειν ἀκουέτω.

\par }\Chap{15}{\PP \VerseOne{1}Ἦσαν δὲ αὐτῷ ἐγγίζοντες πάντες οἱ τελῶναι καὶ οἱ ἁμαρτωλοὶ ἀκούειν αὐτοῦ.
\VS{2}καὶ διεγόγγυζον οἵ τε Φαρισαῖοι καὶ οἱ γραμματεῖς λέγοντες ὅτι Οὗτος ἁμαρτωλοὺς προσδέχεται καὶ συνεσθίει αὐτοῖς.
\par }{\PP \VS{3}Εἶπεν δὲ πρὸς αὐτοὺς τὴν παραβολὴν ταύτην λέγων·
\VS{4}Τίς ἄνθρωπος ἐξ ὑμῶν ἔχων ἑκατὸν πρόβατα καὶ ἀπολέσας ἐξ αὐτῶν ἓν οὐ καταλείπει τὰ ἐνενήκοντα ἐννέα ἐν τῇ ἐρήμῳ καὶ πορεύεται ἐπὶ τὸ ἀπολωλὸς ἕως εὕρῃ αὐτό;
\VS{5}καὶ εὑρὼν ἐπιτίθησιν ἐπὶ τοὺς ὤμους αὐτοῦ χαίρων
\VS{6}καὶ ἐλθὼν εἰς τὸν οἶκον συνκαλεῖ τοὺς φίλους καὶ τοὺς γείτονας λέγων αὐτοῖς· Συνχάρητέ μοι, ὅτι εὗρον τὸ πρόβατόν μου τὸ ἀπολωλός.
\VS{7}λέγω ὑμῖν ὅτι οὕτως χαρὰ ἐν τῷ οὐρανῷ ἔσται ἐπὶ ἑνὶ ἁμαρτωλῷ μετανοοῦντι ἢ ἐπὶ ἐνενήκοντα ἐννέα δικαίοις οἵτινες οὐ χρείαν ἔχουσιν μετανοίας.
\par }{\PP \VS{8}Ἢ τίς γυνὴ δραχμὰς ἔχουσα δέκα ἐὰν ἀπολέσῃ δραχμὴν μίαν, οὐχὶ ἅπτει λύχνον καὶ σαροῖ τὴν οἰκίαν καὶ ζητεῖ ἐπιμελῶς ἕως οὗ εὕρῃ;
\VS{9}καὶ εὑροῦσα συνκαλεῖ τὰς φίλας καὶ γείτονας λέγουσα· Συνχάρητέ μοι, ὅτι εὗρον τὴν δραχμὴν ἣν ἀπώλεσα.
\VS{10}οὕτως, λέγω ὑμῖν, γίνεται χαρὰ ἐνώπιον τῶν ἀγγέλων τοῦ Θεοῦ ἐπὶ ἑνὶ ἁμαρτωλῷ μετανοοῦντι.
\par }{\PP \VS{11}Εἶπεν δέ· Ἄνθρωπός τις εἶχεν δύο υἱούς.
\VS{12}καὶ εἶπεν ὁ νεώτερος αὐτῶν τῷ πατρί· Πάτερ, δός μοι τὸ ἐπιβάλλον μέρος τῆς οὐσίας. ὁ δὲ διεῖλεν αὐτοῖς τὸν βίον.
\VS{13}Καὶ μετ᾽ οὐ πολλὰς ἡμέρας συναγαγὼν πάντα ὁ νεώτερος υἱὸς ἀπεδήμησεν εἰς χώραν μακράν καὶ ἐκεῖ διεσκόρπισεν τὴν οὐσίαν αὐτοῦ ζῶν ἀσώτως.
\VS{14}Δαπανήσαντος δὲ αὐτοῦ πάντα ἐγένετο λιμὸς ἰσχυρὰ κατὰ τὴν χώραν ἐκείνην, καὶ αὐτὸς ἤρξατο ὑστερεῖσθαι.
\VS{15}καὶ πορευθεὶς ἐκολλήθη ἑνὶ τῶν πολιτῶν τῆς χώρας ἐκείνης, καὶ ἔπεμψεν αὐτὸν εἰς τοὺς ἀγροὺς αὐτοῦ βόσκειν χοίρους,
\VS{16}καὶ ἐπεθύμει χορτασθῆναι ἐκ τῶν κερατίων ὧν ἤσθιον οἱ χοῖροι, καὶ οὐδεὶς ἐδίδου αὐτῷ.
\VS{17}Εἰς ἑαυτὸν δὲ ἐλθὼν ἔφη· Πόσοι μίσθιοι τοῦ πατρός μου περισσεύονται ἄρτων, ἐγὼ δὲ λιμῷ ὧδε ἀπόλλυμαι.
\VS{18}ἀναστὰς πορεύσομαι πρὸς τὸν πατέρα μου καὶ ἐρῶ αὐτῷ· Πάτερ, ἥμαρτον εἰς τὸν οὐρανὸν καὶ ἐνώπιόν σου,
\VS{19}οὐκέτι εἰμὶ ἄξιος κληθῆναι υἱός σου· ποίησόν με ὡς ἕνα τῶν μισθίων σου.
\VS{20}Καὶ ἀναστὰς ἦλθεν πρὸς τὸν πατέρα ἑαυτοῦ. ἔτι δὲ αὐτοῦ μακρὰν ἀπέχοντος εἶδεν αὐτὸν ὁ πατὴρ αὐτοῦ καὶ ἐσπλαγχνίσθη καὶ δραμὼν ἐπέπεσεν ἐπὶ τὸν τράχηλον αὐτοῦ καὶ κατεφίλησεν αὐτόν.
\VS{21}Εἶπεν δὲ ὁ υἱὸς αὐτῷ· Πάτερ, ἥμαρτον εἰς τὸν οὐρανὸν καὶ ἐνώπιόν σου, οὐκέτι εἰμὶ ἄξιος κληθῆναι υἱός σου.
\VS{22}Εἶπεν δὲ ὁ πατὴρ πρὸς τοὺς δούλους αὐτοῦ· Ταχὺ ἐξενέγκατε στολὴν τὴν πρώτην καὶ ἐνδύσατε αὐτόν, καὶ δότε δακτύλιον εἰς τὴν χεῖρα αὐτοῦ καὶ ὑποδήματα εἰς τοὺς πόδας,
\VS{23}καὶ φέρετε τὸν μόσχον τὸν σιτευτόν, θύσατε, καὶ φαγόντες εὐφρανθῶμεν,
\VS{24}ὅτι οὗτος ὁ υἱός μου νεκρὸς ἦν καὶ ἀνέζησεν, ἦν ἀπολωλὼς καὶ εὑρέθη. καὶ ἤρξαντο εὐφραίνεσθαι.
\VS{25}Ἦν δὲ ὁ υἱὸς αὐτοῦ ὁ πρεσβύτερος ἐν ἀγρῷ· καὶ ὡς ἐρχόμενος ἤγγισεν τῇ οἰκίᾳ, ἤκουσεν συμφωνίας καὶ χορῶν,
\VS{26}καὶ προσκαλεσάμενος ἕνα τῶν παίδων ἐπυνθάνετο τί ἂν εἴη ταῦτα.
\VS{27}Ὁ δὲ εἶπεν αὐτῷ ὅτι Ὁ ἀδελφός σου ἥκει, καὶ ἔθυσεν ὁ πατήρ σου τὸν μόσχον τὸν σιτευτόν, ὅτι ὑγιαίνοντα αὐτὸν ἀπέλαβεν.
\VS{28}Ὠργίσθη δὲ καὶ οὐκ ἤθελεν εἰσελθεῖν, ὁ δὲ πατὴρ αὐτοῦ ἐξελθὼν παρεκάλει αὐτόν.
\VS{29}Ὁ δὲ ἀποκριθεὶς εἶπεν τῷ πατρὶ αὐτοῦ· Ἰδοὺ τοσαῦτα ἔτη δουλεύω σοι καὶ οὐδέποτε ἐντολήν σου παρῆλθον, καὶ ἐμοὶ οὐδέποτε ἔδωκας ἔριφον ἵνα μετὰ τῶν φίλων μου εὐφρανθῶ·
\VS{30}ὅτε δὲ ὁ υἱός σου οὗτος ὁ καταφαγών σου τὸν βίον μετὰ πορνῶν ἦλθεν, ἔθυσας αὐτῷ τὸν σιτευτὸν μόσχον.
\VS{31}Ὁ δὲ εἶπεν αὐτῷ· Τέκνον, σὺ πάντοτε μετ᾽ ἐμοῦ εἶ, καὶ πάντα τὰ ἐμὰ σά ἐστιν·
\VS{32}εὐφρανθῆναι δὲ καὶ χαρῆναι ἔδει, ὅτι ὁ ἀδελφός σου οὗτος νεκρὸς ἦν καὶ ἔζησεν, καὶ ἀπολωλὼς καὶ εὑρέθη.

\par }\Chap{16}{\PP \VerseOne{1}Ἔλεγεν δὲ καὶ πρὸς τοὺς μαθητάς· Ἄνθρωπός τις ἦν πλούσιος ὃς εἶχεν οἰκονόμον, καὶ οὗτος διεβλήθη αὐτῷ ὡς διασκορπίζων τὰ ὑπάρχοντα αὐτοῦ.
\VS{2}καὶ φωνήσας αὐτὸν εἶπεν αὐτῷ· Τί τοῦτο ἀκούω περὶ σοῦ; ἀπόδος τὸν λόγον τῆς οἰκονομίας σου, οὐ γὰρ δύνῃ ἔτι οἰκονομεῖν.
\VS{3}Εἶπεν δὲ ἐν ἑαυτῷ ὁ οἰκονόμος· Τί ποιήσω, ὅτι ὁ κύριός μου ἀφαιρεῖται τὴν οἰκονομίαν ἀπ᾽ ἐμοῦ; σκάπτειν οὐκ ἰσχύω, ἐπαιτεῖν αἰσχύνομαι.
\VS{4}ἔγνων τί ποιήσω, ἵνα ὅταν μετασταθῶ ἐκ τῆς οἰκονομίας δέξωνταί με εἰς τοὺς οἴκους αὐτῶν.
\VS{5}Καὶ προσκαλεσάμενος ἕνα ἕκαστον τῶν χρεοφειλετῶν τοῦ κυρίου ἑαυτοῦ ἔλεγεν τῷ πρώτῳ· Πόσον ὀφείλεις τῷ κυρίῳ μου;
\VS{6}Ὁ δὲ εἶπεν· Ἑκατὸν βάτους ἐλαίου. Ὁ δὲ εἶπεν αὐτῷ· Δέξαι σου τὰ γράμματα καὶ καθίσας ταχέως γράψον πεντήκοντα.
\VS{7}Ἔπειτα ἑτέρῳ εἶπεν· Σὺ δὲ πόσον ὀφείλεις; Ὁ δὲ εἶπεν· Ἑκατὸν κόρους σίτου. Λέγει αὐτῷ· Δέξαι σου τὰ γράμματα καὶ γράψον ὀγδοήκοντα.
\VS{8}Καὶ ἐπῄνεσεν ὁ κύριος τὸν οἰκονόμον τῆς ἀδικίας ὅτι φρονίμως ἐποίησεν· ὅτι οἱ υἱοὶ τοῦ αἰῶνος τούτου φρονιμώτεροι ὑπὲρ τοὺς υἱοὺς τοῦ φωτὸς εἰς τὴν γενεὰν τὴν ἑαυτῶν εἰσιν.
\VS{9}Καὶ ἐγὼ ὑμῖν λέγω, ἑαυτοῖς ποιήσατε φίλους ἐκ τοῦ μαμωνᾶ τῆς ἀδικίας, ἵνα ὅταν ἐκλίπῃ δέξωνται ὑμᾶς εἰς τὰς αἰωνίους σκηνάς.
\par }{\PP \VS{10}Ὁ πιστὸς ἐν ἐλαχίστῳ καὶ ἐν πολλῷ πιστός ἐστιν, καὶ ὁ ἐν ἐλαχίστῳ ἄδικος καὶ ἐν πολλῷ ἄδικός ἐστιν.
\VS{11}εἰ οὖν ἐν τῷ ἀδίκῳ μαμωνᾷ πιστοὶ οὐκ ἐγένεσθε, τὸ ἀληθινὸν τίς ὑμῖν πιστεύσει;
\VS{12}καὶ εἰ ἐν τῷ ἀλλοτρίῳ πιστοὶ οὐκ ἐγένεσθε, τὸ ὑμέτερον τίς ὑμῖν δώσει;
\par }{\PP \VS{13}Οὐδεὶς οἰκέτης δύναται δυσὶ κυρίοις δουλεύειν· ἢ γὰρ τὸν ἕνα μισήσει καὶ τὸν ἕτερον ἀγαπήσει, ἢ ἑνὸς ἀνθέξεται καὶ τοῦ ἑτέρου καταφρονήσει. οὐ δύνασθε Θεῷ δουλεύειν καὶ μαμωνᾷ.
\par }{\PP \VS{14}Ἤκουον δὲ ταῦτα πάντα οἱ Φαρισαῖοι φιλάργυροι ὑπάρχοντες καὶ ἐξεμυκτήριζον αὐτόν.
\VS{15}καὶ εἶπεν αὐτοῖς· Ὑμεῖς ἐστε οἱ δικαιοῦντες ἑαυτοὺς ἐνώπιον τῶν ἀνθρώπων, ὁ δὲ Θεὸς γινώσκει τὰς καρδίας ὑμῶν· ὅτι τὸ ἐν ἀνθρώποις ὑψηλὸν βδέλυγμα ἐνώπιον τοῦ Θεοῦ.
\par }{\PP \VS{16}Ὁ νόμος καὶ οἱ προφῆται μέχρι Ἰωάννου· ἀπὸ τότε ἡ βασιλεία τοῦ Θεοῦ εὐαγγελίζεται καὶ πᾶς εἰς αὐτὴν βιάζεται.
\VS{17}εὐκοπώτερον δέ ἐστιν τὸν οὐρανὸν καὶ τὴν γῆν παρελθεῖν ἢ τοῦ νόμου μίαν κεραίαν πεσεῖν.
\par }{\PP \VS{18}Πᾶς ὁ ἀπολύων τὴν γυναῖκα αὐτοῦ καὶ γαμῶν ἑτέραν μοιχεύει, καὶ ὁ ἀπολελυμένην ἀπὸ ἀνδρὸς γαμῶν μοιχεύει.
\par }{\PP \VS{19}Ἄνθρωπος δέ τις ἦν πλούσιος, καὶ ἐνεδιδύσκετο πορφύραν καὶ βύσσον εὐφραινόμενος καθ᾽ ἡμέραν λαμπρῶς.
\VS{20}πτωχὸς δέ τις ὀνόματι Λάζαρος ἐβέβλητο πρὸς τὸν πυλῶνα αὐτοῦ εἱλκωμένος
\VS{21}καὶ ἐπιθυμῶν χορτασθῆναι ἀπὸ τῶν πιπτόντων ἀπὸ τῆς τραπέζης τοῦ πλουσίου· ἀλλὰ καὶ οἱ κύνες ἐρχόμενοι ἐπέλειχον τὰ ἕλκη αὐτοῦ.
\VS{22}Ἐγένετο δὲ ἀποθανεῖν τὸν πτωχὸν καὶ ἀπενεχθῆναι αὐτὸν ὑπὸ τῶν ἀγγέλων εἰς τὸν κόλπον Ἀβραάμ· ἀπέθανεν δὲ καὶ ὁ πλούσιος καὶ ἐτάφη.
\VS{23}καὶ ἐν τῷ ᾅδῃ ἐπάρας τοὺς ὀφθαλμοὺς αὐτοῦ, ὑπάρχων ἐν βασάνοις, ὁρᾷ Ἀβραὰμ ἀπὸ μακρόθεν καὶ Λάζαρον ἐν τοῖς κόλποις αὐτοῦ.
\VS{24}Καὶ αὐτὸς φωνήσας εἶπεν· Πάτερ Ἀβραάμ, ἐλέησόν με καὶ πέμψον Λάζαρον ἵνα βάψῃ τὸ ἄκρον τοῦ δακτύλου αὐτοῦ ὕδατος καὶ καταψύξῃ τὴν γλῶσσάν μου, ὅτι ὀδυνῶμαι ἐν τῇ φλογὶ ταύτῃ.
\VS{25}Εἶπεν δὲ Ἀβραάμ· Τέκνον, μνήσθητι ὅτι ἀπέλαβες τὰ ἀγαθά σου ἐν τῇ ζωῇ σου, καὶ Λάζαρος ὁμοίως τὰ κακά· νῦν δὲ ὧδε παρακαλεῖται, σὺ δὲ ὀδυνᾶσαι.
\VS{26}καὶ ἐν πᾶσι τούτοις μεταξὺ ἡμῶν καὶ ὑμῶν χάσμα μέγα ἐστήρικται, ὅπως οἱ θέλοντες διαβῆναι ἔνθεν πρὸς ὑμᾶς μὴ δύνωνται, μηδὲ ἐκεῖθεν πρὸς ἡμᾶς διαπερῶσιν.
\VS{27}Εἶπεν δέ· Ἐρωτῶ σε οὖν, πάτερ, ἵνα πέμψῃς αὐτὸν εἰς τὸν οἶκον τοῦ πατρός μου,
\VS{28}ἔχω γὰρ πέντε ἀδελφούς, ὅπως διαμαρτύρηται αὐτοῖς, ἵνα μὴ καὶ αὐτοὶ ἔλθωσιν εἰς τὸν τόπον τοῦτον τῆς βασάνου.
\VS{29}Λέγει δὲ Ἀβραάμ· Ἔχουσι Μωϋσέα καὶ τοὺς προφήτας· ἀκουσάτωσαν αὐτῶν.
\VS{30}Ὁ δὲ εἶπεν· Οὐχί, πάτερ Ἀβραάμ, ἀλλ᾽ ἐάν τις ἀπὸ νεκρῶν πορευθῇ πρὸς αὐτοὺς μετανοήσουσιν.
\VS{31}Εἶπεν δὲ αὐτῷ· Εἰ Μωϋσέως καὶ τῶν προφητῶν οὐκ ἀκούουσιν, οὐδ᾽ ἐάν τις ἐκ νεκρῶν ἀναστῇ πεισθήσονται.

\par }\Chap{17}{\PP \VerseOne{1}Εἶπεν δὲ πρὸς τοὺς μαθητὰς αὐτοῦ· Ἀνένδεκτόν ἐστιν τοῦ τὰ σκάνδαλα μὴ ἐλθεῖν, πλὴν οὐαὶ δι᾽ οὗ ἔρχεται·
\VS{2}λυσιτελεῖ αὐτῷ εἰ λίθος μυλικὸς περίκειται περὶ τὸν τράχηλον αὐτοῦ καὶ ἔρριπται εἰς τὴν θάλασσαν ἢ ἵνα σκανδαλίσῃ τῶν μικρῶν τούτων ἕνα.
\VS{3}Προσέχετε ἑαυτοῖς.
\par }{\PP ἐὰν ἁμάρτῃ ὁ ἀδελφός σου ἐπιτίμησον αὐτῷ, καὶ ἐὰν μετανοήσῃ ἄφες αὐτῷ.
\VS{4}καὶ ἐὰν ἑπτάκις τῆς ἡμέρας ἁμαρτήσῃ εἰς σὲ καὶ ἑπτάκις ἐπιστρέψῃ πρὸς σὲ λέγων· Μετανοῶ, ἀφήσεις αὐτῷ.
\par }{\PP \VS{5}Καὶ εἶπαν οἱ ἀπόστολοι τῷ Κυρίῳ· Πρόσθες ἡμῖν πίστιν.
\VS{6}Εἶπεν δὲ ὁ Κύριος· Εἰ ἔχετε πίστιν ὡς κόκκον σινάπεως, ἐλέγετε ἂν τῇ συκαμίνῳ ταύτῃ· Ἐκριζώθητι καὶ φυτεύθητι ἐν τῇ θαλάσσῃ· καὶ ὑπήκουσεν ἂν ὑμῖν.
\par }{\PP \VS{7}Τίς δὲ ἐξ ὑμῶν δοῦλον ἔχων ἀροτριῶντα ἢ ποιμαίνοντα, ὃς εἰσελθόντι ἐκ τοῦ ἀγροῦ ἐρεῖ αὐτῷ· Εὐθέως παρελθὼν ἀνάπεσε,
\VS{8}ἀλλ᾽ οὐχὶ ἐρεῖ αὐτῷ· Ἑτοίμασον τί δειπνήσω καὶ περιζωσάμενος διακόνει μοι ἕως φάγω καὶ πίω, καὶ μετὰ ταῦτα φάγεσαι καὶ πίεσαι σύ;
\VS{9}μὴ ἔχει χάριν τῷ δούλῳ ὅτι ἐποίησεν τὰ διαταχθέντα;
\VS{10}οὕτως καὶ ὑμεῖς, ὅταν ποιήσητε πάντα τὰ διαταχθέντα ὑμῖν, λέγετε ὅτι Δοῦλοι ἀχρεῖοί ἐσμεν, ὃ ὠφείλομεν ποιῆσαι πεποιήκαμεν.
\par }{\PP \VS{11}Καὶ ἐγένετο ἐν τῷ πορεύεσθαι εἰς Ἰερουσαλὴμ καὶ αὐτὸς διήρχετο διὰ μέσον Σαμαρείας καὶ Γαλιλαίας.
\par }{\PP \VS{12}καὶ εἰσερχομένου αὐτοῦ εἴς τινα κώμην ἀπήντησαν αὐτῷ δέκα λεπροὶ ἄνδρες, οἳ ἔστησαν πόρρωθεν
\VS{13}καὶ αὐτοὶ ἦραν φωνὴν λέγοντες· Ἰησοῦ Ἐπιστάτα, ἐλέησον ἡμᾶς.
\VS{14}Καὶ ἰδὼν εἶπεν αὐτοῖς· Πορευθέντες ἐπιδείξατε ἑαυτοὺς τοῖς ἱερεῦσιν. καὶ ἐγένετο ἐν τῷ ὑπάγειν αὐτοὺς ἐκαθαρίσθησαν.
\VS{15}Εἷς δὲ ἐξ αὐτῶν, ἰδὼν ὅτι ἰάθη, ὑπέστρεψεν μετὰ φωνῆς μεγάλης δοξάζων τὸν Θεόν,
\VS{16}καὶ ἔπεσεν ἐπὶ πρόσωπον παρὰ τοὺς πόδας αὐτοῦ εὐχαριστῶν αὐτῷ· καὶ αὐτὸς ἦν Σαμαρίτης.
\VS{17}Ἀποκριθεὶς δὲ ὁ Ἰησοῦς εἶπεν· Οὐχὶ οἱ δέκα ἐκαθαρίσθησαν; οἱ δὲ ἐννέα ποῦ;
\VS{18}οὐχ εὑρέθησαν ὑποστρέψαντες δοῦναι δόξαν τῷ Θεῷ εἰ μὴ ὁ ἀλλογενὴς οὗτος;
\VS{19}καὶ εἶπεν αὐτῷ· Ἀναστὰς πορεύου· ἡ πίστις σου σέσωκέν σε.
\par }{\PP \VS{20}Ἐπερωτηθεὶς δὲ ὑπὸ τῶν Φαρισαίων πότε ἔρχεται ἡ βασιλεία τοῦ Θεοῦ ἀπεκρίθη αὐτοῖς καὶ εἶπεν· Οὐκ ἔρχεται ἡ βασιλεία τοῦ Θεοῦ μετὰ παρατηρήσεως,
\VS{21}οὐδὲ ἐροῦσιν· Ἰδοὺ ὧδε ἤ· Ἐκεῖ, ἰδοὺ γὰρ ἡ βασιλεία τοῦ Θεοῦ ἐντὸς ὑμῶν ἐστιν.
\par }{\PP \VS{22}Εἶπεν δὲ πρὸς τοὺς μαθητάς· Ἐλεύσονται ἡμέραι ὅτε ἐπιθυμήσετε μίαν τῶν ἡμερῶν τοῦ Υἱοῦ τοῦ ἀνθρώπου ἰδεῖν καὶ οὐκ ὄψεσθε.
\VS{23}καὶ ἐροῦσιν ὑμῖν· Ἰδοὺ ἐκεῖ, ἤ· Ἰδοὺ ὧδε· μὴ ἀπέλθητε μηδὲ διώξητε.
\VS{24}ὥσπερ γὰρ ἡ ἀστραπὴ ἀστράπτουσα ἐκ τῆς ὑπὸ τὸν οὐρανὸν εἰς τὴν ὑπ᾽ οὐρανὸν λάμπει, οὕτως ἔσται ὁ Υἱὸς τοῦ ἀνθρώπου ἐν τῇ ἡμέρᾳ αὐτοῦ.
\VS{25}πρῶτον δὲ δεῖ αὐτὸν πολλὰ παθεῖν καὶ ἀποδοκιμασθῆναι ἀπὸ τῆς γενεᾶς ταύτης.
\VS{26}Καὶ καθὼς ἐγένετο ἐν ταῖς ἡμέραις Νῶε, οὕτως ἔσται καὶ ἐν ταῖς ἡμέραις τοῦ Υἱοῦ τοῦ ἀνθρώπου·
\VS{27}ἤσθιον, ἔπινον, ἐγάμουν, ἐγαμίζοντο, ἄχρι ἧς ἡμέρας εἰσῆλθεν Νῶε εἰς τὴν κιβωτόν καὶ ἦλθεν ὁ κατακλυσμὸς καὶ ἀπώλεσεν πάντας.
\VS{28}Ὁμοίως καθὼς ἐγένετο ἐν ταῖς ἡμέραις Λώτ· ἤσθιον, ἔπινον, ἠγόραζον, ἐπώλουν, ἐφύτευον, ᾠκοδόμουν·
\VS{29}ᾗ δὲ ἡμέρᾳ ἐξῆλθεν Λὼτ ἀπὸ Σοδόμων, ἔβρεξεν πῦρ καὶ θεῖον ἀπ᾽ οὐρανοῦ καὶ ἀπώλεσεν πάντας.
\VS{30}Κατὰ τὰ αὐτὰ ἔσται ᾗ ἡμέρᾳ ὁ Υἱὸς τοῦ ἀνθρώπου ἀποκαλύπτεται.
\VS{31}ἐν ἐκείνῃ τῇ ἡμέρᾳ ὃς ἔσται ἐπὶ τοῦ δώματος καὶ τὰ σκεύη αὐτοῦ ἐν τῇ οἰκίᾳ, μὴ καταβάτω ἆραι αὐτά, καὶ ὁ ἐν ἀγρῷ ὁμοίως μὴ ἐπιστρεψάτω εἰς τὰ ὀπίσω.
\VS{32}μνημονεύετε τῆς γυναικὸς Λώτ.
\VS{33}ὃς ἐὰν ζητήσῃ τὴν ψυχὴν αὐτοῦ περιποιήσασθαι ἀπολέσει αὐτήν, ὃς δ᾽ ἂν ἀπολέσῃ ζωογονήσει αὐτήν.
\VS{34}λέγω ὑμῖν, ταύτῃ τῇ νυκτὶ ἔσονται δύο ἐπὶ κλίνης μιᾶς, ὁ εἷς παραλημφθήσεται καὶ ὁ ἕτερος ἀφεθήσεται·
\VS{35}ἔσονται δύο ἀλήθουσαι ἐπὶ τὸ αὐτό, ἡ μία παραλημφθήσεται, ἡ δὲ ἑτέρα ἀφεθήσεται.
\VS{37}Καὶ ἀποκριθέντες λέγουσιν αὐτῷ· Ποῦ, Κύριε; Ὁ δὲ εἶπεν αὐτοῖς· Ὅπου τὸ σῶμα, ἐκεῖ καὶ οἱ ἀετοὶ ἐπισυναχθήσονται.

\par }\Chap{18}{\PP \VerseOne{1}Ἔλεγεν δὲ παραβολὴν αὐτοῖς πρὸς τὸ δεῖν πάντοτε προσεύχεσθαι αὐτοὺς καὶ μὴ ἐνκακεῖν,
\VS{2}λέγων· Κριτής τις ἦν ἔν τινι πόλει τὸν Θεὸν μὴ φοβούμενος καὶ ἄνθρωπον μὴ ἐντρεπόμενος.
\VS{3}χήρα δὲ ἦν ἐν τῇ πόλει ἐκείνῃ καὶ ἤρχετο πρὸς αὐτὸν λέγουσα· Ἐκδίκησόν με ἀπὸ τοῦ ἀντιδίκου μου.
\VS{4}Καὶ οὐκ ἤθελεν ἐπὶ χρόνον. μετὰ ταῦτα2 δὲ1 εἶπεν ἐν ἑαυτῷ· Εἰ καὶ τὸν Θεὸν οὐ φοβοῦμαι οὐδὲ ἄνθρωπον ἐντρέπομαι,
\VS{5}διά γε τὸ παρέχειν μοι κόπον τὴν χήραν ταύτην ἐκδικήσω αὐτήν, ἵνα μὴ εἰς τέλος ἐρχομένη ὑπωπιάζῃ με.
\VS{6}Εἶπεν δὲ ὁ Κύριος· Ἀκούσατε τί ὁ κριτὴς τῆς ἀδικίας λέγει·
\VS{7}ὁ δὲ Θεὸς οὐ μὴ ποιήσῃ τὴν ἐκδίκησιν τῶν ἐκλεκτῶν αὐτοῦ τῶν βοώντων αὐτῷ ἡμέρας καὶ νυκτός, καὶ μακροθυμεῖ ἐπ᾽ αὐτοῖς;
\VS{8}λέγω ὑμῖν ὅτι ποιήσει τὴν ἐκδίκησιν αὐτῶν ἐν τάχει. πλὴν ὁ Υἱὸς τοῦ ἀνθρώπου ἐλθὼν ἆρα εὑρήσει τὴν πίστιν ἐπὶ τῆς γῆς;
\par }{\PP \VS{9}Εἶπεν δὲ καὶ πρός τινας τοὺς πεποιθότας ἐφ᾽ ἑαυτοῖς ὅτι εἰσὶν δίκαιοι καὶ ἐξουθενοῦντας τοὺς λοιποὺς τὴν παραβολὴν ταύτην·
\VS{10}Ἄνθρωποι δύο ἀνέβησαν εἰς τὸ ἱερὸν προσεύξασθαι, ὁ εἷς Φαρισαῖος καὶ ὁ ἕτερος τελώνης.
\VS{11}ὁ Φαρισαῖος σταθεὶς πρὸς ἑαυτὸν ταῦτα προσηύχετο· Ὁ Θεός, εὐχαριστῶ σοι ὅτι οὐκ εἰμὶ ὥσπερ οἱ λοιποὶ τῶν ἀνθρώπων, ἅρπαγες, ἄδικοι, μοιχοί, ἢ καὶ ὡς οὗτος ὁ τελώνης·
\VS{12}νηστεύω δὶς τοῦ σαββάτου, ἀποδεκατῶ πάντα ὅσα κτῶμαι.
\VS{13}Ὁ δὲ τελώνης μακρόθεν ἑστὼς οὐκ ἤθελεν οὐδὲ τοὺς ὀφθαλμοὺς ἐπᾶραι εἰς τὸν οὐρανόν, ἀλλ᾽ ἔτυπτεν τὸ στῆθος αὐτοῦ λέγων· Ὁ Θεός, ἱλάσθητί μοι τῷ ἁμαρτωλῷ.
\VS{14}λέγω ὑμῖν, κατέβη οὗτος δεδικαιωμένος εἰς τὸν οἶκον αὐτοῦ παρ᾽ ἐκεῖνον· ὅτι πᾶς ὁ ὑψῶν ἑαυτὸν ταπεινωθήσεται, ὁ δὲ ταπεινῶν ἑαυτὸν ὑψωθήσεται.
\par }{\PP \VS{15}Προσέφερον δὲ αὐτῷ καὶ τὰ βρέφη ἵνα αὐτῶν ἅπτηται· ἰδόντες δὲ οἱ μαθηταὶ ἐπετίμων αὐτοῖς.
\VS{16}Ὁ δὲ Ἰησοῦς προσεκαλέσατο αὐτὰ λέγων· Ἄφετε τὰ παιδία ἔρχεσθαι πρός με καὶ μὴ κωλύετε αὐτά, τῶν γὰρ τοιούτων ἐστὶν ἡ βασιλεία τοῦ Θεοῦ.
\VS{17}ἀμὴν λέγω ὑμῖν, ὃς ἂν μὴ δέξηται τὴν βασιλείαν τοῦ Θεοῦ ὡς παιδίον, οὐ μὴ εἰσέλθῃ εἰς αὐτήν.
\par }{\PP \VS{18}Καὶ ἐπηρώτησέν τις αὐτὸν ἄρχων λέγων· Διδάσκαλε ἀγαθέ, τί ποιήσας ζωὴν αἰώνιον κληρονομήσω;
\VS{19}Εἶπεν δὲ αὐτῷ ὁ Ἰησοῦς· Τί με λέγεις ἀγαθόν; οὐδεὶς ἀγαθὸς εἰ μὴ εἷς ὁ Θεός.
\VS{20}τὰς ἐντολὰς οἶδας· Μὴ μοιχεύσῃς, Μὴ φονεύσῃς, Μὴ κλέψῃς, Μὴ ψευδομαρτυρήσῃς, Τίμα τὸν πατέρα σου καὶ τὴν μητέρα.
\VS{21}Ὁ δὲ εἶπεν· Ταῦτα πάντα ἐφύλαξα ἐκ νεότητος.
\VS{22}Ἀκούσας δὲ ὁ Ἰησοῦς εἶπεν αὐτῷ· Ἔτι ἕν σοι λείπει· πάντα ὅσα ἔχεις πώλησον καὶ διάδος πτωχοῖς, καὶ ἕξεις θησαυρὸν ἐν τοῖς οὐρανοῖς, καὶ δεῦρο ἀκολούθει μοι.
\VS{23}Ὁ δὲ ἀκούσας ταῦτα περίλυπος ἐγενήθη· ἦν γὰρ πλούσιος σφόδρα.
\par }{\PP \VS{24}Ἰδὼν δὲ αὐτὸν ὁ Ἰησοῦς περίλυπον γενόμενον εἶπεν· Πῶς δυσκόλως οἱ τὰ χρήματα ἔχοντες εἰς τὴν βασιλείαν τοῦ Θεοῦ εἰσπορεύονται·
\VS{25}εὐκοπώτερον γάρ ἐστιν κάμηλον διὰ τρήματος βελόνης εἰσελθεῖν ἢ πλούσιον εἰς τὴν βασιλείαν τοῦ Θεοῦ εἰσελθεῖν.
\VS{26}Εἶπαν δὲ οἱ ἀκούσαντες· Καὶ τίς δύναται σωθῆναι;
\VS{27}Ὁ δὲ εἶπεν· Τὰ ἀδύνατα παρὰ ἀνθρώποις δυνατὰ παρὰ τῷ Θεῷ ἐστιν.
\par }{\PP \VS{28}Εἶπεν δὲ ὁ Πέτρος· Ἰδοὺ ἡμεῖς ἀφέντες τὰ ἴδια ἠκολουθήσαμέν σοι.
\VS{29}Ὁ δὲ εἶπεν αὐτοῖς· Ἀμὴν λέγω ὑμῖν ὅτι οὐδείς ἐστιν ὃς ἀφῆκεν οἰκίαν ἢ γυναῖκα ἢ ἀδελφοὺς ἢ γονεῖς ἢ τέκνα ἕνεκεν τῆς βασιλείας τοῦ Θεοῦ,
\VS{30}ὃς οὐχὶ μὴ ἀπολάβῃ πολλαπλασίονα ἐν τῷ καιρῷ τούτῳ καὶ ἐν τῷ αἰῶνι τῷ ἐρχομένῳ ζωὴν αἰώνιον.
\par }{\PP \VS{31}Παραλαβὼν δὲ τοὺς δώδεκα εἶπεν πρὸς αὐτούς· Ἰδοὺ ἀναβαίνομεν εἰς Ἰερουσαλήμ, καὶ τελεσθήσεται πάντα τὰ γεγραμμένα διὰ τῶν προφητῶν τῷ Υἱῷ τοῦ ἀνθρώπου·
\VS{32}παραδοθήσεται γὰρ τοῖς ἔθνεσιν καὶ ἐμπαιχθήσεται καὶ ὑβρισθήσεται καὶ ἐμπτυσθήσεται
\VS{33}καὶ μαστιγώσαντες ἀποκτενοῦσιν αὐτόν, καὶ τῇ ἡμέρᾳ τῇ τρίτῃ ἀναστήσεται.
\VS{34}Καὶ αὐτοὶ οὐδὲν τούτων συνῆκαν καὶ ἦν τὸ ῥῆμα τοῦτο κεκρυμμένον ἀπ᾽ αὐτῶν καὶ οὐκ ἐγίνωσκον τὰ λεγόμενα.
\par }{\PP \VS{35}Ἐγένετο δὲ ἐν τῷ ἐγγίζειν αὐτὸν εἰς Ἰεριχὼ τυφλός τις ἐκάθητο παρὰ τὴν ὁδὸν ἐπαιτῶν.
\VS{36}ἀκούσας δὲ ὄχλου διαπορευομένου ἐπυνθάνετο τί εἴη τοῦτο.
\VS{37}Ἀπήγγειλαν δὲ αὐτῷ ὅτι Ἰησοῦς ὁ Ναζωραῖος παρέρχεται.
\VS{38}Καὶ ἐβόησεν λέγων· Ἰησοῦ υἱὲ Δαυίδ, ἐλέησόν με.
\VS{39}Καὶ οἱ προάγοντες ἐπετίμων αὐτῷ ἵνα σιγήσῃ, αὐτὸς δὲ πολλῷ μᾶλλον ἔκραζεν· Υἱὲ Δαυίδ, ἐλέησόν με.
\VS{40}Σταθεὶς δὲ ὁ Ἰησοῦς ἐκέλευσεν αὐτὸν ἀχθῆναι πρὸς αὐτόν. ἐγγίσαντος δὲ αὐτοῦ ἐπηρώτησεν αὐτόν·
\VS{41}Τί σοι θέλεις ποιήσω; Ὁ δὲ εἶπεν· Κύριε, ἵνα ἀναβλέψω.
\VS{42}Καὶ ὁ Ἰησοῦς εἶπεν αὐτῷ· Ἀνάβλεψον· ἡ πίστις σου σέσωκέν σε.
\VS{43}καὶ παραχρῆμα ἀνέβλεψεν καὶ ἠκολούθει αὐτῷ δοξάζων τὸν Θεόν. καὶ πᾶς ὁ λαὸς ἰδὼν ἔδωκεν αἶνον τῷ Θεῷ.

\par }\Chap{19}{\PP \VerseOne{1}Καὶ εἰσελθὼν διήρχετο τὴν Ἰεριχώ.
\VS{2}Καὶ ἰδοὺ ἀνὴρ ὀνόματι καλούμενος Ζακχαῖος, καὶ αὐτὸς ἦν ἀρχιτελώνης καὶ αὐτὸς πλούσιος·
\VS{3}καὶ ἐζήτει ἰδεῖν τὸν Ἰησοῦν τίς ἐστιν καὶ οὐκ ἠδύνατο ἀπὸ τοῦ ὄχλου, ὅτι τῇ ἡλικίᾳ μικρὸς ἦν.
\VS{4}καὶ προδραμὼν εἰς τὸ ἔμπροσθεν ἀνέβη ἐπὶ συκομορέαν ἵνα ἴδῃ αὐτόν ὅτι ἐκείνης ἤμελλεν διέρχεσθαι.
\VS{5}Καὶ ὡς ἦλθεν ἐπὶ τὸν τόπον, ἀναβλέψας ὁ Ἰησοῦς εἶπεν πρὸς αὐτόν· Ζακχαῖε, σπεύσας κατάβηθι, σήμερον γὰρ ἐν τῷ οἴκῳ σου δεῖ με μεῖναι.
\VS{6}Καὶ σπεύσας κατέβη καὶ ὑπεδέξατο αὐτὸν χαίρων.
\VS{7}καὶ ἰδόντες πάντες διεγόγγυζον λέγοντες ὅτι Παρὰ ἁμαρτωλῷ ἀνδρὶ εἰσῆλθεν καταλῦσαι.
\VS{8}Σταθεὶς δὲ Ζακχαῖος εἶπεν πρὸς τὸν Κύριον· Ἰδοὺ τὰ ἡμίσιά μου τῶν ὑπαρχόντων, Κύριε, τοῖς πτωχοῖς δίδωμι, καὶ εἴ τινός τι ἐσυκοφάντησα ἀποδίδωμι τετραπλοῦν.
\VS{9}Εἶπεν δὲ πρὸς αὐτὸν ὁ Ἰησοῦς ὅτι Σήμερον σωτηρία τῷ οἴκῳ τούτῳ ἐγένετο, καθότι καὶ αὐτὸς υἱὸς Ἀβραάμ ἐστιν·
\VS{10}ἦλθεν γὰρ ὁ Υἱὸς τοῦ ἀνθρώπου ζητῆσαι καὶ σῶσαι τὸ ἀπολωλός.
\par }{\PP \VS{11}Ἀκουόντων δὲ αὐτῶν ταῦτα προσθεὶς εἶπεν παραβολὴν διὰ τὸ ἐγγὺς εἶναι Ἰερουσαλὴμ αὐτὸν καὶ δοκεῖν αὐτοὺς ὅτι παραχρῆμα μέλλει ἡ βασιλεία τοῦ Θεοῦ ἀναφαίνεσθαι.
\VS{12}εἶπεν οὖν· Ἄνθρωπός τις εὐγενὴς ἐπορεύθη εἰς χώραν μακρὰν λαβεῖν ἑαυτῷ βασιλείαν καὶ ὑποστρέψαι.
\VS{13}καλέσας δὲ δέκα δούλους ἑαυτοῦ ἔδωκεν αὐτοῖς δέκα μνᾶς καὶ εἶπεν πρὸς αὐτούς· Πραγματεύσασθε ἐν ᾧ ἔρχομαι.
\VS{14}Οἱ δὲ πολῖται αὐτοῦ ἐμίσουν αὐτόν καὶ ἀπέστειλαν πρεσβείαν ὀπίσω αὐτοῦ λέγοντες· Οὐ θέλομεν τοῦτον βασιλεῦσαι ἐφ᾽ ἡμᾶς.
\VS{15}Καὶ ἐγένετο ἐν τῷ ἐπανελθεῖν αὐτὸν λαβόντα τὴν βασιλείαν καὶ εἶπεν φωνηθῆναι αὐτῷ τοὺς δούλους τούτους οἷς δεδώκει τὸ ἀργύριον, ἵνα γνοῖ τί διεπραγματεύσαντο.
\VS{16}Παρεγένετο δὲ ὁ πρῶτος λέγων· Κύριε, ἡ μνᾶ σου δέκα προσηργάσατο μνᾶς.
\VS{17}Καὶ εἶπεν αὐτῷ· Εὖγε, ἀγαθὲ δοῦλε, ὅτι ἐν ἐλαχίστῳ πιστὸς ἐγένου, ἴσθι ἐξουσίαν ἔχων ἐπάνω δέκα πόλεων.
\VS{18}Καὶ ἦλθεν ὁ δεύτερος λέγων· Ἡ μνᾶ σου, κύριε, ἐποίησεν πέντε μνᾶς.
\VS{19}Εἶπεν δὲ καὶ τούτῳ· Καὶ σὺ ἐπάνω γίνου πέντε πόλεων.
\VS{20}Καὶ ὁ ἕτερος ἦλθεν λέγων· Κύριε, ἰδοὺ ἡ μνᾶ σου ἣν εἶχον ἀποκειμένην ἐν σουδαρίῳ·
\VS{21}ἐφοβούμην γάρ σε, ὅτι ἄνθρωπος αὐστηρὸς εἶ, αἴρεις ὃ οὐκ ἔθηκας καὶ θερίζεις ὃ οὐκ ἔσπειρας.
\VS{22}Λέγει αὐτῷ· Ἐκ τοῦ στόματός σου κρίνω σε, πονηρὲ δοῦλε. ᾔδεις ὅτι ἐγὼ ἄνθρωπος αὐστηρός εἰμι, αἴρων ὃ οὐκ ἔθηκα καὶ θερίζων ὃ οὐκ ἔσπειρα;
\VS{23}καὶ διὰ τί οὐκ ἔδωκάς μου τὸ ἀργύριον ἐπὶ τράπεζαν; κἀγὼ ἐλθὼν σὺν τόκῳ ἂν αὐτὸ ἔπραξα.
\VS{24}Καὶ τοῖς παρεστῶσιν εἶπεν· Ἄρατε ἀπ᾽ αὐτοῦ τὴν μνᾶν καὶ δότε τῷ τὰς δέκα μνᾶς ἔχοντι—
\VS{25}Καὶ εἶπαν αὐτῷ· Κύριε, ἔχει δέκα μνᾶς—
\VS{26}Λέγω ὑμῖν ὅτι παντὶ τῷ ἔχοντι δοθήσεται, ἀπὸ δὲ τοῦ μὴ ἔχοντος καὶ ὃ ἔχει ἀρθήσεται.
\VS{27}πλὴν τοὺς ἐχθρούς μου τούτους τοὺς μὴ θελήσαντάς με βασιλεῦσαι ἐπ᾽ αὐτοὺς ἀγάγετε ὧδε καὶ κατασφάξατε αὐτοὺς ἔμπροσθέν μου.
\par }{\PP \VS{28}Καὶ εἰπὼν ταῦτα ἐπορεύετο ἔμπροσθεν ἀναβαίνων εἰς Ἱεροσόλυμα.
\par }{\PP \VS{29}Καὶ ἐγένετο ὡς ἤγγισεν εἰς Βηθφαγὴ καὶ Βηθανίαν πρὸς τὸ ὄρος τὸ καλούμενον Ἐλαιῶν, ἀπέστειλεν δύο τῶν μαθητῶν
\VS{30}λέγων· Ὑπάγετε εἰς τὴν κατέναντι κώμην, ἐν ᾗ εἰσπορευόμενοι εὑρήσετε πῶλον δεδεμένον, ἐφ᾽ ὃν οὐδεὶς πώποτε ἀνθρώπων ἐκάθισεν, καὶ λύσαντες αὐτὸν ἀγάγετε.
\VS{31}καὶ ἐάν τις ὑμᾶς ἐρωτᾷ· Διὰ τί λύετε; οὕτως ἐρεῖτε· Ὅτι Ὁ Κύριος αὐτοῦ χρείαν ἔχει.
\VS{32}Ἀπελθόντες δὲ οἱ ἀπεσταλμένοι εὗρον καθὼς εἶπεν αὐτοῖς.
\VS{33}λυόντων δὲ αὐτῶν τὸν πῶλον εἶπαν οἱ κύριοι αὐτοῦ πρὸς αὐτούς· Τί λύετε τὸν πῶλον;
\VS{34}Οἱ δὲ εἶπαν· ὅτι Ὁ Κύριος αὐτοῦ χρείαν ἔχει.
\VS{35}Καὶ ἤγαγον αὐτὸν πρὸς τὸν Ἰησοῦν καὶ ἐπιρίψαντες αὐτῶν τὰ ἱμάτια ἐπὶ τὸν πῶλον ἐπεβίβασαν τὸν Ἰησοῦν.
\VS{36}Πορευομένου δὲ αὐτοῦ ὑπεστρώννυον τὰ ἱμάτια ἑαυτῶν ἐν τῇ ὁδῷ.
\VS{37}ἐγγίζοντος δὲ αὐτοῦ ἤδη πρὸς τῇ καταβάσει τοῦ ὄρους τῶν Ἐλαιῶν ἤρξαντο ἅπαν τὸ πλῆθος τῶν μαθητῶν χαίροντες αἰνεῖν τὸν Θεὸν φωνῇ μεγάλῃ περὶ πασῶν ὧν εἶδον δυνάμεων,
\VS{38}λέγοντες· 
\begin{poetryblock}
\par }{\PP \begin{quote}Εὐλογημένος ὁ ἐρχόμενος,\end{quote} 
\par }{\PP \begin{quote}ὁ Βασιλεὺς ἐν ὀνόματι Κυρίου· Ἐν οὐρανῷ εἰρήνη καὶ δόξα ἐν ὑψίστοις.\end{quote}
\end{poetryblock}
\VS{39}Καί τινες τῶν Φαρισαίων ἀπὸ τοῦ ὄχλου εἶπαν πρὸς αὐτόν· Διδάσκαλε, ἐπιτίμησον τοῖς μαθηταῖς σου.
\VS{40}Καὶ ἀποκριθεὶς εἶπεν· Λέγω ὑμῖν, ἐὰν οὗτοι σιωπήσουσιν, οἱ λίθοι κράξουσιν.
\par }{\PP \VS{41}Καὶ ὡς ἤγγισεν ἰδὼν τὴν πόλιν ἔκλαυσεν ἐπ᾽ αὐτήν
\VS{42}λέγων ὅτι Εἰ ἔγνως ἐν τῇ ἡμέρᾳ ταύτῃ καὶ σὺ τὰ πρὸς εἰρήνην· νῦν δὲ ἐκρύβη ἀπὸ ὀφθαλμῶν σου.
\VS{43}ὅτι ἥξουσιν ἡμέραι ἐπὶ σὲ καὶ παρεμβαλοῦσιν οἱ ἐχθροί σου χάρακά σοι καὶ περικυκλώσουσίν σε καὶ συνέξουσίν σε πάντοθεν,
\VS{44}καὶ ἐδαφιοῦσίν σε καὶ τὰ τέκνα σου ἐν σοί, καὶ οὐκ ἀφήσουσιν λίθον ἐπὶ λίθον ἐν σοί, ἀνθ᾽ ὧν οὐκ ἔγνως τὸν καιρὸν τῆς ἐπισκοπῆς σου.
\par }{\PP \VS{45}Καὶ εἰσελθὼν εἰς τὸ ἱερὸν ἤρξατο ἐκβάλλειν τοὺς πωλοῦντας
\VS{46}λέγων αὐτοῖς· Γέγραπται· 
\begin{poetryblock}
\par }{\PP \begin{quote}Καὶ ἔσται ὁ οἶκός μου οἶκος προσευχῆς, ὑμεῖς δὲ αὐτὸν ἐποιήσατε Σπήλαιον λῃστῶν.\end{quote}
\end{poetryblock}
\VS{47}Καὶ ἦν διδάσκων τὸ καθ᾽ ἡμέραν ἐν τῷ ἱερῷ. οἱ δὲ ἀρχιερεῖς καὶ οἱ γραμματεῖς ἐζήτουν αὐτὸν ἀπολέσαι καὶ οἱ πρῶτοι τοῦ λαοῦ,
\VS{48}καὶ οὐχ εὕρισκον τὸ τί ποιήσωσιν, ὁ λαὸς γὰρ ἅπας ἐξεκρέματο αὐτοῦ ἀκούων.

\par }\Chap{20}{\PP \VerseOne{1}Καὶ ἐγένετο ἐν μιᾷ τῶν ἡμερῶν διδάσκοντος αὐτοῦ τὸν λαὸν ἐν τῷ ἱερῷ καὶ εὐαγγελιζομένου ἐπέστησαν οἱ ἀρχιερεῖς καὶ οἱ γραμματεῖς σὺν τοῖς πρεσβυτέροις
\VS{2}καὶ εἶπαν λέγοντες πρὸς αὐτόν· Εἰπὸν ἡμῖν ἐν ποίᾳ ἐξουσίᾳ ταῦτα ποιεῖς, ἢ τίς ἐστιν ὁ δούς σοι τὴν ἐξουσίαν ταύτην;
\VS{3}Ἀποκριθεὶς δὲ εἶπεν πρὸς αὐτούς· Ἐρωτήσω ὑμᾶς κἀγὼ λόγον, καὶ εἴπατέ μοι·
\VS{4}Τὸ βάπτισμα Ἰωάννου ἐξ οὐρανοῦ ἦν ἢ ἐξ ἀνθρώπων;
\VS{5}Οἱ δὲ συνελογίσαντο πρὸς ἑαυτοὺς λέγοντες ὅτι Ἐὰν εἴπωμεν· Ἐξ οὐρανοῦ, ἐρεῖ· Διὰ τί οὐκ ἐπιστεύσατε αὐτῷ;
\VS{6}ἐὰν δὲ εἴπωμεν· Ἐξ ἀνθρώπων, ὁ λαὸς ἅπας καταλιθάσει ἡμᾶς, πεπεισμένος γάρ ἐστιν Ἰωάννην προφήτην εἶναι.
\VS{7}Καὶ ἀπεκρίθησαν μὴ εἰδέναι πόθεν.
\VS{8}Καὶ ὁ Ἰησοῦς εἶπεν αὐτοῖς· Οὐδὲ ἐγὼ λέγω ὑμῖν ἐν ποίᾳ ἐξουσίᾳ ταῦτα ποιῶ.
\par }{\PP \VS{9}Ἤρξατο δὲ πρὸς τὸν λαὸν λέγειν τὴν παραβολὴν ταύτην· Ἄνθρωπος τις ἐφύτευσεν ἀμπελῶνα καὶ ἐξέδετο αὐτὸν γεωργοῖς καὶ ἀπεδήμησεν χρόνους ἱκανούς.
\VS{10}καὶ καιρῷ ἀπέστειλεν πρὸς τοὺς γεωργοὺς δοῦλον ἵνα ἀπὸ τοῦ καρποῦ τοῦ ἀμπελῶνος δώσουσιν αὐτῷ· οἱ δὲ γεωργοὶ ἐξαπέστειλαν αὐτὸν δείραντες κενόν.
\VS{11}Καὶ προσέθετο ἕτερον πέμψαι δοῦλον· οἱ δὲ κἀκεῖνον δείραντες καὶ ἀτιμάσαντες ἐξαπέστειλαν κενόν.
\VS{12}Καὶ προσέθετο τρίτον πέμψαι· οἱ δὲ καὶ τοῦτον τραυματίσαντες ἐξέβαλον.
\VS{13}Εἶπεν δὲ ὁ κύριος τοῦ ἀμπελῶνος· Τί ποιήσω; πέμψω τὸν υἱόν μου τὸν ἀγαπητόν· ἴσως τοῦτον ἐντραπήσονται.
\VS{14}Ἰδόντες δὲ αὐτὸν οἱ γεωργοὶ διελογίζοντο πρὸς ἀλλήλους λέγοντες· Οὗτός ἐστιν ὁ κληρονόμος· ἀποκτείνωμεν αὐτόν, ἵνα ἡμῶν γένηται ἡ κληρονομία.
\VS{15}καὶ ἐκβαλόντες αὐτὸν ἔξω τοῦ ἀμπελῶνος ἀπέκτειναν. Τί οὖν ποιήσει αὐτοῖς ὁ κύριος τοῦ ἀμπελῶνος;
\VS{16}ἐλεύσεται καὶ ἀπολέσει τοὺς γεωργοὺς τούτους καὶ δώσει τὸν ἀμπελῶνα ἄλλοις.
\par }{\PP Ἀκούσαντες δὲ εἶπαν· Μὴ γένοιτο.
\VS{17}Ὁ δὲ ἐμβλέψας αὐτοῖς εἶπεν· Τί οὖν ἐστιν τὸ γεγραμμένον τοῦτο· 
\begin{poetryblock}
\par }{\PP \begin{quote}Λίθον ὃν ἀπεδοκίμασαν οἱ οἰκοδομοῦντες,\end{quote} 
\par }{\PP \begin{quote}Οὗτος ἐγενήθη εἰς κεφαλὴν γωνίας;\end{quote}
\end{poetryblock}
\par }{\PP \VS{18}Πᾶς ὁ πεσὼν ἐπ᾽ ἐκεῖνον τὸν λίθον συνθλασθήσεται· ἐφ᾽ ὃν δ᾽ ἂν πέσῃ, λικμήσει αὐτόν.
\par }{\PP \VS{19}Καὶ ἐζήτησαν οἱ γραμματεῖς καὶ οἱ ἀρχιερεῖς ἐπιβαλεῖν ἐπ᾽ αὐτὸν τὰς χεῖρας ἐν αὐτῇ τῇ ὥρᾳ, καὶ ἐφοβήθησαν τὸν λαόν, ἔγνωσαν γὰρ ὅτι πρὸς αὐτοὺς εἶπεν τὴν παραβολὴν ταύτην.
\par }{\PP \VS{20}Καὶ παρατηρήσαντες ἀπέστειλαν ἐνκαθέτους ὑποκρινομένους ἑαυτοὺς δικαίους εἶναι, ἵνα ἐπιλάβωνται αὐτοῦ λόγου, ὥστε παραδοῦναι αὐτὸν τῇ ἀρχῇ καὶ τῇ ἐξουσίᾳ τοῦ ἡγεμόνος.
\VS{21}καὶ ἐπηρώτησαν αὐτὸν λέγοντες· Διδάσκαλε, οἴδαμεν ὅτι ὀρθῶς λέγεις καὶ διδάσκεις καὶ οὐ λαμβάνεις πρόσωπον, ἀλλ᾽ ἐπ᾽ ἀληθείας τὴν ὁδὸν τοῦ Θεοῦ διδάσκεις·
\VS{22}ἔξεστιν ἡμᾶς Καίσαρι φόρον δοῦναι ἢ οὔ;
\VS{23}Κατανοήσας δὲ αὐτῶν τὴν πανουργίαν εἶπεν πρὸς αὐτούς·
\VS{24}Δείξατέ μοι δηνάριον· τίνος ἔχει εἰκόνα καὶ ἐπιγραφήν; Οἱ δὲ εἶπαν· Καίσαρος.
\VS{25}Ὁ δὲ εἶπεν πρὸς αὐτούς· Τοίνυν ἀπόδοτε τὰ Καίσαρος Καίσαρι καὶ τὰ τοῦ Θεοῦ τῷ Θεῷ.
\VS{26}Καὶ οὐκ ἴσχυσαν ἐπιλαβέσθαι αὐτοῦ ῥήματος ἐναντίον τοῦ λαοῦ καὶ θαυμάσαντες ἐπὶ τῇ ἀποκρίσει αὐτοῦ ἐσίγησαν.
\par }{\PP \VS{27}Προσελθόντες δέ τινες τῶν Σαδδουκαίων, οἱ ἀντιλέγοντες ἀνάστασιν μὴ εἶναι, ἐπηρώτησαν αὐτὸν
\VS{28}λέγοντες· Διδάσκαλε, Μωϋσῆς ἔγραψεν ἡμῖν, ἐάν τινος ἀδελφὸς ἀποθάνῃ ἔχων γυναῖκα, καὶ οὗτος ἄτεκνος ᾖ, ἵνα λάβῃ ὁ ἀδελφὸς αὐτοῦ τὴν γυναῖκα καὶ ἐξαναστήσῃ σπέρμα τῷ ἀδελφῷ αὐτοῦ.
\VS{29}ἑπτὰ οὖν ἀδελφοὶ ἦσαν· καὶ ὁ πρῶτος λαβὼν γυναῖκα ἀπέθανεν ἄτεκνος·
\VS{30}καὶ ὁ δεύτερος
\VS{31}καὶ ὁ τρίτος ἔλαβεν αὐτήν, ὡσαύτως δὲ καὶ οἱ ἑπτὰ οὐ κατέλιπον τέκνα καὶ ἀπέθανον.
\VS{32}ὕστερον καὶ ἡ γυνὴ ἀπέθανεν.
\VS{33}ἡ γυνὴ οὖν ἐν τῇ ἀναστάσει τίνος αὐτῶν γίνεται γυνή; οἱ γὰρ ἑπτὰ ἔσχον αὐτὴν γυναῖκα.
\par }{\PP \VS{34}Καὶ εἶπεν αὐτοῖς ὁ Ἰησοῦς· Οἱ υἱοὶ τοῦ αἰῶνος τούτου γαμοῦσιν καὶ γαμίσκονται,
\VS{35}οἱ δὲ καταξιωθέντες τοῦ αἰῶνος ἐκείνου τυχεῖν καὶ τῆς ἀναστάσεως τῆς ἐκ νεκρῶν οὔτε γαμοῦσιν οὔτε γαμίζονται·
\VS{36}οὐδὲ γὰρ ἀποθανεῖν ἔτι δύνανται, ἰσάγγελοι γάρ εἰσιν καὶ υἱοί εἰσιν Θεοῦ τῆς ἀναστάσεως υἱοὶ ὄντες.
\VS{37}Ὅτι δὲ ἐγείρονται οἱ νεκροὶ, καὶ Μωϋσῆς ἐμήνυσεν ἐπὶ τῆς Βάτου, ὡς λέγει Κύριον Τὸν Θεὸν Ἀβραὰμ καὶ Θεὸν Ἰσαὰκ καὶ Θεὸν Ἰακώβ.
\VS{38}Θεὸς δὲ οὐκ ἔστιν νεκρῶν ἀλλὰ ζώντων, πάντες γὰρ αὐτῷ ζῶσιν.
\par }{\PP \VS{39}Ἀποκριθέντες δέ τινες τῶν γραμματέων εἶπαν· Διδάσκαλε, καλῶς εἶπας.
\VS{40}οὐκέτι γὰρ ἐτόλμων ἐπερωτᾶν αὐτὸν οὐδέν.
\par }{\PP \VS{41}Εἶπεν δὲ πρὸς αὐτούς· Πῶς λέγουσιν τὸν Χριστὸν εἶναι Δαυὶδ υἱόν;
\VS{42}αὐτὸς γὰρ Δαυὶδ λέγει ἐν βίβλῳ ψαλμῶν· 
\begin{poetryblock}
\par }{\PP \begin{quote}Εἶπεν Κύριος τῷ Κυρίῳ μου·\end{quote} 
\par }{\PP \begin{quote}Κάθου ἐκ δεξιῶν μου,\end{quote}
\par }{\PP \begin{quote} \VS{43}Ἕως ἂν θῶ τοὺς ἐχθρούς σου\end{quote} 
\par }{\PP \begin{quote}Ὑποπόδιον τῶν ποδῶν σου.\end{quote}
\end{poetryblock}
\par }{\PP \VS{44}Δαυὶδ οὖν Κύριον αὐτὸν καλεῖ, καὶ πῶς αὐτοῦ υἱός ἐστιν;
\par }{\PP \VS{45}Ἀκούοντος δὲ παντὸς τοῦ λαοῦ εἶπεν τοῖς μαθηταῖς αὐτοῦ·
\VS{46}Προσέχετε ἀπὸ τῶν γραμματέων τῶν θελόντων περιπατεῖν ἐν στολαῖς καὶ φιλούντων ἀσπασμοὺς ἐν ταῖς ἀγοραῖς καὶ πρωτοκαθεδρίας ἐν ταῖς συναγωγαῖς καὶ πρωτοκλισίας ἐν τοῖς δείπνοις,
\VS{47}οἳ κατεσθίουσιν τὰς οἰκίας τῶν χηρῶν καὶ προφάσει μακρὰ προσεύχονται· οὗτοι λήμψονται περισσότερον κρίμα.

\par }\Chap{21}{\PP \VerseOne{1}Ἀναβλέψας δὲ εἶδεν τοὺς βάλλοντας εἰς τὸ γαζοφυλάκιον τὰ δῶρα αὐτῶν πλουσίους.
\VS{2}εἶδεν δέ τινα χήραν πενιχρὰν βάλλουσαν ἐκεῖ λεπτὰ δύο,
\VS{3}Καὶ εἶπεν· Ἀληθῶς λέγω ὑμῖν ὅτι ἡ χήρα αὕτη ἡ πτωχὴ πλεῖον πάντων ἔβαλεν·
\VS{4}πάντες γὰρ οὗτοι ἐκ τοῦ περισσεύοντος αὐτοῖς ἔβαλον εἰς τὰ δῶρα, αὕτη δὲ ἐκ τοῦ ὑστερήματος αὐτῆς πάντα τὸν βίον ὃν εἶχεν ἔβαλεν.
\par }{\PP \VS{5}Καί τινων λεγόντων περὶ τοῦ ἱεροῦ ὅτι λίθοις καλοῖς καὶ ἀναθήμασιν κεκόσμηται εἶπεν·
\VS{6}Ταῦτα ἃ θεωρεῖτε ἐλεύσονται ἡμέραι ἐν αἷς οὐκ ἀφεθήσεται λίθος ἐπὶ λίθῳ ὃς οὐ καταλυθήσεται.
\par }{\PP \VS{7}Ἐπηρώτησαν δὲ αὐτὸν λέγοντες· Διδάσκαλε, πότε οὖν ταῦτα ἔσται καὶ τί τὸ σημεῖον ὅταν μέλλῃ ταῦτα γίνεσθαι;
\VS{8}Ὁ δὲ εἶπεν· Βλέπετε μὴ πλανηθῆτε· πολλοὶ γὰρ ἐλεύσονται ἐπὶ τῷ ὀνόματί μου λέγοντες· Ἐγώ εἰμι, καί· Ὁ καιρὸς ἤγγικεν. μὴ πορευθῆτε ὀπίσω αὐτῶν.
\VS{9}ὅταν δὲ ἀκούσητε πολέμους καὶ ἀκαταστασίας, μὴ πτοηθῆτε· δεῖ γὰρ ταῦτα γενέσθαι πρῶτον, ἀλλ᾽ οὐκ εὐθέως τὸ τέλος.
\par }{\PP \VS{10}Τότε ἔλεγεν αὐτοῖς· Ἐγερθήσεται ἔθνος ἐπ᾽ ἔθνος καὶ βασιλεία ἐπὶ βασιλείαν,
\VS{11}σεισμοί τε μεγάλοι καὶ κατὰ τόπους λιμοὶ καὶ λοιμοὶ ἔσονται, φόβητρά τε καὶ ἀπ᾽ οὐρανοῦ σημεῖα μεγάλα ἔσται.
\par }{\PP \VS{12}Πρὸ δὲ τούτων πάντων ἐπιβαλοῦσιν ἐφ᾽ ὑμᾶς τὰς χεῖρας αὐτῶν καὶ διώξουσιν, παραδιδόντες εἰς τὰς συναγωγὰς καὶ φυλακάς, ἀπαγομένους ἐπὶ βασιλεῖς καὶ ἡγεμόνας ἕνεκεν τοῦ ὀνόματός μου·
\VS{13}ἀποβήσεται ὑμῖν εἰς μαρτύριον.
\VS{14}θέτε οὖν ἐν ταῖς καρδίαις ὑμῶν μὴ προμελετᾶν ἀπολογηθῆναι·
\VS{15}ἐγὼ γὰρ δώσω ὑμῖν στόμα καὶ σοφίαν ᾗ οὐ δυνήσονται ἀντιστῆναι ἢ ἀντειπεῖν ἅπαντες οἱ ἀντικείμενοι ὑμῖν.
\VS{16}Παραδοθήσεσθε δὲ καὶ ὑπὸ γονέων καὶ ἀδελφῶν καὶ συγγενῶν καὶ φίλων, καὶ θανατώσουσιν ἐξ ὑμῶν,
\VS{17}καὶ ἔσεσθε μισούμενοι ὑπὸ πάντων διὰ τὸ ὄνομά μου.
\VS{18}καὶ θρὶξ ἐκ τῆς κεφαλῆς ὑμῶν οὐ μὴ ἀπόληται.
\VS{19}ἐν τῇ ὑπομονῇ ὑμῶν κτήσασθε τὰς ψυχὰς ὑμῶν.
\par }{\PP \VS{20}Ὅταν δὲ ἴδητε κυκλουμένην ὑπὸ στρατοπέδων Ἰερουσαλήμ, τότε γνῶτε ὅτι ἤγγικεν ἡ ἐρήμωσις αὐτῆς.
\VS{21}τότε οἱ ἐν τῇ Ἰουδαίᾳ φευγέτωσαν εἰς τὰ ὄρη καὶ οἱ ἐν μέσῳ αὐτῆς ἐκχωρείτωσαν καὶ οἱ ἐν ταῖς χώραις μὴ εἰσερχέσθωσαν εἰς αὐτήν,
\VS{22}ὅτι ἡμέραι ἐκδικήσεως αὗταί εἰσιν τοῦ πλησθῆναι πάντα τὰ γεγραμμένα.
\VS{23}Οὐαὶ ταῖς ἐν γαστρὶ ἐχούσαις καὶ ταῖς θηλαζούσαις ἐν ἐκείναις ταῖς ἡμέραις· ἔσται γὰρ ἀνάγκη μεγάλη ἐπὶ τῆς γῆς καὶ ὀργὴ τῷ λαῷ τούτῳ,
\VS{24}καὶ πεσοῦνται στόματι μαχαίρης καὶ αἰχμαλωτισθήσονται εἰς τὰ ἔθνη πάντα, καὶ Ἰερουσαλὴμ ἔσται πατουμένη ὑπὸ ἐθνῶν, ἄχρι οὗ πληρωθῶσιν καιροὶ ἐθνῶν.
\par }{\PP \VS{25}Καὶ ἔσονται σημεῖα ἐν ἡλίῳ καὶ σελήνῃ καὶ ἄστροις, καὶ ἐπὶ τῆς γῆς συνοχὴ ἐθνῶν ἐν ἀπορίᾳ ἤχους θαλάσσης καὶ σάλου,
\VS{26}ἀποψυχόντων ἀνθρώπων ἀπὸ φόβου καὶ προσδοκίας τῶν ἐπερχομένων τῇ οἰκουμένῃ, αἱ γὰρ δυνάμεις τῶν οὐρανῶν σαλευθήσονται.
\VS{27}καὶ τότε ὄψονται τὸν Υἱὸν τοῦ ἀνθρώπου ἐρχόμενον ἐν νεφέλῃ μετὰ δυνάμεως καὶ δόξης πολλῆς.
\VS{28}ἀρχομένων δὲ τούτων γίνεσθαι ἀνακύψατε καὶ ἐπάρατε τὰς κεφαλὰς ὑμῶν, διότι ἐγγίζει ἡ ἀπολύτρωσις ὑμῶν.
\par }{\PP \VS{29}Καὶ εἶπεν παραβολὴν αὐτοῖς· Ἴδετε τὴν συκῆν καὶ πάντα τὰ δένδρα·
\VS{30}ὅταν προβάλωσιν ἤδη, βλέποντες ἀφ᾽ ἑαυτῶν γινώσκετε ὅτι ἤδη ἐγγὺς τὸ θέρος ἐστίν·
\VS{31}οὕτως καὶ ὑμεῖς, ὅταν ἴδητε ταῦτα γινόμενα, γινώσκετε ὅτι ἐγγύς ἐστιν ἡ βασιλεία τοῦ Θεοῦ.
\VS{32}ἀμὴν λέγω ὑμῖν ὅτι οὐ μὴ παρέλθῃ ἡ γενεὰ αὕτη ἕως ἂν πάντα γένηται.
\VS{33}ὁ οὐρανὸς καὶ ἡ γῆ παρελεύσονται, οἱ δὲ λόγοι μου οὐ μὴ παρελεύσονται.
\par }{\PP \VS{34}Προσέχετε δὲ ἑαυτοῖς μήποτε βαρηθῶσιν ὑμῶν αἱ καρδίαι ἐν κραιπάλῃ καὶ μέθῃ καὶ μερίμναις βιωτικαῖς καὶ ἐπιστῇ ἐφ᾽ ὑμᾶς αἰφνίδιος ἡ ἡμέρα ἐκείνη
\VS{35}ὡς παγίς· ἐπεισελεύσεται γὰρ ἐπὶ πάντας τοὺς καθημένους ἐπὶ πρόσωπον πάσης τῆς γῆς.
\VS{36}ἀγρυπνεῖτε δὲ ἐν παντὶ καιρῷ δεόμενοι ἵνα κατισχύσητε ἐκφυγεῖν ταῦτα πάντα τὰ μέλλοντα γίνεσθαι καὶ σταθῆναι ἔμπροσθεν τοῦ Υἱοῦ τοῦ ἀνθρώπου.
\par }{\PP \VS{37}Ἦν δὲ τὰς ἡμέρας ἐν τῷ ἱερῷ διδάσκων, τὰς δὲ νύκτας ἐξερχόμενος ηὐλίζετο εἰς τὸ ὄρος τὸ καλούμενον Ἐλαιῶν·
\VS{38}καὶ πᾶς ὁ λαὸς ὤρθριζεν πρὸς αὐτὸν ἐν τῷ ἱερῷ ἀκούειν αὐτοῦ.

\par }\Chap{22}{\PP \VerseOne{1}Ἤγγιζεν δὲ ἡ ἑορτὴ τῶν ἀζύμων ἡ λεγομένη Πάσχα.
\VS{2}καὶ ἐζήτουν οἱ ἀρχιερεῖς καὶ οἱ γραμματεῖς τὸ πῶς ἀνέλωσιν αὐτόν, ἐφοβοῦντο γὰρ τὸν λαόν.
\par }{\PP \VS{3}Εἰσῆλθεν δὲ Σατανᾶς εἰς Ἰούδαν τὸν καλούμενον Ἰσκαριώτην, ὄντα ἐκ τοῦ ἀριθμοῦ τῶν δώδεκα·
\VS{4}καὶ ἀπελθὼν συνελάλησεν τοῖς ἀρχιερεῦσιν καὶ στρατηγοῖς τὸ πῶς αὐτοῖς παραδῷ αὐτόν.
\VS{5}καὶ ἐχάρησαν καὶ συνέθεντο αὐτῷ ἀργύριον δοῦναι.
\VS{6}καὶ ἐξωμολόγησεν, καὶ ἐζήτει εὐκαιρίαν τοῦ παραδοῦναι αὐτὸν ἄτερ ὄχλου αὐτοῖς.
\par }{\PP \VS{7}Ἦλθεν δὲ ἡ ἡμέρα τῶν ἀζύμων, ἐν ᾗ ἔδει θύεσθαι τὸ πάσχα·
\VS{8}καὶ ἀπέστειλεν Πέτρον καὶ Ἰωάννην εἰπών· Πορευθέντες ἑτοιμάσατε ἡμῖν τὸ πάσχα ἵνα φάγωμεν.
\VS{9}Οἱ δὲ εἶπαν αὐτῷ· Ποῦ θέλεις ἑτοιμάσωμεν;
\VS{10}Ὁ δὲ εἶπεν αὐτοῖς· Ἰδοὺ εἰσελθόντων ὑμῶν εἰς τὴν πόλιν συναντήσει ὑμῖν ἄνθρωπος κεράμιον ὕδατος βαστάζων· ἀκολουθήσατε αὐτῷ εἰς τὴν οἰκίαν εἰς ἣν εἰσπορεύεται,
\VS{11}καὶ ἐρεῖτε τῷ οἰκοδεσπότῃ τῆς οἰκίας· Λέγει σοι ὁ Διδάσκαλος· Ποῦ ἐστιν τὸ κατάλυμα ὅπου τὸ πάσχα μετὰ τῶν μαθητῶν μου φάγω;
\VS{12}κἀκεῖνος ὑμῖν δείξει ἀνάγαιον μέγα ἐστρωμένον· ἐκεῖ ἑτοιμάσατε.
\VS{13}Ἀπελθόντες δὲ εὗρον καθὼς εἰρήκει αὐτοῖς καὶ ἡτοίμασαν τὸ πάσχα.
\par }{\PP \VS{14}Καὶ ὅτε ἐγένετο ἡ ὥρα, ἀνέπεσεν καὶ οἱ ἀπόστολοι σὺν αὐτῷ.
\VS{15}καὶ εἶπεν πρὸς αὐτούς· Ἐπιθυμίᾳ ἐπεθύμησα τοῦτο τὸ πάσχα φαγεῖν μεθ᾽ ὑμῶν πρὸ τοῦ με παθεῖν·
\VS{16}λέγω γὰρ ὑμῖν ὅτι οὐ μὴ φάγω αὐτὸ ἕως ὅτου πληρωθῇ ἐν τῇ βασιλείᾳ τοῦ Θεοῦ.
\VS{17}Καὶ δεξάμενος ποτήριον εὐχαριστήσας εἶπεν· Λάβετε τοῦτο καὶ διαμερίσατε εἰς ἑαυτούς·
\VS{18}λέγω γὰρ ὑμῖν, ὅτι οὐ μὴ πίω ἀπὸ τοῦ νῦν ἀπὸ τοῦ γενήματος τῆς ἀμπέλου ἕως οὗ ἡ βασιλεία τοῦ Θεοῦ ἔλθῃ.
\VS{19}Καὶ λαβὼν ἄρτον εὐχαριστήσας ἔκλασεν καὶ ἔδωκεν αὐτοῖς λέγων· Τοῦτό ἐστιν τὸ σῶμά μου τὸ ὑπὲρ ὑμῶν διδόμενον· τοῦτο ποιεῖτε εἰς τὴν ἐμὴν ἀνάμνησιν.
\VS{20}καὶ τὸ ποτήριον ὡσαύτως μετὰ τὸ δειπνῆσαι, λέγων· Τοῦτο τὸ ποτήριον ἡ καινὴ διαθήκη ἐν τῷ αἵματί μου τὸ ὑπὲρ ὑμῶν ἐκχυννόμενον.
\par }{\PP \VS{21}Πλὴν ἰδοὺ ἡ χεὶρ τοῦ παραδιδόντος με μετ᾽ ἐμοῦ ἐπὶ τῆς τραπέζης.
\VS{22}ὅτι ὁ Υἱὸς μὲν τοῦ ἀνθρώπου κατὰ τὸ ὡρισμένον πορεύεται, πλὴν οὐαὶ τῷ ἀνθρώπῳ ἐκείνῳ δι᾽ οὗ παραδίδοται.
\VS{23}Καὶ αὐτοὶ ἤρξαντο συζητεῖν πρὸς ἑαυτοὺς τὸ τίς ἄρα εἴη ἐξ αὐτῶν ὁ τοῦτο μέλλων πράσσειν.
\par }{\PP \VS{24}Ἐγένετο δὲ καὶ φιλονεικία ἐν αὐτοῖς, τὸ τίς αὐτῶν δοκεῖ εἶναι μείζων.
\VS{25}ὁ δὲ εἶπεν αὐτοῖς· Οἱ βασιλεῖς τῶν ἐθνῶν κυριεύουσιν αὐτῶν καὶ οἱ ἐξουσιάζοντες αὐτῶν εὐεργέται καλοῦνται.
\VS{26}ὑμεῖς δὲ οὐχ οὕτως, ἀλλ᾽ ὁ μείζων ἐν ὑμῖν γινέσθω ὡς ὁ νεώτερος καὶ ὁ ἡγούμενος ὡς ὁ διακονῶν.
\VS{27}τίς γὰρ μείζων, ὁ ἀνακείμενος ἢ ὁ διακονῶν; οὐχὶ ὁ ἀνακείμενος; ἐγὼ δὲ ἐν μέσῳ ὑμῶν εἰμι ὡς ὁ διακονῶν.
\par }{\PP \VS{28}Ὑμεῖς δέ ἐστε οἱ διαμεμενηκότες μετ᾽ ἐμοῦ ἐν τοῖς πειρασμοῖς μου·
\VS{29}κἀγὼ διατίθεμαι ὑμῖν καθὼς διέθετό μοι ὁ Πατήρ μου βασιλείαν,
\VS{30}ἵνα ἔσθητε καὶ πίνητε ἐπὶ τῆς τραπέζης μου ἐν τῇ βασιλείᾳ μου, καὶ καθήσεσθε ἐπὶ θρόνων τὰς δώδεκα φυλὰς κρίνοντες τοῦ Ἰσραήλ.
\par }{\PP \VS{31}Σίμων Σίμων, ἰδοὺ ὁ Σατανᾶς ἐξῃτήσατο ὑμᾶς τοῦ σινιάσαι ὡς τὸν σῖτον·
\VS{32}ἐγὼ δὲ ἐδεήθην περὶ σοῦ ἵνα μὴ ἐκλίπῃ ἡ πίστις σου· καὶ σύ ποτε ἐπιστρέψας στήρισον τοὺς ἀδελφούς σου.
\VS{33}Ὁ δὲ εἶπεν αὐτῷ· Κύριε, μετὰ σοῦ ἕτοιμός εἰμι καὶ εἰς φυλακὴν καὶ εἰς θάνατον πορεύεσθαι.
\VS{34}Ὁ δὲ εἶπεν· Λέγω σοι, Πέτρε, οὐ φωνήσει σήμερον ἀλέκτωρ ἕως τρίς με ἀπαρνήσῃ εἰδέναι.
\par }{\PP \VS{35}Καὶ εἶπεν αὐτοῖς· Ὅτε ἀπέστειλα ὑμᾶς ἄτερ βαλλαντίου καὶ πήρας καὶ ὑποδημάτων, μή τινος ὑστερήσατε; Οἱ δὲ εἶπαν· Οὐθενός.
\VS{36}Εἶπεν δὲ αὐτοῖς· Ἀλλὰ νῦν ὁ ἔχων βαλλάντιον ἀράτω, ὁμοίως καὶ πήραν, καὶ ὁ μὴ ἔχων πωλησάτω τὸ ἱμάτιον αὐτοῦ καὶ ἀγορασάτω μάχαιραν.
\VS{37}λέγω γὰρ ὑμῖν ὅτι τοῦτο τὸ γεγραμμένον δεῖ τελεσθῆναι ἐν ἐμοί, Τό· Καὶ μετὰ ἀνόμων ἐλογίσθη· καὶ γὰρ τὸ περὶ ἐμοῦ τέλος ἔχει.
\VS{38}Οἱ δὲ εἶπαν· Κύριε, ἰδοὺ μάχαιραι ὧδε δύο. Ὁ δὲ εἶπεν αὐτοῖς· Ἱκανόν ἐστιν.
\par }{\PP \VS{39}Καὶ ἐξελθὼν ἐπορεύθη κατὰ τὸ ἔθος εἰς τὸ ὄρος τῶν Ἐλαιῶν, ἠκολούθησαν δὲ αὐτῷ καὶ οἱ μαθηταί.
\VS{40}γενόμενος δὲ ἐπὶ τοῦ τόπου εἶπεν αὐτοῖς· Προσεύχεσθε μὴ εἰσελθεῖν εἰς πειρασμόν.
\VS{41}Καὶ αὐτὸς ἀπεσπάσθη ἀπ᾽ αὐτῶν ὡσεὶ λίθου βολήν καὶ θεὶς τὰ γόνατα προσηύχετο
\VS{42}λέγων· Πάτερ, εἰ βούλει παρένεγκε τοῦτο τὸ ποτήριον ἀπ᾽ ἐμοῦ· πλὴν μὴ τὸ θέλημά μου ἀλλὰ τὸ σὸν γινέσθω.
\VS{43}Ὤφθη δὲ αὐτῷ ἄγγελος ἀπ᾽ οὐρανοῦ ἐνισχύων αὐτόν.
\VS{44}καὶ γενόμενος ἐν ἀγωνίᾳ ἐκτενέστερον προσηύχετο· καὶ ἐγένετο ὁ ἱδρὼς αὐτοῦ ὡσεὶ θρόμβοι αἵματος καταβαίνοντες ἐπὶ τὴν γῆν.
\VS{45}Καὶ ἀναστὰς ἀπὸ τῆς προσευχῆς ἐλθὼν πρὸς τοὺς μαθητὰς εὗρεν κοιμωμένους αὐτοὺς ἀπὸ τῆς λύπης,
\VS{46}καὶ εἶπεν αὐτοῖς· Τί καθεύδετε; ἀναστάντες προσεύχεσθε, ἵνα μὴ εἰσέλθητε εἰς πειρασμόν.
\par }{\PP \VS{47}Ἔτι αὐτοῦ λαλοῦντος ἰδοὺ ὄχλος, καὶ ὁ λεγόμενος Ἰούδας εἷς τῶν δώδεκα προήρχετο αὐτούς καὶ ἤγγισεν τῷ Ἰησοῦ φιλῆσαι αὐτόν.
\VS{48}Ἰησοῦς δὲ εἶπεν αὐτῷ· Ἰούδα, φιλήματι τὸν Υἱὸν τοῦ ἀνθρώπου παραδίδως;
\VS{49}Ἰδόντες δὲ οἱ περὶ αὐτὸν τὸ ἐσόμενον εἶπαν· Κύριε, εἰ πατάξομεν ἐν μαχαίρῃ;
\VS{50}καὶ ἐπάταξεν εἷς τις ἐξ αὐτῶν τοῦ ἀρχιερέως τὸν δοῦλον καὶ ἀφεῖλεν τὸ οὖς αὐτοῦ τὸ δεξιόν.
\VS{51}Ἀποκριθεὶς δὲ ὁ Ἰησοῦς εἶπεν· Ἐᾶτε ἕως τούτου· καὶ ἁψάμενος τοῦ ὠτίου ἰάσατο αὐτόν.
\par }{\PP \VS{52}Εἶπεν δὲ Ἰησοῦς πρὸς τοὺς παραγενομένους ἐπ᾽ αὐτὸν ἀρχιερεῖς καὶ στρατηγοὺς τοῦ ἱεροῦ καὶ πρεσβυτέρους· Ὡς ἐπὶ λῃστὴν ἐξήλθατε μετὰ μαχαιρῶν καὶ ξύλων;
\VS{53}καθ᾽ ἡμέραν ὄντος μου μεθ᾽ ὑμῶν ἐν τῷ ἱερῷ οὐκ ἐξετείνατε τὰς χεῖρας ἐπ᾽ ἐμέ, ἀλλ᾽ αὕτη ἐστὶν ὑμῶν ἡ ὥρα καὶ ἡ ἐξουσία τοῦ σκότους.
\par }{\PP \VS{54}Συλλαβόντες δὲ αὐτὸν ἤγαγον καὶ εἰσήγαγον εἰς τὴν οἰκίαν τοῦ ἀρχιερέως· ὁ δὲ Πέτρος ἠκολούθει μακρόθεν.
\VS{55}Περιαψάντων δὲ πῦρ ἐν μέσῳ τῆς αὐλῆς καὶ συνκαθισάντων ἐκάθητο ὁ Πέτρος μέσος αὐτῶν.
\VS{56}ἰδοῦσα δὲ αὐτὸν παιδίσκη τις καθήμενον πρὸς τὸ φῶς καὶ ἀτενίσασα αὐτῷ εἶπεν· Καὶ οὗτος σὺν αὐτῷ ἦν.
\VS{57}Ὁ δὲ ἠρνήσατο λέγων· Οὐκ οἶδα αὐτόν, γύναι.
\VS{58}Καὶ μετὰ βραχὺ ἕτερος ἰδὼν αὐτὸν ἔφη· Καὶ σὺ ἐξ αὐτῶν εἶ. Ὁ δὲ Πέτρος ἔφη· Ἄνθρωπε, οὐκ εἰμί.
\VS{59}Καὶ διαστάσης ὡσεὶ ὥρας μιᾶς ἄλλος τις διϊσχυρίζετο λέγων· Ἐπ᾽ ἀληθείας καὶ οὗτος μετ᾽ αὐτοῦ ἦν, καὶ γὰρ Γαλιλαῖός ἐστιν.
\VS{60}Εἶπεν δὲ ὁ Πέτρος· Ἄνθρωπε, οὐκ οἶδα ὃ λέγεις. καὶ παραχρῆμα ἔτι λαλοῦντος αὐτοῦ ἐφώνησεν ἀλέκτωρ.
\VS{61}καὶ στραφεὶς ὁ Κύριος ἐνέβλεψεν τῷ Πέτρῳ, καὶ ὑπεμνήσθη ὁ Πέτρος τοῦ ῥήματος τοῦ Κυρίου ὡς εἶπεν αὐτῷ ὅτι Πρὶν ἀλέκτορα φωνῆσαι σήμερον ἀπαρνήσῃ με τρίς.
\VS{62}καὶ ἐξελθὼν ἔξω ἔκλαυσεν πικρῶς.
\par }{\PP \VS{63}Καὶ οἱ ἄνδρες οἱ συνέχοντες αὐτὸν ἐνέπαιζον αὐτῷ δέροντες,
\VS{64}καὶ περικαλύψαντες αὐτὸν ἐπηρώτων λέγοντες· Προφήτευσον, τίς ἐστιν ὁ παίσας σε;
\VS{65}καὶ ἕτερα πολλὰ βλασφημοῦντες ἔλεγον εἰς αὐτόν.
\par }{\PP \VS{66}Καὶ ὡς ἐγένετο ἡμέρα, συνήχθη τὸ πρεσβυτέριον τοῦ λαοῦ, ἀρχιερεῖς τε καὶ γραμματεῖς, καὶ ἀπήγαγον αὐτὸν εἰς τὸ συνέδριον αὐτῶν
\VS{67}λέγοντες· Εἰ σὺ εἶ ὁ Χριστός, εἰπὸν ἡμῖν. Εἶπεν δὲ αὐτοῖς· Ἐὰν ὑμῖν εἴπω, οὐ μὴ πιστεύσητε·
\VS{68}ἐὰν δὲ ἐρωτήσω, οὐ μὴ ἀποκριθῆτε.
\VS{69}ἀπὸ τοῦ νῦν δὲ ἔσται ὁ Υἱὸς τοῦ ἀνθρώπου καθήμενος ἐκ δεξιῶν τῆς δυνάμεως τοῦ Θεοῦ.
\VS{70}Εἶπαν δὲ πάντες· Σὺ οὖν εἶ ὁ Υἱὸς τοῦ Θεοῦ; Ὁ δὲ πρὸς αὐτοὺς ἔφη· Ὑμεῖς λέγετε ὅτι ἐγώ εἰμι.
\VS{71}Οἱ δὲ εἶπαν· Τί ἔτι ἔχομεν μαρτυρίας χρείαν; αὐτοὶ γὰρ ἠκούσαμεν ἀπὸ τοῦ στόματος αὐτοῦ.

\par }\Chap{23}{\PP \VerseOne{1}Καὶ ἀναστὰν ἅπαν τὸ πλῆθος αὐτῶν ἤγαγον αὐτὸν ἐπὶ τὸν Πιλᾶτον.
\par }{\PP \VS{2}ἤρξαντο δὲ κατηγορεῖν αὐτοῦ λέγοντες· Τοῦτον εὕραμεν διαστρέφοντα τὸ ἔθνος ἡμῶν καὶ κωλύοντα φόρους Καίσαρι διδόναι καὶ λέγοντα ἑαυτὸν Χριστὸν βασιλέα εἶναι.
\VS{3}Ὁ δὲ Πιλᾶτος ἠρώτησεν αὐτὸν λέγων· Σὺ εἶ ὁ Βασιλεὺς τῶν Ἰουδαίων; Ὁ δὲ ἀποκριθεὶς αὐτῷ ἔφη· Σὺ λέγεις.
\VS{4}Ὁ δὲ Πιλᾶτος εἶπεν πρὸς τοὺς ἀρχιερεῖς καὶ τοὺς ὄχλους· Οὐδὲν εὑρίσκω αἴτιον ἐν τῷ ἀνθρώπῳ τούτῳ.
\VS{5}Οἱ δὲ ἐπίσχυον λέγοντες ὅτι Ἀνασείει τὸν λαὸν διδάσκων καθ᾽ ὅλης τῆς Ἰουδαίας, καὶ ἀρξάμενος ἀπὸ τῆς Γαλιλαίας ἕως ὧδε.
\par }{\PP \VS{6}Πιλᾶτος δὲ ἀκούσας ἐπηρώτησεν εἰ ὁ ἄνθρωπος Γαλιλαῖός ἐστιν,
\VS{7}καὶ ἐπιγνοὺς ὅτι ἐκ τῆς ἐξουσίας Ἡρῴδου ἐστὶν ἀνέπεμψεν αὐτὸν πρὸς Ἡρῴδην, ὄντα καὶ αὐτὸν ἐν Ἱεροσολύμοις ἐν ταύταις ταῖς ἡμέραις.
\par }{\PP \VS{8}Ὁ δὲ Ἡρῴδης ἰδὼν τὸν Ἰησοῦν ἐχάρη λίαν, ἦν γὰρ ἐξ ἱκανῶν χρόνων θέλων ἰδεῖν αὐτὸν διὰ τὸ ἀκούειν περὶ αὐτοῦ καὶ ἤλπιζέν τι σημεῖον ἰδεῖν ὑπ᾽ αὐτοῦ γινόμενον.
\VS{9}ἐπηρώτα δὲ αὐτὸν ἐν λόγοις ἱκανοῖς, αὐτὸς δὲ οὐδὲν ἀπεκρίνατο αὐτῷ.
\VS{10}Εἱστήκεισαν δὲ οἱ ἀρχιερεῖς καὶ οἱ γραμματεῖς εὐτόνως κατηγοροῦντες αὐτοῦ.
\VS{11}ἐξουθενήσας δὲ αὐτὸν καὶ ὁ Ἡρῴδης σὺν τοῖς στρατεύμασιν αὐτοῦ καὶ ἐμπαίξας περιβαλὼν ἐσθῆτα λαμπρὰν ἀνέπεμψεν αὐτὸν τῷ Πιλάτῳ.
\VS{12}Ἐγένοντο δὲ φίλοι ὅ τε Ἡρῴδης καὶ ὁ Πιλᾶτος ἐν αὐτῇ τῇ ἡμέρᾳ μετ᾽ ἀλλήλων· προϋπῆρχον γὰρ ἐν ἔχθρᾳ ὄντες πρὸς αὑτούς.
\par }{\PP \VS{13}Πιλᾶτος δὲ συνκαλεσάμενος τοὺς ἀρχιερεῖς καὶ τοὺς ἄρχοντας καὶ τὸν λαὸν
\VS{14}εἶπεν πρὸς αὐτούς· Προσηνέγκατέ μοι τὸν ἄνθρωπον τοῦτον ὡς ἀποστρέφοντα τὸν λαόν, καὶ ἰδοὺ ἐγὼ ἐνώπιον ὑμῶν ἀνακρίνας οὐθὲν εὗρον ἐν τῷ ἀνθρώπῳ τούτῳ αἴτιον ὧν κατηγορεῖτε κατ᾽ αὐτοῦ.
\VS{15}ἀλλ᾽ οὐδὲ Ἡρῴδης, ἀνέπεμψεν γὰρ αὐτὸν πρὸς ἡμᾶς, καὶ ἰδοὺ οὐδὲν ἄξιον θανάτου ἐστὶν πεπραγμένον αὐτῷ·
\VS{16}παιδεύσας οὖν αὐτὸν ἀπολύσω.
\par }{\PP \VS{18}Ἀνέκραγον δὲ παμπληθεὶ λέγοντες· Αἶρε τοῦτον, ἀπόλυσον δὲ ἡμῖν τὸν Βαραββᾶν·
\VS{19}ὅστις ἦν διὰ στάσιν τινὰ γενομένην ἐν τῇ πόλει καὶ φόνον βληθεὶς ἐν τῇ φυλακῇ.
\VS{20}Πάλιν δὲ ὁ Πιλᾶτος προσεφώνησεν αὐτοῖς θέλων ἀπολῦσαι τὸν Ἰησοῦν.
\VS{21}οἱ δὲ ἐπεφώνουν λέγοντες· Σταύρου σταύρου αὐτόν.
\VS{22}Ὁ δὲ τρίτον εἶπεν πρὸς αὐτούς· Τί γὰρ κακὸν ἐποίησεν οὗτος; οὐδὲν αἴτιον θανάτου εὗρον ἐν αὐτῷ· παιδεύσας οὖν αὐτὸν ἀπολύσω.
\VS{23}Οἱ δὲ ἐπέκειντο φωναῖς μεγάλαις αἰτούμενοι αὐτὸν σταυρωθῆναι, καὶ κατίσχυον αἱ φωναὶ αὐτῶν.
\VS{24}καὶ Πιλᾶτος ἐπέκρινεν γενέσθαι τὸ αἴτημα αὐτῶν·
\VS{25}ἀπέλυσεν δὲ τὸν διὰ στάσιν καὶ φόνον βεβλημένον εἰς φυλακὴν ὃν ᾐτοῦντο, τὸν δὲ Ἰησοῦν παρέδωκεν τῷ θελήματι αὐτῶν.
\par }{\PP \VS{26}Καὶ ὡς ἀπήγαγον αὐτόν, ἐπιλαβόμενοι Σίμωνά τινα Κυρηναῖον ἐρχόμενον ἀπ᾽ ἀγροῦ ἐπέθηκαν αὐτῷ τὸν σταυρὸν φέρειν ὄπισθεν τοῦ Ἰησοῦ.
\par }{\PP \VS{27}Ἠκολούθει δὲ αὐτῷ πολὺ πλῆθος τοῦ λαοῦ καὶ γυναικῶν αἳ ἐκόπτοντο καὶ ἐθρήνουν αὐτόν.
\VS{28}στραφεὶς δὲ πρὸς αὐτὰς ὁ Ἰησοῦς εἶπεν· Θυγατέρες Ἰερουσαλήμ, μὴ κλαίετε ἐπ᾽ ἐμέ· πλὴν ἐφ᾽ ἑαυτὰς κλαίετε καὶ ἐπὶ τὰ τέκνα ὑμῶν,
\VS{29}ὅτι ἰδοὺ ἔρχονται ἡμέραι ἐν αἷς ἐροῦσιν· Μακάριαι αἱ στεῖραι καὶ αἱ κοιλίαι αἳ οὐκ ἐγέννησαν καὶ μαστοὶ οἳ οὐκ ἔθρεψαν.
\VS{30}τότε Ἄρξονται λέγειν τοῖς ὄρεσιν· Πέσετε ἐφ᾽ ἡμᾶς, Καὶ τοῖς βουνοῖς· Καλύψατε ἡμᾶς·
\VS{31}Ὅτι εἰ ἐν τῷ ὑγρῷ ξύλῳ ταῦτα ποιοῦσιν, ἐν τῷ ξηρῷ τί γένηται;
\par }{\PP \VS{32}Ἤγοντο δὲ καὶ ἕτεροι κακοῦργοι δύο σὺν αὐτῷ ἀναιρεθῆναι.
\par }{\PP \VS{33}Καὶ ὅτε ἦλθον ἐπὶ τὸν τόπον τὸν καλούμενον Κρανίον, ἐκεῖ ἐσταύρωσαν αὐτὸν καὶ τοὺς κακούργους, ὃν μὲν ἐκ δεξιῶν ὃν δὲ ἐξ ἀριστερῶν.
\VS{34}Ὁ δὲ Ἰησοῦς ἔλεγεν· Πάτερ, ἄφες αὐτοῖς, οὐ γὰρ οἴδασιν τί ποιοῦσιν. διαμεριζόμενοι δὲ τὰ ἱμάτια αὐτοῦ ἔβαλον κλήρους.
\VS{35}Καὶ εἱστήκει ὁ λαὸς θεωρῶν. ἐξεμυκτήριζον δὲ καὶ οἱ ἄρχοντες λέγοντες· Ἄλλους ἔσωσεν, σωσάτω ἑαυτόν, εἰ οὗτός ἐστιν ὁ Χριστὸς τοῦ Θεοῦ ὁ ἐκλεκτός.
\VS{36}Ἐνέπαιξαν δὲ αὐτῷ καὶ οἱ στρατιῶται προσερχόμενοι, ὄξος προσφέροντες αὐτῷ
\VS{37}καὶ λέγοντες· Εἰ σὺ εἶ ὁ Βασιλεὺς τῶν Ἰουδαίων, σῶσον σεαυτόν.
\VS{38}Ἦν δὲ καὶ ἐπιγραφὴ ἐπ᾽ αὐτῷ· 
\begin{poetryblock}
\par }{\PP \begin{quote}Ο ΒΑΣΙΛΕΥΣ ΤΩΝ ΙΟΥΔΑΙΩΝ ΟΥΤΟΣ.\end{quote}
\end{poetryblock}
\par }{\PP \VS{39}Εἷς δὲ τῶν κρεμασθέντων κακούργων ἐβλασφήμει αὐτόν λέγων· Οὐχὶ σὺ εἶ ὁ Χριστός; σῶσον σεαυτὸν καὶ ἡμᾶς.
\VS{40}Ἀποκριθεὶς δὲ ὁ ἕτερος ἐπιτιμῶν αὐτῷ ἔφη· Οὐδὲ φοβῇ σὺ τὸν Θεόν, ὅτι ἐν τῷ αὐτῷ κρίματι εἶ;
\VS{41}καὶ ἡμεῖς μὲν δικαίως, ἄξια γὰρ ὧν ἐπράξαμεν ἀπολαμβάνομεν· οὗτος δὲ οὐδὲν ἄτοπον ἔπραξεν.
\VS{42}καὶ ἔλεγεν· Ἰησοῦ, μνήσθητί μου ὅταν ἔλθῃς εἰς τὴν βασιλείαν σου.
\VS{43}Καὶ εἶπεν αὐτῷ· Ἀμήν σοι λέγω, σήμερον μετ᾽ ἐμοῦ ἔσῃ ἐν τῷ Παραδείσῳ.
\par }{\PP \VS{44}Καὶ ἦν ἤδη ὡσεὶ ὥρα ἕκτη καὶ σκότος ἐγένετο ἐφ᾽ ὅλην τὴν γῆν ἕως ὥρας ἐνάτης
\VS{45}τοῦ ἡλίου ἐκλιπόντος, ἐσχίσθη δὲ τὸ καταπέτασμα τοῦ ναοῦ μέσον.
\VS{46}Καὶ φωνήσας φωνῇ μεγάλῃ ὁ Ἰησοῦς εἶπεν· Πάτερ, εἰς χεῖράς σου παρατίθεμαι τὸ πνεῦμά μου. τοῦτο δὲ εἰπὼν ἐξέπνευσεν.
\par }{\PP \VS{47}Ἰδὼν δὲ ὁ ἑκατοντάρχης τὸ γενόμενον ἐδόξαζεν τὸν Θεὸν λέγων· Ὄντως ὁ ἄνθρωπος οὗτος δίκαιος ἦν.
\VS{48}καὶ πάντες οἱ συμπαραγενόμενοι ὄχλοι ἐπὶ τὴν θεωρίαν ταύτην, θεωρήσαντες τὰ γενόμενα, τύπτοντες τὰ στήθη ὑπέστρεφον.
\VS{49}εἱστήκεισαν δὲ πάντες οἱ γνωστοὶ αὐτῷ ἀπὸ μακρόθεν καὶ γυναῖκες αἱ συνακολουθοῦσαι αὐτῷ ἀπὸ τῆς Γαλιλαίας ὁρῶσαι ταῦτα.
\par }{\PP \VS{50}Καὶ ἰδοὺ ἀνὴρ ὀνόματι Ἰωσὴφ βουλευτὴς ὑπάρχων καὶ ἀνὴρ ἀγαθὸς καὶ δίκαιος—
\VS{51}οὗτος οὐκ ἦν συνκατατεθειμένος τῇ βουλῇ καὶ τῇ πράξει αὐτῶν— ἀπὸ Ἁριμαθαίας πόλεως τῶν Ἰουδαίων, ὃς προσεδέχετο τὴν βασιλείαν τοῦ Θεοῦ,
\VS{52}οὗτος προσελθὼν τῷ Πιλάτῳ ᾐτήσατο τὸ σῶμα τοῦ Ἰησοῦ
\VS{53}καὶ καθελὼν ἐνετύλιξεν αὐτὸ σινδόνι καὶ ἔθηκεν αὐτὸν ἐν μνήματι λαξευτῷ οὗ οὐκ ἦν οὐδεὶς οὔπω κείμενος.
\VS{54}καὶ ἡμέρα ἦν Παρασκευῆς καὶ σάββατον ἐπέφωσκεν.
\par }{\PP \VS{55}Κατακολουθήσασαι δὲ αἱ γυναῖκες, αἵτινες ἦσαν συνεληλυθυῖαι ἐκ τῆς Γαλιλαίας αὐτῷ, ἐθεάσαντο τὸ μνημεῖον καὶ ὡς ἐτέθη τὸ σῶμα αὐτοῦ,
\VS{56}ὑποστρέψασαι δὲ ἡτοίμασαν ἀρώματα καὶ μύρα. Καὶ τὸ μὲν σάββατον ἡσύχασαν κατὰ τὴν ἐντολήν.

\par }\Chap{24}{\PP \VerseOne{1}Τῇ δὲ μιᾷ τῶν σαββάτων ὄρθρου βαθέως ἐπὶ τὸ μνῆμα ἦλθον φέρουσαι ἃ ἡτοίμασαν ἀρώματα.
\VS{2}εὗρον δὲ τὸν λίθον ἀποκεκυλισμένον ἀπὸ τοῦ μνημείου,
\VS{3}εἰσελθοῦσαι δὲ οὐχ εὗρον τὸ σῶμα τοῦ Κυρίου Ἰησοῦ.
\VS{4}καὶ ἐγένετο ἐν τῷ ἀπορεῖσθαι αὐτὰς περὶ τούτου καὶ ἰδοὺ ἄνδρες δύο ἐπέστησαν αὐταῖς ἐν ἐσθῆτι ἀστραπτούσῃ.
\VS{5}ἐμφόβων δὲ γενομένων αὐτῶν καὶ κλινουσῶν τὰ πρόσωπα εἰς τὴν γῆν εἶπαν πρὸς αὐτάς· Τί ζητεῖτε τὸν ζῶντα μετὰ τῶν νεκρῶν;
\VS{6}οὐκ ἔστιν ὧδε, ἀλλὰ ἠγέρθη. μνήσθητε ὡς ἐλάλησεν ὑμῖν ἔτι ὢν ἐν τῇ Γαλιλαίᾳ
\VS{7}λέγων Τὸν Υἱὸν τοῦ ἀνθρώπου ὅτι δεῖ παραδοθῆναι εἰς χεῖρας ἀνθρώπων ἁμαρτωλῶν καὶ σταυρωθῆναι καὶ τῇ τρίτῃ ἡμέρᾳ ἀναστῆναι.
\VS{8}Καὶ ἐμνήσθησαν τῶν ῥημάτων αὐτοῦ.
\par }{\PP \VS{9}καὶ ὑποστρέψασαι ἀπὸ τοῦ μνημείου ἀπήγγειλαν ταῦτα πάντα τοῖς ἕνδεκα καὶ πᾶσιν τοῖς λοιποῖς.
\VS{10}ἦσαν δὲ ἡ Μαγδαληνὴ Μαρία καὶ Ἰωάννα καὶ Μαρία ἡ Ἰακώβου καὶ αἱ λοιπαὶ σὺν αὐταῖς. ἔλεγον πρὸς τοὺς ἀποστόλους ταῦτα,
\VS{11}καὶ ἐφάνησαν ἐνώπιον αὐτῶν ὡσεὶ λῆρος τὰ ῥήματα ταῦτα, καὶ ἠπίστουν αὐταῖς.
\VS{12}Ὁ δὲ Πέτρος ἀναστὰς ἔδραμεν ἐπὶ τὸ μνημεῖον καὶ παρακύψας βλέπει τὰ ὀθόνια μόνα, καὶ ἀπῆλθεν πρὸς ἑαυτὸν θαυμάζων τὸ γεγονός.
\par }{\PP \VS{13}Καὶ ἰδοὺ δύο ἐξ αὐτῶν ἐν αὐτῇ τῇ ἡμέρᾳ ἦσαν πορευόμενοι εἰς κώμην ἀπέχουσαν σταδίους ἑξήκοντα ἀπὸ Ἰερουσαλήμ, ᾗ ὄνομα Ἐμμαοῦς,
\VS{14}καὶ αὐτοὶ ὡμίλουν πρὸς ἀλλήλους περὶ πάντων τῶν συμβεβηκότων τούτων.
\VS{15}καὶ ἐγένετο ἐν τῷ ὁμιλεῖν αὐτοὺς καὶ συζητεῖν καὶ αὐτὸς Ἰησοῦς ἐγγίσας συνεπορεύετο αὐτοῖς,
\VS{16}οἱ δὲ ὀφθαλμοὶ αὐτῶν ἐκρατοῦντο τοῦ μὴ ἐπιγνῶναι αὐτόν.
\VS{17}Εἶπεν δὲ πρὸς αὐτούς· Τίνες οἱ λόγοι οὗτοι οὓς ἀντιβάλλετε πρὸς ἀλλήλους περιπατοῦντες; Καὶ ἐστάθησαν σκυθρωποί.
\VS{18}ἀποκριθεὶς δὲ εἷς ὀνόματι Κλεοπᾶς εἶπεν πρὸς αὐτόν· Σὺ μόνος παροικεῖς Ἰερουσαλὴμ καὶ οὐκ ἔγνως τὰ γενόμενα ἐν αὐτῇ ἐν ταῖς ἡμέραις ταύταις;
\VS{19}Καὶ εἶπεν αὐτοῖς· Ποῖα; οἱ Δὲ εἶπαν αὐτῷ· Τὰ περὶ Ἰησοῦ τοῦ Ναζαρηνοῦ, ὃς ἐγένετο ἀνὴρ προφήτης δυνατὸς ἐν ἔργῳ καὶ λόγῳ ἐναντίον τοῦ Θεοῦ καὶ παντὸς τοῦ λαοῦ,
\VS{20}ὅπως τε παρέδωκαν αὐτὸν οἱ ἀρχιερεῖς καὶ οἱ ἄρχοντες ἡμῶν εἰς κρίμα θανάτου καὶ ἐσταύρωσαν αὐτόν.
\VS{21}ἡμεῖς δὲ ἠλπίζομεν ὅτι αὐτός ἐστιν ὁ μέλλων λυτροῦσθαι τὸν Ἰσραήλ· ἀλλά γε καὶ σὺν πᾶσιν τούτοις τρίτην ταύτην ἡμέραν ἄγει ἀφ᾽ οὗ ταῦτα ἐγένετο.
\VS{22}Ἀλλὰ καὶ γυναῖκές τινες ἐξ ἡμῶν ἐξέστησαν ἡμᾶς, γενόμεναι ὀρθριναὶ ἐπὶ τὸ μνημεῖον,
\VS{23}καὶ μὴ εὑροῦσαι τὸ σῶμα αὐτοῦ ἦλθον λέγουσαι καὶ ὀπτασίαν ἀγγέλων ἑωρακέναι, οἳ λέγουσιν αὐτὸν ζῆν.
\VS{24}καὶ ἀπῆλθόν τινες τῶν σὺν ἡμῖν ἐπὶ τὸ μνημεῖον καὶ εὗρον οὕτως καθὼς καὶ αἱ γυναῖκες εἶπον, αὐτὸν δὲ οὐκ εἶδον.
\VS{25}Καὶ αὐτὸς εἶπεν πρὸς αὐτούς· Ὦ ἀνόητοι καὶ βραδεῖς τῇ καρδίᾳ τοῦ πιστεύειν ἐπὶ πᾶσιν οἷς ἐλάλησαν οἱ προφῆται·
\VS{26}οὐχὶ ταῦτα ἔδει παθεῖν τὸν Χριστὸν καὶ εἰσελθεῖν εἰς τὴν δόξαν αὐτοῦ;
\VS{27}καὶ ἀρξάμενος ἀπὸ Μωϋσέως καὶ ἀπὸ πάντων τῶν προφητῶν διερμήνευσεν αὐτοῖς ἐν πάσαις ταῖς γραφαῖς τὰ περὶ ἑαυτοῦ.
\par }{\PP \VS{28}Καὶ ἤγγισαν εἰς τὴν κώμην οὗ ἐπορεύοντο, καὶ αὐτὸς προσεποιήσατο πορρώτερον πορεύεσθαι.
\VS{29}καὶ παρεβιάσαντο αὐτὸν λέγοντες· Μεῖνον μεθ᾽ ἡμῶν, ὅτι πρὸς ἑσπέραν ἐστὶν καὶ κέκλικεν ἤδη ἡ ἡμέρα. Καὶ εἰσῆλθεν τοῦ μεῖναι σὺν αὐτοῖς.
\VS{30}καὶ ἐγένετο ἐν τῷ κατακλιθῆναι αὐτὸν μετ᾽ αὐτῶν λαβὼν τὸν ἄρτον εὐλόγησεν καὶ κλάσας ἐπεδίδου αὐτοῖς,
\VS{31}αὐτῶν δὲ διηνοίχθησαν οἱ ὀφθαλμοὶ καὶ ἐπέγνωσαν αὐτόν· καὶ αὐτὸς ἄφαντος ἐγένετο ἀπ᾽ αὐτῶν.
\VS{32}Καὶ εἶπαν πρὸς ἀλλήλους· Οὐχὶ ἡ καρδία ἡμῶν καιομένη ἦν ἐν ἡμῖν ὡς ἐλάλει ἡμῖν ἐν τῇ ὁδῷ, ὡς διήνοιγεν ἡμῖν τὰς γραφάς;
\par }{\PP \VS{33}Καὶ ἀναστάντες αὐτῇ τῇ ὥρᾳ ὑπέστρεψαν εἰς Ἰερουσαλήμ καὶ εὗρον ἠθροισμένους τοὺς ἕνδεκα καὶ τοὺς σὺν αὐτοῖς,
\VS{34}λέγοντας ὅτι Ὄντως ἠγέρθη ὁ Κύριος καὶ ὤφθη Σίμωνι.
\VS{35}Καὶ αὐτοὶ ἐξηγοῦντο τὰ ἐν τῇ ὁδῷ καὶ ὡς ἐγνώσθη αὐτοῖς ἐν τῇ κλάσει τοῦ ἄρτου.
\par }{\PP \VS{36}Ταῦτα δὲ αὐτῶν λαλούντων αὐτὸς ἔστη ἐν μέσῳ αὐτῶν καὶ λέγει αὐτοῖς· Εἰρήνη ὑμῖν.
\VS{37}πτοηθέντες δὲ καὶ ἔμφοβοι γενόμενοι ἐδόκουν πνεῦμα θεωρεῖν.
\VS{38}Καὶ εἶπεν αὐτοῖς· Τί τεταραγμένοι ἐστέ καὶ διὰ τί διαλογισμοὶ ἀναβαίνουσιν ἐν τῇ καρδίᾳ ὑμῶν;
\VS{39}ἴδετε τὰς χεῖράς μου καὶ τοὺς πόδας μου ὅτι ἐγώ εἰμι αὐτός· ψηλαφήσατέ με καὶ ἴδετε, ὅτι πνεῦμα σάρκα καὶ ὀστέα οὐκ ἔχει καθὼς ἐμὲ θεωρεῖτε ἔχοντα.
\VS{40}καὶ τοῦτο εἰπὼν ἔδειξεν αὐτοῖς τὰς χεῖρας καὶ τοὺς πόδας.
\VS{41}Ἔτι δὲ ἀπιστούντων αὐτῶν ἀπὸ τῆς χαρᾶς καὶ θαυμαζόντων εἶπεν αὐτοῖς· Ἔχετέ τι βρώσιμον ἐνθάδε;
\VS{42}οἱ δὲ ἐπέδωκαν αὐτῷ ἰχθύος ὀπτοῦ μέρος·
\VS{43}καὶ λαβὼν ἐνώπιον αὐτῶν ἔφαγεν.
\par }{\PP \VS{44}Εἶπεν δὲ πρὸς αὐτούς· Οὗτοι οἱ λόγοι μου οὓς ἐλάλησα πρὸς ὑμᾶς ἔτι ὢν σὺν ὑμῖν, ὅτι δεῖ πληρωθῆναι πάντα τὰ γεγραμμένα ἐν τῷ νόμῳ Μωϋσέως καὶ τοῖς προφήταις καὶ ψαλμοῖς περὶ ἐμοῦ.
\VS{45}τότε διήνοιξεν αὐτῶν τὸν νοῦν τοῦ συνιέναι τὰς γραφάς·
\VS{46}Καὶ εἶπεν αὐτοῖς ὅτι Οὕτως γέγραπται παθεῖν τὸν Χριστὸν καὶ ἀναστῆναι ἐκ νεκρῶν τῇ τρίτῃ ἡμέρᾳ,
\VS{47}καὶ κηρυχθῆναι ἐπὶ τῷ ὀνόματι αὐτοῦ μετάνοιαν εἰς ἄφεσιν ἁμαρτιῶν εἰς πάντα τὰ ἔθνη. ἀρξάμενοι ἀπὸ Ἰερουσαλήμ
\VS{48}ὑμεῖς μάρτυρες τούτων.
\VS{49}Καὶ ἰδοὺ ἐγὼ ἀποστέλλω τὴν ἐπαγγελίαν τοῦ Πατρός μου ἐφ᾽ ὑμᾶς· ὑμεῖς δὲ καθίσατε ἐν τῇ πόλει ἕως οὗ ἐνδύσησθε ἐξ ὕψους δύναμιν.
\par }{\PP \VS{50}Ἐξήγαγεν δὲ αὐτοὺς ἔξω ἕως πρὸς Βηθανίαν, καὶ ἐπάρας τὰς χεῖρας αὐτοῦ εὐλόγησεν αὐτούς.
\VS{51}καὶ ἐγένετο ἐν τῷ εὐλογεῖν αὐτὸν αὐτοὺς διέστη ἀπ᾽ αὐτῶν καὶ ἀνεφέρετο εἰς τὸν οὐρανόν.
\par }{\PP \VS{52}καὶ αὐτοὶ προσκυνήσαντες αὐτὸν ὑπέστρεψαν εἰς Ἰερουσαλὴμ μετὰ χαρᾶς μεγάλης
\VS{53}καὶ ἦσαν διὰ παντὸς ἐν τῷ ἱερῷ εὐλογοῦντες τὸν Θεόν.
\par }