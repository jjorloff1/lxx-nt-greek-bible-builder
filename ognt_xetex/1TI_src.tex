\NormalFont\ShortTitle{ΠΡΟΣ ΤΙΜΟΘΕΟΝ Α}
{\MT ΠΡΟΣ ΤΙΜΟΘΕΟΝ Α

\par }\ChapOne{1}{\PP \VerseOne{1}Παῦλος ἀπόστολος Χριστοῦ Ἰησοῦ κατ᾽ ἐπιταγὴν Θεοῦ Σωτῆρος ἡμῶν καὶ Χριστοῦ Ἰησοῦ τῆς ἐλπίδος ἡμῶν
\VS{2}Τιμοθέῳ γνησίῳ τέκνῳ ἐν πίστει, Χάρις ἔλεος εἰρήνη ἀπὸ Θεοῦ πατρὸς καὶ Χριστοῦ Ἰησοῦ τοῦ κυρίου ἡμῶν.
\par }{\PP \VS{3}Καθὼς παρεκάλεσά σε προσμεῖναι ἐν Ἐφέσῳ πορευόμενος εἰς Μακεδονίαν, ἵνα παραγγείλῃς τισὶν μὴ ἑτεροδιδασκαλεῖν
\VS{4}μηδὲ προσέχειν μύθοις καὶ γενεαλογίαις ἀπεράντοις, αἵτινες ἐκζητήσεις παρέχουσιν μᾶλλον ἢ οἰκονομίαν Θεοῦ τὴν ἐν πίστει.
\VS{5}Τὸ δὲ τέλος τῆς παραγγελίας ἐστὶν ἀγάπη ἐκ καθαρᾶς καρδίας καὶ συνειδήσεως ἀγαθῆς καὶ πίστεως ἀνυποκρίτου,
\VS{6}ὧν τινες ἀστοχήσαντες ἐξετράπησαν εἰς ματαιολογίαν
\VS{7}θέλοντες εἶναι νομοδιδάσκαλοι, μὴ νοοῦντες μήτε ἃ λέγουσιν μήτε περὶ τίνων διαβεβαιοῦνται.
\VS{8}Οἴδαμεν δὲ ὅτι καλὸς ὁ νόμος, ἐάν τις αὐτῷ νομίμως χρῆται,
\VS{9}εἰδὼς τοῦτο, ὅτι δικαίῳ νόμος οὐ κεῖται, ἀνόμοις δὲ καὶ ἀνυποτάκτοις, ἀσεβέσι= καὶ ἁμαρτωλοῖς, ἀνοσίοις καὶ βεβήλοις, πατρολῴαις καὶ μητρολῴαις, ἀνδροφόνοις
\VS{10}πόρνοις ἀρσενοκοίταις ἀνδραποδισταῖς ψεύσταις ἐπιόρκοις, καὶ εἴ τι ἕτερον τῇ ὑγιαινούσῃ διδασκαλίᾳ ἀντίκειται
\VS{11}κατὰ τὸ εὐαγγέλιον τῆς δόξης τοῦ μακαρίου Θεοῦ, ὃ ἐπιστεύθην ἐγώ.
\par }{\PP \VS{12}Χάριν ἔχω τῷ ἐνδυναμώσαντί με Χριστῷ Ἰησοῦ τῷ Κυρίῳ ἡμῶν, ὅτι πιστόν με ἡγήσατο θέμενος εἰς διακονίαν
\VS{13}τὸ πρότερον ὄντα βλάσφημον καὶ διώκτην καὶ ὑβριστήν, ἀλλὰ= ἠλεήθην, ὅτι ἀγνοῶν ἐποίησα ἐν ἀπιστίᾳ·
\VS{14}ὑπερεπλεόνασεν δὲ ἡ χάρις τοῦ Κυρίου ἡμῶν μετὰ πίστεως καὶ ἀγάπης τῆς ἐν Χριστῷ Ἰησοῦ.
\VS{15}Πιστὸς ὁ λόγος καὶ πάσης ἀποδοχῆς ἄξιος, ὅτι Χριστὸς Ἰησοῦς ἦλθεν εἰς τὸν κόσμον ἁμαρτωλοὺς σῶσαι, ὧν πρῶτός εἰμι ἐγώ.
\VS{16}ἀλλὰ διὰ τοῦτο ἠλεήθην, ἵνα ἐν ἐμοὶ πρώτῳ ἐνδείξηται Χριστὸς Ἰησοῦς τὴν ἅπασαν μακροθυμίαν πρὸς ὑποτύπωσιν τῶν μελλόντων πιστεύειν ἐπ᾽ αὐτῷ εἰς ζωὴν αἰώνιον.
\VS{17}Τῷ δὲ Βασιλεῖ τῶν αἰώνων, ἀφθάρτῳ ἀοράτῳ μόνῳ Θεῷ, τιμὴ καὶ δόξα εἰς τοὺς αἰῶνας τῶν αἰώνων, ἀμήν.
\par }{\PP \VS{18}Ταύτην τὴν παραγγελίαν παρατίθεμαί σοι, τέκνον Τιμόθεε, κατὰ τὰς προαγούσας ἐπὶ σὲ προφητείας, ἵνα στρατεύῃ ἐν αὐταῖς τὴν καλὴν στρατείαν
\VS{19}ἔχων πίστιν καὶ ἀγαθὴν συνείδησιν, ἥν τινες ἀπωσάμενοι περὶ τὴν πίστιν ἐναυάγησαν,
\VS{20}ὧν ἐστιν Ὑμέναιος καὶ Ἀλέξανδρος, οὓς παρέδωκα τῷ Σατανᾷ, ἵνα παιδευθῶσιν μὴ βλασφημεῖν.

\par }\Chap{2}{\PP \VerseOne{1}Παρακαλῶ οὖν πρῶτον πάντων ποιεῖσθαι δεήσεις προσευχάς ἐντεύξεις εὐχαριστίας ὑπὲρ πάντων ἀνθρώπων,
\VS{2}ὑπὲρ βασιλέων καὶ πάντων τῶν ἐν ὑπεροχῇ ὄντων, ἵνα ἤρεμον καὶ ἡσύχιον βίον διάγωμεν ἐν πάσῃ εὐσεβείᾳ καὶ σεμνότητι.
\VS{3}τοῦτο καλὸν καὶ ἀπόδεκτον ἐνώπιον τοῦ Σωτῆρος ἡμῶν Θεοῦ,
\VS{4}ὃς πάντας ἀνθρώπους θέλει σωθῆναι καὶ εἰς ἐπίγνωσιν ἀληθείας ἐλθεῖν.
\begin{poetryblock}
\par }{\PP \begin{quote} \VS{5}Εἷς γὰρ Θεός,\end{quote} 
\par }{\PP \begin{quote}εἷς καὶ μεσίτης Θεοῦ καὶ ἀνθρώπων,\end{quote} 
\par }{\PP \begin{quote}ἄνθρωπος Χριστὸς Ἰησοῦς,\end{quote}
\par }{\PP \begin{quote} \VS{6}ὁ δοὺς ἑαυτὸν ἀντίλυτρον ὑπὲρ πάντων,\end{quote} 
\par }{\PP \begin{quote}τὸ μαρτύριον καιροῖς ἰδίοις.\end{quote}
\end{poetryblock}
\VS{7}εἰς ὃ ἐτέθην ἐγὼ κῆρυξ καὶ ἀπόστολος, ἀλήθειαν λέγω οὐ ψεύδομαι, διδάσκαλος ἐθνῶν ἐν πίστει καὶ ἀληθείᾳ.
\par }{\PP \VS{8}Βούλομαι οὖν προσεύχεσθαι τοὺς ἄνδρας ἐν παντὶ τόπῳ ἐπαίροντας ὁσίους χεῖρας χωρὶς ὀργῆς καὶ διαλογισμοῦ.
\VS{9}Ὡσαύτως καὶ γυναῖκας ἐν καταστολῇ κοσμίῳ μετὰ αἰδοῦς καὶ σωφροσύνης κοσμεῖν ἑαυτάς, μὴ ἐν πλέγμασιν καὶ χρυσίῳ ἢ μαργαρίταις ἢ ἱματισμῷ πολυτελεῖ,
\VS{10}ἀλλ᾽ ὃ πρέπει γυναιξὶν ἐπαγγελλομέναις θεοσέβειαν, δι᾽ ἔργων ἀγαθῶν.
\VS{11}Γυνὴ ἐν ἡσυχίᾳ μανθανέτω ἐν πάσῃ ὑποταγῇ·
\VS{12}διδάσκειν δὲ γυναικὶ οὐκ ἐπιτρέπω οὐδὲ αὐθεντεῖν ἀνδρός, ἀλλ᾽ εἶναι ἐν ἡσυχίᾳ.
\VS{13}Ἀδὰμ γὰρ πρῶτος ἐπλάσθη, εἶτα Εὕα.
\VS{14}καὶ Ἀδὰμ οὐκ ἠπατήθη, ἡ δὲ γυνὴ ἐξαπατηθεῖσα ἐν παραβάσει γέγονεν·
\VS{15}σωθήσεται δὲ διὰ τῆς τεκνογονίας, ἐὰν μείνωσιν ἐν πίστει καὶ ἀγάπῃ καὶ ἁγιασμῷ μετὰ σωφροσύνης·

\par }\Chap{3}{\PP \VerseOne{1}Πιστὸς ὁ λόγος.
\par }{\PP εἴ τις ἐπισκοπῆς ὀρέγεται, καλοῦ ἔργου ἐπιθυμεῖ.
\VS{2}δεῖ οὖν τὸν ἐπίσκοπον ἀνεπίλημπτον εἶναι, μιᾶς γυναικὸς ἄνδρα, νηφάλιον σώφρονα κόσμιον φιλόξενον διδακτικόν,
\VS{3}μὴ πάροινον μὴ πλήκτην, ἀλλὰ= ἐπιεικῆ ἄμαχον ἀφιλάργυρον,
\VS{4}τοῦ ἰδίου οἴκου καλῶς προϊστάμενον, τέκνα ἔχοντα ἐν ὑποταγῇ, μετὰ πάσης σεμνότητος
\VS{5}εἰ δέ τις τοῦ ἰδίου οἴκου προστῆναι οὐκ οἶδεν, πῶς ἐκκλησίας Θεοῦ ἐπιμελήσεται;,
\VS{6}μὴ νεόφυτον, ἵνα μὴ τυφωθεὶς εἰς κρίμα ἐμπέσῃ τοῦ διαβόλου.
\VS{7}δεῖ δὲ καὶ μαρτυρίαν καλὴν ἔχειν ἀπὸ τῶν ἔξωθεν, ἵνα μὴ εἰς ὀνειδισμὸν ἐμπέσῃ καὶ παγίδα τοῦ διαβόλου.
\par }{\PP \VS{8}Διακόνους ὡσαύτως σεμνούς, μὴ διλόγους, μὴ οἴνῳ πολλῷ προσέχοντας, μὴ αἰσχροκερδεῖς,
\VS{9}ἔχοντας τὸ μυστήριον τῆς πίστεως ἐν καθαρᾷ συνειδήσει.
\VS{10}καὶ οὗτοι δὲ δοκιμαζέσθωσαν πρῶτον, εἶτα διακονείτωσαν ἀνέγκλητοι ὄντες.
\VS{11}Γυναῖκας ὡσαύτως σεμνάς, μὴ διαβόλους, νηφαλίους, πιστὰς ἐν πᾶσιν.
\VS{12}Διάκονοι ἔστωσαν μιᾶς γυναικὸς ἄνδρες, τέκνων καλῶς προϊστάμενοι καὶ τῶν ἰδίων οἴκων.
\VS{13}οἱ γὰρ καλῶς διακονήσαντες βαθμὸν ἑαυτοῖς καλὸν περιποιοῦνται καὶ πολλὴν παρρησίαν ἐν πίστει τῇ ἐν Χριστῷ Ἰησοῦ.
\par }{\PP \VS{14}Ταῦτά σοι γράφω ἐλπίζων ἐλθεῖν πρὸς σὲ ἐν τάχει·
\VS{15}ἐὰν δὲ βραδύνω, ἵνα εἰδῇς πῶς δεῖ ἐν οἴκῳ Θεοῦ ἀναστρέφεσθαι, ἥτις ἐστὶν ἐκκλησία Θεοῦ ζῶντος, στῦλος καὶ ἑδραίωμα τῆς ἀληθείας.
\VS{16}Καὶ ὁμολογουμένως μέγα ἐστὶν τὸ τῆς εὐσεβείας μυστήριον· 
\begin{poetryblock}
\par }{\PP \begin{quote}Ὃς ἐφανερώθη ἐν σαρκί,\end{quote} 
\par }{\PP \begin{quote}ἐδικαιώθη ἐν πνεύματι,\end{quote} 
\par }{\PP \begin{quote}ὤφθη ἀγγέλοις,\end{quote} 
\par }{\PP \begin{quote}ἐκηρύχθη ἐν ἔθνεσιν,\end{quote} 
\par }{\PP \begin{quote}ἐπιστεύθη ἐν κόσμῳ,\end{quote} 
\par }{\PP \begin{quote}ἀνελήμφθη ἐν δόξῃ.\end{quote}
\end{poetryblock}

\par }\Chap{4}{\PP \VerseOne{1}Τὸ δὲ πνεῦμα ῥητῶς λέγει ὅτι ἐν ὑστέροις καιροῖς ἀποστήσονταί τινες τῆς πίστεως προσέχοντες πνεύμασιν πλάνοις καὶ διδασκαλίαις δαιμονίων,
\VS{2}ἐν ὑποκρίσει ψευδολόγων, κεκαυστηριασμένων τὴν ἰδίαν συνείδησιν,
\VS{3}κωλυόντων γαμεῖν, ἀπέχεσθαι βρωμάτων, ἃ ὁ Θεὸς ἔκτισεν εἰς μετάλημψιν μετὰ εὐχαριστίας τοῖς πιστοῖς καὶ ἐπεγνωκόσι= τὴν ἀλήθειαν.
\VS{4}ὅτι πᾶν κτίσμα Θεοῦ καλόν καὶ οὐδὲν ἀπόβλητον μετὰ εὐχαριστίας λαμβανόμενον·
\VS{5}ἁγιάζεται γὰρ διὰ λόγου Θεοῦ καὶ ἐντεύξεως.
\par }{\PP \VS{6}Ταῦτα ὑποτιθέμενος τοῖς ἀδελφοῖς καλὸς ἔσῃ διάκονος Χριστοῦ Ἰησοῦ, ἐντρεφόμενος τοῖς λόγοις τῆς πίστεως καὶ τῆς καλῆς διδασκαλίας ᾗ παρηκολούθηκας·
\VS{7}Τοὺς δὲ βεβήλους καὶ γραώδεις μύθους παραιτοῦ. γύμναζε δὲ σεαυτὸν πρὸς εὐσέβειαν·
\VS{8}ἡ γὰρ σωματικὴ γυμνασία πρὸς ὀλίγον ἐστὶν ὠφέλιμος, ἡ δὲ εὐσέβεια πρὸς πάντα ὠφέλιμός ἐστιν ἐπαγγελίαν ἔχουσα ζωῆς τῆς νῦν καὶ τῆς μελλούσης.
\VS{9}πιστὸς ὁ λόγος καὶ πάσης ἀποδοχῆς ἄξιος·
\VS{10}Εἰς τοῦτο γὰρ κοπιῶμεν καὶ ἀγωνιζόμεθα, ὅτι ἠλπίκαμεν ἐπὶ Θεῷ ζῶντι, ὅς ἐστιν Σωτὴρ πάντων ἀνθρώπων μάλιστα πιστῶν.
\par }{\PP \VS{11}Παράγγελλε ταῦτα καὶ δίδασκε.
\VS{12}Μηδείς σου τῆς νεότητος καταφρονείτω, ἀλλὰ τύπος γίνου τῶν πιστῶν ἐν λόγῳ, ἐν ἀναστροφῇ, ἐν ἀγάπῃ, ἐν πίστει, ἐν ἁγνείᾳ.
\VS{13}ἕως ἔρχομαι πρόσεχε τῇ ἀναγνώσει, τῇ παρακλήσει, τῇ διδασκαλίᾳ.
\VS{14}Μὴ ἀμέλει τοῦ ἐν σοὶ χαρίσματος, ὃ ἐδόθη σοι διὰ προφητείας μετὰ ἐπιθέσεως τῶν χειρῶν τοῦ πρεσβυτερίου.
\VS{15}ταῦτα μελέτα, ἐν τούτοις ἴσθι, ἵνα σου ἡ προκοπὴ φανερὰ ᾖ πᾶσιν.
\VS{16}ἔπεχε σεαυτῷ καὶ τῇ διδασκαλίᾳ, ἐπίμενε αὐτοῖς· τοῦτο γὰρ ποιῶν καὶ σεαυτὸν σώσεις καὶ τοὺς ἀκούοντάς σου.

\par }\Chap{5}{\PP \VerseOne{1}Πρεσβυτέρῳ μὴ ἐπιπλήξῃς ἀλλὰ παρακάλει ὡς πατέρα, νεωτέρους ὡς ἀδελφούς,
\VS{2}πρεσβυτέρας ὡς μητέρας, νεωτέρας ὡς ἀδελφὰς ἐν πάσῃ ἁγνείᾳ.
\par }{\PP \VS{3}Χήρας τίμα τὰς ὄντως χήρας.
\VS{4}εἰ δέ τις χήρα τέκνα ἢ ἔκγονα ἔχει, μανθανέτωσαν πρῶτον τὸν ἴδιον οἶκον εὐσεβεῖν καὶ ἀμοιβὰς ἀποδιδόναι τοῖς προγόνοις· τοῦτο γάρ ἐστιν ἀπόδεκτον ἐνώπιον τοῦ Θεοῦ.
\VS{5}Ἡ δὲ ὄντως χήρα καὶ μεμονωμένη ἤλπικεν ἐπὶ Θεὸν καὶ προσμένει ταῖς δεήσεσιν καὶ ταῖς προσευχαῖς νυκτὸς καὶ ἡμέρας,
\VS{6}ἡ δὲ σπαταλῶσα ζῶσα τέθνηκεν.
\VS{7}Καὶ ταῦτα παράγγελλε, ἵνα ἀνεπίλημπτοι ὦσιν.
\VS{8}εἰ δέ τις τῶν ἰδίων καὶ μάλιστα οἰκείων οὐ προνοεῖ, τὴν πίστιν ἤρνηται καὶ ἔστιν ἀπίστου χείρων.
\par }{\PP \VS{9}Χήρα καταλεγέσθω μὴ ἔλαττον ἐτῶν ἑξήκοντα γεγονυῖα, ἑνὸς ἀνδρὸς γυνή,
\VS{10}ἐν ἔργοις καλοῖς μαρτυρουμένη, εἰ ἐτεκνοτρόφησεν, εἰ ἐξενοδόχησεν, εἰ ἁγίων πόδας ἔνιψεν, εἰ θλιβομένοις ἐπήρκεσεν, εἰ παντὶ ἔργῳ ἀγαθῷ ἐπηκολούθησεν.
\VS{11}Νεωτέρας δὲ χήρας παραιτοῦ· ὅταν γὰρ καταστρηνιάσωσιν τοῦ Χριστοῦ, γαμεῖν θέλουσιν
\VS{12}ἔχουσαι κρίμα ὅτι τὴν πρώτην πίστιν ἠθέτησαν·
\VS{13}ἅμα δὲ καὶ ἀργαὶ μανθάνουσιν περιερχόμεναι τὰς οἰκίας, οὐ μόνον δὲ ἀργαὶ ἀλλὰ καὶ φλύαροι καὶ περίεργοι, λαλοῦσαι τὰ μὴ δέοντα.
\VS{14}Βούλομαι οὖν νεωτέρας γαμεῖν, τεκνογονεῖν, οἰκοδεσποτεῖν, μηδεμίαν ἀφορμὴν διδόναι τῷ ἀντικειμένῳ λοιδορίας χάριν·
\VS{15}ἤδη γάρ τινες ἐξετράπησαν ὀπίσω τοῦ Σατανᾶ.
\VS{16}Εἴ τις πιστὴ ἔχει χήρας, ἐπαρκείτω αὐταῖς καὶ μὴ βαρείσθω ἡ ἐκκλησία, ἵνα ταῖς ὄντως χήραις ἐπαρκέσῃ.
\par }{\PP \VS{17}Οἱ καλῶς προεστῶτες πρεσβύτεροι διπλῆς τιμῆς ἀξιούσθωσαν, μάλιστα οἱ κοπιῶντες ἐν λόγῳ καὶ διδασκαλίᾳ.
\VS{18}λέγει γὰρ ἡ γραφή· Βοῦν ἀλοῶντα οὐ φιμώσεις, καί· Ἄξιος ὁ ἐργάτης τοῦ μισθοῦ αὐτοῦ.
\VS{19}Κατὰ πρεσβυτέρου κατηγορίαν μὴ παραδέχου, ἐκτὸς εἰ μὴ ἐπὶ δύο ἢ τριῶν μαρτύρων.
\VS{20}Τοὺς ἁμαρτάνοντας ἐνώπιον πάντων ἔλεγχε, ἵνα καὶ οἱ λοιποὶ φόβον ἔχωσιν.
\VS{21}Διαμαρτύρομαι ἐνώπιον τοῦ Θεοῦ καὶ Χριστοῦ Ἰησοῦ καὶ τῶν ἐκλεκτῶν ἀγγέλων, ἵνα ταῦτα φυλάξῃς χωρὶς προκρίματος, μηδὲν ποιῶν κατὰ πρόσκλισιν.
\VS{22}Χεῖρας ταχέως μηδενὶ ἐπιτίθει μηδὲ κοινώνει ἁμαρτίαις ἀλλοτρίαις· σεαυτὸν ἁγνὸν τήρει.
\par }{\PP \VS{23}Μηκέτι ὑδροπότει, ἀλλὰ= οἴνῳ ὀλίγῳ χρῶ διὰ τὸν στόμαχον καὶ τὰς πυκνάς σου ἀσθενείας.
\VS{24}Τινῶν ἀνθρώπων αἱ ἁμαρτίαι πρόδηλοί εἰσιν προάγουσαι εἰς κρίσιν, τισὶν δὲ καὶ ἐπακολουθοῦσιν·
\VS{25}ὡσαύτως καὶ τὰ ἔργα τὰ καλὰ πρόδηλα, καὶ τὰ ἄλλως ἔχοντα κρυβῆναι οὐ δύνανται.

\par }\Chap{6}{\PP \VerseOne{1}Ὅσοι εἰσὶν ὑπὸ ζυγὸν δοῦλοι, τοὺς ἰδίους δεσπότας πάσης τιμῆς ἀξίους ἡγείσθωσαν, ἵνα μὴ τὸ ὄνομα τοῦ Θεοῦ καὶ ἡ διδασκαλία βλασφημῆται.
\VS{2}οἱ δὲ πιστοὺς ἔχοντες δεσπότας μὴ καταφρονείτωσαν, ὅτι ἀδελφοί εἰσιν, ἀλλὰ μᾶλλον δουλευέτωσαν, ὅτι πιστοί εἰσιν καὶ ἀγαπητοὶ οἱ τῆς εὐεργεσίας ἀντιλαμβανόμενοι.
\par }{\PP Ταῦτα δίδασκε καὶ παρακάλει.
\VS{3}Εἴ τις ἑτεροδιδασκαλεῖ καὶ μὴ προσέρχεται ὑγιαίνουσιν λόγοις τοῖς τοῦ Κυρίου ἡμῶν Ἰησοῦ Χριστοῦ καὶ τῇ κατ᾽ εὐσέβειαν διδασκαλίᾳ,
\VS{4}τετύφωται, μηδὲν ἐπιστάμενος, ἀλλὰ νοσῶν περὶ ζητήσεις καὶ λογομαχίας, ἐξ ὧν γίνεται φθόνος ἔρις βλασφημίαι, ὑπόνοιαι πονηραί,
\VS{5}διαπαρατριβαὶ διεφθαρμένων ἀνθρώπων τὸν νοῦν καὶ ἀπεστερημένων τῆς ἀληθείας, νομιζόντων πορισμὸν εἶναι τὴν εὐσέβειαν.
\par }{\PP \VS{6}Ἔστιν δὲ πορισμὸς μέγας ἡ εὐσέβεια μετὰ αὐταρκείας·
\begin{poetryblock}
\par }{\PP \begin{quote} \VS{7}οὐδὲν γὰρ εἰσηνέγκαμεν εἰς τὸν κόσμον,\end{quote} 
\par }{\PP \begin{quote}ὅτι οὐδὲ ἐξενεγκεῖν τι δυνάμεθα·\end{quote}
\par }{\PP \begin{quote} \VS{8}ἔχοντες δὲ διατροφὰς καὶ σκεπάσματα,\end{quote} 
\par }{\PP \begin{quote}τούτοις ἀρκεσθησόμεθα.\end{quote}
\end{poetryblock}
\par }{\PP \VS{9}Οἱ δὲ βουλόμενοι πλουτεῖν ἐμπίπτουσιν εἰς πειρασμὸν καὶ παγίδα καὶ ἐπιθυμίας πολλὰς ἀνοήτους καὶ βλαβεράς, αἵτινες βυθίζουσιν τοὺς ἀνθρώπους εἰς ὄλεθρον καὶ ἀπώλειαν.
\VS{10}ῥίζα γὰρ πάντων τῶν κακῶν ἐστιν ἡ φιλαργυρία, ἧς τινες ὀρεγόμενοι ἀπεπλανήθησαν ἀπὸ τῆς πίστεως καὶ ἑαυτοὺς περιέπειραν ὀδύναις πολλαῖς.
\par }{\PP \VS{11}Σὺ δέ, ὦ ἄνθρωπε Θεοῦ, ταῦτα φεῦγε· 
\begin{poetryblock}
\par }{\PP \begin{quote}δίωκε δὲ δικαιοσύνην εὐσέβειαν πίστιν,\end{quote} 
\par }{\PP \begin{quote}ἀγάπην ὑπομονήν πραϋπαθίαν.\end{quote}
\par }{\PP \begin{quote} \VS{12}ἀγωνίζου τὸν καλὸν ἀγῶνα τῆς πίστεως,\end{quote} 
\par }{\PP \begin{quote}ἐπιλαβοῦ τῆς αἰωνίου ζωῆς, εἰς ἣν ἐκλήθης\end{quote} 
\par }{\PP \begin{quote}καὶ ὡμολόγησας τὴν καλὴν ὁμολογίαν\end{quote} 
\par }{\PP \begin{quote}ἐνώπιον πολλῶν μαρτύρων.\end{quote}
\end{poetryblock}
\par }{\PP \VS{13}Παραγγέλλω σοι ἐνώπιον τοῦ Θεοῦ τοῦ ζωογονοῦντος τὰ πάντα καὶ Χριστοῦ Ἰησοῦ τοῦ μαρτυρήσαντος ἐπὶ Ποντίου Πιλάτου τὴν καλὴν ὁμολογίαν,
\VS{14}τηρῆσαί σε τὴν ἐντολὴν ἄσπιλον ἀνεπίλημπτον μέχρι τῆς ἐπιφανείας τοῦ Κυρίου ἡμῶν Ἰησοῦ Χριστοῦ,
\VS{15}ἣν καιροῖς ἰδίοις δείξει
\par }{\PP ὁ μακάριος καὶ μόνος Δυνάστης, 
\begin{poetryblock}
\par }{\PP \begin{quote}ὁ Βασιλεὺς τῶν βασιλευόντων\end{quote} 
\par }{\PP \begin{quote}καὶ Κύριος τῶν κυριευόντων,\end{quote}
\par }{\PP \begin{quote} \VS{16}ὁ μόνος ἔχων ἀθανασίαν,\end{quote} 
\par }{\PP \begin{quote}φῶς οἰκῶν ἀπρόσιτον,\end{quote} 
\par }{\PP \begin{quote}ὃν εἶδεν οὐδεὶς ἀνθρώπων οὐδὲ ἰδεῖν δύναται·\end{quote} 
\par }{\PP \begin{quote}ᾧ τιμὴ καὶ κράτος αἰώνιον, ἀμήν.\end{quote}
\end{poetryblock}
\par }{\PP \VS{17}Τοῖς πλουσίοις ἐν τῷ νῦν αἰῶνι παράγγελλε μὴ ὑψηλοφρονεῖν μηδὲ ἠλπικέναι ἐπὶ πλούτου ἀδηλότητι ἀλλ᾽ ἐπὶ Θεῷ τῷ παρέχοντι ἡμῖν πάντα πλουσίως εἰς ἀπόλαυσιν,
\VS{18}ἀγαθοεργεῖν, πλουτεῖν ἐν ἔργοις καλοῖς, εὐμεταδότους εἶναι, κοινωνικούς,
\VS{19}ἀποθησαυρίζοντας ἑαυτοῖς θεμέλιον καλὸν εἰς τὸ μέλλον, ἵνα ἐπιλάβωνται τῆς ὄντως ζωῆς.
\par }{\PP \VS{20}Ὦ Τιμόθεε, τὴν παραθήκην φύλαξον ἐκτρεπόμενος τὰς βεβήλους κενοφωνίας καὶ ἀντιθέσεις τῆς ψευδωνύμου γνώσεως,
\VS{21}ἥν τινες ἐπαγγελλόμενοι περὶ τὴν πίστιν ἠστόχησαν.
\par }{\PP Ἡ χάρις μεθ᾽ ὑμῶν.
\par }