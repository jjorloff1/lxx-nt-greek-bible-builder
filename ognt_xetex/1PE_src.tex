\NormalFont\ShortTitle{ΠΕΤΡΟΥ Α}
{\MT ΠΕΤΡΟΥ Α

\par }\ChapOne{1}{\PP \VerseOne{1}Πέτρος ἀπόστολος Ἰησοῦ Χριστοῦ Ἐκλεκτοῖς παρεπιδήμοις Διασπορᾶς Πόντου, Γαλατίας, Καππαδοκίας, Ἀσίας καὶ Βιθυνίας
\VS{2}κατὰ πρόγνωσιν Θεοῦ Πατρός ἐν ἁγιασμῷ Πνεύματος εἰς ὑπακοὴν καὶ ῥαντισμὸν αἵματος Ἰησοῦ Χριστοῦ, Χάρις ὑμῖν καὶ εἰρήνη πληθυνθείη.
\par }{\PP \VS{3}Εὐλογητὸς ὁ Θεὸς καὶ Πατὴρ τοῦ Κυρίου ἡμῶν Ἰησοῦ Χριστοῦ ὁ κατὰ τὸ πολὺ αὐτοῦ ἔλεος ἀναγεννήσας ἡμᾶς εἰς ἐλπίδα ζῶσαν δι᾽ ἀναστάσεως Ἰησοῦ Χριστοῦ ἐκ νεκρῶν,
\VS{4}εἰς κληρονομίαν ἄφθαρτον καὶ ἀμίαντον καὶ ἀμάραντον τετηρημένην ἐν οὐρανοῖς εἰς ὑμᾶς
\VS{5}τοὺς ἐν δυνάμει Θεοῦ φρουρουμένους διὰ πίστεως εἰς σωτηρίαν ἑτοίμην ἀποκαλυφθῆναι ἐν καιρῷ ἐσχάτῳ
\VS{6}ἐν ᾧ ἀγαλλιᾶσθε ὀλίγον ἄρτι, εἰ δέον ἐστὶν, λυπηθέντες* ἐν ποικίλοις πειρασμοῖς,
\VS{7}ἵνα τὸ δοκίμιον ὑμῶν τῆς πίστεως πολυτιμότερον χρυσίου τοῦ ἀπολλυμένου, διὰ πυρὸς δὲ δοκιμαζομένου εὑρεθῇ εἰς ἔπαινον καὶ δόξαν καὶ τιμὴν ἐν ἀποκαλύψει Ἰησοῦ Χριστοῦ
\VS{8}ὃν οὐκ ἰδόντες ἀγαπᾶτε, εἰς ὃν ἄρτι μὴ ὁρῶντες, πιστεύοντες δὲ ἀγαλλιᾶσθε χαρᾷ ἀνεκλαλήτῳ καὶ δεδοξασμένῃ
\VS{9}κομιζόμενοι τὸ τέλος τῆς πίστεως ὑμῶν σωτηρίαν ψυχῶν.
\VS{10}Περὶ ἧς σωτηρίας ἐξεζήτησαν καὶ ἐξηραύνησαν προφῆται οἱ περὶ τῆς εἰς ὑμᾶς χάριτος προφητεύσαντες
\VS{11}ἐραυνῶντες εἰς τίνα ἢ ποῖον καιρὸν ἐδήλου τὸ ἐν αὐτοῖς Πνεῦμα Χριστοῦ προμαρτυρόμενον τὰ εἰς Χριστὸν παθήματα καὶ τὰς μετὰ ταῦτα δόξας.
\VS{12}οἷς ἀπεκαλύφθη ὅτι οὐχ ἑαυτοῖς, ὑμῖν δὲ διηκόνουν αὐτά ἃ νῦν ἀνηγγέλη ὑμῖν διὰ τῶν εὐαγγελισαμένων ὑμᾶς ἐν Πνεύματι Ἁγίῳ ἀποσταλέντι ἀπ᾽ οὐρανοῦ, εἰς ἃ ἐπιθυμοῦσιν ἄγγελοι παρακύψαι.
\par }{\PP \VS{13}Διὸ ἀναζωσάμενοι τὰς ὀσφύας τῆς διανοίας ὑμῶν νήφοντες τελείως ἐλπίσατε ἐπὶ τὴν φερομένην ὑμῖν χάριν ἐν ἀποκαλύψει Ἰησοῦ Χριστοῦ.
\VS{14}ὡς τέκνα ὑπακοῆς μὴ συσχηματιζόμενοι ταῖς πρότερον ἐν τῇ ἀγνοίᾳ ὑμῶν ἐπιθυμίαις,
\VS{15}ἀλλὰ κατὰ τὸν καλέσαντα ὑμᾶς ἅγιον καὶ αὐτοὶ ἅγιοι ἐν πάσῃ ἀναστροφῇ γενήθητε,
\VS{16}διότι γέγραπται· Ἅγιοι ἔσεσθε, ὅτι ἐγὼ ἅγιος.
\VS{17}Καὶ εἰ Πατέρα ἐπικαλεῖσθε τὸν ἀπροσωπολήμπτως κρίνοντα κατὰ τὸ ἑκάστου ἔργον, ἐν φόβῳ τὸν τῆς παροικίας ὑμῶν χρόνον ἀναστράφητε
\par }{\PP \VS{18}εἰδότες ὅτι οὐ φθαρτοῖς, ἀργυρίῳ ἢ χρυσίῳ, ἐλυτρώθητε 
\begin{poetryblock}
\par }{\PP \begin{quote}¬ἐκ τῆς ματαίας ὑμῶν ἀναστροφῆς πατροπαραδότου\end{quote}
\par }{\PP \begin{quote} \VS{19}¬ἀλλὰ τιμίῳ αἵματι ὡς ἀμνοῦ ἀμώμου καὶ ἀσπίλου Χριστοῦ\end{quote}
\par }{\PP \begin{quote} \VS{20}¬προεγνωσμένου μὲν πρὸ καταβολῆς κόσμου,\end{quote} 
\par }{\PP \begin{quote}¬φανερωθέντος δὲ ἐπ᾽ ἐσχάτου τῶν χρόνων\end{quote} 
\par }{\PP \begin{quote}¬δι᾽ ὑμᾶς\end{quote}
\end{poetryblock}
\par }{\PP \VS{21}τοὺς δι᾽ αὐτοῦ πιστοὺς εἰς Θεὸν 
\begin{poetryblock}
\par }{\PP \begin{quote}¬τὸν ἐγείραντα αὐτὸν ἐκ νεκρῶν καὶ δόξαν αὐτῷ δόντα,\end{quote} 
\par }{\PP \begin{quote}¬ὥστε τὴν πίστιν ὑμῶν καὶ ἐλπίδα εἶναι εἰς Θεόν.\end{quote}
\end{poetryblock}
\par }{\PP \VS{22}Τὰς ψυχὰς ὑμῶν ἡγνικότες ἐν τῇ ὑπακοῇ τῆς ἀληθείας εἰς φιλαδελφίαν ἀνυπόκριτον ἐκ καθαρᾶς καρδίας ἀλλήλους ἀγαπήσατε ἐκτενῶς
\VS{23}ἀναγεγεννημένοι οὐκ ἐκ σπορᾶς φθαρτῆς ἀλλὰ= ἀφθάρτου διὰ λόγου ζῶντος Θεοῦ καὶ μένοντος.
\VS{24}διότι 
\begin{poetryblock}
\par }{\PP \begin{quote}¬Πᾶσα σὰρξ ὡς χόρτος\end{quote} 
\par }{\PP \begin{quote}¬καὶ πᾶσα δόξα αὐτῆς ὡς ἄνθος χόρτου·\end{quote} 
\par }{\PP \begin{quote}¬ἐξηράνθη ὁ χόρτος καὶ τὸ ἄνθος ἐξέπεσεν·\end{quote}
\par }{\PP \begin{quote} \VS{25}¬τὸ δὲ ῥῆμα Κυρίου μένει εἰς τὸν αἰῶνα.\end{quote}
\end{poetryblock}
\par }{\PP Τοῦτο δέ ἐστιν τὸ ῥῆμα τὸ εὐαγγελισθὲν εἰς ὑμᾶς.

\par }\Chap{2}{\PP \VerseOne{1}Ἀποθέμενοι οὖν πᾶσαν κακίαν καὶ πάντα δόλον καὶ ὑποκρίσεις καὶ φθόνους καὶ πάσας καταλαλιάς
\VS{2}ὡς ἀρτιγέννητα βρέφη τὸ λογικὸν ἄδολον γάλα ἐπιποθήσατε, ἵνα ἐν αὐτῷ αὐξηθῆτε εἰς σωτηρίαν,
\VS{3}εἰ ἐγεύσασθε ὅτι χρηστὸς ὁ Κύριος.
\VS{4}Πρὸς ὃν προσερχόμενοι λίθον ζῶντα ὑπὸ ἀνθρώπων μὲν ἀποδεδοκιμασμένον, παρὰ δὲ Θεῷ ἐκλεκτὸν ἔντιμον,
\VS{5}καὶ αὐτοὶ ὡς λίθοι ζῶντες οἰκοδομεῖσθε οἶκος πνευματικὸς εἰς ἱεράτευμα ἅγιον ἀνενέγκαι πνευματικὰς θυσίας εὐπροσδέκτους Θεῷ διὰ Ἰησοῦ Χριστοῦ.
\VS{6}διότι περιέχει ἐν γραφῇ· 
\begin{poetryblock}
\par }{\PP \begin{quote}¬Ἰδοὺ τίθημι ἐν Σιὼν λίθον ἀκρογωνιαῖον ἐκλεκτὸν ἔντιμον,\end{quote} 
\par }{\PP \begin{quote}¬καὶ ὁ πιστεύων ἐπ᾽ αὐτῷ οὐ μὴ καταισχυνθῇ.\end{quote}
\end{poetryblock}
\par }{\PP \VS{7}Ὑμῖν οὖν ἡ τιμὴ τοῖς πιστεύουσιν, ἀπιστοῦσιν δὲ Λίθος ὃν ἀπεδοκίμασαν οἱ οἰκοδομοῦντες, οὗτος ἐγενήθη εἰς κεφαλὴν γωνίας
\VS{8}Καὶ Λίθος προσκόμματος καὶ πέτρα σκανδάλου· Οἳ προσκόπτουσιν τῷ λόγῳ ἀπειθοῦντες εἰς ὃ καὶ ἐτέθησαν.
\VS{9}Ὑμεῖς δὲ γένος ἐκλεκτόν, βασίλειον ἱεράτευμα, ἔθνος ἅγιον, λαὸς εἰς περιποίησιν, ὅπως τὰς ἀρετὰς ἐξαγγείλητε τοῦ ἐκ σκότους ὑμᾶς καλέσαντος εἰς τὸ θαυμαστὸν αὐτοῦ φῶς·
\VS{10}οἵ ποτε οὐ λαὸς, νῦν δὲ λαὸς Θεοῦ, οἱ οὐκ ἠλεημένοι, νῦν δὲ ἐλεηθέντες.
\par }{\PP \VS{11}Ἀγαπητοί, παρακαλῶ ὡς παροίκους καὶ παρεπιδήμους ἀπέχεσθαι τῶν σαρκικῶν ἐπιθυμιῶν αἵτινες στρατεύονται κατὰ τῆς ψυχῆς·
\VS{12}τὴν ἀναστροφὴν ὑμῶν ἐν τοῖς ἔθνεσιν ἔχοντες καλήν, ἵνα ἐν ᾧ καταλαλοῦσιν ὑμῶν ὡς κακοποιῶν ἐκ τῶν καλῶν ἔργων ἐποπτεύοντες δοξάσωσιν τὸν Θεὸν ἐν ἡμέρᾳ ἐπισκοπῆς.
\par }{\PP \VS{13}Ὑποτάγητε πάσῃ ἀνθρωπίνῃ κτίσει διὰ τὸν Κύριον, εἴτε βασιλεῖ ὡς ὑπερέχοντι
\VS{14}εἴτε ἡγεμόσιν ὡς δι᾽ αὐτοῦ πεμπομένοις εἰς ἐκδίκησιν κακοποιῶν, ἔπαινον δὲ ἀγαθοποιῶν,
\VS{15}ὅτι οὕτως ἐστὶν τὸ θέλημα τοῦ Θεοῦ ἀγαθοποιοῦντας φιμοῦν τὴν τῶν ἀφρόνων ἀνθρώπων ἀγνωσίαν,
\VS{16}ὡς ἐλεύθεροι καὶ μὴ ὡς ἐπικάλυμμα ἔχοντες τῆς κακίας τὴν ἐλευθερίαν ἀλλ᾽ ὡς Θεοῦ δοῦλοι.
\VS{17}Πάντας τιμήσατε, τὴν ἀδελφότητα ἀγαπᾶτε, τὸν Θεὸν φοβεῖσθε, τὸν βασιλέα τιμᾶτε.
\par }{\PP \VS{18}Οἱ οἰκέται ὑποτασσόμενοι ἐν παντὶ φόβῳ τοῖς δεσπόταις, οὐ μόνον τοῖς ἀγαθοῖς καὶ ἐπιεικέσιν ἀλλὰ καὶ τοῖς σκολιοῖς.
\VS{19}τοῦτο γὰρ χάρις, εἰ διὰ συνείδησιν Θεοῦ ὑποφέρει τις λύπας πάσχων ἀδίκως.
\VS{20}ποῖον γὰρ κλέος, εἰ ἁμαρτάνοντες καὶ κολαφιζόμενοι ὑπομενεῖτε; ἀλλ᾽ εἰ ἀγαθοποιοῦντες καὶ πάσχοντες ὑπομενεῖτε, τοῦτο χάρις παρὰ Θεῷ.
\begin{poetryblock}
\par }{\PP \begin{quote} \VS{21}¬Εἰς τοῦτο γὰρ ἐκλήθητε,\end{quote} 
\par }{\PP \begin{quote}¬ὅτι καὶ Χριστὸς ἔπαθεν ὑπὲρ ὑμῶν\end{quote} 
\par }{\PP \begin{quote}¬ὑμῖν ὑπολιμπάνων ὑπογραμμὸν,\end{quote} 
\par }{\PP \begin{quote}¬ἵνα ἐπακολουθήσητε τοῖς ἴχνεσιν αὐτοῦ,\end{quote}
\par }{\PP \begin{quote} \VS{22}¬Ὃς ἁμαρτίαν οὐκ ἐποίησεν\end{quote} 
\par }{\PP \begin{quote}¬οὐδὲ εὑρέθη δόλος ἐν τῷ στόματι αὐτοῦ,\end{quote}
\par }{\PP \begin{quote} \VS{23}¬ὃς λοιδορούμενος οὐκ ἀντελοιδόρει,\end{quote} 
\par }{\PP \begin{quote}¬πάσχων οὐκ ἠπείλει,\end{quote} 
\par }{\PP \begin{quote}¬παρεδίδου δὲ τῷ κρίνοντι δικαίως\end{quote}
\par }{\PP \begin{quote} \VS{24}¬ὃς τὰς ἁμαρτίας ἡμῶν αὐτὸς ἀνήνεγκεν\end{quote} 
\par }{\PP \begin{quote}¬ἐν τῷ σώματι αὐτοῦ ἐπὶ τὸ ξύλον,\end{quote} 
\par }{\PP \begin{quote}¬ἵνα ταῖς ἁμαρτίαις ἀπογενόμενοι\end{quote} 
\par }{\PP \begin{quote}¬τῇ δικαιοσύνῃ ζήσωμεν,\end{quote} 
\par }{\PP \begin{quote}¬Οὗ τῷ μώλωπι ἰάθητε.\end{quote}
\par }{\PP \begin{quote} \VS{25}¬Ἦτε γὰρ ὡς πρόβατα πλανώμενοι,\end{quote} 
\par }{\PP \begin{quote}¬ἀλλὰ= ἐπεστράφητε νῦν ἐπὶ τὸν Ποιμένα\end{quote} 
\par }{\PP \begin{quote}¬καὶ Ἐπίσκοπον τῶν ψυχῶν ὑμῶν.\end{quote}
\end{poetryblock}

\par }\Chap{3}{\PP \VerseOne{1}Ὁμοίως αἱ γυναῖκες, ὑποτασσόμεναι τοῖς ἰδίοις ἀνδράσιν, ἵνα καὶ εἴ τινες ἀπειθοῦσιν τῷ λόγῳ, διὰ τῆς τῶν γυναικῶν ἀναστροφῆς ἄνευ λόγου κερδηθήσονται
\VS{2}ἐποπτεύσαντες τὴν ἐν φόβῳ ἁγνὴν ἀναστροφὴν ὑμῶν.
\VS{3}ὧν ἔστω οὐχ ὁ ἔξωθεν ἐμπλοκῆς τριχῶν καὶ περιθέσεως χρυσίων ἢ ἐνδύσεως ἱματίων κόσμος,
\VS{4}ἀλλ᾽ ὁ κρυπτὸς τῆς καρδίας ἄνθρωπος ἐν τῷ ἀφθάρτῳ τοῦ πραέως καὶ ἡσυχίου πνεύματος ὅ ἐστιν ἐνώπιον τοῦ Θεοῦ πολυτελές.
\VS{5}Οὕτως γάρ ποτε καὶ αἱ ἅγιαι γυναῖκες αἱ ἐλπίζουσαι εἰς Θεὸν ἐκόσμουν ἑαυτάς ὑποτασσόμεναι τοῖς ἰδίοις ἀνδράσιν,
\VS{6}ὡς Σάρρα ὑπήκουσεν τῷ Ἀβραάμ κύριον αὐτὸν καλοῦσα ἧς ἐγενήθητε τέκνα ἀγαθοποιοῦσαι καὶ μὴ φοβούμεναι μηδεμίαν πτόησιν.
\par }{\PP \VS{7}Οἱ ἄνδρες ὁμοίως, συνοικοῦντες κατὰ γνῶσιν ὡς ἀσθενεστέρῳ σκεύει τῷ γυναικείῳ, ἀπονέμοντες τιμήν ὡς καὶ συνκληρονόμοις= χάριτος ζωῆς εἰς τὸ μὴ ἐνκόπτεσθαι= τὰς προσευχὰς ὑμῶν.
\par }{\PP \VS{8}Τὸ δὲ τέλος πάντες ὁμόφρονες, συμπαθεῖς, φιλάδελφοι, εὔσπλαγχνοι, ταπεινόφρονες,
\VS{9}μὴ ἀποδιδόντες κακὸν ἀντὶ κακοῦ ἢ λοιδορίαν ἀντὶ λοιδορίας, τοὐναντίον δὲ εὐλογοῦντες, ὅτι εἰς τοῦτο ἐκλήθητε, ἵνα εὐλογίαν κληρονομήσητε.
\begin{poetryblock}
\par }{\PP \begin{quote} \VS{10}¬Ὁ γὰρ θέλων ζωὴν ἀγαπᾶν\end{quote} 
\par }{\PP \begin{quote}¬καὶ ἰδεῖν ἡμέρας ἀγαθὰς\end{quote} 
\par }{\PP \begin{quote}¬παυσάτω τὴν γλῶσσαν ἀπὸ κακοῦ\end{quote} 
\par }{\PP \begin{quote}¬καὶ χείλη τοῦ μὴ λαλῆσαι δόλον,\end{quote}
\par }{\PP \begin{quote} \VS{11}¬ἐκκλινάτω δὲ ἀπὸ κακοῦ καὶ ποιησάτω ἀγαθόν,\end{quote} 
\par }{\PP \begin{quote}¬ζητησάτω εἰρήνην καὶ διωξάτω αὐτήν·\end{quote}
\par }{\PP \begin{quote} \VS{12}¬ὅτι ὀφθαλμοὶ Κυρίου ἐπὶ δικαίους\end{quote} 
\par }{\PP \begin{quote}¬καὶ ὦτα αὐτοῦ εἰς δέησιν αὐτῶν,\end{quote} 
\par }{\PP \begin{quote}¬πρόσωπον δὲ Κυρίου ἐπὶ ποιοῦντας κακά.\end{quote}
\end{poetryblock}
\VS{13}Καὶ τίς ὁ κακώσων ὑμᾶς, ἐὰν τοῦ ἀγαθοῦ ζηλωταὶ γένησθε;
\VS{14}Ἀλλ᾽ εἰ καὶ πάσχοιτε διὰ δικαιοσύνην, μακάριοι. Τὸν δὲ φόβον αὐτῶν μὴ φοβηθῆτε μηδὲ ταραχθῆτε,
\VS{15}Κύριον δὲ τὸν Χριστὸν ἁγιάσατε ἐν ταῖς καρδίαις ὑμῶν, ἕτοιμοι ἀεὶ πρὸς ἀπολογίαν παντὶ τῷ αἰτοῦντι ὑμᾶς λόγον περὶ τῆς ἐν ὑμῖν ἐλπίδος,
\VS{16}ἀλλὰ μετὰ πραΰτητος καὶ φόβου, συνείδησιν ἔχοντες ἀγαθήν, ἵνα ἐν ᾧ καταλαλεῖσθε καταισχυνθῶσιν οἱ ἐπηρεάζοντες ὑμῶν τὴν ἀγαθὴν ἐν Χριστῷ ἀναστροφήν.
\VS{17}κρεῖττον γὰρ ἀγαθοποιοῦντας, εἰ θέλοι τὸ θέλημα τοῦ Θεοῦ, πάσχειν ἢ κακοποιοῦντας.
\begin{poetryblock}
\par }{\PP \begin{quote} \VS{18}¬ὅτι καὶ Χριστὸς ἅπαξ περὶ ἁμαρτιῶν ἔπαθεν,\end{quote} 
\par }{\PP \begin{quote}¬δίκαιος ὑπὲρ ἀδίκων,\end{quote} 
\par }{\PP \begin{quote}¬ἵνα ὑμᾶς προσαγάγῃ τῷ Θεῷ\end{quote} 
\par }{\PP \begin{quote}¬θανατωθεὶς μὲν σαρκὶ,\end{quote} 
\par }{\PP \begin{quote}¬ζωοποιηθεὶς δὲ πνεύματι·\end{quote}
\par }{\PP \begin{quote} \VS{19}¬ἐν ᾧ καὶ τοῖς ἐν φυλακῇ πνεύμασιν\end{quote} 
\par }{\PP \begin{quote}¬πορευθεὶς ἐκήρυξεν\end{quote}
\end{poetryblock}
\par }{\PP \VS{20}ἀπειθήσασίν ποτε, ὅτε ἀπεξεδέχετο ἡ τοῦ Θεοῦ μακροθυμία ἐν ἡμέραις Νῶε κατασκευαζομένης κιβωτοῦ εἰς ἣν ὀλίγοι, τοῦτ᾽ ἔστιν ὀκτὼ ψυχαί, διεσώθησαν δι᾽ ὕδατος
\VS{21}ὃ καὶ ὑμᾶς ἀντίτυπον νῦν σῴζει βάπτισμα, οὐ σαρκὸς ἀπόθεσις ῥύπου ἀλλὰ συνειδήσεως ἀγαθῆς ἐπερώτημα εἰς Θεόν, δι᾽ ἀναστάσεως Ἰησοῦ Χριστοῦ
\VS{22}ὅς ἐστιν ἐν δεξιᾷ τοῦ Θεοῦ πορευθεὶς εἰς οὐρανόν ὑποταγέντων αὐτῷ ἀγγέλων καὶ ἐξουσιῶν καὶ δυνάμεων.

\par }\Chap{4}{\PP \VerseOne{1}Χριστοῦ οὖν παθόντος σαρκὶ καὶ ὑμεῖς τὴν αὐτὴν ἔννοιαν ὁπλίσασθε, ὅτι ὁ παθὼν σαρκὶ πέπαυται ἁμαρτίας
\VS{2}εἰς τὸ μηκέτι ἀνθρώπων ἐπιθυμίαις ἀλλὰ θελήματι Θεοῦ τὸν ἐπίλοιπον ἐν σαρκὶ βιῶσαι χρόνον.
\VS{3}ἀρκετὸς γὰρ ὁ παρεληλυθὼς χρόνος τὸ βούλημα τῶν ἐθνῶν κατειργάσθαι πεπορευμένους ἐν ἀσελγείαις, ἐπιθυμίαις, οἰνοφλυγίαις, κώμοις, πότοις καὶ ἀθεμίτοις εἰδωλολατρίαις.
\VS{4}Ἐν ᾧ ξενίζονται μὴ συντρεχόντων ὑμῶν εἰς τὴν αὐτὴν τῆς ἀσωτίας ἀνάχυσιν βλασφημοῦντες,
\VS{5}οἳ ἀποδώσουσιν λόγον τῷ ἑτοίμως ἔχοντι κρῖναι ζῶντας καὶ νεκρούς.
\VS{6}εἰς τοῦτο γὰρ καὶ νεκροῖς εὐηγγελίσθη, ἵνα κριθῶσι= μὲν κατὰ ἀνθρώπους σαρκί, ζῶσι= δὲ κατὰ Θεὸν πνεύματι.
\par }{\PP \VS{7}Πάντων δὲ τὸ τέλος ἤγγικεν. σωφρονήσατε οὖν καὶ νήψατε εἰς προσευχάς
\VS{8}πρὸ πάντων τὴν εἰς ἑαυτοὺς ἀγάπην ἐκτενῆ ἔχοντες, ὅτι ἀγάπη καλύπτει πλῆθος ἁμαρτιῶν,
\VS{9}φιλόξενοι εἰς ἀλλήλους ἄνευ γογγυσμοῦ,
\VS{10}ἕκαστος καθὼς ἔλαβεν χάρισμα εἰς ἑαυτοὺς αὐτὸ διακονοῦντες ὡς καλοὶ οἰκονόμοι ποικίλης χάριτος Θεοῦ.
\VS{11}εἴ τις λαλεῖ, ὡς λόγια Θεοῦ· εἴ τις διακονεῖ, ὡς ἐξ ἰσχύος ἧς χορηγεῖ ὁ Θεός, ἵνα ἐν πᾶσιν δοξάζηται ὁ Θεὸς διὰ Ἰησοῦ Χριστοῦ ᾧ ἐστιν ἡ δόξα καὶ τὸ κράτος εἰς τοὺς αἰῶνας τῶν αἰώνων, ἀμήν.
\par }{\PP \VS{12}Ἀγαπητοί, μὴ ξενίζεσθε τῇ ἐν ὑμῖν πυρώσει πρὸς πειρασμὸν ὑμῖν γινομένῃ ὡς ξένου ὑμῖν συμβαίνοντος,
\VS{13}ἀλλὰ καθὸ κοινωνεῖτε τοῖς τοῦ Χριστοῦ παθήμασιν, χαίρετε, ἵνα καὶ ἐν τῇ ἀποκαλύψει τῆς δόξης αὐτοῦ χαρῆτε ἀγαλλιώμενοι.
\VS{14}Εἰ ὀνειδίζεσθε ἐν ὀνόματι Χριστοῦ, μακάριοι, ὅτι τὸ τῆς δόξης καὶ τὸ τοῦ Θεοῦ Πνεῦμα ἐφ᾽ ὑμᾶς ἀναπαύεται.
\VS{15}μὴ γάρ τις ὑμῶν πασχέτω ὡς φονεὺς ἢ κλέπτης ἢ κακοποιὸς ἢ ὡς ἀλλοτριεπίσκοπος·
\VS{16}εἰ δὲ ὡς Χριστιανός, μὴ αἰσχυνέσθω, δοξαζέτω δὲ τὸν Θεὸν ἐν τῷ ὀνόματι* τούτῳ.
\VS{17}ὅτι ὁ καιρὸς τοῦ ἄρξασθαι τὸ κρίμα ἀπὸ τοῦ οἴκου τοῦ Θεοῦ· εἰ δὲ πρῶτον ἀφ᾽ ἡμῶν, τί τὸ τέλος τῶν ἀπειθούντων τῷ τοῦ Θεοῦ εὐαγγελίῳ;
\VS{18}καὶ Εἰ ὁ δίκαιος μόλις σώζεται, ὁ ἀσεβὴς καὶ ἁμαρτωλὸς ποῦ φανεῖται;
\VS{19}Ὥστε καὶ οἱ πάσχοντες κατὰ τὸ θέλημα τοῦ Θεοῦ πιστῷ Κτίστῃ παρατιθέσθωσαν τὰς ψυχὰς αὐτῶν ἐν ἀγαθοποιΐᾳ.

\par }\Chap{5}{\PP \VerseOne{1}Πρεσβυτέρους τοὺς ἐν ὑμῖν παρακαλῶ ὁ συμπρεσβύτερος καὶ μάρτυς τῶν τοῦ Χριστοῦ παθημάτων, ὁ καὶ τῆς μελλούσης ἀποκαλύπτεσθαι δόξης κοινωνός·
\VS{2}ποιμάνατε τὸ ἐν ὑμῖν ποίμνιον τοῦ Θεοῦ ἐπισκοποῦντες μὴ ἀναγκαστῶς ἀλλὰ= ἑκουσίως κατὰ Θεόν, μηδὲ αἰσχροκερδῶς ἀλλὰ προθύμως,
\VS{3}μηδ᾽ ὡς κατακυριεύοντες τῶν κλήρων ἀλλὰ τύποι γινόμενοι τοῦ ποιμνίου·
\VS{4}καὶ φανερωθέντος τοῦ Ἀρχιποίμενος κομιεῖσθε τὸν ἀμαράντινον τῆς δόξης στέφανον.
\VS{5}Ὁμοίως, νεώτεροι, ὑποτάγητε πρεσβυτέροις· πάντες δὲ ἀλλήλοις τὴν ταπεινοφροσύνην ἐγκομβώσασθε, ὅτι Ὁ Θεὸς ὑπερηφάνοις ἀντιτάσσεται, ταπεινοῖς δὲ δίδωσιν χάριν.
\par }{\PP \VS{6}Ταπεινώθητε οὖν ὑπὸ τὴν κραταιὰν χεῖρα τοῦ Θεοῦ, ἵνα ὑμᾶς ὑψώσῃ ἐν καιρῷ,
\VS{7}πᾶσαν τὴν μέριμναν ὑμῶν ἐπιρίψαντες ἐπ᾽ αὐτόν, ὅτι αὐτῷ μέλει περὶ ὑμῶν.
\VS{8}Νήψατε, γρηγορήσατε. ὁ ἀντίδικος ὑμῶν διάβολος ὡς λέων ὠρυόμενος περιπατεῖ ζητῶν τινα καταπιεῖν·
\VS{9}ᾧ ἀντίστητε στερεοὶ τῇ πίστει εἰδότες τὰ αὐτὰ τῶν παθημάτων τῇ ἐν κόσμῳ ὑμῶν ἀδελφότητι ἐπιτελεῖσθαι.
\VS{10}Ὁ δὲ Θεὸς πάσης χάριτος, ὁ καλέσας ὑμᾶς εἰς τὴν αἰώνιον αὐτοῦ δόξαν ἐν Χριστῷ ὀλίγον παθόντας αὐτὸς καταρτίσει, στηρίξει, σθενώσει, θεμελιώσει.
\VS{11}αὐτῷ τὸ κράτος εἰς τοὺς αἰῶνας, ἀμήν.
\par }{\PP \VS{12}Διὰ Σιλουανοῦ ὑμῖν τοῦ πιστοῦ ἀδελφοῦ, ὡς λογίζομαι, δι᾽ ὀλίγων ἔγραψα παρακαλῶν καὶ ἐπιμαρτυρῶν ταύτην εἶναι ἀληθῆ χάριν τοῦ Θεοῦ εἰς ἣν στῆτε.
\VS{13}Ἀσπάζεται ὑμᾶς ἡ ἐν Βαβυλῶνι συνεκλεκτὴ καὶ Μάρκος ὁ υἱός μου.
\VS{14}Ἀσπάσασθε ἀλλήλους ἐν φιλήματι ἀγάπης.
\par }{\PP Εἰρήνη ὑμῖν πᾶσιν τοῖς ἐν Χριστῷ.
\par }