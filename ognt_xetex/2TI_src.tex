\NormalFont\ShortTitle{ΠΡΟΣ ΤΙΜΟΘΕΟΝ Β}
{\MT ΠΡΟΣ ΤΙΜΟΘΕΟΝ Β

\par }\ChapOne{1}{\PP \VerseOne{1}Παῦλος ἀπόστολος Χριστοῦ Ἰησοῦ διὰ θελήματος Θεοῦ κατ᾽ ἐπαγγελίαν ζωῆς τῆς ἐν Χριστῷ Ἰησοῦ
\par }{\PP \VS{2}Τιμοθέῳ ἀγαπητῷ τέκνῳ, Χάρις ἔλεος εἰρήνη ἀπὸ Θεοῦ Πατρὸς καὶ Χριστοῦ Ἰησοῦ τοῦ Κυρίου ἡμῶν.
\VS{3}Χάριν ἔχω τῷ Θεῷ, ᾧ λατρεύω ἀπὸ προγόνων ἐν καθαρᾷ συνειδήσει, ὡς ἀδιάλειπτον ἔχω τὴν περὶ σοῦ μνείαν ἐν ταῖς δεήσεσίν μου νυκτὸς καὶ ἡμέρας,
\VS{4}ἐπιποθῶν σε ἰδεῖν, μεμνημένος σου τῶν δακρύων, ἵνα χαρᾶς πληρωθῶ,
\par }{\PP \VS{5}ὑπόμνησιν λαβὼν τῆς ἐν σοὶ ἀνυποκρίτου πίστεως, ἥτις ἐνῴκησεν πρῶτον ἐν τῇ μάμμῃ σου Λωΐδι καὶ τῇ μητρί σου Εὐνίκῃ, πέπεισμαι δὲ ὅτι καὶ ἐν σοί.
\VS{6}Δι᾽ ἣν αἰτίαν ἀναμιμνῄσκω σε ἀναζωπυρεῖν τὸ χάρισμα τοῦ Θεοῦ, ὅ ἐστιν ἐν σοὶ διὰ τῆς ἐπιθέσεως τῶν χειρῶν μου.
\VS{7}οὐ γὰρ ἔδωκεν ἡμῖν ὁ Θεὸς πνεῦμα δειλίας ἀλλὰ δυνάμεως καὶ ἀγάπης καὶ σωφρονισμοῦ.
\par }{\PP \VS{8}Μὴ οὖν ἐπαισχυνθῇς τὸ μαρτύριον τοῦ Κυρίου ἡμῶν μηδὲ ἐμὲ τὸν δέσμιον αὐτοῦ, ἀλλὰ συνκακοπάθησον= τῷ εὐαγγελίῳ κατὰ δύναμιν Θεοῦ,
\VS{9}τοῦ σώσαντος ἡμᾶς ¬καὶ καλέσαντος κλήσει ἁγίᾳ, ¬οὐ κατὰ τὰ ἔργα ἡμῶν ¬ἀλλὰ κατὰ ἰδίαν πρόθεσιν καὶ χάριν, ¬τὴν δοθεῖσαν ἡμῖν ἐν Χριστῷ Ἰησοῦ ¬πρὸ χρόνων αἰωνίων,
\par }{\PP \VS{10}¬φανερωθεῖσαν δὲ νῦν ¬διὰ τῆς ἐπιφανείας τοῦ Σωτῆρος ἡμῶν Χριστοῦ Ἰησοῦ, ¬καταργήσαντος μὲν τὸν θάνατον ¬φωτίσαντος δὲ ζωὴν καὶ ἀφθαρσίαν διὰ τοῦ εὐαγγελίου
\VS{11}εἰς ὃ ἐτέθην ἐγὼ κήρυξ καὶ ἀπόστολος καὶ διδάσκαλος,
\VS{12}Δι᾽ ἣν αἰτίαν καὶ ταῦτα πάσχω· ἀλλ᾽ οὐκ ἐπαισχύνομαι, οἶδα γὰρ ᾧ πεπίστευκα καὶ πέπεισμαι ὅτι δυνατός ἐστιν τὴν παραθήκην μου φυλάξαι εἰς ἐκείνην τὴν ἡμέραν.
\VS{13}Ὑποτύπωσιν ἔχε ὑγιαινόντων λόγων ὧν παρ᾽ ἐμοῦ ἤκουσας ἐν πίστει καὶ ἀγάπῃ τῇ ἐν Χριστῷ Ἰησοῦ·
\par }{\PP \VS{14}τὴν καλὴν παραθήκην φύλαξον διὰ Πνεύματος Ἁγίου τοῦ ἐνοικοῦντος ἐν ἡμῖν.
\VS{15}Οἶδας τοῦτο, ὅτι ἀπεστράφησάν με πάντες οἱ ἐν τῇ Ἀσίᾳ, ὧν ἐστιν Φύγελος καὶ Ἑρμογένης.
\VS{16}Δῴη ἔλεος ὁ Κύριος τῷ Ὀνησιφόρου οἴκῳ, ὅτι πολλάκις με ἀνέψυξεν καὶ τὴν ἅλυσίν μου οὐκ ἐπαισχύνθη,
\VS{17}ἀλλὰ γενόμενος ἐν Ῥώμῃ σπουδαίως ἐζήτησέν με καὶ εὗρεν·
\par }{\PP \VS{18}Δῴη αὐτῷ ὁ Κύριος εὑρεῖν ἔλεος παρὰ Κυρίου ἐν ἐκείνῃ τῇ ἡμέρᾳ. καὶ ὅσα ἐν Ἐφέσῳ διηκόνησεν, βέλτιον σὺ γινώσκεις.

\par }\Chap{2}{\PP \VerseOne{1}Σὺ οὖν, τέκνον μου, ἐνδυναμοῦ ἐν τῇ χάριτι τῇ ἐν Χριστῷ Ἰησοῦ,
\VS{2}καὶ ἃ ἤκουσας παρ᾽ ἐμοῦ διὰ πολλῶν μαρτύρων, ταῦτα παράθου πιστοῖς ἀνθρώποις, οἵτινες ἱκανοὶ ἔσονται καὶ ἑτέρους διδάξαι.
\VS{3}Συνκακοπάθησον= ὡς καλὸς στρατιώτης Χριστοῦ Ἰησοῦ.
\VS{4}οὐδεὶς στρατευόμενος ἐμπλέκεται ταῖς τοῦ βίου πραγματείαις, ἵνα τῷ στρατολογήσαντι ἀρέσῃ.
\VS{5}ἐὰν δὲ καὶ ἀθλῇ τις, οὐ στεφανοῦται ἐὰν μὴ νομίμως ἀθλήσῃ.
\VS{6}τὸν κοπιῶντα γεωργὸν δεῖ πρῶτον τῶν καρπῶν μεταλαμβάνειν.
\par }{\PP \VS{7}νόει ὃ λέγω· δώσει γάρ σοι ὁ Κύριος σύνεσιν ἐν πᾶσιν.
\VS{8}Μνημόνευε Ἰησοῦν Χριστὸν ἐγηγερμένον ἐκ νεκρῶν, ἐκ σπέρματος Δαυίδ, κατὰ τὸ εὐαγγέλιόν μου,
\VS{9}ἐν ᾧ κακοπαθῶ μέχρι δεσμῶν ὡς κακοῦργος, ἀλλὰ= ὁ λόγος τοῦ Θεοῦ οὐ δέδεται·
\VS{10}διὰ τοῦτο πάντα ὑπομένω διὰ τοὺς ἐκλεκτούς, ἵνα καὶ αὐτοὶ σωτηρίας τύχωσιν τῆς ἐν Χριστῷ Ἰησοῦ μετὰ δόξης αἰωνίου.
\VS{11}Πιστὸς ὁ λόγος· ¬Εἰ γὰρ συναπεθάνομεν, καὶ συζήσομεν·
\VS{12}¬εἰ ὑπομένομεν, καὶ συμβασιλεύσομεν· ¬εἰ ἀρνησόμεθα, κἀκεῖνος ἀρνήσεται ἡμᾶς·
\par }{\PP \VS{13}¬εἰ ἀπιστοῦμεν, ἐκεῖνος πιστὸς μένει, ¬ἀρνήσασθαι γὰρ ἑαυτὸν οὐ δύναται.
\VS{14}Ταῦτα ὑπομίμνῃσκε διαμαρτυρόμενος ἐνώπιον τοῦ Θεοῦ μὴ λογομαχεῖν, ἐπ᾽ οὐδὲν χρήσιμον, ἐπὶ καταστροφῇ τῶν ἀκουόντων.
\VS{15}Σπούδασον σεαυτὸν δόκιμον παραστῆσαι τῷ Θεῷ, ἐργάτην ἀνεπαίσχυντον, ὀρθοτομοῦντα τὸν λόγον τῆς ἀληθείας.
\VS{16}Τὰς δὲ βεβήλους κενοφωνίας περιΐστασο· ἐπὶ πλεῖον γὰρ προκόψουσιν ἀσεβείας
\VS{17}καὶ ὁ λόγος αὐτῶν ὡς γάγγραινα νομὴν ἕξει. ὧν ἐστιν Ὑμέναιος καὶ Φιλητός,
\VS{18}οἵτινες περὶ τὴν ἀλήθειαν ἠστόχησαν, λέγοντες τὴν ἀνάστασιν ἤδη γεγονέναι, καὶ ἀνατρέπουσιν τήν τινων πίστιν.
\VS{19}Ὁ μέντοι στερεὸς θεμέλιος τοῦ Θεοῦ ἕστηκεν, ἔχων τὴν σφραγῖδα ταύτην· Ἔγνω Κύριος τοὺς ὄντας αὐτοῦ, καί· Ἀποστήτω ἀπὸ ἀδικίας πᾶς ὁ ὀνομάζων τὸ ὄνομα Κυρίου.
\VS{20}Ἐν μεγάλῃ δὲ οἰκίᾳ οὐκ ἔστιν μόνον σκεύη χρυσᾶ καὶ ἀργυρᾶ ἀλλὰ καὶ ξύλινα καὶ ὀστράκινα, καὶ ἃ μὲν εἰς τιμὴν ἃ δὲ εἰς ἀτιμίαν·
\par }{\PP \VS{21}ἐὰν οὖν τις ἐκκαθάρῃ ἑαυτὸν ἀπὸ τούτων, ἔσται σκεῦος εἰς τιμήν, ἡγιασμένον, εὔχρηστον τῷ δεσπότῃ, εἰς πᾶν ἔργον ἀγαθὸν ἡτοιμασμένον.
\VS{22}Τὰς δὲ νεωτερικὰς ἐπιθυμίας φεῦγε, δίωκε δὲ δικαιοσύνην πίστιν ἀγάπην εἰρήνην μετὰ τῶν ἐπικαλουμένων τὸν Κύριον ἐκ καθαρᾶς καρδίας.
\VS{23}Τὰς δὲ μωρὰς καὶ ἀπαιδεύτους ζητήσεις παραιτοῦ, εἰδὼς ὅτι γεννῶσιν μάχας·
\VS{24}δοῦλον δὲ Κυρίου οὐ δεῖ μάχεσθαι ἀλλὰ= ἤπιον εἶναι πρὸς πάντας, διδακτικόν, ἀνεξίκακον,
\VS{25}ἐν πραΰτητι παιδεύοντα τοὺς ἀντιδιατιθεμένους, μήποτε δώῃ αὐτοῖς ὁ Θεὸς μετάνοιαν εἰς ἐπίγνωσιν ἀληθείας
\par }{\PP \VS{26}καὶ ἀνανήψωσιν ἐκ τῆς τοῦ διαβόλου παγίδος, ἐζωγρημένοι ὑπ᾽ αὐτοῦ εἰς τὸ ἐκείνου θέλημα.

\par }\Chap{3}{\PP \VerseOne{1}Τοῦτο δὲ γίνωσκε, ὅτι ἐν ἐσχάταις ἡμέραις ἐνστήσονται καιροὶ χαλεποί·
\VS{2}ἔσονται γὰρ οἱ ἄνθρωποι φίλαυτοι φιλάργυροι ἀλαζόνες ὑπερήφανοι βλάσφημοι, γονεῦσιν ἀπειθεῖς, ἀχάριστοι ἀνόσιοι
\VS{3}ἄστοργοι ἄσπονδοι διάβολοι ἀκρατεῖς ἀνήμεροι ἀφιλάγαθοι
\VS{4}προδόται προπετεῖς τετυφωμένοι, φιλήδονοι μᾶλλον ἢ φιλόθεοι,
\VS{5}ἔχοντες μόρφωσιν εὐσεβείας τὴν δὲ δύναμιν αὐτῆς ἠρνημένοι· καὶ τούτους ἀποτρέπου.
\VS{6}Ἐκ τούτων γάρ εἰσιν οἱ ἐνδύνοντες εἰς τὰς οἰκίας καὶ αἰχμαλωτίζοντες γυναικάρια σεσωρευμένα ἁμαρτίαις, ἀγόμενα ἐπιθυμίαις ποικίλαις,
\VS{7}πάντοτε μανθάνοντα καὶ μηδέποτε εἰς ἐπίγνωσιν ἀληθείας ἐλθεῖν δυνάμενα.
\VS{8}ὃν τρόπον δὲ Ἰάννης καὶ Ἰαμβρῆς ἀντέστησαν Μωϋσεῖ, οὕτως καὶ οὗτοι ἀνθίστανται τῇ ἀληθείᾳ, ἄνθρωποι κατεφθαρμένοι τὸν νοῦν, ἀδόκιμοι περὶ τὴν πίστιν.
\par }{\PP \VS{9}ἀλλ᾽ οὐ προκόψουσιν ἐπὶ πλεῖον· ἡ γὰρ ἄνοια αὐτῶν ἔκδηλος ἔσται πᾶσιν, ὡς καὶ ἡ ἐκείνων ἐγένετο.
\VS{10}Σὺ δὲ παρηκολούθησάς μου τῇ διδασκαλίᾳ, τῇ ἀγωγῇ, τῇ προθέσει, τῇ πίστει, τῇ μακροθυμίᾳ, τῇ ἀγάπῃ, τῇ ὑπομονῇ,
\VS{11}τοῖς διωγμοῖς, τοῖς παθήμασιν, οἷά μοι ἐγένετο ἐν Ἀντιοχείᾳ, ἐν Ἰκονίῳ, ἐν Λύστροις, οἵους διωγμοὺς ὑπήνεγκα καὶ ἐκ πάντων με ἐρρύσατο ὁ Κύριος.
\VS{12}καὶ πάντες δὲ οἱ θέλοντες εὐσεβῶς ζῆν ἐν Χριστῷ Ἰησοῦ διωχθήσονται.
\VS{13}πονηροὶ δὲ ἄνθρωποι καὶ γόητες προκόψουσιν ἐπὶ τὸ χεῖρον πλανῶντες καὶ πλανώμενοι.
\VS{14}Σὺ δὲ μένε ἐν οἷς ἔμαθες καὶ ἐπιστώθης, εἰδὼς παρὰ τίνων ἔμαθες,
\VS{15}καὶ ὅτι ἀπὸ βρέφους τὰ ἱερὰ γράμματα οἶδας, τὰ δυνάμενά σε σοφίσαι εἰς σωτηρίαν διὰ πίστεως τῆς ἐν Χριστῷ Ἰησοῦ.
\VS{16}πᾶσα γραφὴ θεόπνευστος καὶ ὠφέλιμος πρὸς διδασκαλίαν, πρὸς ἐλεγμόν, πρὸς ἐπανόρθωσιν, πρὸς παιδείαν τὴν ἐν δικαιοσύνῃ,
\par }{\PP \VS{17}ἵνα ἄρτιος ᾖ ὁ τοῦ Θεοῦ ἄνθρωπος, πρὸς πᾶν ἔργον ἀγαθὸν ἐξηρτισμένος.

\par }\Chap{4}{\PP \VerseOne{1}Διαμαρτύρομαι ἐνώπιον τοῦ Θεοῦ καὶ Χριστοῦ Ἰησοῦ τοῦ μέλλοντος κρίνειν ζῶντας καὶ νεκρούς, καὶ τὴν ἐπιφάνειαν αὐτοῦ καὶ τὴν βασιλείαν αὐτοῦ·
\VS{2}κήρυξον τὸν λόγον, ἐπίστηθι εὐκαίρως ἀκαίρως, ἔλεγξον, ἐπιτίμησον, παρακάλεσον, ἐν πάσῃ μακροθυμίᾳ καὶ διδαχῇ.
\VS{3}Ἔσται γὰρ καιρὸς ὅτε τῆς ὑγιαινούσης διδασκαλίας οὐκ ἀνέξονται ἀλλὰ κατὰ τὰς ἰδίας ἐπιθυμίας ἑαυτοῖς ἐπισωρεύσουσιν διδασκάλους κνηθόμενοι τὴν ἀκοήν
\VS{4}καὶ ἀπὸ μὲν τῆς ἀληθείας τὴν ἀκοὴν ἀποστρέψουσιν, ἐπὶ δὲ τοὺς μύθους ἐκτραπήσονται.
\par }{\PP \VS{5}Σὺ δὲ νῆφε ἐν πᾶσιν, κακοπάθησον, ἔργον ποίησον εὐαγγελιστοῦ, τὴν διακονίαν σου πληροφόρησον.
\VS{6}Ἐγὼ γὰρ ἤδη σπένδομαι, καὶ ὁ καιρὸς τῆς ἀναλύσεώς μου ἐφέστηκεν.
\VS{7}τὸν καλὸν ἀγῶνα ἠγώνισμαι, τὸν δρόμον τετέλεκα, τὴν πίστιν τετήρηκα·
\par }{\PP \VS{8}λοιπὸν ἀπόκειταί μοι ὁ τῆς δικαιοσύνης στέφανος, ὃν ἀποδώσει μοι ὁ κύριος ἐν ἐκείνῃ τῇ ἡμέρᾳ, ὁ δίκαιος κριτής, οὐ μόνον δὲ ἐμοὶ ἀλλὰ καὶ πᾶσι= τοῖς ἠγαπηκόσι= τὴν ἐπιφάνειαν αὐτοῦ.
\VS{9}Σπούδασον ἐλθεῖν πρός με ταχέως·
\VS{10}Δημᾶς γάρ με ἐγκατέλιπεν ἀγαπήσας τὸν νῦν αἰῶνα καὶ ἐπορεύθη εἰς Θεσσαλονίκην, Κρήσκης εἰς Γαλατίαν, Τίτος εἰς Δαλματίαν·
\VS{11}Λουκᾶς ἐστιν μόνος μετ᾽ ἐμοῦ. Μᾶρκον ἀναλαβὼν ἄγε μετὰ σεαυτοῦ, ἔστιν γάρ μοι εὔχρηστος εἰς διακονίαν.
\VS{12}Τυχικὸν δὲ ἀπέστειλα εἰς Ἔφεσον.
\VS{13}τὸν φαιλόνην ὃν ἀπέλιπον ἐν Τρῳάδι παρὰ Κάρπῳ ἐρχόμενος φέρε, καὶ τὰ βιβλία μάλιστα τὰς μεμβράνας.
\VS{14}Ἀλέξανδρος ὁ χαλκεὺς πολλά μοι κακὰ ἐνεδείξατο· ἀποδώσει αὐτῷ ὁ Κύριος κατὰ τὰ ἔργα αὐτοῦ·
\VS{15}ὃν καὶ σὺ φυλάσσου, λίαν γὰρ ἀντέστη τοῖς ἡμετέροις λόγοις.
\VS{16}Ἐν τῇ πρώτῃ μου ἀπολογίᾳ οὐδείς μοι παρεγένετο, ἀλλὰ πάντες με ἐγκατέλιπον· μὴ αὐτοῖς λογισθείη·
\VS{17}ὁ δὲ Κύριός μοι παρέστη καὶ ἐνεδυνάμωσέν με, ἵνα δι᾽ ἐμοῦ τὸ κήρυγμα πληροφορηθῇ καὶ ἀκούσωσιν πάντα τὰ ἔθνη, καὶ ἐρρύσθην ἐκ στόματος λέοντος.
\par }{\PP \VS{18}ῥύσεταί με ὁ Κύριος ἀπὸ παντὸς ἔργου πονηροῦ καὶ σώσει εἰς τὴν βασιλείαν αὐτοῦ τὴν ἐπουράνιον· ᾧ ἡ δόξα εἰς τοὺς αἰῶνας τῶν αἰώνων, ἀμήν.
\VS{19}Ἄσπασαι Πρίσκαν καὶ Ἀκύλαν καὶ τὸν Ὀνησιφόρου οἶκον.
\VS{20}Ἔραστος ἔμεινεν ἐν Κορίνθῳ, Τρόφιμον δὲ ἀπέλιπον ἐν Μιλήτῳ ἀσθενοῦντα.
\par }{\PP \VS{21}Σπούδασον πρὸ χειμῶνος ἐλθεῖν. Ἀσπάζεταί σε Εὔβουλος καὶ Πούδης καὶ Λίνος καὶ Κλαυδία καὶ οἱ ἀδελφοὶ πάντες.
\par }{\PP \VS{22}Ὁ Κύριος μετὰ τοῦ πνεύματός σου. ἡ χάρις μεθ᾽ ὑμῶν.
\par }