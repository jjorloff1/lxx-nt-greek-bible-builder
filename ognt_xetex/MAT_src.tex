\NormalFont\ShortTitle{ΚΑΤΑ ΜΑΘΘΑΙΟΝ}
{\MT ΚΑΤΑ ΜΑΘΘΑΙΟΝ

\par }\ChapOne{1}{\PP \VerseOne{1}Βίβλος γενέσεως Ἰησοῦ Χριστοῦ υἱοῦ Δαυὶδ υἱοῦ Ἀβραάμ.
\par }{\PP \postdropcapindent\VS{2}Ἀβραὰμ ἐγέννησεν τὸν Ἰσαάκ, Ἰσαὰκ δὲ ἐγέννησεν τὸν Ἰακώβ, Ἰακὼβ δὲ ἐγέννησεν τὸν Ἰούδαν καὶ τοὺς ἀδελφοὺς αὐτοῦ,
\VS{3}Ἰούδας δὲ ἐγέννησεν τὸν Φαρὲς καὶ τὸν Ζαρὰ ἐκ τῆς Θάμαρ, Φαρὲς δὲ ἐγέννησεν τὸν Ἑσρώμ, Ἑσρὼμ δὲ ἐγέννησεν τὸν Ἀράμ,
\VS{4}Ἀρὰμ δὲ ἐγέννησεν τὸν Ἀμιναδάβ, Ἀμιναδὰβ δὲ ἐγέννησεν τὸν Ναασσών, Ναασσὼν δὲ ἐγέννησεν τὸν Σαλμών,
\VS{5}Σαλμὼν δὲ ἐγέννησεν τὸν Βόες ἐκ τῆς Ῥαχάβ, Βόες δὲ ἐγέννησεν τὸν Ἰωβὴδ ἐκ τῆς Ῥούθ, Ἰωβὴδ δὲ ἐγέννησεν τὸν Ἰεσσαί,
\VS{6}Ἰεσσαὶ δὲ ἐγέννησεν τὸν Δαυὶδ τὸν βασιλέα.
\par }{\PP Δαυὶδ δὲ ἐγέννησεν τὸν Σολομῶνα ἐκ τῆς τοῦ Οὐρίου,
\VS{7}Σολομὼν δὲ ἐγέννησεν τὸν Ῥοβοάμ, Ῥοβοὰμ δὲ ἐγέννησεν τὸν Ἀβιά, Ἀβιὰ δὲ ἐγέννησεν τὸν Ἀσάφ,
\VS{8}Ἀσὰφ δὲ ἐγέννησεν τὸν Ἰωσαφάτ, Ἰωσαφὰτ δὲ ἐγέννησεν τὸν Ἰωράμ, Ἰωρὰμ δὲ ἐγέννησεν τὸν Ὀζίαν,
\VS{9}Ὀζίας δὲ ἐγέννησεν τὸν Ἰωαθάμ, Ἰωαθὰμ δὲ ἐγέννησεν τὸν Ἄχαζ, Ἄχαζ δὲ ἐγέννησεν τὸν Ἑζεκίαν,
\VS{10}Ἑζεκίας δὲ ἐγέννησεν τὸν Μανασσῆ, Μανασσῆς δὲ ἐγέννησεν τὸν Ἀμώς, Ἀμὼς δὲ ἐγέννησεν τὸν Ἰωσίαν,
\VS{11}Ἰωσίας δὲ ἐγέννησεν τὸν Ἰεχονίαν καὶ τοὺς ἀδελφοὺς αὐτοῦ ἐπὶ τῆς μετοικεσίας Βαβυλῶνος.
\par }{\PP \VS{12}Μετὰ δὲ τὴν μετοικεσίαν Βαβυλῶνος Ἰεχονίας ἐγέννησεν τὸν Σαλαθιήλ, Σαλαθιὴλ δὲ ἐγέννησεν τὸν Ζοροβαβέλ,
\VS{13}Ζοροβαβὲλ δὲ ἐγέννησεν τὸν Ἀβιούδ, Ἀβιοὺδ δὲ ἐγέννησεν τὸν Ἐλιακίμ, Ἐλιακὶμ δὲ ἐγέννησεν τὸν Ἀζώρ,
\VS{14}Ἀζὼρ δὲ ἐγέννησεν τὸν Σαδώκ, Σαδὼκ δὲ ἐγέννησεν τὸν Ἀχίμ, Ἀχὶμ δὲ ἐγέννησεν τὸν Ἐλιούδ,
\VS{15}Ἐλιοὺδ δὲ ἐγέννησεν τὸν Ἐλεάζαρ, Ἐλεάζαρ δὲ ἐγέννησεν τὸν Ματθάν, Ματθὰν δὲ ἐγέννησεν τὸν Ἰακώβ,
\VS{16}Ἰακὼβ δὲ ἐγέννησεν τὸν Ἰωσὴφ τὸν ἄνδρα Μαρίας, ἐξ ἧς ἐγεννήθη Ἰησοῦς ὁ λεγόμενος Χριστός.
\par }{\PP \VS{17}Πᾶσαι οὖν αἱ γενεαὶ ἀπὸ Ἀβραὰμ ἕως Δαυὶδ γενεαὶ δεκατέσσαρες, καὶ ἀπὸ Δαυὶδ ἕως τῆς μετοικεσίας Βαβυλῶνος γενεαὶ δεκατέσσαρες, καὶ ἀπὸ τῆς μετοικεσίας Βαβυλῶνος ἕως τοῦ Χριστοῦ γενεαὶ δεκατέσσαρες.
\par }{\PP \VS{18}Τοῦ δὲ Ἰησοῦ Χριστοῦ ἡ γένεσις οὕτως ἦν. μνηστευθείσης τῆς μητρὸς αὐτοῦ Μαρίας τῷ Ἰωσήφ, πρὶν ἢ συνελθεῖν αὐτοὺς εὑρέθη ἐν γαστρὶ ἔχουσα ἐκ πνεύματος ἁγίου.
\VS{19}Ἰωσὴφ δὲ ὁ ἀνὴρ αὐτῆς, δίκαιος ὢν καὶ μὴ θέλων αὐτὴν δειγματίσαι, ἐβουλήθη λάθρᾳ ἀπολῦσαι αὐτήν.
\VS{20}Ταῦτα δὲ αὐτοῦ ἐνθυμηθέντος ἰδοὺ ἄγγελος Κυρίου κατ᾽ ὄναρ ἐφάνη αὐτῷ λέγων· Ἰωσὴφ υἱὸς Δαυίδ, μὴ φοβηθῇς παραλαβεῖν Μαρίαν τὴν γυναῖκά σου· τὸ γὰρ ἐν αὐτῇ γεννηθὲν ἐκ Πνεύματός ἐστιν Ἁγίου.
\VS{21}τέξεται δὲ υἱὸν, καὶ καλέσεις τὸ ὄνομα αὐτοῦ Ἰησοῦν· αὐτὸς γὰρ σώσει τὸν λαὸν αὐτοῦ ἀπὸ τῶν ἁμαρτιῶν αὐτῶν.
\VS{22}Τοῦτο δὲ ὅλον γέγονεν ἵνα πληρωθῇ τὸ ῥηθὲν ὑπὸ Κυρίου διὰ τοῦ προφήτου λέγοντος·
\VS{23}¬Ἰδοὺ ἡ παρθένος ἐν γαστρὶ ἕξει καὶ τέξεται υἱόν, ¬καὶ καλέσουσιν τὸ ὄνομα αὐτοῦ Ἐμμανουήλ,
\par }{\PP ὅ ἐστιν μεθερμηνευόμενον Μεθ᾽ ἡμῶν ὁ Θεός.
\VS{24}Ἐγερθεὶς δὲ ὁ Ἰωσὴφ ἀπὸ τοῦ ὕπνου ἐποίησεν ὡς προσέταξεν αὐτῷ ὁ ἄγγελος Κυρίου καὶ παρέλαβεν τὴν γυναῖκα αὐτοῦ,
\VS{25}καὶ οὐκ ἐγίνωσκεν αὐτὴν ἕως οὗ ἔτεκεν υἱόν· καὶ ἐκάλεσεν τὸ ὄνομα αὐτοῦ Ἰησοῦν.

\par }\Chap{2}{\PP \VerseOne{1}Τοῦ δὲ Ἰησοῦ γεννηθέντος ἐν Βηθλέεμ τῆς Ἰουδαίας ἐν ἡμέραις Ἡρῴδου τοῦ βασιλέως, ἰδοὺ μάγοι ἀπὸ ἀνατολῶν παρεγένοντο εἰς Ἱεροσόλυμα
\VS{2}λέγοντες· Ποῦ ἐστιν ὁ τεχθεὶς βασιλεὺς τῶν Ἰουδαίων; εἴδομεν γὰρ αὐτοῦ τὸν ἀστέρα ἐν τῇ ἀνατολῇ καὶ ἤλθομεν προσκυνῆσαι αὐτῷ.
\VS{3}Ἀκούσας δὲ ὁ βασιλεὺς Ἡρῴδης ἐταράχθη καὶ πᾶσα Ἱεροσόλυμα μετ᾽ αὐτοῦ,
\VS{4}καὶ συναγαγὼν πάντας τοὺς ἀρχιερεῖς καὶ γραμματεῖς τοῦ λαοῦ ἐπυνθάνετο παρ᾽ αὐτῶν ποῦ ὁ Χριστὸς γεννᾶται.
\VS{5}Οἱ δὲ εἶπαν αὐτῷ· Ἐν Βηθλέεμ τῆς Ἰουδαίας· οὕτως γὰρ γέγραπται διὰ τοῦ προφήτου·
\VS{6}¬Καὶ σύ Βηθλέεμ, γῆ Ἰούδα, ¬οὐδαμῶς ἐλαχίστη εἶ ἐν τοῖς ἡγεμόσιν Ἰούδα· ¬ἐκ σοῦ γὰρ ἐξελεύσεται ἡγούμενος, ¬ὅστις ποιμανεῖ τὸν λαόν μου τὸν Ἰσραήλ.
\par }{\PP \VS{7}Τότε Ἡρῴδης λάθρᾳ καλέσας τοὺς μάγους ἠκρίβωσεν παρ᾽ αὐτῶν τὸν χρόνον τοῦ φαινομένου ἀστέρος,
\VS{8}καὶ πέμψας αὐτοὺς εἰς Βηθλέεμ εἶπεν· Πορευθέντες ἐξετάσατε ἀκριβῶς περὶ τοῦ παιδίου· ἐπὰν δὲ εὕρητε, ἀπαγγείλατέ μοι, ὅπως κἀγὼ ἐλθὼν προσκυνήσω αὐτῷ.
\VS{9}Οἱ δὲ ἀκούσαντες τοῦ βασιλέως ἐπορεύθησαν καὶ ἰδοὺ ὁ ἀστὴρ, ὃν εἶδον ἐν τῇ ἀνατολῇ, προῆγεν αὐτούς, ἕως ἐλθὼν ἐστάθη ἐπάνω οὗ ἦν τὸ παιδίον.
\VS{10}ἰδόντες δὲ τὸν ἀστέρα ἐχάρησαν χαρὰν μεγάλην σφόδρα.
\VS{11}καὶ ἐλθόντες εἰς τὴν οἰκίαν εἶδον τὸ παιδίον μετὰ Μαρίας τῆς μητρὸς αὐτοῦ, καὶ πεσόντες προσεκύνησαν αὐτῷ καὶ ἀνοίξαντες τοὺς θησαυροὺς αὐτῶν προσήνεγκαν αὐτῷ δῶρα, χρυσὸν καὶ λίβανον καὶ σμύρναν.
\VS{12}Καὶ χρηματισθέντες κατ᾽ ὄναρ μὴ ἀνακάμψαι πρὸς Ἡρῴδην, δι᾽ ἄλλης ὁδοῦ ἀνεχώρησαν εἰς τὴν χώραν αὐτῶν.
\par }{\PP \VS{13}Ἀναχωρησάντων δὲ αὐτῶν ἰδοὺ ἄγγελος κυρίου φαίνεται κατ᾽ ὄναρ τῷ Ἰωσὴφ λέγων· Ἐγερθεὶς παράλαβε τὸ παιδίον καὶ τὴν μητέρα αὐτοῦ καὶ φεῦγε εἰς Αἴγυπτον καὶ ἴσθι ἐκεῖ ἕως ἂν εἴπω σοι· μέλλει γὰρ Ἡρῴδης ζητεῖν τὸ παιδίον τοῦ ἀπολέσαι αὐτό.
\VS{14}Ὁ δὲ ἐγερθεὶς παρέλαβεν τὸ παιδίον καὶ τὴν μητέρα αὐτοῦ νυκτὸς καὶ ἀνεχώρησεν εἰς Αἴγυπτον,
\VS{15}καὶ ἦν ἐκεῖ ἕως τῆς τελευτῆς Ἡρῴδου· ἵνα πληρωθῇ τὸ ῥηθὲν ὑπὸ κυρίου διὰ τοῦ προφήτου λέγοντος· ¬Ἐξ Αἰγύπτου ἐκάλεσα τὸν υἱόν μου.
\par }{\PP \VS{16}Τότε Ἡρῴδης ἰδὼν ὅτι ἐνεπαίχθη ὑπὸ τῶν μάγων ἐθυμώθη λίαν, καὶ ἀποστείλας ἀνεῖλεν πάντας τοὺς παῖδας τοὺς ἐν Βηθλέεμ καὶ ἐν πᾶσι= τοῖς ὁρίοις αὐτῆς ἀπὸ διετοῦς καὶ κατωτέρω, κατὰ τὸν χρόνον ὃν ἠκρίβωσεν παρὰ τῶν μάγων.
\VS{17}τότε ἐπληρώθη τὸ ῥηθὲν διὰ Ἰερεμίου τοῦ προφήτου λέγοντος·
\VS{18}¬Φωνὴ ἐν Ῥαμὰ ἠκούσθη, ¬κλαυθμὸς καὶ ὀδυρμὸς πολύς· ¬Ῥαχὴλ κλαίουσα τὰ τέκνα αὐτῆς, ¬καὶ οὐκ ἤθελεν παρακληθῆναι, ¬ὅτι οὐκ εἰσίν.
\par }{\PP \VS{19}Τελευτήσαντος δὲ τοῦ Ἡρῴδου ἰδοὺ ἄγγελος Κυρίου φαίνεται κατ᾽ ὄναρ τῷ Ἰωσὴφ ἐν Αἰγύπτῳ
\VS{20}λέγων· Ἐγερθεὶς παράλαβε τὸ παιδίον καὶ τὴν μητέρα αὐτοῦ καὶ πορεύου εἰς γῆν Ἰσραήλ· τεθνήκασιν γὰρ οἱ ζητοῦντες τὴν ψυχὴν τοῦ παιδίου.
\VS{21}Ὁ δὲ ἐγερθεὶς παρέλαβεν τὸ παιδίον καὶ τὴν μητέρα αὐτοῦ καὶ εἰσῆλθεν εἰς γῆν Ἰσραήλ.
\par }{\PP \VS{22}ἀκούσας δὲ ὅτι Ἀρχέλαος βασιλεύει τῆς Ἰουδαίας ἀντὶ τοῦ πατρὸς αὐτοῦ Ἡρῴδου ἐφοβήθη ἐκεῖ ἀπελθεῖν· χρηματισθεὶς δὲ κατ᾽ ὄναρ ἀνεχώρησεν εἰς τὰ μέρη τῆς Γαλιλαίας,
\VS{23}καὶ ἐλθὼν κατῴκησεν εἰς πόλιν λεγομένην Ναζαρέτ· ὅπως πληρωθῇ τὸ ῥηθὲν διὰ τῶν προφητῶν ὅτι Ναζωραῖος κληθήσεται.

\par }\Chap{3}{\PP \VerseOne{1}Ἐν δὲ ταῖς ἡμέραις ἐκείναις παραγίνεται Ἰωάννης ὁ βαπτιστὴς κηρύσσων ἐν τῇ ἐρήμῳ τῆς Ἰουδαίας
\VS{2}καὶ λέγων· Μετανοεῖτε· ἤγγικεν γὰρ ἡ βασιλεία τῶν οὐρανῶν.
\par }{\PP \VS{3}οὗτος γάρ ἐστιν ὁ ῥηθεὶς διὰ Ἠσαΐου τοῦ προφήτου λέγοντος· ¬Φωνὴ βοῶντος ἐν τῇ ἐρήμῳ· ¬Ἑτοιμάσατε τὴν ὁδὸν Κυρίου, ¬εὐθείας ποιεῖτε τὰς τρίβους αὐτοῦ.
\par }{\PP \VS{4}Αὐτὸς δὲ ὁ Ἰωάννης εἶχεν τὸ ἔνδυμα αὐτοῦ ἀπὸ τριχῶν καμήλου καὶ ζώνην δερματίνην περὶ τὴν ὀσφὺν αὐτοῦ, ἡ δὲ τροφὴ ἦν αὐτοῦ ἀκρίδες καὶ μέλι ἄγριον.
\VS{5}Τότε ἐξεπορεύετο πρὸς αὐτὸν Ἱεροσόλυμα καὶ πᾶσα ἡ Ἰουδαία καὶ πᾶσα ἡ περίχωρος τοῦ Ἰορδάνου,
\VS{6}καὶ ἐβαπτίζοντο ἐν τῷ Ἰορδάνῃ ποταμῷ ὑπ᾽ αὐτοῦ ἐξομολογούμενοι τὰς ἁμαρτίας αὐτῶν.
\par }{\PP \VS{7}Ἰδὼν δὲ πολλοὺς τῶν Φαρισαίων καὶ Σαδδουκαίων ἐρχομένους ἐπὶ τὸ βάπτισμα αὐτοῦ εἶπεν αὐτοῖς· Γεννήματα ἐχιδνῶν, τίς ὑπέδειξεν ὑμῖν φυγεῖν ἀπὸ τῆς μελλούσης ὀργῆς;
\VS{8}ποιήσατε οὖν καρπὸν ἄξιον τῆς μετανοίας
\VS{9}καὶ μὴ δόξητε λέγειν ἐν ἑαυτοῖς· Πατέρα ἔχομεν τὸν Ἀβραάμ. λέγω γὰρ ὑμῖν ὅτι δύναται ὁ Θεὸς ἐκ τῶν λίθων τούτων ἐγεῖραι τέκνα τῷ Ἀβραάμ.
\VS{10}ἤδη δὲ ἡ ἀξίνη πρὸς τὴν ῥίζαν τῶν δένδρων κεῖται· πᾶν οὖν δένδρον μὴ ποιοῦν καρπὸν καλὸν ἐκκόπτεται καὶ εἰς πῦρ βάλλεται.
\par }{\PP \VS{11}Ἐγὼ μὲν ὑμᾶς βαπτίζω ἐν ὕδατι εἰς μετάνοιαν, ὁ δὲ ὀπίσω μου ἐρχόμενος ἰσχυρότερός μού ἐστιν, οὗ οὐκ εἰμὶ ἱκανὸς τὰ ὑποδήματα βαστάσαι· αὐτὸς ὑμᾶς βαπτίσει ἐν Πνεύματι Ἁγίῳ καὶ πυρί·
\VS{12}οὗ τὸ πτύον ἐν τῇ χειρὶ αὐτοῦ καὶ διακαθαριεῖ τὴν ἅλωνα αὐτοῦ καὶ συνάξει τὸν σῖτον αὐτοῦ εἰς τὴν ἀποθήκην, τὸ δὲ ἄχυρον κατακαύσει πυρὶ ἀσβέστῳ.
\par }{\PP \VS{13}Τότε παραγίνεται ὁ Ἰησοῦς ἀπὸ τῆς Γαλιλαίας ἐπὶ τὸν Ἰορδάνην πρὸς τὸν Ἰωάννην τοῦ βαπτισθῆναι ὑπ᾽ αὐτοῦ.
\VS{14}ὁ δὲ Ἰωάννης διεκώλυεν αὐτὸν λέγων· Ἐγὼ χρείαν ἔχω ὑπὸ σοῦ βαπτισθῆναι, καὶ σὺ ἔρχῃ πρός με;
\VS{15}Ἀποκριθεὶς δὲ ὁ Ἰησοῦς εἶπεν πρὸς αὐτόν· Ἄφες ἄρτι, οὕτως γὰρ πρέπον ἐστὶν ἡμῖν πληρῶσαι πᾶσαν δικαιοσύνην. τότε ἀφίησιν αὐτόν.
\VS{16}Βαπτισθεὶς δὲ ὁ Ἰησοῦς εὐθὺς ἀνέβη ἀπὸ τοῦ ὕδατος· καὶ ἰδοὺ ἠνεῴχθησαν αὐτῷ οἱ οὐρανοί, καὶ εἶδεν τὸ Πνεῦμα τοῦ Θεοῦ καταβαῖνον ὡσεὶ περιστερὰν καὶ ἐρχόμενον ἐπ᾽ αὐτόν·
\VS{17}καὶ ἰδοὺ φωνὴ ἐκ τῶν οὐρανῶν λέγουσα· Οὗτός ἐστιν ὁ Υἱός μου ὁ ἀγαπητός, ἐν ᾧ εὐδόκησα.

\par }\Chap{4}{\PP \VerseOne{1}Τότε ὁ Ἰησοῦς ἀνήχθη εἰς τὴν ἔρημον ὑπὸ τοῦ Πνεύματος πειρασθῆναι ὑπὸ τοῦ διαβόλου.
\VS{2}καὶ νηστεύσας ἡμέρας τεσσεράκοντα καὶ νύκτας τεσσεράκοντα, ὕστερον ἐπείνασεν.
\VS{3}Καὶ προσελθὼν ὁ πειράζων εἶπεν αὐτῷ· Εἰ Υἱὸς εἶ τοῦ Θεοῦ, εἰπὲ ἵνα οἱ λίθοι οὗτοι ἄρτοι γένωνται.
\VS{4}Ὁ δὲ ἀποκριθεὶς εἶπεν· Γέγραπται· Οὐκ ἐπ᾽ ἄρτῳ μόνῳ ζήσεται ὁ ἄνθρωπος, Ἀλλ᾽ ἐπὶ παντὶ ῥήματι ἐκπορευομένῳ διὰ στόματος Θεοῦ.
\par }{\PP \VS{5}Τότε παραλαμβάνει αὐτὸν ὁ διάβολος εἰς τὴν ἁγίαν πόλιν καὶ ἔστησεν αὐτὸν ἐπὶ τὸ πτερύγιον τοῦ ἱεροῦ
\VS{6}καὶ λέγει αὐτῷ· Εἰ Υἱὸς εἶ τοῦ Θεοῦ, βάλε σεαυτὸν κάτω· γέγραπται γὰρ ὅτι ¬Τοῖς ἀγγέλοις αὐτοῦ ἐντελεῖται περὶ σοῦ ¬καὶ ἐπὶ χειρῶν ἀροῦσίν σε, ¬μήποτε προσκόψῃς πρὸς λίθον τὸν πόδα σου.
\par }{\PP \VS{7}Ἔφη αὐτῷ ὁ Ἰησοῦς· Πάλιν γέγραπται· Οὐκ ἐκπειράσεις Κύριον τὸν Θεόν σου.
\par }{\PP \VS{8}Πάλιν παραλαμβάνει αὐτὸν ὁ διάβολος εἰς ὄρος ὑψηλὸν λίαν καὶ δείκνυσιν αὐτῷ πάσας τὰς βασιλείας τοῦ κόσμου καὶ τὴν δόξαν αὐτῶν
\VS{9}καὶ εἶπεν αὐτῷ· Ταῦτά σοι πάντα δώσω, ἐὰν πεσὼν προσκυνήσῃς μοι.
\VS{10}Τότε λέγει αὐτῷ ὁ Ἰησοῦς· Ὕπαγε, Σατανᾶ· γέγραπται γάρ· Κύριον τὸν θεόν σου προσκυνήσεις καὶ αὐτῷ μόνῳ λατρεύσεις.
\par }{\PP \VS{11}Τότε ἀφίησιν αὐτὸν ὁ διάβολος, καὶ ἰδοὺ ἄγγελοι προσῆλθον καὶ διηκόνουν αὐτῷ.
\par }{\PP \VS{12}Ἀκούσας δὲ ὅτι Ἰωάννης παρεδόθη ἀνεχώρησεν εἰς τὴν Γαλιλαίαν.
\VS{13}καὶ καταλιπὼν τὴν Ναζαρὰ ἐλθὼν κατῴκησεν εἰς Καφαρναοὺμ τὴν παραθαλασσίαν ἐν ὁρίοις Ζαβουλὼν καὶ Νεφθαλίμ·
\VS{14}ἵνα πληρωθῇ τὸ ῥηθὲν διὰ Ἠσαΐου τοῦ προφήτου λέγοντος·
\VS{15}¬Γῆ Ζαβουλὼν καὶ γῆ Νεφθαλίμ, ¬ὁδὸν θαλάσσης, πέραν τοῦ Ἰορδάνου, ¬Γαλιλαία τῶν ἐθνῶν,
\VS{16}¬ὁ λαὸς ὁ καθήμενος ἐν σκοτίᾳ* ¬φῶς εἶδεν μέγα, ¬καὶ τοῖς καθημένοις ἐν χώρᾳ καὶ σκιᾷ θανάτου ¬φῶς ἀνέτειλεν αὐτοῖς.
\par }{\PP \VS{17}Ἀπὸ τότε ἤρξατο ὁ Ἰησοῦς κηρύσσειν καὶ λέγειν· Μετανοεῖτε· ἤγγικεν γὰρ ἡ βασιλεία τῶν οὐρανῶν.
\par }{\PP \VS{18}Περιπατῶν δὲ παρὰ τὴν θάλασσαν τῆς Γαλιλαίας εἶδεν δύο ἀδελφούς, Σίμωνα τὸν λεγόμενον Πέτρον καὶ Ἀνδρέαν τὸν ἀδελφὸν αὐτοῦ, βάλλοντας ἀμφίβληστρον εἰς τὴν θάλασσαν· ἦσαν γὰρ ἁλιεῖς.
\VS{19}καὶ λέγει αὐτοῖς· Δεῦτε ὀπίσω μου, καὶ ποιήσω ὑμᾶς ἁλιεῖς ἀνθρώπων.
\VS{20}οἱ δὲ εὐθέως ἀφέντες τὰ δίκτυα ἠκολούθησαν αὐτῷ.
\VS{21}Καὶ προβὰς ἐκεῖθεν εἶδεν ἄλλους δύο ἀδελφούς, Ἰάκωβον τὸν τοῦ Ζεβεδαίου καὶ Ἰωάννην τὸν ἀδελφὸν αὐτοῦ, ἐν τῷ πλοίῳ μετὰ Ζεβεδαίου τοῦ πατρὸς αὐτῶν καταρτίζοντας τὰ δίκτυα αὐτῶν, καὶ ἐκάλεσεν αὐτούς.
\VS{22}οἱ δὲ εὐθέως ἀφέντες τὸ πλοῖον καὶ τὸν πατέρα αὐτῶν ἠκολούθησαν αὐτῷ.
\par }{\PP \VS{23}Καὶ περιῆγεν ἐν ὅλῃ τῇ Γαλιλαίᾳ διδάσκων ἐν ταῖς συναγωγαῖς αὐτῶν καὶ κηρύσσων τὸ εὐαγγέλιον τῆς βασιλείας καὶ θεραπεύων πᾶσαν νόσον καὶ πᾶσαν μαλακίαν ἐν τῷ λαῷ.
\par }{\PP \VS{24}καὶ ἀπῆλθεν ἡ ἀκοὴ αὐτοῦ εἰς ὅλην τὴν Συρίαν· καὶ προσήνεγκαν αὐτῷ πάντας τοὺς κακῶς ἔχοντας ποικίλαις νόσοις καὶ βασάνοις συνεχομένους καὶ δαιμονιζομένους καὶ σεληνιαζομένους καὶ παραλυτικούς, καὶ ἐθεράπευσεν αὐτούς.
\VS{25}Καὶ ἠκολούθησαν αὐτῷ ὄχλοι πολλοὶ ἀπὸ τῆς Γαλιλαίας καὶ Δεκαπόλεως καὶ Ἱεροσολύμων καὶ Ἰουδαίας καὶ πέραν τοῦ Ἰορδάνου.

\par }\Chap{5}{\PP \VerseOne{1}Ἰδὼν δὲ τοὺς ὄχλους ἀνέβη εἰς τὸ ὄρος, καὶ καθίσαντος αὐτοῦ προσῆλθαν αὐτῷ οἱ μαθηταὶ αὐτοῦ·
\VS{2}καὶ ἀνοίξας τὸ στόμα αὐτοῦ ἐδίδασκεν αὐτοὺς λέγων·
\VS{3}¬Μακάριοι οἱ πτωχοὶ τῷ πνεύματι, ¬Ὅτι αὐτῶν ἐστιν ἡ βασιλεία τῶν οὐρανῶν.
\VS{4}¬Μακάριοι οἱ πενθοῦντες, Ὅτι αὐτοὶ παρακληθήσονται.
\VS{5}¬Μακάριοι οἱ πραεῖς, ¬Ὅτι αὐτοὶ κληρονομήσουσιν τὴν γῆν.
\VS{6}¬Μακάριοι οἱ πεινῶντες καὶ διψῶντες τὴν δικαιοσύνην, ¬Ὅτι αὐτοὶ χορτασθήσονται.
\VS{7}¬Μακάριοι οἱ ἐλεήμονες, ¬Ὅτι αὐτοὶ ἐλεηθήσονται.
\VS{8}¬Μακάριοι οἱ καθαροὶ τῇ καρδίᾳ, ¬Ὅτι αὐτοὶ τὸν Θεὸν ὄψονται.
\VS{9}¬Μακάριοι οἱ εἰρηνοποιοί, ¬Ὅτι αὐτοὶ υἱοὶ Θεοῦ κληθήσονται.
\VS{10}¬Μακάριοι οἱ δεδιωγμένοι ἕνεκεν δικαιοσύνης, ¬Ὅτι αὐτῶν ἐστιν ἡ βασιλεία τῶν οὐρανῶν.
\VS{11}¬Μακάριοί ἐστε ¬ὅταν ὀνειδίσωσιν ὑμᾶς καὶ διώξωσιν καὶ εἴπωσιν πᾶν πονηρὸν καθ᾽ ὑμῶν ψευδόμενοι ἕνεκεν ἐμοῦ.
\VS{12}¬χαίρετε καὶ ἀγαλλιᾶσθε, ὅτι ὁ μισθὸς ὑμῶν πολὺς ἐν τοῖς οὐρανοῖς· οὕτως γὰρ ἐδίωξαν τοὺς προφήτας τοὺς πρὸ ὑμῶν.
\par }{\PP \VS{13}Ὑμεῖς ἐστε τὸ ἅλας τῆς γῆς· ἐὰν δὲ τὸ ἅλας μωρανθῇ, ἐν τίνι ἁλισθήσεται; εἰς οὐδὲν ἰσχύει ἔτι εἰ μὴ βληθὲν ἔξω καταπατεῖσθαι ὑπὸ τῶν ἀνθρώπων.
\par }{\PP \VS{14}Ὑμεῖς ἐστε τὸ φῶς τοῦ κόσμου. οὐ δύναται πόλις κρυβῆναι ἐπάνω ὄρους κειμένη·
\VS{15}οὐδὲ καίουσιν λύχνον καὶ τιθέασιν αὐτὸν ὑπὸ τὸν μόδιον ἀλλ᾽ ἐπὶ τὴν λυχνίαν, καὶ λάμπει πᾶσιν τοῖς ἐν τῇ οἰκίᾳ.
\VS{16}οὕτως λαμψάτω τὸ φῶς ὑμῶν ἔμπροσθεν τῶν ἀνθρώπων, ὅπως ἴδωσιν ὑμῶν τὰ καλὰ ἔργα καὶ δοξάσωσιν τὸν πατέρα ὑμῶν τὸν ἐν τοῖς οὐρανοῖς.
\par }{\PP \VS{17}Μὴ νομίσητε ὅτι ἦλθον καταλῦσαι τὸν νόμον ἢ τοὺς προφήτας· οὐκ ἦλθον καταλῦσαι ἀλλὰ πληρῶσαι.
\VS{18}ἀμὴν γὰρ λέγω ὑμῖν· ἕως ἂν παρέλθῃ ὁ οὐρανὸς καὶ ἡ γῆ, ἰῶτα ἓν ἢ μία κεραία οὐ μὴ παρέλθῃ ἀπὸ τοῦ νόμου, ἕως ἂν πάντα γένηται.
\VS{19}Ὃς ἐὰν οὖν λύσῃ μίαν τῶν ἐντολῶν τούτων τῶν ἐλαχίστων καὶ διδάξῃ οὕτως τοὺς ἀνθρώπους, ἐλάχιστος κληθήσεται ἐν τῇ βασιλείᾳ τῶν οὐρανῶν· ὃς δ᾽ ἂν ποιήσῃ καὶ διδάξῃ, οὗτος μέγας κληθήσεται ἐν τῇ βασιλείᾳ τῶν οὐρανῶν.
\par }{\PP \VS{20}λέγω γὰρ ὑμῖν ὅτι ἐὰν μὴ περισσεύσῃ ὑμῶν ἡ δικαιοσύνη πλεῖον τῶν γραμματέων καὶ Φαρισαίων, οὐ μὴ εἰσέλθητε εἰς τὴν βασιλείαν τῶν οὐρανῶν.
\par }{\PP \VS{21}Ἠκούσατε ὅτι ἐρρέθη τοῖς ἀρχαίοις· Οὐ φονεύσεις· ὃς δ᾽ ἂν φονεύσῃ, ἔνοχος ἔσται τῇ κρίσει.
\VS{22}ἐγὼ δὲ λέγω ὑμῖν ὅτι πᾶς ὁ ὀργιζόμενος τῷ ἀδελφῷ αὐτοῦ ἔνοχος ἔσται τῇ κρίσει· ὃς δ᾽ ἂν εἴπῃ τῷ ἀδελφῷ αὐτοῦ· Ῥακά, ἔνοχος ἔσται τῷ συνεδρίῳ· ὃς δ᾽ ἂν εἴπῃ· Μωρέ, ἔνοχος ἔσται εἰς τὴν γέενναν τοῦ πυρός.
\VS{23}Ἐὰν οὖν προσφέρῃς τὸ δῶρόν σου ἐπὶ τὸ θυσιαστήριον κἀκεῖ μνησθῇς ὅτι ὁ ἀδελφός σου ἔχει τι κατὰ σοῦ,
\VS{24}ἄφες ἐκεῖ τὸ δῶρόν σου ἔμπροσθεν τοῦ θυσιαστηρίου καὶ ὕπαγε πρῶτον διαλλάγηθι τῷ ἀδελφῷ σου, καὶ τότε ἐλθὼν πρόσφερε τὸ δῶρόν σου.
\VS{25}ἴσθι εὐνοῶν τῷ ἀντιδίκῳ σου ταχὺ, ἕως ὅτου εἶ μετ᾽ αὐτοῦ ἐν τῇ ὁδῷ, μήποτέ σε παραδῷ ὁ ἀντίδικος τῷ κριτῇ καὶ ὁ κριτὴς τῷ ὑπηρέτῃ καὶ εἰς φυλακὴν βληθήσῃ·
\VS{26}ἀμὴν λέγω σοι, οὐ μὴ ἐξέλθῃς ἐκεῖθεν, ἕως ἂν ἀποδῷς τὸν ἔσχατον κοδράντην.
\par }{\PP \VS{27}Ἠκούσατε ὅτι ἐρρέθη· Οὐ μοιχεύσεις.
\VS{28}ἐγὼ δὲ λέγω ὑμῖν ὅτι πᾶς ὁ βλέπων γυναῖκα πρὸς τὸ ἐπιθυμῆσαι αὐτὴν ἤδη ἐμοίχευσεν αὐτὴν ἐν τῇ καρδίᾳ αὐτοῦ.
\VS{29}εἰ δὲ ὁ ὀφθαλμός σου ὁ δεξιὸς σκανδαλίζει σε, ἔξελε αὐτὸν καὶ βάλε ἀπὸ σοῦ· συμφέρει γάρ σοι ἵνα ἀπόληται ἓν τῶν μελῶν σου καὶ μὴ ὅλον τὸ σῶμά σου βληθῇ εἰς γέενναν.
\VS{30}καὶ εἰ ἡ δεξιά σου χεὶρ σκανδαλίζει σε, ἔκκοψον αὐτὴν καὶ βάλε ἀπὸ σοῦ· συμφέρει γάρ σοι ἵνα ἀπόληται ἓν τῶν μελῶν σου καὶ μὴ ὅλον τὸ σῶμά σου εἰς γέενναν ἀπέλθῃ.
\par }{\PP \VS{31}Ἐρρέθη δέ· Ὃς ἂν ἀπολύσῃ τὴν γυναῖκα αὐτοῦ, δότω αὐτῇ ἀποστάσιον.
\VS{32}ἐγὼ δὲ λέγω ὑμῖν ὅτι πᾶς ὁ ἀπολύων τὴν γυναῖκα αὐτοῦ παρεκτὸς λόγου πορνείας ποιεῖ αὐτὴν μοιχευθῆναι, καὶ ὃς ἐὰν ἀπολελυμένην γαμήσῃ, μοιχᾶται.
\par }{\PP \VS{33}Πάλιν ἠκούσατε ὅτι ἐρρέθη τοῖς ἀρχαίοις· Οὐκ ἐπιορκήσεις, ἀποδώσεις δὲ τῷ Κυρίῳ τοὺς ὅρκους σου.
\VS{34}ἐγὼ δὲ λέγω ὑμῖν μὴ ὀμόσαι ὅλως· μήτε ἐν τῷ οὐρανῷ, ὅτι θρόνος ἐστὶν τοῦ Θεοῦ,
\VS{35}μήτε ἐν τῇ γῇ, ὅτι ὑποπόδιόν ἐστιν τῶν ποδῶν αὐτοῦ, μήτε εἰς Ἱεροσόλυμα, ὅτι πόλις ἐστὶν τοῦ μεγάλου Βασιλέως,
\VS{36}μήτε ἐν τῇ κεφαλῇ σου ὀμόσῃς, ὅτι οὐ δύνασαι μίαν τρίχα λευκὴν ποιῆσαι ἢ μέλαιναν.
\VS{37}ἔστω δὲ ὁ λόγος ὑμῶν Ναὶ ναί, οὒ Οὔ· τὸ δὲ περισσὸν τούτων ἐκ τοῦ πονηροῦ ἐστιν.
\par }{\PP \VS{38}Ἠκούσατε ὅτι ἐρρέθη· Ὀφθαλμὸν ἀντὶ ὀφθαλμοῦ καὶ ὀδόντα ἀντὶ ὀδόντος.
\VS{39}ἐγὼ δὲ λέγω ὑμῖν μὴ ἀντιστῆναι τῷ πονηρῷ· ἀλλ᾽ ὅστις σε ῥαπίζει εἰς τὴν δεξιὰν σιαγόνα σου, στρέψον αὐτῷ καὶ τὴν ἄλλην·
\VS{40}καὶ τῷ θέλοντί σοι κριθῆναι καὶ τὸν χιτῶνά σου λαβεῖν, ἄφες αὐτῷ καὶ τὸ ἱμάτιον·
\VS{41}καὶ ὅστις σε ἀγγαρεύσει μίλιον ἕν, ὕπαγε μετ᾽ αὐτοῦ δύο.
\VS{42}τῷ αἰτοῦντί σε δός, καὶ τὸν θέλοντα ἀπὸ σοῦ δανίσασθαι μὴ ἀποστραφῇς.
\par }{\PP \VS{43}Ἠκούσατε ὅτι ἐρρέθη· Ἀγαπήσεις τὸν πλησίον σου καὶ μισήσεις τὸν ἐχθρόν σου.
\VS{44}ἐγὼ δὲ λέγω ὑμῖν· ἀγαπᾶτε τοὺς ἐχθροὺς ὑμῶν καὶ προσεύχεσθε ὑπὲρ τῶν διωκόντων ὑμᾶς,
\VS{45}ὅπως γένησθε υἱοὶ τοῦ Πατρὸς ὑμῶν τοῦ ἐν οὐρανοῖς, ὅτι τὸν ἥλιον αὐτοῦ ἀνατέλλει ἐπὶ πονηροὺς καὶ ἀγαθοὺς καὶ βρέχει ἐπὶ δικαίους καὶ ἀδίκους.
\VS{46}ἐὰν γὰρ ἀγαπήσητε τοὺς ἀγαπῶντας ὑμᾶς, τίνα μισθὸν ἔχετε; οὐχὶ καὶ οἱ τελῶναι τὸ αὐτὸ ποιοῦσιν;
\VS{47}καὶ ἐὰν ἀσπάσησθε τοὺς ἀδελφοὺς ὑμῶν μόνον, τί περισσὸν ποιεῖτε; οὐχὶ καὶ οἱ ἐθνικοὶ τὸ αὐτὸ ποιοῦσιν;
\VS{48}Ἔσεσθε οὖν ὑμεῖς τέλειοι ὡς ὁ Πατὴρ ὑμῶν ὁ οὐράνιος τέλειός ἐστιν.

\par }\Chap{6}{\PP \VerseOne{1}Προσέχετε δὲ τὴν δικαιοσύνην ὑμῶν μὴ ποιεῖν ἔμπροσθεν τῶν ἀνθρώπων πρὸς τὸ θεαθῆναι αὐτοῖς· εἰ δὲ μή γε, μισθὸν οὐκ ἔχετε παρὰ τῷ Πατρὶ ὑμῶν τῷ ἐν τοῖς οὐρανοῖς.
\VS{2}Ὅταν οὖν ποιῇς ἐλεημοσύνην, μὴ σαλπίσῃς ἔμπροσθέν σου, ὥσπερ οἱ ὑποκριταὶ ποιοῦσιν ἐν ταῖς συναγωγαῖς καὶ ἐν ταῖς ῥύμαις, ὅπως δοξασθῶσιν ὑπὸ τῶν ἀνθρώπων· ἀμὴν λέγω ὑμῖν, ἀπέχουσιν τὸν μισθὸν αὐτῶν.
\VS{3}σοῦ δὲ ποιοῦντος ἐλεημοσύνην μὴ γνώτω ἡ ἀριστερά σου τί ποιεῖ ἡ δεξιά σου,
\VS{4}ὅπως ᾖ σου ἡ ἐλεημοσύνη ἐν τῷ κρυπτῷ· καὶ ὁ Πατήρ σου ὁ βλέπων ἐν τῷ κρυπτῷ ἀποδώσει σοι.
\VS{5}Καὶ ὅταν προσεύχησθε, οὐκ ἔσεσθε ὡς οἱ ὑποκριταί, ὅτι φιλοῦσιν ἐν ταῖς συναγωγαῖς καὶ ἐν ταῖς γωνίαις τῶν πλατειῶν ἑστῶτες προσεύχεσθαι, ὅπως φανῶσιν τοῖς ἀνθρώποις· ἀμὴν λέγω ὑμῖν, ἀπέχουσιν τὸν μισθὸν αὐτῶν.
\VS{6}σὺ δὲ ὅταν προσεύχῃ, εἴσελθε εἰς τὸ ταμεῖόν σου καὶ κλείσας τὴν θύραν σου πρόσευξαι τῷ Πατρί σου τῷ ἐν τῷ κρυπτῷ· καὶ ὁ Πατήρ σου ὁ βλέπων ἐν τῷ κρυπτῷ ἀποδώσει σοι.
\VS{7}Προσευχόμενοι δὲ μὴ βατταλογήσητε ὥσπερ οἱ ἐθνικοί, δοκοῦσιν γὰρ ὅτι ἐν τῇ πολυλογίᾳ αὐτῶν εἰσακουσθήσονται.
\VS{8}μὴ οὖν ὁμοιωθῆτε αὐτοῖς· οἶδεν γὰρ ὁ Πατὴρ ὑμῶν ὧν χρείαν ἔχετε πρὸ τοῦ ὑμᾶς αἰτῆσαι αὐτόν.
\par }{\PP \VS{9}Οὕτως οὖν προσεύχεσθε ὑμεῖς· ¬Πάτερ ἡμῶν ὁ ἐν τοῖς οὐρανοῖς· ¬Ἁγιασθήτω τὸ ὄνομά σου·
\VS{10}¬Ἐλθέτω ἡ βασιλεία σου· ¬Γενηθήτω τὸ θέλημά σου, ¬Ὡς ἐν οὐρανῷ καὶ ἐπὶ γῆς·
\VS{11}¬Τὸν ἄρτον ἡμῶν τὸν ἐπιούσιον δὸς ἡμῖν σήμερον·
\VS{12}¬Καὶ ἄφες ἡμῖν τὰ ὀφειλήματα ἡμῶν, ¬Ὡς καὶ ἡμεῖς ἀφήκαμεν τοῖς ὀφειλέταις ἡμῶν·
\VS{13}¬Καὶ μὴ εἰσενέγκῃς ἡμᾶς εἰς πειρασμόν, ¬Ἀλλὰ ῥῦσαι ἡμᾶς ἀπὸ τοῦ πονηροῦ.
\par }{\PP \VS{14}Ἐὰν γὰρ ἀφῆτε τοῖς ἀνθρώποις τὰ παραπτώματα αὐτῶν, ἀφήσει καὶ ὑμῖν ὁ Πατὴρ ὑμῶν ὁ οὐράνιος·
\VS{15}ἐὰν δὲ μὴ ἀφῆτε τοῖς ἀνθρώποις, οὐδὲ ὁ Πατὴρ ὑμῶν ἀφήσει τὰ παραπτώματα ὑμῶν.
\par }{\PP \VS{16}Ὅταν δὲ νηστεύητε, μὴ γίνεσθε ὡς οἱ ὑποκριταὶ σκυθρωποί, ἀφανίζουσιν γὰρ τὰ πρόσωπα αὐτῶν ὅπως φανῶσιν τοῖς ἀνθρώποις νηστεύοντες· ἀμὴν λέγω ὑμῖν, ἀπέχουσιν τὸν μισθὸν αὐτῶν.
\VS{17}σὺ δὲ νηστεύων ἄλειψαί σου τὴν κεφαλὴν καὶ τὸ πρόσωπόν σου νίψαι,
\VS{18}ὅπως μὴ φανῇς τοῖς ἀνθρώποις νηστεύων ἀλλὰ τῷ Πατρί σου τῷ ἐν τῷ κρυφαίῳ· καὶ ὁ Πατήρ σου ὁ βλέπων ἐν τῷ κρυφαίῳ ἀποδώσει σοι.
\VS{19}Μὴ θησαυρίζετε ὑμῖν θησαυροὺς ἐπὶ τῆς γῆς, ὅπου σὴς καὶ βρῶσις ἀφανίζει καὶ ὅπου κλέπται διορύσσουσιν καὶ κλέπτουσιν·
\VS{20}θησαυρίζετε δὲ ὑμῖν θησαυροὺς ἐν οὐρανῷ, ὅπου οὔτε σὴς οὔτε βρῶσις ἀφανίζει καὶ ὅπου κλέπται οὐ διορύσσουσιν οὐδὲ κλέπτουσιν·
\VS{21}ὅπου γάρ ἐστιν ὁ θησαυρός σου, ἐκεῖ ἔσται καὶ ἡ καρδία σου.
\par }{\PP \VS{22}Ὁ λύχνος τοῦ σώματός ἐστιν ὁ ὀφθαλμός. ἐὰν οὖν ᾖ ὁ ὀφθαλμός σου ἁπλοῦς, ὅλον τὸ σῶμά σου φωτεινὸν ἔσται·
\VS{23}ἐὰν δὲ ὁ ὀφθαλμός σου πονηρὸς ᾖ, ὅλον τὸ σῶμά σου σκοτεινὸν ἔσται. εἰ οὖν τὸ φῶς τὸ ἐν σοὶ σκότος ἐστίν, τὸ σκότος πόσον.
\par }{\PP \VS{24}Οὐδεὶς δύναται δυσὶ κυρίοις δουλεύειν· ἢ γὰρ τὸν ἕνα μισήσει καὶ τὸν ἕτερον ἀγαπήσει, ἢ ἑνὸς ἀνθέξεται καὶ τοῦ ἑτέρου καταφρονήσει. οὐ δύνασθε Θεῷ δουλεύειν καὶ μαμωνᾷ.
\par }{\PP \VS{25}Διὰ τοῦτο λέγω ὑμῖν· μὴ μεριμνᾶτε τῇ ψυχῇ ὑμῶν τί φάγητε ἢ τί πίητε, μηδὲ τῷ σώματι ὑμῶν τί ἐνδύσησθε. οὐχὶ ἡ ψυχὴ πλεῖόν ἐστιν τῆς τροφῆς καὶ τὸ σῶμα τοῦ ἐνδύματος;
\VS{26}ἐμβλέψατε εἰς τὰ πετεινὰ τοῦ οὐρανοῦ ὅτι οὐ σπείρουσιν οὐδὲ θερίζουσιν οὐδὲ συνάγουσιν εἰς ἀποθήκας, καὶ ὁ Πατὴρ ὑμῶν ὁ οὐράνιος τρέφει αὐτά· οὐχ ὑμεῖς μᾶλλον διαφέρετε αὐτῶν;
\VS{27}τίς δὲ ἐξ ὑμῶν μεριμνῶν δύναται προσθεῖναι ἐπὶ τὴν ἡλικίαν αὐτοῦ πῆχυν ἕνα;
\VS{28}Καὶ περὶ ἐνδύματος τί μεριμνᾶτε; καταμάθετε τὰ κρίνα τοῦ ἀγροῦ πῶς αὐξάνουσιν· οὐ κοπιῶσιν οὐδὲ νήθουσιν·
\VS{29}λέγω δὲ ὑμῖν ὅτι οὐδὲ Σολομὼν ἐν πάσῃ τῇ δόξῃ αὐτοῦ περιεβάλετο ὡς ἓν τούτων.
\VS{30}εἰ δὲ τὸν χόρτον τοῦ ἀγροῦ σήμερον ὄντα καὶ αὔριον εἰς κλίβανον βαλλόμενον ὁ Θεὸς οὕτως ἀμφιέννυσιν, οὐ πολλῷ μᾶλλον ὑμᾶς, ὀλιγόπιστοι;
\VS{31}Μὴ οὖν μεριμνήσητε λέγοντες· Τί φάγωμεν; ἤ· Τί πίωμεν; ἤ· Τί περιβαλώμεθα;
\VS{32}πάντα γὰρ ταῦτα τὰ ἔθνη ἐπιζητοῦσιν· οἶδεν γὰρ ὁ Πατὴρ ὑμῶν ὁ οὐράνιος ὅτι χρῄζετε τούτων ἁπάντων.
\VS{33}ζητεῖτε δὲ πρῶτον τὴν βασιλείαν τοῦ θεοῦ καὶ τὴν δικαιοσύνην αὐτοῦ, καὶ ταῦτα πάντα προστεθήσεται ὑμῖν.
\VS{34}Μὴ οὖν μεριμνήσητε εἰς τὴν αὔριον, ἡ γὰρ αὔριον μεριμνήσει ἑαυτῆς· ἀρκετὸν τῇ ἡμέρᾳ ἡ κακία αὐτῆς.

\par }\Chap{7}{\PP \VerseOne{1}Μὴ κρίνετε, ἵνα μὴ κριθῆτε·
\VS{2}ἐν ᾧ γὰρ κρίματι κρίνετε κριθήσεσθε, καὶ ἐν ᾧ μέτρῳ μετρεῖτε μετρηθήσεται ὑμῖν.
\VS{3}Τί δὲ βλέπεις τὸ κάρφος τὸ ἐν τῷ ὀφθαλμῷ τοῦ ἀδελφοῦ σου, τὴν δὲ ἐν τῷ σῷ ὀφθαλμῷ δοκὸν οὐ κατανοεῖς;
\VS{4}ἢ πῶς ἐρεῖς τῷ ἀδελφῷ σου· Ἄφες ἐκβάλω τὸ κάρφος ἐκ τοῦ ὀφθαλμοῦ σου, καὶ ἰδοὺ ἡ δοκὸς ἐν τῷ ὀφθαλμῷ σοῦ;
\VS{5}ὑποκριτά, ἔκβαλε πρῶτον ἐκ τοῦ ὀφθαλμοῦ σοῦ τὴν δοκόν, καὶ τότε διαβλέψεις ἐκβαλεῖν τὸ κάρφος ἐκ τοῦ ὀφθαλμοῦ τοῦ ἀδελφοῦ σου.
\par }{\PP \VS{6}Μὴ δῶτε τὸ ἅγιον τοῖς κυσίν μηδὲ βάλητε τοὺς μαργαρίτας ὑμῶν ἔμπροσθεν τῶν χοίρων, μήποτε καταπατήσουσιν αὐτοὺς ἐν τοῖς ποσὶν αὐτῶν καὶ στραφέντες ῥήξωσιν ὑμᾶς.
\par }{\PP \VS{7}Αἰτεῖτε καὶ δοθήσεται ὑμῖν, ζητεῖτε καὶ εὑρήσετε, κρούετε καὶ ἀνοιγήσεται ὑμῖν·
\VS{8}πᾶς γὰρ ὁ αἰτῶν λαμβάνει καὶ ὁ ζητῶν εὑρίσκει καὶ τῷ κρούοντι ἀνοιγήσεται.
\VS{9}Ἢ τίς ἐστιν ἐξ ὑμῶν ἄνθρωπος, ὃν αἰτήσει ὁ υἱὸς αὐτοῦ ἄρτον, μὴ λίθον ἐπιδώσει αὐτῷ;
\VS{10}ἢ καὶ ἰχθὺν αἰτήσει, μὴ ὄφιν ἐπιδώσει αὐτῷ;
\VS{11}εἰ οὖν ὑμεῖς πονηροὶ ὄντες οἴδατε δόματα ἀγαθὰ διδόναι τοῖς τέκνοις ὑμῶν, πόσῳ μᾶλλον ὁ Πατὴρ ὑμῶν ὁ ἐν τοῖς οὐρανοῖς δώσει ἀγαθὰ τοῖς αἰτοῦσιν αὐτόν.
\par }{\PP \VS{12}Πάντα οὖν ὅσα ἐὰν θέλητε ἵνα ποιῶσιν ὑμῖν οἱ ἄνθρωποι, οὕτως καὶ ὑμεῖς ποιεῖτε αὐτοῖς· οὗτος γάρ ἐστιν ὁ νόμος καὶ οἱ προφῆται.
\par }{\PP \VS{13}Εἰσέλθατε διὰ τῆς στενῆς πύλης· ὅτι πλατεῖα ἡ πύλη καὶ εὐρύχωρος ἡ ὁδὸς ἡ ἀπάγουσα εἰς τὴν ἀπώλειαν καὶ πολλοί εἰσιν οἱ εἰσερχόμενοι δι᾽ αὐτῆς·
\VS{14}ὅτι* στενὴ ἡ πύλη καὶ τεθλιμμένη ἡ ὁδὸς ἡ ἀπάγουσα εἰς τὴν ζωήν καὶ ὀλίγοι εἰσὶν οἱ εὑρίσκοντες αὐτήν.
\par }{\PP \VS{15}Προσέχετε ἀπὸ τῶν ψευδοπροφητῶν, οἵτινες ἔρχονται πρὸς ὑμᾶς ἐν ἐνδύμασιν προβάτων, ἔσωθεν δέ εἰσιν λύκοι ἅρπαγες.
\VS{16}ἀπὸ τῶν καρπῶν αὐτῶν ἐπιγνώσεσθε αὐτούς. μήτι συλλέγουσιν ἀπὸ ἀκανθῶν σταφυλὰς ἢ ἀπὸ τριβόλων σῦκα;
\VS{17}οὕτως πᾶν δένδρον ἀγαθὸν καρποὺς καλοὺς ποιεῖ, τὸ δὲ σαπρὸν δένδρον καρποὺς πονηροὺς ποιεῖ.
\VS{18}οὐ δύναται δένδρον ἀγαθὸν καρποὺς πονηροὺς ποιεῖν οὐδὲ δένδρον σαπρὸν καρποὺς καλοὺς ποιεῖν.
\VS{19}πᾶν δένδρον μὴ ποιοῦν καρπὸν καλὸν ἐκκόπτεται καὶ εἰς πῦρ βάλλεται.
\VS{20}ἄρα γε ἀπὸ τῶν καρπῶν αὐτῶν ἐπιγνώσεσθε αὐτούς.
\par }{\PP \VS{21}Οὐ πᾶς ὁ λέγων μοι· Κύριε Κύριε, εἰσελεύσεται εἰς τὴν βασιλείαν τῶν οὐρανῶν, ἀλλ᾽ ὁ ποιῶν τὸ θέλημα τοῦ Πατρός μου τοῦ ἐν τοῖς οὐρανοῖς.
\VS{22}πολλοὶ ἐροῦσίν μοι ἐν ἐκείνῃ τῇ ἡμέρᾳ· Κύριε Κύριε, οὐ τῷ σῷ ὀνόματι ἐπροφητεύσαμεν, καὶ τῷ σῷ ὀνόματι δαιμόνια ἐξεβάλομεν, καὶ τῷ σῷ ὀνόματι δυνάμεις πολλὰς ἐποιήσαμεν;
\VS{23}καὶ τότε ὁμολογήσω αὐτοῖς ὅτι Οὐδέποτε ἔγνων ὑμᾶς· ἀποχωρεῖτε ἀπ᾽ ἐμοῦ οἱ ἐργαζόμενοι τὴν ἀνομίαν.
\par }{\PP \VS{24}Πᾶς οὖν ὅστις ἀκούει μου τοὺς λόγους τούτους καὶ ποιεῖ αὐτούς, ὁμοιωθήσεται ἀνδρὶ φρονίμῳ, ὅστις ᾠκοδόμησεν αὐτοῦ τὴν οἰκίαν ἐπὶ τὴν πέτραν·
\VS{25}καὶ κατέβη ἡ βροχὴ καὶ ἦλθον οἱ ποταμοὶ καὶ ἔπνευσαν οἱ ἄνεμοι καὶ προσέπεσαν τῇ οἰκίᾳ ἐκείνῃ, καὶ οὐκ ἔπεσεν, τεθεμελίωτο γὰρ ἐπὶ τὴν πέτραν.
\VS{26}καὶ πᾶς ὁ ἀκούων μου τοὺς λόγους τούτους καὶ μὴ ποιῶν αὐτοὺς ὁμοιωθήσεται ἀνδρὶ μωρῷ, ὅστις ᾠκοδόμησεν αὐτοῦ τὴν οἰκίαν ἐπὶ τὴν ἄμμον·
\VS{27}καὶ κατέβη ἡ βροχὴ καὶ ἦλθον οἱ ποταμοὶ καὶ ἔπνευσαν οἱ ἄνεμοι καὶ προσέκοψαν τῇ οἰκίᾳ ἐκείνῃ, καὶ ἔπεσεν καὶ ἦν ἡ πτῶσις αὐτῆς μεγάλη.
\VS{28}Καὶ ἐγένετο ὅτε ἐτέλεσεν ὁ Ἰησοῦς τοὺς λόγους τούτους, ἐξεπλήσσοντο οἱ ὄχλοι ἐπὶ τῇ διδαχῇ αὐτοῦ·
\VS{29}ἦν γὰρ διδάσκων αὐτοὺς ὡς ἐξουσίαν ἔχων καὶ οὐχ ὡς οἱ γραμματεῖς αὐτῶν.

\par }\Chap{8}{\PP \VerseOne{1}Καταβάντος δὲ αὐτοῦ ἀπὸ τοῦ ὄρους ἠκολούθησαν αὐτῷ ὄχλοι πολλοί.
\VS{2}καὶ ἰδοὺ λεπρὸς προσελθὼν προσεκύνει αὐτῷ λέγων· Κύριε, ἐὰν θέλῃς δύνασαί με καθαρίσαι.
\VS{3}Καὶ ἐκτείνας τὴν χεῖρα ἥψατο αὐτοῦ λέγων· Θέλω, καθαρίσθητι· καὶ εὐθέως ἐκαθαρίσθη αὐτοῦ ἡ λέπρα.
\VS{4}καὶ λέγει αὐτῷ ὁ Ἰησοῦς· Ὅρα μηδενὶ εἴπῃς, ἀλλὰ= ὕπαγε σεαυτὸν δεῖξον τῷ ἱερεῖ καὶ προσένεγκον τὸ δῶρον ὃ προσέταξεν Μωϋσῆς, εἰς μαρτύριον αὐτοῖς.
\par }{\PP \VS{5}Εἰσελθόντος δὲ αὐτοῦ εἰς Καφαρναοὺμ προσῆλθεν αὐτῷ ἑκατόνταρχος παρακαλῶν αὐτὸν
\VS{6}καὶ λέγων· Κύριε, ὁ παῖς μου βέβληται ἐν τῇ οἰκίᾳ παραλυτικός, δεινῶς βασανιζόμενος.
\VS{7}Καὶ λέγει αὐτῷ· Ἐγὼ ἐλθὼν θεραπεύσω αὐτόν.
\VS{8}Καὶ+ ἀποκριθεὶς ὁ ἑκατόνταρχος ἔφη· Κύριε, οὐκ εἰμὶ ἱκανὸς ἵνα μου ὑπὸ τὴν στέγην εἰσέλθῃς, ἀλλὰ μόνον εἰπὲ λόγῳ, καὶ ἰαθήσεται ὁ παῖς μου.
\VS{9}καὶ γὰρ ἐγὼ ἄνθρωπός εἰμι ὑπὸ ἐξουσίαν, ἔχων ὑπ᾽ ἐμαυτὸν στρατιώτας, καὶ λέγω τούτῳ· Πορεύθητι, καὶ πορεύεται, καὶ ἄλλῳ· Ἔρχου, καὶ ἔρχεται, καὶ τῷ δούλῳ μου· Ποίησον τοῦτο, καὶ ποιεῖ.
\VS{10}Ἀκούσας δὲ ὁ Ἰησοῦς ἐθαύμασεν καὶ εἶπεν τοῖς ἀκολουθοῦσιν· Ἀμὴν λέγω ὑμῖν, παρ᾽ οὐδενὶ τοσαύτην πίστιν ἐν τῷ Ἰσραὴλ εὗρον.
\VS{11}λέγω δὲ ὑμῖν ὅτι πολλοὶ ἀπὸ ἀνατολῶν καὶ δυσμῶν ἥξουσιν καὶ ἀνακλιθήσονται μετὰ Ἀβραὰμ καὶ Ἰσαὰκ καὶ Ἰακὼβ ἐν τῇ βασιλείᾳ τῶν οὐρανῶν,
\VS{12}οἱ δὲ υἱοὶ τῆς βασιλείας ἐκβληθήσονται εἰς τὸ σκότος τὸ ἐξώτερον· ἐκεῖ ἔσται ὁ κλαυθμὸς καὶ ὁ βρυγμὸς τῶν ὀδόντων.
\VS{13}Καὶ εἶπεν ὁ Ἰησοῦς τῷ ἑκατοντάρχῃ· Ὕπαγε, ὡς ἐπίστευσας γενηθήτω σοι. καὶ ἰάθη ὁ παῖς αὐτοῦ ἐν τῇ ὥρᾳ ἐκείνῃ.
\par }{\PP \VS{14}Καὶ ἐλθὼν ὁ Ἰησοῦς εἰς τὴν οἰκίαν Πέτρου εἶδεν τὴν πενθερὰν αὐτοῦ βεβλημένην καὶ πυρέσσουσαν·
\VS{15}καὶ ἥψατο τῆς χειρὸς αὐτῆς, καὶ ἀφῆκεν αὐτὴν ὁ πυρετός, καὶ ἠγέρθη καὶ διηκόνει αὐτῷ.
\par }{\PP \VS{16}Ὀψίας δὲ γενομένης προσήνεγκαν αὐτῷ δαιμονιζομένους πολλούς· καὶ ἐξέβαλεν τὰ πνεύματα λόγῳ καὶ πάντας τοὺς κακῶς ἔχοντας ἐθεράπευσεν,
\VS{17}ὅπως πληρωθῇ τὸ ῥηθὲν διὰ Ἠσαΐου τοῦ προφήτου λέγοντος· ¬Αὐτὸς τὰς ἀσθενείας ἡμῶν ἔλαβεν ¬καὶ τὰς νόσους ἐβάστασεν.
\par }{\PP \VS{18}Ἰδὼν δὲ ὁ Ἰησοῦς ὄχλον περὶ αὐτὸν ἐκέλευσεν ἀπελθεῖν εἰς τὸ πέραν.
\VS{19}Καὶ προσελθὼν εἷς γραμματεὺς εἶπεν αὐτῷ· Διδάσκαλε, ἀκολουθήσω σοι ὅπου ἐὰν ἀπέρχῃ.
\VS{20}Καὶ λέγει αὐτῷ ὁ Ἰησοῦς· Αἱ ἀλώπεκες φωλεοὺς ἔχουσιν καὶ τὰ πετεινὰ τοῦ οὐρανοῦ κατασκηνώσεις, ὁ δὲ Υἱὸς τοῦ ἀνθρώπου οὐκ ἔχει ποῦ τὴν κεφαλὴν κλίνῃ.
\VS{21}Ἕτερος δὲ τῶν μαθητῶν αὐτοῦ εἶπεν αὐτῷ· Κύριε, ἐπίτρεψόν μοι πρῶτον ἀπελθεῖν καὶ θάψαι τὸν πατέρα μου.
\VS{22}Ὁ δὲ Ἰησοῦς λέγει αὐτῷ· Ἀκολούθει μοι καὶ ἄφες τοὺς νεκροὺς θάψαι τοὺς ἑαυτῶν νεκρούς.
\par }{\PP \VS{23}Καὶ ἐμβάντι αὐτῷ εἰς τὸ πλοῖον ἠκολούθησαν αὐτῷ οἱ μαθηταὶ αὐτοῦ.
\VS{24}καὶ ἰδοὺ σεισμὸς μέγας ἐγένετο ἐν τῇ θαλάσσῃ, ὥστε τὸ πλοῖον καλύπτεσθαι ὑπὸ τῶν κυμάτων, αὐτὸς δὲ ἐκάθευδεν.
\VS{25}καὶ προσελθόντες ἤγειραν αὐτὸν λέγοντες· Κύριε, σῶσον, ἀπολλύμεθα.
\VS{26}Καὶ λέγει αὐτοῖς· Τί δειλοί ἐστε, ὀλιγόπιστοι; τότε ἐγερθεὶς ἐπετίμησεν τοῖς ἀνέμοις καὶ τῇ θαλάσσῃ, καὶ ἐγένετο γαλήνη μεγάλη.
\VS{27}Οἱ δὲ ἄνθρωποι ἐθαύμασαν λέγοντες· Ποταπός ἐστιν οὗτος ὅτι καὶ οἱ ἄνεμοι καὶ ἡ θάλασσα αὐτῷ ὑπακούουσιν;
\par }{\PP \VS{28}Καὶ ἐλθόντος αὐτοῦ εἰς τὸ πέραν εἰς τὴν χώραν τῶν Γαδαρηνῶν ὑπήντησαν αὐτῷ δύο δαιμονιζόμενοι ἐκ τῶν μνημείων ἐξερχόμενοι, χαλεποὶ λίαν, ὥστε μὴ ἰσχύειν τινὰ παρελθεῖν διὰ τῆς ὁδοῦ ἐκείνης.
\VS{29}Καὶ ἰδοὺ ἔκραξαν λέγοντες· Τί ἡμῖν καὶ σοί, Υἱὲ τοῦ Θεοῦ; ἦλθες ὧδε πρὸ καιροῦ βασανίσαι ἡμᾶς;
\VS{30}Ἦν δὲ μακρὰν ἀπ᾽ αὐτῶν ἀγέλη χοίρων πολλῶν βοσκομένη.
\VS{31}οἱ δὲ δαίμονες παρεκάλουν αὐτὸν λέγοντες· Εἰ ἐκβάλλεις ἡμᾶς, ἀπόστειλον ἡμᾶς εἰς τὴν ἀγέλην τῶν χοίρων.
\VS{32}Καὶ εἶπεν αὐτοῖς· Ὑπάγετε. οἱ δὲ ἐξελθόντες ἀπῆλθον εἰς τοὺς χοίρους· καὶ ἰδοὺ ὥρμησεν πᾶσα ἡ ἀγέλη κατὰ τοῦ κρημνοῦ εἰς τὴν θάλασσαν καὶ ἀπέθανον ἐν τοῖς ὕδασιν.
\VS{33}Οἱ δὲ βόσκοντες ἔφυγον, καὶ ἀπελθόντες εἰς τὴν πόλιν ἀπήγγειλαν πάντα καὶ τὰ τῶν δαιμονιζομένων.
\VS{34}καὶ ἰδοὺ πᾶσα ἡ πόλις ἐξῆλθεν εἰς ὑπάντησιν τῷ Ἰησοῦ καὶ ἰδόντες αὐτὸν παρεκάλεσαν ὅπως μεταβῇ ἀπὸ τῶν ὁρίων αὐτῶν.

\par }\Chap{9}{\PP \VerseOne{1}Καὶ ἐμβὰς εἰς πλοῖον διεπέρασεν καὶ ἦλθεν εἰς τὴν ἰδίαν πόλιν.
\VS{2}Καὶ ἰδοὺ προσέφερον αὐτῷ παραλυτικὸν ἐπὶ κλίνης βεβλημένον. καὶ ἰδὼν ὁ Ἰησοῦς τὴν πίστιν αὐτῶν εἶπεν τῷ παραλυτικῷ· Θάρσει, τέκνον, ἀφίενταί σου αἱ ἁμαρτίαι.
\VS{3}Καὶ ἰδού τινες τῶν γραμματέων εἶπαν ἐν ἑαυτοῖς· Οὗτος βλασφημεῖ.
\VS{4}Καὶ ἰδὼν+ ὁ Ἰησοῦς τὰς ἐνθυμήσεις αὐτῶν εἶπεν· Ἵνατί ἐνθυμεῖσθε πονηρὰ ἐν ταῖς καρδίαις ὑμῶν;
\VS{5}τί γάρ ἐστιν εὐκοπώτερον, εἰπεῖν· Ἀφίενταί σου αἱ ἁμαρτίαι, ἢ εἰπεῖν· Ἔγειρε καὶ περιπάτει;
\VS{6}ἵνα δὲ εἰδῆτε ὅτι ἐξουσίαν ἔχει ὁ Υἱὸς τοῦ ἀνθρώπου ἐπὶ τῆς γῆς ἀφιέναι ἁμαρτίας— τότε λέγει τῷ παραλυτικῷ· Ἐγερθεὶς ἆρόν σου τὴν κλίνην καὶ ὕπαγε εἰς τὸν οἶκόν σου.
\VS{7}καὶ ἐγερθεὶς ἀπῆλθεν εἰς τὸν οἶκον αὐτοῦ.
\VS{8}Ἰδόντες δὲ οἱ ὄχλοι ἐφοβήθησαν καὶ ἐδόξασαν τὸν Θεὸν τὸν δόντα ἐξουσίαν τοιαύτην τοῖς ἀνθρώποις.
\par }{\PP \VS{9}Καὶ παράγων ὁ Ἰησοῦς ἐκεῖθεν εἶδεν ἄνθρωπον καθήμενον ἐπὶ τὸ τελώνιον, Μαθθαῖον λεγόμενον, καὶ λέγει αὐτῷ· Ἀκολούθει μοι. καὶ ἀναστὰς ἠκολούθησεν αὐτῷ.
\par }{\PP \VS{10}Καὶ ἐγένετο αὐτοῦ ἀνακειμένου ἐν τῇ οἰκίᾳ, καὶ ἰδοὺ πολλοὶ τελῶναι καὶ ἁμαρτωλοὶ ἐλθόντες συνανέκειντο τῷ Ἰησοῦ καὶ τοῖς μαθηταῖς αὐτοῦ.
\VS{11}καὶ ἰδόντες οἱ Φαρισαῖοι ἔλεγον τοῖς μαθηταῖς αὐτοῦ· Διὰ τί μετὰ τῶν τελωνῶν καὶ ἁμαρτωλῶν ἐσθίει ὁ διδάσκαλος ὑμῶν;
\VS{12}Ὁ δὲ ἀκούσας εἶπεν· Οὐ χρείαν ἔχουσιν οἱ ἰσχύοντες ἰατροῦ ἀλλ᾽ οἱ κακῶς ἔχοντες.
\VS{13}πορευθέντες δὲ μάθετε τί ἐστιν· Ἔλεος θέλω καὶ οὐ θυσίαν· οὐ γὰρ ἦλθον καλέσαι δικαίους ἀλλὰ= ἁμαρτωλούς.
\VS{14}Τότε προσέρχονται αὐτῷ οἱ μαθηταὶ Ἰωάννου λέγοντες· Διὰ τί ἡμεῖς καὶ οἱ Φαρισαῖοι νηστεύομεν πολλά, οἱ δὲ μαθηταί σου οὐ νηστεύουσιν;
\VS{15}Καὶ εἶπεν αὐτοῖς ὁ Ἰησοῦς· Μὴ δύνανται οἱ υἱοὶ τοῦ νυμφῶνος πενθεῖν ἐφ᾽ ὅσον μετ᾽ αὐτῶν ἐστιν ὁ νυμφίος; ἐλεύσονται δὲ ἡμέραι ὅταν ἀπαρθῇ ἀπ᾽ αὐτῶν ὁ νυμφίος, καὶ τότε νηστεύσουσιν.
\VS{16}Οὐδεὶς δὲ ἐπιβάλλει ἐπίβλημα ῥάκους ἀγνάφου ἐπὶ ἱματίῳ παλαιῷ· αἴρει γὰρ τὸ πλήρωμα αὐτοῦ ἀπὸ τοῦ ἱματίου καὶ χεῖρον σχίσμα γίνεται.
\VS{17}Οὐδὲ βάλλουσιν οἶνον νέον εἰς ἀσκοὺς παλαιούς· εἰ δὲ μή γε, ῥήγνυνται οἱ ἀσκοί καὶ ὁ οἶνος ἐκχεῖται καὶ οἱ ἀσκοὶ ἀπόλλυνται· ἀλλὰ βάλλουσιν οἶνον νέον εἰς ἀσκοὺς καινούς, καὶ ἀμφότεροι συντηροῦνται.
\par }{\PP \VS{18}Ταῦτα αὐτοῦ λαλοῦντος αὐτοῖς, ἰδοὺ ἄρχων εἷς ἐλθὼν προσεκύνει αὐτῷ λέγων ὅτι Ἡ θυγάτηρ μου ἄρτι ἐτελεύτησεν· ἀλλὰ= ἐλθὼν ἐπίθες τὴν χεῖρά σου ἐπ᾽ αὐτήν, καὶ ζήσεται.
\VS{19}Καὶ ἐγερθεὶς ὁ Ἰησοῦς ἠκολούθει* αὐτῷ καὶ οἱ μαθηταὶ αὐτοῦ.
\par }{\PP \VS{20}Καὶ ἰδοὺ γυνὴ αἱμορροοῦσα δώδεκα ἔτη προσελθοῦσα ὄπισθεν ἥψατο τοῦ κρασπέδου τοῦ ἱματίου αὐτοῦ·
\VS{21}ἔλεγεν γὰρ ἐν ἑαυτῇ· Ἐὰν μόνον ἅψωμαι τοῦ ἱματίου αὐτοῦ σωθήσομαι.
\VS{22}Ὁ δὲ Ἰησοῦς στραφεὶς καὶ ἰδὼν αὐτὴν εἶπεν· Θάρσει, θύγατερ· ἡ πίστις σου σέσωκέν σε. καὶ ἐσώθη ἡ γυνὴ ἀπὸ τῆς ὥρας ἐκείνης.
\par }{\PP \VS{23}Καὶ ἐλθὼν ὁ Ἰησοῦς εἰς τὴν οἰκίαν τοῦ ἄρχοντος καὶ ἰδὼν τοὺς αὐλητὰς καὶ τὸν ὄχλον θορυβούμενον
\VS{24}ἔλεγεν· Ἀναχωρεῖτε, οὐ γὰρ ἀπέθανεν τὸ κοράσιον ἀλλὰ καθεύδει. καὶ κατεγέλων αὐτοῦ.
\VS{25}Ὅτε δὲ ἐξεβλήθη ὁ ὄχλος εἰσελθὼν ἐκράτησεν τῆς χειρὸς αὐτῆς, καὶ ἠγέρθη τὸ κοράσιον.
\VS{26}καὶ ἐξῆλθεν ἡ φήμη αὕτη εἰς ὅλην τὴν γῆν ἐκείνην.
\par }{\PP \VS{27}Καὶ παράγοντι ἐκεῖθεν τῷ Ἰησοῦ ἠκολούθησαν αὐτῷ δύο τυφλοὶ κράζοντες καὶ λέγοντες· Ἐλέησον ἡμᾶς, υἱὸς Δαυίδ.
\VS{28}Ἐλθόντι δὲ εἰς τὴν οἰκίαν προσῆλθον αὐτῷ οἱ τυφλοί, καὶ λέγει αὐτοῖς ὁ Ἰησοῦς· Πιστεύετε ὅτι δύναμαι τοῦτο ποιῆσαι; Λέγουσιν αὐτῷ· Ναί Κύριε.
\VS{29}Τότε ἥψατο τῶν ὀφθαλμῶν αὐτῶν λέγων· Κατὰ τὴν πίστιν ὑμῶν γενηθήτω ὑμῖν.
\VS{30}καὶ ἠνεῴχθησαν αὐτῶν οἱ ὀφθαλμοί. καὶ ἐνεβριμήθη αὐτοῖς ὁ Ἰησοῦς λέγων· Ὁρᾶτε μηδεὶς γινωσκέτω.
\VS{31}οἱ δὲ ἐξελθόντες διεφήμισαν αὐτὸν ἐν ὅλῃ τῇ γῇ ἐκείνῃ.
\par }{\PP \VS{32}Αὐτῶν δὲ ἐξερχομένων ἰδοὺ προσήνεγκαν αὐτῷ ἄνθρωπον κωφὸν δαιμονιζόμενον.
\VS{33}καὶ ἐκβληθέντος τοῦ δαιμονίου ἐλάλησεν ὁ κωφός. καὶ ἐθαύμασαν οἱ ὄχλοι λέγοντες· Οὐδέποτε ἐφάνη οὕτως ἐν τῷ Ἰσραήλ.
\VS{34}Οἱ δὲ Φαρισαῖοι ἔλεγον· Ἐν τῷ ἄρχοντι τῶν δαιμονίων ἐκβάλλει τὰ δαιμόνια.
\par }{\PP \VS{35}Καὶ περιῆγεν ὁ Ἰησοῦς τὰς πόλεις πάσας καὶ τὰς κώμας διδάσκων ἐν ταῖς συναγωγαῖς αὐτῶν καὶ κηρύσσων τὸ εὐαγγέλιον τῆς βασιλείας καὶ θεραπεύων πᾶσαν νόσον καὶ πᾶσαν μαλακίαν.
\VS{36}Ἰδὼν δὲ τοὺς ὄχλους ἐσπλαγχνίσθη περὶ αὐτῶν, ὅτι ἦσαν ἐσκυλμένοι καὶ ἐρριμμένοι ὡσεὶ πρόβατα μὴ ἔχοντα ποιμένα.
\VS{37}Τότε λέγει τοῖς μαθηταῖς αὐτοῦ· Ὁ μὲν θερισμὸς πολύς, οἱ δὲ ἐργάται ὀλίγοι·
\VS{38}δεήθητε οὖν τοῦ Κυρίου τοῦ θερισμοῦ ὅπως ἐκβάλῃ ἐργάτας εἰς τὸν θερισμὸν αὐτοῦ.

\par }\Chap{10}{\PP \VerseOne{1}Καὶ προσκαλεσάμενος τοὺς δώδεκα μαθητὰς αὐτοῦ ἔδωκεν αὐτοῖς ἐξουσίαν πνευμάτων ἀκαθάρτων ὥστε ἐκβάλλειν αὐτὰ καὶ θεραπεύειν πᾶσαν νόσον καὶ πᾶσαν μαλακίαν.
\VS{2}Τῶν δὲ δώδεκα ἀποστόλων τὰ ὀνόματά ἐστιν ταῦτα· πρῶτος Σίμων ὁ λεγόμενος Πέτρος καὶ Ἀνδρέας ὁ ἀδελφὸς αὐτοῦ, καὶ Ἰάκωβος ὁ τοῦ Ζεβεδαίου καὶ Ἰωάννης ὁ ἀδελφὸς αὐτοῦ,
\VS{3}Φίλιππος καὶ Βαρθολομαῖος, Θωμᾶς καὶ Μαθθαῖος ὁ τελώνης, Ἰάκωβος ὁ τοῦ Ἁλφαίου καὶ Θαδδαῖος,
\VS{4}Σίμων ὁ Καναναῖος καὶ Ἰούδας ὁ Ἰσκαριώτης ὁ καὶ παραδοὺς αὐτόν.
\par }{\PP \VS{5}Τούτους τοὺς δώδεκα ἀπέστειλεν ὁ Ἰησοῦς παραγγείλας αὐτοῖς λέγων· Εἰς ὁδὸν ἐθνῶν μὴ ἀπέλθητε καὶ εἰς πόλιν Σαμαριτῶν μὴ εἰσέλθητε·
\VS{6}πορεύεσθε δὲ μᾶλλον πρὸς τὰ πρόβατα τὰ ἀπολωλότα οἴκου Ἰσραήλ.
\VS{7}πορευόμενοι δὲ κηρύσσετε λέγοντες ὅτι Ἤγγικεν ἡ βασιλεία τῶν οὐρανῶν.
\VS{8}ἀσθενοῦντας θεραπεύετε, νεκροὺς ἐγείρετε, λεπροὺς καθαρίζετε, δαιμόνια ἐκβάλλετε· δωρεὰν ἐλάβετε, δωρεὰν δότε.
\VS{9}Μὴ κτήσησθε χρυσὸν μηδὲ ἄργυρον μηδὲ χαλκὸν εἰς τὰς ζώνας ὑμῶν,
\VS{10}μὴ πήραν εἰς ὁδὸν μηδὲ δύο χιτῶνας μηδὲ ὑποδήματα μηδὲ ῥάβδον· ἄξιος γὰρ ὁ ἐργάτης τῆς τροφῆς αὐτοῦ.
\VS{11}Εἰς ἣν δ᾽ ἂν πόλιν ἢ κώμην εἰσέλθητε, ἐξετάσατε τίς ἐν αὐτῇ ἄξιός ἐστιν· κἀκεῖ μείνατε ἕως ἂν ἐξέλθητε.
\VS{12}εἰσερχόμενοι δὲ εἰς τὴν οἰκίαν ἀσπάσασθε αὐτήν·
\VS{13}καὶ ἐὰν μὲν ᾖ ἡ οἰκία ἀξία, ἐλθάτω ἡ εἰρήνη ὑμῶν ἐπ᾽ αὐτήν, ἐὰν δὲ μὴ ᾖ ἀξία, ἡ εἰρήνη ὑμῶν πρὸς ὑμᾶς ἐπιστραφήτω.
\VS{14}καὶ ὃς ἂν μὴ δέξηται ὑμᾶς μηδὲ ἀκούσῃ τοὺς λόγους ὑμῶν, ἐξερχόμενοι ἔξω τῆς οἰκίας ἢ τῆς πόλεως ἐκείνης ἐκτινάξατε τὸν κονιορτὸν τῶν ποδῶν ὑμῶν.
\VS{15}ἀμὴν λέγω ὑμῖν, ἀνεκτότερον ἔσται γῇ Σοδόμων καὶ Γομόρρων ἐν ἡμέρᾳ κρίσεως ἢ τῇ πόλει ἐκείνῃ.
\par }{\PP \VS{16}Ἰδοὺ ἐγὼ ἀποστέλλω ὑμᾶς ὡς πρόβατα ἐν μέσῳ λύκων· γίνεσθε οὖν φρόνιμοι ὡς οἱ ὄφεις καὶ ἀκέραιοι ὡς αἱ περιστεραί.
\par }{\PP \VS{17}Προσέχετε δὲ ἀπὸ τῶν ἀνθρώπων· παραδώσουσιν γὰρ ὑμᾶς εἰς συνέδρια καὶ ἐν ταῖς συναγωγαῖς αὐτῶν μαστιγώσουσιν ὑμᾶς·
\VS{18}καὶ ἐπὶ ἡγεμόνας δὲ καὶ βασιλεῖς ἀχθήσεσθε ἕνεκεν ἐμοῦ εἰς μαρτύριον αὐτοῖς καὶ τοῖς ἔθνεσιν.
\VS{19}ὅταν δὲ παραδῶσιν ὑμᾶς, μὴ μεριμνήσητε πῶς ἢ τί λαλήσητε· δοθήσεται γὰρ ὑμῖν ἐν ἐκείνῃ τῇ ὥρᾳ τί λαλήσητε·
\VS{20}οὐ γὰρ ὑμεῖς ἐστε οἱ λαλοῦντες ἀλλὰ τὸ Πνεῦμα τοῦ Πατρὸς ὑμῶν τὸ λαλοῦν ἐν ὑμῖν.
\par }{\PP \VS{21}Παραδώσει δὲ ἀδελφὸς ἀδελφὸν εἰς θάνατον καὶ πατὴρ τέκνον, καὶ ἐπαναστήσονται τέκνα ἐπὶ γονεῖς καὶ θανατώσουσιν αὐτούς.
\VS{22}καὶ ἔσεσθε μισούμενοι ὑπὸ πάντων διὰ τὸ ὄνομά μου· ὁ δὲ ὑπομείνας εἰς τέλος οὗτος σωθήσεται.
\par }{\PP \VS{23}Ὅταν δὲ διώκωσιν ὑμᾶς ἐν τῇ πόλει ταύτῃ, φεύγετε εἰς τὴν ἑτέραν· ἀμὴν γὰρ λέγω ὑμῖν, οὐ μὴ τελέσητε τὰς πόλεις τοῦ Ἰσραὴλ ἕως ἂν ἔλθῃ ὁ Υἱὸς τοῦ ἀνθρώπου.
\par }{\PP \VS{24}Οὐκ ἔστιν μαθητὴς ὑπὲρ τὸν διδάσκαλον οὐδὲ δοῦλος ὑπὲρ τὸν κύριον αὐτοῦ.
\VS{25}ἀρκετὸν τῷ μαθητῇ ἵνα γένηται ὡς ὁ διδάσκαλος αὐτοῦ καὶ ὁ δοῦλος ὡς ὁ κύριος αὐτοῦ. εἰ τὸν οἰκοδεσπότην Βεελζεβοὺλ ἐπεκάλεσαν, πόσῳ μᾶλλον τοὺς οἰκιακοὺς αὐτοῦ.
\par }{\PP \VS{26}Μὴ οὖν φοβηθῆτε αὐτούς· οὐδὲν γάρ ἐστιν κεκαλυμμένον ὃ οὐκ ἀποκαλυφθήσεται καὶ κρυπτὸν ὃ οὐ γνωσθήσεται.
\VS{27}ὃ λέγω ὑμῖν ἐν τῇ σκοτίᾳ εἴπατε ἐν τῷ φωτί, καὶ ὃ εἰς τὸ οὖς ἀκούετε κηρύξατε ἐπὶ τῶν δωμάτων.
\VS{28}Καὶ μὴ φοβεῖσθε ἀπὸ τῶν ἀποκτεννόντων τὸ σῶμα, τὴν δὲ ψυχὴν μὴ δυναμένων ἀποκτεῖναι· φοβεῖσθε δὲ μᾶλλον τὸν δυνάμενον καὶ ψυχὴν καὶ σῶμα ἀπολέσαι ἐν γεέννῃ.
\VS{29}Οὐχὶ δύο στρουθία ἀσσαρίου πωλεῖται; καὶ ἓν ἐξ αὐτῶν οὐ πεσεῖται ἐπὶ τὴν γῆν ἄνευ τοῦ Πατρὸς ὑμῶν.
\VS{30}ὑμῶν δὲ καὶ αἱ τρίχες τῆς κεφαλῆς πᾶσαι ἠριθμημέναι εἰσίν.
\VS{31}μὴ οὖν φοβεῖσθε· πολλῶν στρουθίων διαφέρετε ὑμεῖς.
\par }{\PP \VS{32}Πᾶς οὖν ὅστις ὁμολογήσει ἐν ἐμοὶ ἔμπροσθεν τῶν ἀνθρώπων, ὁμολογήσω κἀγὼ ἐν αὐτῷ ἔμπροσθεν τοῦ Πατρός μου τοῦ ἐν τοῖς οὐρανοῖς·
\VS{33}ὅστις δ᾽ ἂν ἀρνήσηταί με ἔμπροσθεν τῶν ἀνθρώπων, ἀρνήσομαι κἀγὼ αὐτὸν ἔμπροσθεν τοῦ Πατρός μου τοῦ ἐν τοῖς οὐρανοῖς.
\par }{\PP \VS{34}Μὴ νομίσητε ὅτι ἦλθον βαλεῖν εἰρήνην ἐπὶ τὴν γῆν· οὐκ ἦλθον βαλεῖν εἰρήνην ἀλλὰ μάχαιραν.
\VS{35}ἦλθον γὰρ διχάσαι Ἄνθρωπον κατὰ τοῦ πατρὸς αὐτοῦ Καὶ θυγατέρα κατὰ τῆς μητρὸς αὐτῆς Καὶ νύμφην κατὰ τῆς πενθερᾶς αὐτῆς,
\VS{36}Καὶ ἐχθροὶ τοῦ ἀνθρώπου οἱ οἰκιακοὶ αὐτοῦ.
\par }{\PP \VS{37}Ὁ φιλῶν πατέρα ἢ μητέρα ὑπὲρ ἐμὲ οὐκ ἔστιν μου ἄξιος, καὶ ὁ φιλῶν υἱὸν ἢ θυγατέρα ὑπὲρ ἐμὲ οὐκ ἔστιν μου ἄξιος·
\VS{38}καὶ ὃς οὐ λαμβάνει τὸν σταυρὸν αὐτοῦ καὶ ἀκολουθεῖ ὀπίσω μου, οὐκ ἔστιν μου ἄξιος.
\VS{39}ὁ εὑρὼν τὴν ψυχὴν αὐτοῦ ἀπολέσει αὐτήν, καὶ ὁ ἀπολέσας τὴν ψυχὴν αὐτοῦ ἕνεκεν ἐμοῦ εὑρήσει αὐτήν.
\par }{\PP \VS{40}Ὁ δεχόμενος ὑμᾶς ἐμὲ δέχεται, καὶ ὁ ἐμὲ δεχόμενος δέχεται τὸν ἀποστείλαντά με.
\VS{41}ὁ δεχόμενος προφήτην εἰς ὄνομα προφήτου μισθὸν προφήτου λήμψεται, καὶ ὁ δεχόμενος δίκαιον εἰς ὄνομα δικαίου μισθὸν δικαίου λήμψεται.
\VS{42}καὶ ὃς ἂν ποτίσῃ ἕνα τῶν μικρῶν τούτων ποτήριον ψυχροῦ μόνον εἰς ὄνομα μαθητοῦ, ἀμὴν λέγω ὑμῖν, οὐ μὴ ἀπολέσῃ τὸν μισθὸν αὐτοῦ.

\par }\Chap{11}{\PP \VerseOne{1}Καὶ ἐγένετο ὅτε ἐτέλεσεν ὁ Ἰησοῦς διατάσσων τοῖς δώδεκα μαθηταῖς αὐτοῦ, μετέβη ἐκεῖθεν τοῦ διδάσκειν καὶ κηρύσσειν ἐν ταῖς πόλεσιν αὐτῶν.
\VS{2}Ὁ δὲ Ἰωάννης ἀκούσας ἐν τῷ δεσμωτηρίῳ τὰ ἔργα τοῦ Χριστοῦ πέμψας διὰ τῶν μαθητῶν αὐτοῦ
\VS{3}εἶπεν αὐτῷ· Σὺ εἶ ὁ ἐρχόμενος ἢ ἕτερον προσδοκῶμεν;
\VS{4}Καὶ ἀποκριθεὶς ὁ Ἰησοῦς εἶπεν αὐτοῖς· Πορευθέντες ἀπαγγείλατε Ἰωάννῃ ἃ ἀκούετε καὶ βλέπετε·
\VS{5}τυφλοὶ ἀναβλέπουσιν καὶ χωλοὶ περιπατοῦσιν, λεπροὶ καθαρίζονται καὶ κωφοὶ ἀκούουσιν, καὶ νεκροὶ ἐγείρονται καὶ πτωχοὶ εὐαγγελίζονται·
\VS{6}καὶ μακάριός ἐστιν ὃς ἐὰν μὴ σκανδαλισθῇ ἐν ἐμοί.
\par }{\PP \VS{7}Τούτων δὲ πορευομένων ἤρξατο ὁ Ἰησοῦς λέγειν τοῖς ὄχλοις περὶ Ἰωάννου· Τί ἐξήλθατε εἰς τὴν ἔρημον θεάσασθαι; κάλαμον ὑπὸ ἀνέμου σαλευόμενον;
\VS{8}ἀλλὰ τί ἐξήλθατε ἰδεῖν; ἄνθρωπον ἐν μαλακοῖς ἠμφιεσμένον; ἰδοὺ οἱ τὰ μαλακὰ φοροῦντες ἐν τοῖς οἴκοις τῶν βασιλέων εἰσίν.
\VS{9}ἀλλὰ τί ἐξήλθατε ἰδεῖν; προφήτην; ναί λέγω ὑμῖν, καὶ περισσότερον προφήτου.
\VS{10}οὗτός ἐστιν περὶ οὗ γέγραπται· ¬Ἰδοὺ ἐγὼ ἀποστέλλω τὸν ἄγγελόν μου πρὸ προσώπου σου, ¬Ὃς κατασκευάσει τὴν ὁδόν σου ἔμπροσθέν σου.
\par }{\PP \VS{11}Ἀμὴν λέγω ὑμῖν· οὐκ ἐγήγερται ἐν γεννητοῖς γυναικῶν μείζων Ἰωάννου τοῦ Βαπτιστοῦ· ὁ δὲ μικρότερος ἐν τῇ βασιλείᾳ τῶν οὐρανῶν μείζων αὐτοῦ ἐστιν.
\VS{12}ἀπὸ δὲ τῶν ἡμερῶν Ἰωάννου τοῦ Βαπτιστοῦ ἕως ἄρτι ἡ βασιλεία τῶν οὐρανῶν βιάζεται καὶ βιασταὶ ἁρπάζουσιν αὐτήν.
\VS{13}πάντες γὰρ οἱ προφῆται καὶ ὁ νόμος ἕως Ἰωάννου ἐπροφήτευσαν·
\VS{14}καὶ εἰ θέλετε δέξασθαι, αὐτός ἐστιν Ἠλίας ὁ μέλλων ἔρχεσθαι.
\VS{15}Ὁ ἔχων ὦτα ἀκουέτω.
\par }{\PP \VS{16}Τίνι δὲ ὁμοιώσω τὴν γενεὰν ταύτην; ὁμοία ἐστὶν παιδίοις καθημένοις ἐν ταῖς ἀγοραῖς ἃ προσφωνοῦντα τοῖς ἑτέροις
\VS{17}λέγουσιν· ¬Ηὐλήσαμεν ὑμῖν Καὶ οὐκ ὠρχήσασθε, ¬Ἐθρηνήσαμεν Καὶ οὐκ ἐκόψασθε.
\par }{\PP \VS{18}Ἦλθεν γὰρ Ἰωάννης μήτε ἐσθίων μήτε πίνων, καὶ λέγουσιν· Δαιμόνιον ἔχει.
\VS{19}ἦλθεν ὁ Υἱὸς τοῦ ἀνθρώπου ἐσθίων καὶ πίνων, καὶ λέγουσιν· Ἰδοὺ ἄνθρωπος φάγος καὶ οἰνοπότης, τελωνῶν φίλος καὶ ἁμαρτωλῶν. καὶ ἐδικαιώθη ἡ σοφία ἀπὸ τῶν ἔργων αὐτῆς.
\par }{\PP \VS{20}Τότε ἤρξατο ὀνειδίζειν τὰς πόλεις ἐν αἷς ἐγένοντο αἱ πλεῖσται δυνάμεις αὐτοῦ, ὅτι οὐ μετενόησαν·
\VS{21}Οὐαί σοι, Χοραζίν, οὐαί σοι, Βηθσαϊδά· ὅτι εἰ ἐν Τύρῳ καὶ Σιδῶνι ἐγένοντο αἱ δυνάμεις αἱ γενόμεναι ἐν ὑμῖν, πάλαι ἂν ἐν σάκκῳ καὶ σποδῷ μετενόησαν.
\VS{22}πλὴν λέγω ὑμῖν, Τύρῳ καὶ Σιδῶνι ἀνεκτότερον ἔσται ἐν ἡμέρᾳ κρίσεως ἢ ὑμῖν.
\VS{23}Καὶ σύ, Καφαρναούμ, μὴ ἕως οὐρανοῦ ὑψωθήσῃ; ἕως ᾅδου καταβήσῃ· ὅτι εἰ ἐν Σοδόμοις ἐγενήθησαν αἱ δυνάμεις αἱ γενόμεναι ἐν σοί, ἔμεινεν ἂν μέχρι τῆς σήμερον.
\VS{24}πλὴν λέγω ὑμῖν ὅτι γῇ Σοδόμων ἀνεκτότερον ἔσται ἐν ἡμέρᾳ κρίσεως ἢ σοί.
\par }{\PP \VS{25}Ἐν ἐκείνῳ τῷ καιρῷ ἀποκριθεὶς ὁ Ἰησοῦς εἶπεν· Ἐξομολογοῦμαί σοι, Πάτερ, Κύριε τοῦ οὐρανοῦ καὶ τῆς γῆς, ὅτι ἔκρυψας ταῦτα ἀπὸ σοφῶν καὶ συνετῶν καὶ ἀπεκάλυψας αὐτὰ νηπίοις·
\VS{26}ναί ὁ Πατήρ, ὅτι οὕτως εὐδοκία ἐγένετο ἔμπροσθέν σου.
\VS{27}Πάντα μοι παρεδόθη ὑπὸ τοῦ Πατρός μου, καὶ οὐδεὶς ἐπιγινώσκει τὸν Υἱὸν εἰ μὴ ὁ Πατήρ, οὐδὲ τὸν Πατέρα τις ἐπιγινώσκει εἰ μὴ ὁ Υἱὸς καὶ ᾧ ἐὰν βούληται ὁ Υἱὸς ἀποκαλύψαι.
\par }{\PP \VS{28}Δεῦτε πρός με πάντες οἱ κοπιῶντες καὶ πεφορτισμένοι, κἀγὼ ἀναπαύσω ὑμᾶς.
\VS{29}ἄρατε τὸν ζυγόν μου ἐφ᾽ ὑμᾶς καὶ μάθετε ἀπ᾽ ἐμοῦ, ὅτι πραΰς εἰμι καὶ ταπεινὸς τῇ καρδίᾳ, καὶ εὑρήσετε ἀνάπαυσιν ταῖς ψυχαῖς ὑμῶν·
\VS{30}ὁ γὰρ ζυγός μου χρηστὸς καὶ τὸ φορτίον μου ἐλαφρόν ἐστιν.

\par }\Chap{12}{\PP \VerseOne{1}Ἐν ἐκείνῳ τῷ καιρῷ ἐπορεύθη ὁ Ἰησοῦς τοῖς σάββασιν διὰ τῶν σπορίμων· οἱ δὲ μαθηταὶ αὐτοῦ ἐπείνασαν καὶ ἤρξαντο τίλλειν στάχυας καὶ ἐσθίειν.
\VS{2}οἱ δὲ Φαρισαῖοι ἰδόντες εἶπαν αὐτῷ· Ἰδοὺ οἱ μαθηταί σου ποιοῦσιν ὃ οὐκ ἔξεστιν ποιεῖν ἐν σαββάτῳ.
\VS{3}Ὁ δὲ εἶπεν αὐτοῖς· Οὐκ ἀνέγνωτε τί ἐποίησεν Δαυὶδ ὅτε ἐπείνασεν καὶ οἱ μετ᾽ αὐτοῦ,
\VS{4}πῶς εἰσῆλθεν εἰς τὸν οἶκον τοῦ Θεοῦ καὶ τοὺς ἄρτους τῆς προθέσεως ἔφαγον, ὃ οὐκ ἐξὸν ἦν αὐτῷ φαγεῖν οὐδὲ τοῖς μετ᾽ αὐτοῦ εἰ μὴ τοῖς ἱερεῦσιν μόνοις;
\VS{5}Ἢ οὐκ ἀνέγνωτε ἐν τῷ νόμῳ ὅτι τοῖς σάββασιν οἱ ἱερεῖς ἐν τῷ ἱερῷ τὸ σάββατον βεβηλοῦσιν καὶ ἀναίτιοί εἰσιν;
\VS{6}λέγω δὲ ὑμῖν ὅτι τοῦ ἱεροῦ μεῖζόν ἐστιν ὧδε.
\VS{7}Εἰ δὲ ἐγνώκειτε τί ἐστιν· Ἔλεος θέλω καὶ οὐ θυσίαν, οὐκ ἂν κατεδικάσατε τοὺς ἀναιτίους.
\VS{8}κύριος γάρ ἐστιν τοῦ σαββάτου ὁ Υἱὸς τοῦ ἀνθρώπου.
\par }{\PP \VS{9}Καὶ μεταβὰς ἐκεῖθεν ἦλθεν εἰς τὴν συναγωγὴν αὐτῶν·
\VS{10}καὶ ἰδοὺ ἄνθρωπος χεῖρα ἔχων ξηράν. καὶ ἐπηρώτησαν αὐτὸν λέγοντες· Εἰ ἔξεστιν τοῖς σάββασιν θεραπεῦσαι; ἵνα κατηγορήσωσιν αὐτοῦ.
\VS{11}Ὁ δὲ εἶπεν αὐτοῖς· Τίς ἔσται ἐξ ὑμῶν ἄνθρωπος ὃς ἕξει πρόβατον ἕν καὶ ἐὰν ἐμπέσῃ τοῦτο τοῖς σάββασιν εἰς βόθυνον, οὐχὶ κρατήσει αὐτὸ καὶ ἐγερεῖ;
\VS{12}πόσῳ οὖν διαφέρει ἄνθρωπος προβάτου. ὥστε ἔξεστιν τοῖς σάββασιν καλῶς ποιεῖν.
\VS{13}Τότε λέγει τῷ ἀνθρώπῳ· Ἔκτεινόν σου τὴν χεῖρα. καὶ ἐξέτεινεν καὶ ἀπεκατεστάθη ὑγιὴς ὡς ἡ ἄλλη.
\VS{14}ἐξελθόντες δὲ οἱ Φαρισαῖοι συμβούλιον ἔλαβον κατ᾽ αὐτοῦ ὅπως αὐτὸν ἀπολέσωσιν.
\VS{15}Ὁ δὲ Ἰησοῦς γνοὺς ἀνεχώρησεν ἐκεῖθεν. καὶ ἠκολούθησαν αὐτῷ ὄχλοι πολλοί, καὶ ἐθεράπευσεν αὐτοὺς πάντας
\VS{16}καὶ ἐπετίμησεν αὐτοῖς ἵνα μὴ φανερὸν αὐτὸν ποιήσωσιν,
\VS{17}ἵνα πληρωθῇ τὸ ῥηθὲν διὰ Ἠσαΐου τοῦ προφήτου λέγοντος·
\VS{18}¬Ἰδοὺ ὁ παῖς μου ὃν ᾑρέτισα, ¬ὁ ἀγαπητός μου εἰς ὃν εὐδόκησεν ἡ ψυχή μου· ¬θήσω τὸ Πνεῦμά μου ἐπ᾽ αὐτόν, ¬καὶ κρίσιν τοῖς ἔθνεσιν ἀπαγγελεῖ.
\VS{19}¬οὐκ ἐρίσει οὐδὲ κραυγάσει, ¬οὐδὲ ἀκούσει τις ἐν ταῖς πλατείαις τὴν φωνὴν αὐτοῦ.
\VS{20}¬κάλαμον συντετριμμένον οὐ κατεάξει ¬καὶ λίνον τυφόμενον οὐ σβέσει, ¬ἕως ἂν ἐκβάλῃ εἰς νῖκος τὴν κρίσιν.
\VS{21}¬καὶ τῷ ὀνόματι αὐτοῦ ἔθνη ἐλπιοῦσιν.
\par }{\PP \VS{22}Τότε προσηνέχθη αὐτῷ δαιμονιζόμενος τυφλὸς καὶ κωφός, καὶ ἐθεράπευσεν αὐτόν, ὥστε τὸν κωφὸν λαλεῖν καὶ βλέπειν.
\VS{23}καὶ ἐξίσταντο πάντες οἱ ὄχλοι καὶ ἔλεγον· Μήτι οὗτός ἐστιν ὁ υἱὸς Δαυίδ;
\VS{24}Οἱ δὲ Φαρισαῖοι ἀκούσαντες εἶπον· Οὗτος οὐκ ἐκβάλλει τὰ δαιμόνια εἰ μὴ ἐν τῷ Βεελζεβοὺλ ἄρχοντι τῶν δαιμονίων.
\VS{25}Εἰδὼς δὲ τὰς ἐνθυμήσεις αὐτῶν εἶπεν αὐτοῖς· Πᾶσα βασιλεία μερισθεῖσα καθ᾽ ἑαυτῆς ἐρημοῦται καὶ πᾶσα πόλις ἢ οἰκία μερισθεῖσα καθ᾽ ἑαυτῆς οὐ σταθήσεται.
\VS{26}καὶ εἰ ὁ Σατανᾶς τὸν Σατανᾶν ἐκβάλλει, ἐφ᾽ ἑαυτὸν ἐμερίσθη· πῶς οὖν σταθήσεται ἡ βασιλεία αὐτοῦ;
\VS{27}Καὶ εἰ ἐγὼ ἐν Βεελζεβοὺλ ἐκβάλλω τὰ δαιμόνια, οἱ υἱοὶ ὑμῶν ἐν τίνι ἐκβάλλουσιν; διὰ τοῦτο αὐτοὶ κριταὶ ἔσονται ὑμῶν.
\VS{28}εἰ δὲ ἐν Πνεύματι Θεοῦ ἐγὼ ἐκβάλλω τὰ δαιμόνια, ἄρα ἔφθασεν ἐφ᾽ ὑμᾶς ἡ βασιλεία τοῦ Θεοῦ.
\VS{29}Ἢ πῶς δύναταί τις εἰσελθεῖν εἰς τὴν οἰκίαν τοῦ ἰσχυροῦ καὶ τὰ σκεύη αὐτοῦ ἁρπάσαι, ἐὰν μὴ πρῶτον δήσῃ τὸν ἰσχυρόν; καὶ τότε τὴν οἰκίαν αὐτοῦ διαρπάσει.
\VS{30}Ὁ μὴ ὢν μετ᾽ ἐμοῦ κατ᾽ ἐμοῦ ἐστιν, καὶ ὁ μὴ συνάγων μετ᾽ ἐμοῦ σκορπίζει.
\VS{31}Διὰ τοῦτο λέγω ὑμῖν, πᾶσα ἁμαρτία καὶ βλασφημία ἀφεθήσεται τοῖς ἀνθρώποις, ἡ δὲ τοῦ Πνεύματος βλασφημία οὐκ ἀφεθήσεται.
\VS{32}καὶ ὃς ἐὰν εἴπῃ λόγον κατὰ τοῦ Υἱοῦ τοῦ ἀνθρώπου, ἀφεθήσεται αὐτῷ· ὃς δ᾽ ἂν εἴπῃ κατὰ τοῦ Πνεύματος τοῦ Ἁγίου, οὐκ ἀφεθήσεται αὐτῷ οὔτε ἐν τούτῳ τῷ αἰῶνι οὔτε ἐν τῷ μέλλοντι.
\par }{\PP \VS{33}Ἢ ποιήσατε τὸ δένδρον καλὸν καὶ τὸν καρπὸν αὐτοῦ καλόν, ἢ ποιήσατε τὸ δένδρον σαπρὸν καὶ τὸν καρπὸν αὐτοῦ σαπρόν· ἐκ γὰρ τοῦ καρποῦ τὸ δένδρον γινώσκεται.
\VS{34}γεννήματα ἐχιδνῶν, πῶς δύνασθε ἀγαθὰ λαλεῖν πονηροὶ ὄντες; ἐκ γὰρ τοῦ περισσεύματος τῆς καρδίας τὸ στόμα λαλεῖ.
\VS{35}ὁ ἀγαθὸς ἄνθρωπος ἐκ τοῦ ἀγαθοῦ θησαυροῦ ἐκβάλλει ἀγαθά, καὶ ὁ πονηρὸς ἄνθρωπος ἐκ τοῦ πονηροῦ θησαυροῦ ἐκβάλλει πονηρά.
\VS{36}λέγω δὲ ὑμῖν ὅτι πᾶν ῥῆμα ἀργὸν ὃ λαλήσουσιν οἱ ἄνθρωποι ἀποδώσουσιν περὶ αὐτοῦ λόγον ἐν ἡμέρᾳ κρίσεως·
\VS{37}ἐκ γὰρ τῶν λόγων σου δικαιωθήσῃ, καὶ ἐκ τῶν λόγων σου καταδικασθήσῃ.
\par }{\PP \VS{38}Τότε ἀπεκρίθησαν αὐτῷ τινες τῶν γραμματέων καὶ Φαρισαίων λέγοντες· Διδάσκαλε, θέλομεν ἀπὸ σοῦ σημεῖον ἰδεῖν.
\VS{39}Ὁ δὲ ἀποκριθεὶς εἶπεν αὐτοῖς· Γενεὰ πονηρὰ καὶ μοιχαλὶς σημεῖον ἐπιζητεῖ, καὶ σημεῖον οὐ δοθήσεται αὐτῇ εἰ μὴ τὸ σημεῖον Ἰωνᾶ τοῦ προφήτου.
\VS{40}ὥσπερ γὰρ ἦν Ἰωνᾶς ἐν τῇ κοιλίᾳ τοῦ κήτους τρεῖς ἡμέρας καὶ τρεῖς νύκτας, οὕτως ἔσται ὁ Υἱὸς τοῦ ἀνθρώπου ἐν τῇ καρδίᾳ τῆς γῆς τρεῖς ἡμέρας καὶ τρεῖς νύκτας.
\VS{41}Ἄνδρες Νινευῖται ἀναστήσονται ἐν τῇ κρίσει μετὰ τῆς γενεᾶς ταύτης καὶ κατακρινοῦσιν αὐτήν, ὅτι μετενόησαν εἰς τὸ κήρυγμα Ἰωνᾶ, καὶ ἰδοὺ πλεῖον Ἰωνᾶ ὧδε.
\VS{42}βασίλισσα νότου ἐγερθήσεται ἐν τῇ κρίσει μετὰ τῆς γενεᾶς ταύτης καὶ κατακρινεῖ αὐτήν, ὅτι ἦλθεν ἐκ τῶν περάτων τῆς γῆς ἀκοῦσαι τὴν σοφίαν Σολομῶνος, καὶ ἰδοὺ πλεῖον Σολομῶνος ὧδε.
\par }{\PP \VS{43}Ὅταν δὲ τὸ ἀκάθαρτον πνεῦμα ἐξέλθῃ ἀπὸ τοῦ ἀνθρώπου, διέρχεται δι᾽ ἀνύδρων τόπων ζητοῦν ἀνάπαυσιν καὶ οὐχ εὑρίσκει.
\VS{44}τότε λέγει· Εἰς τὸν οἶκόν μου ἐπιστρέψω ὅθεν ἐξῆλθον· καὶ ἐλθὸν εὑρίσκει σχολάζοντα σεσαρωμένον καὶ κεκοσμημένον.
\VS{45}τότε πορεύεται καὶ παραλαμβάνει μεθ᾽ ἑαυτοῦ ἑπτὰ ἕτερα πνεύματα πονηρότερα ἑαυτοῦ καὶ εἰσελθόντα κατοικεῖ ἐκεῖ· καὶ γίνεται τὰ ἔσχατα τοῦ ἀνθρώπου ἐκείνου χείρονα τῶν πρώτων. οὕτως ἔσται καὶ τῇ γενεᾷ ταύτῃ τῇ πονηρᾷ.
\par }{\PP \VS{46}Ἔτι αὐτοῦ λαλοῦντος τοῖς ὄχλοις ἰδοὺ ἡ μήτηρ καὶ οἱ ἀδελφοὶ αὐτοῦ εἱστήκεισαν ἔξω ζητοῦντες αὐτῷ λαλῆσαι.
\VS{47}εἶπεν δέ τις αὐτῷ· Ἰδοὺ ἡ μήτηρ σου καὶ οἱ ἀδελφοί σου ἔξω ἑστήκασιν ζητοῦντές σοι λαλῆσαι.
\VS{48}Ὁ δὲ ἀποκριθεὶς εἶπεν τῷ λέγοντι αὐτῷ· Τίς ἐστιν ἡ μήτηρ μου καὶ τίνες εἰσὶν οἱ ἀδελφοί μου;
\VS{49}καὶ ἐκτείνας τὴν χεῖρα αὐτοῦ ἐπὶ τοὺς μαθητὰς αὐτοῦ εἶπεν· Ἰδοὺ ἡ μήτηρ μου καὶ οἱ ἀδελφοί μου.
\VS{50}ὅστις γὰρ ἂν ποιήσῃ τὸ θέλημα τοῦ Πατρός μου τοῦ ἐν οὐρανοῖς αὐτός μου ἀδελφὸς καὶ ἀδελφὴ καὶ μήτηρ ἐστίν.

\par }\Chap{13}{\PP \VerseOne{1}Ἐν τῇ ἡμέρᾳ ἐκείνῃ ἐξελθὼν ὁ Ἰησοῦς τῆς οἰκίας ἐκάθητο παρὰ τὴν θάλασσαν·
\VS{2}καὶ συνήχθησαν πρὸς αὐτὸν ὄχλοι πολλοί, ὥστε αὐτὸν εἰς πλοῖον ἐμβάντα καθῆσθαι, καὶ πᾶς ὁ ὄχλος ἐπὶ τὸν αἰγιαλὸν εἱστήκει.
\par }{\PP \VS{3}Καὶ ἐλάλησεν αὐτοῖς πολλὰ ἐν παραβολαῖς λέγων· Ἰδοὺ ἐξῆλθεν ὁ σπείρων τοῦ σπείρειν.
\VS{4}καὶ ἐν τῷ σπείρειν αὐτὸν ἃ μὲν ἔπεσεν παρὰ τὴν ὁδόν, καὶ ἐλθόντα τὰ πετεινὰ κατέφαγεν αὐτά.
\VS{5}Ἄλλα δὲ ἔπεσεν ἐπὶ τὰ πετρώδη ὅπου οὐκ εἶχεν γῆν πολλήν, καὶ εὐθέως ἐξανέτειλεν διὰ τὸ μὴ ἔχειν βάθος γῆς·
\VS{6}ἡλίου δὲ ἀνατείλαντος ἐκαυματίσθη καὶ διὰ τὸ μὴ ἔχειν ῥίζαν ἐξηράνθη.
\VS{7}Ἄλλα δὲ ἔπεσεν ἐπὶ τὰς ἀκάνθας, καὶ ἀνέβησαν αἱ ἄκανθαι καὶ ἔπνιξαν αὐτά.
\VS{8}Ἄλλα δὲ ἔπεσεν ἐπὶ τὴν γῆν τὴν καλὴν καὶ ἐδίδου καρπόν, ὃ μὲν ἑκατὸν, ὃ δὲ ἑξήκοντα, ὃ δὲ τριάκοντα.
\VS{9}Ὁ ἔχων ὦτα ἀκουέτω.
\par }{\PP \VS{10}Καὶ προσελθόντες οἱ μαθηταὶ εἶπαν αὐτῷ· Διὰ τί ἐν παραβολαῖς λαλεῖς αὐτοῖς;
\VS{11}Ὁ δὲ ἀποκριθεὶς εἶπεν αὐτοῖς· Ὅτι Ὑμῖν δέδοται γνῶναι τὰ μυστήρια τῆς βασιλείας τῶν οὐρανῶν, ἐκείνοις δὲ οὐ δέδοται.
\VS{12}ὅστις γὰρ ἔχει, δοθήσεται αὐτῷ καὶ περισσευθήσεται· ὅστις δὲ οὐκ ἔχει, καὶ ὃ ἔχει ἀρθήσεται ἀπ᾽ αὐτοῦ.
\VS{13}διὰ τοῦτο ἐν παραβολαῖς αὐτοῖς λαλῶ, Ὅτι βλέποντες οὐ βλέπουσιν Καὶ ἀκούοντες οὐκ ἀκούουσιν οὐδὲ συνίουσιν,
\VS{14}Καὶ ἀναπληροῦται αὐτοῖς ἡ προφητεία Ἠσαΐου ἡ λέγουσα· ¬Ἀκοῇ ἀκούσετε καὶ οὐ μὴ συνῆτε, ¬Καὶ βλέποντες βλέψετε καὶ οὐ μὴ ἴδητε.
\VS{15}¬Ἐπαχύνθη γὰρ ἡ καρδία τοῦ λαοῦ τούτου, ¬Καὶ τοῖς ὠσὶν βαρέως ἤκουσαν ¬Καὶ τοὺς ὀφθαλμοὺς αὐτῶν ἐκάμμυσαν, ¬Μήποτε ἴδωσιν τοῖς ὀφθαλμοῖς ¬Καὶ τοῖς ὠσὶν ἀκούσωσιν ¬Καὶ τῇ καρδίᾳ συνῶσιν ¬Καὶ ἐπιστρέψωσιν Καὶ ἰάσομαι αὐτούς.
\par }{\PP \VS{16}Ὑμῶν δὲ μακάριοι οἱ ὀφθαλμοὶ ὅτι βλέπουσιν καὶ τὰ ὦτα ὑμῶν ὅτι ἀκούουσιν.
\VS{17}ἀμὴν γὰρ λέγω ὑμῖν ὅτι πολλοὶ προφῆται καὶ δίκαιοι ἐπεθύμησαν ἰδεῖν ἃ βλέπετε καὶ οὐκ εἶδαν, καὶ ἀκοῦσαι ἃ ἀκούετε καὶ οὐκ ἤκουσαν.
\par }{\PP \VS{18}Ὑμεῖς οὖν ἀκούσατε τὴν παραβολὴν τοῦ σπείραντος.
\VS{19}Παντὸς ἀκούοντος τὸν λόγον τῆς βασιλείας καὶ μὴ συνιέντος ἔρχεται ὁ πονηρὸς καὶ ἁρπάζει τὸ ἐσπαρμένον ἐν τῇ καρδίᾳ αὐτοῦ, οὗτός ἐστιν ὁ παρὰ τὴν ὁδὸν σπαρείς.
\VS{20}Ὁ δὲ ἐπὶ τὰ πετρώδη σπαρείς, οὗτός ἐστιν ὁ τὸν λόγον ἀκούων καὶ εὐθὺς μετὰ χαρᾶς λαμβάνων αὐτόν,
\VS{21}οὐκ ἔχει δὲ ῥίζαν ἐν ἑαυτῷ ἀλλὰ πρόσκαιρός ἐστιν, γενομένης δὲ θλίψεως ἢ διωγμοῦ διὰ τὸν λόγον εὐθὺς σκανδαλίζεται.
\VS{22}Ὁ δὲ εἰς τὰς ἀκάνθας σπαρείς, οὗτός ἐστιν ὁ τὸν λόγον ἀκούων, καὶ ἡ μέριμνα τοῦ αἰῶνος καὶ ἡ ἀπάτη τοῦ πλούτου συμπνίγει τὸν λόγον καὶ ἄκαρπος γίνεται.
\VS{23}Ὁ δὲ ἐπὶ τὴν καλὴν γῆν σπαρείς, οὗτός ἐστιν ὁ τὸν λόγον ἀκούων καὶ συνιείς, ὃς δὴ καρποφορεῖ καὶ ποιεῖ ὃ μὲν ἑκατὸν, ὃ δὲ ἑξήκοντα, ὃ δὲ τριάκοντα.
\par }{\PP \VS{24}Ἄλλην παραβολὴν παρέθηκεν αὐτοῖς λέγων· Ὡμοιώθη ἡ βασιλεία τῶν οὐρανῶν ἀνθρώπῳ σπείραντι καλὸν σπέρμα ἐν τῷ ἀγρῷ αὐτοῦ.
\VS{25}ἐν δὲ τῷ καθεύδειν τοὺς ἀνθρώπους ἦλθεν αὐτοῦ ὁ ἐχθρὸς καὶ ἐπέσπειρεν ζιζάνια ἀνὰ μέσον τοῦ σίτου καὶ ἀπῆλθεν.
\VS{26}ὅτε δὲ ἐβλάστησεν ὁ χόρτος καὶ καρπὸν ἐποίησεν, τότε ἐφάνη καὶ τὰ ζιζάνια.
\VS{27}Προσελθόντες δὲ οἱ δοῦλοι τοῦ οἰκοδεσπότου εἶπον αὐτῷ· Κύριε, οὐχὶ καλὸν σπέρμα ἔσπειρας ἐν τῷ σῷ ἀγρῷ; πόθεν οὖν ἔχει ζιζάνια;
\VS{28}Ὁ δὲ ἔφη αὐτοῖς· Ἐχθρὸς ἄνθρωπος τοῦτο ἐποίησεν. Οἱ δὲ δοῦλοι λέγουσιν αὐτῷ· Θέλεις οὖν ἀπελθόντες συλλέξωμεν αὐτά;
\VS{29}Ὁ δέ φησιν· Οὔ, μήποτε συλλέγοντες τὰ ζιζάνια ἐκριζώσητε ἅμα αὐτοῖς τὸν σῖτον.
\VS{30}ἄφετε συναυξάνεσθαι ἀμφότερα ἕως τοῦ θερισμοῦ, καὶ ἐν καιρῷ τοῦ θερισμοῦ ἐρῶ τοῖς θερισταῖς· Συλλέξατε πρῶτον τὰ ζιζάνια καὶ δήσατε αὐτὰ εἰς δέσμας πρὸς τὸ κατακαῦσαι αὐτά, τὸν δὲ σῖτον συναγάγετε εἰς τὴν ἀποθήκην μου.
\par }{\PP \VS{31}Ἄλλην παραβολὴν παρέθηκεν αὐτοῖς λέγων· Ὁμοία ἐστὶν ἡ βασιλεία τῶν οὐρανῶν κόκκῳ σινάπεως, ὃν λαβὼν ἄνθρωπος ἔσπειρεν ἐν τῷ ἀγρῷ αὐτοῦ·
\VS{32}ὃ μικρότερον μέν ἐστιν πάντων τῶν σπερμάτων, ὅταν δὲ αὐξηθῇ μεῖζον τῶν λαχάνων ἐστὶν καὶ γίνεται δένδρον, ὥστε ἐλθεῖν τὰ πετεινὰ τοῦ οὐρανοῦ καὶ κατασκηνοῦν ἐν τοῖς κλάδοις αὐτοῦ.
\par }{\PP \VS{33}Ἄλλην παραβολὴν ἐλάλησεν αὐτοῖς· Ὁμοία ἐστὶν ἡ βασιλεία τῶν οὐρανῶν ζύμῃ, ἣν λαβοῦσα γυνὴ ἐνέκρυψεν εἰς ἀλεύρου σάτα τρία ἕως οὗ ἐζυμώθη ὅλον.
\VS{34}Ταῦτα πάντα ἐλάλησεν ὁ Ἰησοῦς ἐν παραβολαῖς τοῖς ὄχλοις καὶ χωρὶς παραβολῆς οὐδὲν ἐλάλει αὐτοῖς,
\VS{35}ὅπως πληρωθῇ τὸ ῥηθὲν διὰ τοῦ προφήτου λέγοντος· ¬Ἀνοίξω ἐν παραβολαῖς τὸ στόμα μου, ¬ἐρεύξομαι κεκρυμμένα ἀπὸ καταβολῆς κόσμου.
\par }{\PP \VS{36}Τότε ἀφεὶς τοὺς ὄχλους ἦλθεν εἰς τὴν οἰκίαν. Καὶ προσῆλθον αὐτῷ οἱ μαθηταὶ αὐτοῦ λέγοντες· Διασάφησον ἡμῖν τὴν παραβολὴν τῶν ζιζανίων τοῦ ἀγροῦ.
\VS{37}Ὁ δὲ ἀποκριθεὶς εἶπεν· Ὁ σπείρων τὸ καλὸν σπέρμα ἐστὶν ὁ Υἱὸς τοῦ ἀνθρώπου,
\VS{38}ὁ δὲ ἀγρός ἐστιν ὁ κόσμος, τὸ δὲ καλὸν σπέρμα οὗτοί εἰσιν οἱ υἱοὶ τῆς βασιλείας· τὰ δὲ ζιζάνιά εἰσιν οἱ υἱοὶ τοῦ πονηροῦ,
\VS{39}ὁ δὲ ἐχθρὸς ὁ σπείρας αὐτά ἐστιν ὁ διάβολος, ὁ δὲ θερισμὸς συντέλεια αἰῶνός ἐστιν, οἱ δὲ θερισταὶ ἄγγελοί εἰσιν.
\VS{40}Ὥσπερ οὖν συλλέγεται τὰ ζιζάνια καὶ πυρὶ κατακαίεται, οὕτως ἔσται ἐν τῇ συντελείᾳ τοῦ αἰῶνος·
\VS{41}ἀποστελεῖ ὁ Υἱὸς τοῦ ἀνθρώπου τοὺς ἀγγέλους αὐτοῦ, καὶ συλλέξουσιν ἐκ τῆς βασιλείας αὐτοῦ πάντα τὰ σκάνδαλα καὶ τοὺς ποιοῦντας τὴν ἀνομίαν
\VS{42}καὶ βαλοῦσιν αὐτοὺς εἰς τὴν κάμινον τοῦ πυρός· ἐκεῖ ἔσται ὁ κλαυθμὸς καὶ ὁ βρυγμὸς τῶν ὀδόντων.
\VS{43}τότε οἱ δίκαιοι ἐκλάμψουσιν ὡς ὁ ἥλιος ἐν τῇ βασιλείᾳ τοῦ Πατρὸς αὐτῶν. Ὁ ἔχων ὦτα ἀκουέτω.
\par }{\PP \VS{44}Ὁμοία ἐστὶν ἡ βασιλεία τῶν οὐρανῶν θησαυρῷ κεκρυμμένῳ ἐν τῷ ἀγρῷ, ὃν εὑρὼν ἄνθρωπος ἔκρυψεν, καὶ ἀπὸ τῆς χαρᾶς αὐτοῦ ὑπάγει καὶ πωλεῖ πάντα ὅσα ἔχει καὶ ἀγοράζει τὸν ἀγρὸν ἐκεῖνον.
\par }{\PP \VS{45}Πάλιν ὁμοία ἐστὶν ἡ βασιλεία τῶν οὐρανῶν ἀνθρώπῳ ἐμπόρῳ ζητοῦντι καλοὺς μαργαρίτας·
\VS{46}εὑρὼν δὲ ἕνα πολύτιμον μαργαρίτην ἀπελθὼν πέπρακεν πάντα ὅσα εἶχεν καὶ ἠγόρασεν αὐτόν.
\par }{\PP \VS{47}Πάλιν ὁμοία ἐστὶν ἡ βασιλεία τῶν οὐρανῶν σαγήνῃ βληθείσῃ εἰς τὴν θάλασσαν καὶ ἐκ παντὸς γένους συναγαγούσῃ·
\VS{48}ἣν ὅτε ἐπληρώθη ἀναβιβάσαντες ἐπὶ τὸν αἰγιαλὸν καὶ καθίσαντες συνέλεξαν τὰ καλὰ εἰς ἄγγη, τὰ δὲ σαπρὰ ἔξω ἔβαλον.
\VS{49}Οὕτως ἔσται ἐν τῇ συντελείᾳ τοῦ αἰῶνος· ἐξελεύσονται οἱ ἄγγελοι καὶ ἀφοριοῦσιν τοὺς πονηροὺς ἐκ μέσου τῶν δικαίων
\VS{50}καὶ βαλοῦσιν αὐτοὺς εἰς τὴν κάμινον τοῦ πυρός· ἐκεῖ ἔσται ὁ κλαυθμὸς καὶ ὁ βρυγμὸς τῶν ὀδόντων.
\par }{\PP \VS{51}Συνήκατε ταῦτα πάντα; Λέγουσιν αὐτῷ· Ναί.
\VS{52}Ὁ δὲ εἶπεν αὐτοῖς· Διὰ τοῦτο πᾶς γραμματεὺς μαθητευθεὶς τῇ βασιλείᾳ τῶν οὐρανῶν ὅμοιός ἐστιν ἀνθρώπῳ οἰκοδεσπότῃ, ὅστις ἐκβάλλει ἐκ τοῦ θησαυροῦ αὐτοῦ καινὰ καὶ παλαιά.
\par }{\PP \VS{53}Καὶ ἐγένετο ὅτε ἐτέλεσεν ὁ Ἰησοῦς τὰς παραβολὰς ταύτας, μετῆρεν ἐκεῖθεν.
\VS{54}καὶ ἐλθὼν εἰς τὴν πατρίδα αὐτοῦ ἐδίδασκεν αὐτοὺς ἐν τῇ συναγωγῇ αὐτῶν, ὥστε ἐκπλήσσεσθαι αὐτοὺς καὶ λέγειν· Πόθεν τούτῳ ἡ σοφία αὕτη καὶ αἱ δυνάμεις;
\VS{55}οὐχ οὗτός ἐστιν ὁ τοῦ τέκτονος υἱός; οὐχ ἡ μήτηρ αὐτοῦ λέγεται Μαριὰμ καὶ οἱ ἀδελφοὶ αὐτοῦ Ἰάκωβος καὶ Ἰωσὴφ καὶ Σίμων καὶ Ἰούδας;
\VS{56}καὶ αἱ ἀδελφαὶ αὐτοῦ οὐχὶ πᾶσαι πρὸς ἡμᾶς εἰσιν; πόθεν οὖν τούτῳ ταῦτα πάντα;
\VS{57}καὶ ἐσκανδαλίζοντο ἐν αὐτῷ. Ὁ δὲ Ἰησοῦς εἶπεν αὐτοῖς· Οὐκ ἔστιν προφήτης ἄτιμος εἰ μὴ ἐν τῇ πατρίδι καὶ ἐν τῇ οἰκίᾳ αὐτοῦ.
\VS{58}καὶ οὐκ ἐποίησεν ἐκεῖ δυνάμεις πολλὰς διὰ τὴν ἀπιστίαν αὐτῶν.

\par }\Chap{14}{\PP \VerseOne{1}Ἐν ἐκείνῳ τῷ καιρῷ ἤκουσεν Ἡρῴδης ὁ τετραάρχης τὴν ἀκοὴν Ἰησοῦ,
\VS{2}καὶ εἶπεν τοῖς παισὶν αὐτοῦ· Οὗτός ἐστιν Ἰωάννης ὁ Βαπτιστής· αὐτὸς ἠγέρθη ἀπὸ τῶν νεκρῶν καὶ διὰ τοῦτο αἱ δυνάμεις ἐνεργοῦσιν ἐν αὐτῷ.
\par }{\PP \VS{3}Ὁ γὰρ Ἡρῴδης κρατήσας τὸν Ἰωάννην+ ἔδησεν αὐτὸν καὶ ἐν φυλακῇ ἀπέθετο διὰ Ἡρῳδιάδα τὴν γυναῖκα Φιλίππου τοῦ ἀδελφοῦ αὐτοῦ·
\VS{4}ἔλεγεν γὰρ ὁ Ἰωάννης αὐτῷ· Οὐκ ἔξεστίν σοι ἔχειν αὐτήν.
\VS{5}καὶ θέλων αὐτὸν ἀποκτεῖναι ἐφοβήθη τὸν ὄχλον, ὅτι ὡς προφήτην αὐτὸν εἶχον.
\par }{\PP \VS{6}Γενεσίοις δὲ γενομένοις τοῦ Ἡρῴδου ὠρχήσατο ἡ θυγάτηρ τῆς Ἡρῳδιάδος ἐν τῷ μέσῳ καὶ ἤρεσεν τῷ Ἡρῴδῃ,
\VS{7}ὅθεν μεθ᾽ ὅρκου ὡμολόγησεν αὐτῇ δοῦναι ὃ ἐὰν αἰτήσηται.
\VS{8}Ἡ δὲ προβιβασθεῖσα ὑπὸ τῆς μητρὸς αὐτῆς· Δός μοι, φησίν, Ὧδε ἐπὶ πίνακι τὴν κεφαλὴν Ἰωάννου τοῦ Βαπτιστοῦ.
\VS{9}Καὶ λυπηθεὶς ὁ βασιλεὺς διὰ τοὺς ὅρκους καὶ τοὺς συνανακειμένους ἐκέλευσεν δοθῆναι,
\VS{10}καὶ πέμψας ἀπεκεφάλισεν τὸν Ἰωάννην ἐν τῇ φυλακῇ.
\VS{11}Καὶ ἠνέχθη ἡ κεφαλὴ αὐτοῦ ἐπὶ πίνακι καὶ ἐδόθη τῷ κορασίῳ, καὶ ἤνεγκεν τῇ μητρὶ αὐτῆς.
\VS{12}καὶ προσελθόντες οἱ μαθηταὶ αὐτοῦ ἦραν τὸ πτῶμα καὶ ἔθαψαν αὐτόν καὶ ἐλθόντες ἀπήγγειλαν τῷ Ἰησοῦ.
\par }{\PP \VS{13}Ἀκούσας δὲ ὁ Ἰησοῦς ἀνεχώρησεν ἐκεῖθεν ἐν πλοίῳ εἰς ἔρημον τόπον κατ᾽ ἰδίαν· καὶ ἀκούσαντες οἱ ὄχλοι ἠκολούθησαν αὐτῷ πεζῇ ἀπὸ τῶν πόλεων.
\VS{14}Καὶ ἐξελθὼν εἶδεν πολὺν ὄχλον καὶ ἐσπλαγχνίσθη ἐπ᾽ αὐτοῖς καὶ ἐθεράπευσεν τοὺς ἀρρώστους αὐτῶν.
\par }{\PP \VS{15}Ὀψίας δὲ γενομένης προσῆλθον αὐτῷ οἱ μαθηταὶ λέγοντες· Ἔρημός ἐστιν ὁ τόπος καὶ ἡ ὥρα ἤδη παρῆλθεν· ἀπόλυσον τοὺς ὄχλους, ἵνα ἀπελθόντες εἰς τὰς κώμας ἀγοράσωσιν ἑαυτοῖς βρώματα.
\VS{16}Ὁ δὲ Ἰησοῦς εἶπεν αὐτοῖς· Οὐ χρείαν ἔχουσιν ἀπελθεῖν, δότε αὐτοῖς ὑμεῖς φαγεῖν.
\VS{17}Οἱ δὲ λέγουσιν αὐτῷ· Οὐκ ἔχομεν ὧδε εἰ μὴ πέντε ἄρτους καὶ δύο ἰχθύας.
\VS{18}Ὁ δὲ εἶπεν· Φέρετέ μοι ὧδε αὐτούς.
\VS{19}καὶ κελεύσας τοὺς ὄχλους ἀνακλιθῆναι ἐπὶ τοῦ χόρτου, λαβὼν τοὺς πέντε ἄρτους καὶ τοὺς δύο ἰχθύας, ἀναβλέψας εἰς τὸν οὐρανὸν εὐλόγησεν καὶ κλάσας ἔδωκεν τοῖς μαθηταῖς τοὺς ἄρτους, οἱ δὲ μαθηταὶ τοῖς ὄχλοις.
\VS{20}Καὶ ἔφαγον πάντες καὶ ἐχορτάσθησαν, καὶ ἦραν τὸ περισσεῦον τῶν κλασμάτων δώδεκα κοφίνους πλήρεις.
\VS{21}οἱ δὲ ἐσθίοντες ἦσαν ἄνδρες ὡσεὶ πεντακισχίλιοι χωρὶς γυναικῶν καὶ παιδίων.
\par }{\PP \VS{22}Καὶ εὐθέως ἠνάγκασεν τοὺς μαθητὰς ἐμβῆναι εἰς τὸ πλοῖον καὶ προάγειν αὐτὸν εἰς τὸ πέραν, ἕως οὗ ἀπολύσῃ τοὺς ὄχλους.
\VS{23}καὶ ἀπολύσας τοὺς ὄχλους ἀνέβη εἰς τὸ ὄρος κατ᾽ ἰδίαν προσεύξασθαι. ὀψίας δὲ γενομένης μόνος ἦν ἐκεῖ.
\VS{24}τὸ δὲ πλοῖον ἤδη σταδίους πολλοὺς ἀπὸ τῆς γῆς ἀπεῖχεν βασανιζόμενον ὑπὸ τῶν κυμάτων, ἦν γὰρ ἐναντίος ὁ ἄνεμος.
\VS{25}Τετάρτῃ δὲ φυλακῇ τῆς νυκτὸς ἦλθεν πρὸς αὐτοὺς περιπατῶν ἐπὶ τὴν θάλασσαν.
\VS{26}οἱ δὲ μαθηταὶ ἰδόντες αὐτὸν ἐπὶ τῆς θαλάσσης περιπατοῦντα ἐταράχθησαν λέγοντες ὅτι Φάντασμά ἐστιν, καὶ ἀπὸ τοῦ φόβου ἔκραξαν.
\VS{27}Εὐθὺς δὲ ἐλάλησεν ὁ Ἰησοῦς αὐτοῖς λέγων· Θαρσεῖτε, ἐγώ εἰμι· μὴ φοβεῖσθε.
\VS{28}Ἀποκριθεὶς δὲ αὐτῷ ὁ Πέτρος εἶπεν· Κύριε, εἰ σὺ εἶ, κέλευσόν με ἐλθεῖν πρὸς σὲ ἐπὶ τὰ ὕδατα.
\VS{29}Ὁ δὲ εἶπεν· Ἐλθέ. Καὶ καταβὰς ἀπὸ τοῦ πλοίου ὁ Πέτρος περιεπάτησεν ἐπὶ τὰ ὕδατα καὶ ἦλθεν πρὸς τὸν Ἰησοῦν.
\VS{30}βλέπων δὲ τὸν ἄνεμον ἰσχυρὸν ἐφοβήθη, καὶ ἀρξάμενος καταποντίζεσθαι ἔκραξεν λέγων· Κύριε, σῶσόν με.
\VS{31}Εὐθέως δὲ ὁ Ἰησοῦς ἐκτείνας τὴν χεῖρα ἐπελάβετο αὐτοῦ καὶ λέγει αὐτῷ· Ὀλιγόπιστε, εἰς τί ἐδίστασας;
\VS{32}Καὶ ἀναβάντων αὐτῶν εἰς τὸ πλοῖον ἐκόπασεν ὁ ἄνεμος.
\VS{33}οἱ δὲ ἐν τῷ πλοίῳ προσεκύνησαν αὐτῷ λέγοντες· Ἀληθῶς Θεοῦ Υἱὸς εἶ.
\par }{\PP \VS{34}Καὶ διαπεράσαντες ἦλθον ἐπὶ τὴν γῆν εἰς Γεννησαρέτ.
\VS{35}καὶ ἐπιγνόντες αὐτὸν οἱ ἄνδρες τοῦ τόπου ἐκείνου ἀπέστειλαν εἰς ὅλην τὴν περίχωρον ἐκείνην καὶ προσήνεγκαν αὐτῷ πάντας τοὺς κακῶς ἔχοντας
\VS{36}καὶ παρεκάλουν αὐτὸν ἵνα μόνον ἅψωνται τοῦ κρασπέδου τοῦ ἱματίου αὐτοῦ· καὶ ὅσοι ἥψαντο διεσώθησαν.

\par }\Chap{15}{\PP \VerseOne{1}Τότε προσέρχονται τῷ Ἰησοῦ ἀπὸ Ἱεροσολύμων Φαρισαῖοι καὶ γραμματεῖς λέγοντες·
\VS{2}Διὰ τί οἱ μαθηταί σου παραβαίνουσιν τὴν παράδοσιν τῶν πρεσβυτέρων; οὐ γὰρ νίπτονται τὰς χεῖρας αὐτῶν ὅταν ἄρτον ἐσθίωσιν.
\VS{3}Ὁ δὲ ἀποκριθεὶς εἶπεν αὐτοῖς· Διὰ τί καὶ ὑμεῖς παραβαίνετε τὴν ἐντολὴν τοῦ Θεοῦ διὰ τὴν παράδοσιν ὑμῶν;
\VS{4}ὁ γὰρ Θεὸς εἶπεν· Τίμα τὸν πατέρα καὶ τὴν μητέρα, καί· Ὁ κακολογῶν πατέρα ἢ μητέρα θανάτῳ τελευτάτω.
\VS{5}ὑμεῖς δὲ λέγετε· Ὃς ἂν εἴπῃ τῷ πατρὶ ἢ τῇ μητρί· Δῶρον ὃ ἐὰν ἐξ ἐμοῦ ὠφεληθῇς,
\VS{6}οὐ μὴ τιμήσει τὸν πατέρα αὐτοῦ· καὶ ἠκυρώσατε τὸν λόγον τοῦ Θεοῦ διὰ τὴν παράδοσιν ὑμῶν.
\VS{7}ὑποκριταί, καλῶς ἐπροφήτευσεν περὶ ὑμῶν Ἠσαΐας λέγων·
\VS{8}¬Ὁ λαὸς οὗτος τοῖς χείλεσίν με τιμᾷ, ¬Ἡ δὲ καρδία αὐτῶν πόρρω ἀπέχει ἀπ᾽ ἐμοῦ·
\VS{9}¬Μάτην δὲ σέβονταί με ¬Διδάσκοντες διδασκαλίας ἐντάλματα ἀνθρώπων.
\par }{\PP \VS{10}Καὶ προσκαλεσάμενος τὸν ὄχλον εἶπεν αὐτοῖς· Ἀκούετε καὶ συνίετε·
\VS{11}οὐ τὸ εἰσερχόμενον εἰς τὸ στόμα κοινοῖ τὸν ἄνθρωπον, ἀλλὰ τὸ ἐκπορευόμενον ἐκ τοῦ στόματος τοῦτο κοινοῖ τὸν ἄνθρωπον.
\par }{\PP \VS{12}Τότε προσελθόντες οἱ μαθηταὶ λέγουσιν αὐτῷ· Οἶδας ὅτι οἱ Φαρισαῖοι ἀκούσαντες τὸν λόγον ἐσκανδαλίσθησαν;
\VS{13}Ὁ δὲ ἀποκριθεὶς εἶπεν· Πᾶσα φυτεία ἣν οὐκ ἐφύτευσεν ὁ Πατήρ μου ὁ οὐράνιος ἐκριζωθήσεται.
\VS{14}ἄφετε αὐτούς· τυφλοί εἰσιν ὁδηγοί τυφλῶν· τυφλὸς δὲ τυφλὸν ἐὰν ὁδηγῇ, ἀμφότεροι εἰς βόθυνον πεσοῦνται.
\par }{\PP \VS{15}Ἀποκριθεὶς δὲ ὁ Πέτρος εἶπεν αὐτῷ· Φράσον ἡμῖν τὴν παραβολήν ταύτην.
\VS{16}Ὁ δὲ εἶπεν· Ἀκμὴν καὶ ὑμεῖς ἀσύνετοί ἐστε;
\VS{17}οὐ νοεῖτε ὅτι πᾶν τὸ εἰσπορευόμενον εἰς τὸ στόμα εἰς τὴν κοιλίαν χωρεῖ καὶ εἰς ἀφεδρῶνα ἐκβάλλεται;
\VS{18}τὰ δὲ ἐκπορευόμενα ἐκ τοῦ στόματος ἐκ τῆς καρδίας ἐξέρχεται, κἀκεῖνα κοινοῖ τὸν ἄνθρωπον.
\VS{19}ἐκ γὰρ τῆς καρδίας ἐξέρχονται διαλογισμοὶ πονηροί, φόνοι, μοιχεῖαι, πορνεῖαι, κλοπαί, ψευδομαρτυρίαι, βλασφημίαι.
\VS{20}ταῦτά ἐστιν τὰ κοινοῦντα τὸν ἄνθρωπον, τὸ δὲ ἀνίπτοις χερσὶν φαγεῖν οὐ κοινοῖ τὸν ἄνθρωπον.
\par }{\PP \VS{21}Καὶ ἐξελθὼν ἐκεῖθεν ὁ Ἰησοῦς ἀνεχώρησεν εἰς τὰ μέρη Τύρου καὶ Σιδῶνος.
\VS{22}καὶ ἰδοὺ γυνὴ Χαναναία ἀπὸ τῶν ὁρίων ἐκείνων ἐξελθοῦσα ἔκραζεν λέγουσα· Ἐλέησόν με, Κύριε υἱὸς Δαυίδ· ἡ θυγάτηρ μου κακῶς δαιμονίζεται.
\VS{23}Ὁ δὲ οὐκ ἀπεκρίθη αὐτῇ λόγον. καὶ προσελθόντες οἱ μαθηταὶ αὐτοῦ ἠρώτουν αὐτὸν λέγοντες· Ἀπόλυσον αὐτήν, ὅτι κράζει ὄπισθεν ἡμῶν.
\VS{24}Ὁ δὲ ἀποκριθεὶς εἶπεν· Οὐκ ἀπεστάλην εἰ μὴ εἰς τὰ πρόβατα τὰ ἀπολωλότα οἴκου Ἰσραήλ.
\VS{25}Ἡ δὲ ἐλθοῦσα προσεκύνει αὐτῷ λέγουσα· Κύριε, βοήθει μοι.
\VS{26}Ὁ δὲ ἀποκριθεὶς εἶπεν· Οὐκ ἔστιν καλὸν λαβεῖν τὸν ἄρτον τῶν τέκνων καὶ βαλεῖν τοῖς κυναρίοις.
\VS{27}Ἡ δὲ εἶπεν· Ναί κύριε, καὶ γὰρ τὰ κυνάρια ἐσθίει ἀπὸ τῶν ψιχίων τῶν πιπτόντων ἀπὸ τῆς τραπέζης τῶν κυρίων αὐτῶν.
\VS{28}Τότε ἀποκριθεὶς ὁ Ἰησοῦς εἶπεν αὐτῇ· Ὦ γύναι, μεγάλη σου ἡ πίστις· γενηθήτω σοι ὡς θέλεις. καὶ ἰάθη ἡ θυγάτηρ αὐτῆς ἀπὸ τῆς ὥρας ἐκείνης.
\par }{\PP \VS{29}Καὶ μεταβὰς ἐκεῖθεν ὁ Ἰησοῦς ἦλθεν παρὰ τὴν θάλασσαν τῆς Γαλιλαίας, καὶ ἀναβὰς εἰς τὸ ὄρος ἐκάθητο ἐκεῖ.
\VS{30}καὶ προσῆλθον αὐτῷ ὄχλοι πολλοὶ ἔχοντες μεθ᾽ ἑαυτῶν χωλούς, τυφλούς, κυλλούς, κωφούς, καὶ ἑτέρους πολλούς καὶ ἔρριψαν αὐτοὺς παρὰ τοὺς πόδας αὐτοῦ, καὶ ἐθεράπευσεν αὐτούς·
\VS{31}ὥστε τὸν ὄχλον θαυμάσαι βλέποντας κωφοὺς λαλοῦντας, κυλλοὺς ὑγιεῖς καὶ χωλοὺς περιπατοῦντας καὶ τυφλοὺς βλέποντας· καὶ ἐδόξασαν τὸν Θεὸν Ἰσραήλ.
\par }{\PP \VS{32}Ὁ δὲ Ἰησοῦς προσκαλεσάμενος τοὺς μαθητὰς αὐτοῦ εἶπεν· Σπλαγχνίζομαι ἐπὶ τὸν ὄχλον, ὅτι ἤδη ἡμέραι τρεῖς προσμένουσίν μοι καὶ οὐκ ἔχουσιν τί φάγωσιν· καὶ ἀπολῦσαι αὐτοὺς νήστεις οὐ θέλω, μήποτε ἐκλυθῶσιν ἐν τῇ ὁδῷ.
\VS{33}Καὶ λέγουσιν αὐτῷ οἱ μαθηταί· Πόθεν ἡμῖν ἐν ἐρημίᾳ ἄρτοι τοσοῦτοι ὥστε χορτάσαι ὄχλον τοσοῦτον;
\VS{34}Καὶ λέγει αὐτοῖς ὁ Ἰησοῦς· Πόσους ἄρτους ἔχετε; Οἱ δὲ εἶπαν· Ἑπτά καὶ ὀλίγα ἰχθύδια.
\VS{35}Καὶ παραγγείλας τῷ ὄχλῳ ἀναπεσεῖν ἐπὶ τὴν γῆν
\VS{36}ἔλαβεν τοὺς ἑπτὰ ἄρτους καὶ τοὺς ἰχθύας καὶ εὐχαριστήσας ἔκλασεν καὶ ἐδίδου τοῖς μαθηταῖς, οἱ δὲ μαθηταὶ τοῖς ὄχλοις.
\VS{37}Καὶ ἔφαγον πάντες καὶ ἐχορτάσθησαν. καὶ τὸ περισσεῦον τῶν κλασμάτων ἦραν ἑπτὰ σπυρίδας πλήρεις.
\VS{38}οἱ δὲ ἐσθίοντες ἦσαν τετρακισχίλιοι ἄνδρες χωρὶς γυναικῶν καὶ παιδίων.
\par }{\PP \VS{39}Καὶ ἀπολύσας τοὺς ὄχλους ἐνέβη εἰς τὸ πλοῖον καὶ ἦλθεν εἰς τὰ ὅρια Μαγαδάν.

\par }\Chap{16}{\PP \VerseOne{1}Καὶ προσελθόντες οἱ Φαρισαῖοι καὶ Σαδδουκαῖοι πειράζοντες ἐπηρώτησαν αὐτὸν σημεῖον ἐκ τοῦ οὐρανοῦ ἐπιδεῖξαι αὐτοῖς.
\VS{2}Ὁ δὲ ἀποκριθεὶς εἶπεν αὐτοῖς· Ὀψίας γενομένης λέγετε· Εὐδία, πυρράζει γὰρ ὁ οὐρανός·
\VS{3}καὶ πρωΐ· Σήμερον χειμών, πυρράζει γὰρ στυγνάζων ὁ οὐρανός. τὸ μὲν πρόσωπον τοῦ οὐρανοῦ γινώσκετε διακρίνειν, τὰ δὲ σημεῖα τῶν καιρῶν οὐ δύνασθε;
\VS{4}γενεὰ πονηρὰ καὶ μοιχαλὶς σημεῖον ἐπιζητεῖ, καὶ σημεῖον οὐ δοθήσεται αὐτῇ εἰ μὴ τὸ σημεῖον Ἰωνᾶ. καὶ καταλιπὼν αὐτοὺς ἀπῆλθεν.
\par }{\PP \VS{5}Καὶ ἐλθόντες οἱ μαθηταὶ εἰς τὸ πέραν ἐπελάθοντο ἄρτους λαβεῖν.
\VS{6}ὁ δὲ Ἰησοῦς εἶπεν αὐτοῖς· Ὁρᾶτε καὶ προσέχετε ἀπὸ τῆς ζύμης τῶν Φαρισαίων καὶ Σαδδουκαίων.
\VS{7}Οἱ δὲ διελογίζοντο ἐν ἑαυτοῖς λέγοντες Ὅτι Ἄρτους οὐκ ἐλάβομεν.
\VS{8}Γνοὺς δὲ ὁ Ἰησοῦς εἶπεν· Τί διαλογίζεσθε ἐν ἑαυτοῖς, ὀλιγόπιστοι, ὅτι ἄρτους οὐκ ἔχετε;
\VS{9}οὔπω νοεῖτε, οὐδὲ μνημονεύετε τοὺς πέντε ἄρτους τῶν πεντακισχιλίων καὶ πόσους κοφίνους ἐλάβετε;
\VS{10}οὐδὲ τοὺς ἑπτὰ ἄρτους τῶν τετρακισχιλίων καὶ πόσας σπυρίδας ἐλάβετε;
\VS{11}πῶς οὐ νοεῖτε ὅτι οὐ περὶ ἄρτων εἶπον ὑμῖν; προσέχετε δὲ ἀπὸ τῆς ζύμης τῶν Φαρισαίων καὶ Σαδδουκαίων.
\VS{12}Τότε συνῆκαν ὅτι οὐκ εἶπεν προσέχειν ἀπὸ τῆς ζύμης τῶν ἄρτων ἀλλὰ= ἀπὸ τῆς διδαχῆς τῶν Φαρισαίων καὶ Σαδδουκαίων.
\par }{\PP \VS{13}Ἐλθὼν δὲ ὁ Ἰησοῦς εἰς τὰ μέρη Καισαρείας τῆς Φιλίππου ἠρώτα τοὺς μαθητὰς αὐτοῦ λέγων· Τίνα λέγουσιν οἱ ἄνθρωποι εἶναι τὸν Υἱὸν τοῦ ἀνθρώπου;
\VS{14}Οἱ δὲ εἶπαν· Οἱ μὲν Ἰωάννην τὸν Βαπτιστήν, ἄλλοι δὲ Ἠλίαν, ἕτεροι δὲ Ἰερεμίαν ἢ ἕνα τῶν προφητῶν.
\VS{15}Λέγει αὐτοῖς· Ὑμεῖς δὲ τίνα με λέγετε εἶναι;
\VS{16}Ἀποκριθεὶς δὲ Σίμων Πέτρος εἶπεν· Σὺ εἶ ὁ Χριστὸς ὁ Υἱὸς τοῦ Θεοῦ τοῦ ζῶντος.
\VS{17}Ἀποκριθεὶς δὲ ὁ Ἰησοῦς εἶπεν αὐτῷ· Μακάριος εἶ, Σίμων Βαριωνᾶ, ὅτι σὰρξ καὶ αἷμα οὐκ ἀπεκάλυψέν σοι ἀλλ᾽ ὁ Πατήρ μου ὁ ἐν τοῖς οὐρανοῖς.
\VS{18}κἀγὼ δέ σοι λέγω ὅτι σὺ εἶ Πέτρος, καὶ ἐπὶ ταύτῃ τῇ πέτρᾳ οἰκοδομήσω μου τὴν ἐκκλησίαν καὶ πύλαι ᾅδου οὐ κατισχύσουσιν αὐτῆς.
\VS{19}δώσω σοι τὰς κλεῖδας τῆς βασιλείας τῶν οὐρανῶν, καὶ ὃ ἐὰν δήσῃς ἐπὶ τῆς γῆς ἔσται δεδεμένον ἐν τοῖς οὐρανοῖς, καὶ ὃ ἐὰν λύσῃς ἐπὶ τῆς γῆς ἔσται λελυμένον ἐν τοῖς οὐρανοῖς.
\VS{20}Τότε διεστείλατο τοῖς μαθηταῖς ἵνα μηδενὶ εἴπωσιν ὅτι αὐτός ἐστιν ὁ Χριστός.
\par }{\PP \VS{21}Ἀπὸ τότε ἤρξατο ὁ Ἰησοῦς δεικνύειν τοῖς μαθηταῖς αὐτοῦ ὅτι δεῖ αὐτὸν εἰς Ἱεροσόλυμα ἀπελθεῖν καὶ πολλὰ παθεῖν ἀπὸ τῶν πρεσβυτέρων καὶ ἀρχιερέων καὶ γραμματέων καὶ ἀποκτανθῆναι καὶ τῇ τρίτῃ ἡμέρᾳ ἐγερθῆναι.
\VS{22}Καὶ προσλαβόμενος αὐτὸν ὁ Πέτρος ἤρξατο ἐπιτιμᾶν αὐτῷ λέγων· Ἵλεώς σοι, Κύριε· οὐ μὴ ἔσται σοι τοῦτο.
\VS{23}Ὁ δὲ στραφεὶς εἶπεν τῷ Πέτρῳ· Ὕπαγε ὀπίσω μου, Σατανᾶ· σκάνδαλον εἶ ἐμοῦ, ὅτι οὐ φρονεῖς τὰ τοῦ Θεοῦ ἀλλὰ τὰ τῶν ἀνθρώπων.
\par }{\PP \VS{24}Τότε ὁ Ἰησοῦς εἶπεν τοῖς μαθηταῖς αὐτοῦ· Εἴ τις θέλει ὀπίσω μου ἐλθεῖν, ἀπαρνησάσθω ἑαυτὸν καὶ ἀράτω τὸν σταυρὸν αὐτοῦ καὶ ἀκολουθείτω μοι.
\VS{25}ὃς γὰρ ἐὰν θέλῃ τὴν ψυχὴν αὐτοῦ σῶσαι ἀπολέσει αὐτήν· ὃς δ᾽ ἂν ἀπολέσῃ τὴν ψυχὴν αὐτοῦ ἕνεκεν ἐμοῦ εὑρήσει αὐτήν.
\VS{26}τί γὰρ ὠφεληθήσεται ἄνθρωπος ἐὰν τὸν κόσμον ὅλον κερδήσῃ τὴν δὲ ψυχὴν αὐτοῦ ζημιωθῇ; ἢ τί δώσει ἄνθρωπος ἀντάλλαγμα τῆς ψυχῆς αὐτοῦ;
\VS{27}μέλλει γὰρ ὁ Υἱὸς τοῦ ἀνθρώπου ἔρχεσθαι ἐν τῇ δόξῃ τοῦ Πατρὸς αὐτοῦ μετὰ τῶν ἀγγέλων αὐτοῦ, καὶ τότε ἀποδώσει ἑκάστῳ κατὰ τὴν πρᾶξιν αὐτοῦ.
\VS{28}Ἀμὴν λέγω ὑμῖν ὅτι εἰσίν τινες τῶν ὧδε ἑστώτων οἵτινες οὐ μὴ γεύσωνται θανάτου ἕως ἂν ἴδωσιν τὸν Υἱὸν τοῦ ἀνθρώπου ἐρχόμενον ἐν τῇ βασιλείᾳ αὐτοῦ.

\par }\Chap{17}{\PP \VerseOne{1}Καὶ μεθ᾽ ἡμέρας ἓξ παραλαμβάνει ὁ Ἰησοῦς τὸν Πέτρον καὶ Ἰάκωβον καὶ Ἰωάννην τὸν ἀδελφὸν αὐτοῦ καὶ ἀναφέρει αὐτοὺς εἰς ὄρος ὑψηλὸν κατ᾽ ἰδίαν.
\VS{2}καὶ μετεμορφώθη ἔμπροσθεν αὐτῶν, καὶ ἔλαμψεν τὸ πρόσωπον αὐτοῦ ὡς ὁ ἥλιος, τὰ δὲ ἱμάτια αὐτοῦ ἐγένετο λευκὰ ὡς τὸ φῶς.
\VS{3}Καὶ ἰδοὺ ὤφθη αὐτοῖς Μωϋσῆς καὶ Ἠλίας συλλαλοῦντες μετ᾽ αὐτοῦ.
\VS{4}ἀποκριθεὶς δὲ ὁ Πέτρος εἶπεν τῷ Ἰησοῦ· Κύριε, καλόν ἐστιν ἡμᾶς ὧδε εἶναι· εἰ θέλεις, ποιήσω ὧδε τρεῖς σκηνάς, σοὶ μίαν καὶ Μωϋσεῖ μίαν καὶ Ἠλίᾳ μίαν.
\VS{5}Ἔτι αὐτοῦ λαλοῦντος ἰδοὺ νεφέλη φωτεινὴ ἐπεσκίασεν αὐτούς, καὶ ἰδοὺ φωνὴ ἐκ τῆς νεφέλης λέγουσα· Οὗτός ἐστιν ὁ Υἱός μου ὁ ἀγαπητός, ἐν ᾧ εὐδόκησα· ἀκούετε αὐτοῦ.
\VS{6}καὶ ἀκούσαντες οἱ μαθηταὶ ἔπεσαν ἐπὶ πρόσωπον αὐτῶν καὶ ἐφοβήθησαν σφόδρα.
\VS{7}Καὶ προσῆλθεν ὁ Ἰησοῦς καὶ ἁψάμενος αὐτῶν εἶπεν· Ἐγέρθητε καὶ μὴ φοβεῖσθε.
\VS{8}ἐπάραντες δὲ τοὺς ὀφθαλμοὺς αὐτῶν οὐδένα εἶδον εἰ μὴ αὐτὸν Ἰησοῦν μόνον.
\par }{\PP \VS{9}Καὶ καταβαινόντων αὐτῶν ἐκ τοῦ ὄρους ἐνετείλατο αὐτοῖς ὁ Ἰησοῦς λέγων· Μηδενὶ εἴπητε τὸ ὅραμα ἕως οὗ ὁ Υἱὸς τοῦ ἀνθρώπου ἐκ νεκρῶν ἐγερθῇ.
\VS{10}Καὶ ἐπηρώτησαν αὐτὸν οἱ μαθηταὶ λέγοντες· Τί οὖν οἱ γραμματεῖς λέγουσιν ὅτι Ἠλίαν δεῖ ἐλθεῖν πρῶτον;
\VS{11}Ὁ δὲ ἀποκριθεὶς εἶπεν· Ἠλίας μὲν ἔρχεται καὶ ἀποκαταστήσει πάντα·
\VS{12}λέγω δὲ ὑμῖν ὅτι Ἠλίας ἤδη ἦλθεν, καὶ οὐκ ἐπέγνωσαν αὐτὸν ἀλλὰ= ἐποίησαν ἐν αὐτῷ ὅσα ἠθέλησαν· οὕτως καὶ ὁ Υἱὸς τοῦ ἀνθρώπου μέλλει πάσχειν ὑπ᾽ αὐτῶν.
\VS{13}τότε συνῆκαν οἱ μαθηταὶ ὅτι περὶ Ἰωάννου τοῦ Βαπτιστοῦ εἶπεν αὐτοῖς.
\par }{\PP \VS{14}Καὶ ἐλθόντων πρὸς τὸν ὄχλον προσῆλθεν αὐτῷ ἄνθρωπος γονυπετῶν αὐτὸν
\VS{15}καὶ λέγων· Κύριε, ἐλέησόν μου τὸν υἱόν, ὅτι σεληνιάζεται καὶ κακῶς πάσχει· πολλάκις γὰρ πίπτει εἰς τὸ πῦρ καὶ πολλάκις εἰς τὸ ὕδωρ.
\VS{16}καὶ προσήνεγκα αὐτὸν τοῖς μαθηταῖς σου, καὶ οὐκ ἠδυνήθησαν αὐτὸν θεραπεῦσαι.
\VS{17}Ἀποκριθεὶς δὲ ὁ Ἰησοῦς εἶπεν· Ὦ γενεὰ ἄπιστος καὶ διεστραμμένη, ἕως πότε μεθ᾽ ὑμῶν ἔσομαι; ἕως πότε ἀνέξομαι ὑμῶν; φέρετέ μοι αὐτὸν ὧδε.
\VS{18}καὶ ἐπετίμησεν αὐτῷ ὁ Ἰησοῦς καὶ ἐξῆλθεν ἀπ᾽ αὐτοῦ τὸ δαιμόνιον καὶ ἐθεραπεύθη ὁ παῖς ἀπὸ τῆς ὥρας ἐκείνης.
\par }{\PP \VS{19}Τότε προσελθόντες οἱ μαθηταὶ τῷ Ἰησοῦ κατ᾽ ἰδίαν εἶπον· Διὰ τί ἡμεῖς οὐκ ἠδυνήθημεν ἐκβαλεῖν αὐτό;
\VS{20}Ὁ δὲ λέγει αὐτοῖς· Διὰ τὴν ὀλιγοπιστίαν ὑμῶν· ἀμὴν γὰρ λέγω ὑμῖν, ἐὰν ἔχητε πίστιν ὡς κόκκον σινάπεως, ἐρεῖτε τῷ ὄρει τούτῳ· Μετάβα ἔνθεν ἐκεῖ, καὶ μεταβήσεται· καὶ οὐδὲν ἀδυνατήσει ὑμῖν.
\par }{\PP \VS{22}Συστρεφομένων δὲ αὐτῶν ἐν τῇ Γαλιλαίᾳ εἶπεν αὐτοῖς ὁ Ἰησοῦς· Μέλλει ὁ Υἱὸς τοῦ ἀνθρώπου παραδίδοσθαι εἰς χεῖρας ἀνθρώπων,
\VS{23}καὶ ἀποκτενοῦσιν αὐτόν, καὶ τῇ τρίτῃ ἡμέρᾳ ἐγερθήσεται. καὶ ἐλυπήθησαν σφόδρα.
\par }{\PP \VS{24}Ἐλθόντων δὲ αὐτῶν εἰς Καφαρναοὺμ προσῆλθον οἱ τὰ δίδραχμα λαμβάνοντες τῷ Πέτρῳ καὶ εἶπαν· Ὁ διδάσκαλος ὑμῶν οὐ τελεῖ τὰ δίδραχμα;
\VS{25}Λέγει· Ναί. Καὶ ἐλθόντα εἰς τὴν οἰκίαν προέφθασεν αὐτὸν ὁ Ἰησοῦς λέγων· Τί σοι δοκεῖ, Σίμων; οἱ βασιλεῖς τῆς γῆς ἀπὸ τίνων λαμβάνουσιν τέλη ἢ κῆνσον; ἀπὸ τῶν υἱῶν αὐτῶν ἢ ἀπὸ τῶν ἀλλοτρίων;
\VS{26}Εἰπόντος δέ· Ἀπὸ τῶν ἀλλοτρίων, ἔφη αὐτῷ ὁ Ἰησοῦς· Ἄρα Γε ἐλεύθεροί εἰσιν οἱ υἱοί.
\VS{27}ἵνα δὲ μὴ σκανδαλίσωμεν αὐτούς, πορευθεὶς εἰς θάλασσαν βάλε ἄγκιστρον καὶ τὸν ἀναβάντα πρῶτον ἰχθὺν ἆρον, καὶ ἀνοίξας τὸ στόμα αὐτοῦ εὑρήσεις στατῆρα· ἐκεῖνον λαβὼν δὸς αὐτοῖς ἀντὶ ἐμοῦ καὶ σοῦ.

\par }\Chap{18}{\PP \VerseOne{1}Ἐν ἐκείνῃ τῇ ὥρᾳ προσῆλθον οἱ μαθηταὶ τῷ Ἰησοῦ λέγοντες· Τίς ἄρα μείζων ἐστὶν ἐν τῇ βασιλείᾳ τῶν οὐρανῶν;
\VS{2}Καὶ προσκαλεσάμενος παιδίον ἔστησεν αὐτὸ ἐν μέσῳ αὐτῶν
\VS{3}καὶ εἶπεν· Ἀμὴν λέγω ὑμῖν, ἐὰν μὴ στραφῆτε καὶ γένησθε ὡς τὰ παιδία, οὐ μὴ εἰσέλθητε εἰς τὴν βασιλείαν τῶν οὐρανῶν.
\VS{4}ὅστις οὖν ταπεινώσει ἑαυτὸν ὡς τὸ παιδίον τοῦτο, οὗτός ἐστιν ὁ μείζων ἐν τῇ βασιλείᾳ τῶν οὐρανῶν.
\VS{5}καὶ ὃς ἐὰν δέξηται ἓν παιδίον τοιοῦτο ἐπὶ τῷ ὀνόματί μου, ἐμὲ δέχεται.
\par }{\PP \VS{6}Ὃς δ᾽ ἂν σκανδαλίσῃ ἕνα τῶν μικρῶν τούτων τῶν πιστευόντων εἰς ἐμέ, συμφέρει αὐτῷ ἵνα κρεμασθῇ μύλος ὀνικὸς περὶ τὸν τράχηλον αὐτοῦ καὶ καταποντισθῇ ἐν τῷ πελάγει τῆς θαλάσσης.
\VS{7}Οὐαὶ τῷ κόσμῳ ἀπὸ τῶν σκανδάλων· ἀνάγκη γὰρ ἐλθεῖν τὰ σκάνδαλα, πλὴν οὐαὶ τῷ ἀνθρώπῳ δι᾽ οὗ τὸ σκάνδαλον ἔρχεται.
\VS{8}Εἰ δὲ ἡ χείρ σου ἢ ὁ πούς σου σκανδαλίζει σε, ἔκκοψον αὐτὸν καὶ βάλε ἀπὸ σοῦ· καλόν σοί ἐστιν εἰσελθεῖν εἰς τὴν ζωὴν κυλλὸν ἢ χωλόν ἢ δύο χεῖρας ἢ δύο πόδας ἔχοντα βληθῆναι εἰς τὸ πῦρ τὸ αἰώνιον.
\VS{9}καὶ εἰ ὁ ὀφθαλμός σου σκανδαλίζει σε, ἔξελε αὐτὸν καὶ βάλε ἀπὸ σοῦ· καλόν σοί ἐστιν μονόφθαλμον εἰς τὴν ζωὴν εἰσελθεῖν ἢ δύο ὀφθαλμοὺς ἔχοντα βληθῆναι εἰς τὴν γέενναν τοῦ πυρός.
\par }{\PP \VS{10}Ὁρᾶτε μὴ καταφρονήσητε ἑνὸς τῶν μικρῶν τούτων· λέγω γὰρ ὑμῖν ὅτι οἱ ἄγγελοι αὐτῶν ἐν οὐρανοῖς διὰ παντὸς βλέπουσι= τὸ πρόσωπον τοῦ Πατρός μου τοῦ ἐν οὐρανοῖς.
\par }{\PP \VS{12}Τί ὑμῖν δοκεῖ; ἐὰν γένηταί τινι ἀνθρώπῳ ἑκατὸν πρόβατα καὶ πλανηθῇ ἓν ἐξ αὐτῶν, οὐχὶ ἀφήσει τὰ ἐνενήκοντα ἐννέα ἐπὶ τὰ ὄρη καὶ πορευθεὶς ζητεῖ τὸ πλανώμενον;
\VS{13}καὶ ἐὰν γένηται εὑρεῖν αὐτό, ἀμὴν λέγω ὑμῖν ὅτι χαίρει ἐπ᾽ αὐτῷ μᾶλλον ἢ ἐπὶ τοῖς ἐνενήκοντα ἐννέα τοῖς μὴ πεπλανημένοις.
\VS{14}οὕτως οὐκ ἔστιν θέλημα ἔμπροσθεν τοῦ Πατρὸς ὑμῶν τοῦ ἐν οὐρανοῖς ἵνα ἀπόληται ἓν τῶν μικρῶν τούτων.
\par }{\PP \VS{15}Ἐὰν δὲ ἁμαρτήσῃ εἰς σὲ ὁ ἀδελφός σου, ὕπαγε ἔλεγξον αὐτὸν μεταξὺ σοῦ καὶ αὐτοῦ μόνου. ἐάν σου ἀκούσῃ, ἐκέρδησας τὸν ἀδελφόν σου·
\VS{16}ἐὰν δὲ μὴ ἀκούσῃ, παράλαβε μετὰ σοῦ ἔτι ἕνα ἢ δύο, ἵνα Ἐπὶ στόματος δύο μαρτύρων ἢ τριῶν σταθῇ πᾶν ῥῆμα·
\VS{17}ἐὰν δὲ παρακούσῃ αὐτῶν, εἰπὲ+ τῇ ἐκκλησίᾳ· ἐὰν δὲ καὶ τῆς ἐκκλησίας παρακούσῃ, ἔστω σοι ὥσπερ ὁ ἐθνικὸς καὶ ὁ τελώνης.
\VS{18}Ἀμὴν λέγω ὑμῖν· ὅσα ἐὰν δήσητε ἐπὶ τῆς γῆς ἔσται δεδεμένα ἐν οὐρανῷ, καὶ ὅσα ἐὰν λύσητε ἐπὶ τῆς γῆς ἔσται λελυμένα ἐν οὐρανῷ.
\par }{\PP \VS{19}Πάλιν ἀμὴν λέγω ὑμῖν ὅτι ἐὰν δύο συμφωνήσωσιν ἐξ ὑμῶν ἐπὶ τῆς γῆς περὶ παντὸς πράγματος οὗ ἐὰν αἰτήσωνται, γενήσεται αὐτοῖς παρὰ τοῦ Πατρός μου τοῦ ἐν οὐρανοῖς.
\VS{20}οὗ γάρ εἰσιν δύο ἢ τρεῖς συνηγμένοι εἰς τὸ ἐμὸν ὄνομα, ἐκεῖ εἰμι ἐν μέσῳ αὐτῶν.
\par }{\PP \VS{21}Τότε προσελθὼν ὁ Πέτρος εἶπεν αὐτῷ· Κύριε, ποσάκις ἁμαρτήσει εἰς ἐμὲ ὁ ἀδελφός μου καὶ ἀφήσω αὐτῷ; ἕως ἑπτάκις;
\VS{22}Λέγει αὐτῷ ὁ Ἰησοῦς· Οὐ λέγω σοι ἕως ἑπτάκις ἀλλὰ= ἕως ἑβδομηκοντάκις ἑπτά.
\par }{\PP \VS{23}Διὰ τοῦτο ὡμοιώθη ἡ βασιλεία τῶν οὐρανῶν ἀνθρώπῳ βασιλεῖ, ὃς ἠθέλησεν συνᾶραι λόγον μετὰ τῶν δούλων αὐτοῦ.
\VS{24}ἀρξαμένου δὲ αὐτοῦ συναίρειν προσηνέχθη αὐτῷ εἷς ὀφειλέτης μυρίων ταλάντων.
\VS{25}μὴ ἔχοντος δὲ αὐτοῦ ἀποδοῦναι ἐκέλευσεν αὐτὸν ὁ κύριος πραθῆναι καὶ τὴν γυναῖκα καὶ τὰ τέκνα καὶ πάντα ὅσα ἔχει, καὶ ἀποδοθῆναι.
\VS{26}Πεσὼν οὖν ὁ δοῦλος προσεκύνει αὐτῷ λέγων· Μακροθύμησον ἐπ᾽ ἐμοί, καὶ πάντα ἀποδώσω σοι.
\VS{27}Σπλαγχνισθεὶς δὲ ὁ κύριος τοῦ δούλου ἐκείνου ἀπέλυσεν αὐτόν καὶ τὸ δάνειον ἀφῆκεν αὐτῷ.
\VS{28}Ἐξελθὼν δὲ ὁ δοῦλος ἐκεῖνος εὗρεν ἕνα τῶν συνδούλων αὐτοῦ, ὃς ὤφειλεν αὐτῷ ἑκατὸν δηνάρια, καὶ κρατήσας αὐτὸν ἔπνιγεν λέγων· Ἀπόδος εἴ τι ὀφείλεις.
\VS{29}Πεσὼν οὖν ὁ σύνδουλος αὐτοῦ παρεκάλει αὐτὸν λέγων· Μακροθύμησον ἐπ᾽ ἐμοί, καὶ ἀποδώσω σοι.
\VS{30}Ὁ δὲ οὐκ ἤθελεν ἀλλὰ= ἀπελθὼν ἔβαλεν αὐτὸν εἰς φυλακὴν ἕως ἀποδῷ τὸ ὀφειλόμενον.
\VS{31}Ἰδόντες οὖν οἱ σύνδουλοι αὐτοῦ τὰ γενόμενα ἐλυπήθησαν σφόδρα καὶ ἐλθόντες διεσάφησαν τῷ κυρίῳ ἑαυτῶν πάντα τὰ γενόμενα.
\VS{32}Τότε προσκαλεσάμενος αὐτὸν ὁ κύριος αὐτοῦ λέγει αὐτῷ· Δοῦλε πονηρέ, πᾶσαν τὴν ὀφειλὴν ἐκείνην ἀφῆκά σοι, ἐπεὶ παρεκάλεσάς με·
\VS{33}οὐκ ἔδει καὶ σὲ ἐλεῆσαι τὸν σύνδουλόν σου, ὡς κἀγὼ σὲ ἠλέησα;
\VS{34}καὶ ὀργισθεὶς ὁ κύριος αὐτοῦ παρέδωκεν αὐτὸν τοῖς βασανισταῖς ἕως οὗ ἀποδῷ πᾶν τὸ ὀφειλόμενον.
\VS{35}Οὕτως καὶ ὁ Πατήρ μου ὁ οὐράνιος ποιήσει ὑμῖν, ἐὰν μὴ ἀφῆτε ἕκαστος τῷ ἀδελφῷ αὐτοῦ ἀπὸ τῶν καρδιῶν ὑμῶν.

\par }\Chap{19}{\PP \VerseOne{1}Καὶ ἐγένετο ὅτε ἐτέλεσεν ὁ Ἰησοῦς τοὺς λόγους τούτους, μετῆρεν ἀπὸ τῆς Γαλιλαίας καὶ ἦλθεν εἰς τὰ ὅρια τῆς Ἰουδαίας πέραν τοῦ Ἰορδάνου.
\VS{2}καὶ ἠκολούθησαν αὐτῷ ὄχλοι πολλοί, καὶ ἐθεράπευσεν αὐτοὺς ἐκεῖ.
\par }{\PP \VS{3}Καὶ προσῆλθον αὐτῷ Φαρισαῖοι πειράζοντες αὐτὸν καὶ λέγοντες· Εἰ ἔξεστιν ἀνθρώπῳ ἀπολῦσαι τὴν γυναῖκα αὐτοῦ κατὰ πᾶσαν αἰτίαν;
\VS{4}Ὁ δὲ ἀποκριθεὶς εἶπεν· Οὐκ ἀνέγνωτε ὅτι ὁ κτίσας ἀπ᾽ ἀρχῆς Ἄρσεν καὶ θῆλυ ἐποίησεν αὐτοὺς;
\VS{5}καὶ εἶπεν· Ἕνεκα τούτου καταλείψει ἄνθρωπος τὸν πατέρα καὶ τὴν μητέρα καὶ κολληθήσεται τῇ γυναικὶ αὐτοῦ, καὶ ἔσονται οἱ δύο εἰς σάρκα μίαν.
\VS{6}ὥστε οὐκέτι εἰσὶν δύο ἀλλὰ σὰρξ μία. ὃ οὖν ὁ Θεὸς συνέζευξεν ἄνθρωπος μὴ χωριζέτω.
\VS{7}Λέγουσιν αὐτῷ· Τί οὖν Μωϋσῆς ἐνετείλατο δοῦναι βιβλίον ἀποστασίου καὶ ἀπολῦσαι αὐτήν;
\VS{8}Λέγει αὐτοῖς Ὅτι Μωϋσῆς πρὸς τὴν σκληροκαρδίαν ὑμῶν ἐπέτρεψεν ὑμῖν ἀπολῦσαι τὰς γυναῖκας ὑμῶν, ἀπ᾽ ἀρχῆς δὲ οὐ γέγονεν οὕτως.
\VS{9}λέγω δὲ ὑμῖν ὅτι ὃς ἂν ἀπολύσῃ τὴν γυναῖκα αὐτοῦ μὴ ἐπὶ πορνείᾳ καὶ γαμήσῃ ἄλλην μοιχᾶται.
\VS{10}Λέγουσιν αὐτῷ οἱ μαθηταί αὐτοῦ· Εἰ οὕτως ἐστὶν ἡ αἰτία τοῦ ἀνθρώπου μετὰ τῆς γυναικός, οὐ συμφέρει γαμῆσαι.
\VS{11}Ὁ δὲ εἶπεν αὐτοῖς· Οὐ πάντες χωροῦσιν τὸν λόγον τοῦτον ἀλλ᾽ οἷς δέδοται.
\VS{12}εἰσὶν γὰρ εὐνοῦχοι οἵτινες ἐκ κοιλίας μητρὸς ἐγεννήθησαν οὕτως, καὶ εἰσὶν εὐνοῦχοι οἵτινες εὐνουχίσθησαν ὑπὸ τῶν ἀνθρώπων, καὶ εἰσὶν εὐνοῦχοι οἵτινες εὐνούχισαν ἑαυτοὺς διὰ τὴν βασιλείαν τῶν οὐρανῶν. ὁ δυνάμενος χωρεῖν χωρείτω.
\par }{\PP \VS{13}Τότε προσηνέχθησαν αὐτῷ παιδία ἵνα τὰς χεῖρας ἐπιθῇ αὐτοῖς καὶ προσεύξηται· οἱ δὲ μαθηταὶ ἐπετίμησαν αὐτοῖς.
\VS{14}ὁ δὲ Ἰησοῦς εἶπεν· Ἄφετε τὰ παιδία καὶ μὴ κωλύετε αὐτὰ ἐλθεῖν πρός με, τῶν γὰρ τοιούτων ἐστὶν ἡ βασιλεία τῶν οὐρανῶν.
\VS{15}καὶ ἐπιθεὶς τὰς χεῖρας αὐτοῖς ἐπορεύθη ἐκεῖθεν.
\par }{\PP \VS{16}Καὶ ἰδοὺ εἷς προσελθὼν αὐτῷ εἶπεν· Διδάσκαλε, τί ἀγαθὸν ποιήσω ἵνα σχῶ ζωὴν αἰώνιον;
\VS{17}Ὁ δὲ εἶπεν αὐτῷ· Τί με ἐρωτᾷς περὶ τοῦ ἀγαθοῦ; εἷς ἐστιν ὁ ἀγαθός· εἰ δὲ θέλεις εἰς τὴν ζωὴν εἰσελθεῖν, τήρησον τὰς ἐντολάς.
\VS{18}Λέγει αὐτῷ· Ποίας; Ὁ δὲ Ἰησοῦς εἶπεν· Τὸ Οὐ φονεύσεις, Οὐ μοιχεύσεις, Οὐ κλέψεις, Οὐ ψευδομαρτυρήσεις,
\VS{19}Τίμα τὸν πατέρα καὶ τὴν μητέρα, καὶ Ἀγαπήσεις τὸν πλησίον σου ὡς σεαυτόν.
\VS{20}Λέγει αὐτῷ ὁ νεανίσκος· Πάντα ταῦτα ἐφύλαξα· τί ἔτι ὑστερῶ;
\VS{21}Ἔφη αὐτῷ ὁ Ἰησοῦς· Εἰ θέλεις τέλειος εἶναι, ὕπαγε πώλησόν σου τὰ ὑπάρχοντα καὶ δὸς τοῖς πτωχοῖς, καὶ ἕξεις θησαυρὸν ἐν οὐρανοῖς, καὶ δεῦρο ἀκολούθει μοι.
\VS{22}Ἀκούσας δὲ ὁ νεανίσκος τὸν λόγον ἀπῆλθεν λυπούμενος· ἦν γὰρ ἔχων κτήματα πολλά.
\par }{\PP \VS{23}Ὁ δὲ Ἰησοῦς εἶπεν τοῖς μαθηταῖς αὐτοῦ· Ἀμὴν λέγω ὑμῖν ὅτι πλούσιος δυσκόλως εἰσελεύσεται εἰς τὴν βασιλείαν τῶν οὐρανῶν.
\VS{24}πάλιν δὲ λέγω ὑμῖν, εὐκοπώτερόν ἐστιν κάμηλον διὰ τρυπήματος ῥαφίδος διελθεῖν+ ἢ πλούσιον εἰσελθεῖν εἰς τὴν βασιλείαν τοῦ Θεοῦ.
\VS{25}Ἀκούσαντες δὲ οἱ μαθηταὶ ἐξεπλήσσοντο σφόδρα λέγοντες· Τίς ἄρα δύναται σωθῆναι;
\VS{26}Ἐμβλέψας δὲ ὁ Ἰησοῦς εἶπεν αὐτοῖς· Παρὰ ἀνθρώποις τοῦτο ἀδύνατόν ἐστιν, παρὰ δὲ Θεῷ πάντα δυνατά.
\par }{\PP \VS{27}Τότε ἀποκριθεὶς ὁ Πέτρος εἶπεν αὐτῷ· Ἰδοὺ ἡμεῖς ἀφήκαμεν πάντα καὶ ἠκολουθήσαμέν σοι· τί ἄρα ἔσται ἡμῖν;
\VS{28}Ὁ δὲ Ἰησοῦς εἶπεν αὐτοῖς· Ἀμὴν λέγω ὑμῖν ὅτι ὑμεῖς οἱ ἀκολουθήσαντές μοι ἐν τῇ παλινγενεσίᾳ,= ὅταν καθίσῃ ὁ Υἱὸς τοῦ ἀνθρώπου ἐπὶ θρόνου δόξης αὐτοῦ, καθήσεσθε καὶ ὑμεῖς ἐπὶ δώδεκα θρόνους κρίνοντες τὰς δώδεκα φυλὰς τοῦ Ἰσραήλ.
\VS{29}καὶ πᾶς ὅστις ἀφῆκεν οἰκίας ἢ ἀδελφοὺς ἢ ἀδελφὰς ἢ πατέρα ἢ μητέρα ἢ τέκνα ἢ ἀγροὺς ἕνεκεν τοῦ ὀνόματός μου, ἑκατονταπλασίονα λήμψεται καὶ ζωὴν αἰώνιον κληρονομήσει.
\VS{30}Πολλοὶ δὲ ἔσονται πρῶτοι ἔσχατοι καὶ ἔσχατοι πρῶτοι.

\par }\Chap{20}{\PP \VerseOne{1}Ὁμοία γάρ ἐστιν ἡ βασιλεία τῶν οὐρανῶν ἀνθρώπῳ οἰκοδεσπότῃ, ὅστις ἐξῆλθεν ἅμα πρωῒ μισθώσασθαι ἐργάτας εἰς τὸν ἀμπελῶνα αὐτοῦ.
\VS{2}συμφωνήσας δὲ μετὰ τῶν ἐργατῶν ἐκ δηναρίου τὴν ἡμέραν ἀπέστειλεν αὐτοὺς εἰς τὸν ἀμπελῶνα αὐτοῦ.
\VS{3}Καὶ ἐξελθὼν περὶ τρίτην ὥραν εἶδεν ἄλλους ἑστῶτας ἐν τῇ ἀγορᾷ ἀργούς
\VS{4}καὶ ἐκείνοις εἶπεν· Ὑπάγετε καὶ ὑμεῖς εἰς τὸν ἀμπελῶνα, καὶ ὃ ἐὰν ᾖ δίκαιον δώσω ὑμῖν.
\VS{5}οἱ δὲ ἀπῆλθον. Πάλιν δὲ ἐξελθὼν περὶ ἕκτην καὶ ἐνάτην ὥραν ἐποίησεν ὡσαύτως.
\VS{6}Περὶ δὲ τὴν ἑνδεκάτην ἐξελθὼν εὗρεν ἄλλους ἑστῶτας καὶ λέγει αὐτοῖς· Τί ὧδε ἑστήκατε ὅλην τὴν ἡμέραν ἀργοί;
\VS{7}Λέγουσιν αὐτῷ· Ὅτι οὐδεὶς ἡμᾶς ἐμισθώσατο. Λέγει αὐτοῖς· Ὑπάγετε καὶ ὑμεῖς εἰς τὸν ἀμπελῶνα.
\VS{8}Ὀψίας δὲ γενομένης λέγει ὁ κύριος τοῦ ἀμπελῶνος τῷ ἐπιτρόπῳ αὐτοῦ· Κάλεσον τοὺς ἐργάτας καὶ ἀπόδος αὐτοῖς τὸν μισθόν ἀρξάμενος ἀπὸ τῶν ἐσχάτων ἕως τῶν πρώτων.
\VS{9}Καὶ ἐλθόντες οἱ περὶ τὴν ἑνδεκάτην ὥραν ἔλαβον ἀνὰ δηνάριον.
\VS{10}καὶ ἐλθόντες οἱ πρῶτοι ἐνόμισαν ὅτι πλεῖον λήμψονται· καὶ ἔλαβον τὸ ἀνὰ δηνάριον καὶ αὐτοί.
\VS{11}Λαβόντες δὲ ἐγόγγυζον κατὰ τοῦ οἰκοδεσπότου
\VS{12}λέγοντες· Οὗτοι οἱ ἔσχατοι μίαν ὥραν ἐποίησαν, καὶ ἴσους ἡμῖν αὐτοὺς ἐποίησας τοῖς βαστάσασι= τὸ βάρος τῆς ἡμέρας καὶ τὸν καύσωνα.
\VS{13}Ὁ δὲ ἀποκριθεὶς ἑνὶ αὐτῶν εἶπεν· Ἑταῖρε, οὐκ ἀδικῶ σε· οὐχὶ δηναρίου συνεφώνησάς μοι;
\VS{14}ἆρον τὸ σὸν καὶ ὕπαγε. θέλω δὲ τούτῳ τῷ ἐσχάτῳ δοῦναι ὡς καὶ σοί·
\VS{15}ἢ οὐκ ἔξεστίν μοι ὃ θέλω ποιῆσαι ἐν τοῖς ἐμοῖς; ἢ ὁ ὀφθαλμός σου πονηρός ἐστιν ὅτι ἐγὼ ἀγαθός εἰμι;
\VS{16}Οὕτως ἔσονται οἱ ἔσχατοι πρῶτοι καὶ οἱ πρῶτοι ἔσχατοι.
\par }{\PP \VS{17}Καὶ ἀναβαίνων ὁ Ἰησοῦς εἰς Ἱεροσόλυμα παρέλαβεν τοὺς δώδεκα μαθητὰς κατ᾽ ἰδίαν καὶ ἐν τῇ ὁδῷ εἶπεν αὐτοῖς·
\VS{18}Ἰδοὺ ἀναβαίνομεν εἰς Ἱεροσόλυμα, καὶ ὁ Υἱὸς τοῦ ἀνθρώπου παραδοθήσεται τοῖς ἀρχιερεῦσιν καὶ γραμματεῦσιν, καὶ κατακρινοῦσιν αὐτὸν θανάτῳ
\VS{19}καὶ παραδώσουσιν αὐτὸν τοῖς ἔθνεσιν εἰς τὸ ἐμπαῖξαι καὶ μαστιγῶσαι καὶ σταυρῶσαι, καὶ τῇ τρίτῃ ἡμέρᾳ ἐγερθήσεται.
\par }{\PP \VS{20}Τότε προσῆλθεν αὐτῷ ἡ μήτηρ τῶν υἱῶν Ζεβεδαίου μετὰ τῶν υἱῶν αὐτῆς προσκυνοῦσα καὶ αἰτοῦσά τι ἀπ᾽ αὐτοῦ.
\VS{21}Ὁ δὲ εἶπεν αὐτῇ· Τί θέλεις; Λέγει αὐτῷ· Εἰπὲ ἵνα καθίσωσιν οὗτοι οἱ δύο υἱοί μου εἷς ἐκ δεξιῶν σου καὶ εἷς ἐξ εὐωνύμων σου ἐν τῇ βασιλείᾳ σου.
\VS{22}Ἀποκριθεὶς δὲ ὁ Ἰησοῦς εἶπεν· Οὐκ οἴδατε τί αἰτεῖσθε. δύνασθε πιεῖν τὸ ποτήριον ὃ ἐγὼ μέλλω πίνειν; Λέγουσιν αὐτῷ· Δυνάμεθα.
\VS{23}Λέγει αὐτοῖς· Τὸ μὲν ποτήριόν μου πίεσθε, τὸ δὲ καθίσαι ἐκ δεξιῶν μου καὶ ἐξ εὐωνύμων οὐκ ἔστιν ἐμὸν τοῦτο δοῦναι, ἀλλ᾽ οἷς ἡτοίμασται ὑπὸ τοῦ Πατρός μου.
\par }{\PP \VS{24}Καὶ ἀκούσαντες οἱ δέκα ἠγανάκτησαν περὶ τῶν δύο ἀδελφῶν.
\VS{25}ὁ δὲ Ἰησοῦς προσκαλεσάμενος αὐτοὺς εἶπεν· Οἴδατε ὅτι οἱ ἄρχοντες τῶν ἐθνῶν κατακυριεύουσιν αὐτῶν καὶ οἱ μεγάλοι κατεξουσιάζουσιν αὐτῶν.
\VS{26}οὐχ οὕτως ἔσται ἐν ὑμῖν, ἀλλ᾽ ὃς ἐὰν θέλῃ ἐν ὑμῖν μέγας γενέσθαι ἔσται ὑμῶν διάκονος,
\VS{27}καὶ ὃς ἂν θέλῃ ἐν ὑμῖν εἶναι πρῶτος ἔσται ὑμῶν δοῦλος·
\VS{28}ὥσπερ ὁ Υἱὸς τοῦ ἀνθρώπου οὐκ ἦλθεν διακονηθῆναι ἀλλὰ διακονῆσαι καὶ δοῦναι τὴν ψυχὴν αὐτοῦ λύτρον ἀντὶ πολλῶν.
\par }{\PP \VS{29}Καὶ ἐκπορευομένων αὐτῶν ἀπὸ Ἰεριχὼ ἠκολούθησεν αὐτῷ ὄχλος πολύς.
\VS{30}καὶ ἰδοὺ δύο τυφλοὶ καθήμενοι παρὰ τὴν ὁδόν ἀκούσαντες ὅτι Ἰησοῦς παράγει, ἔκραξαν λέγοντες· Ἐλέησον ἡμᾶς, Κύριε, υἱὸς Δαυίδ.
\VS{31}Ὁ δὲ ὄχλος ἐπετίμησεν αὐτοῖς ἵνα σιωπήσωσιν· οἱ δὲ μεῖζον ἔκραξαν λέγοντες· Ἐλέησον ἡμᾶς, Κύριε, υἱὸς Δαυίδ.
\VS{32}Καὶ στὰς ὁ Ἰησοῦς ἐφώνησεν αὐτοὺς καὶ εἶπεν· Τί θέλετε ποιήσω ὑμῖν;
\VS{33}Λέγουσιν αὐτῷ· Κύριε, ἵνα ἀνοιγῶσιν οἱ ὀφθαλμοὶ ἡμῶν.
\VS{34}Σπλαγχνισθεὶς δὲ ὁ Ἰησοῦς ἥψατο τῶν ὀμμάτων αὐτῶν, καὶ εὐθέως ἀνέβλεψαν καὶ ἠκολούθησαν αὐτῷ.

\par }\Chap{21}{\PP \VerseOne{1}Καὶ ὅτε ἤγγισαν εἰς Ἱεροσόλυμα καὶ ἦλθον εἰς Βηθφαγὴ εἰς τὸ ὄρος τῶν Ἐλαιῶν, τότε Ἰησοῦς ἀπέστειλεν δύο μαθητὰς
\VS{2}λέγων αὐτοῖς· Πορεύεσθε εἰς τὴν κώμην τὴν κατέναντι ὑμῶν, καὶ εὐθέως εὑρήσετε ὄνον δεδεμένην καὶ πῶλον μετ᾽ αὐτῆς· λύσαντες ἀγάγετέ μοι.
\VS{3}καὶ ἐάν τις ὑμῖν εἴπῃ τι, ἐρεῖτε ὅτι Ὁ Κύριος αὐτῶν χρείαν ἔχει· εὐθὺς δὲ ἀποστελεῖ αὐτούς.
\VS{4}Τοῦτο δὲ γέγονεν ἵνα πληρωθῇ τὸ ῥηθὲν διὰ τοῦ προφήτου λέγοντος·
\VS{5}¬Εἴπατε τῇ θυγατρὶ Σιών· ¬Ἰδοὺ ὁ Βασιλεύς σου ἔρχεταί σοι ¬πραῢς καὶ ἐπιβεβηκὼς ἐπὶ ὄνον ¬καὶ ἐπὶ πῶλον υἱὸν ὑποζυγίου.
\par }{\PP \VS{6}Πορευθέντες δὲ οἱ μαθηταὶ καὶ ποιήσαντες καθὼς συνέταξεν αὐτοῖς ὁ Ἰησοῦς
\VS{7}ἤγαγον τὴν ὄνον καὶ τὸν πῶλον καὶ ἐπέθηκαν ἐπ᾽ αὐτῶν τὰ ἱμάτια, καὶ ἐπεκάθισεν ἐπάνω αὐτῶν.
\VS{8}Ὁ δὲ πλεῖστος ὄχλος ἔστρωσαν ἑαυτῶν τὰ ἱμάτια ἐν τῇ ὁδῷ, ἄλλοι δὲ ἔκοπτον κλάδους ἀπὸ τῶν δένδρων καὶ ἐστρώννυον ἐν τῇ ὁδῷ.
\VS{9}Οἱ δὲ ὄχλοι οἱ προάγοντες αὐτὸν καὶ οἱ ἀκολουθοῦντες ἔκραζον λέγοντες· ¬Ὡσαννὰ τῷ υἱῷ Δαυίδ· ¬Εὐλογημένος ὁ ἐρχόμενος ἐν ὀνόματι Κυρίου· ¬Ὡσαννὰ ἐν τοῖς ὑψίστοις.
\par }{\PP \VS{10}Καὶ εἰσελθόντος αὐτοῦ εἰς Ἱεροσόλυμα ἐσείσθη πᾶσα ἡ πόλις λέγουσα· Τίς ἐστιν οὗτος;
\VS{11}Οἱ δὲ ὄχλοι ἔλεγον· Οὗτός ἐστιν ὁ προφήτης Ἰησοῦς ὁ ἀπὸ Ναζαρὲθ τῆς Γαλιλαίας.
\par }{\PP \VS{12}Καὶ εἰσῆλθεν Ἰησοῦς εἰς τὸ ἱερόν καὶ ἐξέβαλεν πάντας τοὺς πωλοῦντας καὶ ἀγοράζοντας ἐν τῷ ἱερῷ, καὶ τὰς τραπέζας τῶν κολλυβιστῶν κατέστρεψεν καὶ τὰς καθέδρας τῶν πωλούντων τὰς περιστεράς,
\VS{13}καὶ λέγει αὐτοῖς· Γέγραπται· ¬Ὁ οἶκός μου οἶκος προσευχῆς κληθήσεται, ¬ὑμεῖς δὲ αὐτὸν ποιεῖτε Σπήλαιον λῃστῶν.
\par }{\PP \VS{14}Καὶ προσῆλθον αὐτῷ τυφλοὶ καὶ χωλοὶ ἐν τῷ ἱερῷ, καὶ ἐθεράπευσεν αὐτούς.
\VS{15}ἰδόντες δὲ οἱ ἀρχιερεῖς καὶ οἱ γραμματεῖς τὰ θαυμάσια ἃ ἐποίησεν καὶ τοὺς παῖδας τοὺς κράζοντας ἐν τῷ ἱερῷ καὶ λέγοντας· Ὡσαννὰ τῷ υἱῷ Δαυίδ, ἠγανάκτησαν
\VS{16}καὶ εἶπαν αὐτῷ· Ἀκούεις τί οὗτοι λέγουσιν; Ὁ δὲ Ἰησοῦς λέγει αὐτοῖς· Ναί. οὐδέποτε ἀνέγνωτε ὅτι ¬Ἐκ στόματος νηπίων καὶ θηλαζόντων Κατηρτίσω αἶνον;
\par }{\PP \VS{17}Καὶ καταλιπὼν αὐτοὺς ἐξῆλθεν ἔξω τῆς πόλεως εἰς Βηθανίαν καὶ ηὐλίσθη ἐκεῖ.
\par }{\PP \VS{18}Πρωῒ δὲ ἐπανάγων εἰς τὴν πόλιν ἐπείνασεν.
\VS{19}καὶ ἰδὼν συκῆν μίαν ἐπὶ τῆς ὁδοῦ ἦλθεν ἐπ᾽ αὐτήν καὶ οὐδὲν εὗρεν ἐν αὐτῇ εἰ μὴ φύλλα μόνον, καὶ λέγει αὐτῇ· μηκέτι ἐκ σοῦ καρπὸς γένηται εἰς τὸν αἰῶνα. καὶ ἐξηράνθη παραχρῆμα ἡ συκῆ.
\par }{\PP \VS{20}Καὶ ἰδόντες οἱ μαθηταὶ ἐθαύμασαν λέγοντες· Πῶς παραχρῆμα ἐξηράνθη ἡ συκῆ;
\VS{21}Ἀποκριθεὶς δὲ ὁ Ἰησοῦς εἶπεν αὐτοῖς· Ἀμὴν λέγω ὑμῖν, ἐὰν ἔχητε πίστιν καὶ μὴ διακριθῆτε, οὐ μόνον τὸ τῆς συκῆς ποιήσετε, ἀλλὰ κἂν τῷ ὄρει τούτῳ εἴπητε· Ἄρθητι καὶ βλήθητι εἰς τὴν θάλασσαν, γενήσεται·
\VS{22}καὶ πάντα ὅσα ἂν αἰτήσητε ἐν τῇ προσευχῇ πιστεύοντες λήμψεσθε.
\par }{\PP \VS{23}Καὶ ἐλθόντος αὐτοῦ εἰς τὸ ἱερὸν προσῆλθον αὐτῷ διδάσκοντι οἱ ἀρχιερεῖς καὶ οἱ πρεσβύτεροι τοῦ λαοῦ λέγοντες· Ἐν ποίᾳ ἐξουσίᾳ ταῦτα ποιεῖς; καὶ τίς σοι ἔδωκεν τὴν ἐξουσίαν ταύτην;
\VS{24}Ἀποκριθεὶς δὲ ὁ Ἰησοῦς εἶπεν αὐτοῖς· Ἐρωτήσω ὑμᾶς κἀγὼ λόγον ἕνα, ὃν ἐὰν εἴπητέ μοι κἀγὼ ὑμῖν ἐρῶ ἐν ποίᾳ ἐξουσίᾳ ταῦτα ποιῶ·
\VS{25}τὸ βάπτισμα τὸ Ἰωάννου πόθεν ἦν; ἐξ οὐρανοῦ ἢ ἐξ ἀνθρώπων; Οἱ δὲ διελογίζοντο ἐν ἑαυτοῖς λέγοντες· Ἐὰν εἴπωμεν· Ἐξ οὐρανοῦ, ἐρεῖ ἡμῖν· Διὰ τί οὖν οὐκ ἐπιστεύσατε αὐτῷ;
\VS{26}ἐὰν δὲ εἴπωμεν· Ἐξ ἀνθρώπων, φοβούμεθα τὸν ὄχλον, πάντες γὰρ ὡς προφήτην ἔχουσιν τὸν Ἰωάννην.
\VS{27}καὶ ἀποκριθέντες τῷ Ἰησοῦ εἶπαν· Οὐκ οἴδαμεν. Ἔφη αὐτοῖς καὶ αὐτός· Οὐδὲ ἐγὼ λέγω ὑμῖν ἐν ποίᾳ ἐξουσίᾳ ταῦτα ποιῶ.
\par }{\PP \VS{28}Τί δὲ ὑμῖν δοκεῖ; ἄνθρωπος εἶχεν τέκνα δύο. καὶ προσελθὼν τῷ πρώτῳ εἶπεν· Τέκνον, ὕπαγε σήμερον ἐργάζου ἐν τῷ ἀμπελῶνι.
\VS{29}Ὁ δὲ ἀποκριθεὶς εἶπεν· Οὐ θέλω, ὕστερον δὲ μεταμεληθεὶς ἀπῆλθεν.
\VS{30}Προσελθὼν δὲ τῷ ἑτέρῳ+ εἶπεν ὡσαύτως. Ὁ δὲ ἀποκριθεὶς εἶπεν· Ἐγώ, κύριε, καὶ οὐκ ἀπῆλθεν.
\VS{31}Τίς ἐκ τῶν δύο ἐποίησεν τὸ θέλημα τοῦ πατρός; Λέγουσιν· Ὁ πρῶτος. Λέγει αὐτοῖς ὁ Ἰησοῦς· Ἀμὴν λέγω ὑμῖν ὅτι οἱ τελῶναι καὶ αἱ πόρναι προάγουσιν ὑμᾶς εἰς τὴν βασιλείαν τοῦ Θεοῦ.
\VS{32}ἦλθεν γὰρ Ἰωάννης πρὸς ὑμᾶς ἐν ὁδῷ δικαιοσύνης, καὶ οὐκ ἐπιστεύσατε αὐτῷ, οἱ δὲ τελῶναι καὶ αἱ πόρναι ἐπίστευσαν αὐτῷ· ὑμεῖς δὲ ἰδόντες οὐδὲ μετεμελήθητε ὕστερον τοῦ πιστεῦσαι αὐτῷ.
\par }{\PP \VS{33}Ἄλλην παραβολὴν ἀκούσατε. Ἄνθρωπος ἦν οἰκοδεσπότης ὅστις ἐφύτευσεν ἀμπελῶνα καὶ φραγμὸν αὐτῷ περιέθηκεν καὶ ὤρυξεν ἐν αὐτῷ ληνὸν καὶ ᾠκοδόμησεν πύργον καὶ ἐξέδετο αὐτὸν γεωργοῖς καὶ ἀπεδήμησεν.
\VS{34}Ὅτε δὲ ἤγγισεν ὁ καιρὸς τῶν καρπῶν, ἀπέστειλεν τοὺς δούλους αὐτοῦ πρὸς τοὺς γεωργοὺς λαβεῖν τοὺς καρποὺς αὐτοῦ.
\VS{35}καὶ λαβόντες οἱ γεωργοὶ τοὺς δούλους αὐτοῦ ὃν μὲν ἔδειραν, ὃν δὲ ἀπέκτειναν, ὃν δὲ ἐλιθοβόλησαν.
\VS{36}Πάλιν ἀπέστειλεν ἄλλους δούλους πλείονας τῶν πρώτων, καὶ ἐποίησαν αὐτοῖς ὡσαύτως.
\VS{37}Ὕστερον δὲ ἀπέστειλεν πρὸς αὐτοὺς τὸν υἱὸν αὐτοῦ λέγων· Ἐντραπήσονται τὸν υἱόν μου.
\VS{38}Οἱ δὲ γεωργοὶ ἰδόντες τὸν υἱὸν εἶπον ἐν ἑαυτοῖς· Οὗτός ἐστιν ὁ κληρονόμος· δεῦτε ἀποκτείνωμεν αὐτὸν καὶ σχῶμεν τὴν κληρονομίαν αὐτοῦ,
\VS{39}καὶ λαβόντες αὐτὸν ἐξέβαλον ἔξω τοῦ ἀμπελῶνος καὶ ἀπέκτειναν.
\VS{40}Ὅταν οὖν ἔλθῃ ὁ κύριος τοῦ ἀμπελῶνος, τί ποιήσει τοῖς γεωργοῖς ἐκείνοις;
\VS{41}Λέγουσιν αὐτῷ· Κακοὺς κακῶς ἀπολέσει αὐτούς καὶ τὸν ἀμπελῶνα ἐκδώσεται ἄλλοις γεωργοῖς, οἵτινες ἀποδώσουσιν αὐτῷ τοὺς καρποὺς ἐν τοῖς καιροῖς αὐτῶν.
\par }{\PP \VS{42}Λέγει αὐτοῖς ὁ Ἰησοῦς· Οὐδέποτε ἀνέγνωτε ἐν ταῖς γραφαῖς· ¬Λίθον ὃν ἀπεδοκίμασαν οἱ οἰκοδομοῦντες, ¬Οὗτος ἐγενήθη εἰς κεφαλὴν γωνίας· ¬Παρὰ Κυρίου ἐγένετο αὕτη ¬Καὶ ἔστιν θαυμαστὴ ἐν ὀφθαλμοῖς ἡμῶν;
\par }{\PP \VS{43}Διὰ τοῦτο λέγω ὑμῖν ὅτι ἀρθήσεται ἀφ᾽ ὑμῶν ἡ βασιλεία τοῦ Θεοῦ καὶ δοθήσεται ἔθνει ποιοῦντι τοὺς καρποὺς αὐτῆς.
\VS{44}καὶ ὁ πεσὼν ἐπὶ τὸν λίθον τοῦτον συνθλασθήσεται· ἐφ᾽ ὃν δ᾽ ἂν πέσῃ λικμήσει αὐτόν.
\par }{\PP \VS{45}Καὶ ἀκούσαντες οἱ ἀρχιερεῖς καὶ οἱ Φαρισαῖοι τὰς παραβολὰς αὐτοῦ ἔγνωσαν ὅτι περὶ αὐτῶν λέγει·
\VS{46}καὶ ζητοῦντες αὐτὸν κρατῆσαι ἐφοβήθησαν τοὺς ὄχλους, ἐπεὶ εἰς προφήτην αὐτὸν εἶχον.

\par }\Chap{22}{\PP \VerseOne{1}Καὶ ἀποκριθεὶς ὁ Ἰησοῦς πάλιν εἶπεν ἐν παραβολαῖς αὐτοῖς λέγων·
\VS{2}Ὡμοιώθη ἡ βασιλεία τῶν οὐρανῶν ἀνθρώπῳ βασιλεῖ, ὅστις ἐποίησεν γάμους τῷ υἱῷ αὐτοῦ.
\VS{3}καὶ ἀπέστειλεν τοὺς δούλους αὐτοῦ καλέσαι τοὺς κεκλημένους εἰς τοὺς γάμους, καὶ οὐκ ἤθελον ἐλθεῖν.
\VS{4}Πάλιν ἀπέστειλεν ἄλλους δούλους λέγων· Εἴπατε τοῖς κεκλημένοις· Ἰδοὺ τὸ ἄριστόν μου ἡτοίμακα, οἱ ταῦροί μου καὶ τὰ σιτιστὰ τεθυμένα καὶ πάντα ἕτοιμα· δεῦτε εἰς τοὺς γάμους.
\VS{5}Οἱ δὲ ἀμελήσαντες ἀπῆλθον, ὃς μὲν εἰς τὸν ἴδιον ἀγρόν, ὃς δὲ ἐπὶ τὴν ἐμπορίαν αὐτοῦ·
\VS{6}οἱ δὲ λοιποὶ κρατήσαντες τοὺς δούλους αὐτοῦ ὕβρισαν καὶ ἀπέκτειναν.
\VS{7}Ὁ δὲ βασιλεὺς ὠργίσθη καὶ πέμψας τὰ στρατεύματα αὐτοῦ ἀπώλεσεν τοὺς φονεῖς ἐκείνους καὶ τὴν πόλιν αὐτῶν ἐνέπρησεν.
\VS{8}τότε λέγει τοῖς δούλοις αὐτοῦ· Ὁ μὲν γάμος ἕτοιμός ἐστιν, οἱ δὲ κεκλημένοι οὐκ ἦσαν ἄξιοι·
\VS{9}πορεύεσθε οὖν ἐπὶ τὰς διεξόδους τῶν ὁδῶν καὶ ὅσους ἐὰν εὕρητε καλέσατε εἰς τοὺς γάμους.
\VS{10}Καὶ ἐξελθόντες οἱ δοῦλοι ἐκεῖνοι εἰς τὰς ὁδοὺς συνήγαγον πάντας οὓς εὗρον, πονηρούς τε καὶ ἀγαθούς· καὶ ἐπλήσθη ὁ γάμος ἀνακειμένων.
\VS{11}Εἰσελθὼν δὲ ὁ βασιλεὺς θεάσασθαι τοὺς ἀνακειμένους εἶδεν ἐκεῖ ἄνθρωπον οὐκ ἐνδεδυμένον ἔνδυμα γάμου,
\VS{12}καὶ λέγει αὐτῷ· Ἑταῖρε, πῶς εἰσῆλθες ὧδε μὴ ἔχων ἔνδυμα γάμου; Ὁ δὲ ἐφιμώθη.
\VS{13}Τότε ὁ βασιλεὺς εἶπεν τοῖς διακόνοις· Δήσαντες αὐτοῦ πόδας καὶ χεῖρας ἐκβάλετε αὐτὸν εἰς τὸ σκότος τὸ ἐξώτερον· ἐκεῖ ἔσται ὁ κλαυθμὸς καὶ ὁ βρυγμὸς τῶν ὀδόντων.
\VS{14}Πολλοὶ γάρ εἰσιν κλητοὶ, ὀλίγοι δὲ ἐκλεκτοί.
\par }{\PP \VS{15}Τότε πορευθέντες οἱ Φαρισαῖοι συμβούλιον ἔλαβον ὅπως αὐτὸν παγιδεύσωσιν ἐν λόγῳ.
\VS{16}καὶ ἀποστέλλουσιν αὐτῷ τοὺς μαθητὰς αὐτῶν μετὰ τῶν Ἡρῳδιανῶν λέγοντες· Διδάσκαλε, οἴδαμεν ὅτι ἀληθὴς εἶ καὶ τὴν ὁδὸν τοῦ Θεοῦ ἐν ἀληθείᾳ διδάσκεις καὶ οὐ μέλει σοι περὶ οὐδενός· οὐ γὰρ βλέπεις εἰς πρόσωπον ἀνθρώπων,
\VS{17}εἰπὲ+ οὖν ἡμῖν τί σοι δοκεῖ· ἔξεστιν δοῦναι κῆνσον Καίσαρι ἢ οὔ;
\VS{18}Γνοὺς δὲ ὁ Ἰησοῦς τὴν πονηρίαν αὐτῶν εἶπεν· Τί με πειράζετε, ὑποκριταί;
\VS{19}ἐπιδείξατέ μοι τὸ νόμισμα τοῦ κήνσου. Οἱ δὲ προσήνεγκαν αὐτῷ δηνάριον.
\VS{20}Καὶ λέγει αὐτοῖς· Τίνος ἡ εἰκὼν αὕτη καὶ ἡ ἐπιγραφή;
\VS{21}Λέγουσιν αὐτῷ· Καίσαρος. Τότε λέγει αὐτοῖς· Ἀπόδοτε οὖν τὰ Καίσαρος Καίσαρι καὶ τὰ τοῦ Θεοῦ τῷ Θεῷ.
\VS{22}Καὶ ἀκούσαντες ἐθαύμασαν, καὶ ἀφέντες αὐτὸν ἀπῆλθαν.
\par }{\PP \VS{23}Ἐν ἐκείνῃ τῇ ἡμέρᾳ προσῆλθον αὐτῷ Σαδδουκαῖοι, λέγοντες μὴ εἶναι ἀνάστασιν, καὶ ἐπηρώτησαν αὐτὸν
\VS{24}λέγοντες· Διδάσκαλε, Μωϋσῆς εἶπεν· Ἐάν τις ἀποθάνῃ μὴ ἔχων τέκνα, ἐπιγαμβρεύσει ὁ ἀδελφὸς αὐτοῦ τὴν γυναῖκα αὐτοῦ καὶ ἀναστήσει σπέρμα τῷ ἀδελφῷ αὐτοῦ.
\VS{25}ἦσαν δὲ παρ᾽ ἡμῖν ἑπτὰ ἀδελφοί· καὶ ὁ πρῶτος γήμας ἐτελεύτησεν, καὶ μὴ ἔχων σπέρμα ἀφῆκεν τὴν γυναῖκα αὐτοῦ τῷ ἀδελφῷ αὐτοῦ·
\VS{26}ὁμοίως καὶ ὁ δεύτερος καὶ ὁ τρίτος ἕως τῶν ἑπτά.
\VS{27}ὕστερον δὲ πάντων ἀπέθανεν ἡ γυνή.
\VS{28}ἐν τῇ ἀναστάσει οὖν τίνος τῶν ἑπτὰ ἔσται γυνή; πάντες γὰρ ἔσχον αὐτήν·
\VS{29}Ἀποκριθεὶς δὲ ὁ Ἰησοῦς εἶπεν αὐτοῖς· Πλανᾶσθε μὴ εἰδότες τὰς γραφὰς μηδὲ τὴν δύναμιν τοῦ Θεοῦ·
\VS{30}ἐν γὰρ τῇ ἀναστάσει οὔτε γαμοῦσιν οὔτε γαμίζονται, ἀλλ᾽ ὡς ἄγγελοι ἐν τῷ οὐρανῷ εἰσιν.
\VS{31}περὶ δὲ τῆς ἀναστάσεως τῶν νεκρῶν οὐκ ἀνέγνωτε τὸ ῥηθὲν ὑμῖν ὑπὸ τοῦ Θεοῦ λέγοντος·
\VS{32}Ἐγώ εἰμι ὁ Θεὸς Ἀβραὰμ καὶ ὁ Θεὸς Ἰσαὰκ καὶ ὁ Θεὸς Ἰακώβ; οὐκ ἔστιν ὁ Θεὸς νεκρῶν ἀλλὰ ζώντων.
\VS{33}Καὶ ἀκούσαντες οἱ ὄχλοι ἐξεπλήσσοντο ἐπὶ τῇ διδαχῇ αὐτοῦ.
\par }{\PP \VS{34}Οἱ δὲ Φαρισαῖοι ἀκούσαντες ὅτι ἐφίμωσεν τοὺς Σαδδουκαίους συνήχθησαν ἐπὶ τὸ αὐτό,
\VS{35}καὶ ἐπηρώτησεν εἷς ἐξ αὐτῶν νομικὸς πειράζων αὐτόν·
\VS{36}Διδάσκαλε, ποία ἐντολὴ μεγάλη ἐν τῷ νόμῳ;
\VS{37}Ὁ δὲ ἔφη αὐτῷ· Ἀγαπήσεις κύριον τὸν Θεόν σου ἐν ὅλῃ τῇ καρδίᾳ σου καὶ ἐν ὅλῃ τῇ ψυχῇ σου καὶ ἐν ὅλῃ τῇ διανοίᾳ σου·
\VS{38}αὕτη ἐστὶν ἡ μεγάλη καὶ πρώτη ἐντολή.
\VS{39}δευτέρα δὲ ὁμοία αὐτῇ· Ἀγαπήσεις τὸν πλησίον σου ὡς σεαυτόν.
\VS{40}ἐν ταύταις ταῖς δυσὶν ἐντολαῖς ὅλος ὁ νόμος κρέμαται καὶ οἱ προφῆται.
\par }{\PP \VS{41}Συνηγμένων δὲ τῶν Φαρισαίων ἐπηρώτησεν αὐτοὺς ὁ Ἰησοῦς
\VS{42}λέγων· Τί ὑμῖν δοκεῖ περὶ τοῦ Χριστοῦ; τίνος υἱός ἐστιν; Λέγουσιν αὐτῷ· Τοῦ Δαυίδ.
\VS{43}Λέγει αὐτοῖς· Πῶς οὖν Δαυὶδ ἐν Πνεύματι καλεῖ αὐτὸν Κύριον λέγων·
\VS{44}¬Εἶπεν Κύριος τῷ Κυρίῳ μου· ¬Κάθου ἐκ δεξιῶν μου, ¬Ἕως ἂν θῶ τοὺς ἐχθρούς σου ¬Ὑποκάτω τῶν ποδῶν σου;
\par }{\PP \VS{45}Εἰ οὖν Δαυὶδ καλεῖ αὐτὸν Κύριον, πῶς υἱὸς αὐτοῦ ἐστιν;
\VS{46}Καὶ οὐδεὶς ἐδύνατο ἀποκριθῆναι αὐτῷ λόγον οὐδὲ ἐτόλμησέν τις ἀπ᾽ ἐκείνης τῆς ἡμέρας ἐπερωτῆσαι αὐτὸν οὐκέτι.

\par }\Chap{23}{\PP \VerseOne{1}Τότε ὁ Ἰησοῦς ἐλάλησεν τοῖς ὄχλοις καὶ τοῖς μαθηταῖς αὐτοῦ
\VS{2}λέγων· Ἐπὶ τῆς Μωϋσέως καθέδρας ἐκάθισαν οἱ γραμματεῖς καὶ οἱ Φαρισαῖοι.
\VS{3}πάντα οὖν ὅσα ἐὰν εἴπωσιν ὑμῖν ποιήσατε καὶ τηρεῖτε, κατὰ δὲ τὰ ἔργα αὐτῶν μὴ ποιεῖτε· λέγουσιν γὰρ καὶ οὐ ποιοῦσιν.
\VS{4}δεσμεύουσιν δὲ φορτία βαρέα καὶ δυσβάστακτα καὶ ἐπιτιθέασιν ἐπὶ τοὺς ὤμους τῶν ἀνθρώπων, αὐτοὶ δὲ τῷ δακτύλῳ αὐτῶν οὐ θέλουσιν κινῆσαι αὐτά.
\VS{5}Πάντα δὲ τὰ ἔργα αὐτῶν ποιοῦσιν πρὸς τὸ θεαθῆναι τοῖς ἀνθρώποις· πλατύνουσιν γὰρ τὰ φυλακτήρια αὐτῶν καὶ μεγαλύνουσιν τὰ κράσπεδα,
\VS{6}φιλοῦσιν δὲ τὴν πρωτοκλισίαν ἐν τοῖς δείπνοις καὶ τὰς πρωτοκαθεδρίας ἐν ταῖς συναγωγαῖς
\VS{7}καὶ τοὺς ἀσπασμοὺς ἐν ταῖς ἀγοραῖς καὶ καλεῖσθαι ὑπὸ τῶν ἀνθρώπων Ῥαββί.
\VS{8}Ὑμεῖς δὲ μὴ κληθῆτε Ῥαββί· εἷς γάρ ἐστιν ὑμῶν ὁ διδάσκαλος, πάντες δὲ ὑμεῖς ἀδελφοί ἐστε.
\VS{9}καὶ πατέρα μὴ καλέσητε ὑμῶν ἐπὶ τῆς γῆς, εἷς γάρ ἐστιν ὑμῶν ὁ Πατὴρ ὁ οὐράνιος.
\VS{10}μηδὲ κληθῆτε καθηγηταί, ὅτι καθηγητὴς ὑμῶν ἐστιν εἷς ὁ Χριστός.
\VS{11}ὁ δὲ μείζων ὑμῶν ἔσται ὑμῶν διάκονος.
\VS{12}Ὅστις δὲ ὑψώσει ἑαυτὸν ταπεινωθήσεται καὶ ὅστις ταπεινώσει ἑαυτὸν ὑψωθήσεται.
\par }{\PP \VS{13}Οὐαὶ δὲ ὑμῖν, γραμματεῖς καὶ Φαρισαῖοι ὑποκριταί, ὅτι κλείετε τὴν βασιλείαν τῶν οὐρανῶν ἔμπροσθεν τῶν ἀνθρώπων· ὑμεῖς γὰρ οὐκ εἰσέρχεσθε οὐδὲ τοὺς εἰσερχομένους ἀφίετε εἰσελθεῖν.
\par }{\PP \VS{15}Οὐαὶ ὑμῖν, γραμματεῖς καὶ Φαρισαῖοι ὑποκριταί, ὅτι περιάγετε τὴν θάλασσαν καὶ τὴν ξηρὰν ποιῆσαι ἕνα προσήλυτον, καὶ ὅταν γένηται ποιεῖτε αὐτὸν υἱὸν γεέννης διπλότερον ὑμῶν.
\par }{\PP \VS{16}Οὐαὶ ὑμῖν, ὁδηγοὶ τυφλοὶ οἱ λέγοντες· Ὃς ἂν ὀμόσῃ ἐν τῷ ναῷ, οὐδέν ἐστιν· ὃς δ᾽ ἂν ὀμόσῃ ἐν τῷ χρυσῷ τοῦ ναοῦ, ὀφείλει.
\VS{17}μωροὶ καὶ τυφλοί, τίς γὰρ μείζων ἐστίν, ὁ χρυσὸς ἢ ὁ ναὸς ὁ ἁγιάσας τὸν χρυσόν;
\VS{18}καί· Ὃς ἂν ὀμόσῃ ἐν τῷ θυσιαστηρίῳ, οὐδέν ἐστιν· ὃς δ᾽ ἂν ὀμόσῃ ἐν τῷ δώρῳ τῷ ἐπάνω αὐτοῦ, ὀφείλει.
\VS{19}τυφλοί, τί γὰρ μεῖζον, τὸ δῶρον ἢ τὸ θυσιαστήριον τὸ ἁγιάζον τὸ δῶρον;
\VS{20}ὁ οὖν ὀμόσας ἐν τῷ θυσιαστηρίῳ ὀμνύει ἐν αὐτῷ καὶ ἐν πᾶσι= τοῖς ἐπάνω αὐτοῦ·
\VS{21}καὶ ὁ ὀμόσας ἐν τῷ ναῷ ὀμνύει ἐν αὐτῷ καὶ ἐν τῷ κατοικοῦντι αὐτόν,
\VS{22}καὶ ὁ ὀμόσας ἐν τῷ οὐρανῷ ὀμνύει ἐν τῷ θρόνῳ τοῦ Θεοῦ καὶ ἐν τῷ καθημένῳ ἐπάνω αὐτοῦ.
\par }{\PP \VS{23}Οὐαὶ ὑμῖν, γραμματεῖς καὶ Φαρισαῖοι ὑποκριταί, ὅτι ἀποδεκατοῦτε τὸ ἡδύοσμον καὶ τὸ ἄνηθον καὶ τὸ κύμινον καὶ ἀφήκατε τὰ βαρύτερα τοῦ νόμου, τὴν κρίσιν καὶ τὸ ἔλεος καὶ τὴν πίστιν· ταῦτα δὲ ἔδει ποιῆσαι κἀκεῖνα μὴ ἀφιέναι.
\VS{24}ὁδηγοὶ τυφλοί, οἱ διϋλίζοντες τὸν κώνωπα, τὴν δὲ κάμηλον καταπίνοντες.
\par }{\PP \VS{25}Οὐαὶ ὑμῖν, γραμματεῖς καὶ Φαρισαῖοι ὑποκριταί, ὅτι καθαρίζετε τὸ ἔξωθεν τοῦ ποτηρίου καὶ τῆς παροψίδος, ἔσωθεν δὲ γέμουσιν ἐξ ἁρπαγῆς καὶ ἀκρασίας.
\VS{26}Φαρισαῖε τυφλέ, καθάρισον πρῶτον τὸ ἐντὸς τοῦ ποτηρίου, ἵνα γένηται καὶ τὸ ἐκτὸς αὐτοῦ καθαρόν.
\par }{\PP \VS{27}Οὐαὶ ὑμῖν, γραμματεῖς καὶ Φαρισαῖοι ὑποκριταί, ὅτι παρομοιάζετε τάφοις κεκονιαμένοις, οἵτινες ἔξωθεν μὲν φαίνονται ὡραῖοι, ἔσωθεν δὲ γέμουσιν ὀστέων νεκρῶν καὶ πάσης ἀκαθαρσίας.
\VS{28}οὕτως καὶ ὑμεῖς ἔξωθεν μὲν φαίνεσθε τοῖς ἀνθρώποις δίκαιοι, ἔσωθεν δέ ἐστε μεστοὶ ὑποκρίσεως καὶ ἀνομίας.
\par }{\PP \VS{29}Οὐαὶ ὑμῖν, γραμματεῖς καὶ Φαρισαῖοι ὑποκριταί, ὅτι οἰκοδομεῖτε τοὺς τάφους τῶν προφητῶν καὶ κοσμεῖτε τὰ μνημεῖα τῶν δικαίων,
\VS{30}καὶ λέγετε· Εἰ ἤμεθα ἐν ταῖς ἡμέραις τῶν πατέρων ἡμῶν, οὐκ ἂν ἤμεθα αὐτῶν κοινωνοὶ ἐν τῷ αἵματι τῶν προφητῶν.
\VS{31}ὥστε μαρτυρεῖτε ἑαυτοῖς ὅτι υἱοί ἐστε τῶν φονευσάντων τοὺς προφήτας.
\VS{32}καὶ ὑμεῖς πληρώσατε τὸ μέτρον τῶν πατέρων ὑμῶν.
\VS{33}ὄφεις, γεννήματα ἐχιδνῶν, πῶς φύγητε ἀπὸ τῆς κρίσεως τῆς γεέννης;
\par }{\PP \VS{34}Διὰ τοῦτο ἰδοὺ ἐγὼ ἀποστέλλω πρὸς ὑμᾶς προφήτας καὶ σοφοὺς καὶ γραμματεῖς· ἐξ αὐτῶν ἀποκτενεῖτε καὶ σταυρώσετε καὶ ἐξ αὐτῶν μαστιγώσετε ἐν ταῖς συναγωγαῖς ὑμῶν καὶ διώξετε ἀπὸ πόλεως εἰς πόλιν·
\VS{35}ὅπως ἔλθῃ ἐφ᾽ ὑμᾶς πᾶν αἷμα δίκαιον ἐκχυννόμενον ἐπὶ τῆς γῆς ἀπὸ τοῦ αἵματος Ἅβελ τοῦ δικαίου ἕως τοῦ αἵματος Ζαχαρίου υἱοῦ Βαραχίου, ὃν ἐφονεύσατε μεταξὺ τοῦ ναοῦ καὶ τοῦ θυσιαστηρίου.
\VS{36}ἀμὴν λέγω ὑμῖν, ἥξει ταῦτα πάντα ἐπὶ τὴν γενεὰν ταύτην.
\par }{\PP \VS{37}Ἰερουσαλὴμ Ἰερουσαλήμ, ἡ ἀποκτείνουσα τοὺς προφήτας καὶ λιθοβολοῦσα τοὺς ἀπεσταλμένους πρὸς αὐτήν, ποσάκις ἠθέλησα ἐπισυναγαγεῖν τὰ τέκνα σου, ὃν τρόπον ὄρνις ἐπισυνάγει τὰ νοσσία αὐτῆς ὑπὸ τὰς πτέρυγας, καὶ οὐκ ἠθελήσατε.
\VS{38}ἰδοὺ ἀφίεται ὑμῖν ὁ οἶκος ὑμῶν ἔρημος.
\VS{39}λέγω γὰρ ὑμῖν, οὐ μή με ἴδητε ἀπ᾽ ἄρτι ἕως ἂν εἴπητε· ¬Εὐλογημένος ὁ ἐρχόμενος ἐν ὀνόματι Κυρίου.

\par }\Chap{24}{\PP \VerseOne{1}Καὶ ἐξελθὼν ὁ Ἰησοῦς ἀπὸ τοῦ ἱεροῦ ἐπορεύετο, καὶ προσῆλθον οἱ μαθηταὶ αὐτοῦ ἐπιδεῖξαι αὐτῷ τὰς οἰκοδομὰς τοῦ ἱεροῦ.
\VS{2}ὁ δὲ ἀποκριθεὶς εἶπεν αὐτοῖς· Οὐ βλέπετε ταῦτα πάντα; ἀμὴν λέγω ὑμῖν, οὐ μὴ ἀφεθῇ ὧδε λίθος ἐπὶ λίθον ὃς οὐ καταλυθήσεται.
\par }{\PP \VS{3}Καθημένου δὲ αὐτοῦ ἐπὶ τοῦ ὄρους τῶν Ἐλαιῶν προσῆλθον αὐτῷ οἱ μαθηταὶ κατ᾽ ἰδίαν λέγοντες· Εἰπὲ ἡμῖν, πότε ταῦτα ἔσται καὶ τί τὸ σημεῖον τῆς σῆς παρουσίας καὶ συντελείας τοῦ αἰῶνος;
\par }{\PP \VS{4}Καὶ ἀποκριθεὶς ὁ Ἰησοῦς εἶπεν αὐτοῖς· Βλέπετε μή τις ὑμᾶς πλανήσῃ·
\VS{5}πολλοὶ γὰρ ἐλεύσονται ἐπὶ τῷ ὀνόματί μου λέγοντες· Ἐγώ εἰμι ὁ Χριστός, καὶ πολλοὺς πλανήσουσιν.
\VS{6}μελλήσετε δὲ ἀκούειν πολέμους καὶ ἀκοὰς πολέμων· ὁρᾶτε μὴ θροεῖσθε· δεῖ γὰρ γενέσθαι, ἀλλ᾽ οὔπω ἐστὶν τὸ τέλος.
\VS{7}ἐγερθήσεται γὰρ ἔθνος ἐπὶ ἔθνος καὶ βασιλεία ἐπὶ βασιλείαν καὶ ἔσονται λιμοὶ καὶ σεισμοὶ κατὰ τόπους·
\VS{8}πάντα δὲ ταῦτα ἀρχὴ ὠδίνων.
\par }{\PP \VS{9}Τότε παραδώσουσιν ὑμᾶς εἰς θλῖψιν καὶ ἀποκτενοῦσιν ὑμᾶς, καὶ ἔσεσθε μισούμενοι ὑπὸ πάντων τῶν ἐθνῶν διὰ τὸ ὄνομά μου.
\VS{10}καὶ τότε σκανδαλισθήσονται πολλοὶ καὶ ἀλλήλους παραδώσουσιν καὶ μισήσουσιν ἀλλήλους·
\VS{11}καὶ πολλοὶ ψευδοπροφῆται ἐγερθήσονται καὶ πλανήσουσιν πολλούς·
\VS{12}Καὶ διὰ τὸ πληθυνθῆναι τὴν ἀνομίαν ψυγήσεται ἡ ἀγάπη τῶν πολλῶν.
\VS{13}ὁ δὲ ὑπομείνας εἰς τέλος οὗτος σωθήσεται.
\VS{14}Καὶ κηρυχθήσεται τοῦτο τὸ εὐαγγέλιον τῆς βασιλείας ἐν ὅλῃ τῇ οἰκουμένῃ εἰς μαρτύριον πᾶσιν τοῖς ἔθνεσιν, καὶ τότε ἥξει τὸ τέλος.
\par }{\PP \VS{15}Ὅταν οὖν ἴδητε Τὸ βδέλυγμα τῆς ἐρημώσεως τὸ ῥηθὲν διὰ Δανιὴλ τοῦ προφήτου ἑστὸς ἐν τόπῳ ἁγίῳ, ὁ ἀναγινώσκων νοείτω,
\VS{16}τότε οἱ ἐν τῇ Ἰουδαίᾳ φευγέτωσαν εἰς τὰ ὄρη,
\VS{17}ὁ ἐπὶ τοῦ δώματος μὴ καταβάτω ἆραι τὰ ἐκ τῆς οἰκίας αὐτοῦ,
\VS{18}καὶ ὁ ἐν τῷ ἀγρῷ μὴ ἐπιστρεψάτω ὀπίσω ἆραι τὸ ἱμάτιον αὐτοῦ.
\VS{19}Οὐαὶ δὲ ταῖς ἐν γαστρὶ ἐχούσαις καὶ ταῖς θηλαζούσαις ἐν ἐκείναις ταῖς ἡμέραις.
\par }{\PP \VS{20}προσεύχεσθε δὲ ἵνα μὴ γένηται ἡ φυγὴ ὑμῶν χειμῶνος μηδὲ σαββάτῳ.
\VS{21}ἔσται γὰρ τότε θλῖψις μεγάλη οἵα οὐ γέγονεν ἀπ᾽ ἀρχῆς κόσμου ἕως τοῦ νῦν οὐδ᾽ οὐ μὴ γένηται.
\VS{22}καὶ εἰ μὴ ἐκολοβώθησαν αἱ ἡμέραι ἐκεῖναι, οὐκ ἂν ἐσώθη πᾶσα σάρξ· διὰ δὲ τοὺς ἐκλεκτοὺς κολοβωθήσονται αἱ ἡμέραι ἐκεῖναι.
\par }{\PP \VS{23}Τότε ἐάν τις ὑμῖν εἴπῃ· Ἰδοὺ ὧδε ὁ Χριστός, ἤ· Ὧδε, μὴ πιστεύσητε·
\VS{24}ἐγερθήσονται γὰρ ψευδόχριστοι καὶ ψευδοπροφῆται καὶ δώσουσιν σημεῖα μεγάλα καὶ τέρατα ὥστε πλανῆσαι, εἰ δυνατὸν, καὶ τοὺς ἐκλεκτούς.
\VS{25}ἰδοὺ προείρηκα ὑμῖν.
\VS{26}Ἐὰν οὖν εἴπωσιν ὑμῖν· Ἰδοὺ ἐν τῇ ἐρήμῳ ἐστίν, μὴ ἐξέλθητε· Ἰδοὺ ἐν τοῖς ταμείοις, μὴ πιστεύσητε·
\VS{27}ὥσπερ γὰρ ἡ ἀστραπὴ ἐξέρχεται ἀπὸ ἀνατολῶν καὶ φαίνεται ἕως δυσμῶν, οὕτως ἔσται ἡ παρουσία τοῦ Υἱοῦ τοῦ ἀνθρώπου·
\VS{28}ὅπου ἐὰν ᾖ τὸ πτῶμα, ἐκεῖ συναχθήσονται οἱ ἀετοί.
\par }{\PP \VS{29}Εὐθέως δὲ μετὰ τὴν θλῖψιν τῶν ἡμερῶν ἐκείνων ¬Ὁ ἥλιος σκοτισθήσεται, ¬Καὶ ἡ σελήνη οὐ δώσει τὸ φέγγος αὐτῆς, ¬Καὶ οἱ ἀστέρες πεσοῦνται ἀπὸ τοῦ οὐρανοῦ, ¬Καὶ αἱ δυνάμεις τῶν οὐρανῶν σαλευθήσονται.
\par }{\PP \VS{30}Καὶ τότε φανήσεται τὸ σημεῖον τοῦ Υἱοῦ τοῦ ἀνθρώπου ἐν οὐρανῷ, καὶ τότε κόψονται πᾶσαι αἱ φυλαὶ τῆς γῆς καὶ ὄψονται τὸν Υἱὸν τοῦ ἀνθρώπου ἐρχόμενον ἐπὶ τῶν νεφελῶν τοῦ οὐρανοῦ μετὰ δυνάμεως καὶ δόξης πολλῆς·
\VS{31}καὶ ἀποστελεῖ τοὺς ἀγγέλους αὐτοῦ μετὰ σάλπιγγος μεγάλης, καὶ ἐπισυνάξουσιν τοὺς ἐκλεκτοὺς αὐτοῦ ἐκ τῶν τεσσάρων ἀνέμων ἀπ᾽ ἄκρων οὐρανῶν ἕως τῶν ἄκρων αὐτῶν.
\VS{32}Ἀπὸ δὲ τῆς συκῆς μάθετε τὴν παραβολήν· ὅταν ἤδη ὁ κλάδος αὐτῆς γένηται ἁπαλὸς καὶ τὰ φύλλα ἐκφύῃ, γινώσκετε ὅτι ἐγγὺς τὸ θέρος·
\VS{33}οὕτως καὶ ὑμεῖς, ὅταν ἴδητε πάντα ταῦτα, γινώσκετε ὅτι ἐγγύς ἐστιν ἐπὶ θύραις.
\VS{34}ἀμὴν λέγω ὑμῖν ὅτι οὐ μὴ παρέλθῃ ἡ γενεὰ αὕτη ἕως ἂν πάντα ταῦτα γένηται.
\VS{35}ὁ οὐρανὸς καὶ ἡ γῆ παρελεύσεται, οἱ δὲ λόγοι μου οὐ μὴ παρέλθωσιν.
\par }{\PP \VS{36}Περὶ δὲ τῆς ἡμέρας ἐκείνης καὶ ὥρας οὐδεὶς οἶδεν, οὐδὲ οἱ ἄγγελοι τῶν οὐρανῶν οὐδὲ ὁ Υἱός, εἰ μὴ ὁ Πατὴρ μόνος.
\par }{\PP \VS{37}ὥσπερ γὰρ αἱ ἡμέραι τοῦ Νῶε, οὕτως ἔσται ἡ παρουσία τοῦ Υἱοῦ τοῦ ἀνθρώπου.
\VS{38}ὡς γὰρ ἦσαν ἐν ταῖς ἡμέραις ἐκείναις ταῖς πρὸ τοῦ κατακλυσμοῦ τρώγοντες καὶ πίνοντες, γαμοῦντες καὶ γαμίζοντες, ἄχρι ἧς ἡμέρας εἰσῆλθεν Νῶε εἰς τὴν κιβωτόν,
\VS{39}καὶ οὐκ ἔγνωσαν ἕως ἦλθεν ὁ κατακλυσμὸς καὶ ἦρεν ἅπαντας, οὕτως ἔσται καὶ ἡ παρουσία τοῦ Υἱοῦ τοῦ ἀνθρώπου.
\VS{40}τότε δύο ἔσονται ἐν τῷ ἀγρῷ, εἷς παραλαμβάνεται καὶ εἷς ἀφίεται·
\VS{41}δύο ἀλήθουσαι ἐν τῷ μύλῳ, μία παραλαμβάνεται καὶ μία ἀφίεται.
\VS{42}Γρηγορεῖτε οὖν, ὅτι οὐκ οἴδατε ποίᾳ ἡμέρᾳ ὁ κύριος ὑμῶν ἔρχεται.
\VS{43}ἐκεῖνο δὲ γινώσκετε ὅτι εἰ ᾔδει ὁ οἰκοδεσπότης ποίᾳ φυλακῇ ὁ κλέπτης ἔρχεται, ἐγρηγόρησεν ἂν καὶ οὐκ ἂν εἴασεν διορυχθῆναι τὴν οἰκίαν αὐτοῦ.
\VS{44}διὰ τοῦτο καὶ ὑμεῖς γίνεσθε ἕτοιμοι, ὅτι ᾗ οὐ δοκεῖτε ὥρᾳ ὁ Υἱὸς τοῦ ἀνθρώπου ἔρχεται.
\par }{\PP \VS{45}Τίς ἄρα ἐστὶν ὁ πιστὸς δοῦλος καὶ φρόνιμος ὃν κατέστησεν ὁ κύριος ἐπὶ τῆς οἰκετείας αὐτοῦ τοῦ δοῦναι αὐτοῖς τὴν τροφὴν ἐν καιρῷ;
\VS{46}μακάριος ὁ δοῦλος ἐκεῖνος ὃν ἐλθὼν ὁ κύριος αὐτοῦ εὑρήσει οὕτως ποιοῦντα·
\VS{47}ἀμὴν λέγω ὑμῖν ὅτι ἐπὶ πᾶσιν τοῖς ὑπάρχουσιν αὐτοῦ καταστήσει αὐτόν.
\VS{48}Ἐὰν δὲ εἴπῃ ὁ κακὸς δοῦλος ἐκεῖνος ἐν τῇ καρδίᾳ αὐτοῦ· Χρονίζει μου ὁ κύριος,
\VS{49}καὶ ἄρξηται τύπτειν τοὺς συνδούλους αὐτοῦ, ἐσθίῃ δὲ καὶ πίνῃ μετὰ τῶν μεθυόντων,
\VS{50}ἥξει ὁ κύριος τοῦ δούλου ἐκείνου ἐν ἡμέρᾳ ᾗ οὐ προσδοκᾷ καὶ ἐν ὥρᾳ ᾗ οὐ γινώσκει,
\VS{51}καὶ διχοτομήσει αὐτὸν καὶ τὸ μέρος αὐτοῦ μετὰ τῶν ὑποκριτῶν θήσει· ἐκεῖ ἔσται ὁ κλαυθμὸς καὶ ὁ βρυγμὸς τῶν ὀδόντων.

\par }\Chap{25}{\PP \VerseOne{1}Τότε ὁμοιωθήσεται ἡ βασιλεία τῶν οὐρανῶν δέκα παρθένοις, αἵτινες λαβοῦσαι τὰς λαμπάδας ἑαυτῶν ἐξῆλθον εἰς ὑπάντησιν τοῦ νυμφίου.
\VS{2}πέντε δὲ ἐξ αὐτῶν ἦσαν μωραὶ καὶ πέντε φρόνιμοι.
\VS{3}αἱ γὰρ μωραὶ λαβοῦσαι τὰς λαμπάδας αὐτῶν οὐκ ἔλαβον μεθ᾽ ἑαυτῶν ἔλαιον.
\VS{4}αἱ δὲ φρόνιμοι ἔλαβον ἔλαιον ἐν τοῖς ἀγγείοις μετὰ τῶν λαμπάδων ἑαυτῶν.
\VS{5}χρονίζοντος δὲ τοῦ νυμφίου ἐνύσταξαν πᾶσαι καὶ ἐκάθευδον.
\VS{6}Μέσης δὲ νυκτὸς κραυγὴ γέγονεν· Ἰδοὺ ὁ νυμφίος, ἐξέρχεσθε εἰς ἀπάντησιν αὐτοῦ.
\VS{7}Τότε ἠγέρθησαν πᾶσαι αἱ παρθένοι ἐκεῖναι καὶ ἐκόσμησαν τὰς λαμπάδας ἑαυτῶν.
\VS{8}αἱ δὲ μωραὶ ταῖς φρονίμοις εἶπαν· Δότε ἡμῖν ἐκ τοῦ ἐλαίου ὑμῶν, ὅτι αἱ λαμπάδες ἡμῶν σβέννυνται.
\VS{9}Ἀπεκρίθησαν δὲ αἱ φρόνιμοι λέγουσαι· Μήποτε οὐ μὴ ἀρκέσῃ ἡμῖν καὶ ὑμῖν· πορεύεσθε μᾶλλον πρὸς τοὺς πωλοῦντας καὶ ἀγοράσατε ἑαυταῖς.
\VS{10}Ἀπερχομένων δὲ αὐτῶν ἀγοράσαι ἦλθεν ὁ νυμφίος, καὶ αἱ ἕτοιμοι εἰσῆλθον μετ᾽ αὐτοῦ εἰς τοὺς γάμους καὶ ἐκλείσθη ἡ θύρα.
\VS{11}Ὕστερον δὲ ἔρχονται καὶ αἱ λοιπαὶ παρθένοι λέγουσαι· Κύριε κύριε, ἄνοιξον ἡμῖν.
\VS{12}Ὁ δὲ ἀποκριθεὶς εἶπεν· Ἀμὴν λέγω ὑμῖν, οὐκ οἶδα ὑμᾶς.
\VS{13}Γρηγορεῖτε οὖν, ὅτι οὐκ οἴδατε τὴν ἡμέραν οὐδὲ τὴν ὥραν.
\VS{14}Ὥσπερ γὰρ ἄνθρωπος ἀποδημῶν ἐκάλεσεν τοὺς ἰδίους δούλους καὶ παρέδωκεν αὐτοῖς τὰ ὑπάρχοντα αὐτοῦ,
\VS{15}καὶ ᾧ μὲν ἔδωκεν πέντε τάλαντα, ᾧ δὲ δύο, ᾧ δὲ ἕν, ἑκάστῳ κατὰ τὴν ἰδίαν δύναμιν, καὶ ἀπεδήμησεν. Εὐθέως
\VS{16}Πορευθεὶς ὁ τὰ πέντε τάλαντα λαβὼν ἠργάσατο ἐν αὐτοῖς καὶ ἐκέρδησεν ἄλλα πέντε·
\VS{17}ὡσαύτως ὁ τὰ δύο ἐκέρδησεν ἄλλα δύο.
\VS{18}ὁ δὲ τὸ ἓν λαβὼν ἀπελθὼν ὤρυξεν γῆν καὶ ἔκρυψεν τὸ ἀργύριον τοῦ κυρίου αὐτοῦ.
\VS{19}Μετὰ δὲ πολὺν χρόνον ἔρχεται ὁ κύριος τῶν δούλων ἐκείνων καὶ συναίρει λόγον μετ᾽ αὐτῶν.
\VS{20}καὶ προσελθὼν ὁ τὰ πέντε τάλαντα λαβὼν προσήνεγκεν ἄλλα πέντε τάλαντα λέγων· Κύριε, πέντε τάλαντά μοι παρέδωκας· ἴδε ἄλλα πέντε τάλαντα ἐκέρδησα.
\VS{21}Ἔφη αὐτῷ ὁ κύριος αὐτοῦ· Εὖ, δοῦλε ἀγαθὲ καὶ πιστέ, ἐπὶ ὀλίγα ἦς πιστός, ἐπὶ πολλῶν σε καταστήσω· εἴσελθε εἰς τὴν χαρὰν τοῦ κυρίου σου.
\VS{22}Προσελθὼν δὲ καὶ ὁ τὰ δύο τάλαντα εἶπεν· Κύριε, δύο τάλαντά μοι παρέδωκας· ἴδε ἄλλα δύο τάλαντα ἐκέρδησα.
\VS{23}Ἔφη αὐτῷ ὁ κύριος αὐτοῦ· Εὖ, δοῦλε ἀγαθὲ καὶ πιστέ, ἐπὶ ὀλίγα ἦς πιστός, ἐπὶ πολλῶν σε καταστήσω· εἴσελθε εἰς τὴν χαρὰν τοῦ κυρίου σου.
\VS{24}Προσελθὼν δὲ καὶ ὁ τὸ ἓν τάλαντον εἰληφὼς εἶπεν· Κύριε, ἔγνων σε ὅτι σκληρὸς εἶ ἄνθρωπος, θερίζων ὅπου οὐκ ἔσπειρας καὶ συνάγων ὅθεν οὐ διεσκόρπισας,
\VS{25}καὶ φοβηθεὶς ἀπελθὼν ἔκρυψα τὸ τάλαντόν σου ἐν τῇ γῇ· ἴδε ἔχεις τὸ σόν.
\VS{26}Ἀποκριθεὶς δὲ ὁ κύριος αὐτοῦ εἶπεν αὐτῷ· Πονηρὲ δοῦλε καὶ ὀκνηρέ, ᾔδεις ὅτι θερίζω ὅπου οὐκ ἔσπειρα καὶ συνάγω ὅθεν οὐ διεσκόρπισα;
\VS{27}ἔδει σε οὖν βαλεῖν τὰ ἀργύριά μου τοῖς τραπεζίταις, καὶ ἐλθὼν ἐγὼ ἐκομισάμην ἂν τὸ ἐμὸν σὺν τόκῳ.
\VS{28}Ἄρατε οὖν ἀπ᾽ αὐτοῦ τὸ τάλαντον καὶ δότε τῷ ἔχοντι τὰ δέκα τάλαντα·
\VS{29}τῷ γὰρ ἔχοντι παντὶ δοθήσεται καὶ περισσευθήσεται, τοῦ δὲ μὴ ἔχοντος καὶ ὃ ἔχει ἀρθήσεται ἀπ᾽ αὐτοῦ.
\VS{30}καὶ τὸν ἀχρεῖον δοῦλον ἐκβάλετε εἰς τὸ σκότος τὸ ἐξώτερον· ἐκεῖ ἔσται ὁ κλαυθμὸς καὶ ὁ βρυγμὸς τῶν ὀδόντων.
\par }{\PP \VS{31}Ὅταν δὲ ἔλθῃ ὁ υἱὸς τοῦ ἀνθρώπου ἐν τῇ δόξῃ αὐτοῦ καὶ πάντες οἱ ἄγγελοι μετ᾽ αὐτοῦ, τότε καθίσει ἐπὶ θρόνου δόξης αὐτοῦ·
\VS{32}καὶ συναχθήσονται ἔμπροσθεν αὐτοῦ πάντα τὰ ἔθνη, καὶ ἀφορίσει αὐτοὺς ἀπ᾽ ἀλλήλων, ὥσπερ ὁ ποιμὴν ἀφορίζει τὰ πρόβατα ἀπὸ τῶν ἐρίφων,
\VS{33}καὶ στήσει τὰ μὲν πρόβατα ἐκ δεξιῶν αὐτοῦ, τὰ δὲ ἐρίφια ἐξ εὐωνύμων.
\VS{34}Τότε ἐρεῖ ὁ Βασιλεὺς τοῖς ἐκ δεξιῶν αὐτοῦ· Δεῦτε οἱ εὐλογημένοι τοῦ Πατρός μου, κληρονομήσατε τὴν ἡτοιμασμένην ὑμῖν βασιλείαν ἀπὸ καταβολῆς κόσμου.
\VS{35}ἐπείνασα γὰρ καὶ ἐδώκατέ μοι φαγεῖν, ἐδίψησα καὶ ἐποτίσατέ με, ξένος ἤμην καὶ συνηγάγετέ με,
\VS{36}γυμνὸς καὶ περιεβάλετέ με, ἠσθένησα καὶ ἐπεσκέψασθέ με, ἐν φυλακῇ ἤμην καὶ ἤλθατε πρός με.
\VS{37}Τότε ἀποκριθήσονται αὐτῷ οἱ δίκαιοι λέγοντες· Κύριε, πότε σε εἴδομεν πεινῶντα καὶ ἐθρέψαμεν, ἢ διψῶντα καὶ ἐποτίσαμεν;
\VS{38}πότε δέ σε εἴδομεν ξένον καὶ συνηγάγομεν, ἢ γυμνὸν καὶ περιεβάλομεν;
\VS{39}πότε δέ σε εἴδομεν ἀσθενοῦντα ἢ ἐν φυλακῇ καὶ ἤλθομεν πρός σε;
\VS{40}Καὶ ἀποκριθεὶς ὁ Βασιλεὺς ἐρεῖ αὐτοῖς· Ἀμὴν λέγω ὑμῖν, ἐφ᾽ ὅσον ἐποιήσατε ἑνὶ τούτων τῶν ἀδελφῶν μου τῶν ἐλαχίστων, ἐμοὶ ἐποιήσατε.
\VS{41}Τότε ἐρεῖ καὶ τοῖς ἐξ εὐωνύμων· Πορεύεσθε ἀπ᾽ ἐμοῦ οἱ κατηραμένοι εἰς τὸ πῦρ τὸ αἰώνιον τὸ ἡτοιμασμένον τῷ διαβόλῳ καὶ τοῖς ἀγγέλοις αὐτοῦ.
\VS{42}ἐπείνασα γὰρ καὶ οὐκ ἐδώκατέ μοι φαγεῖν, ἐδίψησα καὶ οὐκ ἐποτίσατέ με,
\VS{43}ξένος ἤμην καὶ οὐ συνηγάγετέ με, γυμνὸς καὶ οὐ περιεβάλετέ με, ἀσθενὴς καὶ ἐν φυλακῇ καὶ οὐκ ἐπεσκέψασθέ με.
\VS{44}Τότε ἀποκριθήσονται καὶ αὐτοὶ λέγοντες· Κύριε, πότε σε εἴδομεν πεινῶντα ἢ διψῶντα ἢ ξένον ἢ γυμνὸν ἢ ἀσθενῆ ἢ ἐν φυλακῇ καὶ οὐ διηκονήσαμέν σοι;
\VS{45}Τότε ἀποκριθήσεται αὐτοῖς λέγων· Ἀμὴν λέγω ὑμῖν, ἐφ᾽ ὅσον οὐκ ἐποιήσατε ἑνὶ τούτων τῶν ἐλαχίστων, οὐδὲ ἐμοὶ ἐποιήσατε.
\VS{46}Καὶ ἀπελεύσονται οὗτοι εἰς κόλασιν αἰώνιον, οἱ δὲ δίκαιοι εἰς ζωὴν αἰώνιον.

\par }\Chap{26}{\PP \VerseOne{1}Καὶ ἐγένετο ὅτε ἐτέλεσεν ὁ Ἰησοῦς πάντας τοὺς λόγους τούτους, εἶπεν τοῖς μαθηταῖς αὐτοῦ·
\VS{2}Οἴδατε ὅτι μετὰ δύο ἡμέρας τὸ πάσχα γίνεται, καὶ ὁ Υἱὸς τοῦ ἀνθρώπου παραδίδοται εἰς τὸ σταυρωθῆναι.
\par }{\PP \VS{3}Τότε συνήχθησαν οἱ ἀρχιερεῖς καὶ οἱ πρεσβύτεροι τοῦ λαοῦ εἰς τὴν αὐλὴν τοῦ ἀρχιερέως τοῦ λεγομένου Καϊάφα
\VS{4}καὶ συνεβουλεύσαντο ἵνα τὸν Ἰησοῦν δόλῳ κρατήσωσιν καὶ ἀποκτείνωσιν·
\VS{5}ἔλεγον δέ· Μὴ ἐν τῇ ἑορτῇ, ἵνα μὴ θόρυβος γένηται ἐν τῷ λαῷ.
\par }{\PP \VS{6}Τοῦ δὲ Ἰησοῦ γενομένου ἐν Βηθανίᾳ ἐν οἰκίᾳ Σίμωνος τοῦ λεπροῦ,
\VS{7}προσῆλθεν αὐτῷ γυνὴ ἔχουσα ἀλάβαστρον μύρου βαρυτίμου καὶ κατέχεεν ἐπὶ τῆς κεφαλῆς αὐτοῦ ἀνακειμένου.
\VS{8}Ἰδόντες δὲ οἱ μαθηταὶ ἠγανάκτησαν λέγοντες· Εἰς τί ἡ ἀπώλεια αὕτη;
\VS{9}ἐδύνατο γὰρ τοῦτο πραθῆναι πολλοῦ καὶ δοθῆναι πτωχοῖς.
\VS{10}Γνοὺς δὲ ὁ Ἰησοῦς εἶπεν αὐτοῖς· Τί κόπους παρέχετε τῇ γυναικί; ἔργον γὰρ καλὸν ἠργάσατο εἰς ἐμέ·
\VS{11}πάντοτε γὰρ τοὺς πτωχοὺς ἔχετε μεθ᾽ ἑαυτῶν, ἐμὲ δὲ οὐ πάντοτε ἔχετε·
\VS{12}βαλοῦσα γὰρ αὕτη τὸ μύρον τοῦτο ἐπὶ τοῦ σώματός μου πρὸς τὸ ἐνταφιάσαι με ἐποίησεν.
\VS{13}ἀμὴν λέγω ὑμῖν, ὅπου ἐὰν κηρυχθῇ τὸ εὐαγγέλιον τοῦτο ἐν ὅλῳ τῷ κόσμῳ, λαληθήσεται καὶ ὃ ἐποίησεν αὕτη εἰς μνημόσυνον αὐτῆς.
\par }{\PP \VS{14}Τότε πορευθεὶς εἷς τῶν δώδεκα, ὁ λεγόμενος Ἰούδας Ἰσκαριώτης, πρὸς τοὺς ἀρχιερεῖς
\VS{15}εἶπεν· Τί θέλετέ μοι δοῦναι, κἀγὼ ὑμῖν παραδώσω αὐτόν; οἱ δὲ ἔστησαν αὐτῷ τριάκοντα ἀργύρια.
\VS{16}καὶ ἀπὸ τότε ἐζήτει εὐκαιρίαν ἵνα αὐτὸν παραδῷ.
\VS{17}Τῇ δὲ πρώτῃ τῶν ἀζύμων προσῆλθον οἱ μαθηταὶ τῷ Ἰησοῦ λέγοντες· Ποῦ θέλεις ἑτοιμάσωμέν σοι φαγεῖν τὸ πάσχα;
\VS{18}Ὁ δὲ εἶπεν· Ὑπάγετε εἰς τὴν πόλιν πρὸς τὸν δεῖνα καὶ εἴπατε αὐτῷ· Ὁ Διδάσκαλος λέγει· Ὁ καιρός μου ἐγγύς ἐστιν, πρὸς σὲ ποιῶ τὸ πάσχα μετὰ τῶν μαθητῶν μου.
\VS{19}καὶ ἐποίησαν οἱ μαθηταὶ ὡς συνέταξεν αὐτοῖς ὁ Ἰησοῦς καὶ ἡτοίμασαν τὸ πάσχα.
\par }{\PP \VS{20}Ὀψίας δὲ γενομένης ἀνέκειτο μετὰ τῶν δώδεκα.
\VS{21}καὶ ἐσθιόντων αὐτῶν εἶπεν· Ἀμὴν λέγω ὑμῖν ὅτι εἷς ἐξ ὑμῶν παραδώσει με.
\VS{22}Καὶ λυπούμενοι σφόδρα ἤρξαντο λέγειν αὐτῷ εἷς ἕκαστος· Μήτι ἐγώ εἰμι, Κύριε;
\VS{23}Ὁ δὲ ἀποκριθεὶς εἶπεν· Ὁ ἐμβάψας μετ᾽ ἐμοῦ τὴν χεῖρα ἐν τῷ τρυβλίῳ οὗτός με παραδώσει.
\VS{24}ὁ μὲν Υἱὸς τοῦ ἀνθρώπου ὑπάγει καθὼς γέγραπται περὶ αὐτοῦ, οὐαὶ δὲ τῷ ἀνθρώπῳ ἐκείνῳ δι᾽ οὗ ὁ Υἱὸς τοῦ ἀνθρώπου παραδίδοται· καλὸν ἦν αὐτῷ εἰ οὐκ ἐγεννήθη ὁ ἄνθρωπος ἐκεῖνος.
\VS{25}Ἀποκριθεὶς δὲ Ἰούδας ὁ παραδιδοὺς αὐτὸν εἶπεν· Μήτι ἐγώ εἰμι, ῥαββί; Λέγει αὐτῷ· Σὺ εἶπας.
\par }{\PP \VS{26}Ἐσθιόντων δὲ αὐτῶν λαβὼν ὁ Ἰησοῦς ἄρτον καὶ εὐλογήσας ἔκλασεν καὶ δοὺς τοῖς μαθηταῖς εἶπεν· Λάβετε φάγετε, τοῦτό ἐστιν τὸ σῶμά μου.
\VS{27}Καὶ λαβὼν ποτήριον καὶ εὐχαριστήσας ἔδωκεν αὐτοῖς λέγων· Πίετε ἐξ αὐτοῦ πάντες,
\VS{28}τοῦτο γάρ ἐστιν τὸ αἷμά μου τῆς διαθήκης τὸ περὶ πολλῶν ἐκχυννόμενον εἰς ἄφεσιν ἁμαρτιῶν.
\VS{29}λέγω δὲ ὑμῖν, οὐ μὴ πίω ἀπ᾽ ἄρτι ἐκ τούτου τοῦ γενήματος τῆς ἀμπέλου ἕως τῆς ἡμέρας ἐκείνης ὅταν αὐτὸ πίνω μεθ᾽ ὑμῶν καινὸν ἐν τῇ βασιλείᾳ τοῦ Πατρός μου.
\VS{30}Καὶ ὑμνήσαντες ἐξῆλθον εἰς τὸ ὄρος τῶν Ἐλαιῶν.
\par }{\PP \VS{31}Τότε λέγει αὐτοῖς ὁ Ἰησοῦς· Πάντες ὑμεῖς σκανδαλισθήσεσθε ἐν ἐμοὶ ἐν τῇ νυκτὶ ταύτῃ, γέγραπται γάρ· ¬Πατάξω τὸν ποιμένα, ¬Καὶ διασκορπισθήσονται τὰ πρόβατα τῆς ποίμνης.
\par }{\PP \VS{32}Μετὰ δὲ τὸ ἐγερθῆναί με προάξω ὑμᾶς εἰς τὴν Γαλιλαίαν.
\VS{33}Ἀποκριθεὶς δὲ ὁ Πέτρος εἶπεν αὐτῷ· Εἰ πάντες σκανδαλισθήσονται ἐν σοί, ἐγὼ οὐδέποτε σκανδαλισθήσομαι.
\VS{34}Ἔφη αὐτῷ ὁ Ἰησοῦς· Ἀμὴν λέγω σοι ὅτι ἐν ταύτῃ τῇ νυκτὶ πρὶν ἀλέκτορα φωνῆσαι τρὶς ἀπαρνήσῃ με.
\VS{35}Λέγει αὐτῷ ὁ Πέτρος· Κἂν δέῃ με σὺν σοὶ ἀποθανεῖν, οὐ μή σε ἀπαρνήσομαι. ὁμοίως καὶ πάντες οἱ μαθηταὶ εἶπαν.
\par }{\PP \VS{36}Τότε ἔρχεται μετ᾽ αὐτῶν ὁ Ἰησοῦς εἰς χωρίον λεγόμενον Γεθσημανὶ καὶ λέγει τοῖς μαθηταῖς· Καθίσατε αὐτοῦ ἕως οὗ ἀπελθὼν ἐκεῖ προσεύξωμαι.
\VS{37}Καὶ παραλαβὼν τὸν Πέτρον καὶ τοὺς δύο υἱοὺς Ζεβεδαίου ἤρξατο λυπεῖσθαι καὶ ἀδημονεῖν.
\VS{38}τότε λέγει αὐτοῖς· Περίλυπός ἐστιν ἡ ψυχή μου ἕως θανάτου· μείνατε ὧδε καὶ γρηγορεῖτε μετ᾽ ἐμοῦ.
\VS{39}Καὶ προελθὼν μικρὸν ἔπεσεν ἐπὶ πρόσωπον αὐτοῦ προσευχόμενος καὶ λέγων· Πάτερ μου, εἰ δυνατόν ἐστιν, παρελθάτω ἀπ᾽ ἐμοῦ τὸ ποτήριον τοῦτο· πλὴν οὐχ ὡς ἐγὼ θέλω ἀλλ᾽ ὡς σύ.
\VS{40}Καὶ ἔρχεται πρὸς τοὺς μαθητὰς καὶ εὑρίσκει αὐτοὺς καθεύδοντας, καὶ λέγει τῷ Πέτρῳ· Οὕτως οὐκ ἰσχύσατε μίαν ὥραν γρηγορῆσαι μετ᾽ ἐμοῦ;
\VS{41}γρηγορεῖτε καὶ προσεύχεσθε, ἵνα μὴ εἰσέλθητε εἰς πειρασμόν· τὸ μὲν πνεῦμα πρόθυμον ἡ δὲ σὰρξ ἀσθενής.
\VS{42}Πάλιν ἐκ δευτέρου ἀπελθὼν προσηύξατο λέγων· Πάτερ μου, εἰ οὐ δύναται τοῦτο παρελθεῖν ἐὰν μὴ αὐτὸ πίω, γενηθήτω τὸ θέλημά σου.
\VS{43}καὶ ἐλθὼν πάλιν εὗρεν αὐτοὺς καθεύδοντας, ἦσαν γὰρ αὐτῶν οἱ ὀφθαλμοὶ βεβαρημένοι.
\VS{44}Καὶ ἀφεὶς αὐτοὺς πάλιν ἀπελθὼν προσηύξατο ἐκ τρίτου τὸν αὐτὸν λόγον εἰπὼν πάλιν.
\VS{45}τότε ἔρχεται πρὸς τοὺς μαθητὰς καὶ λέγει αὐτοῖς· Καθεύδετε τὸ λοιπὸν καὶ ἀναπαύεσθε· ἰδοὺ ἤγγικεν ἡ ὥρα καὶ ὁ Υἱὸς τοῦ ἀνθρώπου παραδίδοται εἰς χεῖρας ἁμαρτωλῶν.
\VS{46}ἐγείρεσθε ἄγωμεν· ἰδοὺ ἤγγικεν ὁ παραδιδούς με.
\par }{\PP \VS{47}Καὶ ἔτι αὐτοῦ λαλοῦντος ἰδοὺ Ἰούδας εἷς τῶν δώδεκα ἦλθεν καὶ μετ᾽ αὐτοῦ ὄχλος πολὺς μετὰ μαχαιρῶν καὶ ξύλων ἀπὸ τῶν ἀρχιερέων καὶ πρεσβυτέρων τοῦ λαοῦ.
\VS{48}Ὁ δὲ παραδιδοὺς αὐτὸν ἔδωκεν αὐτοῖς σημεῖον λέγων· Ὃν ἂν φιλήσω αὐτός ἐστιν, κρατήσατε αὐτόν.
\VS{49}καὶ εὐθέως προσελθὼν τῷ Ἰησοῦ εἶπεν· Χαῖρε, ῥαββί, καὶ κατεφίλησεν αὐτόν.
\VS{50}Ὁ δὲ Ἰησοῦς εἶπεν αὐτῷ· Ἑταῖρε, ἐφ᾽ ὃ πάρει. Τότε προσελθόντες ἐπέβαλον τὰς χεῖρας ἐπὶ τὸν Ἰησοῦν καὶ ἐκράτησαν αὐτόν.
\VS{51}καὶ ἰδοὺ εἷς τῶν μετὰ Ἰησοῦ ἐκτείνας τὴν χεῖρα ἀπέσπασεν τὴν μάχαιραν αὐτοῦ καὶ πατάξας τὸν δοῦλον τοῦ ἀρχιερέως ἀφεῖλεν αὐτοῦ τὸ ὠτίον.
\VS{52}Τότε λέγει αὐτῷ ὁ Ἰησοῦς· Ἀπόστρεψον τὴν μάχαιράν σου εἰς τὸν τόπον αὐτῆς· πάντες γὰρ οἱ λαβόντες μάχαιραν ἐν μαχαίρῃ ἀπολοῦνται.
\VS{53}ἢ δοκεῖς ὅτι οὐ δύναμαι παρακαλέσαι τὸν Πατέρα μου, καὶ παραστήσει μοι ἄρτι πλείω δώδεκα λεγιῶνας ἀγγέλων;
\VS{54}πῶς οὖν πληρωθῶσιν αἱ γραφαὶ ὅτι οὕτως δεῖ γενέσθαι;
\VS{55}Ἐν ἐκείνῃ τῇ ὥρᾳ εἶπεν ὁ Ἰησοῦς τοῖς ὄχλοις· Ὡς ἐπὶ λῃστὴν ἐξήλθατε μετὰ μαχαιρῶν καὶ ξύλων συλλαβεῖν με; καθ᾽ ἡμέραν ἐν τῷ ἱερῷ ἐκαθεζόμην διδάσκων καὶ οὐκ ἐκρατήσατέ με.
\VS{56}Τοῦτο δὲ ὅλον γέγονεν ἵνα πληρωθῶσιν αἱ γραφαὶ τῶν προφητῶν. Τότε οἱ μαθηταὶ πάντες ἀφέντες αὐτὸν ἔφυγον.
\par }{\PP \VS{57}Οἱ δὲ κρατήσαντες τὸν Ἰησοῦν ἀπήγαγον πρὸς Καϊάφαν τὸν ἀρχιερέα, ὅπου οἱ γραμματεῖς καὶ οἱ πρεσβύτεροι συνήχθησαν.
\VS{58}ὁ δὲ Πέτρος ἠκολούθει αὐτῷ ἀπὸ μακρόθεν ἕως τῆς αὐλῆς τοῦ ἀρχιερέως καὶ εἰσελθὼν ἔσω ἐκάθητο μετὰ τῶν ὑπηρετῶν ἰδεῖν τὸ τέλος.
\par }{\PP \VS{59}Οἱ δὲ ἀρχιερεῖς καὶ τὸ συνέδριον ὅλον ἐζήτουν ψευδομαρτυρίαν κατὰ τοῦ Ἰησοῦ ὅπως αὐτὸν θανατώσωσιν,
\VS{60}καὶ οὐχ εὗρον πολλῶν προσελθόντων ψευδομαρτύρων. Ὕστερον δὲ προσελθόντες δύο
\VS{61}εἶπαν· Οὗτος ἔφη· Δύναμαι καταλῦσαι τὸν ναὸν τοῦ Θεοῦ καὶ διὰ τριῶν ἡμερῶν οἰκοδομῆσαι.
\VS{62}Καὶ ἀναστὰς ὁ ἀρχιερεὺς εἶπεν αὐτῷ· Οὐδὲν ἀποκρίνῃ τί οὗτοί σου καταμαρτυροῦσιν;
\VS{63}Ὁ δὲ Ἰησοῦς ἐσιώπα. Καὶ ὁ ἀρχιερεὺς εἶπεν αὐτῷ· Ἐξορκίζω σε κατὰ τοῦ Θεοῦ τοῦ ζῶντος ἵνα ἡμῖν εἴπῃς εἰ σὺ εἶ ὁ Χριστὸς ὁ Υἱὸς τοῦ Θεοῦ.
\VS{64}Λέγει αὐτῷ ὁ Ἰησοῦς· Σὺ εἶπας. πλὴν λέγω ὑμῖν· ἀπ᾽ ἄρτι ὄψεσθε τὸν Υἱὸν τοῦ ἀνθρώπου καθήμενον ἐκ δεξιῶν τῆς δυνάμεως καὶ ἐρχόμενον ἐπὶ τῶν νεφελῶν τοῦ οὐρανοῦ.
\VS{65}Τότε ὁ ἀρχιερεὺς διέρρηξεν τὰ ἱμάτια αὐτοῦ λέγων· Ἐβλασφήμησεν· τί ἔτι χρείαν ἔχομεν μαρτύρων; ἴδε νῦν ἠκούσατε τὴν βλασφημίαν·
\VS{66}τί ὑμῖν δοκεῖ; Οἱ δὲ ἀποκριθέντες εἶπαν· Ἔνοχος θανάτου ἐστίν.
\par }{\PP \VS{67}Τότε ἐνέπτυσαν εἰς τὸ πρόσωπον αὐτοῦ καὶ ἐκολάφισαν αὐτόν, οἱ δὲ ἐράπισαν
\VS{68}λέγοντες· Προφήτευσον ἡμῖν, Χριστέ, τίς ἐστιν ὁ παίσας σε;
\par }{\PP \VS{69}Ὁ δὲ Πέτρος ἐκάθητο ἔξω ἐν τῇ αὐλῇ· καὶ προσῆλθεν αὐτῷ μία παιδίσκη λέγουσα· Καὶ σὺ ἦσθα μετὰ Ἰησοῦ τοῦ Γαλιλαίου.
\VS{70}Ὁ δὲ ἠρνήσατο ἔμπροσθεν πάντων λέγων· Οὐκ οἶδα τί λέγεις.
\VS{71}Ἐξελθόντα δὲ εἰς τὸν πυλῶνα εἶδεν αὐτὸν ἄλλη καὶ λέγει τοῖς ἐκεῖ· Οὗτος ἦν μετὰ Ἰησοῦ τοῦ Ναζωραίου.
\VS{72}Καὶ πάλιν ἠρνήσατο μετὰ ὅρκου ὅτι Οὐκ οἶδα τὸν ἄνθρωπον.
\VS{73}Μετὰ μικρὸν δὲ προσελθόντες οἱ ἑστῶτες εἶπον τῷ Πέτρῳ· Ἀληθῶς καὶ σὺ ἐξ αὐτῶν εἶ, καὶ γὰρ ἡ λαλιά σου δῆλόν σε ποιεῖ.
\VS{74}Τότε ἤρξατο καταθεματίζειν καὶ ὀμνύειν ὅτι Οὐκ οἶδα τὸν ἄνθρωπον. Καὶ εὐθέως ἀλέκτωρ ἐφώνησεν.
\par }{\PP \VS{75}Καὶ ἐμνήσθη ὁ Πέτρος τοῦ ῥήματος Ἰησοῦ εἰρηκότος ὅτι Πρὶν ἀλέκτορα φωνῆσαι τρὶς ἀπαρνήσῃ με· καὶ ἐξελθὼν ἔξω ἔκλαυσεν πικρῶς.

\par }\Chap{27}{\PP \VerseOne{1}Πρωΐας δὲ γενομένης συμβούλιον ἔλαβον πάντες οἱ ἀρχιερεῖς καὶ οἱ πρεσβύτεροι τοῦ λαοῦ κατὰ τοῦ Ἰησοῦ ὥστε θανατῶσαι αὐτόν·
\VS{2}καὶ δήσαντες αὐτὸν ἀπήγαγον καὶ παρέδωκαν Πιλάτῳ τῷ ἡγεμόνι.
\par }{\PP \VS{3}Τότε ἰδὼν Ἰούδας ὁ παραδιδοὺς αὐτὸν ὅτι κατεκρίθη, μεταμεληθεὶς ἔστρεψεν τὰ τριάκοντα ἀργύρια τοῖς ἀρχιερεῦσιν καὶ πρεσβυτέροις
\VS{4}λέγων· Ἥμαρτον παραδοὺς αἷμα ἀθῷον. Οἱ δὲ εἶπαν· Τί πρὸς ἡμᾶς; σὺ ὄψῃ.
\VS{5}Καὶ ῥίψας τὰ ἀργύρια εἰς τὸν ναὸν ἀνεχώρησεν, καὶ ἀπελθὼν ἀπήγξατο.
\VS{6}Οἱ δὲ ἀρχιερεῖς λαβόντες τὰ ἀργύρια εἶπαν· Οὐκ ἔξεστιν βαλεῖν αὐτὰ εἰς τὸν κορβανᾶν, ἐπεὶ τιμὴ αἵματός ἐστιν.
\VS{7}συμβούλιον δὲ λαβόντες ἠγόρασαν ἐξ αὐτῶν τὸν ἀγρὸν τοῦ κεραμέως εἰς ταφὴν τοῖς ξένοις.
\VS{8}διὸ ἐκλήθη ὁ ἀγρὸς ἐκεῖνος Ἀγρὸς αἵματος ἕως τῆς σήμερον.
\VS{9}τότε ἐπληρώθη τὸ ῥηθὲν διὰ Ἰερεμίου τοῦ προφήτου λέγοντος· Καὶ ἔλαβον τὰ τριάκοντα ἀργύρια, τὴν τιμὴν τοῦ τετιμημένου ὃν ἐτιμήσαντο ἀπὸ υἱῶν Ἰσραήλ,
\VS{10}καὶ ἔδωκαν αὐτὰ εἰς τὸν ἀγρὸν τοῦ κεραμέως, καθὰ συνέταξέν μοι Κύριος.
\par }{\PP \VS{11}Ὁ δὲ Ἰησοῦς ἐστάθη ἔμπροσθεν τοῦ ἡγεμόνος· καὶ ἐπηρώτησεν αὐτὸν ὁ ἡγεμὼν λέγων· Σὺ εἶ ὁ Βασιλεὺς τῶν Ἰουδαίων; Ὁ δὲ Ἰησοῦς ἔφη· Σὺ λέγεις.
\VS{12}Καὶ ἐν τῷ κατηγορεῖσθαι αὐτὸν ὑπὸ τῶν ἀρχιερέων καὶ πρεσβυτέρων οὐδὲν ἀπεκρίνατο.
\VS{13}Τότε λέγει αὐτῷ ὁ Πιλᾶτος· Οὐκ ἀκούεις πόσα σου καταμαρτυροῦσιν;
\VS{14}Καὶ οὐκ ἀπεκρίθη αὐτῷ πρὸς οὐδὲ ἓν ῥῆμα, ὥστε θαυμάζειν τὸν ἡγεμόνα λίαν.
\par }{\PP \VS{15}Κατὰ δὲ ἑορτὴν εἰώθει ὁ ἡγεμὼν ἀπολύειν ἕνα τῷ ὄχλῳ δέσμιον ὃν ἤθελον.
\VS{16}εἶχον δὲ τότε δέσμιον ἐπίσημον λεγόμενον Ἰησοῦν Βαραββᾶν.
\VS{17}συνηγμένων οὖν αὐτῶν εἶπεν αὐτοῖς ὁ Πιλᾶτος· Τίνα θέλετε ἀπολύσω ὑμῖν, Ἰησοῦν τὸν Βαραββᾶν ἢ Ἰησοῦν τὸν λεγόμενον Χριστόν;
\VS{18}ᾔδει γὰρ ὅτι διὰ φθόνον παρέδωκαν αὐτόν.
\par }{\PP \VS{19}Καθημένου δὲ αὐτοῦ ἐπὶ τοῦ βήματος ἀπέστειλεν πρὸς αὐτὸν ἡ γυνὴ αὐτοῦ λέγουσα· Μηδὲν σοὶ καὶ τῷ δικαίῳ ἐκείνῳ· πολλὰ γὰρ ἔπαθον σήμερον κατ᾽ ὄναρ δι᾽ αὐτόν.
\par }{\PP \VS{20}Οἱ δὲ ἀρχιερεῖς καὶ οἱ πρεσβύτεροι ἔπεισαν τοὺς ὄχλους ἵνα αἰτήσωνται τὸν Βαραββᾶν, τὸν δὲ Ἰησοῦν ἀπολέσωσιν.
\VS{21}Ἀποκριθεὶς δὲ ὁ ἡγεμὼν εἶπεν αὐτοῖς· Τίνα θέλετε ἀπὸ τῶν δύο ἀπολύσω ὑμῖν; Οἱ δὲ εἶπαν· Τὸν Βαραββᾶν.
\VS{22}Λέγει αὐτοῖς ὁ Πιλᾶτος· Τί οὖν ποιήσω Ἰησοῦν τὸν λεγόμενον Χριστόν; Λέγουσιν πάντες· Σταυρωθήτω.
\VS{23}Ὁ δὲ ἔφη· Τί γὰρ κακὸν ἐποίησεν; Οἱ δὲ περισσῶς ἔκραζον λέγοντες· Σταυρωθήτω.
\par }{\PP \VS{24}Ἰδὼν δὲ ὁ Πιλᾶτος ὅτι οὐδὲν ὠφελεῖ ἀλλὰ μᾶλλον θόρυβος γίνεται, λαβὼν ὕδωρ ἀπενίψατο τὰς χεῖρας ἀπέναντι τοῦ ὄχλου λέγων· Ἀθῷός εἰμι ἀπὸ τοῦ αἵματος τούτου· ὑμεῖς ὄψεσθε.
\VS{25}Καὶ ἀποκριθεὶς πᾶς ὁ λαὸς εἶπεν· Τὸ αἷμα αὐτοῦ ἐφ᾽ ἡμᾶς καὶ ἐπὶ τὰ τέκνα ἡμῶν.
\VS{26}Τότε ἀπέλυσεν αὐτοῖς τὸν Βαραββᾶν, τὸν δὲ Ἰησοῦν φραγελλώσας παρέδωκεν ἵνα σταυρωθῇ.
\par }{\PP \VS{27}Τότε οἱ στρατιῶται τοῦ ἡγεμόνος παραλαβόντες τὸν Ἰησοῦν εἰς τὸ πραιτώριον συνήγαγον ἐπ᾽ αὐτὸν ὅλην τὴν σπεῖραν.
\VS{28}καὶ ἐκδύσαντες αὐτὸν χλαμύδα κοκκίνην περιέθηκαν αὐτῷ,
\VS{29}καὶ πλέξαντες στέφανον ἐξ ἀκανθῶν ἐπέθηκαν ἐπὶ τῆς κεφαλῆς αὐτοῦ καὶ κάλαμον ἐν τῇ δεξιᾷ αὐτοῦ, καὶ γονυπετήσαντες ἔμπροσθεν αὐτοῦ ἐνέπαιξαν αὐτῷ λέγοντες· Χαῖρε, Βασιλεῦ τῶν Ἰουδαίων,
\VS{30}καὶ ἐμπτύσαντες εἰς αὐτὸν ἔλαβον τὸν κάλαμον καὶ ἔτυπτον εἰς τὴν κεφαλὴν αὐτοῦ.
\VS{31}Καὶ ὅτε ἐνέπαιξαν αὐτῷ, ἐξέδυσαν αὐτὸν τὴν χλαμύδα καὶ ἐνέδυσαν αὐτὸν τὰ ἱμάτια αὐτοῦ καὶ ἀπήγαγον αὐτὸν εἰς τὸ σταυρῶσαι.
\VS{32}Ἐξερχόμενοι δὲ εὗρον ἄνθρωπον Κυρηναῖον ὀνόματι Σίμωνα, τοῦτον ἠγγάρευσαν ἵνα ἄρῃ τὸν σταυρὸν αὐτοῦ.
\par }{\PP \VS{33}Καὶ ἐλθόντες εἰς τόπον λεγόμενον Γολγοθᾶ, ὅ ἐστιν κρανίου τόπος λεγόμενος,
\VS{34}ἔδωκαν αὐτῷ πιεῖν οἶνον μετὰ χολῆς μεμιγμένον· καὶ γευσάμενος οὐκ ἠθέλησεν πιεῖν.
\VS{35}Σταυρώσαντες δὲ αὐτὸν διεμερίσαντο τὰ ἱμάτια αὐτοῦ βάλλοντες κλῆρον,
\VS{36}καὶ καθήμενοι ἐτήρουν αὐτὸν ἐκεῖ.
\VS{37}καὶ ἐπέθηκαν ἐπάνω τῆς κεφαλῆς αὐτοῦ τὴν αἰτίαν αὐτοῦ γεγραμμένην· ¬ΟΥΤΟΣ ΕΣΤΙΝ ΙΗΣΟΥΣ Ο ΒΑΣΙΛΕΥΣ ΤΩΝ ΙΟΥΔΑΙΩΝ.
\par }{\PP \VS{38}Τότε σταυροῦνται σὺν αὐτῷ δύο λῃσταί, εἷς ἐκ δεξιῶν καὶ εἷς ἐξ εὐωνύμων.
\VS{39}Οἱ δὲ παραπορευόμενοι ἐβλασφήμουν αὐτὸν κινοῦντες τὰς κεφαλὰς αὐτῶν
\VS{40}καὶ λέγοντες· Ὁ καταλύων τὸν ναὸν καὶ ἐν τρισὶν ἡμέραις οἰκοδομῶν, σῶσον σεαυτόν, εἰ Υἱὸς εἶ τοῦ Θεοῦ, καὶ κατάβηθι ἀπὸ τοῦ σταυροῦ.
\VS{41}Ὁμοίως καὶ οἱ ἀρχιερεῖς ἐμπαίζοντες μετὰ τῶν γραμματέων καὶ πρεσβυτέρων ἔλεγον·
\VS{42}Ἄλλους ἔσωσεν, ἑαυτὸν οὐ δύναται σῶσαι· Βασιλεὺς Ἰσραήλ ἐστιν, καταβάτω νῦν ἀπὸ τοῦ σταυροῦ καὶ πιστεύσομεν ἐπ᾽ αὐτόν.
\VS{43}πέποιθεν ἐπὶ τὸν Θεόν, ῥυσάσθω νῦν εἰ θέλει αὐτόν· εἶπεν γὰρ ὅτι Θεοῦ εἰμι Υἱός.
\VS{44}Τὸ δ᾽ αὐτὸ καὶ οἱ λῃσταὶ οἱ συσταυρωθέντες σὺν αὐτῷ ὠνείδιζον αὐτόν.
\par }{\PP \VS{45}Ἀπὸ δὲ ἕκτης ὥρας σκότος ἐγένετο ἐπὶ πᾶσαν τὴν γῆν ἕως ὥρας ἐνάτης.
\VS{46}περὶ δὲ τὴν ἐνάτην ὥραν ἀνεβόησεν ὁ Ἰησοῦς φωνῇ μεγάλῃ λέγων· ¬Ἠλὶ ἠλὶ λεμὰ σαβαχθάνι;
\par }{\PP τοῦτ᾽ ἔστιν· Θεέ μου θεέ μου, ἵνατί με ἐγκατέλιπες;
\VS{47}Τινὲς δὲ τῶν ἐκεῖ ἑστηκότων ἀκούσαντες ἔλεγον ὅτι Ἠλίαν φωνεῖ οὗτος.
\VS{48}καὶ εὐθέως δραμὼν εἷς ἐξ αὐτῶν καὶ λαβὼν σπόγγον πλήσας τε ὄξους καὶ περιθεὶς καλάμῳ ἐπότιζεν αὐτόν.
\VS{49}Οἱ δὲ λοιποὶ ἔλεγον· Ἄφες ἴδωμεν εἰ ἔρχεται Ἠλίας σώσων αὐτόν.
\VS{50}Ὁ δὲ Ἰησοῦς πάλιν κράξας φωνῇ μεγάλῃ ἀφῆκεν τὸ πνεῦμα.
\par }{\PP \VS{51}Καὶ ἰδοὺ τὸ καταπέτασμα τοῦ ναοῦ ἐσχίσθη ἀπ᾽ ἄνωθεν ἕως κάτω εἰς δύο καὶ ἡ γῆ ἐσείσθη καὶ αἱ πέτραι ἐσχίσθησαν,
\VS{52}καὶ τὰ μνημεῖα ἀνεῴχθησαν καὶ πολλὰ σώματα τῶν κεκοιμημένων ἁγίων ἠγέρθησαν,
\VS{53}καὶ ἐξελθόντες ἐκ τῶν μνημείων μετὰ τὴν ἔγερσιν αὐτοῦ εἰσῆλθον εἰς τὴν ἁγίαν πόλιν καὶ ἐνεφανίσθησαν πολλοῖς.
\par }{\PP \VS{54}Ὁ δὲ ἑκατόνταρχος καὶ οἱ μετ᾽ αὐτοῦ τηροῦντες τὸν Ἰησοῦν ἰδόντες τὸν σεισμὸν καὶ τὰ γενόμενα ἐφοβήθησαν σφόδρα, λέγοντες· Ἀληθῶς Θεοῦ Υἱὸς ἦν οὗτος.
\par }{\PP \VS{55}Ἦσαν δὲ ἐκεῖ γυναῖκες πολλαὶ ἀπὸ μακρόθεν θεωροῦσαι, αἵτινες ἠκολούθησαν τῷ Ἰησοῦ ἀπὸ τῆς Γαλιλαίας διακονοῦσαι αὐτῷ·
\VS{56}ἐν αἷς ἦν Μαρία ἡ Μαγδαληνή καὶ Μαρία ἡ τοῦ Ἰακώβου καὶ Ἰωσὴφ μήτηρ καὶ ἡ μήτηρ τῶν υἱῶν Ζεβεδαίου.
\par }{\PP \VS{57}Ὀψίας δὲ γενομένης ἦλθεν ἄνθρωπος πλούσιος ἀπὸ Ἁριμαθαίας, τοὔνομα Ἰωσήφ, ὃς καὶ αὐτὸς ἐμαθητεύθη τῷ Ἰησοῦ·
\VS{58}οὗτος προσελθὼν τῷ Πιλάτῳ ᾐτήσατο τὸ σῶμα τοῦ Ἰησοῦ. τότε ὁ Πιλᾶτος ἐκέλευσεν ἀποδοθῆναι.
\VS{59}καὶ λαβὼν τὸ σῶμα ὁ Ἰωσὴφ ἐνετύλιξεν αὐτὸ ἐν σινδόνι καθαρᾷ
\VS{60}καὶ ἔθηκεν αὐτὸ ἐν τῷ καινῷ αὐτοῦ μνημείῳ ὃ ἐλατόμησεν ἐν τῇ πέτρᾳ καὶ προσκυλίσας λίθον μέγαν τῇ θύρᾳ τοῦ μνημείου ἀπῆλθεν.
\VS{61}Ἦν δὲ ἐκεῖ Μαριὰμ ἡ Μαγδαληνὴ καὶ ἡ ἄλλη Μαρία καθήμεναι ἀπέναντι τοῦ τάφου.
\par }{\PP \VS{62}Τῇ δὲ ἐπαύριον, ἥτις ἐστὶν μετὰ τὴν Παρασκευήν, συνήχθησαν οἱ ἀρχιερεῖς καὶ οἱ Φαρισαῖοι πρὸς Πιλᾶτον
\VS{63}λέγοντες· Κύριε, ἐμνήσθημεν ὅτι ἐκεῖνος ὁ πλάνος εἶπεν ἔτι ζῶν· Μετὰ τρεῖς ἡμέρας ἐγείρομαι.
\VS{64}κέλευσον οὖν ἀσφαλισθῆναι τὸν τάφον ἕως τῆς τρίτης ἡμέρας, μήποτε ἐλθόντες οἱ μαθηταὶ αὐτοῦ κλέψωσιν αὐτὸν καὶ εἴπωσιν τῷ λαῷ· Ἠγέρθη ἀπὸ τῶν νεκρῶν, καὶ ἔσται ἡ ἐσχάτη πλάνη χείρων τῆς πρώτης.
\VS{65}Ἔφη αὐτοῖς ὁ Πιλᾶτος· Ἔχετε κουστωδίαν· ὑπάγετε ἀσφαλίσασθε ὡς οἴδατε.
\VS{66}οἱ δὲ πορευθέντες ἠσφαλίσαντο τὸν τάφον σφραγίσαντες τὸν λίθον μετὰ τῆς κουστωδίας.

\par }\Chap{28}{\PP \VerseOne{1}Ὀψὲ δὲ σαββάτων, τῇ ἐπιφωσκούσῃ εἰς μίαν σαββάτων ἦλθεν Μαριὰμ ἡ Μαγδαληνὴ καὶ ἡ ἄλλη Μαρία θεωρῆσαι τὸν τάφον.
\VS{2}Καὶ ἰδοὺ σεισμὸς ἐγένετο μέγας· ἄγγελος γὰρ Κυρίου καταβὰς ἐξ οὐρανοῦ καὶ προσελθὼν ἀπεκύλισεν τὸν λίθον καὶ ἐκάθητο ἐπάνω αὐτοῦ.
\VS{3}ἦν δὲ ἡ εἰδέα αὐτοῦ ὡς ἀστραπὴ καὶ τὸ ἔνδυμα αὐτοῦ λευκὸν ὡς χιών.
\VS{4}ἀπὸ δὲ τοῦ φόβου αὐτοῦ ἐσείσθησαν οἱ τηροῦντες καὶ ἐγενήθησαν ὡς νεκροί.
\VS{5}Ἀποκριθεὶς δὲ ὁ ἄγγελος εἶπεν ταῖς γυναιξίν· Μὴ φοβεῖσθε ὑμεῖς, οἶδα γὰρ ὅτι Ἰησοῦν τὸν ἐσταυρωμένον ζητεῖτε·
\VS{6}οὐκ ἔστιν ὧδε, ἠγέρθη γὰρ καθὼς εἶπεν· δεῦτε ἴδετε τὸν τόπον ὅπου ἔκειτο.
\VS{7}καὶ ταχὺ πορευθεῖσαι εἴπατε τοῖς μαθηταῖς αὐτοῦ ὅτι Ἠγέρθη ἀπὸ τῶν νεκρῶν, καὶ ἰδοὺ προάγει ὑμᾶς εἰς τὴν Γαλιλαίαν, ἐκεῖ αὐτὸν ὄψεσθε· ἰδοὺ εἶπον ὑμῖν.
\par }{\PP \VS{8}Καὶ ἀπελθοῦσαι ταχὺ ἀπὸ τοῦ μνημείου μετὰ φόβου καὶ χαρᾶς μεγάλης ἔδραμον ἀπαγγεῖλαι τοῖς μαθηταῖς αὐτοῦ.
\VS{9}καὶ ἰδοὺ Ἰησοῦς ὑπήντησεν αὐταῖς λέγων· Χαίρετε. αἱ δὲ προσελθοῦσαι ἐκράτησαν αὐτοῦ τοὺς πόδας καὶ προσεκύνησαν αὐτῷ.
\VS{10}τότε λέγει αὐταῖς ὁ Ἰησοῦς· Μὴ φοβεῖσθε· ὑπάγετε ἀπαγγείλατε τοῖς ἀδελφοῖς μου ἵνα ἀπέλθωσιν εἰς τὴν Γαλιλαίαν, κἀκεῖ με ὄψονται.
\par }{\PP \VS{11}Πορευομένων δὲ αὐτῶν ἰδού τινες τῆς κουστωδίας ἐλθόντες εἰς τὴν πόλιν ἀπήγγειλαν τοῖς ἀρχιερεῦσιν ἅπαντα τὰ γενόμενα.
\VS{12}καὶ συναχθέντες μετὰ τῶν πρεσβυτέρων συμβούλιόν τε λαβόντες ἀργύρια ἱκανὰ ἔδωκαν τοῖς στρατιώταις
\VS{13}λέγοντες· Εἴπατε ὅτι Οἱ μαθηταὶ αὐτοῦ νυκτὸς ἐλθόντες ἔκλεψαν αὐτὸν ἡμῶν κοιμωμένων.
\VS{14}καὶ ἐὰν ἀκουσθῇ τοῦτο ἐπὶ τοῦ ἡγεμόνος, ἡμεῖς πείσομεν αὐτὸν καὶ ὑμᾶς ἀμερίμνους ποιήσομεν.
\VS{15}Οἱ δὲ λαβόντες τὰ ἀργύρια ἐποίησαν ὡς ἐδιδάχθησαν. Καὶ διεφημίσθη ὁ λόγος οὗτος παρὰ Ἰουδαίοις μέχρι τῆς σήμερον ἡμέρας.
\par }{\PP \VS{16}Οἱ δὲ ἕνδεκα μαθηταὶ ἐπορεύθησαν εἰς τὴν Γαλιλαίαν εἰς τὸ ὄρος οὗ ἐτάξατο αὐτοῖς ὁ Ἰησοῦς,
\VS{17}καὶ ἰδόντες αὐτὸν προσεκύνησαν, οἱ δὲ ἐδίστασαν.
\VS{18}Καὶ προσελθὼν ὁ Ἰησοῦς ἐλάλησεν αὐτοῖς λέγων· Ἐδόθη μοι πᾶσα ἐξουσία ἐν οὐρανῷ καὶ ἐπὶ τῆς γῆς.
\VS{19}πορευθέντες οὖν μαθητεύσατε πάντα τὰ ἔθνη, βαπτίζοντες αὐτοὺς εἰς τὸ ὄνομα τοῦ Πατρὸς καὶ τοῦ Υἱοῦ καὶ τοῦ Ἁγίου Πνεύματος,
\VS{20}διδάσκοντες αὐτοὺς τηρεῖν πάντα ὅσα ἐνετειλάμην ὑμῖν· καὶ ἰδοὺ ἐγὼ μεθ᾽ ὑμῶν εἰμι πάσας τὰς ἡμέρας ἕως τῆς συντελείας τοῦ αἰῶνος.
\par }