\NormalFont\ShortTitle{ΠΡΟΣ ΤΙΤΟΝ}
{\MT ΠΡΟΣ ΤΙΤΟΝ

\par }\ChapOne{1}{\PP \VerseOne{1}Παῦλος δοῦλος Θεοῦ, ἀπόστολος δὲ Ἰησοῦ Χριστοῦ κατὰ πίστιν ἐκλεκτῶν Θεοῦ καὶ ἐπίγνωσιν ἀληθείας τῆς κατ᾽ εὐσέβειαν
\VS{2}ἐπ᾽ ἐλπίδι ζωῆς αἰωνίου, ἣν ἐπηγγείλατο ὁ ἀψευδὴς Θεὸς πρὸ χρόνων αἰωνίων,
\VS{3}ἐφανέρωσεν δὲ καιροῖς ἰδίοις τὸν λόγον αὐτοῦ ἐν κηρύγματι, ὃ ἐπιστεύθην ἐγὼ κατ᾽ ἐπιταγὴν τοῦ Σωτῆρος ἡμῶν Θεοῦ,
\VS{4}Τίτῳ γνησίῳ τέκνῳ κατὰ κοινὴν πίστιν, Χάρις καὶ εἰρήνη ἀπὸ Θεοῦ Πατρὸς καὶ Χριστοῦ Ἰησοῦ τοῦ Σωτῆρος ἡμῶν.
\par }{\PP \VS{5}Τούτου χάριν ἀπέλιπόν σε ἐν Κρήτῃ, ἵνα τὰ λείποντα ἐπιδιορθώσῃ καὶ καταστήσῃς κατὰ πόλιν πρεσβυτέρους, ὡς ἐγώ σοι διεταξάμην,
\VS{6}εἴ τίς ἐστιν ἀνέγκλητος, μιᾶς γυναικὸς ἀνήρ, τέκνα ἔχων πιστά, μὴ ἐν κατηγορίᾳ ἀσωτίας ἢ ἀνυπότακτα.
\VS{7}Δεῖ γὰρ τὸν ἐπίσκοπον ἀνέγκλητον εἶναι ὡς Θεοῦ οἰκονόμον, μὴ αὐθάδη, μὴ ὀργίλον, μὴ πάροινον, μὴ πλήκτην, μὴ αἰσχροκερδῆ,
\VS{8}ἀλλὰ φιλόξενον φιλάγαθον σώφρονα δίκαιον ὅσιον ἐγκρατῆ,
\VS{9}ἀντεχόμενον τοῦ κατὰ τὴν διδαχὴν πιστοῦ λόγου, ἵνα δυνατὸς ᾖ καὶ παρακαλεῖν ἐν τῇ διδασκαλίᾳ τῇ ὑγιαινούσῃ καὶ τοὺς ἀντιλέγοντας ἐλέγχειν.
\VS{10}Εἰσὶν γὰρ πολλοὶ καὶ ἀνυπότακτοι, ματαιολόγοι καὶ φρεναπάται, μάλιστα οἱ ἐκ τῆς περιτομῆς,
\VS{11}οὓς δεῖ ἐπιστομίζειν, οἵτινες ὅλους οἴκους ἀνατρέπουσιν διδάσκοντες ἃ μὴ δεῖ αἰσχροῦ κέρδους χάριν.
\VS{12}εἶπέν τις ἐξ αὐτῶν ἴδιος αὐτῶν προφήτης· 
\begin{poetryblock}
\par }{\PP \begin{quote}Κρῆτες ἀεὶ ψεῦσται, κακὰ θηρία, γαστέρες ἀργαί.\end{quote}
\end{poetryblock}
\par }{\PP \VS{13}Ἡ μαρτυρία αὕτη ἐστὶν ἀληθής. δι᾽ ἣν αἰτίαν ἔλεγχε αὐτοὺς ἀποτόμως, ἵνα ὑγιαίνωσιν ἐν τῇ πίστει,
\VS{14}μὴ προσέχοντες Ἰουδαϊκοῖς μύθοις καὶ ἐντολαῖς ἀνθρώπων ἀποστρεφομένων τὴν ἀλήθειαν.
\VS{15}Πάντα καθαρὰ τοῖς καθαροῖς· τοῖς δὲ μεμιαμμένοις καὶ ἀπίστοις οὐδὲν καθαρόν, ἀλλὰ μεμίανται αὐτῶν καὶ ὁ νοῦς καὶ ἡ συνείδησις.
\VS{16}Θεὸν ὁμολογοῦσιν εἰδέναι, τοῖς δὲ ἔργοις ἀρνοῦνται, βδελυκτοὶ ὄντες καὶ ἀπειθεῖς καὶ πρὸς πᾶν ἔργον ἀγαθὸν ἀδόκιμοι.

\par }\Chap{2}{\PP \VerseOne{1}Σὺ δὲ λάλει ἃ πρέπει τῇ ὑγιαινούσῃ διδασκαλίᾳ.
\VS{2}Πρεσβύτας νηφαλίους εἶναι, σεμνούς, σώφρονας, ὑγιαίνοντας τῇ πίστει, τῇ ἀγάπῃ, τῇ ὑπομονῇ·
\VS{3}πρεσβύτιδας ὡσαύτως ἐν καταστήματι ἱεροπρεπεῖς, μὴ διαβόλους μηδὲ οἴνῳ πολλῷ δεδουλωμένας, καλοδιδασκάλους,
\VS{4}ἵνα σωφρονίζωσιν τὰς νέας φιλάνδρους εἶναι, φιλοτέκνους
\VS{5}σώφρονας ἁγνάς οἰκουργούς ἀγαθάς, ὑποτασσομένας τοῖς ἰδίοις ἀνδράσιν, ἵνα μὴ ὁ λόγος τοῦ Θεοῦ βλασφημῆται.
\par }{\PP \VS{6}Τοὺς νεωτέρους ὡσαύτως παρακάλει σωφρονεῖν
\VS{7}Περὶ πάντα, σεαυτὸν παρεχόμενος τύπον καλῶν ἔργων, ἐν τῇ διδασκαλίᾳ ἀφθορίαν, σεμνότητα,
\VS{8}λόγον ὑγιῆ ἀκατάγνωστον, ἵνα ὁ ἐξ ἐναντίας ἐντραπῇ μηδὲν ἔχων λέγειν περὶ ἡμῶν φαῦλον.
\par }{\PP \VS{9}Δούλους ἰδίοις δεσπόταις ὑποτάσσεσθαι ἐν πᾶσιν, εὐαρέστους εἶναι, μὴ ἀντιλέγοντας,
\VS{10}μὴ νοσφιζομένους, ἀλλὰ πᾶσαν πίστιν ἐνδεικνυμένους ἀγαθήν, ἵνα τὴν διδασκαλίαν τὴν τοῦ Σωτῆρος ἡμῶν Θεοῦ κοσμῶσιν ἐν πᾶσιν.
\par }{\PP \VS{11}Ἐπεφάνη γὰρ ἡ χάρις τοῦ Θεοῦ σωτήριος πᾶσιν ἀνθρώποις
\VS{12}παιδεύουσα ἡμᾶς, ἵνα ἀρνησάμενοι τὴν ἀσέβειαν καὶ τὰς κοσμικὰς ἐπιθυμίας σωφρόνως καὶ δικαίως καὶ εὐσεβῶς ζήσωμεν ἐν τῷ νῦν αἰῶνι,
\VS{13}προσδεχόμενοι τὴν μακαρίαν ἐλπίδα καὶ ἐπιφάνειαν τῆς δόξης τοῦ μεγάλου Θεοῦ καὶ Σωτῆρος ἡμῶν Ἰησοῦ Χριστοῦ,
\VS{14}ὃς ἔδωκεν ἑαυτὸν ὑπὲρ ἡμῶν, ἵνα λυτρώσηται ἡμᾶς ἀπὸ πάσης ἀνομίας καὶ καθαρίσῃ ἑαυτῷ λαὸν περιούσιον, ζηλωτὴν καλῶν ἔργων.
\VS{15}Ταῦτα λάλει καὶ παρακάλει καὶ ἔλεγχε μετὰ πάσης ἐπιταγῆς· μηδείς σου περιφρονείτω.

\par }\Chap{3}{\PP \VerseOne{1}Ὑπομίμνῃσκε αὐτοὺς ἀρχαῖς ἐξουσίαις ὑποτάσσεσθαι, πειθαρχεῖν, πρὸς πᾶν ἔργον ἀγαθὸν ἑτοίμους εἶναι,
\VS{2}μηδένα βλασφημεῖν, ἀμάχους εἶναι, ἐπιεικεῖς, πᾶσαν ἐνδεικνυμένους πραΰτητα πρὸς πάντας ἀνθρώπους.
\VS{3}Ἦμεν γάρ ποτε καὶ ἡμεῖς ἀνόητοι, ἀπειθεῖς, πλανώμενοι, δουλεύοντες ἐπιθυμίαις καὶ ἡδοναῖς ποικίλαις, ἐν κακίᾳ καὶ φθόνῳ διάγοντες, στυγητοί, μισοῦντες ἀλλήλους.
\par }{\PP \VS{4}Ὅτε δὲ ἡ χρηστότης καὶ ἡ φιλανθρωπία ἐπεφάνη 
\begin{poetryblock}
\par }{\PP \begin{quote}τοῦ Σωτῆρος ἡμῶν Θεοῦ,\end{quote}
\par }{\PP \begin{quote} \VS{5}οὐκ ἐξ ἔργων τῶν ἐν δικαιοσύνῃ\end{quote} 
\par }{\PP \begin{quote}ἃ ἐποιήσαμεν ἡμεῖς\end{quote} 
\par }{\PP \begin{quote}ἀλλὰ κατὰ τὸ αὐτοῦ ἔλεος\end{quote} 
\par }{\PP \begin{quote}ἔσωσεν ἡμᾶς διὰ λουτροῦ παλινγενεσίας\end{quote} 
\par }{\PP \begin{quote}καὶ ἀνακαινώσεως Πνεύματος Ἁγίου,\end{quote}
\par }{\PP \begin{quote} \VS{6}οὗ ἐξέχεεν ἐφ᾽ ἡμᾶς πλουσίως\end{quote} 
\par }{\PP \begin{quote}διὰ Ἰησοῦ Χριστοῦ τοῦ Σωτῆρος ἡμῶν,\end{quote}
\par }{\PP \begin{quote} \VS{7}ἵνα δικαιωθέντες τῇ ἐκείνου χάριτι\end{quote} 
\par }{\PP \begin{quote}κληρονόμοι γενηθῶμεν κατ᾽ ἐλπίδα ζωῆς αἰωνίου.\end{quote}
\end{poetryblock}
\par }{\PP \VS{8}Πιστὸς ὁ λόγος· καὶ περὶ τούτων βούλομαί σε διαβεβαιοῦσθαι, ἵνα φροντίζωσιν καλῶν ἔργων προΐστασθαι οἱ πεπιστευκότες Θεῷ· ταῦτά ἐστιν καλὰ καὶ ὠφέλιμα τοῖς ἀνθρώποις.
\VS{9}Μωρὰς δὲ ζητήσεις καὶ γενεαλογίας καὶ ἔρεις καὶ μάχας νομικὰς περιΐστασο· εἰσὶν γὰρ ἀνωφελεῖς καὶ μάταιοι.
\VS{10}αἱρετικὸν ἄνθρωπον μετὰ μίαν καὶ δευτέραν νουθεσίαν παραιτοῦ,
\VS{11}εἰδὼς ὅτι ἐξέστραπται ὁ τοιοῦτος καὶ ἁμαρτάνει ὢν αὐτοκατάκριτος.
\par }{\PP \VS{12}Ὅταν πέμψω Ἀρτεμᾶν πρὸς σὲ ἢ Τυχικόν, σπούδασον ἐλθεῖν πρός με εἰς Νικόπολιν, ἐκεῖ γὰρ κέκρικα παραχειμάσαι.
\VS{13}Ζηνᾶν τὸν νομικὸν καὶ Ἀπολλῶν σπουδαίως πρόπεμψον, ἵνα μηδὲν αὐτοῖς λείπῃ.
\VS{14}μανθανέτωσαν δὲ καὶ οἱ ἡμέτεροι καλῶν ἔργων προΐστασθαι εἰς τὰς ἀναγκαίας χρείας, ἵνα μὴ ὦσιν ἄκαρποι.
\par }{\PP \VS{15}Ἀσπάζονταί σε οἱ μετ᾽ ἐμοῦ πάντες. Ἄσπασαι τοὺς φιλοῦντας ἡμᾶς ἐν πίστει.
\par }{\PP Ἡ χάρις μετὰ πάντων ὑμῶν.
\par }