\NormalFont\ShortTitle{ΠΡΟΣ ΚΟΡΙΝΘΙΟΥΣ Β}
{\MT ΠΡΟΣ ΚΟΡΙΝΘΙΟΥΣ Β

\par }\ChapOne{1}{\PP \VerseOne{1}Παῦλος ἀπόστολος Χριστοῦ Ἰησοῦ διὰ θελήματος Θεοῦ καὶ Τιμόθεος ὁ ἀδελφὸς Τῇ ἐκκλησίᾳ τοῦ Θεοῦ τῇ οὔσῃ ἐν Κορίνθῳ σὺν τοῖς ἁγίοις πᾶσιν τοῖς οὖσιν ἐν ὅλῃ τῇ Ἀχαΐᾳ,
\par }{\PP \VS{2}Χάρις ὑμῖν καὶ εἰρήνη ἀπὸ Θεοῦ Πατρὸς ἡμῶν καὶ Κυρίου Ἰησοῦ Χριστοῦ.
\VS{3}Εὐλογητὸς ὁ Θεὸς καὶ Πατὴρ τοῦ Κυρίου ἡμῶν Ἰησοῦ Χριστοῦ, ὁ Πατὴρ τῶν οἰκτιρμῶν καὶ Θεὸς πάσης παρακλήσεως,
\VS{4}ὁ παρακαλῶν ἡμᾶς ἐπὶ πάσῃ τῇ θλίψει ἡμῶν εἰς τὸ δύνασθαι ἡμᾶς παρακαλεῖν τοὺς ἐν πάσῃ θλίψει διὰ τῆς παρακλήσεως ἧς παρακαλούμεθα αὐτοὶ ὑπὸ τοῦ Θεοῦ.
\VS{5}ὅτι καθὼς περισσεύει τὰ παθήματα τοῦ Χριστοῦ εἰς ἡμᾶς, οὕτως διὰ τοῦ Χριστοῦ περισσεύει καὶ ἡ παράκλησις ἡμῶν.
\VS{6}Εἴτε δὲ θλιβόμεθα, ὑπὲρ τῆς ὑμῶν παρακλήσεως καὶ σωτηρίας· εἴτε παρακαλούμεθα, ὑπὲρ τῆς ὑμῶν παρακλήσεως τῆς ἐνεργουμένης ἐν ὑπομονῇ τῶν αὐτῶν παθημάτων ὧν καὶ ἡμεῖς πάσχομεν.
\par }{\PP \VS{7}καὶ ἡ ἐλπὶς ἡμῶν βεβαία ὑπὲρ ὑμῶν εἰδότες ὅτι ὡς κοινωνοί ἐστε τῶν παθημάτων, οὕτως καὶ τῆς παρακλήσεως.
\VS{8}Οὐ γὰρ θέλομεν ὑμᾶς ἀγνοεῖν, ἀδελφοί, ὑπὲρ τῆς θλίψεως ἡμῶν τῆς γενομένης ἐν τῇ Ἀσίᾳ, ὅτι καθ᾽ ὑπερβολὴν ὑπὲρ δύναμιν ἐβαρήθημεν ὥστε ἐξαπορηθῆναι ἡμᾶς καὶ τοῦ ζῆν·
\VS{9}ἀλλὰ= αὐτοὶ ἐν ἑαυτοῖς τὸ ἀπόκριμα τοῦ θανάτου ἐσχήκαμεν, ἵνα μὴ πεποιθότες ὦμεν ἐφ᾽ ἑαυτοῖς ἀλλ᾽ ἐπὶ τῷ Θεῷ τῷ ἐγείροντι τοὺς νεκρούς·
\VS{10}ὃς ἐκ τηλικούτου θανάτου ἐρρύσατο ἡμᾶς καὶ ῥύσεται, εἰς ὃν ἠλπίκαμεν ὅτι καὶ ἔτι ῥύσεται,
\par }{\PP \VS{11}συνυπουργούντων καὶ ὑμῶν ὑπὲρ ἡμῶν τῇ δεήσει, ἵνα ἐκ πολλῶν προσώπων τὸ εἰς ἡμᾶς χάρισμα διὰ πολλῶν εὐχαριστηθῇ ὑπὲρ ἡμῶν.
\VS{12}Ἡ γὰρ καύχησις ἡμῶν αὕτη ἐστίν, τὸ μαρτύριον τῆς συνειδήσεως ἡμῶν, ὅτι ἐν ἁγιότητι* καὶ εἰλικρινείᾳ τοῦ Θεοῦ, καὶ οὐκ ἐν σοφίᾳ σαρκικῇ ἀλλ᾽ ἐν χάριτι Θεοῦ, ἀνεστράφημεν ἐν τῷ κόσμῳ, περισσοτέρως δὲ πρὸς ὑμᾶς.
\VS{13}οὐ γὰρ ἄλλα γράφομεν ὑμῖν ἀλλ᾽ ἢ ἃ ἀναγινώσκετε ἢ καὶ ἐπιγινώσκετε· ἐλπίζω δὲ ὅτι ἕως τέλους ἐπιγνώσεσθε,
\par }{\PP \VS{14}καθὼς καὶ ἐπέγνωτε ἡμᾶς ἀπὸ μέρους, ὅτι καύχημα ὑμῶν ἐσμεν καθάπερ καὶ ὑμεῖς ἡμῶν ἐν τῇ ἡμέρᾳ τοῦ Κυρίου ἡμῶν Ἰησοῦ.
\VS{15}Καὶ ταύτῃ τῇ πεποιθήσει ἐβουλόμην πρότερον πρὸς ὑμᾶς ἐλθεῖν, ἵνα δευτέραν χάριν σχῆτε,
\VS{16}καὶ δι᾽ ὑμῶν διελθεῖν εἰς Μακεδονίαν καὶ πάλιν ἀπὸ Μακεδονίας ἐλθεῖν πρὸς ὑμᾶς καὶ ὑφ᾽ ὑμῶν προπεμφθῆναι εἰς τὴν Ἰουδαίαν.
\VS{17}Τοῦτο οὖν βουλόμενος μήτι ἄρα τῇ ἐλαφρίᾳ ἐχρησάμην; ἢ ἃ βουλεύομαι κατὰ σάρκα βουλεύομαι, ἵνα ᾖ παρ᾽ ἐμοὶ τό Ναί ναὶ καὶ τὸ Οὔ οὔ;
\VS{18}πιστὸς δὲ ὁ Θεὸς ὅτι ὁ λόγος ἡμῶν ὁ πρὸς ὑμᾶς οὐκ ἔστιν Ναί καὶ Οὔ.
\VS{19}ὁ τοῦ Θεοῦ γὰρ Υἱὸς Ἰησοῦς Χριστὸς ὁ ἐν ὑμῖν δι᾽ ἡμῶν κηρυχθείς, δι᾽ ἐμοῦ καὶ Σιλουανοῦ καὶ Τιμοθέου, οὐκ ἐγένετο Ναί καὶ Οὔ ἀλλὰ Ναί ἐν αὐτῷ γέγονεν.
\VS{20}ὅσαι γὰρ ἐπαγγελίαι Θεοῦ, ἐν αὐτῷ τὸ Ναί· διὸ καὶ δι᾽ αὐτοῦ τὸ Ἀμὴν τῷ Θεῷ πρὸς δόξαν δι᾽ ἡμῶν.
\VS{21}Ὁ δὲ βεβαιῶν ἡμᾶς σὺν ὑμῖν εἰς Χριστὸν καὶ χρίσας ἡμᾶς Θεός,
\par }{\PP \VS{22}ὁ καὶ σφραγισάμενος ἡμᾶς καὶ δοὺς τὸν ἀρραβῶνα τοῦ Πνεύματος ἐν ταῖς καρδίαις ἡμῶν.
\VS{23}Ἐγὼ δὲ μάρτυρα τὸν Θεὸν ἐπικαλοῦμαι ἐπὶ τὴν ἐμὴν ψυχήν, ὅτι φειδόμενος ὑμῶν οὐκέτι ἦλθον εἰς Κόρινθον.
\VS{24}οὐχ ὅτι κυριεύομεν ὑμῶν τῆς πίστεως ἀλλὰ συνεργοί ἐσμεν τῆς χαρᾶς ὑμῶν· τῇ γὰρ πίστει ἑστήκατε.

\par }\Chap{2}{\PP \VerseOne{1}Ἔκρινα γὰρ ἐμαυτῷ τοῦτο τὸ μὴ πάλιν ἐν λύπῃ πρὸς ὑμᾶς ἐλθεῖν.
\VS{2}εἰ γὰρ ἐγὼ λυπῶ ὑμᾶς, καὶ τίς ὁ εὐφραίνων με εἰ μὴ ὁ λυπούμενος ἐξ ἐμοῦ;
\VS{3}καὶ ἔγραψα τοῦτο αὐτὸ, ἵνα μὴ ἐλθὼν λύπην σχῶ ἀφ᾽ ὧν ἔδει με χαίρειν, πεποιθὼς ἐπὶ πάντας ὑμᾶς ὅτι ἡ ἐμὴ χαρὰ πάντων ὑμῶν ἐστιν.
\par }{\PP \VS{4}ἐκ γὰρ πολλῆς θλίψεως καὶ συνοχῆς καρδίας ἔγραψα ὑμῖν διὰ πολλῶν δακρύων, οὐχ ἵνα λυπηθῆτε ἀλλὰ τὴν ἀγάπην ἵνα γνῶτε ἣν ἔχω περισσοτέρως εἰς ὑμᾶς.
\VS{5}Εἰ δέ τις λελύπηκεν, οὐκ ἐμὲ λελύπηκεν, ἀλλὰ= ἀπὸ μέρους, ἵνα μὴ ἐπιβαρῶ, πάντας ὑμᾶς.
\VS{6}ἱκανὸν τῷ τοιούτῳ ἡ ἐπιτιμία αὕτη ἡ ὑπὸ τῶν πλειόνων,
\VS{7}ὥστε τοὐναντίον μᾶλλον ὑμᾶς χαρίσασθαι καὶ παρακαλέσαι, μή πως τῇ περισσοτέρᾳ λύπῃ καταποθῇ ὁ τοιοῦτος.
\VS{8}διὸ παρακαλῶ ὑμᾶς κυρῶσαι εἰς αὐτὸν ἀγάπην·
\VS{9}Εἰς τοῦτο γὰρ καὶ ἔγραψα, ἵνα γνῶ τὴν δοκιμὴν ὑμῶν, εἰ εἰς πάντα ὑπήκοοί ἐστε.
\VS{10}ᾧ δέ τι χαρίζεσθε, κἀγώ· καὶ γὰρ ἐγὼ ὃ κεχάρισμαι, εἴ τι κεχάρισμαι, δι᾽ ὑμᾶς ἐν προσώπῳ Χριστοῦ,
\par }{\PP \VS{11}ἵνα μὴ πλεονεκτηθῶμεν ὑπὸ τοῦ Σατανᾶ· οὐ γὰρ αὐτοῦ τὰ νοήματα ἀγνοοῦμεν.
\VS{12}Ἐλθὼν δὲ εἰς τὴν Τρῳάδα εἰς τὸ εὐαγγέλιον τοῦ Χριστοῦ καὶ θύρας μοι ἀνεῳγμένης ἐν Κυρίῳ,
\par }{\PP \VS{13}οὐκ ἔσχηκα ἄνεσιν τῷ πνεύματί μου τῷ μὴ εὑρεῖν με Τίτον τὸν ἀδελφόν μου, ἀλλὰ= ἀποταξάμενος αὐτοῖς ἐξῆλθον εἰς Μακεδονίαν.
\VS{14}Τῷ δὲ Θεῷ χάρις τῷ πάντοτε θριαμβεύοντι ἡμᾶς ἐν τῷ Χριστῷ καὶ τὴν ὀσμὴν τῆς γνώσεως αὐτοῦ φανεροῦντι δι᾽ ἡμῶν ἐν παντὶ τόπῳ·
\VS{15}ὅτι Χριστοῦ εὐωδία ἐσμὲν τῷ Θεῷ ἐν τοῖς σωζομένοις καὶ ἐν τοῖς ἀπολλυμένοις,
\VS{16}οἷς μὲν ὀσμὴ ἐκ θανάτου εἰς θάνατον, οἷς δὲ ὀσμὴ ἐκ ζωῆς εἰς ζωήν. καὶ πρὸς ταῦτα τίς ἱκανός;
\par }{\PP \VS{17}Οὐ γάρ ἐσμεν ὡς οἱ πολλοὶ καπηλεύοντες τὸν λόγον τοῦ Θεοῦ, ἀλλ᾽ ὡς ἐξ εἰλικρινείας, ἀλλ᾽ ὡς ἐκ Θεοῦ κατέναντι Θεοῦ ἐν Χριστῷ λαλοῦμεν.

\par }\Chap{3}{\PP \VerseOne{1}Ἀρχόμεθα πάλιν ἑαυτοὺς συνιστάνειν; ἢ μὴ χρῄζομεν ὥς τινες συστατικῶν ἐπιστολῶν πρὸς ὑμᾶς ἢ ἐξ ὑμῶν;
\VS{2}ἡ ἐπιστολὴ ἡμῶν ὑμεῖς ἐστε, ἐνγεγραμμένη= ἐν ταῖς καρδίαις ἡμῶν, γινωσκομένη καὶ ἀναγινωσκομένη ὑπὸ πάντων ἀνθρώπων,
\par }{\PP \VS{3}φανερούμενοι ὅτι ἐστὲ ἐπιστολὴ Χριστοῦ διακονηθεῖσα ὑφ᾽ ἡμῶν, ἐνγεγραμμένη= οὐ μέλανι ἀλλὰ Πνεύματι Θεοῦ ζῶντος, οὐκ ἐν πλαξὶν λιθίναις ἀλλ᾽ ἐν πλαξὶν καρδίαις σαρκίναις.
\VS{4}Πεποίθησιν δὲ τοιαύτην ἔχομεν διὰ τοῦ Χριστοῦ πρὸς τὸν Θεόν.
\VS{5}οὐχ ὅτι ἀφ᾽ ἑαυτῶν ἱκανοί ἐσμεν λογίσασθαί τι ὡς ἐξ ἑαυτῶν, ἀλλ᾽ ἡ ἱκανότης ἡμῶν ἐκ τοῦ Θεοῦ,
\VS{6}ὃς καὶ ἱκάνωσεν ἡμᾶς διακόνους καινῆς διαθήκης, οὐ γράμματος ἀλλὰ πνεύματος· τὸ γὰρ γράμμα ἀποκτέννει, τὸ δὲ πνεῦμα ζωοποιεῖ.
\VS{7}Εἰ δὲ ἡ διακονία τοῦ θανάτου ἐν γράμμασιν ἐντετυπωμένη λίθοις ἐγενήθη ἐν δόξῃ, ὥστε μὴ δύνασθαι ἀτενίσαι τοὺς υἱοὺς Ἰσραὴλ εἰς τὸ πρόσωπον Μωϋσέως διὰ τὴν δόξαν τοῦ προσώπου αὐτοῦ τὴν καταργουμένην,
\VS{8}πῶς οὐχὶ μᾶλλον ἡ διακονία τοῦ πνεύματος ἔσται ἐν δόξῃ;
\VS{9}εἰ γὰρ τῇ διακονία τῆς κατακρίσεως δόξα, πολλῷ μᾶλλον περισσεύει ἡ διακονία τῆς δικαιοσύνης δόξῃ.
\VS{10}καὶ γὰρ οὐ δεδόξασται τὸ δεδοξασμένον ἐν τούτῳ τῷ μέρει εἵνεκεν τῆς ὑπερβαλλούσης δόξης.
\par }{\PP \VS{11}εἰ γὰρ τὸ καταργούμενον διὰ δόξης, πολλῷ μᾶλλον τὸ μένον ἐν δόξῃ.
\VS{12}Ἔχοντες οὖν τοιαύτην ἐλπίδα πολλῇ παρρησίᾳ χρώμεθα
\VS{13}καὶ οὐ καθάπερ Μωϋσῆς ἐτίθει κάλυμμα ἐπὶ τὸ πρόσωπον αὐτοῦ πρὸς τὸ μὴ ἀτενίσαι τοὺς υἱοὺς Ἰσραὴλ εἰς τὸ τέλος τοῦ καταργουμένου.
\VS{14}Ἀλλὰ= ἐπωρώθη τὰ νοήματα αὐτῶν. ἄχρι γὰρ τῆς σήμερον ἡμέρας τὸ αὐτὸ κάλυμμα ἐπὶ τῇ ἀναγνώσει τῆς παλαιᾶς διαθήκης μένει, μὴ ἀνακαλυπτόμενον ὅτι ἐν Χριστῷ καταργεῖται·
\VS{15}ἀλλ᾽ ἕως σήμερον ἡνίκα ἂν ἀναγινώσκηται Μωϋσῆς, κάλυμμα ἐπὶ τὴν καρδίαν αὐτῶν κεῖται·
\VS{16}ἡνίκα δὲ ἐὰν ἐπιστρέψῃ πρὸς Κύριον, περιαιρεῖται τὸ κάλυμμα.
\VS{17}Ὁ δὲ Κύριος τὸ Πνεῦμά ἐστιν· οὗ δὲ τὸ Πνεῦμα Κυρίου, ἐλευθερία.
\par }{\PP \VS{18}ἡμεῖς δὲ πάντες ἀνακεκαλυμμένῳ προσώπῳ τὴν δόξαν Κυρίου κατοπτριζόμενοι τὴν αὐτὴν εἰκόνα μεταμορφούμεθα ἀπὸ δόξης εἰς δόξαν καθάπερ ἀπὸ Κυρίου Πνεύματος.

\par }\Chap{4}{\PP \VerseOne{1}Διὰ τοῦτο, ἔχοντες τὴν διακονίαν ταύτην καθὼς ἠλεήθημεν, οὐκ ἐγκακοῦμεν
\VS{2}ἀλλὰ= ἀπειπάμεθα τὰ κρυπτὰ τῆς αἰσχύνης, μὴ περιπατοῦντες ἐν πανουργίᾳ μηδὲ δολοῦντες τὸν λόγον τοῦ Θεοῦ ἀλλὰ τῇ φανερώσει τῆς ἀληθείας συνιστάνοντες ἑαυτοὺς πρὸς πᾶσαν συνείδησιν ἀνθρώπων ἐνώπιον τοῦ Θεοῦ.
\VS{3}Εἰ δὲ καὶ ἔστιν κεκαλυμμένον τὸ εὐαγγέλιον ἡμῶν, ἐν τοῖς ἀπολλυμένοις ἐστὶν κεκαλυμμένον,
\VS{4}ἐν οἷς ὁ θεὸς τοῦ αἰῶνος τούτου ἐτύφλωσεν τὰ νοήματα τῶν ἀπίστων εἰς τὸ μὴ αὐγάσαι τὸν φωτισμὸν τοῦ εὐαγγελίου τῆς δόξης τοῦ Χριστοῦ, ὅς ἐστιν εἰκὼν τοῦ Θεοῦ.
\VS{5}Οὐ γὰρ ἑαυτοὺς κηρύσσομεν ἀλλὰ= Ἰησοῦν Χριστὸν Κύριον, ἑαυτοὺς δὲ δούλους ὑμῶν διὰ Ἰησοῦν.
\par }{\PP \VS{6}ὅτι ὁ Θεὸς ὁ εἰπών· Ἐκ σκότους φῶς λάμψει, ὃς ἔλαμψεν ἐν ταῖς καρδίαις ἡμῶν πρὸς φωτισμὸν τῆς γνώσεως τῆς δόξης τοῦ Θεοῦ ἐν προσώπῳ Ἰησοῦ Χριστοῦ.
\VS{7}Ἔχομεν δὲ τὸν θησαυρὸν τοῦτον ἐν ὀστρακίνοις σκεύεσιν, ἵνα ἡ ὑπερβολὴ τῆς δυνάμεως ᾖ τοῦ Θεοῦ καὶ μὴ ἐξ ἡμῶν·
\VS{8}ἐν παντὶ θλιβόμενοι ἀλλ᾽ οὐ στενοχωρούμενοι, ἀπορούμενοι ἀλλ᾽ οὐκ ἐξαπορούμενοι,
\VS{9}διωκόμενοι ἀλλ᾽ οὐκ ἐγκαταλειπόμενοι, καταβαλλόμενοι ἀλλ᾽ οὐκ ἀπολλύμενοι,
\VS{10}πάντοτε τὴν νέκρωσιν τοῦ Ἰησοῦ ἐν τῷ σώματι περιφέροντες, ἵνα καὶ ἡ ζωὴ τοῦ Ἰησοῦ ἐν τῷ σώματι ἡμῶν φανερωθῇ.
\VS{11}ἀεὶ γὰρ ἡμεῖς οἱ ζῶντες εἰς θάνατον παραδιδόμεθα διὰ Ἰησοῦν, ἵνα καὶ ἡ ζωὴ τοῦ Ἰησοῦ φανερωθῇ ἐν τῇ θνητῇ σαρκὶ ἡμῶν.
\VS{12}ὥστε ὁ θάνατος ἐν ἡμῖν ἐνεργεῖται, ἡ δὲ ζωὴ ἐν ὑμῖν.
\VS{13}Ἔχοντες δὲ τὸ αὐτὸ πνεῦμα τῆς πίστεως κατὰ τὸ γεγραμμένον· Ἐπίστευσα, διὸ ἐλάλησα, καὶ ἡμεῖς πιστεύομεν, διὸ καὶ λαλοῦμεν,
\VS{14}εἰδότες ὅτι ὁ ἐγείρας τὸν Κύριον Ἰησοῦν καὶ ἡμᾶς σὺν Ἰησοῦ ἐγερεῖ καὶ παραστήσει σὺν ὑμῖν.
\par }{\PP \VS{15}τὰ γὰρ πάντα δι᾽ ὑμᾶς, ἵνα ἡ χάρις πλεονάσασα διὰ τῶν πλειόνων τὴν εὐχαριστίαν περισσεύσῃ εἰς τὴν δόξαν τοῦ Θεοῦ.
\VS{16}Διὸ οὐκ ἐγκακοῦμεν, ἀλλ᾽ εἰ καὶ ὁ ἔξω ἡμῶν ἄνθρωπος διαφθείρεται, ἀλλ᾽ ὁ ἔσω ἡμῶν ἀνακαινοῦται ἡμέρᾳ καὶ ἡμέρᾳ.
\VS{17}τὸ γὰρ παραυτίκα ἐλαφρὸν τῆς θλίψεως ἡμῶν καθ᾽ ὑπερβολὴν εἰς ὑπερβολὴν αἰώνιον βάρος δόξης κατεργάζεται ἡμῖν,
\par }{\PP \VS{18}μὴ σκοπούντων ἡμῶν τὰ βλεπόμενα ἀλλὰ τὰ μὴ βλεπόμενα· τὰ γὰρ βλεπόμενα πρόσκαιρα, τὰ δὲ μὴ βλεπόμενα αἰώνια.

\par }\Chap{5}{\PP \VerseOne{1}Οἴδαμεν γὰρ ὅτι ἐὰν ἡ ἐπίγειος ἡμῶν οἰκία τοῦ σκήνους καταλυθῇ, οἰκοδομὴν ἐκ Θεοῦ ἔχομεν, οἰκίαν ἀχειροποίητον αἰώνιον ἐν τοῖς οὐρανοῖς.
\VS{2}καὶ γὰρ ἐν τούτῳ στενάζομεν τὸ οἰκητήριον ἡμῶν τὸ ἐξ οὐρανοῦ ἐπενδύσασθαι ἐπιποθοῦντες,
\VS{3}εἴ γε καὶ ἐνδυσάμενοι* οὐ γυμνοὶ εὑρεθησόμεθα.
\VS{4}καὶ γὰρ οἱ ὄντες ἐν τῷ σκήνει στενάζομεν βαρούμενοι, ἐφ᾽ ᾧ οὐ θέλομεν ἐκδύσασθαι ἀλλ᾽ ἐπενδύσασθαι, ἵνα καταποθῇ τὸ θνητὸν ὑπὸ τῆς ζωῆς.
\VS{5}ὁ δὲ κατεργασάμενος ἡμᾶς εἰς αὐτὸ τοῦτο Θεός, ὁ δοὺς ἡμῖν τὸν ἀρραβῶνα τοῦ Πνεύματος.
\VS{6}Θαρροῦντες οὖν πάντοτε καὶ εἰδότες ὅτι ἐνδημοῦντες ἐν τῷ σώματι ἐκδημοῦμεν ἀπὸ τοῦ Κυρίου·
\VS{7}διὰ πίστεως γὰρ περιπατοῦμεν, οὐ διὰ εἴδους·
\VS{8}Θαρροῦμεν δὲ καὶ εὐδοκοῦμεν μᾶλλον ἐκδημῆσαι ἐκ τοῦ σώματος καὶ ἐνδημῆσαι πρὸς τὸν Κύριον.
\VS{9}διὸ καὶ φιλοτιμούμεθα, εἴτε ἐνδημοῦντες εἴτε ἐκδημοῦντες, εὐάρεστοι αὐτῷ εἶναι.
\par }{\PP \VS{10}τοὺς γὰρ πάντας ἡμᾶς φανερωθῆναι δεῖ ἔμπροσθεν τοῦ βήματος τοῦ Χριστοῦ, ἵνα κομίσηται ἕκαστος τὰ διὰ τοῦ σώματος πρὸς ἃ ἔπραξεν, εἴτε ἀγαθὸν εἴτε φαῦλον.
\VS{11}Εἰδότες οὖν τὸν φόβον τοῦ Κυρίου ἀνθρώπους πείθομεν, Θεῷ δὲ πεφανερώμεθα· ἐλπίζω δὲ καὶ ἐν ταῖς συνειδήσεσιν ὑμῶν πεφανερῶσθαι.
\VS{12}οὐ πάλιν ἑαυτοὺς συνιστάνομεν ὑμῖν ἀλλὰ= ἀφορμὴν διδόντες ὑμῖν καυχήματος ὑπὲρ ἡμῶν, ἵνα ἔχητε πρὸς τοὺς ἐν προσώπῳ καυχωμένους καὶ μὴ ἐν καρδίᾳ.
\VS{13}Εἴτε γὰρ ἐξέστημεν, Θεῷ· εἴτε σωφρονοῦμεν, ὑμῖν.
\VS{14}ἡ γὰρ ἀγάπη τοῦ Χριστοῦ συνέχει ἡμᾶς, κρίναντας τοῦτο, ὅτι εἷς ὑπὲρ πάντων ἀπέθανεν, ἄρα οἱ πάντες ἀπέθανον·
\VS{15}καὶ ὑπὲρ πάντων ἀπέθανεν, ἵνα οἱ ζῶντες μηκέτι ἑαυτοῖς ζῶσιν ἀλλὰ τῷ ὑπὲρ αὐτῶν ἀποθανόντι καὶ ἐγερθέντι.
\VS{16}Ὥστε ἡμεῖς ἀπὸ τοῦ νῦν οὐδένα οἴδαμεν κατὰ σάρκα· εἰ καὶ ἐγνώκαμεν κατὰ σάρκα Χριστόν, ἀλλὰ νῦν οὐκέτι γινώσκομεν.
\VS{17}ὥστε εἴ τις ἐν Χριστῷ, καινὴ κτίσις· τὰ ἀρχαῖα παρῆλθεν, ἰδοὺ γέγονεν καινά.
\VS{18}Τὰ δὲ πάντα ἐκ τοῦ Θεοῦ τοῦ καταλλάξαντος ἡμᾶς ἑαυτῷ διὰ Χριστοῦ καὶ δόντος ἡμῖν τὴν διακονίαν τῆς καταλλαγῆς,
\VS{19}ὡς ὅτι Θεὸς ἦν ἐν Χριστῷ κόσμον καταλλάσσων ἑαυτῷ, μὴ λογιζόμενος αὐτοῖς τὰ παραπτώματα αὐτῶν καὶ θέμενος ἐν ἡμῖν τὸν λόγον τῆς καταλλαγῆς.
\VS{20}Ὑπὲρ Χριστοῦ οὖν πρεσβεύομεν ὡς τοῦ Θεοῦ παρακαλοῦντος δι᾽ ἡμῶν· δεόμεθα ὑπὲρ Χριστοῦ, καταλλάγητε τῷ Θεῷ.
\par }{\PP \VS{21}τὸν μὴ γνόντα ἁμαρτίαν ὑπὲρ ἡμῶν ἁμαρτίαν ἐποίησεν, ἵνα ἡμεῖς γενώμεθα δικαιοσύνη Θεοῦ ἐν αὐτῷ.

\par }\Chap{6}{\PP \VerseOne{1}Συνεργοῦντες δὲ καὶ παρακαλοῦμεν μὴ εἰς κενὸν τὴν χάριν τοῦ Θεοῦ δέξασθαι ὑμᾶς·
\par }{\PP \VS{2}λέγει γάρ· ¬Καιρῷ δεκτῷ ἐπήκουσά σου ¬καὶ ἐν ἡμέρᾳ σωτηρίας ἐβοήθησά σοι. Ἰδοὺ νῦν καιρὸς εὐπρόσδεκτος, ἰδοὺ νῦν ἡμέρα σωτηρίας.
\VS{3}Μηδεμίαν ἐν μηδενὶ διδόντες προσκοπήν, ἵνα μὴ μωμηθῇ ἡ διακονία,
\VS{4}ἀλλ᾽ ἐν παντὶ συνιστάντες+ ἑαυτοὺς ὡς Θεοῦ διάκονοι, ἐν ὑπομονῇ πολλῇ, ἐν θλίψεσιν, ἐν ἀνάγκαις, ἐν στενοχωρίαις,
\VS{5}ἐν πληγαῖς, ἐν φυλακαῖς, ἐν ἀκαταστασίαις, ἐν κόποις, ἐν ἀγρυπνίαις, ἐν νηστείαις,
\VS{6}ἐν ἁγνότητι, ἐν γνώσει, ἐν μακροθυμίᾳ, ἐν χρηστότητι, ἐν Πνεύματι Ἁγίῳ, ἐν ἀγάπῃ ἀνυποκρίτῳ,
\VS{7}ἐν λόγῳ ἀληθείας, ἐν δυνάμει Θεοῦ· διὰ τῶν ὅπλων τῆς δικαιοσύνης τῶν δεξιῶν καὶ ἀριστερῶν,
\VS{8}διὰ δόξης καὶ ἀτιμίας, διὰ δυσφημίας καὶ εὐφημίας· ὡς πλάνοι καὶ ἀληθεῖς,
\VS{9}ὡς ἀγνοούμενοι καὶ ἐπιγινωσκόμενοι, ὡς ἀποθνήσκοντες καὶ ἰδοὺ ζῶμεν, ὡς παιδευόμενοι καὶ μὴ θανατούμενοι,
\par }{\PP \VS{10}ὡς λυπούμενοι ἀεὶ δὲ χαίροντες, ὡς πτωχοὶ πολλοὺς δὲ πλουτίζοντες, ὡς μηδὲν ἔχοντες καὶ πάντα κατέχοντες.
\VS{11}Τὸ στόμα ἡμῶν ἀνέῳγεν πρὸς ὑμᾶς, Κορίνθιοι, ἡ καρδία ἡμῶν πεπλάτυνται·
\VS{12}οὐ στενοχωρεῖσθε ἐν ἡμῖν, στενοχωρεῖσθε δὲ ἐν τοῖς σπλάγχνοις ὑμῶν·
\par }{\PP \VS{13}τὴν δὲ αὐτὴν ἀντιμισθίαν, ὡς τέκνοις λέγω, πλατύνθητε καὶ ὑμεῖς.
\VS{14}Μὴ γίνεσθε ἑτεροζυγοῦντες ἀπίστοις· τίς γὰρ μετοχὴ δικαιοσύνῃ καὶ ἀνομίᾳ, ἢ τίς κοινωνία φωτὶ πρὸς σκότος;
\VS{15}τίς δὲ συμφώνησις Χριστοῦ πρὸς Βελιάρ, ἢ τίς μερὶς πιστῷ μετὰ ἀπίστου;
\VS{16}τίς δὲ συνκατάθεσις= ναῷ Θεοῦ μετὰ εἰδώλων; ἡμεῖς γὰρ ναὸς Θεοῦ ἐσμεν ζῶντος, καθὼς εἶπεν ὁ Θεὸς ὅτι ¬Ἐνοικήσω ἐν αὐτοῖς καὶ ἐμπεριπατήσω ¬καὶ ἔσομαι αὐτῶν Θεός καὶ αὐτοὶ ἔσονταί μου λαός.
\VS{17}¬Διὸ ἐξέλθατε ἐκ μέσου αὐτῶν ¬καὶ ἀφορίσθητε, λέγει Κύριος, ¬καὶ ἀκαθάρτου μὴ ἅπτεσθε· ¬κἀγὼ εἰσδέξομαι ὑμᾶς
\par }{\PP \VS{18}¬Καὶ Ἔσομαι ὑμῖν εἰς Πατέρα ¬καὶ ὑμεῖς ἔσεσθέ μοι εἰς υἱοὺς καὶ θυγατέρας, ¬λέγει Κύριος Παντοκράτωρ.

\par }\Chap{7}{\PP \VerseOne{1}Ταύτας οὖν ἔχοντες τὰς ἐπαγγελίας, ἀγαπητοί, καθαρίσωμεν ἑαυτοὺς ἀπὸ παντὸς μολυσμοῦ σαρκὸς καὶ πνεύματος, ἐπιτελοῦντες ἁγιωσύνην ἐν φόβῳ Θεοῦ.
\VS{2}Χωρήσατε ἡμᾶς· οὐδένα ἠδικήσαμεν, οὐδένα ἐφθείραμεν, οὐδένα ἐπλεονεκτήσαμεν.
\VS{3}πρὸς κατάκρισιν οὐ λέγω· προείρηκα γὰρ ὅτι ἐν ταῖς καρδίαις ἡμῶν ἐστε εἰς τὸ συναποθανεῖν καὶ συζῆν.
\par }{\PP \VS{4}πολλή μοι παρρησία πρὸς ὑμᾶς, πολλή μοι καύχησις ὑπὲρ ὑμῶν· πεπλήρωμαι τῇ παρακλήσει, ὑπερπερισσεύομαι τῇ χαρᾷ ἐπὶ πάσῃ τῇ θλίψει ἡμῶν.
\VS{5}Καὶ γὰρ ἐλθόντων ἡμῶν εἰς Μακεδονίαν οὐδεμίαν ἔσχηκεν ἄνεσιν ἡ σὰρξ ἡμῶν ἀλλ᾽ ἐν παντὶ θλιβόμενοι· ἔξωθεν μάχαι, ἔσωθεν φόβοι.
\VS{6}ἀλλ᾽ ὁ παρακαλῶν τοὺς ταπεινοὺς παρεκάλεσεν ἡμᾶς ὁ Θεὸς ἐν τῇ παρουσίᾳ Τίτου,
\VS{7}οὐ μόνον δὲ ἐν τῇ παρουσίᾳ αὐτοῦ ἀλλὰ καὶ ἐν τῇ παρακλήσει ᾗ παρεκλήθη ἐφ᾽ ὑμῖν, ἀναγγέλλων ἡμῖν τὴν ὑμῶν ἐπιπόθησιν, τὸν ὑμῶν ὀδυρμόν, τὸν ὑμῶν ζῆλον ὑπὲρ ἐμοῦ ὥστε με μᾶλλον χαρῆναι.
\VS{8}Ὅτι εἰ καὶ ἐλύπησα ὑμᾶς ἐν τῇ ἐπιστολῇ, οὐ μεταμέλομαι· εἰ καὶ μετεμελόμην, βλέπω γὰρ ὅτι ἡ ἐπιστολὴ ἐκείνη εἰ καὶ πρὸς ὥραν ἐλύπησεν ὑμᾶς,
\VS{9}νῦν χαίρω, οὐχ ὅτι ἐλυπήθητε ἀλλ᾽ ὅτι ἐλυπήθητε εἰς μετάνοιαν· ἐλυπήθητε γὰρ κατὰ Θεόν, ἵνα ἐν μηδενὶ ζημιωθῆτε ἐξ ἡμῶν.
\VS{10}ἡ γὰρ κατὰ Θεὸν λύπη μετάνοιαν εἰς σωτηρίαν ἀμεταμέλητον ἐργάζεται· ἡ δὲ τοῦ κόσμου λύπη θάνατον κατεργάζεται.
\VS{11}Ἰδοὺ γὰρ αὐτὸ τοῦτο τὸ κατὰ Θεὸν λυπηθῆναι πόσην κατειργάσατο ὑμῖν σπουδήν, ἀλλὰ= ἀπολογίαν, ἀλλὰ= ἀγανάκτησιν, ἀλλὰ φόβον, ἀλλὰ= ἐπιπόθησιν, ἀλλὰ ζῆλον, ἀλλὰ= ἐκδίκησιν. ἐν παντὶ συνεστήσατε ἑαυτοὺς ἁγνοὺς εἶναι τῷ πράγματι.
\VS{12}ἄρα εἰ καὶ ἔγραψα ὑμῖν, οὐχ ἕνεκεν τοῦ ἀδικήσαντος οὐδὲ ἕνεκεν τοῦ ἀδικηθέντος ἀλλ᾽ ἕνεκεν τοῦ φανερωθῆναι τὴν σπουδὴν ὑμῶν τὴν ὑπὲρ ἡμῶν πρὸς ὑμᾶς ἐνώπιον τοῦ Θεοῦ.
\VS{13}διὰ τοῦτο παρακεκλήμεθα. Ἐπὶ δὲ τῇ παρακλήσει ἡμῶν περισσοτέρως μᾶλλον ἐχάρημεν ἐπὶ τῇ χαρᾷ Τίτου, ὅτι ἀναπέπαυται τὸ πνεῦμα αὐτοῦ ἀπὸ πάντων ὑμῶν·
\VS{14}ὅτι εἴ τι αὐτῷ ὑπὲρ ὑμῶν κεκαύχημαι, οὐ κατῃσχύνθην, ἀλλ᾽ ὡς πάντα ἐν ἀληθείᾳ ἐλαλήσαμεν ὑμῖν, οὕτως καὶ ἡ καύχησις ἡμῶν ἡ ἐπὶ Τίτου ἀλήθεια ἐγενήθη.
\VS{15}καὶ τὰ σπλάγχνα αὐτοῦ περισσοτέρως εἰς ὑμᾶς ἐστιν ἀναμιμνῃσκομένου τὴν πάντων ὑμῶν ὑπακοήν, ὡς μετὰ φόβου καὶ τρόμου ἐδέξασθε αὐτόν.
\par }{\PP \VS{16}χαίρω ὅτι ἐν παντὶ θαρρῶ ἐν ὑμῖν.

\par }\Chap{8}{\PP \VerseOne{1}Γνωρίζομεν δὲ ὑμῖν, ἀδελφοί, τὴν χάριν τοῦ Θεοῦ τὴν δεδομένην ἐν ταῖς ἐκκλησίαις τῆς Μακεδονίας,
\VS{2}ὅτι ἐν πολλῇ δοκιμῇ θλίψεως ἡ περισσεία τῆς χαρᾶς αὐτῶν καὶ ἡ κατὰ βάθους πτωχεία αὐτῶν ἐπερίσσευσεν εἰς τὸ πλοῦτος τῆς ἁπλότητος αὐτῶν·
\VS{3}ὅτι κατὰ δύναμιν, μαρτυρῶ, καὶ παρὰ δύναμιν, αὐθαίρετοι
\VS{4}μετὰ πολλῆς παρακλήσεως δεόμενοι ἡμῶν τὴν χάριν καὶ τὴν κοινωνίαν τῆς διακονίας τῆς εἰς τοὺς ἁγίους,
\VS{5}καὶ οὐ καθὼς ἠλπίσαμεν ἀλλ᾽ ἑαυτοὺς ἔδωκαν πρῶτον τῷ Κυρίῳ καὶ ἡμῖν διὰ θελήματος Θεοῦ
\VS{6}Εἰς τὸ παρακαλέσαι ἡμᾶς Τίτον, ἵνα καθὼς προενήρξατο οὕτως καὶ ἐπιτελέσῃ εἰς ὑμᾶς καὶ τὴν χάριν ταύτην.
\VS{7}ἀλλ᾽ ὥσπερ ἐν παντὶ περισσεύετε, πίστει καὶ λόγῳ καὶ γνώσει καὶ πάσῃ σπουδῇ καὶ τῇ ἐξ ἡμῶν ἐν ὑμῖν ἀγάπῃ, ἵνα καὶ ἐν ταύτῃ τῇ χάριτι περισσεύητε.
\VS{8}Οὐ κατ᾽ ἐπιταγὴν λέγω ἀλλὰ διὰ τῆς ἑτέρων σπουδῆς καὶ τὸ τῆς ὑμετέρας ἀγάπης γνήσιον δοκιμάζων·
\VS{9}Γινώσκετε γὰρ τὴν χάριν τοῦ Κυρίου ἡμῶν Ἰησοῦ Χριστοῦ, ὅτι δι᾽ ὑμᾶς ἐπτώχευσεν πλούσιος ὤν, ἵνα ὑμεῖς τῇ ἐκείνου πτωχείᾳ πλουτήσητε.
\VS{10}καὶ γνώμην ἐν τούτῳ δίδωμι· τοῦτο γὰρ ὑμῖν συμφέρει, οἵτινες οὐ μόνον τὸ ποιῆσαι ἀλλὰ καὶ τὸ θέλειν προενήρξασθε ἀπὸ πέρυσι·
\VS{11}νυνὶ δὲ καὶ τὸ ποιῆσαι ἐπιτελέσατε, ὅπως καθάπερ ἡ προθυμία τοῦ θέλειν, οὕτως καὶ τὸ ἐπιτελέσαι ἐκ τοῦ ἔχειν.
\VS{12}εἰ γὰρ ἡ προθυμία πρόκειται, καθὸ ἐὰν ἔχῃ εὐπρόσδεκτος, οὐ καθὸ οὐκ ἔχει.
\VS{13}Οὐ γὰρ ἵνα ἄλλοις ἄνεσις, ὑμῖν θλῖψις, ἀλλ᾽ ἐξ ἰσότητος·
\VS{14}ἐν τῷ νῦν καιρῷ τὸ ὑμῶν περίσσευμα εἰς τὸ ἐκείνων ὑστέρημα, ἵνα καὶ τὸ ἐκείνων περίσσευμα γένηται εἰς τὸ ὑμῶν ὑστέρημα, ὅπως γένηται ἰσότης,
\par }{\PP \VS{15}καθὼς γέγραπται· Ὁ τὸ πολὺ οὐκ ἐπλεόνασεν, καὶ ὁ τὸ ὀλίγον οὐκ ἠλαττόνησεν.
\VS{16}Χάρις δὲ τῷ Θεῷ τῷ διδόντι* τὴν αὐτὴν σπουδὴν ὑπὲρ ὑμῶν ἐν τῇ καρδίᾳ Τίτου,
\VS{17}ὅτι τὴν μὲν παράκλησιν ἐδέξατο, σπουδαιότερος δὲ ὑπάρχων αὐθαίρετος ἐξῆλθεν πρὸς ὑμᾶς.
\VS{18}Συνεπέμψαμεν δὲ μετ᾽ αὐτοῦ τὸν ἀδελφὸν οὗ ὁ ἔπαινος ἐν τῷ εὐαγγελίῳ διὰ πασῶν τῶν ἐκκλησιῶν,
\VS{19}οὐ μόνον δὲ, ἀλλὰ καὶ χειροτονηθεὶς ὑπὸ τῶν ἐκκλησιῶν συνέκδημος ἡμῶν σὺν τῇ χάριτι ταύτῃ τῇ διακονουμένῃ ὑφ᾽ ἡμῶν πρὸς τὴν αὐτοῦ τοῦ Κυρίου δόξαν καὶ προθυμίαν ἡμῶν,
\VS{20}στελλόμενοι τοῦτο, μή τις ἡμᾶς μωμήσηται ἐν τῇ ἁδρότητι ταύτῃ τῇ διακονουμένῃ ὑφ᾽ ἡμῶν·
\VS{21}προνοοῦμεν γὰρ καλὰ οὐ μόνον ἐνώπιον Κυρίου ἀλλὰ καὶ ἐνώπιον ἀνθρώπων.
\VS{22}Συνεπέμψαμεν δὲ αὐτοῖς τὸν ἀδελφὸν ἡμῶν ὃν ἐδοκιμάσαμεν ἐν πολλοῖς πολλάκις σπουδαῖον ὄντα, νυνὶ δὲ πολὺ σπουδαιότερον πεποιθήσει πολλῇ τῇ εἰς ὑμᾶς.
\VS{23}εἴτε ὑπὲρ Τίτου, κοινωνὸς ἐμὸς καὶ εἰς ὑμᾶς συνεργός· εἴτε ἀδελφοὶ ἡμῶν, ἀπόστολοι ἐκκλησιῶν, δόξα Χριστοῦ.
\par }{\PP \VS{24}τὴν οὖν ἔνδειξιν τῆς ἀγάπης ὑμῶν καὶ ἡμῶν καυχήσεως ὑπὲρ ὑμῶν εἰς αὐτοὺς ἐνδεικνύμενοι εἰς πρόσωπον τῶν ἐκκλησιῶν.

\par }\Chap{9}{\PP \VerseOne{1}Περὶ μὲν γὰρ τῆς διακονίας τῆς εἰς τοὺς ἁγίους περισσόν μοί ἐστιν τὸ γράφειν ὑμῖν·
\VS{2}οἶδα γὰρ τὴν προθυμίαν ὑμῶν ἣν ὑπὲρ ὑμῶν καυχῶμαι Μακεδόσιν, ὅτι Ἀχαΐα παρεσκεύασται ἀπὸ πέρυσι, καὶ τὸ ὑμῶν ζῆλος ἠρέθισεν τοὺς πλείονας.
\VS{3}Ἔπεμψα δὲ τοὺς ἀδελφούς, ἵνα μὴ τὸ καύχημα ἡμῶν τὸ ὑπὲρ ὑμῶν κενωθῇ ἐν τῷ μέρει τούτῳ, ἵνα καθὼς ἔλεγον παρεσκευασμένοι ἦτε,
\VS{4}μή πως ἐὰν ἔλθωσιν σὺν ἐμοὶ Μακεδόνες καὶ εὕρωσιν ὑμᾶς ἀπαρασκευάστους καταισχυνθῶμεν ἡμεῖς, ἵνα μὴ λέγωμεν* ὑμεῖς, ἐν τῇ ὑποστάσει ταύτῃ.
\par }{\PP \VS{5}ἀναγκαῖον οὖν ἡγησάμην παρακαλέσαι τοὺς ἀδελφοὺς, ἵνα προέλθωσιν εἰς ὑμᾶς καὶ προκαταρτίσωσιν τὴν προεπηγγελμένην εὐλογίαν ὑμῶν, ταύτην ἑτοίμην εἶναι οὕτως ὡς εὐλογίαν καὶ μὴ ὡς πλεονεξίαν.
\VS{6}Τοῦτο δέ, ὁ σπείρων φειδομένως φειδομένως καὶ θερίσει, καὶ ὁ σπείρων ἐπ᾽ εὐλογίαις ἐπ᾽ εὐλογίαις καὶ θερίσει.
\VS{7}ἕκαστος καθὼς προῄρηται τῇ καρδίᾳ, μὴ ἐκ λύπης ἢ ἐξ ἀνάγκης· ἱλαρὸν γὰρ δότην ἀγαπᾷ ὁ Θεός.
\VS{8}δυνατεῖ δὲ ὁ Θεὸς πᾶσαν χάριν περισσεῦσαι εἰς ὑμᾶς, ἵνα ἐν παντὶ πάντοτε πᾶσαν αὐτάρκειαν ἔχοντες περισσεύητε εἰς πᾶν ἔργον ἀγαθόν,
\VS{9}καθὼς γέγραπται· ¬Ἐσκόρπισεν, ἔδωκεν τοῖς πένησιν, ¬ἡ δικαιοσύνη αὐτοῦ μένει εἰς τὸν αἰῶνα.
\VS{10}Ὁ δὲ ἐπιχορηγῶν σπόρον τῷ σπείροντι καὶ ἄρτον εἰς βρῶσιν χορηγήσει καὶ πληθυνεῖ τὸν σπόρον ὑμῶν καὶ αὐξήσει τὰ γενήματα τῆς δικαιοσύνης ὑμῶν.
\VS{11}ἐν παντὶ πλουτιζόμενοι εἰς πᾶσαν ἁπλότητα, ἥτις κατεργάζεται δι᾽ ἡμῶν εὐχαριστίαν τῷ Θεῷ·
\VS{12}ὅτι ἡ διακονία τῆς λειτουργίας ταύτης οὐ μόνον ἐστὶν προσαναπληροῦσα τὰ ὑστερήματα τῶν ἁγίων, ἀλλὰ καὶ περισσεύουσα διὰ πολλῶν εὐχαριστιῶν τῷ Θεῷ.
\VS{13}διὰ τῆς δοκιμῆς τῆς διακονίας ταύτης δοξάζοντες τὸν Θεὸν ἐπὶ τῇ ὑποταγῇ τῆς ὁμολογίας ὑμῶν εἰς τὸ εὐαγγέλιον τοῦ Χριστοῦ καὶ ἁπλότητι τῆς κοινωνίας εἰς αὐτοὺς καὶ εἰς πάντας,
\VS{14}καὶ αὐτῶν δεήσει ὑπὲρ ὑμῶν ἐπιποθούντων ὑμᾶς διὰ τὴν ὑπερβάλλουσαν χάριν τοῦ Θεοῦ ἐφ᾽ ὑμῖν.
\par }{\PP \VS{15}Χάρις τῷ Θεῷ ἐπὶ τῇ ἀνεκδιηγήτῳ αὐτοῦ δωρεᾷ.

\par }\Chap{10}{\PP \VerseOne{1}Αὐτὸς δὲ ἐγὼ Παῦλος παρακαλῶ ὑμᾶς διὰ τῆς πραΰτητος καὶ ἐπιεικείας τοῦ Χριστοῦ, ὃς κατὰ πρόσωπον μὲν ταπεινὸς ἐν ὑμῖν, ἀπὼν δὲ θαρρῶ εἰς ὑμᾶς·
\VS{2}δέομαι δὲ τὸ μὴ παρὼν θαρρῆσαι τῇ πεποιθήσει ᾗ λογίζομαι τολμῆσαι ἐπί τινας τοὺς λογιζομένους ἡμᾶς ὡς κατὰ σάρκα περιπατοῦντας.
\VS{3}Ἐν σαρκὶ γὰρ περιπατοῦντες οὐ κατὰ σάρκα στρατευόμεθα,
\VS{4}τὰ γὰρ ὅπλα τῆς στρατείας ἡμῶν οὐ σαρκικὰ ἀλλὰ δυνατὰ τῷ Θεῷ πρὸς καθαίρεσιν ὀχυρωμάτων, λογισμοὺς καθαιροῦντες
\VS{5}καὶ πᾶν ὕψωμα ἐπαιρόμενον κατὰ τῆς γνώσεως τοῦ Θεοῦ, καὶ αἰχμαλωτίζοντες πᾶν νόημα εἰς τὴν ὑπακοὴν τοῦ Χριστοῦ,
\par }{\PP \VS{6}καὶ ἐν ἑτοίμῳ ἔχοντες ἐκδικῆσαι πᾶσαν παρακοήν, ὅταν πληρωθῇ ὑμῶν ἡ ὑπακοή.
\VS{7}Τὰ κατὰ πρόσωπον βλέπετε. εἴ τις πέποιθεν ἑαυτῷ Χριστοῦ εἶναι, τοῦτο λογιζέσθω πάλιν ἐφ᾽ ἑαυτοῦ, ὅτι καθὼς αὐτὸς Χριστοῦ, οὕτως καὶ ἡμεῖς.
\VS{8}ἐάν τε γὰρ περισσότερόν τι καυχήσωμαι περὶ τῆς ἐξουσίας ἡμῶν ἧς ἔδωκεν ὁ Κύριος εἰς οἰκοδομὴν καὶ οὐκ εἰς καθαίρεσιν ὑμῶν, οὐκ αἰσχυνθήσομαι.
\VS{9}ἵνα μὴ δόξω ὡς ἂν ἐκφοβεῖν ὑμᾶς διὰ τῶν ἐπιστολῶν·
\VS{10}Ὅτι Αἱ ἐπιστολαὶ μέν, φησίν, Βαρεῖαι καὶ ἰσχυραί, ἡ δὲ παρουσία τοῦ σώματος ἀσθενὴς καὶ ὁ λόγος ἐξουθενημένος.
\par }{\PP \VS{11}τοῦτο λογιζέσθω ὁ τοιοῦτος, ὅτι οἷοί ἐσμεν τῷ λόγῳ δι᾽ ἐπιστολῶν ἀπόντες, τοιοῦτοι καὶ παρόντες τῷ ἔργῳ.
\VS{12}Οὐ γὰρ τολμῶμεν ἐνκρῖναι= ἢ συνκρῖναι= ἑαυτούς τισιν τῶν ἑαυτοὺς συνιστανόντων, ἀλλὰ= αὐτοὶ ἐν ἑαυτοῖς ἑαυτοὺς μετροῦντες καὶ συνκρίνοντες= ἑαυτοὺς ἑαυτοῖς οὐ συνιᾶσιν.
\VS{13}ἡμεῖς δὲ οὐκ εἰς τὰ ἄμετρα καυχησόμεθα ἀλλὰ κατὰ τὸ μέτρον τοῦ κανόνος οὗ ἐμέρισεν ἡμῖν ὁ Θεὸς μέτρου, ἐφικέσθαι ἄχρι καὶ ὑμῶν.
\VS{14}οὐ γὰρ ὡς μὴ ἐφικνούμενοι εἰς ὑμᾶς ὑπερεκτείνομεν ἑαυτούς, ἄχρι γὰρ καὶ ὑμῶν ἐφθάσαμεν ἐν τῷ εὐαγγελίῳ τοῦ Χριστοῦ,
\VS{15}οὐκ εἰς τὰ ἄμετρα καυχώμενοι ἐν ἀλλοτρίοις κόποις, ἐλπίδα δὲ ἔχοντες αὐξανομένης τῆς πίστεως ὑμῶν ἐν ὑμῖν μεγαλυνθῆναι κατὰ τὸν κανόνα ἡμῶν εἰς περισσείαν
\VS{16}εἰς τὰ ὑπερέκεινα ὑμῶν εὐαγγελίσασθαι, οὐκ ἐν ἀλλοτρίῳ κανόνι εἰς τὰ ἕτοιμα καυχήσασθαι.
\VS{17}Ὁ δὲ καυχώμενος ἐν Κυρίῳ καυχάσθω·
\par }{\PP \VS{18}οὐ γὰρ ὁ ἑαυτὸν συνιστάνων, ἐκεῖνός ἐστιν δόκιμος, ἀλλὰ= ὃν ὁ Κύριος συνίστησιν.

\par }\Chap{11}{\PP \VerseOne{1}Ὄφελον ἀνείχεσθέ μου μικρόν τι ἀφροσύνης· ἀλλὰ καὶ ἀνέχεσθέ μου.
\VS{2}ζηλῶ γὰρ ὑμᾶς Θεοῦ ζήλῳ, ἡρμοσάμην γὰρ ὑμᾶς ἑνὶ ἀνδρὶ παρθένον ἁγνὴν παραστῆσαι τῷ Χριστῷ·
\VS{3}Φοβοῦμαι δὲ μή πως, ὡς ὁ ὄφις ἐξηπάτησεν Εὕαν ἐν τῇ πανουργίᾳ αὐτοῦ, φθαρῇ τὰ νοήματα ὑμῶν ἀπὸ τῆς ἁπλότητος καὶ τῆς ἁγνότητος τῆς εἰς τὸν Χριστόν.
\par }{\PP \VS{4}εἰ μὲν γὰρ ὁ ἐρχόμενος ἄλλον Ἰησοῦν κηρύσσει ὃν οὐκ ἐκηρύξαμεν, ἢ πνεῦμα ἕτερον λαμβάνετε ὃ οὐκ ἐλάβετε, ἢ εὐαγγέλιον ἕτερον ὃ οὐκ ἐδέξασθε, καλῶς ἀνέχεσθε.
\VS{5}Λογίζομαι γὰρ μηδὲν ὑστερηκέναι τῶν Ὑπερλίαν ἀποστόλων.
\VS{6}εἰ δὲ καὶ ἰδιώτης τῷ λόγῳ, ἀλλ᾽ οὐ τῇ γνώσει, ἀλλ᾽ ἐν παντὶ φανερώσαντες ἐν πᾶσιν εἰς ὑμᾶς.
\VS{7}Ἢ ἁμαρτίαν ἐποίησα ἐμαυτὸν ταπεινῶν ἵνα ὑμεῖς ὑψωθῆτε, ὅτι δωρεὰν τὸ τοῦ Θεοῦ εὐαγγέλιον εὐηγγελισάμην ὑμῖν;
\VS{8}ἄλλας ἐκκλησίας ἐσύλησα λαβὼν ὀψώνιον πρὸς τὴν ὑμῶν διακονίαν,
\VS{9}καὶ παρὼν πρὸς ὑμᾶς καὶ ὑστερηθεὶς οὐ κατενάρκησα οὐθενός· τὸ γὰρ ὑστέρημά μου προσανεπλήρωσαν οἱ ἀδελφοὶ ἐλθόντες ἀπὸ Μακεδονίας, καὶ ἐν παντὶ ἀβαρῆ ἐμαυτὸν ὑμῖν ἐτήρησα καὶ τηρήσω.
\VS{10}ἔστιν ἀλήθεια Χριστοῦ ἐν ἐμοὶ ὅτι ἡ καύχησις αὕτη οὐ φραγήσεται εἰς ἐμὲ ἐν τοῖς κλίμασιν τῆς Ἀχαΐας.
\par }{\PP \VS{11}διὰ τί; ὅτι οὐκ ἀγαπῶ ὑμᾶς; ὁ Θεὸς οἶδεν.
\VS{12}Ὃ δὲ ποιῶ, καὶ ποιήσω, ἵνα ἐκκόψω τὴν ἀφορμὴν τῶν θελόντων ἀφορμήν, ἵνα ἐν ᾧ καυχῶνται εὑρεθῶσιν καθὼς καὶ ἡμεῖς.
\VS{13}οἱ γὰρ τοιοῦτοι ψευδαπόστολοι, ἐργάται δόλιοι, μετασχηματιζόμενοι εἰς ἀποστόλους Χριστοῦ.
\VS{14}καὶ οὐ θαῦμα· αὐτὸς γὰρ ὁ Σατανᾶς μετασχηματίζεται εἰς ἄγγελον φωτός.
\par }{\PP \VS{15}οὐ μέγα οὖν εἰ καὶ οἱ διάκονοι αὐτοῦ μετασχηματίζονται ὡς διάκονοι δικαιοσύνης· ὧν τὸ τέλος ἔσται κατὰ τὰ ἔργα αὐτῶν.
\VS{16}Πάλιν λέγω, μή τίς με δόξῃ ἄφρονα εἶναι· εἰ δὲ μή γε, κἂν ὡς ἄφρονα δέξασθέ με, ἵνα κἀγὼ μικρόν τι καυχήσωμαι.
\VS{17}ὃ λαλῶ, οὐ κατὰ Κύριον λαλῶ ἀλλ᾽ ὡς ἐν ἀφροσύνῃ, ἐν ταύτῃ τῇ ὑποστάσει τῆς καυχήσεως.
\VS{18}ἐπεὶ πολλοὶ καυχῶνται κατὰ σάρκα, κἀγὼ καυχήσομαι.
\VS{19}ἡδέως γὰρ ἀνέχεσθε τῶν ἀφρόνων φρόνιμοι ὄντες·
\VS{20}ἀνέχεσθε γὰρ εἴ τις ὑμᾶς καταδουλοῖ, εἴ τις κατεσθίει, εἴ τις λαμβάνει, εἴ τις ἐπαίρεται, εἴ τις εἰς πρόσωπον ὑμᾶς δέρει.
\par }{\PP \VS{21}κατὰ ἀτιμίαν λέγω, ὡς ὅτι ἡμεῖς ἠσθενήκαμεν. Ἐν ᾧ δ᾽ ἄν τις τολμᾷ, ἐν ἀφροσύνῃ λέγω, τολμῶ κἀγώ.
\VS{22}Ἑβραῖοί εἰσιν; κἀγώ. Ἰσραηλῖταί εἰσιν; κἀγώ. σπέρμα Ἀβραάμ εἰσιν; κἀγώ.
\VS{23}διάκονοι Χριστοῦ εἰσιν; παραφρονῶν λαλῶ, ὑπὲρ ἐγώ· ἐν κόποις περισσοτέρως, ἐν φυλακαῖς περισσοτέρως, ἐν πληγαῖς ὑπερβαλλόντως, ἐν θανάτοις πολλάκις.
\VS{24}Ὑπὸ Ἰουδαίων πεντάκις τεσσεράκοντα παρὰ μίαν ἔλαβον,
\VS{25}τρὶς ἐραβδίσθην,= ἅπαξ ἐλιθάσθην, τρὶς ἐναυάγησα, νυχθήμερον ἐν τῷ βυθῷ πεποίηκα·
\VS{26}ὁδοιπορίαις πολλάκις, κινδύνοις ποταμῶν, κινδύνοις λῃστῶν, κινδύνοις ἐκ γένους, κινδύνοις ἐξ ἐθνῶν, κινδύνοις ἐν πόλει, κινδύνοις ἐν ἐρημίᾳ, κινδύνοις ἐν θαλάσσῃ, κινδύνοις ἐν ψευδαδέλφοις,
\VS{27}κόπῳ καὶ μόχθῳ, ἐν ἀγρυπνίαις πολλάκις, ἐν λιμῷ καὶ δίψει, ἐν νηστείαις πολλάκις, ἐν ψύχει καὶ γυμνότητι·
\VS{28}Χωρὶς τῶν παρεκτὸς ἡ ἐπίστασίς μοι ἡ καθ᾽ ἡμέραν, ἡ μέριμνα πασῶν τῶν ἐκκλησιῶν.
\VS{29}τίς ἀσθενεῖ καὶ οὐκ ἀσθενῶ; τίς σκανδαλίζεται καὶ οὐκ ἐγὼ πυροῦμαι;
\VS{30}Εἰ καυχᾶσθαι δεῖ, τὰ τῆς ἀσθενείας μου καυχήσομαι.
\VS{31}ὁ Θεὸς καὶ Πατὴρ τοῦ Κυρίου Ἰησοῦ οἶδεν, ὁ ὢν εὐλογητὸς εἰς τοὺς αἰῶνας, ὅτι οὐ ψεύδομαι.
\VS{32}ἐν Δαμασκῷ ὁ ἐθνάρχης Ἁρέτα τοῦ βασιλέως ἐφρούρει τὴν πόλιν Δαμασκηνῶν πιάσαι με,
\par }{\PP \VS{33}καὶ διὰ θυρίδος ἐν σαργάνῃ ἐχαλάσθην διὰ τοῦ τείχους καὶ ἐξέφυγον τὰς χεῖρας αὐτοῦ.

\par }\Chap{12}{\PP \VerseOne{1}Καυχᾶσθαι δεῖ, οὐ συμφέρον μέν, ἐλεύσομαι δὲ εἰς ὀπτασίας καὶ ἀποκαλύψεις Κυρίου.
\VS{2}οἶδα ἄνθρωπον ἐν Χριστῷ πρὸ ἐτῶν δεκατεσσάρων, εἴτε ἐν σώματι οὐκ οἶδα, εἴτε ἐκτὸς τοῦ σώματος οὐκ οἶδα, ὁ Θεὸς οἶδεν, ἁρπαγέντα τὸν τοιοῦτον ἕως τρίτου οὐρανοῦ.
\VS{3}καὶ οἶδα τὸν τοιοῦτον ἄνθρωπον, εἴτε ἐν σώματι εἴτε χωρὶς τοῦ σώματος οὐκ οἶδα, ὁ Θεὸς οἶδεν,
\VS{4}ὅτι ἡρπάγη εἰς τὸν Παράδεισον καὶ ἤκουσεν ἄρρητα ῥήματα ἃ οὐκ ἐξὸν ἀνθρώπῳ λαλῆσαι.
\VS{5}Ὑπὲρ τοῦ τοιούτου καυχήσομαι, ὑπὲρ δὲ ἐμαυτοῦ οὐ καυχήσομαι εἰ μὴ ἐν ταῖς ἀσθενείαις.
\VS{6}ἐὰν γὰρ θελήσω καυχήσασθαι, οὐκ ἔσομαι ἄφρων, ἀλήθειαν γὰρ ἐρῶ· φείδομαι δέ, μή τις εἰς ἐμὲ λογίσηται ὑπὲρ ὃ βλέπει με ἢ ἀκούει τι ἐξ ἐμοῦ
\VS{7}καὶ τῇ ὑπερβολῇ τῶν ἀποκαλύψεων. Διὸ ἵνα μὴ ὑπεραίρωμαι, ἐδόθη μοι σκόλοψ τῇ σαρκί, ἄγγελος Σατανᾶ, ἵνα με κολαφίζῃ, ἵνα μὴ ὑπεραίρωμαι.
\VS{8}ὑπὲρ τούτου τρὶς τὸν Κύριον παρεκάλεσα ἵνα ἀποστῇ ἀπ᾽ ἐμοῦ.
\VS{9}καὶ εἴρηκέν μοι· Ἀρκεῖ σοι ἡ χάρις μου, ἡ γὰρ δύναμις ἐν ἀσθενείᾳ τελεῖται. Ἥδιστα οὖν μᾶλλον καυχήσομαι ἐν ταῖς ἀσθενείαις μου, ἵνα ἐπισκηνώσῃ ἐπ᾽ ἐμὲ ἡ δύναμις τοῦ Χριστοῦ.
\par }{\PP \VS{10}διὸ εὐδοκῶ ἐν ἀσθενείαις, ἐν ὕβρεσιν, ἐν ἀνάγκαις, ἐν διωγμοῖς καὶ στενοχωρίαις, ὑπὲρ Χριστοῦ· ὅταν γὰρ ἀσθενῶ, τότε δυνατός εἰμι.
\VS{11}Γέγονα ἄφρων, ὑμεῖς με ἠναγκάσατε. ἐγὼ γὰρ ὤφειλον ὑφ᾽ ὑμῶν συνίστασθαι· οὐδὲν γὰρ ὑστέρησα τῶν Ὑπερλίαν ἀποστόλων εἰ καὶ οὐδέν εἰμι.
\VS{12}τὰ μὲν σημεῖα τοῦ ἀποστόλου κατειργάσθη ἐν ὑμῖν ἐν πάσῃ ὑπομονῇ, σημείοις τε καὶ τέρασιν καὶ δυνάμεσιν.
\par }{\PP \VS{13}τί γάρ ἐστιν ὃ ἡσσώθητε ὑπὲρ τὰς λοιπὰς ἐκκλησίας, εἰ μὴ ὅτι αὐτὸς ἐγὼ οὐ κατενάρκησα ὑμῶν; χαρίσασθέ μοι τὴν ἀδικίαν ταύτην.
\VS{14}Ἰδοὺ τρίτον τοῦτο ἑτοίμως ἔχω ἐλθεῖν πρὸς ὑμᾶς, καὶ οὐ καταναρκήσω· οὐ γὰρ ζητῶ τὰ ὑμῶν ἀλλὰ= ὑμᾶς. οὐ γὰρ ὀφείλει τὰ τέκνα τοῖς γονεῦσιν θησαυρίζειν ἀλλὰ= οἱ γονεῖς τοῖς τέκνοις.
\VS{15}ἐγὼ δὲ ἥδιστα δαπανήσω καὶ ἐκδαπανηθήσομαι ὑπὲρ τῶν ψυχῶν ὑμῶν. εἰ περισσοτέρως ὑμᾶς ἀγαπῶν, ἧσσον ἀγαπῶμαι;
\VS{16}Ἔστω δέ, ἐγὼ οὐ κατεβάρησα ὑμᾶς· ἀλλὰ= ὑπάρχων πανοῦργος δόλῳ ὑμᾶς ἔλαβον.
\VS{17}μή τινα ὧν ἀπέσταλκα πρὸς ὑμᾶς, δι᾽ αὐτοῦ ἐπλεονέκτησα ὑμᾶς;
\par }{\PP \VS{18}παρεκάλεσα Τίτον καὶ συναπέστειλα τὸν ἀδελφόν· μήτι ἐπλεονέκτησεν ὑμᾶς Τίτος; οὐ τῷ αὐτῷ Πνεύματι περιεπατήσαμεν; οὐ τοῖς αὐτοῖς ἴχνεσιν;
\VS{19}Πάλαι δοκεῖτε ὅτι ὑμῖν ἀπολογούμεθα. κατέναντι Θεοῦ ἐν Χριστῷ λαλοῦμεν· τὰ δὲ πάντα, ἀγαπητοί, ὑπὲρ τῆς ὑμῶν οἰκοδομῆς.
\VS{20}φοβοῦμαι γὰρ μή πως ἐλθὼν οὐχ οἵους θέλω εὕρω ὑμᾶς κἀγὼ εὑρεθῶ ὑμῖν οἷον οὐ θέλετε· μή πως ἔρις, ζῆλος, θυμοί, ἐριθεῖαι, καταλαλιαί, ψιθυρισμοί, φυσιώσεις, ἀκαταστασίαι·
\par }{\PP \VS{21}μὴ πάλιν ἐλθόντος μου ταπεινώσῃ με ὁ Θεός μου πρὸς ὑμᾶς καὶ πενθήσω πολλοὺς τῶν προημαρτηκότων καὶ μὴ μετανοησάντων ἐπὶ τῇ ἀκαθαρσίᾳ καὶ πορνείᾳ καὶ ἀσελγείᾳ ᾗ ἔπραξαν.

\par }\Chap{13}{\PP \VerseOne{1}Τρίτον τοῦτο ἔρχομαι πρὸς ὑμᾶς· Ἐπὶ στόματος δύο μαρτύρων καὶ τριῶν σταθήσεται πᾶν ῥῆμα.
\VS{2}Προείρηκα καὶ προλέγω, ὡς παρὼν τὸ δεύτερον καὶ ἀπὼν νῦν, τοῖς προημαρτηκόσιν καὶ τοῖς λοιποῖς πᾶσιν, ὅτι ἐὰν ἔλθω εἰς τὸ πάλιν οὐ φείσομαι,
\VS{3}ἐπεὶ δοκιμὴν ζητεῖτε τοῦ ἐν ἐμοὶ λαλοῦντος Χριστοῦ, ὃς εἰς ὑμᾶς οὐκ ἀσθενεῖ ἀλλὰ δυνατεῖ ἐν ὑμῖν.
\par }{\PP \VS{4}καὶ γὰρ ἐσταυρώθη ἐξ ἀσθενείας, ἀλλὰ ζῇ ἐκ δυνάμεως Θεοῦ. καὶ γὰρ ἡμεῖς ἀσθενοῦμεν ἐν αὐτῷ, ἀλλὰ ζήσομεν σὺν αὐτῷ ἐκ δυνάμεως Θεοῦ εἰς ὑμᾶς.
\VS{5}Ἑαυτοὺς πειράζετε εἰ ἐστὲ ἐν τῇ πίστει, ἑαυτοὺς δοκιμάζετε· ἢ οὐκ ἐπιγινώσκετε ἑαυτοὺς ὅτι Ἰησοῦς Χριστὸς ἐν ὑμῖν; εἰ μήτι ἀδόκιμοί ἐστε.
\VS{6}ἐλπίζω δὲ ὅτι γνώσεσθε ὅτι ἡμεῖς οὐκ ἐσμὲν ἀδόκιμοι.
\VS{7}Εὐχόμεθα δὲ πρὸς τὸν Θεὸν μὴ ποιῆσαι ὑμᾶς κακὸν μηδέν, οὐχ ἵνα ἡμεῖς δόκιμοι φανῶμεν, ἀλλ᾽ ἵνα ὑμεῖς τὸ καλὸν ποιῆτε, ἡμεῖς δὲ ὡς ἀδόκιμοι ὦμεν.
\VS{8}οὐ γὰρ δυνάμεθά τι κατὰ τῆς ἀληθείας ἀλλὰ= ὑπὲρ τῆς ἀληθείας.
\VS{9}χαίρομεν γὰρ ὅταν ἡμεῖς ἀσθενῶμεν, ὑμεῖς δὲ δυνατοὶ ἦτε· τοῦτο καὶ εὐχόμεθα, τὴν ὑμῶν κατάρτισιν.
\par }{\PP \VS{10}Διὰ τοῦτο ταῦτα ἀπὼν γράφω, ἵνα παρὼν μὴ ἀποτόμως χρήσωμαι κατὰ τὴν ἐξουσίαν ἣν ὁ Κύριος ἔδωκέν μοι εἰς οἰκοδομὴν καὶ οὐκ εἰς καθαίρεσιν.
\VS{11}Λοιπόν, ἀδελφοί, χαίρετε, καταρτίζεσθε, παρακαλεῖσθε, τὸ αὐτὸ φρονεῖτε, εἰρηνεύετε, καὶ ὁ Θεὸς τῆς ἀγάπης καὶ εἰρήνης ἔσται μεθ᾽ ὑμῶν.
\par }{\PP \VS{12}Ἀσπάσασθε ἀλλήλους ἐν ἁγίῳ φιλήματι. Ἀσπάζονται ὑμᾶς οἱ ἅγιοι πάντες.
\par }{\PP \VS{13}Ἡ χάρις τοῦ Κυρίου Ἰησοῦ Χριστοῦ καὶ ἡ ἀγάπη τοῦ Θεοῦ καὶ ἡ κοινωνία τοῦ Ἁγίου Πνεύματος μετὰ πάντων ὑμῶν.
\par }